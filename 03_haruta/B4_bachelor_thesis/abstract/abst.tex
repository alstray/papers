\documentclass{abst}

\begin{document}

%%%%%%%%%%%%%%%%%%%%%%%%%%%%%%%%%%%%%%%%%%%%%%%%%%%%%%%%%%%%%%%%%%%
\研究室名{番原研究室}
\氏名{春~田~~穂~高}
\卒論題目{%
  解集合プログラミングを用いたグラフ彩色問題の解法に関する考察}

%%%%%%%%%%%%%%%%%%%%%%%%%%%%%%%%%%%%%%%%%%%%%%%%%%%%%%%%%%%%%%%%%%%
\卒論要旨{%

\textbf{グラフ彩色問題} (Graph Coloring Problem; GCP)は,
与えられた有限無向グラフ$G$について,隣接する頂点が同色にならない
ように各頂点を塗りわけるとき,必要となる最小の色数 (彩色数と呼ばれる)
を求める問題である.
グラフ彩色問題はNP困難な問題であり,最適化コンパイラのレジスタ割
り付けや無線の周波数割り当て等の応用がある.
自然数$k\geq 3$について,グラフ$G$が$k$色以下で彩色可能かどう
かを決定するグラフ彩色判定問題はNP完全である.
$G$が$k$色以下で彩色可能で,$k-1$色以下で彩色可能でないとき,
$G$の彩色数は$k$となる.

\vskip .5em

グラフ彩色における\textbf{同色頂点数最小化(最大化)問題}とは,
グラフ彩色判定問題の実行可能解のうち,
同色(例えば,赤色)で塗られる頂点数の最小値(最大値)を求める問題である.
\textbf{多色頂点数最大化問題}は,
グラフ彩色判定問題の制約を満たしつつ,
多色(2色以上)で塗られる頂点数の最大値を求める問題である.
これらの問題は,グラフ彩色問題の関連問題として,
Knuth の教科書 The Art of Computer Programming (TAOCP) でも取り
上げられている.

\vskip .5em

\textbf{解集合プログラミング}(Answer Set Programming; ASP)
は論理プログラミングから派生した比較的新しいプログラミングパラダイムである.
ASP 言語は一階論理に基づいた知識表現言語の一種である.
論理プログラムはルールの有限集合である.
ASP システムは安定モデル意味論
に基づく解集合を計算するシステムである.
近年,SAT 技術を応用した高速な ASP システムが開発され,
スケジューリング,プランニング,
制約充足問題,制約最適化問題,有界モデル検査
など様々な分野への実用的応用が急速に拡大している.

\vskip .5em

%%%%%%%%%%%%%%%%%%%%%%%%%%%%%%%%%%%%%%%%%%%%%%%%%%%%%%%%%%
% Proposal
本論文では,グラフ彩色における同色頂点数最小化問題,
同色頂点数最大化問題,多色頂点数最大化問題を解く
ASP 符号化について述べる.
これらの問題に対して ASP を用いる利点としては,
ASP 言語の高い表現力,
充足不能コアに基づく最適化探索,
高速な解の列挙等
が挙げられる.

\vskip .5em

同色頂点数最小化問題と同色頂点数最大化問題を解く ASP 符号化は,
制約充足問題の SAT 符号化法の一つである直接符号化法
に基づいている.
一方,多色頂点数最大化問題を解くASP 符号化は,多値符号化法をベースにしている.
多値符号化法は各変数が複数の値を取ることを許す.そのため,
多色頂点数最大化問題の一つの解は,基のグラフ彩色判定問題の複数の実行可
能解に対応する圧縮解とみなすことができる.
%%% 実験内容
提案符号化の有効性を評価するために,
\textsf{McGregor}グラフをベンチマークとして,
高速 ASP システム \textsf{clingo}を用いて実行実験を行った.
その結果,提案符号化は,
同色頂点数最小化問題,同色頂点数最大化問題,多色頂点数最大化問題のすべ
てについて,Knuth の教科書に記載されていない新しい最適解を発見すること
に成功した.
% これらの結果は,\textsf{clingo}に実装された充足不能コアに
% 基づく最適化探索によって得られた.
  
}

\end{document}

%%% Local Variables:
%%% mode: latex
%%% TeX-master: t
%%% End:
