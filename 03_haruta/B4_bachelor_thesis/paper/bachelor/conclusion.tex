%%%%%%%%%%%%%%%%%%%%%%%%%%%%%%%%%%%%%%%%%%%%%%%%%%%%%%%%%% 
\chapter{結論}\label{chap]conclusion}
%%%%%%%%%%%%%%%%%%%%%%%%%%%%%%%%%%%%%%%%%%%%%%%%%%%%%%%%%%
本論文では,
解集合プログラミング(ASP)を用いた
グラフ彩色における
同色頂点数最小化問題,
同色頂点数最大化問題,
多色頂点数最大化問題
の解法について述べた.
これらの問題を解くASP符号化として,
\textsf{minimize},\textsf{maximize},\textsf{mult}符号化
を考案した.
% \textsf{color}~符号化はグラフ彩色判定問題を解く符号化である.
% \textsf{minimize}~符号化は\textsf{color}~符号化をベースに,
% ASPシステムの最小化関数を用いて,
% グラフ彩色における同色頂点数最小化問題を解く符号化である.
% \textsf{maximize}~符号化は\textsf{color}~符号化をベースに,
% ASPシステムの最小化関数を用いて,
% グラフ彩色における同色頂点数最大化問題を解く符号化である.
% \textsf{mult}~符号化は\textsf{color}~符号化をベースに,
% ASPシステムの個数制約と最大化関数を用いて,
% グラフ彩色における多色頂点数最大化問題を解く符号化である.
%%%%%%%%%%%%%%%%%%%%%%%%%%%%%%%%%%%%%%%%%%%%%%%%%%%%%%%%%%
考案した符号化の有効性を評価するために,
$N$次の\code{McGregor}グラフ~\cite{Knuth:TAOCP:SAT}
($3\leq N\leq 140$, 計138問)を使用して実験を行った.
%
% その結果,136問において彩色できると判定した.
% また,グラフ彩色問題を全解列挙する実験をベンチマーク8問を使用し行った.
% その結果,5問において全解列挙することができた.
実験の結果,
同色頂点数最小化問題について,
得られた最適値の数は BB 法で11問,USC 法で15問であり,USC 法が優れた結果を出した.
また,$N=17,18$の2問について,Knuth の教科書にない最適値を発見することができた.
%
同色頂点数最大化問題について,
得られた最適値の数は BB 法で6問,USC 法で31問であり,USC法が優れた結果を出した.
また,$17\leq N\leq 26$, $28\leq N\leq 31$, $34\leq N\leq 36$の17問について,
Knuth の教科書にない最適値を発見することができた.
%
多色頂点数最大化問題について,
得られた最適値の数は BB 法で7問,USC 法で10問であり,USC法が優れた結果を出した.
また,$N=11,12$次の\code{McGregor}グラフに対して,
Knuth の教科書にない最適値を発見することができた.
さらに,$N=12$次の\code{McGregor}グラフに対する
多色頂点数最大化問題の最適解が,
基のグラフ彩色判定問題の約680億の実行可能解を
表現していることが確認できた.
%
% \textsf{maximize}~符号化では作成したベンチマーク問題の内,
% 36問を使用し実験を行った.
% その結果,31問の最適値と新たに17問の解を発見した.
% \textsf{mult}~符号化では作成したベンチマーク問題の内,
% 13問を使用し実験を行った.
% その結果,10問の最適値と新たに2問の解を発見した.
% また,グラフ彩色問題の全解列挙と\textsf{mult}~符号化の実験結果より,
% グラフ彩色における多色頂点数最大化問題で得られる解の圧縮率を求めた.
% その結果,グラフのサイズが大きいほど圧縮率が減少していくことが確認できた.

%%%%%%%%%%%%%%%%%%%%%%%%%%%%%%%%%%%%%%%%%%%%%%%%%%%%%%%%%%
今後の課題として,
多色頂点数最大化問題の最適解が基のグラフ彩色判定問題の多くの
実行可能解を表す性質を応用し,様々な組合せ問題に対して,
実行可能解の圧縮表現に関する研究調査を進めたいと考えている.

%%% Local Variables:
%%% mode: latex
%%% TeX-master: "paper"
%%% End:
