%%%%%%%%%%%%%%%%%%%%%%%%%%%%%%%%%%%%%%%%%%%%%%%%%%%%%%%%%% 
\chapter{実行実験}\label{chap:experiments}
%%%%%%%%%%%%%%%%%%%%%%%%%%%%%%%%%%%%%%%%%%%%%%%%%%%%%%%%%% 

第4章で提案した ASP 符号化の有効性を評価するために,
\code{McGregor}グラフをベンチマークとして実行実験を行った.
彩色に使う色数は$k=4$とした.
使用した ASP システムは{\clingo}のバージョン5.5.0である.
実行環境は,Mac mini Intel Core i7 3.2GHz 64GB メモリである.

%%%%%%%%%%%%%%%%%%%%%%%%%%%%%%%%%%%%%%%%%%%%%%%%%%%%%%%%%%
\section{グラフ彩色判定問題の実験結果}
%%%%%%%%%%%%%%%%%%%%%%%%%%%%%%%%%%%%%%%%%%%%%%%%%%%%%%%%%%

%%%%%%%%%%%%%%%%%%%%%%%%%%%%%%%%%
\begin{table}[ht]
  \centering
  \tabcolsep = 5mm
  \renewcommand{\arraystretch}{1.1}
  \begin{tabular}{r|r|r}
 \hline
 N& 解の総数& CPU時間\\
 3&144*&0.002\\
 4&2376*&0.009\\
 5&43536*&0.143\\
 6&2589768*&7.796\\
 7&224442336*&865.500\\
 8&$\geq$816623222&-\\
 9&$\geq$676088853&-\\
 10&$\geq$504392039&-\\
\end{tabular}
%table:enum
\end{table}
%%%%%%%%%%%%%%%%%%%%%%%%%%%%%%%%%

まず最初に,グラフ彩色判定問題を解く ASP 符号化
(コード~\ref{code:colorsample.lp})を用いて,
$N$次の\code{McGregor}グラフ ($3 \leq N\leq 140$: 計138問)
の単解を求める実験を行った.
{\clingo}のオプションには\textit{trendy}を使用し,
制限時間は1問30分とした.
その結果,$3 \leq N\leq 140$ の136問について,
制限時間内に実行可能解を求めることができた.

さらに,グラフ彩色問題の実行可能解を全列挙する実験も行った.
制限時間は1問1時間とした.
表~\ref{table:enum}に実験結果を示す.
左から\code{McGregor}グラフの次数,
得られた解の総数,
全列挙できたかどうかの別,
解を求めるのに要した CPU 時間
を表している.
表より,
$3 \leq N\leq 7$の\code{McGregor}グラフに対して,
その解を全列挙することができた.
例えば,$7$次の\code{McGregor}グラフの解の総数は,
224,442,336 であることがわかった.

%%%%%%%%%%%%%%%%%%%%%%%%%%%%%%%%%%%%%%%%%%%%%%%%%%%%%%%%%%
\section{同色頂点数最小化問題と同色頂点数最大化問題の実験結果}
%%%%%%%%%%%%%%%%%%%%%%%%%%%%%%%%%%%%%%%%%%%%%%%%%%%%%%%%%%

%%%%%%%%%%%%%%%%%%%%%%%%%%%%%%%%%
\begin{table}[htp]
  \tabcolsep = 5mm
%  \renewcommand{\arraystretch}{1.2}
  \begin{minipage}[t]{0.45\linewidth}
    \centering
    \begin{tabular}{r|c|c}
 %n& f(n)\\
 N&BB &USC\\
 \hline
 3&2*&2*\\
 4&2*&2*\\
 5&3*&3*\\
 6&4*&4*\\
 7&5*&5*\\
 8&7*&7*\\
 9&7*&7*\\
 10&7*&7*\\
 11&8*&8*\\
 12&9*&9*\\
 13&10*&10*\\
 14&12&\textbf{12*}\\
 15&\textbf{12}&49\\
 16&16&\textbf{12*}\\
 17&21&\textbf{\textcolor{red}{13*}}\\
 18&19&\textbf{\textcolor{red}{14*}}\\
 19&\textbf{20}&58\\
 20&\textbf{22}&59\\\hline
\end{tabular}

%table:min
    \caption{同色頂点数最小化問題の実験結果}
    \label{table:min}
  \end{minipage}
  \begin{minipage}[t]{0.45\linewidth}
    \centering
    \begin{table}[t]\scriptsize
%\renewcommand{\arraystrech}{1.2}
\begin{tabular}{r|cc||r|cc||r|cc}
 \hline
 %n& g(n)\\
 N&BB &USC&N&BB&USC&N&BB&USC\\
 \hline
 3&4*&4*&    15&71&\textbf{77*}&
 27&\textbf{185}&180 \\
 4&6*&6*&    16&71&\textbf{88*}&
 28&199&\textbf{\textcolor{red}{266*}} \\
 5&10*&10*&    17&76&\textbf{\textcolor{red}{99*}}&
 29&221&\textbf{\textcolor{red}{285*}} \\
 6&13*&13*&    18&91&\textbf{\textcolor{red}{111*}}&
 30&224&\textbf{\textcolor{red}{305*}} \\
 7&17*&17*&    19&92&\textbf{\textcolor{red}{123*}}&
 31&247&\textbf{\textcolor{red}{325*}} \\
 8&23*&23*&    20&109&\textbf{\textcolor{red}{137*}}&
 32&255&255 \\
 9&28&\textbf{28*}&    21&121&\textbf{\textcolor{red}{150*}}&
 33&\textbf{280}&278 \\
 10&35&\textbf{35*}&    22&122&\textbf{\textcolor{red}{165*}}&
 34&296&\textbf{\textcolor{red}{391*}} \\
 11&42&\textbf{42*}&    23&137&\textbf{\textcolor{red}{180*}}&
 35&310&\textbf{\textcolor{red}{414*}} \\
 12&49&\textbf{50*}&    24&146&\textbf{\textcolor{red}{196*}}&
 36&338&\textbf{\textcolor{red}{438*}} \\
 13&56&\textbf{58*}&    25&162&\textbf{\textcolor{red}{212*}}&
 37&\textbf{348}&340 \\
 14&56&\textbf{68*}&    26&180&\textbf{\textcolor{red}{230*}}&
 38&371&371 \\ 
\end{tabular}
\end{table}
%table:max
    \caption{実験結果: 同色頂点数最大化問題}
    \label{table:max}
  \end{minipage}
\end{table}
%%%%%%%%%%%%%%%%%%%%%%%%%%%%%%%%%

グラフ彩色における同色頂点数最小化問題の実験結果について述べる.
{\clingo}のオプションには\textit{trendy}を使用し,一問あたりの制限時
間は1時間とした.
%
表~\ref{table:min}に実験結果を示す.
左から\code{McGregor}グラフの次数,
{\clingo}を用いて得られた最適値・最良値を表している.
BB と USC は{\clingo}の最適値探索オプションであり,
BB は分枝限定法,USC は充足不能コアに基づいている.
記号``*''は,その値が最適値であることを証明できたことを意味する.
2つの最適値探索において,より良い結果を得られた方を太字で示している.
赤字は Knuth の TAOCP~\cite{Knuth:TAOCP:SAT}
に記載されていない最適値を示している.
%
得られた最適値の数は BB 法で11問,USC 法で15問であり,USC 法が優れた結果を出した.
また,$N=17,18$の2問について,Knuth の教科書にない最適値を発見することができた.

つぎに,同色頂点数最大化問題の実験結果を表~\ref{table:max}に示す.
表の見方は,同色頂点数最小化問題と同じである.
表より,最適値の数は BB 法で6問,USC 法で31問であり,USC法が優れた結果を出した.
また,$17\leq N\leq 26$, $28\leq N\leq 31$, $34\leq N\leq 36$の17問について,
Knuth の教科書にない最適値を発見することができた.

%%%%%%%%%%%%%%%%%%%%%%%%%%%%%%%%%%%%%%%%%%%%%%%%%%%%%%%%%%
\section{多色頂点数最大化問題の実験結果}
%%%%%%%%%%%%%%%%%%%%%%%%%%%%%%%%%%%%%%%%%%%%%%%%%%%%%%%%%%

%%%%%%%%%%%%%%%%%%%%%%%%%%%%%%%%%
\begin{table}[htbp]
%  \tabcolsep = 5mm
  \renewcommand{\arraystretch}{1.1}  
  \begin{minipage}[t]{0.3\linewidth}
    \centering
    \begin{tabular}{r|r|r}
 %n& h(n)\\
 N&BB &USC\\
 \hline
 3&1*&1*\\
 4&3*&3*\\
 5&4*&4*\\
 6&7*&7*\\
 7&9*&9*\\
 8&13*&13*\\
 9&18*&18*\\
 10&23&\textbf{23*}\\
 11&27&\textbf{\textcolor{red}{29*}}\\
 12&34&\textbf{\textcolor{red}{36*}}\\
 13&\textbf{39}&15\\
 14&\textbf{44}&11\\
 15&\textbf{49}&20\\\hline
\end{tabular}

%table:mult
    \caption{多色頂点数最大化問題の実験結果}
    \label{table:mult}
  \end{minipage}
  \begin{minipage}[t]{0.65\linewidth}
    \centering
    \begin{tabular}{r|c|c|c}
 \hline
 n& h(n)& 圧縮された解の個数& 圧縮率 \\
 \hline
 3&	1&	2&	1.3889 \\
 4&	3&	8&	0.3367 \\
 5&	4&	16&	0.0368 \\
 6&	7&	128&	0.0049 \\
 7&	9&	512&	0.0002 \\
 8&	13&	8192&	- \\
 9&	18&	262144&	- \\
 10&	23&	8388608&	- \\
 11&	29&	536870912&	- \\
 12&	36&	68719476736&	- \\
\end{tabular}
\caption{多色頂点数最大化問題の解の圧縮率}
\label{table:com}%table:com
    \caption{多色頂点数最大化問題の解の個数}
    \label{table:com}
  \end{minipage}
\end{table}
%%%%%%%%%%%%%%%%%%%%%%%%%%%%%%%%%

多色頂点数最大化問題の実験結果について述べる.
実験には{\clingo}のオプション~\textit{trendy}を使用し,
制限時間は1問あたり1時間とした.
%
表~\ref{table:mult}に実験結果を示す.
表の見方は,同色頂点数最小化(最大化)問題と同じである.
表より,最適値の数は BB 法で7問,USC 法で10問であり,USC法が優れた結果を出した.
また,$N=11,12$次の\code{McGregor}グラフに対して,
Knuth の教科書にない最適値を発見することができた.

第~\ref{chap:background}章で述べたように,
多色頂点数最大化問題の最適解は,
基のグラフ彩色判定問題の複数の実行可能解に対応する.
表~\ref{table:com}に,対応結果をまとめたものを示す.
左から,
\code{McGregor}グラフの次数$N$,
多色頂点数最大化問題の最適値,
対応する基のグラフ彩色判定問題の解の個数,
多色頂点数最大化問題の最適値を得るのに要した CPU 時間を表している.
表より,
$12$次の\code{McGregor}グラフに対する
多色頂点数最大化問題の1つの最適解は,
基のグラフ彩色判定問題の約680億の実行可能解を
表現していることがわかる.

%%%%%%%%%%%%%%%%%%%%%%%%%%%%%%%%%%%%%%%%%%%%%%%%%%%%%%%%%%

%%% Local Variables:
%%% mode: latex
%%% TeX-master: "paper"
%%% End:
