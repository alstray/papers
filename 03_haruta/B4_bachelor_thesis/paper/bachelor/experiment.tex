%%%%%%%%%%%%%%%%%%%%%%%%%%%%%%%%%%%%%%%%%%%%%%%%%%%%%%%%%% 
\chapter{実行実験}
%%%%%%%%%%%%%%%%%%%%%%%%%%%%%%%%%%%%%%%%%%%%%%%%%%%%%%%%%% 

%%%%%%%%%%%%%%%%%%%%%%%%%%%%%%%%%%%%%%%%%%%%%%%%%%%%%%%%%%
\section{実験概要}
%%%%%%%%%%%%%%%%%%%%%%%%%%%%%%%%%%%%%%%%%%%%%%%%%%%%%%%%%%

4章で示した4種のASP符号化を評価するにあたり,ベンチマークを作成した.
これらのASP符号化の評価にはグラフ~\code{G}と色数~\code{k}が必要である.
今回は評価のために~\code{McGregor}グラフを使用した.
~\code{McGregor}グラフは4彩色可能なグラフである.
ベンチマークとして~\code{McGregor}グラフの~\code{order}3$\sim$140までの
138問を作成した.

使用したASPシステムは{\clingo}のバージョン5.5.0である.
実行環境は,~Mac mini Intel Core i7 3.2GHz 64GBメモリである.

%%%%%%%%%%%%%%%%%%%%%%%%%%%%%%%%%%%%%%%%%%%%%%%%%%%%%%%%%%
\section{グラフ彩色判定問題の実験結果}
%%%%%%%%%%%%%%%%%%%%%%%%%%%%%%%%%%%%%%%%%%%%%%%%%%%%%%%%%%

本節では,グラフ彩色判定問題の実験結果について記述する.
実験には~{\clingo}のオプションは~\textit{trendy}を使用し,制限時間は30分とした.
ベンチマーク問題は~\code{McGregor}グラフの
~\code{order}3$\sim$140までの合計138問である.
実験結果としては~\code{order}3$\sim$138の合計136問において,
彩色できると判定した.

%%%%%%%%%%%%%%%%%%%%%%%%%%%%%%%%%%%%%%%%%%%%%%%%%%%%%%%%%%
\section{グラフ彩色における同色頂点数最小化問題の実験結果}
%%%%%%%%%%%%%%%%%%%%%%%%%%%%%%%%%%%%%%%%%%%%%%%%%%%%%%%%%%

%%%%%%%%%%%%%%%%%%%%%%%%%%%%%%%%%%%%%
\begin{table}[t]\footnotesize
 \caption{グラフ彩色における同色頂点数最小問題}
 \label{table:min}
 \centering
 \begin{tabular}{|l|c|c|}\hline
  & & \\
 \end{tabular}
\end{table}%table:min
%%%%%%%%%%%%%%%%%%%%%%%%%%%%%%%%%%%%%

本節では,グラフ彩色における同色頂点数最小化問題の実験結果について記述する.
実験には~{\clingo}のオプションは~\textit{trendy}を使用し,制限時間は1時間とした.
比較のために分枝限定法(branch bound)(以下BB法と示す)と
~\code{un sat core}~法において実行した.
ベンチマーク問題は~\code{McGregor}グラフの
~\code{order}3$\sim$20までの合計18問である.

図~\ref{table:min}に実験結果を示す.
各問題毎に最適値を求められた部分は赤字で示している.


%%%%%%%%%%%%%%%%%%%%%%%%%%%%%%%%%%%%%%%%%%%%%%%%%%%%%%%%%%
\section{グラフ彩色における同色頂点数最大化問題の実験結果}
%%%%%%%%%%%%%%%%%%%%%%%%%%%%%%%%%%%%%%%%%%%%%%%%%%%%%%%%%%

本節では,グラフ彩色における同色頂点数最大化問題の実験結果について記述する.
実験には~{\clingo}のオプションは~\textit{trendy}を使用し,制限時間は1時間とした.
比較のために分枝限定法(branch bound)(以下BB法と示す)と
~\code{un sat core}~法において実行した.
ベンチマーク問題は~\code{McGregor}グラフの
~\code{order}3$\sim$38までの合計36問である.

%%%%%%%%%%%%%%%%%%%%%%%%%%%%%%%%%%%%%%%%%%%%%%%%%%%%%%%%%%
\section{グラフ彩色における多色頂点数最大化問題の実験結果}
%%%%%%%%%%%%%%%%%%%%%%%%%%%%%%%%%%%%%%%%%%%%%%%%%%%%%%%%%%

本節では,グラフ彩色における多色頂点数最大化問題の実験結果について記述する.
実験には~{\clingo}のオプションは~\textit{trendy}を使用し,制限時間は1時間とした.
比較のために分枝限定法(branch bound)(以下BB法と示す)と
~\code{un sat core}~法において実行した.
ベンチマーク問題は~\code{McGregor}グラフの
~\code{order}3$\sim$15までの合計13問である.

%%% Local Variables:
%%% mode: latex
%%% TeX-master: "paper"
%%% End:
