%%%%%%%%%%%%%%%%%%%%%%%%%%%%%%%%%%%%%%%%%%%%%%%%%%%%%%%%%% 
\chapter{実行実験}
%%%%%%%%%%%%%%%%%%%%%%%%%%%%%%%%%%%%%%%%%%%%%%%%%%%%%%%%%% 

%%%%%%%%%%%%%%%%%%%%%%%%%%%%%%%%%%%%%%%%%%%%%%%%%%%%%%%%%%
\section{実験概要}
%%%%%%%%%%%%%%%%%%%%%%%%%%%%%%%%%%%%%%%%%%%%%%%%%%%%%%%%%%

4章で示した4種のASP符号化を評価するにあたり,ベンチマークを作成した.
これらのASP符号化の評価にはグラフ~\code{G}と色数~\code{k}が必要である.
今回は評価のために~\code{McGregor}グラフを使用した.
~\code{McGregor}グラフは4彩色可能なグラフである.
ベンチマークとして~\code{McGregor}グラフの~\code{order}3$\sim$140までの
138問を作成した.

使用したASPシステムは{\clingo}のバージョン5.5.0である.
実行環境は,~Mac mini Intel Core i7 3.2GHz 64GBメモリである.

%%%%%%%%%%%%%%%%%%%%%%%%%%%%%%%%%%%%%%%%%%%%%%%%%%%%%%%%%%
\section{グラフ彩色判定問題の実験結果}
%%%%%%%%%%%%%%%%%%%%%%%%%%%%%%%%%%%%%%%%%%%%%%%%%%%%%%%%%%

本節では,グラフ彩色判定問題の実験結果について記述する.
実験には~{\clingo}のオプション~\textit{trendy}を使用し,制限時間は30分とした.
ベンチマーク問題は~\code{McGregor}グラフの
~\code{order}3$\sim$140までの合計138問である.
実験結果としては~\code{order}3$\sim$138の合計136問において,
彩色できると判定した.

また,追加実験として,グラフ彩色問題を全解列挙する実験を行った.
実験には~{\clingo}のオプション~\textit{trendy}を使用し,制限時間は1時間とした.
ベンチマーク問題は~\code{McGregor}グラフの
~\code{order}3$\sim$10までの合計8問である.
表~\ref{table:enum}に実験結果を示す.
各問題毎に全解列挙ができた問題は*で示している.
時間内に全解列挙ができなかった問題は$\geq$で示している.
~\code{order}3$\sim$7までのグラフで制限時間内に全解列挙を求めることができた.

%%%%%%%%%%%%%%%%%%%%%%%%%%%%%%%%%
\begin{table}[tb]
  \begin{minipage}[t]{0.45\linewidth}
    \centering
    \begin{tabular}{r|c|c}
 %n& f(n)\\
 N&BB &USC\\
 \hline
 3&2*&2*\\
 4&2*&2*\\
 5&3*&3*\\
 6&4*&4*\\
 7&5*&5*\\
 8&7*&7*\\
 9&7*&7*\\
 10&7*&7*\\
 11&8*&8*\\
 12&9*&9*\\
 13&10*&10*\\
 14&12&\textbf{12*}\\
 15&\textbf{12}&49\\
 16&16&\textbf{12*}\\
 17&21&\textbf{\textcolor{red}{13*}}\\
 18&19&\textbf{\textcolor{red}{14*}}\\
 19&\textbf{20}&58\\
 20&\textbf{22}&59\\\hline
\end{tabular}

%table:min
  \end{minipage}
  \begin{minipage}[t]{0.45\linewidth}
    \centering
    \begin{table}[t]\scriptsize
%\renewcommand{\arraystrech}{1.2}
\begin{tabular}{r|cc||r|cc||r|cc}
 \hline
 %n& g(n)\\
 N&BB &USC&N&BB&USC&N&BB&USC\\
 \hline
 3&4*&4*&    15&71&\textbf{77*}&
 27&\textbf{185}&180 \\
 4&6*&6*&    16&71&\textbf{88*}&
 28&199&\textbf{\textcolor{red}{266*}} \\
 5&10*&10*&    17&76&\textbf{\textcolor{red}{99*}}&
 29&221&\textbf{\textcolor{red}{285*}} \\
 6&13*&13*&    18&91&\textbf{\textcolor{red}{111*}}&
 30&224&\textbf{\textcolor{red}{305*}} \\
 7&17*&17*&    19&92&\textbf{\textcolor{red}{123*}}&
 31&247&\textbf{\textcolor{red}{325*}} \\
 8&23*&23*&    20&109&\textbf{\textcolor{red}{137*}}&
 32&255&255 \\
 9&28&\textbf{28*}&    21&121&\textbf{\textcolor{red}{150*}}&
 33&\textbf{280}&278 \\
 10&35&\textbf{35*}&    22&122&\textbf{\textcolor{red}{165*}}&
 34&296&\textbf{\textcolor{red}{391*}} \\
 11&42&\textbf{42*}&    23&137&\textbf{\textcolor{red}{180*}}&
 35&310&\textbf{\textcolor{red}{414*}} \\
 12&49&\textbf{50*}&    24&146&\textbf{\textcolor{red}{196*}}&
 36&338&\textbf{\textcolor{red}{438*}} \\
 13&56&\textbf{58*}&    25&162&\textbf{\textcolor{red}{212*}}&
 37&\textbf{348}&340 \\
 14&56&\textbf{68*}&    26&180&\textbf{\textcolor{red}{230*}}&
 38&371&371 \\ 
\end{tabular}
\end{table}
%table:max
  \end{minipage}
\end{table}
%%%%%%%%%%%%%%%%%%%%%%%%%%%%%%%%%

%%%%%%%%%%%%%%%%%%%%%%%%%%%%%%%%%
\begin{table}[tb]
  \begin{minipage}[t]{0.45\linewidth}
    \centering
    \begin{tabular}{r|r|r}
 %n& h(n)\\
 N&BB &USC\\
 \hline
 3&1*&1*\\
 4&3*&3*\\
 5&4*&4*\\
 6&7*&7*\\
 7&9*&9*\\
 8&13*&13*\\
 9&18*&18*\\
 10&23&\textbf{23*}\\
 11&27&\textbf{\textcolor{red}{29*}}\\
 12&34&\textbf{\textcolor{red}{36*}}\\
 13&\textbf{39}&15\\
 14&\textbf{44}&11\\
 15&\textbf{49}&20\\\hline
\end{tabular}

%table:mult
  \end{minipage}
  \begin{minipage}[t]{0.45\linewidth}
    \centering
    \begin{tabular}{r|r|r}
 \hline
 N& 解の総数& CPU時間\\
 3&144*&0.002\\
 4&2376*&0.009\\
 5&43536*&0.143\\
 6&2589768*&7.796\\
 7&224442336*&865.500\\
 8&$\geq$816623222&-\\
 9&$\geq$676088853&-\\
 10&$\geq$504392039&-\\
\end{tabular}
%table:enum
  \end{minipage}
\end{table}
%%%%%%%%%%%%%%%%%%%%%%%%%%%%%%%%%

%%%%%%%%%%%%%%%%%%%%%%%%%%%%%%%%%
\begin{table}[tb]
  \centering
  \begin{tabular}{r|c|c|c}
 \hline
 n& h(n)& 圧縮された解の個数&圧縮率\\
 \hline
 3& 1   & 144       & 1.3888 \\
 4& 3   & 2376      & 0.3367 \\
 5& 4   & 43536     & 0.0367 \\
 6& 7   & 2589768   & 0.0049 \\
 7& 9   & 224442336 & 0.0002 \\
 8& 13  & -         & -      \\
 9& 18  & -         & -      \\
 10&23  & -         & -      \\
\end{tabular}
\caption{多色頂点最大化問題の解の圧縮率}
\label{table:com}%table:com
\end{table}
%%%%%%%%%%%%%%%%%%%%%%%%%%%%%%%%%

%%%%%%%%%%%%%%%%%%%%%%%%%%%%%%%%%%%%%%%%%%%%%%%%%%%%%%%%%%
\section{グラフ彩色における同色頂点数最小化問題の実験結果}
%%%%%%%%%%%%%%%%%%%%%%%%%%%%%%%%%%%%%%%%%%%%%%%%%%%%%%%%%%

本節では,グラフ彩色における同色頂点数最小化問題の実験結果について記述する.
実験には~{\clingo}のオプション~\textit{trendy}を使用し,制限時間は1時間とした.
比較のために分枝限定法(branch bound)(以下BB法と示す)と
~\code{un sat core}~法(以下USC法と示す)において実行した.
ベンチマーク問題は~\code{McGregor}グラフの
~\code{order}3$\sim$20までの合計18問である.

表~\ref{table:min}に実験結果を示す.
各問題毎に最適値を求めることができた箇所は*で示している.
また,2つの方法において,より良い結果を得られた方は太字で示している.
赤字はD.~E~.Knuthの教科書
The Art of Computer Programming~\cite{Knuth:TAOCP:SAT}
に記載されていない最適値を示している.
最適値の数はBB法で11問,USC法で15問であり,USC法が優れた結果を出した.
また,~\code{order}17と18において,新たに最適値を発見することができた.


%%%%%%%%%%%%%%%%%%%%%%%%%%%%%%%%%%%%%%%%%%%%%%%%%%%%%%%%%%
\section{グラフ彩色における同色頂点数最大化問題の実験結果}
%%%%%%%%%%%%%%%%%%%%%%%%%%%%%%%%%%%%%%%%%%%%%%%%%%%%%%%%%%

本節では,グラフ彩色における同色頂点数最大化問題の実験結果について記述する.
実験には~{\clingo}のオプション~\textit{trendy}を使用し,制限時間は1時間とした.
比較のために分枝限定法(branch bound)(以下BB法と示す)と
~\code{un sat core}~法(以下USC法と示す)において実行した.
ベンチマーク問題は~\code{McGregor}グラフの
~\code{order}3$\sim$38までの合計36問である.

表~\ref{table:max}に実験結果を示す.
各問題毎に最適値を求めることができた箇所は*で示している.
また,2つの方法において,より良い結果を得られた方は太字で示している.
赤字はD.~E~.Knuthの教科書
The Art of Computer Programming~\cite{Knuth:TAOCP:SAT}
に記載されていない最適値を示している.
最適値の数はBB法で6問,USC法で31問であり,USC法が優れた結果を出した.
また,~\code{order}17$\sim$26,28$\sim$31,34$\sim$36において,
新たに最適値を発見することができた.

%%%%%%%%%%%%%%%%%%%%%%%%%%%%%%%%%%%%%%%%%%%%%%%%%%%%%%%%%%
\section{グラフ彩色における多色頂点数最大化問題の実験結果}
%%%%%%%%%%%%%%%%%%%%%%%%%%%%%%%%%%%%%%%%%%%%%%%%%%%%%%%%%%

本節では,グラフ彩色における多色頂点数最大化問題の実験結果について記述する.
実験には~{\clingo}のオプション~\textit{trendy}を使用し,制限時間は1時間とした.
比較のために分枝限定法(branch bound)(以下BB法と示す)と
~\code{un sat core}~法(以下USC法と示す)において実行した.
ベンチマーク問題は~\code{McGregor}グラフの
~\code{order}3$\sim$15までの合計13問である.

表~\ref{table:mult}に実験結果を示す.
各問題毎に最適値を求めることができた箇所は*で示している.
また,2つの方法において,より良い結果を得られた方は太字で示してる.
赤字はD.~E~.Knuthの教科書
The Art of Computer Programming~\cite{Knuth:TAOCP:SAT}
に記載されていない最適値を示している.
最適値の数はBB法で7問,USC法で10問であり,USC法が優れた結果を出した.
また,~\code{order}11と12において,新たに最適値を発見することができた.

さらに,表~\ref{table:mult}と表~\ref{table:enum}から
解の圧縮率を計算し,表~\ref{table:com}を示す.
表中の\code{h(n)}は多色頂点数最大化問題で得られた最適値を示す.
表中より,~\code{order}が大きくなるほど
圧縮率は減少しているのが分かる.


%%%%%%%%%%%%%%%%%%%%%%%%%%%%%%%%%%%%%%%%%%%%%%%%%%%%%%%%%%

%%% Local Variables:
%%% mode: latex
%%% TeX-master: "paper"
%%% End:
