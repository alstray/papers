%%%%%%%%%%%%%%%%%%%%%%%%%%%%%%%%%%%%%%%%%%%%%%%%%%%%%%%%%% 
\chapter{実行実験}
%%%%%%%%%%%%%%%%%%%%%%%%%%%%%%%%%%%%%%%%%%%%%%%%%%%%%%%%%% 

%%%%%%%%%%%%%%%%%%%%%%%%%%%%%%%%%%%%%%%%%%%%%%%%%%%%%%%%%%
\section{実験概要}
%%%%%%%%%%%%%%%%%%%%%%%%%%%%%%%%%%%%%%%%%%%%%%%%%%%%%%%%%%

4章で示した4種のASP符号化を評価するにあたり,ベンチマークを作成した.
これらのASP符号化の評価にはグラフ~\code{G}と色数~\code{k}が必要である.
今回は評価のために~\code{McGregor}グラフを使用した.
~\code{McGregor}グラフは4彩色可能なグラフである.
ベンチマークとして~\code{McGregor}グラフの~\code{order}3$\sim$140までの
138問を作成した.

使用したASPシステムは{\clingo}のバージョン5.5.0である.
実行環境は,~Mac mini Intel Core i7 3.2GHz 64GBメモリである.

%%%%%%%%%%%%%%%%%%%%%%%%%%%%%%%%%%%%%%%%%%%%%%%%%%%%%%%%%%
\section{グラフ彩色判定問題の実験結果}
%%%%%%%%%%%%%%%%%%%%%%%%%%%%%%%%%%%%%%%%%%%%%%%%%%%%%%%%%%

本節では,グラフ彩色判定問題の実験結果について記述する.
実験には~{\clingo}のオプション~\textit{trendy}を使用し,制限時間は30分とした.
ベンチマーク問題は~\code{McGregor}グラフの
~\code{order}3$\sim$140までの合計138問である.
実験結果としては~\code{order}3$\sim$138の合計136問において,
彩色できると判定した.

また,追加実験として,グラフ彩色問題の全解列挙を行う実験を行った.
実験には~{\clingo}のオプション~\textit{trendy}を使用し,制限時間は1時間とした.
ベンチマーク問題は~\code{McGregor}グラフの
~\code{order}3$\sim$10までの合計8問である.
図~\ref{table:enum}に実験結果を示す.
各問題毎に全解列挙ができた問題は*で示している.
時間内に全解列挙ができなかった問題は$\geq$で示している.

%%%%%%%%%%%%%%%%%%%%%%%%%%%%%%%%%
\begin{table}[tb]
  \begin{minipage}[t]{0.45\linewidth}
    \centering
    \begin{table}[t]\scriptsize
\begin{tabular}{r|c|c||r|c|c}
 \hline
 %n& f(n)\\
 n&BB &USC&n&BB&USC\\
 \hline
 3&2*&2*&12&9*&9*\\
 4&2*&2*&13&10*&10*\\
 5&3*&3*&14&12&\textbf{12*}\\
 6&4*&4*&15&\textbf{12}&49\\
 7&5*&5*&16&16&\textbf{12*}\\
 8&7*&7*&17&21&\textbf{\textcolor{red}{13*}}\\
 9&7*&7*&18&19&\textbf{\textcolor{red}{14*}}\\
 10&7*&7*&19&\textbf{20}&58\\
 11&8*&8*&20&\textbf{22}&59\\
\end{tabular}
\end{table} %table:min
  \end{minipage}
  \begin{minipage}[t]{0.45\linewidth}
    \centering
    \begin{tabular}{r|c|c}
 %n& g(n)\\
 N&BB &USC\\
 \hline
 3&4*&4*\\
 4&6*&6*\\
 5&10*&10*\\
 6&13*&13*\\
 7&17*&17*\\
 8&23*&23*\\
 9&28&\textbf{28*}\\
 10&35&\textbf{35*}\\
 11&42&\textbf{42*}\\
 12&49&\textbf{50*}\\
 13&56&\textbf{58*}\\
 14&56&\textbf{68*}\\
 15&71&\textbf{77*}\\
 16&71&\textbf{88*}\\
 17&76&\textbf{\textcolor{red}{99*}}\\
 18&91&\textbf{\textcolor{red}{111*}}\\
 19&92&\textbf{\textcolor{red}{123*}}\\
 20&109&\textbf{\textcolor{red}{137*}}\\
 21&121&\textbf{\textcolor{red}{150*}}\\
 22&122&\textbf{\textcolor{red}{165*}}\\
 23&137&\textbf{\textcolor{red}{180*}}\\
 24&146&\textbf{\textcolor{red}{196*}}\\
 25&162&\textbf{\textcolor{red}{212*}}\\
 26&180&\textbf{\textcolor{red}{230*}}\\
 27&\textbf{185}&180\\
 28&199&\textbf{\textcolor{red}{266*}}\\
 29&221&\textbf{\textcolor{red}{285*}}\\
 30&224&\textbf{\textcolor{red}{305*}}\\
 31&247&\textbf{\textcolor{red}{325*}}\\
 32&255&255\\
 33&\textbf{280}&278\\
 34&296&\textbf{\textcolor{red}{391*}}\\
 35&310&\textbf{\textcolor{red}{414*}}\\
 36&338&\textbf{\textcolor{red}{438*}}\\
 37&\textbf{348}&340\\
 38&371&371\\\hline
\end{tabular}
 %table:max
  \end{minipage}
\end{table}
%%%%%%%%%%%%%%%%%%%%%%%%%%%%%%%%%

%%%%%%%%%%%%%%%%%%%%%%%%%%%%%%%%%
\begin{table}[tb]
  \begin{minipage}[t]{0.45\linewidth}
    \centering
    \begin{table}[t]\scriptsize
\begin{tabular}{r|c|c}
 \hline
 %n& h(n)\\
 n&BB &USC\\
 \hline
 3&1*&1*\\
 4&3*&3*\\
 5&4*&4*\\
 6&7*&7*\\
 7&9*&9*\\
 8&13*&13*\\
 9&18*&18*\\
 10&23&\textbf{23*}\\
 11&27&\textbf{\textcolor{red}{29*}}\\
 12&34&\textbf{\textcolor{red}{36*}}\\
 13&\textbf{39}&15\\
 14&\textbf{44}&11\\
 15&\textbf{49}&20\\
\end{tabular}
\end{table}
%table:mult
  \end{minipage}
  \begin{minipage}[t]{0.45\linewidth}
    \centering
    \begin{table}[tb]
\begin{tabular}{r|c|c}
 \hline
 N& 解の総数& CPU時間\\
 3&144*&0.002\\
 4&2376*&0.009\\
 5&43536*&0.143\\
 6&2589768*&7.796\\
 7&224442336*&865.500\\
 8&$\geq$816623222&-\\
 9&$\geq$676088853&-\\
 10&$\geq$504392039&-\\
\end{tabular}
\end{table}%table:enum
  \end{minipage}
\end{table}
%%%%%%%%%%%%%%%%%%%%%%%%%%%%%%%%%

%%%%%%%%%%%%%%%%%%%%%%%%%%%%%%%%%%%%%%%%%%%%%%%%%%%%%%%%%%
\section{グラフ彩色における同色頂点数最小化問題の実験結果}
%%%%%%%%%%%%%%%%%%%%%%%%%%%%%%%%%%%%%%%%%%%%%%%%%%%%%%%%%%

本節では,グラフ彩色における同色頂点数最小化問題の実験結果について記述する.
実験には~{\clingo}のオプション~\textit{trendy}を使用し,制限時間は1時間とした.
比較のために分枝限定法(branch bound)(以下BB法と示す)と
~\code{un sat core}~法において実行した.
ベンチマーク問題は~\code{McGregor}グラフの
~\code{order}3$\sim$20までの合計18問である.

図~\ref{table:min}に実験結果を示す.
各問題毎に最適値を求めることができた箇所は*で示している.
また,2つの方法において,より良い結果を得られた方は太字で示している.


%%%%%%%%%%%%%%%%%%%%%%%%%%%%%%%%%%%%%%%%%%%%%%%%%%%%%%%%%%
\section{グラフ彩色における同色頂点数最大化問題の実験結果}
%%%%%%%%%%%%%%%%%%%%%%%%%%%%%%%%%%%%%%%%%%%%%%%%%%%%%%%%%%

本節では,グラフ彩色における同色頂点数最大化問題の実験結果について記述する.
実験には~{\clingo}のオプション~\textit{trendy}を使用し,制限時間は1時間とした.
比較のために分枝限定法(branch bound)(以下BB法と示す)と
~\code{un sat core}~法において実行した.
ベンチマーク問題は~\code{McGregor}グラフの
~\code{order}3$\sim$38までの合計36問である.

図~\ref{table:max}に実験結果を示す.
各問題毎に最適値を求めることができた箇所は*で示している.
また,2つの方法において,より良い結果を得られた方は太字で示している.

%%%%%%%%%%%%%%%%%%%%%%%%%%%%%%%%%%%%%%%%%%%%%%%%%%%%%%%%%%
\section{グラフ彩色における多色頂点数最大化問題の実験結果}
%%%%%%%%%%%%%%%%%%%%%%%%%%%%%%%%%%%%%%%%%%%%%%%%%%%%%%%%%%

本節では,グラフ彩色における多色頂点数最大化問題の実験結果について記述する.
実験には~{\clingo}のオプション~\textit{trendy}を使用し,制限時間は1時間とした.
比較のために分枝限定法(branch bound)(以下BB法と示す)と
~\code{un sat core}~法において実行した.
ベンチマーク問題は~\code{McGregor}グラフの
~\code{order}3$\sim$15までの合計13問である.

図~\ref{table:mult}に実験結果を示す.
各問題毎に最適値を求めることができた箇所は*で示している.
また,2つの方法において,より良い結果を得られた方は太字で示してる.

%%% Local Variables:
%%% mode: latex
%%% TeX-master: "paper"
%%% End:
