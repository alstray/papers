%%%%%%%%%%%%%%%%%%%%%%%%%%%%%%%%%%%%%%%%%%%%%%%%%%%%%%%%%% 
\chapter{実行実験}
%%%%%%%%%%%%%%%%%%%%%%%%%%%%%%%%%%%%%%%%%%%%%%%%%%%%%%%%%% 

%%%%%%%%%%%%%%%%%%%%%%%%%%%%%%%%%%%%%%%%%%%%%%%%%%%%%%%%%%
\section{実験概要}
%%%%%%%%%%%%%%%%%%%%%%%%%%%%%%%%%%%%%%%%%%%%%%%%%%%%%%%%%%

\, 4章で示した4種のASP符号化を評価するにあたり,ベンチマークを作成した.
これらのASP符号化の評価にはグラフ\code{G}と色数\code{k}が必要である.
今回は評価のためにMcGregorグラフを使用した.
McGregorグラフは4彩色可能なグラフである.
ベンチマークとしてMcGregorグラフのオーダ3から140までの
138問を作成した,\\
\, 使用したASPシステムは{\clingo}のバージョン5.5.0である.
実行環境は,Mac mini 3.2GHz 64GBメモリである.

%%%%%%%%%%%%%%%%%%%%%%%%%%%%%%%%%%%%%%%%%%%%%%%%%%%%%%%%%%
\section{グラフ彩色判定問題の実験結果}
%%%%%%%%%%%%%%%%%%%%%%%%%%%%%%%%%%%%%%%%%%%%%%%%%%%%%%%%%%

\, 本節では,グラフ彩色判定問題の実験結果について記述する.
実験には{\clingo}のオプションは\textit{trendy}を使用し,制限時間は30分とした.
ベンチマーク問題はMcGregorグラフのオーダ3から140までの合計138問である.
%%
表に実験結果を示す.

%%%%%%%%%%%%%%%%%%%%%%%%%%%%%%%%%%%%%%%%%%%%%%%%%%%%%%%%%%
\section{グラフ彩色における同色頂点数最小化問題の実験結果}
%%%%%%%%%%%%%%%%%%%%%%%%%%%%%%%%%%%%%%%%%%%%%%%%%%%%%%%%%%

\, 本節では,グラフ彩色における同色頂点数最小化問題の実験結果について記述する.
実験には{\clingo}のオプションは\textit{trendy}を使用し,制限時間は1時間とした.
ベンチマーク問題はMcGregorグラフのオーダ3から20までの合計18問である.

%%%%%%%%%%%%%%%%%%%%%%%%%%%%%%%%%%%%%%%%%%%%%%%%%%%%%%%%%%
\section{グラフ彩色における同色頂点数最大化問題の実験結果}
%%%%%%%%%%%%%%%%%%%%%%%%%%%%%%%%%%%%%%%%%%%%%%%%%%%%%%%%%%

\, 本節では,グラフ彩色における同色頂点数最大化問題の実験結果について記述する.
実験には{\clingo}のオプションは\textit{trendy}を使用し,制限時間は1時間とした.
ベンチマーク問題はMcGregorグラフのオーダ3から38までの合計36問である.

%%%%%%%%%%%%%%%%%%%%%%%%%%%%%%%%%%%%%%%%%%%%%%%%%%%%%%%%%%
\section{グラフ彩色における多色頂点数最大化問題の実験結果}
%%%%%%%%%%%%%%%%%%%%%%%%%%%%%%%%%%%%%%%%%%%%%%%%%%%%%%%%%%

\, 本節では,グラフ彩色における多色頂点数最大化問題の実験結果について記述する.
実験には{\clingo}のオプションは\textit{trendy}を使用し,制限時間は1時間とした.
ベンチマーク問題はMcGregorグラフのオーダ3から15までの合計13問である.

%%% Local Variables:
%%% mode: latex
%%% TeX-master: "paper"
%%% End:
