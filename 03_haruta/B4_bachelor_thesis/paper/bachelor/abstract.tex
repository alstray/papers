%%%%%%%%%%%%%%%%%%%%%%%%%%%%%%%%%%%%%%%%%%%%%%%%%%%%%%%%%% 
\chapter*{概要}
\pagenumbering{roman}
%%%%%%%%%%%%%%%%%%%%%%%%%%%%%%%%%%%%%%%%%%%%%%%%%%%%%%%%%% 

本論文では,グラフ彩色における同色頂点数最小化問題,
同色頂点数最大化問題,多色頂点数最大化問題を解く
ASP 符号化について述べる.
これらの問題に対して ASP を用いる利点としては,
ASP 言語の高い表現力,
充足不能コアに基づく最適化探索,
高速な解の列挙等
が挙げられる.
同色頂点数最小化問題と同色頂点数最大化問題を解く ASP 符号化は,
制約充足問題の SAT 符号化法の一つである直接符号化法
に基づいている.
一方,多色頂点数最大化問題を解くASP 符号化は,多値符号化法をベースにしている.
多値符号化法は各変数が複数の値を取ることを許す.そのため,
多色頂点数最大化問題の一つの解は,基のグラフ彩色判定問題の複数の実行可
能解に対応する圧縮解とみなすことができる.
%%
提案符号化の有効性を評価するために,
\textsf{McGregor}グラフをベンチマークとして,
高速 ASP システム \textsf{clingo}を用いて実行実験を行った.
その結果,提案符号化は,
同色頂点数最小化問題,同色頂点数最大化問題,多色頂点数最大化問題のすべ
てについて,Knuth の教科書に記載されていない新しい最適解を発見すること
に成功した.

%%% Local Variables:
%%% mode: japanese-latex
%%% TeX-master: "paper"
%%% End: