%%%%%%%%%%%%%%%%%%%%%%%%%%%%%%%%%%%%%%%%%%%%%%%%%%%%%%%%%% 
\chapter{提案するASP符号化}\label{chap:encoding}
%%%%%%%%%%%%%%%%%%%%%%%%%%%%%%%%%%%%%%%%%%%%%%%%%%%%%%%%%% 

本章では,本論文が対象とする以下の3つの最適化問題を解く ASP 符号化につ
いて述べる.

\begin{itemize}
\item 同色頂点数最小化問題
\item 同色頂点数最大化問題
\item 多色頂点数最大化問題
\end{itemize}

考案した ASP 符号化は,
第~\ref{chap:asp}章で示した
グラフ彩色判定問題の\code{color}符号化を拡張したものである.

% %%%%%%%%%%%%%%%%%%%%%%%%%%%%%%%%%%%%%%%%%%%%%%%%%%%%%%%%%%
% \section{ASPファクト形式}
% %%%%%%%%%%%%%%%%%%%%%%%%%%%%%%%%%%%%%%%%%%%%%%%%%%%%%%%%%%

% %%%%%%%%%%%%%%%%%%%%%%%%%%%%%%
% \lstinputlisting[float=t,caption={%
% 同色頂点数最小化問題のASP符号化},%
% captionpos=b,frame=single,label=code:asp.lp,%
% numbers=left,%
% breaklines=true,%
% columns=fullflexible,keepspaces=true,%
% basicstyle=\ttfamily\footnotesize]{code/aspfact.lp}
% %%%%%%%%%%%%%%%%%%%%%%%%%%%%%%

% 本節では,グラフ彩色判定問題を例としてASPファクト形式について説明する.
% 図~\ref{fig:graph}のグラフを
% ファクト形式で表したものをコード~\ref{code:asp.lp}に示す.

% 1行目と2行目はそれぞれ頂点数と辺数を宣言している.
% 4行目ではアトム~\code{node}によって頂点1から頂点6まで表している.
% 10行目ではアトム~\code{edge}によって辺を定義している.
% \code{edge(X,Y)}は頂点~\code{X}と頂点~\code{Y}の間に辺が存在していることを意味する.

% %%%%%%%%%%%%%%%%%%%%%%%%%%%%%%%%%%%%%%%%%%%%%%%%%%%%%%%%%%
% \section{グラフ彩色判定問題のASP符号化}
% %%%%%%%%%%%%%%%%%%%%%%%%%%%%%%%%%%%%%%%%%%%%%%%%%%%%%%%%%% 

% %%%%%%%%%%%%%%%%%%%%%%%%%%%%%%
% \lstinputlisting[float=t,caption={%
% color 符号化},%
% captionpos=b,frame=single,label=code:color.lp,%
% numbers=left,%
% breaklines=true,%
% columns=fullflexible,keepspaces=true,%
% basicstyle=\ttfamily\footnotesize]{code/color1.lp}
% %%%%%%%%%%%%%%%%%%%%%%%%%%%%%%

% コード~\ref{code:color.lp}にグラフ彩色判定問題のASP符号化,
% \textsf{color}~符号化を示す.
% 符号化中の~\code{k}は色数を表しており,実行時に与えられる.
% 1行目では彩色する色数を指定している.
% 2行目のルールでは各頂点~\code{node(X)}に対して,
% その頂点が色~\code{C}で彩色することを意味するアトム~\code{color(X,C)}が
% 各頂点に対して唯一存在することを選択子を用いて導入している.
% 3行目のルールでは各辺~\code{X-Y}に対して,
% 頂点~\code{X}と頂点~\code{Y}が同じ色で彩色されることを禁止している.


%%%%%%%%%%%%%%%%%%%%%%%%%%%%%%%%%%%%%%%%%%%%%%%%%%%%%%%%%% 
\section{同色頂点数最小化問題のASP符号化}
%%%%%%%%%%%%%%%%%%%%%%%%%%%%%%%%%%%%%%%%%%%%%%%%%%%%%%%%%%

%%%%%%%%%%%%%%%%%%%%%%%%%%%%%%
\lstinputlisting[float=t,caption={%
  同色頂点数最小化問題のASP符号化 (\code{minimize}符号化)},%
captionpos=b,frame=single,label=code:color1.lp,%
numbers=left,%
breaklines=true,%
columns=fullflexible,keepspaces=true,%
basicstyle=\ttfamily]{code/color_min.lp}
%%%%%%%%%%%%%%%%%%%%%%%%%%%%%%

グラフ彩色に関する同色頂点数最小化問題は,グラフ彩色判定問題の実行可
能解のうち,同色(例えば,赤色)で塗られる頂点数の最小値を求める問題である.
%
コード~\ref{code:color1.lp}にこの問題を解く ASP 符号化を示す.
1行目は,\code{#const}宣言を用いて定数\code{v}を定義している.
この定数\code{v}は,彩色の際に最小化したい色を表している.
便宜上初期値を\code{1}としているが,実行に{\clingo}のコマンドラインオ
プションから変更できる.
3行目の定数\code{k}は,彩色に使用できる色数を表す.
7行目は,\code{#minimize}関数を用いて,色\code{v}で塗られる頂点の個数
を最小化している.その他のルールは,
第~\ref{chap:asp}章で示した\code{color}符号化 (コー
ド~\ref{code:colorsample.lp})と同じである.
%
本論文では,この符号化を\code{minimize符号化}と呼ぶ.

%%%%%%%%%%%%%%%%%%%%%%%%%%%%%%%%%%%%%%%%%%%%%%%%%%%%%%%%%% 
\section{同色頂点数最大化問題のASP符号化}
%%%%%%%%%%%%%%%%%%%%%%%%%%%%%%%%%%%%%%%%%%%%%%%%%%%%%%%%%%

%%%%%%%%%%%%%%%%%%%%%%%%%%%%%%
\lstinputlisting[float=t,caption={%
  同色頂点数最大化問題のASP符号化 (\code{maximize}符号化)},%
captionpos=b,frame=single,label=code:color2.lp,%
numbers=left,%
breaklines=true,%
columns=fullflexible,keepspaces=true,%
basicstyle=\ttfamily]{code/color_max.lp}
%%%%%%%%%%%%%%%%%%%%%%%%%%%%%%

グラフ彩色に関する同色頂点数最大化問題は,グラフ彩色判定問題の実行可
能解のうち,同色(例えば,赤色)で塗られる頂点数の最大値を求める問題である.
コード~\ref{code:color2.lp}にこの問題を解く ASP 符号化を示す.
\code{minimize符号化}との違いは,コード~\ref{code:color1.lp}の
\code{#minimize}関数を\code{#maximize}関数に置き換えた点である.
%
本論文では,この符号化を\code{maximize符号化}と呼ぶ.

%%%%%%%%%%%%%%%%%%%%%%%%%%%%%%%%%%%%%%%%%%%%%%%%%%%%%%%%%% 
\section{多色頂点数最大化問題のASP符号化}
%%%%%%%%%%%%%%%%%%%%%%%%%%%%%%%%%%%%%%%%%%%%%%%%%%%%%%%%%%

%%%%%%%%%%%%%%%%%%%%%%%%%%%%%%
\lstinputlisting[float=t,caption={%
  多色頂点数最大化問題のASP符号化 (\code{mult}符号化)},%
captionpos=b,frame=single,label=code:color3.lp,%
numbers=left,%
breaklines=true,%
columns=fullflexible,keepspaces=true,%
basicstyle=\ttfamily]{code/color_mult.lp}
%%%%%%%%%%%%%%%%%%%%%%%%%%%%%%

多色頂点数最大化問題は,グラフ彩色判定問題の制約を満たしつつ,
多色(2色以上)で塗られる頂点数の最大値を求める問題である.
この問題の最適解は,基のグラフ彩色判定問題の複数の実行可能解に対応す
るため,基の問題の圧縮解とみなすことができる.

コード~\ref{code:color3.lp}にこの問題を解く ASP 符号化を示す.
この符号化は,他の符号化と同様,グラフ彩色判定問題を解く
\code{color}符号化 (コード~\ref{code:colorsample.lp})
をベースとしている.
他の符号化との大きな違いは,2行目の個数制約に上限値がなく,
各頂点が少なくとも一つの色で塗られる,すなわち,多色彩色を許す点である.
また,多色彩色可能な頂点を表す補助アトム~\code{mult(X)}を導入している
点も異なる.
5行目のルールでは,各頂点\code{X}に対して,\code{X}が2色以上で彩色でき
る場合,\code{mult(X)}を生成している.
7行目は,\code{#maximize}関数を使って,
\code{mult(X)}の数,つまり多色で彩色できる頂点数を最大化している.
%
本論文では,この符号化を\code{mult}符号化と呼ぶ.



%%% Local Variables:
%%% mode: latex
%%% TeX-master: "paper"
%%% End:
