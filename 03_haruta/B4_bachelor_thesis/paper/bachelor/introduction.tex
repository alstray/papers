%%%%%%%%%%%%%%%%%%%%%%%%%%%%%%%%%%%%%%%%%%%%%%%%%%%%%%%%%% 
\chapter{序論}
\pagenumbering{arabic}
%%%%%%%%%%%%%%%%%%%%%%%%%%%%%%%%%%%%%%%%%%%%%%%%%%%%%%%%%% 

%%% GCP
\textbf{グラフ彩色問題} (GCP; Graph Coloring Problem)は,
与えられた有限無向グラフ$G$について,隣接する頂点が同色にならない
ように各頂点を塗りわけるとき,必要となる最小の色数 (彩色数と呼ばれる)
を求める問題である.
グラフ彩色問題はNP困難な問題であり,最適化コンパイラのレジスタ割
り付けや無線の周波数割り当て等の応用がある.
自然数$k\geq 3$について,グラフ$G$が$k$色以下で彩色可能かどう
かを決定するグラフ彩色判定問題はNP完全である.
$G$が$k$色以下で彩色可能で,$k-1$色以下で彩色可能でないとき,
$G$の彩色数は$k$となる.

グラフ彩色における\textbf{同色頂点数最小化(最大化)問題}とは,
グラフ彩色判定問題の実行可能解のうち,
同色(例えば,赤色)で塗られる頂点数の最小値(最大値)を求める問題である.
\textbf{多色頂点数最大化問題}は,
グラフ彩色判定問題の制約を満たしつつ,
多色(2色以上)で塗られる頂点数の最大値を求める問題である.
これらの問題は,グラフ彩色問題の関連問題として,
Knuth の教科書 The Art of Computer Programming (TAOCP) でも取り
上げられている~\cite{Knuth:TAOCP:SAT}.

%%% ASP
\textbf{解集合プログラミング}(Answer Set Programming; ASP\cite{%
  Baral03:cambridge,%
  Gelfond88:iclp,%
  Inoue08:jssst,%
  Niemela99:amai})
は論理プログラミングから派生した比較的新しいプログラミングパラダイムである.
ASP 言語は一階論理に基づいた知識表現言語の一種である.
論理プログラムはルールの有限集合である.
ASP システムは安定モデル意味論~\cite{Gelfond88:iclp}
に基づく解集合を計算するシステムである.
近年,SAT 技術~\cite{JSAI:InoueT10}
を応用した高速な ASP システムが開発され,
スケジューリング,プランニング,
制約充足問題,制約最適化問題,有界モデル検査
など様々な分野への実用的応用が急速に拡大している~\cite{ergele16a,anor/Banbara2019}.
% ASPシステムは豊富な表現力が特徴である.
% 個数制約や選択子といった構文があり,
% 複雑な問題の制約を簡潔に記述することが可能である.
% グラフ彩色問題とその関連問題に対してASPを用いる利点としては,
% ASP言語の高い表現力,充足不能コアに基づく最適化,
% 高速な解の列挙等が挙げられる.

%%%%%%%%%%%%%%%%%%%%%%%%%%%%%%%%%%%%%%%%%%%%%%%%%%%%%%%%%%
% Proposal
本論文では,グラフ彩色における同色頂点数最小化問題,
同色頂点数最大化問題,多色頂点数最大化問題を解く
ASP 符号化について述べる.
これらの問題に対して ASP を用いる利点としては,
ASP 言語の高い表現力,
充足不能コアに基づく最適化探索,
高速な解の列挙等
が挙げられる.

同色頂点数最小化問題と同色頂点数最大化問題を解く ASP 符号化は,
制約充足問題の SAT 符号化法の一つである直接符号化法
に基づいている~\cite{JSAI:TamuraTB10}.
一方,多色頂点数最大化問題を解くASP 符号化は,多値符号化法をベースにしている.
多値符号化法は各変数が複数の値を取ることを許す.そのため,
多色頂点数最大化問題の一つの解は,基のグラフ彩色判定問題の複数の実行可
能解に対応する圧縮解とみなすことができる.

%%% 実験内容
提案符号化の有効性を評価するために,
\textsf{McGregor}グラフ~\cite{Knuth:TAOCP:SAT}をベンチマークとして,
高速 ASP システム \textsf{clingo}を用いて実行実験を行った.
その結果,提案符号化は,
同色頂点数最小化問題,同色頂点数最大化問題,多色頂点数最大化問題のすべ
てについて,Knuth の教科書に記載されていない新しい最適解を発見すること
に成功した.
% これらの結果は,\textsf{clingo}に実装された充足不能コアに
% 基づく最適化探索によって得られた.
また,$N=12$次の\textsf{McGregor}グラフに対する
多色頂点数最大化問題の最適解が,
基のグラフ彩色判定問題の約680億の実行可能解を
表現していることが確認できた.


% 新たにベンチマークを作成した.
% 本研究では
% \code{McGregor}グラフ~\cite{Knuth:TAOCP:SAT}の
% ~\code{order}3$\sim$140までの計138問を作成した.
% \textsf{color}~符号化では作成したベンチマーク138問を使用し,
% 実験を行った.
% その結果,136問において彩色できると判定した.
% また,グラフ彩色問題を全解列挙する実験をベンチマーク8問を使用し行った.
% その結果,5問において全解列挙することができた.
% \textsf{minimize}~符号化では作成したベンチマーク問題の内,
% 18問を使用し実験を行った.
% その結果,15問の最適値と新たに2問の解を発見した.
% \textsf{maximize}~符号化では作成したベンチマーク問題の内,
% 36問を使用し実験を行った.
% その結果,31問の最適値と新たに17問の解を発見した.
% \textsf{mult}~符号化では作成したベンチマーク問題の内,
% 13問を使用し実験を行った.
% その結果,10問の最適値と新たに2問の解を発見した.
% また,グラフ彩色問題の全解列挙と\textsf{mult}~符号化の実験結果より,
% グラフ彩色における多色頂点数最大化問題で得られる解の圧縮率を求めた.
% その結果,グラフのサイズが大きいほど圧縮率が減少していくことが確認できた.

%%%%%%%%%%%%%%%%%%%%%%%%%%%%%%%%%%%%%%%%%%%%%%%%%%%%%%%%%%

以降,本論文の構成は以下の通りである.
第\ref{chap:background}章では本論文が対象とする
グラフ彩色における同色頂点数最小化問題,同色頂点数最大化問題,多色頂点
数最大化問題について述べる.
第\ref{chap:asp}章では解集合プログラミングについて述べる.
第\ref{chap:encoding}章では提案する ASP 符号化について述べる.
第\ref{chap:experiments}章で実験結果を示したのち,
第\ref{chap]conclusion}章で本論文の結論を述べる.

%%% Local Variables:
%%% mode: latex
%%% TeX-master: "paper"
%%% End:
