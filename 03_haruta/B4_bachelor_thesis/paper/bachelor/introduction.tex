%%%%%%%%%%%%%%%%%%%%%%%%%%%%%%%%%%%%%%%%%%%%%%%%%%%%%%%%%% 
\chapter{序論}
\pagenumbering{arabic}
%%%%%%%%%%%%%%%%%%%%%%%%%%%%%%%%%%%%%%%%%%%%%%%%%%%%%%%%%% 

\textbf{グラフ彩色問題(Graph Coloring)}は,
与えられたグラフに対して,
各頂点を隣接する頂点と異なる色で彩色する問題である.
グラフ彩色問題に対して,\code{k}彩色可能であるか判定する問題を
グラフ彩色判定問題と呼ぶ.
また,グラフ彩色問題において,
同色で彩色される頂点数を最小化,最大化する問題をそれぞれ,
グラフ彩色における同色頂点数最小化問題,同色頂点数最大化問題と呼ぶ.
さらに,グラフ彩色問題において,
2色以上で彩色できる頂点数を最大化する問題を
グラフ彩色における多色頂点数最大化問題と呼ぶ.
本研究ではグラフ彩色判定問題と
その発展問題である,グラフ彩色における同色頂点数最小化問題,
同色頂点数最大化問題,多色頂点数最大化問題を対象とする.


%%%%%%%%%%%%%%%%%%%%%%%%%%%%%%%%%%%%%%%%%%%%%%%%%%%%%%%%%% 
%解集合プログラミング
解集合プログラミング(Answer Set Programming; ASP\cite{%
  Baral03:cambridge,%
  Gelfond88:iclp,%
  Niemela99:amai,%
  Inoue08:jssst})
は論理プログラミングから派生した比較的新しいプログラミングパラダイムであり,
一般拡張選言プログラムをベースとしている.
ASP言語は一階論理に基づいた知識表現言語の一種である.
論理プログラムはルールの有限集合である.
ASPシステムは安定モデル意味論~\cite{Gelfond88:iclp}
に基づく解集合を計算するシステムである.
近年,SATソルバー技術を応用した高速なASPシステムが開発され,
スケジューリング,プランニング,
制約充足問題,制約最適化問題,有界モデル検査
など様々な分野への実用的応用が急速に拡大している.
ASPシステムは豊富な表現力が特徴である.
個数制約や選択子といった構文があり,
複雑な問題の制約を簡潔に記述することが可能である.
グラフ彩色問題とその関連問題に対してASPを用いる利点としては,
ASP言語の高い表現力,充足不能コアに基づく最適化,
高速な解の列挙等が挙げられる.

%%%%%%%%%%%%%%%%%%%%%%%%%%%%%%%%%%%%%%%%%%%%%%%%%%%%%%%%%%
%符号化
本論文では,\code{McGregor}グラフ~\cite{Knuth:TAOCP:SAT}において.
解集合プログラミング(ASP)を用いた
グラフ彩色判定問題,グラフ彩色における同色頂点数最小化問題,
同色頂点数最大化問題,多色頂点数最大化問題の解法
について述べる.
本研究では,グラフをASPファクト形式で表現した.
また,グラフ彩色判定問題を始めとした各問題を解くASP符号化
\textsf{color},\textsf{minimize},\textsf{maximize},\textsf{mult}
を提案した.
\textsf{color}~符号化はグラフ彩色判定問題を解く符号化である.
\textsf{minimize}~符号化は\textsf{color}~符号化をベースに,
ASPシステムの最小化関数を用いて,
グラフ彩色における同色頂点数最小化問題を解く符号化である.
\textsf{maximize}~符号化は\textsf{color}~符号化をベースに,
ASPシステムの最小化関数を用いて,
グラフ彩色における同色頂点数最大化問題を解く符号化である.
\textsf{mult}~符号化は\textsf{color}~符号化をベースに,
ASPシステムの個数制約と最大化関数を用いて,
グラフ彩色における多色頂点数最大化問題を解く符号化である.

%%%%%%%%%%%%%%%%%%%%%%%%%%%%%%%%%%%%%%%%%%%%%%%%%%%%%%%%%%
%実験内容
提案した4種の符号化を評価するために,
新たにベンチマークを作成した.
本研究では
\code{McGregor}グラフ~\cite{Knuth:TAOCP:SAT}の
~\code{order}3$\sim$140までの計138問を作成した.
\textsf{color}~符号化では作成したベンチマーク138問を使用し,
実験を行った.
その結果,136問において彩色できると判定した.
また,\textsf{minimize}~符号化では作成したベンチマーク問題の内,
18問を使用し実験を行った.
その結果,15問の最適値と新たに2問の解を発見した.
\textsf{maximize}~符号化では作成したベンチマーク問題の内,
36問を使用し実験を行った.
その結果,31問の最適値と新たに16問の解を発見した.
\textsf{mult}~符号化では作成したベンチマーク問題の内,
13問を使用し実験を行った.
その結果,10問の最適値と新たに2問の解を発見した.

%%%%%%%%%%%%%%%%%%%%%%%%%%%%%%%%%%%%%%%%%%%%%%%%%%%%%%%%%%

本論文の構成は以下の通りである.
第2章では本論文が対象とする
グラフ彩色問題とその関連問題について述べる.
第3章では解集合プログラミングについて述べる.
第4章ではグラフ彩色判定問題とその関連問題を解く
ASP符号化について示す.
第5章では提案した符号化の実行実験について述べる.
最後に,第6章で本論文の結論を述べる.


%%% Local Variables:
%%% mode: latex
%%% TeX-master: "paper"
%%% End:
