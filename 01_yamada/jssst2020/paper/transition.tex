\section{遷移問題への拡張}


\subsection{遷移問題の概要}
根付き全域森の遷移問題への拡張について,新たに
\textbf{初期状態}と\textbf{目的状態}を定義する.初期状態と
目的状態は,それぞれ対象となるグラフの根付き全域森の実行
可能解である.

根付き全域森の遷移問題は,入力として初期状態と目的状態が
与えられ,各状態で根付き全域森の制約を満たしながら,
状態が遷移する際に変化する辺の数は$d$個以下である.という
\textbf{遷移制約}を新たに設定する.

\ref{fig:trans}は,遷移制約を$d=2$とした場合での遷移問題の例である.

\begin{comment}
%%%%%%%%%%%%%%%%%%%%%%%%%%%%%%%%%
\begin{figure}[htbp]
 \centering
 \begin{subfigure}{}
  \centering
  \input{tikz/tikz-trans-1}
  \caption{$t=0$ (初期状態)}
 \end{subfigure}
 \hspace{1cm}
 \begin{subfigure}{}
  \centering
  \input{tikz/tikz-trans-2}
  \caption{$t=1$}
 \end{subfigure}
 %\\ \vspace{0.1cm}
 \begin{subfigure}{}
  \centering
  %%%%%%%%%%%%%%%%%%%%%%%%%%%%%%%%%%%%%%%%%%%%%%%%%%
% 実行例(t=3) (第6章で使う)
%%%%%%%%%%%%%%%%%%%%%%%%%%%%%%%%%%%%%%%%%%%%%%%%%%
\begin{tikzpicture}[scale=0.6]

 % 設定
 \tikzset{node/.style={circle,draw=black,fill=white}}

 \definecolor{edge1}{RGB}{191,0,0}
 \definecolor{node1}{RGB}{249,200,200}
 \definecolor{edge3}{RGB}{38,38,134}
 \definecolor{node3}{RGB}{200,200,249}

 % 補助線
 % \draw [help lines,blue] (0,0) grid (20,6);

 % node %
 \node[circle, ultra thick, draw=edge1, fill=node1](out1){1};
 \node[node, fill=node3, right=of out1] (out2){2};
 \node[circle, ultra thick, draw=edge3,fill=node3, right=of out2](out3){3};
 \node[node, fill=node3, below=of out1] (out4){4};
 \node[node, fill=node3, below=of out2] (out5){5};
 \node[node, fill=node3, below=of out3] (out6){6};

 \foreach \u / \v in {}
 \draw [very thick, edge1] (\u) -- (\v);

 \foreach \u / \v in {out2/out3,out2/out5,out4/out5,out5/out6}
 \draw [very thick, edge3](\u) -- (\v);
\end{tikzpicture}

%%%%%%%%%%%%%%%%%%%%%%%%%%%%%%%%%%%%%%%%%%%%%%%%%%%%%%%%%%
%%% Local Variables:
%%% mode: japanese-latex
%%% TeX-master: paper.tex
%%% End:

  \caption{$t=3$ (目的状態)}
 \end{subfigure}
 \hspace{1cm}
 \begin{subfigure}{}
  \centering
  \input{tikz/tikz-trans-3}
  \caption{$t=2$}
 \end{subfigure}
 %
 \caption{実行例(コード\ref{code:transition.log})のグラフ表現}
 \label{fig:result-trans}
\end{figure}
%%%%%%%%%%%%%%%%%%%%%%%%%%%%%%%%% 
\end{comment}

\subsection{ASP符号化}
今回考案した符号化は,外部プログラムによるループを用いたインクリメンタル探索を
行うものと,ASPソルバー{\clingo}のインクリメンタル探索ライブラリを使用した2種類の
符号化である.

外部プログラムを用いる符号化をコード\ref{code:roop}に示す.遷移問題への拡張について
基本的には,コード\ref{code:srf2}に出現する各アトムに,状態数を表す引数\code{T}を
追加する.これにより各状態\code{T}における根付き全域森を探索する.1行目のファクト
\code{t(0..t)}は,0から$t$までの各状態数を定義している.なお,$t$は外部プログラムに
よって与えられる変数($t \geq 0$)である.この状態数を増やしていくことでインクリメンタル
探索を行う.

4行目のルールは,与えられる初期状態と状態0における根付き全域森が一致する
ことを表す制約である.アトム\code{init_Forest(X,Y)}は初期状態での根付き全域森に含まれる
辺を表す.
同様に7行目のルールで,与えられる終了状態と状態$t$における根付き全域森が一致することを
表す.アトム\code{goal_Forest(X,Y)}は終了状態での根付き全域森に含まれる辺を表す.
10〜21行目までは,先に述べたように根付き全域森の各制約に状態数\code{T}を追加したルール
である.

24行目から26行目までは,遷移制約を表すルールであり,24,25行目は,状態\code{T-1}で森に
含まれていない辺が状態\code{T}で森に含まれる辺となった場合,あるいはその逆
をアトム\code{dist(X,Y,T)}として生成する.26行目で各状態\code{T}について変化した辺の数
である\code{dist(X,Y,T)}の数が$d$以下であることを一貫性制約で表している.

次に,ライブラリを用いたインクリメンタル探索を行う符号化をコード\ref{code:incmode}に示す.

