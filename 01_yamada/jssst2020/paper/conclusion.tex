\section{おわりに}\label{chap:conc}

本稿では,解集合プログラミングを用いた配電網問題に関する考察を述べた.
トポロジ制約のみを満たす配電網問題を根付き全域森探索問題として帰着させる
ことで,根付き全域森の各制約をASP言語の高い表現力によって数個のルールによって
表現した.特に改良符号化の特徴である符号化を適用した際の制約数が少なく抑える
ことができるという点は,評価実験において,基本符号化よりも高い拡張性を持つ
ことを示した.また,改良符号化は,辺の数が40,000を超えるような大規模な問題も
解いており,配電網問題に対するASPの有効性が確認できた.

遷移問題への拡張においては,初期状態と目的状態を設定し,新たに遷移制約を追加する
ことで拡張を行った.これらの制約条件を追加しても,ASP言語の高い表現力より6つ程度の
ルールを追加するだけで符号化を記述することができた.また,今回考案した2種類の手法の
うち,外部プログラムによる探索よりも\clingo のライブラリを活用することで高速に遷移
問題を解くことができることを評価実験によって確認した.実験では,遷移数が10回という
問題を解くことも確認でき,遷移問題に対してもASPの有効性が確認できた.

今後の課題として,今回はトポロジ制約のみを対象としたため,電気制約への対応を考えて
いく必要がある.しかし,電気制約は複雑な実数の計算を扱う必要があるため,純粋なASP
ソルバーだけで対応することは極めて困難であると思われる.そのため,トポロジ制約のみ
をASPソルバーで解き,その解について背景理論ソルバーを適用し,最適化探索を行うなどの
実装方法について考えていきたい.

また,遷移問題についても,ASPの利点である拡張性を活用して,さらなる応用として遷移問題
に関する様々な制約を追加することができるので,現実の配電網問題の背景を反映するような
制約を追加した際の評価についても行いたい.
