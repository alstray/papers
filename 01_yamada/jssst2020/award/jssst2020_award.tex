% 以下の3行は変更しないこと.
\documentclass[T]{compsoft}
\taikai{2020}
\pagestyle {empty}

\usepackage [dvipdfmx] {graphicx}

% ユーザが定義したマクロなどはここに置く.ただし学会誌のスタイルの
% 再定義は原則として避けること.

\begin{document}

% 論文のタイトル
\title{受賞者による受賞研究紹介}

% 著者
% 和文論文の場合,姓と名の間には半角スペースを入れ,
% 複数の著者の間は全角スペースで区切る
%
\author{山田 健太郎}

\maketitle \thispagestyle {empty}

%%% 本文 %%%
\section{解集合プログラミングを用いた配電網問題の解法に関する一考察 (山田 健太郎)}
この度は学生奨励賞にご選出いただきまして,ありがとうございます.
名誉ある賞を受賞することができ,大変嬉しく思っております.

解集合プログラミング(Answer Set programming; ASP)は,論理プログラミングから
派生した宣言的プログラミングパラダイムです.近年では,SATソルバーの実装技術
を応用した高速ASPシステムが実現され,プランニングやシステム生物学など様々な
分野への応用が拡大しています.

本研究では,このASP技術を求解困難な組合せ最適化問題の
1つである配電網問題に適用するために,問題が満たすべき制約のうち,供給経路に関する
トポロジ制約に注目し,根付き全域森問題とその解の遷移問題についての解法を提案しました.

今後の課題としては,もう一つの制約である電気制約への対応を行っていきたいと考えています.
しかし,電気制約は,電流の適正範囲,電圧の適正範囲など,様々な複雑な制約があるため,
さらなる研究調査をしていくとともに,背景理論付きASP (ASP Modulo Thories)を用いた実装を予定しています.

また研究に関して,私が魅力を感じている点としましては,ASPの高い汎用性だと思っています.
具体的には,ASP言語は高い表現力を持つので,問題の制約を簡潔に記述できるということ,さらには,
制約を記述するだけで,解きたい問題ごとに専用の解くアルゴリズムを考えなくても
ASPソルバーによって,ある程度高速に解を求めることが出来るということが挙げられます.
ASPは,主にAIの分野で研究されていますが,現在主流の機械学習の影に隠れてしまっていると思われます.
なので,本研究はまだ初期段階ではありますが,研究を進めることでASP分野の発展の一助になることができれば
よいと考えています.

最後になりますが,共著者の皆様,研究に関して様々なご意見をくださった方々に感謝申し上げます.
本受賞を糧に今後の研究の発展に向け,精進して参ります.


\end{document}
