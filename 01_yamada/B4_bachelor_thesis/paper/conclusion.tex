%%%%%%%%%%%%%%%%%%%%%%%%%%%%%%%%%%%%%%%%%%%%%%%%%%%%%%%%%% 
\chapter{結論} \label{chap:conc}
%%%%%%%%%%%%%%%%%%%%%%%%%%%%%%%%%%%%%%%%%%%%%%%%%%%%%%%%%%

%% まとめ
本研究では,ASPを用いた根付き全域森探索問題の解法について述べた.
考案した2種類のASP符号化(\code{srf1},\code{srf2})は,
根付き全域森の制約をASP言語の高い表現力によって数行で表現することができた.
特に,\code{srf2}符号化の特長である,基礎化後の制約数を少なく抑えるという点は,
評価実験において,\code{srf1}符号化よりも多くの問題を解く結果を示した.
また,\code{srf2}符号化は,辺の数が40,000を超える大規模な問題に対して
解を求められたという高い拡張性も示した.
さらには,今回行った評価実験で使った問題の多くを高速に
解いており,配電網問題に対するASPの有効性が確認できた.

遷移問題への拡張については,設定した新たな制約を追加しても,
ASP言語の高い表現力により,
5つのルールを追加することで問題を解くことが可能であることが確認できた.


%% 今後の課題
今後の課題として,今回の研究では遷移問題の評価実験を行うことが出来なかったので,
それについて行いたい.さらなる遷移問題の応用として,様々な制約を追加することが
可能であるので,問題背景から考えられる制約の設定を追加し,その評価を行いたい.

また,トポロジ制約のみを対象とした配電網問題においては,ASPの有効性が確認できた.
しかし,電力制約においては複雑な実数の計算を扱う必要があり,
ASPソルバーだけでは実数による制約を扱うのが難しい.そのため,
トポロジ制約のみをASPソルバーで解き,その解について背景理論ソルバーを適用し,
最適化探索を行うなどの実装方法について考えたい.

%%% Local Variables:
%%% mode: japanese-latex
%%% TeX-master: "paper"
%%% End:
