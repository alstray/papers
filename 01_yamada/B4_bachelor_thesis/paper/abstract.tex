%%%%%%%%%%%%%%%%%%%%%%%%%%%%%%%%%%%%%%%%%%%%%%%%%%%%%%%%%% 
\chapter*{概要}
\pagenumbering{roman}
%%%%%%%%%%%%%%%%%%%%%%%%%%%%%%%%%%%%%%%%%%%%%%%%%%%%%%%%%% 

本論文では,解集合プログラミングを用いた配電網問題の解法について述べる.
配電網問題は求解困難な組合せ最適化問題の一種である.
配電網問題は,トポロジ制約と,電気制約を満たしつつ,
電力の損失を最小にするスイッチの開閉状態を求めることが目的である.
研究対象とするトポロジ制約のみの配電網問題は,
与えられた連結グラフと根と呼ばれるノードから,
根付き全域森を探索する問題に帰着できることが知られている.

解集合プログラミング(ASP)は,
論理プログラミングから派生したプログラミングパラダイムである.
ASPシステムは,一階論理に基づくASP言語によって記述された論理プログラムから
解集合を計算するシステムである.

本論文では,ASPを用いた根付き全域森探索問題
(SRFP)の解法について述べる.
まず,根付き全域森探索問題を解く2種類のASP符号化
\code{srf1}と\code{srf2}を考案した.
\code{srf1}符号化は,SRFPの根付き連結制約を
at-least-one制約とat-most-one制約で表現した基本的な符号化に対し,
\code{srf2}符号化は,根付き連結制約をASPの個数制約を用いて表現している点が
特長である.
\code{srf2}符号化は,\code{srf1}符号化と比較すると基礎化後の制約数を
少なく抑えることができるため,大規模な問題に対する有効性が期待できる.

考案した符号化の有効性を評価するために,
様々なスイッチ数をもつ85問の問題を用いて,
実行実験を行なった.
その結果,
\code{srf2}符号化は,\code{srf1}符号化より多くの問題を
解くことに成功した.
また,\code{srf2}符号化はスイッチ数が40,000個を超えるような大規模な問
題も解いており,配電網問題に対する ASP の有効性が確認できた.

%%% Local Variables:
%%% mode: japanese-latex
%%% TeX-master: "paper"
%%% End: