%%%%%%%%%%%%%%%%%%%%%%%%%%%%%%%%%%%%%%%%%%%%%%%%%%
%% 提案手法の比較表
%%%%%%%%%%%%%%%%%%%%%%%%%%%%%%%%%%%%%%%%%%%%%%%%%%
\begin{tabular}[tb]{cp{6cm}p{2.5cm}}

 \rowcolor[RGB]{60,60,134}
 \textcolor{white}{符号化} &
	 \multicolumn{1}{c}{\textcolor{white}{根付き連結制約の表現方法}} & 
		 \begin{tabular}{c} 
		  \textcolor{white}{~基礎化後の}\\\textcolor{white}{~ルール数} 
		 \end{tabular} \\
 \rowcolor[RGB]{230,230,249}
   \begin{tabular}{c}提案\\符号化1\end{tabular} & 
	 \begin{tabular}{l} at-least-one制約と\\
	 at-most-one制約を用いて表現 \end{tabular}& 
	 \vspace{-0.4cm}\alert{$|V|\left(1+ {|R|\choose 2} \right)$} \\
 \rowcolor[RGB]{210,210,255}
  \begin{tabular}{c}提案\\符号化2 \end{tabular} & 
	  \begin{tabular}{l} ASPの個数制約を用いて表現 \end{tabular} & 
		 \multicolumn{1}{c}{~\alert{$|V|$}} 
\end{tabular}
 