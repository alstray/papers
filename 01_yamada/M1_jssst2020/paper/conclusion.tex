\section{おわりに}\label{chap:conc}

本稿では,根付き全域森問題とその解の遷移問題について,
解集合プログラミング技術を用いた解法を提案した.
%
根付き全域森問題を解く2種類のASP符号化を考案した.
特に,改良符号化は,根付き全域森問題の連結制約をASPの個数制約で表現す
ることにより,基礎化後のルール数を少なく抑えるよう工夫されている.
%
解の遷移問題については,マルチショットASP解法を利用して解く方法を
提案した.この方法は,ASPシステムが同様の探索失敗を避けるために獲得し
た学習節を(部分的に)保持することで,無駄な探索を行うことなく,制約を追
加した問題を連続的に解くことができる.
%
DNETの問題集,および,Graph Coloring and its Generalizations の問題を
元に生成した問題集を用いた実験の結果,
大規模な根付き全域森問題に対する改良符号化の有効性,
遷移問題に対するマルチショットASP解法の優位性が確認できた.
%
今後の課題としては,
配電網問題の電気制約を,背景理論付きASP (ASP Modulo Theories) を用いて
実装することが挙げられる.

%%% Local Variables:
%%% mode: japanese-latex
%%% TeX-master: "paper"
%%% End:
