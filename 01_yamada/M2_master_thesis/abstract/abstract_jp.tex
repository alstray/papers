\documentclass[dvipdfmx,a4paper]{jsarticle}

\title{\vspace{-3cm}解集合プログラミングを用いた\\配電網問題の解法に関する研究}
\author{氏名:山田 健太郎}
\date{学生番号:252005214}

\usepackage{url}
\pagestyle{empty}

\begin{document}
\maketitle

配電網の効率的な構成および制御は,スマートグリッドや災害時の停電復旧などを
支える重要な研究課題である.
%
\textbf{配電網問題}
(Power Distribution Network Problem)%~\cite{Minato:dnet:ZDD,DBLP:journals/tsg/InoueTWKYKTMH14})
は,一般に,供給経路に関する\textbf{トポロジ制約}と,
電流・電圧に関する\textbf{電気制約}を満たす%配電網の構成
スイッチの開閉状態
を求める問題である.
%
\textbf{配電網遷移問題}は,配電網問題とその2つの実行可能解が与え
られたとき,一方の解から他方の解へ,遷移制約を満たしつつ実行可能解のみ
を経由して到達できるかどうかを判定し,到達可能であればその最短経路を求
める問題である.

\textbf{解集合プログラミング}(Answer Set Programing; ASP)は,
論理プログラミングから派生したプログラミングパラダイムである.
ASPシステムは,一階論理に基づくASP言語によって記述された論理プログラムから
解集合を計算するシステムである.
配電網(遷移)問題に対して ASP を用いる利点としては,
ASP 言語の高い表現力,
高速な解列挙,
遷移問題を解くために有用なマルチショットASP解法
などが挙げられる.

本論文では,電気制約として電流制約のみを考慮した配電網問題および配電網
遷移問題に対して,解集合プログラミングを用いた解法を提案する.
提案解法では,まず与えられた問題インスタンスを ASP のファクト形式に変
換した後,そのファクトと配電網(遷移)問題を解くための ASP 符号化を結合
した上で,高速 ASP システムを用いて解を求める.

配電網問題を解くための ASP 符号化は,トポロジ制約,電流制約の2つのパー
トから構成される.さらに,トポロジ制約は非閉路制約と根付き連結制約の2つ
から構成される.
トポロジ制約に対して,基本符号化,改良符号化,有向符号化の3つを考案した.
基本符号化は,根付き連結制約を \textit{at-least-one} 制約と \textit{at-most-one} 制約で
表現した基本的な符号化である.
改良符号化は,根付き連結制約を ASP の個数制約で表現することにより,
基礎化後のルール数を少なく抑えられる.
有向符号化は,無向グラフの各辺$u-v$に対して,2つの弧
$u\rightarrow v$と$v\rightarrow u$を対応させることで有向グラフ
化して解く符号化である.
この有向グラフ化により,非閉路制約を簡潔に表現できる.

配電網遷移問題を解くには,複数の配電網問題を繰り返し解く必要がある.
しかし,各問題中の制約の大部分は共通であるため,ASP システムが同一の
探索空間を何度も調べることになり,求解効率が低下するという問題点がある.
この問題を解決するために,マルチショット ASP 解法%~\cite{DBLP:conf/rweb/KaminskiSW17}
を利用した符号化を提案する.
この符号化は,配電網問題の ASP 符号化の自然な拡張となっている.
マルチショット ASP 解法を利用することにより,
ASP システムが同様の探索失敗を避けるために
獲得した学習節を
(部分的に)保持することで,無駄な探索を行うことなく,制約を追加した論理
プログラムを連続的に解くことができる.

提案解法の有効性を評価するために,
DNET% (Power Distribution Network Evaluation Tool)
~\footnote{%
\url{https://github.com/takemaru/dnet}}
に公開されている配電網問題を含む,様々なスイッチ数をもつ,トポロジ制約のみの配電網問題
を用いて実行実験を行なった.
その結果,有向符号化は,基本符号化と改良符号化と比較して,より多くの問
題をより高速に解くことが確認できた.
%
配電網遷移問題の実行実験については,DNETで公開されている実用規模の
問題({\sf fukui-tepco})に対して,実行可能解のペアをランダムに選び,
合計 1000 問の配電網遷移問題を生成しベンチマーク問題として使用した.
その結果,すべての問題の到達可能性を判定することができ,
得られた最短ステップ長の最大値は7であった.また,マルチショットASP解法を
利用することで,通常の解法と比較して,平均で3.8倍の高速化を実現した.


\end{document}