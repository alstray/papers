%%%% 補助スライド
\appendix
\backupbegin

\begin{frame}{~}
 \centering
 - 補足用 -
\end{frame} 
%%%%%%%%%%%%%%%%%%%%%%%%%%%%%%%%%%%%%%%%%%%%%%%%%% 
%% 根付き全域森問題
%%%%%%%%%%%%%%%%%%%%%%%%%%%%%%%%%%%%%%%%%%%%%%%%%%
\begin{frame}{根付き全域森問題}
 \begin{alertblock}{}
  トポロジ制約を満たす配電網構成は,グラフと根と呼ばれる特別なノードから,
  \alert{\bf 根付き全域森}を求める部分グラフ探索問題に帰着できる.
 \end{alertblock}
 \vfill
 \begin{block}{根付き全域森 (Spanning Rooted Forest) [川原・湊 '12]}
  グラフ$G=(V,E)$と,
  \textbf{根}と呼ばれる$V$上のノードが与えられたとき,
  $G$上の根付き全域森とは,以下の条件を満たす$G$の部分グラフ$G'=(V,E'),\ E' \subseteq E$である.
  \begin{enumerate}
   \item $G'$はサイクルを持たない. (\alert{\bf 非閉路制約})
   \item $G'$の各連結成分は,ちょうど1つの根を含む. (\alert{\bf 根付き連結制約})
  \end{enumerate}
 \end{block}
\end{frame}
%%%%%%%%%%%%%%%%%%%%%%%%%%%%%%%%%%%%%%%%%%%%%%%%%%
%% 根付き全域森問題の例
%%%%%%%%%%%%%%%%%%%%%%%%%%%%%%%%%%%%%%%%%%%%%%%%%%
\begin{frame}{根付き全域森問題の例}
  \begin{columns}
    \begin{column}{0.45\textwidth}\centering
      \begin{exampleblock}{入力例}
	\centering
	%%%%%%%%%%%%%%%%%%%%%%%%%%%%%%%%%%%%%%%%%%%%%%%%%%
% 根付き全域森の例
%%%%%%%%%%%%%%%%%%%%%%%%%%%%%%%%%%%%%%%%%%%%%%%%%%

\begin{tikzpicture}[scale=0.5]

 % 設定
 \tikzset{node/.style={circle,draw=black,fill=white}}

 \definecolor{edge1}{RGB}{191,0,0}
 \definecolor{node1}{RGB}{249,200,200}
 \definecolor{edge3}{RGB}{38,38,134}
 \definecolor{node3}{RGB}{200,200,249}

 % 補助線
 % \draw [help lines,blue] (0,0) grid (20,6);

 % 入力されるグラフ
 % node %
 \node[circle, ultra thick,draw=edge1,fill=node1] (in1) {1};
 \node[node,right= of in1] (in2){2};
 \node[circle, ultra thick, draw=edge3,fill=node3, right=of in2](in3){3};
 \node[node,below= of in1] (in4){4};
 \node[node,below= of in2] (in5){5};
 \node[node,below= of in3] (in6){6};

 % 辺
 \foreach \u / \v in {in1/in2,in2/in3,in1/in4,in2/in5,in3/in6,in4/in5,in5/in6}
 \draw (\u) -- (\v);

\end{tikzpicture}

%%%%%%%%%%%%%%%%%%%%%%%%%%%%%%%%%%%%%%%%%%%%%%%%%%%%%%%%%%
%%% Local Variables:
%%% mode: japanese-latex
%%% TeX-master: ``slide''
%%% End:

      \end{exampleblock}
    \end{column}
    \begin{column}{0.05\textwidth}\centering
      $\Rightarrow$
    \end{column}
    \begin{column}{0.45\textwidth}\centering
      \begin{exampleblock}{解の例}
        \centering
        %%%%%%%%%%%%%%%%%%%%%%%%%%%%%%%%%%%%%%%%%%%%%%%%%%
% 根付き全域森の例
%%%%%%%%%%%%%%%%%%%%%%%%%%%%%%%%%%%%%%%%%%%%%%%%%%

\begin{tikzpicture}[scale=0.5]

 % 設定
 \tikzset{node/.style={circle,draw=black,fill=white}}

 \definecolor{edge1}{RGB}{191,0,0}
 \definecolor{node1}{RGB}{249,200,200}
 \definecolor{edge3}{RGB}{38,38,134}
 \definecolor{node3}{RGB}{200,200,249}

 % 補助線
 % \draw [help lines,blue] (0,0) grid (20,6);

 % node %
 \node[circle, ultra thick, draw=edge1, fill=node1](out1){1};
 \node[node, fill=node1, right=of out1] (out2){2};
 \node[circle, ultra thick, draw=edge3,fill=node3, right=of out2](out3){3};
 \node[node, fill=node1, below=of out1] (out4){4};
 \node[node, fill=node3, below=of out2] (out5){5};
 \node[node, fill=node3, below=of out3] (out6){6};

 \foreach \u / \v in {out1/out2,out1/out4}
 \draw [very thick, edge1] (\u) -- (\v);

 \foreach \u / \v in {out3/out6,out5/out6}
 \draw [very thick, edge3](\u) -- (\v);

\end{tikzpicture}

%%%%%%%%%%%%%%%%%%%%%%%%%%%%%%%%%%%%%%%%%%%%%%%%%%%%%%%%%%
%%% Local Variables:
%%% mode: japanese-latex
%%% TeX-master: ``slide''
%%% End:

      \end{exampleblock}
    \end{column}
  \end{columns}
  \vfill
  \begin{itemize}
  \item \structure{\bf 配電網とグラフの対応} \\
	 \begin{center}
      \begin{minipage}[c]{0.7\textwidth}
	   \begin{block}{}
		\centering
		\begin{tabular}{c|ccc}
		配電網 & セクション & スイッチ & 変電所 \\
		\hline
		グラフ & ノード & 辺 & 根
		\end{tabular}
	   \end{block}
      \end{minipage}
	 \end{center}\vfill
   \item \structure{\bf 配電網問題のトポロジ制約}
		 \begin{itemize}
		  \item 停電(変電所と結ばれないセクション)
		  \item 短絡(供給経路上のループ,複数の変電所と結ばれるセクション)
		 \end{itemize}
  \end{itemize}
\end{frame}

% %%%%%%%%%%%%%%%%%%%%%%%%%%%%%%%%%%%%%%%%%%%%%%%%%%
% %% 配電網問題の例
% %%%%%%%%%%%%%%%%%%%%%%%%%%%%%%%%%%%%%%%%%%%%%%%%%%
% \begin{frame}{配電網問題の例}
 
%  \begin{block}{配電網問題}
%   与えられた配電網$D=(S,SW)$から,以下の制約を満たす
%   配電網の構成(スイッチの開閉状態$X$)が存在するかどうかを判定する問題.
% %  \vspace{-0.5em}
%  \begin{itemize}\small
%   \item $X ~\textrm{によって定まる配電網構成に停電と短絡が発生しない}$ (\textbf{トポロジ制約})
%   \item $J_i = \displaystyle\sum_{j\in S_i^{down}} I_j, \quad J_i \leq J^{max} 
%         \quad (\forall s_{i}\in S)\qquad\qquad\qquad\qquad\quad~~$ (\textbf{電流制約})
%  \end{itemize}
%  \end{block}
%   \begin{columns}
%     \begin{column}{0.45\textwidth}\centering
%       \begin{exampleblock}{入力例}
% 	\centering
% 	\scalebox{0.3}{%%%%%%%%%%%%%%%%%%%%%%%%%%%%%%%%%%%%%%%%%%%%%%%%%%
% 配電網 例 (第1章で使う)
%%%%%%%%%%%%%%%%%%%%%%%%%%%%%%%%%%%%%%%%%%%%%%%%%%

\begin{tikzpicture}

 % setting
 \tikzset{customer/.style={rectangle,thick,draw=black,minimum size=0.5cm}}
 \tikzset{on_switch/.style={rectangle,fill=black}}
 \tikzset{off_switch/.style={rectangle,draw=black,fill=white}}
 
 \tikzset{node distance =1cm};

 % substation (x, y, label)
 \newcommand{\substation}[3]{
 \draw [very thick] (#1,#2) circle [radius=0.225cm] node[draw=white,minimum size=1cm](#3){};
 \draw [very thick] (#1+0.225,#2)--(#1+0.35,#2)--(#1+0.35,#2+0.3);
 \draw [very thick] (#1-0.225,#2)--(#1-0.35,#2)--(#1-0.35,#2-0.3);
 \draw [very thick] (#1,#2+0.225)--(#1,#2+0.35);
 \draw [very thick] (#1,#2-0.225)--(#1,#2-0.35);
 \draw [very thick] [domain=-0.284:-0.159] plot(\x+#1,\x+#2);
 \draw [very thick] [domain=0.159:0.284] plot(\x+#1,\x+#2);
 \draw [very thick] [domain=-0.284:-0.159] plot(\x+#1,-\x+#2);
 \draw [very thick] [domain=0.159:0.284] plot(\x+#1,-\x+#2);
 }

 %switch node (position, label, cap)
 %% right switch
 \newcommand{\swnodeR}[4]{
 \coordinate[#1] (#2);
 \node[#1,customer] (#2){#4};
 \node[circle, draw=black, text width=0.2cm, 
 right=0cm of #2, scale=0.3, thick] {};
 \node[right=0cm of #2,scale=0.3, minimum size=0.8cm] (#3){};
 }
 %% left switch
 \newcommand{\swnodeL}[4]{
 %\coordinate[#1] (#2);
 \node[#1,customer] (#2){#4};
 \node[circle, draw=black, fill=white, text width=0.2cm, 
 left=0cm of #2, scale=0.3, thick] (#3){};
 }
 % above switch
 \newcommand{\swnodeA}[4]{
 \coordinate[#1] (#2);
 \node[#1,customer] (#2){#4};
 \node[circle, draw=black, text width=0.2cm, 
 above=0cm of #2, scale=0.3, thick] (#3){};
 }
 % below switch
 \newcommand{\swnodeB}[4]{
 \coordinate[#1] (#2);
 \node[#1,customer] (#2){#4};
 \node[circle, draw=black, text width=0.2cm, 
 below=0cm of #2, scale=0.3, thick] {};
 \node[below=0cm of #2,scale=0.3,minimum size=0.8cm] (#3){};
 }
 
 \substation{0}{0}{sub};
 
 % root1
 \node[customer,fill=black!40,below =4.5cm of sub] (root1) { };
 \swnodeL{left =of root1}{node1}{sw1}{ };
 
 \swnodeR{left=of node1}{node2}{sw2}{ };
 \node[customer,left=of node2] (junc1){ };
 \swnodeL{left =of junc1}{node3}{sw3}{ };
 \swnodeA{above= of junc1}{node4}{sw4}{ }

 \swnodeR{left=of node3}{node5}{sw5}{ };
 \swnodeA{above =of node5}{node6}{sw6}{ };

 \swnodeB{above =of node6}{node7}{sw7}{ };
 \swnodeA{above =of node7}{node29}{sw29}{ };

 \swnodeB{above =of node29}{node30}{sw30}{ };
 
 \swnodeB{above =of node4}{node8}{sw8}{ };
 \swnodeA{above =of node8}{node31}{sw31}{ };
 
 \swnodeB{above =of node31}{node32}{sw32}{ };

 \swnodeR{right =of root1}{node17}{sw17}{ };

 \swnodeL{right =of node17}{node18}{sw18}{ };

 % root2
 \node[customer,fill=black!10,above=4.5cm of sub] (root2) { };
 \swnodeL{left =of root2}{node9}{sw9}{ };

 \swnodeR{left=of node9}{node10}{sw10}{ };
 \node[customer,left=of node10] (junc2){ };
 \swnodeL{left =of junc2}{node11}{sw11}{ };
 \swnodeB{below =of junc2}{node12}{sw12}{ };
 
 \swnodeR{left =of node11}{node13}{sw13}{ };
 \swnodeB{below =of node13}{node14}{sw14}{ };
 
 \swnodeA{below =of node14}{node15}{sw15}{ };

 \swnodeA{below =of node12}{node16}{sw16}{ };

 \swnodeR{right =of root2}{node22}{sw22}{ };

 \swnodeL{right =of node22}{node23}{sw23}{ };
 
 % root3
 \node[customer,pattern=north east lines,right=5.2cm of sub] (root3) { };
 \swnodeB{below =1.4of root3}{node24}{sw24}{ };
 \swnodeA{above =1.4of root3}{node25}{sw25}{ };

 \swnodeA{below =of node24}{node19}{sw19}{ };
 \swnodeL{below =1.3of node19}{node20}{sw20}{ };
 
 \swnodeR{left =of node20}{node21}{sw21}{ };

 \swnodeB{above =of node25}{node26}{sw26}{ };
 \swnodeL{above =1.3of node26}{node27}{sw27}{ };

 \swnodeR{left =of node27}{node28}{sw28}{ };
 
 % sections
 \foreach \v / \u / \t in {root1/sub/$s_a$,root1/node1/$s_1$,node2/junc1/$s_2$, %
 junc1/node3/$s_3$,junc1/node4/$s_4$,node5/node6/$s_5$,node7/node29/$s_6$,node30/node15/$s_7$, %
 sub/root2/$s_b$,root2/node9/$s_8$,node10/junc2/$s_9$,junc2/node11/$s_{10}$,node12/junc2/$s_{11}$, %
 node14/node13/$s_{12}$,node8/node31/$s_{13}$,node32/node16/$s_{14}$,node17/root1/$s_{15}$, %
 node22/root2/$s_{16}$,root3/sub/$s_c$,node24/root3/$s_{17}$,root3/node25/$s_{18}$, %
 node20/node19/$s_{19}$,node26/node27/$s_{20}$,node21/node18/$s_{21}$,node28/node23/$s_{22}$} %
 \draw[thick] (\v) --  node[auto=right]{\t} (\u);

 % switches
 %% horizontal
 \foreach \v / \u / \t in {sw1/sw2/$sw_{1}$,sw3/sw5/$sw_{2}$,sw9/sw10/$sw_{11}$,sw11/sw13/$sw_{12}$,sw18/sw17/$sw_{13}$,
 sw23/sw22/$sw_{15}$,sw20/sw21/$sw_{14}$,sw27/sw28/$sw_{16}$}
 \draw[very thick] (\v) -- node[below=0.2of \v]{\t} (\u.45);
 %% vertical
 \foreach \v / \u / \t in {sw4/sw8/$sw_{4}$,sw6/sw7/$sw_{3}$,sw29/sw30/$sw_{5}$,sw31/sw32/$sw_{6}$,sw15/sw14/$sw_{7}$,%
 sw16/sw12/$sw_{8}$,sw19/sw24/$sw_{9}$,sw25/sw26/$sw_{10}$}
 \draw[very thick] (\v) -- node[auto=below]{\t~~~~~~~~~~} (\u.-30);

\end{tikzpicture}

%%%%%%%%%%%%%%%%%%%%%%%%%%%%%%%%%%%%%%%%%%%%%%%%%%%%%%%%%%
%%% Local Variables:
%%% mode: japanese-latex
%%% TeX-master: paper.tex
%%% End:
}
%       \end{exampleblock}
%     \end{column}
%     \begin{column}{0.05\textwidth}\centering
%       $\Rightarrow$
%     \end{column}
%     \begin{column}{0.45\textwidth}\centering
%       \begin{exampleblock}{解の例}
%         \centering
%         \scalebox{0.3}{\input{tikz/tikz-test-output}}
%       \end{exampleblock}
%     \end{column}
%   \end{columns}
% \end{frame}
% %%%%%%%%%%%%%%%%%%%%%%%%%%%%%%%%%%%%%%%%%%%%%%%%%%
% %% 辺の数の表
% %%%%%%%%%%%%%%%%%%%%%%%%%%%%%%%%%%%%%%%%%%%%%%%%%%
% \begin{frame}{実験結果:解けた問題数による比較}
 
% %\begin{textblock*}{\linewidth}(10pt, 30pt)
% \begin{table}[t]
%  \centering
%  \begin{tabular}[t]{ccccc}
 \rowcolor[RGB]{0,96,0}
 \color{white} 辺の範囲 & \color{white}問題数 & 
		 \color{white}基本符号化 & \color{white}改良符号化 & \color{white}有向符号化\\
 %%%%%%%%
 \rowcolor[RGB]{230,239,230}
 ~~~~\;\:1 ~ 1,000 & 30 & \alert{30} & \alert{30} & \alert{30} \\
 \rowcolor[RGB]{196,230,196}
 1,001 ~ 4,000 & 20 & \alert{20} & \alert{20} & \alert{20} \\
 \rowcolor[RGB]{230,239,230}
 4,001 ~ 7,000 & 11 & 9 & 10 & \alert{11} \\
 \rowcolor[RGB]{196,230,196}
 ~\:7,001 ~ 10,000 & 8 & 4 & 6 &\alert{7} \\
 \rowcolor[RGB]{230,239,230}
 10,001 ~ 20,000 & 9 & 2 & 5 &\alert{9} \\
 \rowcolor[RGB]{196,230,196}
 20,001 ~ 30,000 & 2 & 1 & \alert{2} & \alert{2} \\
 \rowcolor[RGB]{230,239,230}
 30,001 ~ 40,000 & 1 & 0 & 0 & \alert{1} \\
 \rowcolor[RGB]{196,230,196}
 40,001 ~ 50,000 & 4 & 0 & 2 & \alert{4} \\
 %%%%%%%% 合計
 \noalign{\hrule height 0.5pt}
 \rowcolor[RGB]{230,239,230}
 計 & 85 & 66 & 75 & \alert{84} \\
 
\end{tabular}

% \end{table}\vfill

% \begin{itemize}
%  \item 有向符号化は,ベンチマーク問題85問中,\textbf{84問}を解いている.
%  \item 大規模な問題に対しても有向符号化は,優位性を示した.
% \end{itemize}\vfill
% %\end{textblock*}
% \end{frame}
% %%%%%%%%%%%%%%%%%%%%%%%%%%%%%%%%%%%%%%%%%%%%%%%%%%
% %% 配電網遷移問題の例
% %%%%%%%%%%%%%%%%%%%%%%%%%%%%%%%%%%%%%%%%%%%%%%%%%%
% \begin{frame}{配電網遷移問題の例(1/4)}
%  \vfill
%  \begin{figure}[t]
%   \centering
%   \scalebox{0.65}{\input{tikz/tikz-test-core-0}}
%   \vspace{-0.1cm}
%   \caption*{\structure{$\mathbf{t=0}$ \textbf{(スタート状態)}}}
%  \end{figure}
% \end{frame}
% %
% \begin{frame}{配電網遷移問題の例(2/4)}
%   \vfill
%  \begin{figure}[t]
%   \centering
%   \scalebox{0.65}{\input{tikz/tikz-test-core-1}}
%   \vspace{-0.1cm}
%   \caption*{\structure{$\mathbf{t=1}$}}
%  \end{figure}
% \end{frame}
% %
% \begin{frame}{配電網遷移問題の例(3/4)}
%   \vfill
%  \begin{figure}[t]
%   \centering\hspace{-0.1cm}
%   \scalebox{0.65}{\begin{tikzpicture}

 % setting
 \tikzset{customer/.style={rectangle,thick,draw=black,minimum size=0.5cm}}
 \tikzset{on_switch/.style={rectangle,fill=black}}
 \tikzset{off_switch/.style={rectangle,draw=black,fill=white}}
 
 \tikzset{node distance =1cm};

 % substation (x, y, label)
 \newcommand{\substation}[3]{
 \draw [very thick] (#1,#2) circle [radius=0.225cm] node[draw=white,minimum size=1cm](#3){};
 \draw [very thick] (#1+0.225,#2)--(#1+0.35,#2)--(#1+0.35,#2+0.3);
 \draw [very thick] (#1-0.225,#2)--(#1-0.35,#2)--(#1-0.35,#2-0.3);
 \draw [very thick] (#1,#2+0.225)--(#1,#2+0.35);
 \draw [very thick] (#1,#2-0.225)--(#1,#2-0.35);
 \draw [very thick] [domain=-0.284:-0.159] plot(\x+#1,\x+#2);
 \draw [very thick] [domain=0.159:0.284] plot(\x+#1,\x+#2);
 \draw [very thick] [domain=-0.284:-0.159] plot(\x+#1,-\x+#2);
 \draw [very thick] [domain=0.159:0.284] plot(\x+#1,-\x+#2);
 }

 %switch node (position, label, cap)
 %% right switch
 \newcommand{\swnodeR}[4]{
 \coordinate[#1] (#2);
 \node[#1,customer] (#2){#4};
 \node[circle, draw=black, text width=0.2cm, 
 right=0cm of #2, scale=0.3, thick] {};
 \node[right=0cm of #2,scale=0.3, minimum size=0.8cm] (#3){};
 }
 %% left switch
 \newcommand{\swnodeL}[4]{
 %\coordinate[#1] (#2);
 \node[#1,customer] (#2){#4};
 \node[circle, draw=black, fill=white, text width=0.2cm, 
 left=0cm of #2, scale=0.3, thick] (#3){};
 }
 % above switch
 \newcommand{\swnodeA}[4]{
 \coordinate[#1] (#2);
 \node[#1,customer] (#2){#4};
 \node[circle, draw=black, text width=0.2cm, 
 above=0cm of #2, scale=0.3, thick] (#3){};
 }
 % below switch
 \newcommand{\swnodeB}[4]{
 \coordinate[#1] (#2);
 \node[#1,customer] (#2){#4};
 \node[circle, draw=black, text width=0.2cm, 
 below=0cm of #2, scale=0.3, thick] {};
 \node[below=0cm of #2,scale=0.3,minimum size=0.8cm] (#3){};
 }
 
 \substation{0}{0}{sub};
 
 % root1
 \node[customer,fill=black!40,below =4.5cm of sub] (root1) { };
 \swnodeL{left =of root1,fill=black!40}{node1}{sw1}{ };
 
 \swnodeR{left=of node1,fill=black!40}{node2}{sw2}{ };
 \node[customer,left=of node2,fill=black!40] (junc1){ };
 \swnodeL{left =of junc1,fill=black!40}{node3}{sw3}{ };
 \swnodeA{above= of junc1,fill=black!40}{node4}{sw4}{ }

 \swnodeR{left=of node3,fill=black!40}{node5}{sw5}{ };
 \swnodeA{above =of node5,fill=black!40}{node6}{sw6}{ };

 \swnodeB{above =of node6,fill=black!40}{node7}{sw7}{ };
 \swnodeA{above =of node7,fill=black!40}{node29}{sw29}{ };

 \swnodeB{above =of node29,fill=black!10}{node30}{sw30}{ };
 
 \swnodeB{above =of node4,fill=black!40}{node8}{sw8}{ };
 \swnodeA{above =of node8,fill=black!40}{node31}{sw31}{ };
 
 \swnodeB{above =of node31,fill=black!10}{node32}{sw32}{ };

 \swnodeR{right =of root1,fill=black!40}{node17}{sw17}{ };

 \swnodeL{right =of node17,fill=black!40}{node18}{sw18}{ };

 % root2
 \node[customer,fill=black!20,above=4.5cm of sub, fill=black!10] (root2) { };
 \swnodeL{left =of root2,fill=black!10}{node9}{sw9}{ };

 \swnodeR{left=of node9,fill=black!10}{node10}{sw10}{ };
 \node[customer,left=of node10,fill=black!10] (junc2){ };
 \swnodeL{left =of junc2,fill=black!10}{node11}{sw11}{ };
 \swnodeB{below =of junc2,fill=black!10}{node12}{sw12}{ };
 
 \swnodeR{left =of node11,fill=black!10}{node13}{sw13}{ };
 \swnodeB{below =of node13,fill=black!10}{node14}{sw14}{ };
 
 \swnodeA{below =of node14,fill=black!10}{node15}{sw15}{ };

 \swnodeA{below =of node12,fill=black!10}{node16}{sw16}{ };

 \swnodeR{right =of root2,fill=black!10}{node22}{sw22}{ };

 \swnodeL{right =of node22,fill=black!10}{node23}{sw23}{ };
 
 % root3
 \node[customer,fill=black!20,right=5.2cm of sub,pattern=north east lines] (root3) { };
 \swnodeB{below =1.4of root3,pattern=north east lines}{node24}{sw24}{ };
 \swnodeA{above =1.4of root3,pattern=north east lines}{node25}{sw25}{ };

 \swnodeA{below =of node24,pattern=north east lines}{node19}{sw19}{ };
 \swnodeL{below =1.3of node19,pattern=north east lines}{node20}{sw20}{ };
 
 \swnodeR{left =of node20,fill=black!40}{node21}{sw21}{ };

 \swnodeB{above =of node25,fill=black!10}{node26}{sw26}{ };
 \swnodeL{above =1.3of node26,fill=black!10}{node27}{sw27}{ };

 \swnodeR{left =of node27,fill=black!10}{node28}{sw28}{ };
 
 % sections
 \foreach \v / \u / \t in {root1/sub/$s_a$,root1/node1/$s_1$,node2/junc1/$s_2$, %
 junc1/node3/$s_3$,junc1/node4/$s_4$,node5/node6/$s_5$,node7/node29/$s_6$,node30/node15/$s_7$, %
 sub/root2/$s_b$,root2/node9/$s_8$,node10/junc2/$s_9$,junc2/node11/$s_{10}$,node12/junc2/$s_{11}$, %
 node14/node13/$s_{12}$,node8/node31/$s_{13}$,node32/node16/$s_{14}$,node17/root1/$s_{15}$, %
 node22/root2/$s_{16}$,root3/sub/$s_c$,node24/root3/$s_{17}$,root3/node25/$s_{18}$, %
 node20/node19/$s_{19}$,node26/node27/$s_{20}$,node21/node18/$s_{21}$,node28/node23/$s_{22}$} %
 \draw[thick] (\v) --  node[auto=right]{\t} (\u);

 % switches
 %% horizontal
 \foreach \v / \u / \t in {sw1/sw2/$sw_{1}$,sw3/sw5/$sw_{2}$,sw9/sw10/$sw_{11}$,
 sw11/sw13/$sw_{12}$,sw18/sw17/$sw_{13}$,sw23/sw22/$sw_{15}$,sw27/sw28/$sw_{16}$}
 \draw[very thick] (\v) -- node[below=0.2of \v]{\t} (\u.center);
 
 \foreach \v / \u in {sw20/sw21}
 \draw[very thick] (\v) -- (\u.45);
 %% vertical
 \foreach \v / \u / \t in {sw4/sw8/$sw_{4}$,sw6/sw7/$sw_{3}$,sw15/sw14/$sw_{7}$,%
 sw16/sw12/$sw_{8}$,sw19/sw24/$sw_{9}$}
 \draw[very thick] (\v) -- node[auto=below]{\t~~~~~~~~~~} (\u.center);

 \foreach \v / \u in {sw29/sw30,sw31/sw32,sw25/sw26}
 \draw[very thick] (\v) -- (\u.-30);
\end{tikzpicture}

%%%%%%%%%%%%%%%%%%%%%%%%%%%%%%%%%%%%%%%%%%%%%%%%%%%%%%%%%%
%%% Local Variables:
%%% mode: japanese-latex
%%% TeX-master: paper.tex
%%% End:
}
%   \vspace{-0.1cm}
%   \caption*{\structure{$\mathbf{t=2}$}}
%  \end{figure}
% \end{frame}
% %
% \begin{frame}{配電網遷移問題の例(4/4)}
%   \vfill
%  \begin{figure}[t]
%   \centering\hspace{-0.1cm}
%   \scalebox{0.65}{\input{tikz/tikz-test-core-3}}
%   \vspace{-0.1cm}
%   \caption*{\structure{$\mathbf{t=3}$ \textbf{(ゴール状態)}}}
%  \end{figure}
% \end{frame}

% %%%%%%%%%%%%%%%%%%%%%%%%%%%%%%%%%%%%%%%%%%%%%%%%%%
% %% 卒業研究
% %%%%%%%%%%%%%%%%%%%%%%%%%%%%%%%%%%%%%%%%%%%%%%%%%%
% \begin{frame}{配電網問題のASP符号化(卒業研究)}
%    %\scalebox{0.9}{\centering\begin{figure*}[t]
  \centering
  \thicklines
  \setlength{\unitlength}{1.28pt}
  \small
  \begin{picture}(280,57)(4,-10)
    \put( -35, 20){\dashbox(70,24){\shortstack{組合せ最適化問題\\のインスタンス}}}
    \put( 45, 20){\framebox(50,24){変換器}}
    \put(105, 20){\dashbox(70,24){\shortstack{ASPファクト}}}
    \put(105,-10){\dashbox(70,24){\shortstack{ASP符号化\\(論理プログラム)}}}
    \put(185,-10){\framebox(60,54){}}
    \put(189, 25){\framebox(52,12){ASPソルバー}}
    \put(190, -5){\framebox(50,12){LNPS}}
    % \put(180, 20){\framebox(50,24){ASPシステム}}
    \put(255, 20){\dashbox(70,24){\shortstack{組合せ最適化問題\\の最適解}}}
    \put(  35, 32){\vector(1,0){10}}
    \put(  95, 32){\vector(1,0){10}}
    \put(175, 32){\vector(1,0){10}}
    \put(245, 32){\vector(1,0){10}}
    \put(175, +2){\line(1,0){4}}
    \put(179, +2){\line(0,1){30}}
    \put(205,  7){\vector(0,1){17}}
    \put(225, 24){\vector(0,-1){17}}
    \put(190, 48){提案ソルバー}
  \end{picture}  
\caption{提案ソルバー\textit{asprior}の構成}
\label{fig:arch}
\end{figure*}

%%% Local Variables: 
%%% mode: latex
%%% TeX-master: "paper"
%%% End: 
}
%    \begin{block}{根付き全域森問題の2種類のASP符号化を考案}
%      \begin{itemize}
%      \item \alert{\bf 基本符号化}
%        \begin{itemize}
%        \item 根付き全域森問題の制約を,\textbf{ASPのルール7個}で簡潔に記述
%        \end{itemize}
%      \item \alert{\bf 改良符号化}
%        \begin{itemize}
%        \item ASPシステムは,変数を含む論理プログラムを,変数を含まない
%          論理プログラムに\textbf{基礎化}したのち解集合を計算する.
%        \item 根付き連結制約をASPの個数制約で表現することにより,
%          \textbf{基礎化後のルール数を少なく抑える}よう工夫されている.
%        \item これにより,改良符号化は大規模な問題への有効性が期待できる.
%        \end{itemize}
%      \end{itemize}
%    \end{block}
%  \begin{itemize}
%   \renewcommand{\thefootnote}{\fnsymbol{footnote}}
%   \setcounter{footnote}{1}
%   \item \textit{DNET}\footnote{https://github.com/takemaru/dnet},
%         および,\textit{Graph Coloring and its Generalizations}
%         \footnote{https://mat.tepper.cmu.edu/COLOR04/}%
%         の問題をベースに独自に生成した配電網問題を用いて評価実験を行った.
%   \item 結果として,改良符号化は,基本符号化と比較して,より多くの問題を高速に解いた.
%  \end{itemize}
% \end{frame}

% %%%%%%%%%%%%%%%%%%%%%%%%%%%%%%%%%%%%%%%%%%%%%%%%%%
% %% 電気制約
% %%%%%%%%%%%%%%%%%%%%%%%%%%%%%%%%%%%%%%%%%%%%%%%%%%
% \begin{frame}{電気制約の効率的な実装}
%  \begin{itemize}
%   \item \alert{\bf 電気制約}は,送電する電流$\cdot$電圧の適正範囲を保証する制約.
%   \begin{itemize}
%    \item 供給経路の各区間で許容電流を超えない.
%    \item 電気抵抗による電圧降下が許容範囲を超えない.
%    \item etc.
%   \end{itemize}
%   \item 電流と電圧が影響し合う\structure{\bf 実数ドメイン上の制約}によって表される.
%   \item 実数ドメイン上の制約は,純粋なASPのみで扱うのは\alert{\bf 困難}.
% 		\begin{itemize}
% 		 \item \structure{\bf 方針1:} 簡易的な電流の電気制約について実装する.
% 		 \item \structure{\bf 方針2:} ASPMT技術により,背景理論ソルバーと連携して厳密に\\
%                \hspace{4zw}\!電流・電圧の制約を実装する.
% 		\end{itemize}
%  \end{itemize}
% \end{frame}
%%%%%%%%%%%%%%%%%%%%%%%%%%%%%%%%%%%%%%%%%%%%%%%%%%
%% 電流制約
%%%%%%%%%%%%%%%%%%%%%%%%%%%%%%%%%%%%%%%%%%%%%%%%%%
\begin{frame}{電流制約}
\begin{block}{電流制約}\small
 \centering
 \vskip -1em
 \begin{align*}
  J_i = \displaystyle\sum_{j\in S_i^{down}} I_j, \quad J_i \leq J^{max} 
  \quad (\forall s_{i}\in S)
 \end{align*}\vskip -1em
\begin{tabular}{ll}
 $S$ & セクションの集合 \\
 $s_i^{down}$ & セクション$i$より下流にあるセクション \\
 $I_i$ & セクション$i$の負荷電流 \\
 $J_i$ & セクション$i$に流れる電流 \\
 $J^{max}$ & 電流の許容範囲 (入力)
\end{tabular}
\end{block}\vfill
 \begin{exampleblock}{電流の計算例}
  \centering
  %%%%%%%%%%%%%%%%%%%%%%%%%%%%%%%%%%%%%%%%%%%%%%%%%%
% 電気制約の例
%%%%%%%%%%%%%%%%%%%%%%%%%%%%%%%%%%%%%%%%%%%%%%%%%%

\begin{tikzpicture}[scale=0.5]

 % 設定
 \tikzset{node/.style={rectangle, draw=black,fill=white}}

 \definecolor{edge1}{RGB}{191,0,0}
 \definecolor{node1}{RGB}{249,200,200}
 \definecolor{edge3}{RGB}{38,38,134}
 \definecolor{node3}{RGB}{200,200,249}

 % 補助線
 % \draw [help lines,blue] (0,0) grid (20,6);

 % node %
 \node[circle, ultra thick, draw=edge1, fill=node1,minimum size=1cm](1){};
 \node[node, thick, fill=node1, draw=edge1, right=2.5cm of 1] (2){};
 \node[node, thick, fill=node1, draw=edge1, right=3cm of 2] (3){};
 \node[node, thick, fill=node1, draw=edge1, right=2.5cm of 3] (4){};

 % 変電所 %
 \begin{scope}[scale=1.5]
 \draw [ultra thick, draw=edge1] (0,0) circle [radius=0.225cm] node[minimum size=0.5cm](root1){};
 \draw [ultra thick, draw=edge1] (0.225,0)--(0.35,0)--(0.35,0.35);
 \draw [ultra thick, draw=edge1] (-0.225,0)--(-0.35,0)--(-0.35,-0.35);
 \draw [ultra thick, draw=edge1] (0,0.225)--(0,0.35);
 \draw [ultra thick, draw=edge1] (0,-0.225)--(0,-0.35);
 \draw [ultra thick, draw=edge1] [domain=-0.284:-0.159] plot(\x,\x);
 \draw [ultra thick, draw=edge1] [domain=0.159:0.284] plot(\x,\x);
 \draw [ultra thick, draw=edge1] [domain=-0.284:-0.159] plot(\x,-\x);
 \draw [ultra thick, draw=edge1] [domain=0.159:0.284] plot(\x,-\x);
 \end{scope}

 \draw [line width=3.5pt, edge1] (1) -- %
 node[above, font=\Large, label=below:\color{black}{$I_i\colon\quad$30A}]
 {\textbf{$J_i\colon\quad$\!\!60A}}(5,0) -- (2);
 
 \draw [line width=2.5pt, edge1] (2) -- %
 node[above, font=\Large, label=below:\color{black}{20A}] {\textbf{30A}}(11,0) -- (3);

 \draw [line width=1.5pt, edge1] (3) -- %
 node[above, font=\large, label=below:\color{black}{10A}] {\textbf{10A}}(17,0) -- (4);

\end{tikzpicture}

%%%%%%%%%%%%%%%%%%%%%%%%%%%%%%%%%%%%%%%%%%%%%%%%%%%%%%%%%%
%%% Local Variables:
%%% mode: japanese-latex
%%% TeX-master: ``slide''
%%% End:

 \end{exampleblock}\vfill
 \begin{itemize}
  \item 電流が許容範囲を超えると電線が焼き切れる事故につながる.
 \end{itemize}\vfill
\end{frame}
%%%%%%%%%%%%%%%%%%%%%%%%%%%%%%%%%%%%%%%%%%%%%%%%%%
%% 電流制約
%%%%%%%%%%%%%%%%%%%%%%%%%%%%%%%%%%%%%%%%%%%%%%%%%%
\begin{frame}{電圧制約}
\begin{block}{電圧制約}\small
 \centering
 \vskip -1em
 \begin{align*}
  V_i = V_0 - \displaystyle\sum_{s_j\in S_i^{up}\cup \{s_i\}} Z_j 
  \left[
  \displaystyle\sum_{s_k\in S_j^{down}} I_k + \frac{I_j}{2}
  \right], \quad V_i, \geq V^{min} \quad (\forall s_{i}\in S)
 \end{align*}\vskip -5pt
\begin{tabular}{ll}
 %$S$ & セクションの集合 \\
 $s_i^{up}$ & セクション$i$より上流にあるセクション \\
 $s_i^{down}$ & セクション$i$より下流にあるセクション \\
 $I_i$ & セクション$i$の電流 \\
 $Z_i$ & セクション$i$のインピーダンス \\
 $V_i$ & セクション$i$における電圧 \\
 $V^{min}$ & 電圧の許容範囲 (入力)
\end{tabular}
\end{block}\vfill
 \begin{itemize}
  \item 電圧が許容範囲を下回ると,電化製品などが適切に動作できない恐れがある.
  \item 純粋なASPのみで扱うことが困難な制約である.
 \end{itemize}\vfill
\end{frame}
\backupend

%%% Local Variables:
%%% mode: japanese-latex
%%% TeX-master: "slide.tex"
%%% End:
