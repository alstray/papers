%%%%%%%%%%%%%%%%%%%%%%%%%%%%%%%%%%%%%%%%%%%%%%%%%%%%%%%%%% 
\chapter*{概要}
\pagenumbering{roman}
%%%%%%%%%%%%%%%%%%%%%%%%%%%%%%%%%%%%%%%%%%%%%%%%%%%%%%%%%% 
% 配電網問題 (Power Distribution Network Problem)
% は,一般に,供給経路に関するトポロジ制約と,
% 電流・電圧に関する電気制約を満たす配電網の構成
% (スイッチの開閉状態)を求める問題である.
% トポロジ制約は,
% 供給経路上のループや複数の変電所と結ばれることによる短絡,
% 変電所と結ばれないことによる停電
% が発生しないことを保証する.
% 電気制約は,供給経路の各区間で許容電流を超えないこと,
% 電気抵抗による電圧降下が許容範囲を超えないことを保証する.
% %
% 配電網遷移問題は,配電網問題とその2つの実行可能解が与え
% られたとき,一方の解から他方の解へ,遷移制約を満たしつつ実行可能解のみ
% を経由して到達できるかどうかを判定し,到達可能であればその最短経路を求
% める問題である.
% この問題は,配電網の構成制御における災害時の停電復旧などへの応用を狙い
% としている.

本論文では,配電網問題および配電網遷移問題に対して,解集合プログラミン
グ(Answer Set Programing; ASP)を用いた解法を提案する.
提案解法では,まず与えられた問題インスタンスを ASP のファクト形式に変換
した後,そのファクトと配電網(遷移)問題を解くための ASP 符号化を結合した
上で,高速 ASP システム\textit{clingo}を用いて解を求める.
トポロジ制約に対して,基本符号化,改良符号化,有向符号化の3つを考案した.
特に,有向符号化は,無向グラフの各辺$u-v$に対して,2つの弧
$u\rightarrow v$と$v\rightarrow u$を対応させることで有向グラフ
化して解く符号化である.
この有向グラフ化により,非閉路制約を簡潔に表現できる.

配電網遷移問題を解くには,複数の配電網問題を繰り返し解く必要がある.
しかし,各問題中の制約の大部分は共通であるため,ASP システムが同一の
探索空間を何度も調べることになり,求解効率が低下するという問題点がある.
この問題を解決するために,マルチショット ASP 解法を利用した符号化を提案する.
この符号化は,配電網問題の ASP 符号化の自然な拡張となっている.
マルチショット ASP 解法を利用することにより,
ASP システムが同様の探索失敗を避けるために獲得した学習節を
(部分的に)保持することで,無駄な探索を行うことなく,制約を追加した論理
プログラムを連続的に解くことができる.

提案解法の有効性を評価するために,
DNET (Power Distribution Network Evaluation Tool)
に公開されている配電網問題(全3問)と,
Graph Coloring and its Generalizations
に公開されているグラフを基に独自に生成したトポロジ制約のみの配電網問題
(計82問)を用いて実行実験を行なった.
その結果,有向符号化は,基本符号化と改良符号化と比較して,より多くの問
題をより高速に解くことが確認できた.
%
配電網遷移問題の実行実験については,DNETで公開されている実用規模の
問題({\sf fukui-tepco})に対して,実行可能解のペアをランダムに選び,
合計 1000 問の配電網遷移問題を生成しベンチマーク問題として使用した.
その結果,すべての問題の到達可能性を判定することができた.
また,マルチショットASP解法を導入することにより,
通常の解法と比較して,平均で3.8倍の高速化を実現した.



%%% Local Variables:
%%% mode: japanese-latex
%%% TeX-master: "paper"
%%% End:
