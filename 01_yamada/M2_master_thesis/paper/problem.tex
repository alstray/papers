\chapter{配電網問題と配電網遷移問題}\label{chap:problem}
%\subsection{配電網問題}

配電網問題は,
与えられた配電網$D=(S,SW)$から,以下の制約を満たす
配電網の構成(スイッチの開閉状態)が存在するかどうかを判定する問題である.
$S$は配電区間を表すセクションの集合,
$SW$は配電区間を結ぶスイッチの集合である.
本論文では,配電網の構成を閉じたスイッチの集合$X\subset SW$で表す.
すなわち,$X$がこの問題の実行可能解である.
$I_{i}$は各セクション$s_{i}\in S$の電流,
$J^{max}$はセクションにおける許容電流を表し,
ともに入力として与えられる.

% 
\begin{eqnarray}
& X ~\textrm{によって定まる配電網構成に停電と短絡が発生しない}   \label{topology}\\
% & J_i = \displaystyle\sum_{j\in S_i^{down}} I_j \qquad (\forall s_{i}\in S)\nonumber\\
% & J_i \leq J_{i}^{max} \qquad (\forall s_{i}\in S)\label{electrical}
& J_i = \displaystyle\sum_{j\in S_i^{down}} I_j, \qquad J_i \leq J^{max}
  \qquad (\forall s_{i}\in S)\label{electrical}
\end{eqnarray}
%

制約~(\ref{topology})は\textbf{トポロジ制約}と呼ばれる.
トポロジ制約を満たす配電網構成は,配電網$D$に対応するグラフから,
\textbf{根付き全域森}と呼ばれる部分グラフを求める問題に帰着できる~\cite{Minato:dnet:netuki}.
直観的には,セクションがノード,スイッチが辺,変電所が根ノードに対応する.
根付き全域森の定義は以下の通りである.

\theoremstyle{definition}
\newtheorem*{definition*}{定義}
\begin{definition*}
  グラフ$G=(V,E)$と,根と呼ばれる$V$上のノードの集合が与えられたとする.
  このとき,$G$上の根付き全域森とは,以下の制約を満たす$G$の部分グラフ
  $G'=(V,E'), E' \subseteq E$ である.
  \begin{itemize}
  \item $G'$はサイクルを持たない.(非閉路制約)
  \item $G'$の各連結成分は,ちょうど1つの根を含む.(根付き連結制約)
  \end{itemize}
\end{definition*}

制約~(\ref{electrical})は\textbf{電流制約}と呼ばれる.
$S_{i}^{down}$は,各セクション$s_{i}$に対して,自身を含む
下流のセクションの集合を表す.
$J_{i}$はセクション$s_{i}$およびその下流の電流の総和を表し,
許容電流$J^{max}$以下でなければならない.
%
上で述べた配電網問題の定義は,論文~\cite{Minato:dnet:ZDD}に基づいている.
論文~\cite{Minato:dnet:ZDD}では
配電損失を最小にする最適化問題として定義されている.
本論文では,後で述べる配電網遷移問題への拡張を主眼とするため,
配電網問題を判定問題として定義している.

%%%%%%%%%%%%%%%%%%%%%%%%%%%%%%%
\begin{figure}[tb]
\rotatebox{90}{
\begin{minipage}{\textheight}
 \centering
 \scalebox{1.1}{%%%%%%%%%%%%%%%%%%%%%%%%%%%%%%%%%%%%%%%%%%%%%%%%%%
% 配電網 例 (第1章で使う)
%%%%%%%%%%%%%%%%%%%%%%%%%%%%%%%%%%%%%%%%%%%%%%%%%%

\begin{tikzpicture}

 % setting
 \tikzset{customer/.style={rectangle,thick,draw=black,minimum size=0.5cm}}
 \tikzset{on_switch/.style={rectangle,fill=black}}
 \tikzset{off_switch/.style={rectangle,draw=black,fill=white}}
 
 \tikzset{node distance =1cm};

 % substation (x, y, label)
 \newcommand{\substation}[3]{
 \draw [very thick] (#1,#2) circle [radius=0.225cm] node[draw=white,minimum size=1cm](#3){};
 \draw [very thick] (#1+0.225,#2)--(#1+0.35,#2)--(#1+0.35,#2+0.3);
 \draw [very thick] (#1-0.225,#2)--(#1-0.35,#2)--(#1-0.35,#2-0.3);
 \draw [very thick] (#1,#2+0.225)--(#1,#2+0.35);
 \draw [very thick] (#1,#2-0.225)--(#1,#2-0.35);
 \draw [very thick] [domain=-0.284:-0.159] plot(\x+#1,\x+#2);
 \draw [very thick] [domain=0.159:0.284] plot(\x+#1,\x+#2);
 \draw [very thick] [domain=-0.284:-0.159] plot(\x+#1,-\x+#2);
 \draw [very thick] [domain=0.159:0.284] plot(\x+#1,-\x+#2);
 }

 %switch node (position, label, cap)
 %% right switch
 \newcommand{\swnodeR}[4]{
 \coordinate[#1] (#2);
 \node[#1,customer] (#2){#4};
 \node[circle, draw=black, text width=0.2cm, 
 right=0cm of #2, scale=0.3, thick] {};
 \node[right=0cm of #2,scale=0.3, minimum size=0.8cm] (#3){};
 }
 %% left switch
 \newcommand{\swnodeL}[4]{
 %\coordinate[#1] (#2);
 \node[#1,customer] (#2){#4};
 \node[circle, draw=black, fill=white, text width=0.2cm, 
 left=0cm of #2, scale=0.3, thick] (#3){};
 }
 % above switch
 \newcommand{\swnodeA}[4]{
 \coordinate[#1] (#2);
 \node[#1,customer] (#2){#4};
 \node[circle, draw=black, text width=0.2cm, 
 above=0cm of #2, scale=0.3, thick] (#3){};
 }
 % below switch
 \newcommand{\swnodeB}[4]{
 \coordinate[#1] (#2);
 \node[#1,customer] (#2){#4};
 \node[circle, draw=black, text width=0.2cm, 
 below=0cm of #2, scale=0.3, thick] {};
 \node[below=0cm of #2,scale=0.3,minimum size=0.8cm] (#3){};
 }
 
 \substation{0}{0}{sub};
 
 % root1
 \node[customer,fill=black!40,below =4.5cm of sub] (root1) { };
 \swnodeL{left =of root1}{node1}{sw1}{ };
 
 \swnodeR{left=of node1}{node2}{sw2}{ };
 \node[customer,left=of node2] (junc1){ };
 \swnodeL{left =of junc1}{node3}{sw3}{ };
 \swnodeA{above= of junc1}{node4}{sw4}{ }

 \swnodeR{left=of node3}{node5}{sw5}{ };
 \swnodeA{above =of node5}{node6}{sw6}{ };

 \swnodeB{above =of node6}{node7}{sw7}{ };
 \swnodeA{above =of node7}{node29}{sw29}{ };

 \swnodeB{above =of node29}{node30}{sw30}{ };
 
 \swnodeB{above =of node4}{node8}{sw8}{ };
 \swnodeA{above =of node8}{node31}{sw31}{ };
 
 \swnodeB{above =of node31}{node32}{sw32}{ };

 \swnodeR{right =of root1}{node17}{sw17}{ };

 \swnodeL{right =of node17}{node18}{sw18}{ };

 % root2
 \node[customer,fill=black!10,above=4.5cm of sub] (root2) { };
 \swnodeL{left =of root2}{node9}{sw9}{ };

 \swnodeR{left=of node9}{node10}{sw10}{ };
 \node[customer,left=of node10] (junc2){ };
 \swnodeL{left =of junc2}{node11}{sw11}{ };
 \swnodeB{below =of junc2}{node12}{sw12}{ };
 
 \swnodeR{left =of node11}{node13}{sw13}{ };
 \swnodeB{below =of node13}{node14}{sw14}{ };
 
 \swnodeA{below =of node14}{node15}{sw15}{ };

 \swnodeA{below =of node12}{node16}{sw16}{ };

 \swnodeR{right =of root2}{node22}{sw22}{ };

 \swnodeL{right =of node22}{node23}{sw23}{ };
 
 % root3
 \node[customer,pattern=north east lines,right=5.2cm of sub] (root3) { };
 \swnodeB{below =1.4of root3}{node24}{sw24}{ };
 \swnodeA{above =1.4of root3}{node25}{sw25}{ };

 \swnodeA{below =of node24}{node19}{sw19}{ };
 \swnodeL{below =1.3of node19}{node20}{sw20}{ };
 
 \swnodeR{left =of node20}{node21}{sw21}{ };

 \swnodeB{above =of node25}{node26}{sw26}{ };
 \swnodeL{above =1.3of node26}{node27}{sw27}{ };

 \swnodeR{left =of node27}{node28}{sw28}{ };
 
 % sections
 \foreach \v / \u / \t in {root1/sub/$s_a$,root1/node1/$s_1$,node2/junc1/$s_2$, %
 junc1/node3/$s_3$,junc1/node4/$s_4$,node5/node6/$s_5$,node7/node29/$s_6$,node30/node15/$s_7$, %
 sub/root2/$s_b$,root2/node9/$s_8$,node10/junc2/$s_9$,junc2/node11/$s_{10}$,node12/junc2/$s_{11}$, %
 node14/node13/$s_{12}$,node8/node31/$s_{13}$,node32/node16/$s_{14}$,node17/root1/$s_{15}$, %
 node22/root2/$s_{16}$,root3/sub/$s_c$,node24/root3/$s_{17}$,root3/node25/$s_{18}$, %
 node20/node19/$s_{19}$,node26/node27/$s_{20}$,node21/node18/$s_{21}$,node28/node23/$s_{22}$} %
 \draw[thick] (\v) --  node[auto=right]{\t} (\u);

 % switches
 %% horizontal
 \foreach \v / \u / \t in {sw1/sw2/$sw_{1}$,sw3/sw5/$sw_{2}$,sw9/sw10/$sw_{11}$,sw11/sw13/$sw_{12}$,sw18/sw17/$sw_{13}$,
 sw23/sw22/$sw_{15}$,sw20/sw21/$sw_{14}$,sw27/sw28/$sw_{16}$}
 \draw[very thick] (\v) -- node[below=0.2of \v]{\t} (\u.45);
 %% vertical
 \foreach \v / \u / \t in {sw4/sw8/$sw_{4}$,sw6/sw7/$sw_{3}$,sw29/sw30/$sw_{5}$,sw31/sw32/$sw_{6}$,sw15/sw14/$sw_{7}$,%
 sw16/sw12/$sw_{8}$,sw19/sw24/$sw_{9}$,sw25/sw26/$sw_{10}$}
 \draw[very thick] (\v) -- node[auto=below]{\t~~~~~~~~~~} (\u.-30);

\end{tikzpicture}

%%%%%%%%%%%%%%%%%%%%%%%%%%%%%%%%%%%%%%%%%%%%%%%%%%%%%%%%%%
%%% Local Variables:
%%% mode: japanese-latex
%%% TeX-master: paper.tex
%%% End:
}
  \caption{配電網問題の例}
  \label{fig:test-input}
\end{minipage}
}
\end{figure}
%%%%%%%%%%%%%%%%%%%%%%%%%%%%%%%
% %%%%%%%%%%%%%%%%%%%%%%%%%%%%%%%
% \begin{table}[tbp]
%  \centering
%  \caption{負荷電流の一覧~(A)}
%  \label{table:current}
%  \begin{tabular}{ccc|ccc|ccc|ccc|ccc}
 \noalign{\hrule height 1pt}
 $I_{a}$ &=& 16 & $I_{b}$ & = & 41 & $I_{c}$ & = & 16 & $I_{1}$ & = & 40 & $I_{2}$ & = & 5 \\
 $I_{3}$ & = & 34 & $I_{4}$ & = & 0 & $I_{5}$ & = & 11 & $I_{6}$ & = & 34 & $I_{7}$ & = & 24 \\
 $I_{8}$ & = & 31 & $I_{9}$ & = & 45 & $I_{10}$ & = & 24 & $I_{11}$ & = & 0 & $I_{12}$ & = & 45 \\
 $I_{13}$ & = & 21 & $I_{14}$ & = & 20 & $I_{15}$ & = & 0 & $I_{16}$ & = & 0 & $I_{17}$ & = & 0 \\
 $I_{18}$ & = & 0 & $I_{19}$ & = & 35 & $I_{20}$ & = & 20 & $I_{21}$ & = & 35 & $I_{22}$ & = & 20 \\
 \noalign{\hrule height 1pt}
\end{tabular}
% \end{table}
% %%%%%%%%%%%%%%%%%%%%%%%%%%%%%%%
%%%%%%%%%%%%%%%%%%%%%%%%%%%%%%%%%%%%% 
\begin{figure}[tb]
\rotatebox{90}{
\begin{minipage}{\textheight}
 \centering
 \scalebox{1.1}{%%%%%%%%%%%%%%%%%%%%%%%%%%%%%%%%%%%%%%%%%%%%%%%%%%
% 配電網 例 (第1章で使う)
%%%%%%%%%%%%%%%%%%%%%%%%%%%%%%%%%%%%%%%%%%%%%%%%%%

\begin{tikzpicture}

 % setting
 \tikzset{customer/.style={rectangle,thick,draw=black,minimum size=0.5cm}}
 \tikzset{on_switch/.style={rectangle,fill=black}}
 \tikzset{off_switch/.style={rectangle,draw=black,fill=white}}
 
 \tikzset{node distance =1cm};

 % substation (x, y, label)
 \newcommand{\substation}[3]{
 \draw [very thick] (#1,#2) circle [radius=0.225cm] node[draw=none,minimum size=1cm](#3){};
 \draw [very thick] (#1+0.225,#2)--(#1+0.35,#2)--(#1+0.35,#2+0.3);
 \draw [very thick] (#1-0.225,#2)--(#1-0.35,#2)--(#1-0.35,#2-0.3);
 \draw [very thick] (#1,#2+0.225)--(#1,#2+0.35);
 \draw [very thick] (#1,#2-0.225)--(#1,#2-0.35);
 \draw [very thick] [domain=-0.284:-0.159] plot(\x+#1,\x+#2);
 \draw [very thick] [domain=0.159:0.284] plot(\x+#1,\x+#2);
 \draw [very thick] [domain=-0.284:-0.159] plot(\x+#1,-\x+#2);
 \draw [very thick] [domain=0.159:0.284] plot(\x+#1,-\x+#2);
 }

 %switch node (position, label, cap)
 %% right switch
 \newcommand{\swnodeR}[4]{
 \coordinate[#1] (#2);
 \node[#1,customer] (#2){#4};
 \node[circle, draw=black, text width=0.2cm, 
 right=0cm of #2, scale=0.3, thick] {};
 \node[right=0cm of #2,scale=0.3, minimum size=0.8cm] (#3){};
 }
 %% left switch
 \newcommand{\swnodeL}[4]{
 %\coordinate[#1] (#2);
 \node[#1,customer] (#2){#4};
 \node[circle, draw=black, fill=white, text width=0.2cm, 
 left=0cm of #2, scale=0.3, thick] (#3){};
 }
 % above switch
 \newcommand{\swnodeA}[4]{
 \coordinate[#1] (#2);
 \node[#1,customer] (#2){#4};
 \node[circle, draw=black, text width=0.2cm, 
 above=0cm of #2, scale=0.3, thick] (#3){};
 }
 % below switch
 \newcommand{\swnodeB}[4]{
 \coordinate[#1] (#2);
 \node[#1,customer] (#2){#4};
 \node[circle, draw=black, text width=0.2cm, 
 below=0cm of #2, scale=0.3, thick] {};
 \node[below=0cm of #2,scale=0.3,minimum size=0.8cm] (#3){};
 }
 
 \substation{0}{0}{sub};
 
 % root1
 \node[customer,fill=purple!60,below =4.5cm of sub] (root1) { };
 \swnodeL{left =of root1,fill=purple!60}{node1}{sw1}{ };
 
 \swnodeR{left=of node1,fill=purple!60}{node2}{sw2}{ };
 \node[customer,left=of node2,fill=purple!60] (junc1){ };
 \swnodeL{left =of junc1,fill=purple!60}{node3}{sw3}{ };
 \swnodeA{above= of junc1,fill=purple!60}{node4}{sw4}{ }

 \swnodeR{left=of node3,fill=purple!60}{node5}{sw5}{ };
 \swnodeA{above =of node5,fill=purple!60}{node6}{sw6}{ };

 \swnodeB{above =of node6,fill=cyan!80}{node7}{sw7}{ };
 \swnodeA{above =of node7,fill=cyan!80}{node29}{sw29}{ };

 \swnodeB{above =of node29,fill=cyan!80}{node30}{sw30}{ };
 
 \swnodeB{above =of node4,fill=purple!60}{node8}{sw8}{ };
 \swnodeA{above =of node8,fill=purple!60}{node31}{sw31}{ };
 
 \swnodeB{above =of node31,fill=cyan!80}{node32}{sw32}{ };

 \swnodeR{right =of root1,fill=purple!60}{node17}{sw17}{ };

 \swnodeL{right =of node17,fill=purple!60}{node18}{sw18}{ };

 % root2
 \node[customer,fill=cyan!80,above=4.5cm of sub, fill=cyan!80] (root2) { };
 \swnodeL{left =of root2,fill=cyan!80}{node9}{sw9}{ };

 \swnodeR{left=of node9,fill=cyan!80}{node10}{sw10}{ };
 \node[customer,left=of node10,fill=cyan!80] (junc2){ };
 \swnodeL{left =of junc2,fill=cyan!80}{node11}{sw11}{ };
 \swnodeB{below =of junc2,fill=cyan!80}{node12}{sw12}{ };
 
 \swnodeR{left =of node11,fill=cyan!80}{node13}{sw13}{ };
 \swnodeB{below =of node13,fill=cyan!80}{node14}{sw14}{ };
 
 \swnodeA{below =of node14,fill=cyan!80}{node15}{sw15}{ };

 \swnodeA{below =of node12,fill=cyan!80}{node16}{sw16}{ };

 \swnodeR{right =of root2,fill=cyan!80}{node22}{sw22}{ };

 \swnodeL{right =of node22,fill=cyan!80}{node23}{sw23}{ };
 
 % root3
 \node[customer,fill=cyan!80,right=5.2cm of sub,fill=yellow!80] (root3) { };
 \swnodeB{below =1.4of root3,fill=yellow!80}{node24}{sw24}{ };
 \swnodeA{above =1.4of root3,fill=yellow!80}{node25}{sw25}{ };

 \swnodeA{below =of node24,fill=yellow!80}{node19}{sw19}{ };
 \swnodeL{below =1.3of node19,fill=yellow!80}{node20}{sw20}{ };
 
 \swnodeR{left =of node20,fill=purple!60}{node21}{sw21}{ };

 \swnodeB{above =of node25,fill=yellow!80}{node26}{sw26}{ };
 \swnodeL{above =1.3of node26,fill=yellow!80}{node27}{sw27}{ };

 \swnodeR{left =of node27,fill=cyan!80}{node28}{sw28}{ };
 
 % sections
 \foreach \v / \u / \t in {root1/sub/$s_a$,root1/node1/$s_1$,node2/junc1/$s_2$, %
 junc1/node3/$s_3$,junc1/node4/$s_4$,node5/node6/$s_5$,node7/node29/$s_6$,node30/node15/$s_7$, %
 sub/root2/$s_b$,root2/node9/$s_8$,node10/junc2/$s_9$,junc2/node11/$s_{10}$,node12/junc2/$s_{11}$, %
 node14/node13/$s_{12}$,node8/node31/$s_{13}$,node32/node16/$s_{14}$,node17/root1/$s_{15}$, %
 node22/root2/$s_{16}$,root3/sub/$s_c$,node24/root3/$s_{17}$,root3/node25/$s_{18}$, %
 node20/node19/$s_{19}$,node26/node27/$s_{20}$,node21/node18/$s_{21}$,node28/node23/$s_{22}$} %
 \draw[thick] (\v) --  node[auto=right]{\t} (\u);

 % switches
 %% horizontal
 \foreach \v / \u / \t in {sw1/sw2/$sw_{1}$,sw3/sw5/$sw_{2}$,sw9/sw10/$sw_{11}$,
 sw11/sw13/$sw_{12}$,sw18/sw17/$sw_{13}$,sw23/sw22/$sw_{15}$}
 \draw[very thick] (\v) -- node[below=0.2of \v]{\t} (\u.center);
 
 \foreach \v / \u in {sw20/sw21,sw27/sw28}
 \draw[very thick] (\v) -- (\u.45);
 %% vertical
 \foreach \v / \u / \t in {sw4/sw8/$sw_{4}$,sw29/sw30/$sw_{5}$,sw15/sw14/$sw_{7}$,%
 sw16/sw12/$sw_{8}$,sw19/sw24/$sw_{9}$,sw25/sw26/$sw_{10}$}
 \draw[very thick] (\v) -- node[auto=below]{\t~~~~~~~~~~} (\u.center);

 \foreach \v / \u in {sw6/sw7,sw31/sw32}
 \draw[very thick] (\v) -- (\u.-30);
\end{tikzpicture}

%%%%%%%%%%%%%%%%%%%%%%%%%%%%%%%%%%%%%%%%%%%%%%%%%%%%%%%%%%
%%% Local Variables:
%%% mode: japanese-latex
%%% TeX-master: paper.tex
%%% End:
}
 \caption{配電網問題(図\ref{fig:test-input})の解の一例}
 \label{fig:test-output}
\end{minipage}
}
\end{figure}
%%%%%%%%%%%%%%%%%%%%%%%%%%%%%%%%%%%%% 

DNET
% \footnote{\url{https://github.com/takemaru/dnet}}
に公開されている配電網問題の例を図~\ref{fig:test-input}に示す.
この例は,
25箇所のセクション$S=\{s_{a},s_{b},s_{c},s_{1},\ldots, s_{22}\}$,
16個のスイッチ$SW=\{sw_{1},\ldots, sw_{16}\}$,
3箇所の変電所$\{s_{a}, s_{b}, s_{c}\}$
から構成されている.
ここでは,変電所に直接つながるセクションをそれぞれ変電所とする.
%
図\ref{fig:test-output}に,この配電網問題の解の一例を示す.
閉じたスイッチは,
\[X=\{sw_{1},sw_{2},sw_{4},sw_{5},sw_{7},sw_{8},sw_{9},%
sw_{10},sw_{11},sw_{12},sw_{13},sw_{15}\}\]
となる.
変電所$s_{a}$から電力の供給を受けるセクションの端点を赤色で,
変電所$s_{b}$からを青色で,
変電所$s_{c}$からを黄色で表している.
トポロジ制約を満たし,停電と短絡が発生しないことがわかる.
また,電流制約について,例えば,セクション$s_2$の下流にあるセクションの
集合は,$S_2^{down}=\{s_{2},s_{3},s_{4},s_{5},s_{13}\}$となる.したがって,
$J_{2}=I_{2}+I_{3}+I_{4}+I_{5}+I_{13}$のように計算される.

% %%%%%%%%%%%%%%%%%%%%%%%%%%%%%%%%%%%%% 
% \newcommand{\lw}[1]{\smash{\lower-15.ex\hbox{#1}}}
% \begin{figure*}[tb]
%   %\renewcommand{\arraystretch}{0.9}
%   \tabcolsep = 3mm
%   \centering
%   \begin{tabular}{ccc}
%     $t=0$ (スタート状態) & & $t=1$\\
%     \scalebox{0.45}{\begin{tikzpicture}

 % setting
 \tikzset{customer/.style={rectangle,thick,draw=black,minimum size=0.5cm}}
 \tikzset{on_switch/.style={rectangle,fill=black}}
 \tikzset{off_switch/.style={rectangle,draw=black,fill=white}}
 
 \tikzset{node distance =1cm};

 % substation (x, y, label)
 \newcommand{\substation}[3]{
 \draw [very thick] (#1,#2) circle [radius=0.225cm] node[draw=white,minimum size=1cm](#3){};
 \draw [very thick] (#1+0.225,#2)--(#1+0.35,#2)--(#1+0.35,#2+0.3);
 \draw [very thick] (#1-0.225,#2)--(#1-0.35,#2)--(#1-0.35,#2-0.3);
 \draw [very thick] (#1,#2+0.225)--(#1,#2+0.35);
 \draw [very thick] (#1,#2-0.225)--(#1,#2-0.35);
 \draw [very thick] [domain=-0.284:-0.159] plot(\x+#1,\x+#2);
 \draw [very thick] [domain=0.159:0.284] plot(\x+#1,\x+#2);
 \draw [very thick] [domain=-0.284:-0.159] plot(\x+#1,-\x+#2);
 \draw [very thick] [domain=0.159:0.284] plot(\x+#1,-\x+#2);
 }

 %switch node (position, label, cap)
 %% right switch
 \newcommand{\swnodeR}[4]{
 \coordinate[#1] (#2);
 \node[#1,customer] (#2){#4};
 \node[circle, draw=black, text width=0.2cm, 
 right=0cm of #2, scale=0.3, thick] {};
 \node[right=0cm of #2,scale=0.3, minimum size=0.8cm] (#3){};
 }
 %% left switch
 \newcommand{\swnodeL}[4]{
 %\coordinate[#1] (#2);
 \node[#1,customer] (#2){#4};
 \node[circle, draw=black, fill=white, text width=0.2cm, 
 left=0cm of #2, scale=0.3, thick] (#3){};
 }
 % above switch
 \newcommand{\swnodeA}[4]{
 \coordinate[#1] (#2);
 \node[#1,customer] (#2){#4};
 \node[circle, draw=black, text width=0.2cm, 
 above=0cm of #2, scale=0.3, thick] (#3){};
 }
 % below switch
 \newcommand{\swnodeB}[4]{
 \coordinate[#1] (#2);
 \node[#1,customer] (#2){#4};
 \node[circle, draw=black, text width=0.2cm, 
 below=0cm of #2, scale=0.3, thick] {};
 \node[below=0cm of #2,scale=0.3,minimum size=0.8cm] (#3){};
 }
 
 \substation{0}{0}{sub};
 
 % root1
 \node[customer,fill=purple!60,below =4.5cm of sub] (root1) { };
 \swnodeL{left =of root1,fill=purple!60}{node1}{sw1}{ };
 
 \swnodeR{left=of node1,fill=purple!60}{node2}{sw2}{ };
 \node[customer,left=of node2,fill=purple!60] (junc1){ };
 \swnodeL{left =of junc1,fill=purple!60}{node3}{sw3}{ };
 \swnodeA{above= of junc1,fill=purple!60}{node4}{sw4}{ }

 \swnodeR{left=of node3,fill=purple!60}{node5}{sw5}{ };
 \swnodeA{above =of node5,fill=purple!60}{node6}{sw6}{ };

 \swnodeB{above =of node6,fill=cyan!80}{node7}{sw7}{ };
 \swnodeA{above =of node7,fill=cyan!80}{node29}{sw29}{ };

 \swnodeB{above =of node29,fill=cyan!80}{node30}{sw30}{ };
 
 \swnodeB{above =of node4,fill=purple!60}{node8}{sw8}{ };
 \swnodeA{above =of node8,fill=purple!60}{node31}{sw31}{ };
 
 \swnodeB{above =of node31,fill=cyan!80}{node32}{sw32}{ };

 \swnodeR{right =of root1,fill=purple!60}{node17}{sw17}{ };

 \swnodeL{right =of node17,fill=purple!60}{node18}{sw18}{ };

 % root2
 \node[customer,fill=black!20,above=4.5cm of sub, fill=cyan!80] (root2) { };
 \swnodeL{left =of root2,fill=cyan!80}{node9}{sw9}{ };

 \swnodeR{left=of node9,fill=cyan!80}{node10}{sw10}{ };
 \node[customer,left=of node10,fill=cyan!80] (junc2){ };
 \swnodeL{left =of junc2,fill=cyan!80}{node11}{sw11}{ };
 \swnodeB{below =of junc2,fill=cyan!80}{node12}{sw12}{ };
 
 \swnodeR{left =of node11,fill=cyan!80}{node13}{sw13}{ };
 \swnodeB{below =of node13,fill=cyan!80}{node14}{sw14}{ };
 
 \swnodeA{below =of node14,fill=cyan!80}{node15}{sw15}{ };

 \swnodeA{below =of node12,fill=cyan!80}{node16}{sw16}{ };

 \swnodeR{right =of root2,fill=cyan!80}{node22}{sw22}{ };

 \swnodeL{right =of node22,fill=cyan!80}{node23}{sw23}{ };
 
 % root3
 \node[customer,fill=black!20,right=5.2cm of sub,fill=yellow!80] (root3) { };
 \swnodeB{below =1.4of root3,fill=yellow!80}{node24}{sw24}{ };
 \swnodeA{above =1.4of root3,fill=yellow!80}{node25}{sw25}{ };

 \swnodeA{below =of node24,fill=yellow!80}{node19}{sw19}{ };
 \swnodeL{below =1.3of node19,fill=yellow!80}{node20}{sw20}{ };
 
 \swnodeR{left =of node20,fill=purple!60}{node21}{sw21}{ };

 \swnodeB{above =of node25,fill=yellow!80}{node26}{sw26}{ };
 \swnodeL{above =1.3of node26,fill=yellow!80}{node27}{sw27}{ };

 \swnodeR{left =of node27,fill=cyan!80}{node28}{sw28}{ };
 
 % sections
 \foreach \v / \u / \t in {root1/sub/$s_a$,root1/node1/$s_1$,node2/junc1/$s_2$, %
 junc1/node3/$s_3$,junc1/node4/$s_4$,node5/node6/$s_5$,node7/node29/$s_6$,node30/node15/$s_7$, %
 sub/root2/$s_b$,root2/node9/$s_8$,node10/junc2/$s_9$,junc2/node11/$s_{10}$,node12/junc2/$s_{11}$, %
 node14/node13/$s_{12}$,node8/node31/$s_{13}$,node32/node16/$s_{14}$,node17/root1/$s_{15}$, %
 node22/root2/$s_{16}$,root3/sub/$s_c$,node24/root3/$s_{17}$,root3/node25/$s_{18}$, %
 node20/node19/$s_{19}$,node26/node27/$s_{20}$,node21/node18/$s_{21}$,node28/node23/$s_{22}$} %
 \draw[thick] (\v) --  node[auto=right]{\t} (\u);

 % switches
 %% horizontal
 \foreach \v / \u / \t in {sw1/sw2/$sw_{1}$,sw3/sw5/$sw_{2}$,sw9/sw10/$sw_{11}$,
 sw11/sw13/$sw_{12}$,sw18/sw17/$sw_{13}$,sw23/sw22/$sw_{15}$}
 \draw[very thick] (\v) -- node[below=0.2of \v]{\t} (\u.center);
 
 \foreach \v / \u in {sw20/sw21,sw27/sw28}
 \draw[very thick] (\v) -- (\u.45);
 %% vertical
 \foreach \v / \u / \t in {sw4/sw8/$sw_{4}$,sw29/sw30/$sw_{5}$,sw15/sw14/$sw_{7}$,%
 sw16/sw12/$sw_{8}$,sw19/sw24/$sw_{9}$,sw25/sw26/$sw_{10}$}
 \draw[very thick] (\v) -- node[auto=below]{\t~~~~~~~~~~} (\u.center);

 \foreach \v / \u in {sw6/sw7,sw31/sw32}
 \draw[very thick] (\v) -- (\u.-30);
\end{tikzpicture}

%%%%%%%%%%%%%%%%%%%%%%%%%%%%%%%%%%%%%%%%%%%%%%%%%%%%%%%%%%
%%% Local Variables:
%%% mode: japanese-latex
%%% TeX-master: paper.tex
%%% End:
}
%     & \lw{$\Rightarrow$} & 
%     \scalebox{0.45}{\begin{tikzpicture}

 % setting
 \tikzset{customer/.style={rectangle,thick,draw=black,minimum size=0.5cm}}
 \tikzset{on_switch/.style={rectangle,fill=black}}
 \tikzset{off_switch/.style={rectangle,draw=black,fill=white}}
 
 \tikzset{node distance =1cm};

 % substation (x, y, label)
 \newcommand{\substation}[3]{
 \draw [very thick] (#1,#2) circle [radius=0.225cm] node[draw=white,minimum size=1cm](#3){};
 \draw [very thick] (#1+0.225,#2)--(#1+0.35,#2)--(#1+0.35,#2+0.3);
 \draw [very thick] (#1-0.225,#2)--(#1-0.35,#2)--(#1-0.35,#2-0.3);
 \draw [very thick] (#1,#2+0.225)--(#1,#2+0.35);
 \draw [very thick] (#1,#2-0.225)--(#1,#2-0.35);
 \draw [very thick] [domain=-0.284:-0.159] plot(\x+#1,\x+#2);
 \draw [very thick] [domain=0.159:0.284] plot(\x+#1,\x+#2);
 \draw [very thick] [domain=-0.284:-0.159] plot(\x+#1,-\x+#2);
 \draw [very thick] [domain=0.159:0.284] plot(\x+#1,-\x+#2);
 }

 %switch node (position, label, cap)
 %% right switch
 \newcommand{\swnodeR}[4]{
 \coordinate[#1] (#2);
 \node[#1,customer] (#2){#4};
 \node[circle, draw=black, text width=0.2cm, 
 right=0cm of #2, scale=0.3, thick] {};
 \node[right=0cm of #2,scale=0.3, minimum size=0.8cm] (#3){};
 }
 %% left switch
 \newcommand{\swnodeL}[4]{
 %\coordinate[#1] (#2);
 \node[#1,customer] (#2){#4};
 \node[circle, draw=black, fill=white, text width=0.2cm, 
 left=0cm of #2, scale=0.3, thick] (#3){};
 }
 % above switch
 \newcommand{\swnodeA}[4]{
 \coordinate[#1] (#2);
 \node[#1,customer] (#2){#4};
 \node[circle, draw=black, text width=0.2cm, 
 above=0cm of #2, scale=0.3, thick] (#3){};
 }
 % below switch
 \newcommand{\swnodeB}[4]{
 \coordinate[#1] (#2);
 \node[#1,customer] (#2){#4};
 \node[circle, draw=black, text width=0.2cm, 
 below=0cm of #2, scale=0.3, thick] {};
 \node[below=0cm of #2,scale=0.3,minimum size=0.8cm] (#3){};
 }
 
 \substation{0}{0}{sub};
 
 % root1
 \node[customer,fill=purple!60,below =4.5cm of sub] (root1) { };
 \swnodeL{left =of root1,fill=purple!60}{node1}{sw1}{ };
 
 \swnodeR{left=of node1,fill=purple!60}{node2}{sw2}{ };
 \node[customer,left=of node2,fill=purple!60] (junc1){ };
 \swnodeL{left =of junc1,fill=purple!60}{node3}{sw3}{ };
 \swnodeA{above= of junc1,fill=purple!60}{node4}{sw4}{ }

 \swnodeR{left=of node3,fill=purple!60}{node5}{sw5}{ };
 \swnodeA{above =of node5,fill=purple!60}{node6}{sw6}{ };

 \swnodeB{above =of node6,fill=purple!60}{node7}{sw7}{ };
 \swnodeA{above =of node7,fill=purple!60}{node29}{sw29}{ };

 \swnodeB{above =of node29,fill=cyan!80}{node30}{sw30}{ };
 
 \swnodeB{above =of node4,fill=purple!60}{node8}{sw8}{ };
 \swnodeA{above =of node8,fill=purple!60}{node31}{sw31}{ };
 
 \swnodeB{above =of node31,fill=cyan!80}{node32}{sw32}{ };

 \swnodeR{right =of root1,fill=purple!60}{node17}{sw17}{ };

 \swnodeL{right =of node17,fill=purple!60}{node18}{sw18}{ };

 % root2
 \node[customer,fill=black!20,above=4.5cm of sub, fill=cyan!80] (root2) { };
 \swnodeL{left =of root2,fill=cyan!80}{node9}{sw9}{ };

 \swnodeR{left=of node9,fill=cyan!80}{node10}{sw10}{ };
 \node[customer,left=of node10,fill=cyan!80] (junc2){ };
 \swnodeL{left =of junc2,fill=cyan!80}{node11}{sw11}{ };
 \swnodeB{below =of junc2,fill=cyan!80}{node12}{sw12}{ };
 
 \swnodeR{left =of node11,fill=cyan!80}{node13}{sw13}{ };
 \swnodeB{below =of node13,fill=cyan!80}{node14}{sw14}{ };
 
 \swnodeA{below =of node14,fill=cyan!80}{node15}{sw15}{ };

 \swnodeA{below =of node12,fill=cyan!80}{node16}{sw16}{ };

 \swnodeR{right =of root2,fill=cyan!80}{node22}{sw22}{ };

 \swnodeL{right =of node22,fill=cyan!80}{node23}{sw23}{ };
 
 % root3
 \node[customer,fill=black!20,right=5.2cm of sub,fill=yellow!80] (root3) { };
 \swnodeB{below =1.4of root3,fill=yellow!80}{node24}{sw24}{ };
 \swnodeA{above =1.4of root3,fill=yellow!80}{node25}{sw25}{ };

 \swnodeA{below =of node24,fill=yellow!80}{node19}{sw19}{ };
 \swnodeL{below =1.3of node19,fill=yellow!80}{node20}{sw20}{ };
 
 \swnodeR{left =of node20,fill=purple!60}{node21}{sw21}{ };

 \swnodeB{above =of node25,fill=yellow!80}{node26}{sw26}{ };
 \swnodeL{above =1.3of node26,fill=yellow!80}{node27}{sw27}{ };

 \swnodeR{left =of node27,fill=cyan!80}{node28}{sw28}{ };
 
 % sections
 \foreach \v / \u / \t in {root1/sub/$s_a$,root1/node1/$s_1$,node2/junc1/$s_2$, %
 junc1/node3/$s_3$,junc1/node4/$s_4$,node5/node6/$s_5$,node7/node29/$s_6$,node30/node15/$s_7$, %
 sub/root2/$s_b$,root2/node9/$s_8$,node10/junc2/$s_9$,junc2/node11/$s_{10}$,node12/junc2/$s_{11}$, %
 node14/node13/$s_{12}$,node8/node31/$s_{13}$,node32/node16/$s_{14}$,node17/root1/$s_{15}$, %
 node22/root2/$s_{16}$,root3/sub/$s_c$,node24/root3/$s_{17}$,root3/node25/$s_{18}$, %
 node20/node19/$s_{19}$,node26/node27/$s_{20}$,node21/node18/$s_{21}$,node28/node23/$s_{22}$} %
 \draw[thick] (\v) --  node[auto=right]{\t} (\u);

 % switches
 %% horizontal
 \foreach \v / \u / \t in {sw1/sw2/$sw_{1}$,sw3/sw5/$sw_{2}$,sw9/sw10/$sw_{11}$,
 sw11/sw13/$sw_{12}$,sw18/sw17/$sw_{13}$,sw23/sw22/$sw_{15}$}
 \draw[very thick] (\v) -- node[below=0.2of \v]{\t} (\u.center);
 
 \foreach \v / \u in {sw20/sw21,sw27/sw28}
 \draw[very thick] (\v) -- (\u.45);
 %% vertical
 \foreach \v / \u / \t in {sw4/sw8/$sw_{4}$,sw6/sw7/$sw_{3}$,sw15/sw14/$sw_{7}$,%
 sw16/sw12/$sw_{8}$,sw19/sw24/$sw_{9}$,sw25/sw26/$sw_{10}$}
 \draw[very thick] (\v) -- node[auto=below]{\t~~~~~~~~~~} (\u.center);

 \foreach \v / \u in {sw29/sw30,sw31/sw32}
 \draw[very thick] (\v) -- (\u.-30);
\end{tikzpicture}

%%%%%%%%%%%%%%%%%%%%%%%%%%%%%%%%%%%%%%%%%%%%%%%%%%%%%%%%%%
%%% Local Variables:
%%% mode: japanese-latex
%%% TeX-master: paper.tex
%%% End:
}\\
%     & & $\Downarrow$\\
%     & & \\
%     \scalebox{0.45}{\begin{tikzpicture}

 % setting
 \tikzset{customer/.style={rectangle,thick,draw=black,minimum size=0.5cm}}
 \tikzset{on_switch/.style={rectangle,fill=black}}
 \tikzset{off_switch/.style={rectangle,draw=black,fill=white}}
 
 \tikzset{node distance =1cm};

 % substation (x, y, label)
 \newcommand{\substation}[3]{
 \draw [very thick] (#1,#2) circle [radius=0.225cm] node[draw=white,minimum size=1cm](#3){};
 \draw [very thick] (#1+0.225,#2)--(#1+0.35,#2)--(#1+0.35,#2+0.3);
 \draw [very thick] (#1-0.225,#2)--(#1-0.35,#2)--(#1-0.35,#2-0.3);
 \draw [very thick] (#1,#2+0.225)--(#1,#2+0.35);
 \draw [very thick] (#1,#2-0.225)--(#1,#2-0.35);
 \draw [very thick] [domain=-0.284:-0.159] plot(\x+#1,\x+#2);
 \draw [very thick] [domain=0.159:0.284] plot(\x+#1,\x+#2);
 \draw [very thick] [domain=-0.284:-0.159] plot(\x+#1,-\x+#2);
 \draw [very thick] [domain=0.159:0.284] plot(\x+#1,-\x+#2);
 }

 %switch node (position, label, cap)
 %% right switch
 \newcommand{\swnodeR}[4]{
 \coordinate[#1] (#2);
 \node[#1,customer] (#2){#4};
 \node[circle, draw=black, text width=0.2cm, 
 right=0cm of #2, scale=0.3, thick] {};
 \node[right=0cm of #2,scale=0.3, minimum size=0.8cm] (#3){};
 }
 %% left switch
 \newcommand{\swnodeL}[4]{
 %\coordinate[#1] (#2);
 \node[#1,customer] (#2){#4};
 \node[circle, draw=black, fill=white, text width=0.2cm, 
 left=0cm of #2, scale=0.3, thick] (#3){};
 }
 % above switch
 \newcommand{\swnodeA}[4]{
 \coordinate[#1] (#2);
 \node[#1,customer] (#2){#4};
 \node[circle, draw=black, text width=0.2cm, 
 above=0cm of #2, scale=0.3, thick] (#3){};
 }
 % below switch
 \newcommand{\swnodeB}[4]{
 \coordinate[#1] (#2);
 \node[#1,customer] (#2){#4};
 \node[circle, draw=black, text width=0.2cm, 
 below=0cm of #2, scale=0.3, thick] {};
 \node[below=0cm of #2,scale=0.3,minimum size=0.8cm] (#3){};
 }
 
 \substation{0}{0}{sub};
 
 % root1
 \node[customer,fill=purple!60,below =4.5cm of sub] (root1) { };
 \swnodeL{left =of root1,fill=purple!60}{node1}{sw1}{ };
 
 \swnodeR{left=of node1,fill=purple!60}{node2}{sw2}{ };
 \node[customer,left=of node2,fill=purple!60] (junc1){ };
 \swnodeL{left =of junc1,fill=purple!60}{node3}{sw3}{ };
 \swnodeA{above= of junc1,fill=purple!60}{node4}{sw4}{ }

 \swnodeR{left=of node3,fill=purple!60}{node5}{sw5}{ };
 \swnodeA{above =of node5,fill=purple!60}{node6}{sw6}{ };

 \swnodeB{above =of node6,fill=purple!60}{node7}{sw7}{ };
 \swnodeA{above =of node7,fill=purple!60}{node29}{sw29}{ };

 \swnodeB{above =of node29,fill=cyan!80}{node30}{sw30}{ };
 
 \swnodeB{above =of node4,fill=purple!60}{node8}{sw8}{ };
 \swnodeA{above =of node8,fill=purple!60}{node31}{sw31}{ };
 
 \swnodeB{above =of node31,fill=cyan!80}{node32}{sw32}{ };

 \swnodeR{right =of root1,fill=purple!60}{node17}{sw17}{ };

 \swnodeL{right =of node17,fill=purple!60}{node18}{sw18}{ };

 % root2
 \node[customer,fill=cyan!80,above=4.5cm of sub, fill=cyan!80] (root2) { };
 \swnodeL{left =of root2,fill=cyan!80}{node9}{sw9}{ };

 \swnodeR{left=of node9,fill=cyan!80}{node10}{sw10}{ };
 \node[customer,left=of node10,fill=cyan!80] (junc2){ };
 \swnodeL{left =of junc2,fill=cyan!80}{node11}{sw11}{ };
 \swnodeB{below =of junc2,fill=cyan!80}{node12}{sw12}{ };
 
 \swnodeR{left =of node11,fill=cyan!80}{node13}{sw13}{ };
 \swnodeB{below =of node13,fill=cyan!80}{node14}{sw14}{ };
 
 \swnodeA{below =of node14,fill=cyan!80}{node15}{sw15}{ };

 \swnodeA{below =of node12,fill=cyan!80}{node16}{sw16}{ };

 \swnodeR{right =of root2,fill=cyan!80}{node22}{sw22}{ };

 \swnodeL{right =of node22,fill=cyan!80}{node23}{sw23}{ };
 
 % root3
 \node[customer,fill=cyan!80,right=5.2cm of sub,fill=yellow!80] (root3) { };
 \swnodeB{below =1.4of root3,fill=yellow!80}{node24}{sw24}{ };
 \swnodeA{above =1.4of root3,fill=yellow!80}{node25}{sw25}{ };

 \swnodeA{below =of node24,fill=purple!60}{node19}{sw19}{ };
 \swnodeL{below =1.3of node19,fill=purple!60}{node20}{sw20}{ };
 
 \swnodeR{left =of node20,fill=purple!60}{node21}{sw21}{ };

 \swnodeB{above =of node25,fill=cyan!80}{node26}{sw26}{ };
 \swnodeL{above =1.3of node26,fill=cyan!80}{node27}{sw27}{ };

 \swnodeR{left =of node27,fill=cyan!80}{node28}{sw28}{ };
 
 % sections
 \foreach \v / \u / \t in {root1/sub/$s_a$,root1/node1/$s_1$,node2/junc1/$s_2$, %
 junc1/node3/$s_3$,junc1/node4/$s_4$,node5/node6/$s_5$,node7/node29/$s_6$,node30/node15/$s_7$, %
 sub/root2/$s_b$,root2/node9/$s_8$,node10/junc2/$s_9$,junc2/node11/$s_{10}$,node12/junc2/$s_{11}$, %
 node14/node13/$s_{12}$,node8/node31/$s_{13}$,node32/node16/$s_{14}$,node17/root1/$s_{15}$, %
 node22/root2/$s_{16}$,root3/sub/$s_c$,node24/root3/$s_{17}$,root3/node25/$s_{18}$, %
 node20/node19/$s_{19}$,node26/node27/$s_{20}$,node21/node18/$s_{21}$,node28/node23/$s_{22}$} %
 \draw[thick] (\v) --  node[auto=right]{\t} (\u);

 % switches
 %% horizontal
 \foreach \v / \u / \t in {sw1/sw2/$sw_{1}$,sw3/sw5/$sw_{2}$,sw9/sw10/$sw_{11}$,
 sw11/sw13/$sw_{12}$,sw18/sw17/$sw_{13}$,sw20/sw21/$sw_{14}$,sw23/sw22/$sw_{15}$,
 sw27/sw28/$sw_{16}$}
 \draw[very thick] (\v) -- node[below=0.2of \v]{\t} (\u.center);
 
 % \foreach \v / \u in {}
 % \draw[very thick] (\v) -- (\u.45);
 %% vertical
 \foreach \v / \u / \t in {sw4/sw8/$sw_{4}$,sw6/sw7/$sw_{3}$,sw15/sw14/$sw_{7}$,%
 sw16/sw12/$sw_{8}$}
 \draw[very thick] (\v) -- node[auto=below]{\t~~~~~~~~~~} (\u.center);

 \foreach \v / \u in {sw29/sw30,sw31/sw32,sw25/sw26,sw19/sw24}
 \draw[very thick] (\v) -- (\u.-30);
\end{tikzpicture}

%%%%%%%%%%%%%%%%%%%%%%%%%%%%%%%%%%%%%%%%%%%%%%%%%%%%%%%%%%
%%% Local Variables:
%%% mode: japanese-latex
%%% TeX-master: paper.tex
%%% End:
}
%     & \lw{$\Leftarrow$} &
%     \scalebox{0.45}{\begin{tikzpicture}

 % setting
 \tikzset{customer/.style={rectangle,thick,draw=black,minimum size=0.5cm}}
 \tikzset{on_switch/.style={rectangle,fill=black}}
 \tikzset{off_switch/.style={rectangle,draw=black,fill=white}}
 
 \tikzset{node distance =1cm};

 % substation (x, y, label)
 \newcommand{\substation}[3]{
 \draw [very thick] (#1,#2) circle [radius=0.225cm] node[draw=none,minimum size=1cm](#3){};
 \draw [very thick] (#1+0.225,#2)--(#1+0.35,#2)--(#1+0.35,#2+0.3);
 \draw [very thick] (#1-0.225,#2)--(#1-0.35,#2)--(#1-0.35,#2-0.3);
 \draw [very thick] (#1,#2+0.225)--(#1,#2+0.35);
 \draw [very thick] (#1,#2-0.225)--(#1,#2-0.35);
 \draw [very thick] [domain=-0.284:-0.159] plot(\x+#1,\x+#2);
 \draw [very thick] [domain=0.159:0.284] plot(\x+#1,\x+#2);
 \draw [very thick] [domain=-0.284:-0.159] plot(\x+#1,-\x+#2);
 \draw [very thick] [domain=0.159:0.284] plot(\x+#1,-\x+#2);
 }

 %switch node (position, label, cap)
 %% right switch
 \newcommand{\swnodeR}[4]{
 \coordinate[#1] (#2);
 \node[#1,customer] (#2){#4};
 \node[circle, draw=black, text width=0.2cm, 
 right=0cm of #2, scale=0.3, thick] {};
 \node[right=0cm of #2,scale=0.3, minimum size=0.8cm] (#3){};
 }
 %% left switch
 \newcommand{\swnodeL}[4]{
 %\coordinate[#1] (#2);
 \node[#1,customer] (#2){#4};
 \node[circle, draw=black, fill=white, text width=0.2cm, 
 left=0cm of #2, scale=0.3, thick] (#3){};
 }
 % above switch
 \newcommand{\swnodeA}[4]{
 \coordinate[#1] (#2);
 \node[#1,customer] (#2){#4};
 \node[circle, draw=black, text width=0.2cm, 
 above=0cm of #2, scale=0.3, thick] (#3){};
 }
 % below switch
 \newcommand{\swnodeB}[4]{
 \coordinate[#1] (#2);
 \node[#1,customer] (#2){#4};
 \node[circle, draw=black, text width=0.2cm, 
 below=0cm of #2, scale=0.3, thick] {};
 \node[below=0cm of #2,scale=0.3,minimum size=0.8cm] (#3){};
 }
 
 \substation{0}{0}{sub};
 
 % root1
 \node[customer,fill=purple!60,below =4.5cm of sub] (root1) { };
 \swnodeL{left =of root1,fill=purple!60}{node1}{sw1}{ };
 
 \swnodeR{left=of node1,fill=purple!60}{node2}{sw2}{ };
 \node[customer,left=of node2,fill=purple!60] (junc1){ };
 \swnodeL{left =of junc1,fill=purple!60}{node3}{sw3}{ };
 \swnodeA{above= of junc1,fill=purple!60}{node4}{sw4}{ }

 \swnodeR{left=of node3,fill=purple!60}{node5}{sw5}{ };
 \swnodeA{above =of node5,fill=purple!60}{node6}{sw6}{ };

 \swnodeB{above =of node6,fill=purple!60}{node7}{sw7}{ };
 \swnodeA{above =of node7,fill=purple!60}{node29}{sw29}{ };

 \swnodeB{above =of node29,fill=cyan!80}{node30}{sw30}{ };
 
 \swnodeB{above =of node4,fill=purple!60}{node8}{sw8}{ };
 \swnodeA{above =of node8,fill=purple!60}{node31}{sw31}{ };
 
 \swnodeB{above =of node31,fill=cyan!80}{node32}{sw32}{ };

 \swnodeR{right =of root1,fill=purple!60}{node17}{sw17}{ };

 \swnodeL{right =of node17,fill=purple!60}{node18}{sw18}{ };

 % root2
 \node[customer,fill=black!20,above=4.5cm of sub, fill=cyan!80] (root2) { };
 \swnodeL{left =of root2,fill=cyan!80}{node9}{sw9}{ };

 \swnodeR{left=of node9,fill=cyan!80}{node10}{sw10}{ };
 \node[customer,left=of node10,fill=cyan!80] (junc2){ };
 \swnodeL{left =of junc2,fill=cyan!80}{node11}{sw11}{ };
 \swnodeB{below =of junc2,fill=cyan!80}{node12}{sw12}{ };
 
 \swnodeR{left =of node11,fill=cyan!80}{node13}{sw13}{ };
 \swnodeB{below =of node13,fill=cyan!80}{node14}{sw14}{ };
 
 \swnodeA{below =of node14,fill=cyan!80}{node15}{sw15}{ };

 \swnodeA{below =of node12,fill=cyan!80}{node16}{sw16}{ };

 \swnodeR{right =of root2,fill=cyan!80}{node22}{sw22}{ };

 \swnodeL{right =of node22,fill=cyan!80}{node23}{sw23}{ };
 
 % root3
 \node[customer,fill=black!20,right=5.2cm of sub,fill=yellow!80] (root3) { };
 \swnodeB{below =1.4of root3,fill=yellow!80}{node24}{sw24}{ };
 \swnodeA{above =1.4of root3,fill=yellow!80}{node25}{sw25}{ };

 \swnodeA{below =of node24,fill=yellow!80}{node19}{sw19}{ };
 \swnodeL{below =1.3of node19,fill=yellow!80}{node20}{sw20}{ };
 
 \swnodeR{left =of node20,fill=purple!60}{node21}{sw21}{ };

 \swnodeB{above =of node25,fill=cyan!80}{node26}{sw26}{ };
 \swnodeL{above =1.3of node26,fill=cyan!80}{node27}{sw27}{ };

 \swnodeR{left =of node27,fill=cyan!80}{node28}{sw28}{ };
 
 % sections
 \foreach \v / \u / \t in {root1/sub/$s_a$,root1/node1/$s_1$,node2/junc1/$s_2$, %
 junc1/node3/$s_3$,junc1/node4/$s_4$,node5/node6/$s_5$,node7/node29/$s_6$,node30/node15/$s_7$, %
 sub/root2/$s_b$,root2/node9/$s_8$,node10/junc2/$s_9$,junc2/node11/$s_{10}$,node12/junc2/$s_{11}$, %
 node14/node13/$s_{12}$,node8/node31/$s_{13}$,node32/node16/$s_{14}$,node17/root1/$s_{15}$, %
 node22/root2/$s_{16}$,root3/sub/$s_c$,node24/root3/$s_{17}$,root3/node25/$s_{18}$, %
 node20/node19/$s_{19}$,node26/node27/$s_{20}$,node21/node18/$s_{21}$,node28/node23/$s_{22}$} %
 \draw[thick] (\v) --  node[auto=right]{\t} (\u);

 % switches
 %% horizontal
 \foreach \v / \u / \t in {sw1/sw2/$sw_{1}$,sw3/sw5/$sw_{2}$,sw9/sw10/$sw_{11}$,
 sw11/sw13/$sw_{12}$,sw18/sw17/$sw_{13}$,sw23/sw22/$sw_{15}$,sw27/sw28/$sw_{16}$}
 \draw[very thick] (\v) -- node[below=0.2of \v]{\t} (\u.center);
 
 \foreach \v / \u in {sw20/sw21}
 \draw[very thick] (\v) -- (\u.45);
 %% vertical
 \foreach \v / \u / \t in {sw4/sw8/$sw_{4}$,sw6/sw7/$sw_{3}$,sw15/sw14/$sw_{7}$,%
 sw16/sw12/$sw_{8}$,sw19/sw24/$sw_{9}$}
 \draw[very thick] (\v) -- node[auto=below]{\t~~~~~~~~~~} (\u.center);

 \foreach \v / \u in {sw29/sw30,sw31/sw32,sw25/sw26}
 \draw[very thick] (\v) -- (\u.-30);

 \coordinate[above=0.3of node26](C);
 \draw[very thick, draw=red] (C) circle[x radius=0.8,y radius=1.8];

\end{tikzpicture}

%%%%%%%%%%%%%%%%%%%%%%%%%%%%%%%%%%%%%%%%%%%%%%%%%%%%%%%%%%
%%% Local Variables:
%%% mode: japanese-latex
%%% TeX-master: paper.tex
%%% End:
}\\
%     $t=3$ (ゴール状態) & & $t=2$
%   \end{tabular}
%   \caption{配電網遷移問題 (遷移制約$d=2$) の解の一例}
%   \label{fig:test-core}
% \end{figure*}
%%%%%%%%%%%%%%%%%%%%%%%%%%%%%%%%%%%%%
\begin{figure*}[tb]
\rotatebox{90}{
\begin{minipage}{\textheight}
 \centering
 \scalebox{1.1}{\begin{tikzpicture}

 % setting
 \tikzset{customer/.style={rectangle,thick,draw=black,minimum size=0.5cm}}
 \tikzset{on_switch/.style={rectangle,fill=black}}
 \tikzset{off_switch/.style={rectangle,draw=black,fill=white}}
 
 \tikzset{node distance =1cm};

 % substation (x, y, label)
 \newcommand{\substation}[3]{
 \draw [very thick] (#1,#2) circle [radius=0.225cm] node[draw=white,minimum size=1cm](#3){};
 \draw [very thick] (#1+0.225,#2)--(#1+0.35,#2)--(#1+0.35,#2+0.3);
 \draw [very thick] (#1-0.225,#2)--(#1-0.35,#2)--(#1-0.35,#2-0.3);
 \draw [very thick] (#1,#2+0.225)--(#1,#2+0.35);
 \draw [very thick] (#1,#2-0.225)--(#1,#2-0.35);
 \draw [very thick] [domain=-0.284:-0.159] plot(\x+#1,\x+#2);
 \draw [very thick] [domain=0.159:0.284] plot(\x+#1,\x+#2);
 \draw [very thick] [domain=-0.284:-0.159] plot(\x+#1,-\x+#2);
 \draw [very thick] [domain=0.159:0.284] plot(\x+#1,-\x+#2);
 }

 %switch node (position, label, cap)
 %% right switch
 \newcommand{\swnodeR}[4]{
 \coordinate[#1] (#2);
 \node[#1,customer] (#2){#4};
 \node[circle, draw=black, text width=0.2cm, 
 right=0cm of #2, scale=0.3, thick] {};
 \node[right=0cm of #2,scale=0.3, minimum size=0.8cm] (#3){};
 }
 %% left switch
 \newcommand{\swnodeL}[4]{
 %\coordinate[#1] (#2);
 \node[#1,customer] (#2){#4};
 \node[circle, draw=black, fill=white, text width=0.2cm, 
 left=0cm of #2, scale=0.3, thick] (#3){};
 }
 % above switch
 \newcommand{\swnodeA}[4]{
 \coordinate[#1] (#2);
 \node[#1,customer] (#2){#4};
 \node[circle, draw=black, text width=0.2cm, 
 above=0cm of #2, scale=0.3, thick] (#3){};
 }
 % below switch
 \newcommand{\swnodeB}[4]{
 \coordinate[#1] (#2);
 \node[#1,customer] (#2){#4};
 \node[circle, draw=black, text width=0.2cm, 
 below=0cm of #2, scale=0.3, thick] {};
 \node[below=0cm of #2,scale=0.3,minimum size=0.8cm] (#3){};
 }
 
 \substation{0}{0}{sub};
 
 % root1
 \node[customer,fill=purple!60,below =4.5cm of sub] (root1) { };
 \swnodeL{left =of root1,fill=purple!60}{node1}{sw1}{ };
 
 \swnodeR{left=of node1,fill=purple!60}{node2}{sw2}{ };
 \node[customer,left=of node2,fill=purple!60] (junc1){ };
 \swnodeL{left =of junc1,fill=purple!60}{node3}{sw3}{ };
 \swnodeA{above= of junc1,fill=purple!60}{node4}{sw4}{ }

 \swnodeR{left=of node3,fill=purple!60}{node5}{sw5}{ };
 \swnodeA{above =of node5,fill=purple!60}{node6}{sw6}{ };

 \swnodeB{above =of node6,fill=cyan!80}{node7}{sw7}{ };
 \swnodeA{above =of node7,fill=cyan!80}{node29}{sw29}{ };

 \swnodeB{above =of node29,fill=cyan!80}{node30}{sw30}{ };
 
 \swnodeB{above =of node4,fill=purple!60}{node8}{sw8}{ };
 \swnodeA{above =of node8,fill=purple!60}{node31}{sw31}{ };
 
 \swnodeB{above =of node31,fill=cyan!80}{node32}{sw32}{ };

 \swnodeR{right =of root1,fill=purple!60}{node17}{sw17}{ };

 \swnodeL{right =of node17,fill=purple!60}{node18}{sw18}{ };

 % root2
 \node[customer,fill=black!20,above=4.5cm of sub, fill=cyan!80] (root2) { };
 \swnodeL{left =of root2,fill=cyan!80}{node9}{sw9}{ };

 \swnodeR{left=of node9,fill=cyan!80}{node10}{sw10}{ };
 \node[customer,left=of node10,fill=cyan!80] (junc2){ };
 \swnodeL{left =of junc2,fill=cyan!80}{node11}{sw11}{ };
 \swnodeB{below =of junc2,fill=cyan!80}{node12}{sw12}{ };
 
 \swnodeR{left =of node11,fill=cyan!80}{node13}{sw13}{ };
 \swnodeB{below =of node13,fill=cyan!80}{node14}{sw14}{ };
 
 \swnodeA{below =of node14,fill=cyan!80}{node15}{sw15}{ };

 \swnodeA{below =of node12,fill=cyan!80}{node16}{sw16}{ };

 \swnodeR{right =of root2,fill=cyan!80}{node22}{sw22}{ };

 \swnodeL{right =of node22,fill=cyan!80}{node23}{sw23}{ };
 
 % root3
 \node[customer,fill=black!20,right=5.2cm of sub,fill=yellow!80] (root3) { };
 \swnodeB{below =1.4of root3,fill=yellow!80}{node24}{sw24}{ };
 \swnodeA{above =1.4of root3,fill=yellow!80}{node25}{sw25}{ };

 \swnodeA{below =of node24,fill=yellow!80}{node19}{sw19}{ };
 \swnodeL{below =1.3of node19,fill=yellow!80}{node20}{sw20}{ };
 
 \swnodeR{left =of node20,fill=purple!60}{node21}{sw21}{ };

 \swnodeB{above =of node25,fill=yellow!80}{node26}{sw26}{ };
 \swnodeL{above =1.3of node26,fill=yellow!80}{node27}{sw27}{ };

 \swnodeR{left =of node27,fill=cyan!80}{node28}{sw28}{ };
 
 % sections
 \foreach \v / \u / \t in {root1/sub/$s_a$,root1/node1/$s_1$,node2/junc1/$s_2$, %
 junc1/node3/$s_3$,junc1/node4/$s_4$,node5/node6/$s_5$,node7/node29/$s_6$,node30/node15/$s_7$, %
 sub/root2/$s_b$,root2/node9/$s_8$,node10/junc2/$s_9$,junc2/node11/$s_{10}$,node12/junc2/$s_{11}$, %
 node14/node13/$s_{12}$,node8/node31/$s_{13}$,node32/node16/$s_{14}$,node17/root1/$s_{15}$, %
 node22/root2/$s_{16}$,root3/sub/$s_c$,node24/root3/$s_{17}$,root3/node25/$s_{18}$, %
 node20/node19/$s_{19}$,node26/node27/$s_{20}$,node21/node18/$s_{21}$,node28/node23/$s_{22}$} %
 \draw[thick] (\v) --  node[auto=right]{\t} (\u);

 % switches
 %% horizontal
 \foreach \v / \u / \t in {sw1/sw2/$sw_{1}$,sw3/sw5/$sw_{2}$,sw9/sw10/$sw_{11}$,
 sw11/sw13/$sw_{12}$,sw18/sw17/$sw_{13}$,sw23/sw22/$sw_{15}$}
 \draw[very thick] (\v) -- node[below=0.2of \v]{\t} (\u.center);
 
 \foreach \v / \u in {sw20/sw21,sw27/sw28}
 \draw[very thick] (\v) -- (\u.45);
 %% vertical
 \foreach \v / \u / \t in {sw4/sw8/$sw_{4}$,sw29/sw30/$sw_{5}$,sw15/sw14/$sw_{7}$,%
 sw16/sw12/$sw_{8}$,sw19/sw24/$sw_{9}$,sw25/sw26/$sw_{10}$}
 \draw[very thick] (\v) -- node[auto=below]{\t~~~~~~~~~~} (\u.center);

 \foreach \v / \u in {sw6/sw7,sw31/sw32}
 \draw[very thick] (\v) -- (\u.-30);
\end{tikzpicture}

%%%%%%%%%%%%%%%%%%%%%%%%%%%%%%%%%%%%%%%%%%%%%%%%%%%%%%%%%%
%%% Local Variables:
%%% mode: japanese-latex
%%% TeX-master: paper.tex
%%% End:
}
 \caption{配電網遷移問題 (遷移制約$d=2$) の解の一例 \textbf{($\mathbf{t=0}$ (スタート状態))}}
 \label{fig:test-core}
\end{minipage}
}
\end{figure*}
%
\begin{figure*}[tb]
\rotatebox{90}{
\begin{minipage}{\textheight}
 \centering
 \scalebox{1.1}{\begin{tikzpicture}

 % setting
 \tikzset{customer/.style={rectangle,thick,draw=black,minimum size=0.5cm}}
 \tikzset{on_switch/.style={rectangle,fill=black}}
 \tikzset{off_switch/.style={rectangle,draw=black,fill=white}}
 
 \tikzset{node distance =1cm};

 % substation (x, y, label)
 \newcommand{\substation}[3]{
 \draw [very thick] (#1,#2) circle [radius=0.225cm] node[draw=white,minimum size=1cm](#3){};
 \draw [very thick] (#1+0.225,#2)--(#1+0.35,#2)--(#1+0.35,#2+0.3);
 \draw [very thick] (#1-0.225,#2)--(#1-0.35,#2)--(#1-0.35,#2-0.3);
 \draw [very thick] (#1,#2+0.225)--(#1,#2+0.35);
 \draw [very thick] (#1,#2-0.225)--(#1,#2-0.35);
 \draw [very thick] [domain=-0.284:-0.159] plot(\x+#1,\x+#2);
 \draw [very thick] [domain=0.159:0.284] plot(\x+#1,\x+#2);
 \draw [very thick] [domain=-0.284:-0.159] plot(\x+#1,-\x+#2);
 \draw [very thick] [domain=0.159:0.284] plot(\x+#1,-\x+#2);
 }

 %switch node (position, label, cap)
 %% right switch
 \newcommand{\swnodeR}[4]{
 \coordinate[#1] (#2);
 \node[#1,customer] (#2){#4};
 \node[circle, draw=black, text width=0.2cm, 
 right=0cm of #2, scale=0.3, thick] {};
 \node[right=0cm of #2,scale=0.3, minimum size=0.8cm] (#3){};
 }
 %% left switch
 \newcommand{\swnodeL}[4]{
 %\coordinate[#1] (#2);
 \node[#1,customer] (#2){#4};
 \node[circle, draw=black, fill=white, text width=0.2cm, 
 left=0cm of #2, scale=0.3, thick] (#3){};
 }
 % above switch
 \newcommand{\swnodeA}[4]{
 \coordinate[#1] (#2);
 \node[#1,customer] (#2){#4};
 \node[circle, draw=black, text width=0.2cm, 
 above=0cm of #2, scale=0.3, thick] (#3){};
 }
 % below switch
 \newcommand{\swnodeB}[4]{
 \coordinate[#1] (#2);
 \node[#1,customer] (#2){#4};
 \node[circle, draw=black, text width=0.2cm, 
 below=0cm of #2, scale=0.3, thick] {};
 \node[below=0cm of #2,scale=0.3,minimum size=0.8cm] (#3){};
 }
 
 \substation{0}{0}{sub};
 
 % root1
 \node[customer,fill=purple!60,below =4.5cm of sub] (root1) { };
 \swnodeL{left =of root1,fill=purple!60}{node1}{sw1}{ };
 
 \swnodeR{left=of node1,fill=purple!60}{node2}{sw2}{ };
 \node[customer,left=of node2,fill=purple!60] (junc1){ };
 \swnodeL{left =of junc1,fill=purple!60}{node3}{sw3}{ };
 \swnodeA{above= of junc1,fill=purple!60}{node4}{sw4}{ }

 \swnodeR{left=of node3,fill=purple!60}{node5}{sw5}{ };
 \swnodeA{above =of node5,fill=purple!60}{node6}{sw6}{ };

 \swnodeB{above =of node6,fill=purple!60}{node7}{sw7}{ };
 \swnodeA{above =of node7,fill=purple!60}{node29}{sw29}{ };

 \swnodeB{above =of node29,fill=cyan!80}{node30}{sw30}{ };
 
 \swnodeB{above =of node4,fill=purple!60}{node8}{sw8}{ };
 \swnodeA{above =of node8,fill=purple!60}{node31}{sw31}{ };
 
 \swnodeB{above =of node31,fill=cyan!80}{node32}{sw32}{ };

 \swnodeR{right =of root1,fill=purple!60}{node17}{sw17}{ };

 \swnodeL{right =of node17,fill=purple!60}{node18}{sw18}{ };

 % root2
 \node[customer,fill=black!20,above=4.5cm of sub, fill=cyan!80] (root2) { };
 \swnodeL{left =of root2,fill=cyan!80}{node9}{sw9}{ };

 \swnodeR{left=of node9,fill=cyan!80}{node10}{sw10}{ };
 \node[customer,left=of node10,fill=cyan!80] (junc2){ };
 \swnodeL{left =of junc2,fill=cyan!80}{node11}{sw11}{ };
 \swnodeB{below =of junc2,fill=cyan!80}{node12}{sw12}{ };
 
 \swnodeR{left =of node11,fill=cyan!80}{node13}{sw13}{ };
 \swnodeB{below =of node13,fill=cyan!80}{node14}{sw14}{ };
 
 \swnodeA{below =of node14,fill=cyan!80}{node15}{sw15}{ };

 \swnodeA{below =of node12,fill=cyan!80}{node16}{sw16}{ };

 \swnodeR{right =of root2,fill=cyan!80}{node22}{sw22}{ };

 \swnodeL{right =of node22,fill=cyan!80}{node23}{sw23}{ };
 
 % root3
 \node[customer,fill=black!20,right=5.2cm of sub,fill=yellow!80] (root3) { };
 \swnodeB{below =1.4of root3,fill=yellow!80}{node24}{sw24}{ };
 \swnodeA{above =1.4of root3,fill=yellow!80}{node25}{sw25}{ };

 \swnodeA{below =of node24,fill=yellow!80}{node19}{sw19}{ };
 \swnodeL{below =1.3of node19,fill=yellow!80}{node20}{sw20}{ };
 
 \swnodeR{left =of node20,fill=purple!60}{node21}{sw21}{ };

 \swnodeB{above =of node25,fill=yellow!80}{node26}{sw26}{ };
 \swnodeL{above =1.3of node26,fill=yellow!80}{node27}{sw27}{ };

 \swnodeR{left =of node27,fill=cyan!80}{node28}{sw28}{ };
 
 % sections
 \foreach \v / \u / \t in {root1/sub/$s_a$,root1/node1/$s_1$,node2/junc1/$s_2$, %
 junc1/node3/$s_3$,junc1/node4/$s_4$,node5/node6/$s_5$,node7/node29/$s_6$,node30/node15/$s_7$, %
 sub/root2/$s_b$,root2/node9/$s_8$,node10/junc2/$s_9$,junc2/node11/$s_{10}$,node12/junc2/$s_{11}$, %
 node14/node13/$s_{12}$,node8/node31/$s_{13}$,node32/node16/$s_{14}$,node17/root1/$s_{15}$, %
 node22/root2/$s_{16}$,root3/sub/$s_c$,node24/root3/$s_{17}$,root3/node25/$s_{18}$, %
 node20/node19/$s_{19}$,node26/node27/$s_{20}$,node21/node18/$s_{21}$,node28/node23/$s_{22}$} %
 \draw[thick] (\v) --  node[auto=right]{\t} (\u);

 % switches
 %% horizontal
 \foreach \v / \u / \t in {sw1/sw2/$sw_{1}$,sw3/sw5/$sw_{2}$,sw9/sw10/$sw_{11}$,
 sw11/sw13/$sw_{12}$,sw18/sw17/$sw_{13}$,sw23/sw22/$sw_{15}$}
 \draw[very thick] (\v) -- node[below=0.2of \v]{\t} (\u.center);
 
 \foreach \v / \u in {sw20/sw21,sw27/sw28}
 \draw[very thick] (\v) -- (\u.45);
 %% vertical
 \foreach \v / \u / \t in {sw4/sw8/$sw_{4}$,sw6/sw7/$sw_{3}$,sw15/sw14/$sw_{7}$,%
 sw16/sw12/$sw_{8}$,sw19/sw24/$sw_{9}$,sw25/sw26/$sw_{10}$}
 \draw[very thick] (\v) -- node[auto=below]{\t~~~~~~~~~~} (\u.center);

 \foreach \v / \u in {sw29/sw30,sw31/sw32}
 \draw[very thick] (\v) -- (\u.-30);
\end{tikzpicture}

%%%%%%%%%%%%%%%%%%%%%%%%%%%%%%%%%%%%%%%%%%%%%%%%%%%%%%%%%%
%%% Local Variables:
%%% mode: japanese-latex
%%% TeX-master: paper.tex
%%% End:
}
 \setcounter{figure}{2}
 \caption{配電網遷移問題 (遷移制約$d=2$) の解の一例 \textbf{($\mathbf{t=1}$)}}
 %\label{fig:test-output}
\end{minipage}
}
\end{figure*}
%
\begin{figure*}[tb]
\rotatebox{90}{
\begin{minipage}{\textheight}
 \centering
 \scalebox{1.1}{\begin{tikzpicture}

 % setting
 \tikzset{customer/.style={rectangle,thick,draw=black,minimum size=0.5cm}}
 \tikzset{on_switch/.style={rectangle,fill=black}}
 \tikzset{off_switch/.style={rectangle,draw=black,fill=white}}
 
 \tikzset{node distance =1cm};

 % substation (x, y, label)
 \newcommand{\substation}[3]{
 \draw [very thick] (#1,#2) circle [radius=0.225cm] node[draw=none,minimum size=1cm](#3){};
 \draw [very thick] (#1+0.225,#2)--(#1+0.35,#2)--(#1+0.35,#2+0.3);
 \draw [very thick] (#1-0.225,#2)--(#1-0.35,#2)--(#1-0.35,#2-0.3);
 \draw [very thick] (#1,#2+0.225)--(#1,#2+0.35);
 \draw [very thick] (#1,#2-0.225)--(#1,#2-0.35);
 \draw [very thick] [domain=-0.284:-0.159] plot(\x+#1,\x+#2);
 \draw [very thick] [domain=0.159:0.284] plot(\x+#1,\x+#2);
 \draw [very thick] [domain=-0.284:-0.159] plot(\x+#1,-\x+#2);
 \draw [very thick] [domain=0.159:0.284] plot(\x+#1,-\x+#2);
 }

 %switch node (position, label, cap)
 %% right switch
 \newcommand{\swnodeR}[4]{
 \coordinate[#1] (#2);
 \node[#1,customer] (#2){#4};
 \node[circle, draw=black, text width=0.2cm, 
 right=0cm of #2, scale=0.3, thick] {};
 \node[right=0cm of #2,scale=0.3, minimum size=0.8cm] (#3){};
 }
 %% left switch
 \newcommand{\swnodeL}[4]{
 %\coordinate[#1] (#2);
 \node[#1,customer] (#2){#4};
 \node[circle, draw=black, fill=white, text width=0.2cm, 
 left=0cm of #2, scale=0.3, thick] (#3){};
 }
 % above switch
 \newcommand{\swnodeA}[4]{
 \coordinate[#1] (#2);
 \node[#1,customer] (#2){#4};
 \node[circle, draw=black, text width=0.2cm, 
 above=0cm of #2, scale=0.3, thick] (#3){};
 }
 % below switch
 \newcommand{\swnodeB}[4]{
 \coordinate[#1] (#2);
 \node[#1,customer] (#2){#4};
 \node[circle, draw=black, text width=0.2cm, 
 below=0cm of #2, scale=0.3, thick] {};
 \node[below=0cm of #2,scale=0.3,minimum size=0.8cm] (#3){};
 }
 
 \substation{0}{0}{sub};
 
 % root1
 \node[customer,fill=purple!60,below =4.5cm of sub] (root1) { };
 \swnodeL{left =of root1,fill=purple!60}{node1}{sw1}{ };
 
 \swnodeR{left=of node1,fill=purple!60}{node2}{sw2}{ };
 \node[customer,left=of node2,fill=purple!60] (junc1){ };
 \swnodeL{left =of junc1,fill=purple!60}{node3}{sw3}{ };
 \swnodeA{above= of junc1,fill=purple!60}{node4}{sw4}{ }

 \swnodeR{left=of node3,fill=purple!60}{node5}{sw5}{ };
 \swnodeA{above =of node5,fill=purple!60}{node6}{sw6}{ };

 \swnodeB{above =of node6,fill=purple!60}{node7}{sw7}{ };
 \swnodeA{above =of node7,fill=purple!60}{node29}{sw29}{ };

 \swnodeB{above =of node29,fill=cyan!80}{node30}{sw30}{ };
 
 \swnodeB{above =of node4,fill=purple!60}{node8}{sw8}{ };
 \swnodeA{above =of node8,fill=purple!60}{node31}{sw31}{ };
 
 \swnodeB{above =of node31,fill=cyan!80}{node32}{sw32}{ };

 \swnodeR{right =of root1,fill=purple!60}{node17}{sw17}{ };

 \swnodeL{right =of node17,fill=purple!60}{node18}{sw18}{ };

 % root2
 \node[customer,fill=black!20,above=4.5cm of sub, fill=cyan!80] (root2) { };
 \swnodeL{left =of root2,fill=cyan!80}{node9}{sw9}{ };

 \swnodeR{left=of node9,fill=cyan!80}{node10}{sw10}{ };
 \node[customer,left=of node10,fill=cyan!80] (junc2){ };
 \swnodeL{left =of junc2,fill=cyan!80}{node11}{sw11}{ };
 \swnodeB{below =of junc2,fill=cyan!80}{node12}{sw12}{ };
 
 \swnodeR{left =of node11,fill=cyan!80}{node13}{sw13}{ };
 \swnodeB{below =of node13,fill=cyan!80}{node14}{sw14}{ };
 
 \swnodeA{below =of node14,fill=cyan!80}{node15}{sw15}{ };

 \swnodeA{below =of node12,fill=cyan!80}{node16}{sw16}{ };

 \swnodeR{right =of root2,fill=cyan!80}{node22}{sw22}{ };

 \swnodeL{right =of node22,fill=cyan!80}{node23}{sw23}{ };
 
 % root3
 \node[customer,fill=black!20,right=5.2cm of sub,fill=yellow!80] (root3) { };
 \swnodeB{below =1.4of root3,fill=yellow!80}{node24}{sw24}{ };
 \swnodeA{above =1.4of root3,fill=yellow!80}{node25}{sw25}{ };

 \swnodeA{below =of node24,fill=yellow!80}{node19}{sw19}{ };
 \swnodeL{below =1.3of node19,fill=yellow!80}{node20}{sw20}{ };
 
 \swnodeR{left =of node20,fill=purple!60}{node21}{sw21}{ };

 \swnodeB{above =of node25,fill=cyan!80}{node26}{sw26}{ };
 \swnodeL{above =1.3of node26,fill=cyan!80}{node27}{sw27}{ };

 \swnodeR{left =of node27,fill=cyan!80}{node28}{sw28}{ };
 
 % sections
 \foreach \v / \u / \t in {root1/sub/$s_a$,root1/node1/$s_1$,node2/junc1/$s_2$, %
 junc1/node3/$s_3$,junc1/node4/$s_4$,node5/node6/$s_5$,node7/node29/$s_6$,node30/node15/$s_7$, %
 sub/root2/$s_b$,root2/node9/$s_8$,node10/junc2/$s_9$,junc2/node11/$s_{10}$,node12/junc2/$s_{11}$, %
 node14/node13/$s_{12}$,node8/node31/$s_{13}$,node32/node16/$s_{14}$,node17/root1/$s_{15}$, %
 node22/root2/$s_{16}$,root3/sub/$s_c$,node24/root3/$s_{17}$,root3/node25/$s_{18}$, %
 node20/node19/$s_{19}$,node26/node27/$s_{20}$,node21/node18/$s_{21}$,node28/node23/$s_{22}$} %
 \draw[thick] (\v) --  node[auto=right]{\t} (\u);

 % switches
 %% horizontal
 \foreach \v / \u / \t in {sw1/sw2/$sw_{1}$,sw3/sw5/$sw_{2}$,sw9/sw10/$sw_{11}$,
 sw11/sw13/$sw_{12}$,sw18/sw17/$sw_{13}$,sw23/sw22/$sw_{15}$,sw27/sw28/$sw_{16}$}
 \draw[very thick] (\v) -- node[below=0.2of \v]{\t} (\u.center);
 
 \foreach \v / \u in {sw20/sw21}
 \draw[very thick] (\v) -- (\u.45);
 %% vertical
 \foreach \v / \u / \t in {sw4/sw8/$sw_{4}$,sw6/sw7/$sw_{3}$,sw15/sw14/$sw_{7}$,%
 sw16/sw12/$sw_{8}$,sw19/sw24/$sw_{9}$}
 \draw[very thick] (\v) -- node[auto=below]{\t~~~~~~~~~~} (\u.center);

 \foreach \v / \u in {sw29/sw30,sw31/sw32,sw25/sw26}
 \draw[very thick] (\v) -- (\u.-30);

 \coordinate[above=0.3of node26](C);
 \draw[very thick, draw=red] (C) circle[x radius=0.8,y radius=1.8];

\end{tikzpicture}

%%%%%%%%%%%%%%%%%%%%%%%%%%%%%%%%%%%%%%%%%%%%%%%%%%%%%%%%%%
%%% Local Variables:
%%% mode: japanese-latex
%%% TeX-master: paper.tex
%%% End:
}
 \setcounter{figure}{2}
 \caption{配電網遷移問題 (遷移制約$d=2$) の解の一例 \textbf{($\mathbf{t=2}$)}}
 %\label{fig:test-output}
\end{minipage}
}
\end{figure*}
%
\begin{figure*}[tb]
\rotatebox{90}{
\begin{minipage}{\textheight}
 \centering
 \scalebox{1.1}{\begin{tikzpicture}

 % setting
 \tikzset{customer/.style={rectangle,thick,draw=black,minimum size=0.5cm}}
 \tikzset{on_switch/.style={rectangle,fill=black}}
 \tikzset{off_switch/.style={rectangle,draw=black,fill=white}}
 
 \tikzset{node distance =1cm};

 % substation (x, y, label)
 \newcommand{\substation}[3]{
 \draw [very thick] (#1,#2) circle [radius=0.225cm] node[draw=white,minimum size=1cm](#3){};
 \draw [very thick] (#1+0.225,#2)--(#1+0.35,#2)--(#1+0.35,#2+0.3);
 \draw [very thick] (#1-0.225,#2)--(#1-0.35,#2)--(#1-0.35,#2-0.3);
 \draw [very thick] (#1,#2+0.225)--(#1,#2+0.35);
 \draw [very thick] (#1,#2-0.225)--(#1,#2-0.35);
 \draw [very thick] [domain=-0.284:-0.159] plot(\x+#1,\x+#2);
 \draw [very thick] [domain=0.159:0.284] plot(\x+#1,\x+#2);
 \draw [very thick] [domain=-0.284:-0.159] plot(\x+#1,-\x+#2);
 \draw [very thick] [domain=0.159:0.284] plot(\x+#1,-\x+#2);
 }

 %switch node (position, label, cap)
 %% right switch
 \newcommand{\swnodeR}[4]{
 \coordinate[#1] (#2);
 \node[#1,customer] (#2){#4};
 \node[circle, draw=black, text width=0.2cm, 
 right=0cm of #2, scale=0.3, thick] {};
 \node[right=0cm of #2,scale=0.3, minimum size=0.8cm] (#3){};
 }
 %% left switch
 \newcommand{\swnodeL}[4]{
 %\coordinate[#1] (#2);
 \node[#1,customer] (#2){#4};
 \node[circle, draw=black, fill=white, text width=0.2cm, 
 left=0cm of #2, scale=0.3, thick] (#3){};
 }
 % above switch
 \newcommand{\swnodeA}[4]{
 \coordinate[#1] (#2);
 \node[#1,customer] (#2){#4};
 \node[circle, draw=black, text width=0.2cm, 
 above=0cm of #2, scale=0.3, thick] (#3){};
 }
 % below switch
 \newcommand{\swnodeB}[4]{
 \coordinate[#1] (#2);
 \node[#1,customer] (#2){#4};
 \node[circle, draw=black, text width=0.2cm, 
 below=0cm of #2, scale=0.3, thick] {};
 \node[below=0cm of #2,scale=0.3,minimum size=0.8cm] (#3){};
 }
 
 \substation{0}{0}{sub};
 
 % root1
 \node[customer,fill=purple!60,below =4.5cm of sub] (root1) { };
 \swnodeL{left =of root1,fill=purple!60}{node1}{sw1}{ };
 
 \swnodeR{left=of node1,fill=purple!60}{node2}{sw2}{ };
 \node[customer,left=of node2,fill=purple!60] (junc1){ };
 \swnodeL{left =of junc1,fill=purple!60}{node3}{sw3}{ };
 \swnodeA{above= of junc1,fill=purple!60}{node4}{sw4}{ }

 \swnodeR{left=of node3,fill=purple!60}{node5}{sw5}{ };
 \swnodeA{above =of node5,fill=purple!60}{node6}{sw6}{ };

 \swnodeB{above =of node6,fill=purple!60}{node7}{sw7}{ };
 \swnodeA{above =of node7,fill=purple!60}{node29}{sw29}{ };

 \swnodeB{above =of node29,fill=cyan!80}{node30}{sw30}{ };
 
 \swnodeB{above =of node4,fill=purple!60}{node8}{sw8}{ };
 \swnodeA{above =of node8,fill=purple!60}{node31}{sw31}{ };
 
 \swnodeB{above =of node31,fill=cyan!80}{node32}{sw32}{ };

 \swnodeR{right =of root1,fill=purple!60}{node17}{sw17}{ };

 \swnodeL{right =of node17,fill=purple!60}{node18}{sw18}{ };

 % root2
 \node[customer,fill=cyan!80,above=4.5cm of sub, fill=cyan!80] (root2) { };
 \swnodeL{left =of root2,fill=cyan!80}{node9}{sw9}{ };

 \swnodeR{left=of node9,fill=cyan!80}{node10}{sw10}{ };
 \node[customer,left=of node10,fill=cyan!80] (junc2){ };
 \swnodeL{left =of junc2,fill=cyan!80}{node11}{sw11}{ };
 \swnodeB{below =of junc2,fill=cyan!80}{node12}{sw12}{ };
 
 \swnodeR{left =of node11,fill=cyan!80}{node13}{sw13}{ };
 \swnodeB{below =of node13,fill=cyan!80}{node14}{sw14}{ };
 
 \swnodeA{below =of node14,fill=cyan!80}{node15}{sw15}{ };

 \swnodeA{below =of node12,fill=cyan!80}{node16}{sw16}{ };

 \swnodeR{right =of root2,fill=cyan!80}{node22}{sw22}{ };

 \swnodeL{right =of node22,fill=cyan!80}{node23}{sw23}{ };
 
 % root3
 \node[customer,fill=cyan!80,right=5.2cm of sub,fill=yellow!80] (root3) { };
 \swnodeB{below =1.4of root3,fill=yellow!80}{node24}{sw24}{ };
 \swnodeA{above =1.4of root3,fill=yellow!80}{node25}{sw25}{ };

 \swnodeA{below =of node24,fill=purple!60}{node19}{sw19}{ };
 \swnodeL{below =1.3of node19,fill=purple!60}{node20}{sw20}{ };
 
 \swnodeR{left =of node20,fill=purple!60}{node21}{sw21}{ };

 \swnodeB{above =of node25,fill=cyan!80}{node26}{sw26}{ };
 \swnodeL{above =1.3of node26,fill=cyan!80}{node27}{sw27}{ };

 \swnodeR{left =of node27,fill=cyan!80}{node28}{sw28}{ };
 
 % sections
 \foreach \v / \u / \t in {root1/sub/$s_a$,root1/node1/$s_1$,node2/junc1/$s_2$, %
 junc1/node3/$s_3$,junc1/node4/$s_4$,node5/node6/$s_5$,node7/node29/$s_6$,node30/node15/$s_7$, %
 sub/root2/$s_b$,root2/node9/$s_8$,node10/junc2/$s_9$,junc2/node11/$s_{10}$,node12/junc2/$s_{11}$, %
 node14/node13/$s_{12}$,node8/node31/$s_{13}$,node32/node16/$s_{14}$,node17/root1/$s_{15}$, %
 node22/root2/$s_{16}$,root3/sub/$s_c$,node24/root3/$s_{17}$,root3/node25/$s_{18}$, %
 node20/node19/$s_{19}$,node26/node27/$s_{20}$,node21/node18/$s_{21}$,node28/node23/$s_{22}$} %
 \draw[thick] (\v) --  node[auto=right]{\t} (\u);

 % switches
 %% horizontal
 \foreach \v / \u / \t in {sw1/sw2/$sw_{1}$,sw3/sw5/$sw_{2}$,sw9/sw10/$sw_{11}$,
 sw11/sw13/$sw_{12}$,sw18/sw17/$sw_{13}$,sw20/sw21/$sw_{14}$,sw23/sw22/$sw_{15}$,
 sw27/sw28/$sw_{16}$}
 \draw[very thick] (\v) -- node[below=0.2of \v]{\t} (\u.center);
 
 % \foreach \v / \u in {}
 % \draw[very thick] (\v) -- (\u.45);
 %% vertical
 \foreach \v / \u / \t in {sw4/sw8/$sw_{4}$,sw6/sw7/$sw_{3}$,sw15/sw14/$sw_{7}$,%
 sw16/sw12/$sw_{8}$}
 \draw[very thick] (\v) -- node[auto=below]{\t~~~~~~~~~~} (\u.center);

 \foreach \v / \u in {sw29/sw30,sw31/sw32,sw25/sw26,sw19/sw24}
 \draw[very thick] (\v) -- (\u.-30);
\end{tikzpicture}

%%%%%%%%%%%%%%%%%%%%%%%%%%%%%%%%%%%%%%%%%%%%%%%%%%%%%%%%%%
%%% Local Variables:
%%% mode: japanese-latex
%%% TeX-master: paper.tex
%%% End:
}
 \setcounter{figure}{2}
 \caption{配電網遷移問題 (遷移制約$d=2$) の解の一例 \textbf{($\mathbf{t=3}$ (ゴール状態))}}
 %\label{fig:test-output}
\end{minipage}
}
\end{figure*}
%%%%%%%%%%%%%%%%%%%%%%%%%%%%%%%%%%%%%

\textbf{配電網遷移問題}は,配電網問題とその2つの実行可能解が与え
られたとき,一方の解(スタート状態)から他方の解(ゴール状態)へ,遷移制約
を満たしつつ,実行可能解のみを経由して最短ステップ長でのスイッチの切替
手順を求める問題である.
% この問題は,配電網の構成制御における障害時の復旧予測への応用を狙いとし
% ている.
本研究では,
各ステップ$t$で切替可能なスイッチの数を$d$個に制限する一般的な
遷移制約を用いる.

配電網遷移問題の解の一例を図\ref{fig:test-core}に示す.
この例では,各ステップ$t$で切替可能なスイッチの数を$d=2$個に制限している.
この解のステップ長は3であり,スタート状態($t=0$)からゴール状態($t=3$)
まで,配電網問題の制約を満たしながら遷移していることがわかる.
例えば,ステップ$t=0$から$t=1$への遷移では,スイッチ$\{sw_3,sw_5\}$が
それぞれ切り替わっている.

% 根付き全域森問題の入力例となるグラフを図\ref{fig:test-netsuki-input}に示す.
% 図\ref{fig:test-netsuki-input}は,図\ref{fig:test-input}で示した配電網に対応しており,
% 配電区間$\{s_i ~|~ 1 \leq i \leq 22\}$は,スイッチで区切られる1つのまとまりごとに
% \textbf{ノード}に対応する.例えば,区間$\{s_2,s_3,s_4\}$は,ノード5に対応している.
% スイッチ$\{sw_1,\ldots,sw_{16}\}$は,\textbf{辺}に対応する.
% また,図中の色付きノード$\{r_1,r_2,r_3\}$は変電所を含むことを意味しており,
% \textbf{根}に対応している.

% 根付き全域森の例を図\ref{fig:test-netsuki-output}に示す.根付き全域森は,
% 各連結成分が必ずちょうど1つの根をもつ木構造を形成することで,
% 非閉路制約と根付き連結制約を満たす.図\ref{fig:test-netsuki-output}は,
% 図\ref{fig:test-output}の配電網問題の解に対応している.


%%% Local Variables:
%%% mode: japanese-latex
%%% TeX-master: "paper"
%%% End:
