\chapter{配電網遷移問題への拡張}\label{chap:core}

%%%%%%%%%%%%%%%%%%%%%%%%%%%%%%%%%
\lstinputlisting[float=tb,caption={%
  配電網遷移問題(図~\ref{fig:test-core})のファクト表現},%
captionpos=b,frame=single,label=code:test-core.lp,%
xrightmargin=1zw,% 
xleftmargin=1zw,% 
numbersep=5pt,%
numbers=left,%
breaklines=true,%
columns=fullflexible,keepspaces=true,%
basicstyle=\ttfamily\footnotesize]{code/core-input.lp}
%%%%%%%%%%%%%%%%%%%%%%%%%%%%%%%%%
\lstinputlisting[float=tb,caption={%
  配電網遷移問題のシングルショット符号化},%
captionpos=b,frame=single,label=code:singleshot.lp,%
xrightmargin=1zw,% 
xleftmargin=1zw,% 
numbersep=5pt,%
numbers=left,%
breaklines=true,%
columns=fullflexible,keepspaces=true,%
basicstyle=\ttfamily\small]{code/singleshot.lp}
%%%%%%%%%%%%%%%%%%%%%%%%%%%%%%%%%
\lstinputlisting[float=tb,caption={%
  配電網遷移問題のマルチショット符号化},%
captionpos=b,frame=single,label=code:pw-core.lp,%
xrightmargin=1zw,% 
xleftmargin=1zw,% 
numbersep=5pt,%
numbers=left,%
breaklines=true,%
columns=fullflexible,keepspaces=true,%
basicstyle=\ttfamily\small]{code/pw-core.lp}
%%%%%%%%%%%%%%%%%%%%%%%%%%%%%%%%%

配電網遷移問題は,配電網問題とその2つの実行可能解が与え
られたとき,一方の解(スタート状態)から他方の解(ゴール状態)へ,遷移制約
を満たしつつ,実行可能解のみを経由して最短ステップ長でのスイッチの切替
手順を求める問題である.
各ステップ$t$で切替可能なスイッチの数を$d$個に制限する一般的な
遷移制約を用いる.

本節では,まず,配電網遷移問題インスタンスのASPファクト形式について述べる.
次に,第~\ref{chap:encode}節で示した配電網問題のASP符号化を拡張し,
配電網遷移問題のASP符号化を提案する.
最後に,提案するASP符号化を用いて行った実行実験の結果を示す.

%%%%%%%%%%%%%%%%%%%%%%%%%%%%%%%%%
\textbf{ファクト形式.}
配電網遷移問題は,配電網問題インスタンスに加え,
新たにスタート状態とゴール状態が入力として与えられる.
コード\ref{code:test-core.lp}に,
配電網遷移問題の例(図\ref{fig:test-core})における
スタート状態($t=0$)とゴール状態($t=3$)のファクト表現を示す.
スタート状態における閉じたスイッチは,\code{init_switch/1}によって表される.
また,ゴール状態での閉じたスイッチは,\code{goal_switch/1}によって表される.

%%%%%%%%%%%%%%%%%%%%%%%%%%%%%%%%%
\textbf{シングルショット符号化.}
配電網遷移問題のASP符号化をコード\ref{code:singleshot.lp}に示す.
この符号化は,与えられた配電網遷移問題に対して,ステップ長\code{t}の解が
存在するかを判定し,存在する場合はその解を返す論理プログラムである.
配電網問題のトポロジ制約の有向符号化(コード\ref{code:srf3.lp}),
及び,電流制約のASP符号化(コード\ref{code:electrical.lp})からの拡張点は,
以下の通りである.
\begin{itemize}
 \item 新しくステップを表すアトム\code{t(TT)}を導入.
 \item 制約を表すルールの各アトムにステップを表す項\code{TT}を
       引数として追加.
 \item スタート状態とステップ\code{0},ゴール状態とステップ\code{t}を対応させる
       ルールを追加.
 \item 遷移制約を表すルールを追加.
\end{itemize}
2行目の\code{t(0..t).}は,\code{t(0).},\code{t(1).}, ~... \code{t(t).}に展開され,
各ステップの識別子を表す.定数\code{t}はステップ長を表す整数値であり,
実行時に与えられる.5行目のルールは,スタート状態とステップ\code{0}の閉じたスイッチが
一致することを強制する.同様に,8行目のルールで,ゴール状態とステップ\code{t}の閉じた
スイッチが一致することを強制する.
%
遷移制約は,41--43行目のルールで表される.
\code{changed(SW,t)}は,ステップ\code{t-1}とステップ\code{t}の間でスイッチ\code{SW}
の状態が変化したことを意味する.
43行目のルールで,各ステップ\code{t}において,変化したスイッチの数が\code{d}であること
を保証している.

\textbf{マルチショット符号化.}
シングルショット符号化は, 配電網問題のトポロジ制約と電流制約の符号化の自然な拡張に
なっている.この符号化を用いて遷移問題を解くには,ステップ長\code{t}を増やしながら,
複数の問題を繰り返し解く必要がある.しかし,各問題中の制約の大部分は共通であるため,
ASPシステムが同一の探索空間を何度も調べることになり,求解効率が低下するという問題点
がある.

この問題を解決するために,ASP システム \clingo のマルチショットASP解法を適用する.
この解法は,ASPシステムが同様の探索失敗を避けるために獲得した学習節を(部分的に)
保持することで,無駄な探索を行うことなく,制約を追加した論理プログラムを連続的に
解くことができる.そのため,求解性能の向上が期待できる.

ASPシステム \clingo のマルチショットASP解法ライブラリを用いたASP符号化を
コード\ref{code:pw-core.lp}に示す.
この符号化は,
\code{base},\code{step(t)},\code{check(t)}の3パートから構成される.
%
初めに\code{base}パートには,ステップ\code{t=0}で満たすべき制約を記述する.
ここでは5行目のルールで,スタート状態とステップ\code{0}の対応を記述している.
%
次に\code{step(t)}パートには,各ステップ\code{t}において満たすべき制約を記述する.
ここでは,有向符号化と電流制約の符号化を拡張したルールを記述している(12--39行目).
さらに,42--44行目に遷移制約を表すルールを記述している.
%
最後に,\code{step(t)}パートでは,プログラムの終了条件を記述する.
ここでは,50行目でゴール状態とステップ\code{t}の対応を記述する.
なお,\code{t}がインクリメントされると,一つ前の不要になった終了条件は,\code{query(t)}
の真偽を動的に操作することにより無効化される.

\textbf{実行実験.}
%
DNETで公開されている実用規模の配電網問題
({\sf fukui-tepco},スイッチ数 468,変電所の数 72,$J^{max}=300$)をベースにした.
この問題の実行可能解から,スタート状態を10個,ゴール状態を100個
をランダムに選び,それらを組み合わせた計1000問の配電網遷移問題を生
成し,ベンチマーク問題とした.
実験に用いたASPシステムと実験環境は~\ref{chap:exp}章で示したものと同じである.

シングルショット符号化(コード~\ref{code:singleshot.lp})と,
マルチショット符号化(コード~\ref{code:pw-core.lp})の実行結果を
表~\ref{table:core}に示す.
左から順に,
最短ステップ長,解けた問題数,各符号化の平均CPU時間(秒),シングルショット符号化と
マルチショット符号化の平均CPU時間の比率を示している.
今回行った実行実験では,全てのベンチマーク問題の到達可能性を判定するこ
とができ,すべて到達可能であった.
また,最大で最短ステップ長が7の問題を解くことができた.
マルチショットASP解法を導入することで,通常の解法と比較して,
平均で3.8倍の高速化を実現することができた.

%%%%%%%%%%%%%%%%%%%%%%%%%%%%%%
\begin{table*}[t]
  \centering
  \caption{配電網遷移問題のASP符号化の実行結果}
  \label{table:core}
  \begin{tabular}{ccrrr}
 \rowcolor[RGB]{0,96,0}
\color{white}最短ステップ長 & \color{white}問題数 
     & \multicolumn{1}{c}{\color{white}シングルショット} 
         & \multicolumn{1}{c}{\color{white}マルチショット} 
             & \multicolumn{1}{c}{\color{white}シングル/マルチ} \\
 \rowcolor[RGB]{230,239,230}
1 & 6 & 1.677 & \alert{1.035} & 1.620 \\
 \rowcolor[RGB]{196,230,196}
2 & 62 & 3.507 & \alert{1.608} & 2.180 \\
 \rowcolor[RGB]{230,239,230}
3 & 189 & 6.089 & \alert{2.155} & 2.826 \\
 \rowcolor[RGB]{196,230,196}
4 & 312 & 9.294 & \alert{2.734} & 3.399 \\
 \rowcolor[RGB]{230,239,230}
5 & 280 & 13.338 & \alert{3.361} & 3.968 \\
 \rowcolor[RGB]{196,230,196}
6 & 130 & 18.303 & \alert{4.165} & 4.394 \\
 \rowcolor[RGB]{230,239,230}
7 & 21 & 24.483 & \alert{5.086} & 4.814 \\
\noalign{\hrule height 0.5pt}
 \rowcolor[RGB]{196,230,196}
計 & 1000 & 76.691 & \alert{20.114} & 3.807 \\
\end{tabular}


  %\begin{tabular}{c|c|c|c}
\noalign{\hrule height 1pt}
問題名 & ステップ数$t$ & 問題数 & 平均CPU時間 \\   
\noalign{\hrule height 1pt}
%& 0 & 2 & 0.016 \\
test & 1 & 12 & 0.019 \\
& 2 & 33 & 0.023 \\
& 3 & 64 & 0.026 \\
& 4 & 33 & 0.030 \\
\noalign{\hrule height 1pt}
%0 & 2 & 0.018 \\
baran32 & 1 & 12 & 0.022 \\
& 2 & 66 & 0.027 \\
& 3 & 68 & 0.032 \\
& 4 & 28 & 0.038 \\
\noalign{\hrule height 1pt}
%0 & 2 & 0.508 \\
fukui-tepco & 1 & 11 & 1.012 \\
& 2 & 28 & 1.603 \\
& 3 & 64 & 2.140 \\
& 4 & 38 & 2.724 \\
& 5 & 11 & 3.361 \\
\noalign{\hrule height 1pt}
\end{tabular}

\end{table*}
%%%%%%%%%%%%%%%%%%%%%%%%%%%%%%

%%% Local Variables:
%%% mode: japanese-latex
%%% TeX-master: "paper"
%%% End:

