%%%% 補助スライド
\appendix
\backupbegin

\begin{frame}{~}
 \centering
 - 補足用 -
\end{frame} 

% %%%%%%%%%%%%%%%%%%%%%%%%%%%%%%%%%%%%%%%%%%%%%%%%%%
% %% 改良符号化 (到達可能性)
% %%%%%%%%%%%%%%%%%%%%%%%%%%%%%%%%%%%%%%%%%%%%%%%%%%
% \begin{frame}[fragile]{改良符号化1: 到達可能性}
% \begin{exampleblock}{}\small
% \begin{lstlisting}
% (1) { inForest(X,Y) } :- edge(X,Y).
% \end{lstlisting}
% \end{exampleblock}
% \begin{itemize}
%  \item (1) 各辺\code{(X,Y)について},根付き全域森に含まれること意味する \\
% 	  アトム\code{inForest(X,Y)}を導入する.
% \end{itemize}
% \begin{exampleblock}{}\small
% \begin{lstlisting}
% (2) reached(R,R) :- root(R).
% (3) reached(X,R) :- reached(Y,R), inForest(Y,X).
% (4) reached(X,R) :- reached(Y,R), inForest(X,Y).
% \end{lstlisting}
% \end{exampleblock}
% \vfill
% \begin{itemize}
% \item アトム\code{reached(X,R)}は,ノード\code{X}が根ノード\code{R}から到達可能であることを意味する.
% %\item (2) 各根ノード\code{R}について,自分自身から到達可能であることを表す.
% \item (3) ノード\code{Y}が根ノード\code{R}から到達可能かつ,辺\code{(Y,X)}が根付き全域森に含まれるならば,
% 	  ノード\code{X}も同じ根ノード\code{R}から到達可能であることを表す.
% \end{itemize}
% \end{frame}
%%%%%%%%%%%%%%%%%%%%%%%%%%%%%%%%%%%%%%%%%%%%%%%%%%
%% 改良符号化 (根付き連結制約)
%%%%%%%%%%%%%%%%%%%%%%%%%%%%%%%%%%%%%%%%%%%%%%%%%%
\begin{frame}[fragile]{根付き連結制約}
\begin{exampleblock}{基本符号化}\small
\begin{lstlisting}
   :- node(X),  not reached(X,_).
   :- reached(X,R1), reached(X,R2), R1 < R2.
\end{lstlisting}
\end{exampleblock}
\begin{exampleblock}{改良符号化1・2}\small
\begin{lstlisting}
   :- node(X), not 1 { reached(X,R) } 1.
\end{lstlisting}
\end{exampleblock}
\vfill
\begin{itemize}
\item 各ノード\code{X}について,ちょうど1つの根からのみ到達可能であることを意味する.
\end{itemize}
\end{frame}
%%%%%%%%%%%%%%%%%%%%%%%%%%%%%%%%%%%%%%%%%%%%%%%%%%
%% ルール数の比較
%%%%%%%%%%%%%%%%%%%%%%%%%%%%%%%%%%%%%%%%%%%%%%%%%%
\begin{frame}{基礎化後のルール数}
  \begin{itemize}
  \item グラフのノード数を$|V|$,根ノードの数を$|R|$とする.
  \end{itemize}
  \begin{table}[t]
    \centering
    %%%%%%%%%%%%%%%%%%%%%%%%%%%%%%%%%%%%%%%%%%%%%%%%%%
%% 提案手法の比較表
%%%%%%%%%%%%%%%%%%%%%%%%%%%%%%%%%%%%%%%%%%%%%%%%%%
\begin{tabular}[tb]{cp{6cm}p{2.5cm}}

 \rowcolor[RGB]{60,60,134}
 \textcolor{white}{符号化} &
	 \multicolumn{1}{c}{\textcolor{white}{根付き連結制約の表現方法}} & 
		 \begin{tabular}{c} 
		  \textcolor{white}{~基礎化後の}\\\textcolor{white}{~ルール数} 
		 \end{tabular} \\
 \rowcolor[RGB]{230,230,249}
   \begin{tabular}{c}基本\\符号化\end{tabular} & 
	 \begin{tabular}{l} at-least-one制約と\\
	 at-most-one制約を用いて表現 \end{tabular}& 
	 \vspace{-0.4cm}\alert{$|V|\left(1+ {|R|\choose 2} \right)$} \\
 \rowcolor[RGB]{210,210,255}
  \begin{tabular}{c}改良\\符号化1$\cdot$2 \end{tabular} & 
	  \begin{tabular}{l} ASPの個数制約を用いて表現 \end{tabular} & 
		 \multicolumn{1}{c}{~\alert{$|V|$}} 
\end{tabular}
 
  \end{table}
\end{frame}
%%%%%%%%%%%%%%%%%%%%%%%%%%%%%%%%%%%%%%%%%%%%%%%%%%
%% 改良符号化 (非閉路制約)
%%%%%%%%%%%%%%%%%%%%%%%%%%%%%%%%%%%%%%%%%%%%%%%%%%
\begin{frame}[fragile]{非閉路制約}
\begin{minipage}[c]{1.01\textwidth}
\begin{exampleblock}{基本符号化・改良符号化1 (辺数の制約)}\small
\begin{lstlisting}
    :- root(R),
       not 1 #sum{ 1,X:reached(X,R) ;
                  -1,X,Y:inForest(X,Y),reached(X,R),reached(Y,R)
                 } 1.
\end{lstlisting}
\end{exampleblock}
\begin{exampleblock}{改良符号化2 (ノードの入次数の制約)}\small
\begin{lstlisting}
    :- root(R), inForest(_,R).
    :- node(X), not root(X), not 1 { inForest(_,X) } 1.
\end{lstlisting}
\end{exampleblock}
\end{minipage}
\vfill
\begin{itemize}
\item 各連結成分が\structure{\bf 木の性質}を満たすことにより,サイクルを持たない
	  ことを保証する.
\end{itemize}
\end{frame}

%%%%%%%%%%%%%%%%%%%%%%%%%%%%%%%%%%%%%%%%%%%%%%%%%%
%% ASPのコード
%%%%%%%%%%%%%%%%%%%%%%%%%%%%%%%%%%%%%%%%%%%%%%%%%%
\begin{frame}[fragile]{補足 : 根付き全域森 改良符号化2}
%%%%%%%%%%%%%%%%%%%%%%%%%%%%%%%%% 
\lstinputlisting[frame=single,label=code:roop,%
xleftmargin=1zw,%
xrightmargin=1zw,%
numbersep=5pt,%
numbers=left,%
breaklines=true,%
columns=fullflexible,keepspaces=true,%
basicstyle=\ttfamily\scriptsize]{code/dnet-directed.lp}
%%%%%%%%%%%%%%%%%%%%%%%%%%%%%%%%%
\end{frame}

% \begin{frame}[fragile]{補足 : 遷移問題 シングルショット符号化}

% \begin{multicols*}{2}
% %%%%%%%%%%%%%%%%%%%%%%%%%%%%%%%%% 
% \lstinputlisting[frame=single,label=code:roop,%
% xleftmargin=1zw,%
% xrightmargin=1zw,%
% numbersep=5pt,%
% numbers=left,%
% breaklines=true,%
% columns=fullflexible,keepspaces=true,%
% basicstyle=\ttfamily\tiny]{code/trans-const.lp}
% %%%%%%%%%%%%%%%%%%%%%%%%%%%%%%%%%
% \end{multicols*}
% \end{frame}

% \begin{frame}[fragile]{補足 : 遷移問題 マルチショット符号化}
% \begin{multicols*}{2}
% %%%%%%%%%%%%%%%%%%%%%%%%%%%%%%%%%
% \lstinputlisting[frame=single,label=code:incmode,% 
% xleftmargin=1zw,%
% xrightmargin=1zw,%
% numbersep=5pt,%
% numbers=left,%
% breaklines=true,%
% columns=fullflexible,keepspaces=true,%
% basicstyle=\ttfamily\tiny]{code/dnet-trans.lp}
% %%%%%%%%%%%%%%%%%%%%%%%%%%%%%%%%% 
% \end{multicols*}
% \end{frame}

\backupend
