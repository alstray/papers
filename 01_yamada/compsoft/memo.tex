# コンピュータソフトウェア投稿メモ

## 論文種別
研究論文

## 査読種別
通常論文

## 募集論文
一般論文

## 会員の別
会員投稿

## 題目
解集合プログラミングを用いた配電網問題の解法
Solving Power Distribution Network Problems with Answer Set Programming

## ページ数
13

## 概要

配電網の効率的な構成および制御はスマートグリッドや災害時の停電復旧など
を支える重要な研究課題である.
配電網問題は,供給経路に関するトポロジ制約と,電流・電圧に関する電気制
約を満たす配電網の構成(スイッチの開閉状態)を求める問題である.
配電網遷移問題は,配電網問題とその2つの実行可能解が与えられたとき,一
方の解から他方の解へ,遷移制約を満たしつつ実行可能解のみを経由して到達
できるかを判定する問題である.
本論文では,解集合プログラミング(ASP)を用いた配電網問題の解法および配
電網遷移問題への拡張について述べる.提案解法では,まず与えられた問題イ
ンスタンスを ASP のファクト形式に変換した後,そのファクトと配電網(遷移)問
題を解くための ASP 符号化を結合した上で,高速 ASP システムを用いて解を
求める.提案解法の評価として,実用規模の問題を含む問題集を用いた実験結
果について述べる.

Power Distribution Network Problem (PDNP) can be generally 
defined as determining the configuration of power distribution network,
Power Distribution Network Reconfiguration Problem (PDNRP) can be
defined as, for a given PDNP instance and two feasible configurations,
determining the reachability from one configuration to another one
while satisfying the transition constraints.
This paper focuses on searching the shortest path of PDNRPs.

In this paper, we propose an approach to solving PDNPs and PDNRPs
based on Answer Set Programming (ASP).
In our approach, at first, a problem instance is converted into 
a set of ASP facts. 
Then, the facts combined with ASP encoding
for PDNP/PDNRP solving can be solved by using ASP systems,
in our case clingo.

To evaluate the effectiveness of our approach, we conduct experiments
using 1000 benchmark instances based on
fukui-tepco, which is a practical PDNP instance of DNET.

## 著者全員の情報

### 氏名
山田 健太郎
### ふりがな
やまだ けんたろう
### 所属
名古屋大学 大学院情報学研究科
### 電話番号
090-8561-5908
### 電子メールアドレス
yken66@nagoya-u.jp
### 会員番号
なし

### 氏名
湊 真一
### ふりがな
みなと しんいち
### 所属
京都大学 大学院情報学研究科
### 電話番号
???
### 電子メールアドレス
minato@i.kyoto-u.ac.jp
### 会員番号
???

### 氏名
田村 直之
### ふりがな
たむら なおゆき
### 所属
神戸大学 情報基盤センター
### 電話番号
???
### 電子メールアドレス
tamura@kobe-u.ac.jp
### 会員番号
???

### 氏名
番原 睦則
### ふりがな
ばんばら むつのり
### 所属
名古屋大学 大学院情報学研究科
### 電話番号
???
### 電子メールアドレス
banbara@nagoya-u.jp
### 会員番号
???

### 確認事項
本論文と内容的に重なりがある論文で,査読中のものはない.

### 投稿年月日
2022年 X月 X日
