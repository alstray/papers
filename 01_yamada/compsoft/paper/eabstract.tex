Power Distribution Network Problem (PDNP) can be generally 
defined as determining the configuration of power distribution network,
Power Distribution Network Reconfiguration Problem (PDNRP) can be
defined as, for a given PDNP instance and two feasible configurations,
determining the reachability from one configuration to another one
while satisfying the transition constraints.
This paper focuses on searching the shortest path of PDNRPs.
%
In this paper, we propose an approach to solving PDNPs and PDNRPs
based on Answer Set Programming (ASP).
In our approach, at first, a problem instance is converted into 
a set of ASP facts. 
Then, the facts combined with ASP encoding
for PDNP/PDNRP solving can be solved by using ASP systems,
in our case \textit{clingo}.
%
To evaluate the effectiveness of our approach, we conduct experiments
using 1000 benchmark instances based on
\textsf{fukui-tepco}, which is a practical PDNP instance of DNET.

%%% Local Variables:
%%% mode: japanese-latex
%%% TeX-master: "paper"
%%% End:
