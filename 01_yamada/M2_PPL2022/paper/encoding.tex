\section{配電網問題のASP符号化}\label{chap:encode}

第~\ref{chap:intro}節の図~\ref{fig:arch}に示したように,
提案解法では,まず与えられた問題インスタンスを ASP のファクト形式に変
換した後,そのファクトと配電網問題を解くための ASP 符号化を結合した上
で,高速 ASP システム{\clingo}を用いて解を求める.
%
本節では,まずxxで配電網問題インスタンスのファクト形式について述べる.
xx で前処理として補助アトムを導入したのち,xxx でトポロジ制約のASP符号
化として,基本符号化,改良符号化,有向符号化の3つを提案する.
最後に,xxx で電流制約のASP符号化を示す.

%%%%%%%%%%%%%%%%%%%%%%%%%%%%%%%%%%%%%%%%%%%%%%%%%%%%%%%%%%%%%%%%%
\subsection{配電網問題インスタンスのファクト形式}

%%%%%%%%%%%%%%%%%%%%%%%%%%%%%%%%%
\lstinputlisting[float=tb,caption={%
  配電網問題(図~\ref{fig:test-input})のファクト表現},%
captionpos=b,frame=single,label=code:test.lp,%
xrightmargin=1zw,% 
xleftmargin=1zw,% 
numbersep=5pt,%
numbers=none,%
breaklines=true,%
columns=fullflexible,keepspaces=true,%
basicstyle=\ttfamily\scriptsize]{code/test.lp}
%%%%%%%%%%%%%%%%%%%%%%%%%%%%%%%%%

図~\ref{fig:test-input}で示した
配電網問題の例の ASP ファクト表現をコード\ref{code:test.lp}に示す.
このファクト表現は
DNETに公開されているYAML形式の問題インスタンスを,ほぼ一対一に ASP の
ファクトに変換したものである.
%
\code{dnet_node}アトムは,図~\ref{fig:test-input}の四角ノード($\Box$)
に対応し,セクションとスイッチの結合関係を表している.
例えば,1行目の
\code{dnet_node(1, ("s_a1"; "s_1"; "s_15"))}は,
赤色の四角ノードに対応し,
3つのセクション
\code{s_a}, \code{s_1}, \code{s_15}
が結合していることを表す.
アトム中のセミコロン(\code{;})は ASP の略記法であり,実行時には
\code{dnet_node(1, "s_a1")}
\code{dnet_node(1, "s_1")}
\code{dnet_node(1, "s_15")}
の3つのアトムに展開される.
%
4行目の\code{dnet_node(4, ("s_1"; "sw_1"))}は,
セクション\code{s_1}とスイッチ\code{sw_1}
が結合していることを表す.

%%%%%%%%%%%%%%%%%%%%%%%%%%%%%%%%%
\subsection{前処理:補助アトムの導入}
配電網問題を解くための前処理となるASP符号化をコード\ref{code:prepro.lp}に示す.
1行目のルールは,スイッチを含むノード(スイッチノード)番号を表すアトム\code{swt_node(X)}
を導入する.この\code{swt_node(X)}は,ノード番号\code{X}はスイッチを含むことを意味する.
反対に,2行目のルールは,スイッチを含まないノード(ジャンクションノード)番号を表すアトム
\code{jct_node(X)}を導入する.この\code{jct_node(X)}は,ノード\code{X}には,
スイッチが含まれないことを意味する.

4行目のルールは,各区間がどのノード番号に属するかを表す
アトム\code{section(S,X)}を導入する.
また,5行目で区間を意味するアトム\code{section(S)}を導入する.

7,8行目のルールでは,各区間について,スイッチノードまたはジャンクションノードに
含まれていることを表すアトム\code{swt_node(S)},\code{jct_node(S)}を導入する.

10行目のルールで,各スイッチがどの2つの区間を接続されているかを表すアトム
\code{switch(SW,S,T)}を導入する.このアトム\code{switch(SW,S,T)}は,
区間\code{(S,T)}はスイッチ\code{SW}で接続されていることを意味する.

14行目からのルールは根付き全域森問題の入力を生成するルールである.
14,15行目のルールは,根付き全域森問題のノードと区間を対応させるアトム\code{node/2}を導入する.
アトム\code{node(X,S)}は,各区間\code{S}は根付き全域森のノード\code{X}に含まれることを意味する.
14行目のルールで,ジャンクションノードはそのまま根付き全域森のノードとして定義され,ジャンクション
ノードに属する区間は,そのまま対応づけられる.
15行目のルールでは,スイッチノードに含まれ,ジャンクションノードに含まれない各区間について,
属するノード番号のうち,小さい方の番号を根付き全域森問題のノードとして定義する.

18行目は,根ノードを表すアトム\code{root(X)}を導入する.この\code{root(X)}は,\code{node(X,S)}
のうち,区間\code{S}が変電所と直接つながっているならば,根ノードであることを意味する.また,
19行目で各\code{node/2}を,根付き全域森問題の入力のノードを表すアトム\code{node/1}とする.
20行目で,辺を表すアトム\code{edge(X,Y)}を導入する.この\code{edge(X,Y)}は任意の2つの
ノード\code{(X,Y)}が辺で結ばれていることを意味する.


%%%%%%%%%%%%%%%%%%%%%%%%%%%%%%%%%
\lstinputlisting[float=tb,caption={%
  根付き全域森問題の基本符号化},%
captionpos=b,frame=single,label=code:srf1.lp,%
xrightmargin=1zw,% 
xleftmargin=1zw,% 
numbersep=5pt,%
numbers=left,%
breaklines=true,%
columns=fullflexible,keepspaces=true,%
basicstyle=\ttfamily\scriptsize]{code/srf1.lp}
%%%%%%%%%%%%%%%%%%%%%%%%%%%%%%%%%
\lstinputlisting[float=tb,caption={%
  根付き全域森問題の改良符号化},%
captionpos=b,frame=single,label=code:srf2.lp,%
xrightmargin=1zw,% 
xleftmargin=1zw,% 
numbersep=5pt,%
numbers=left,%
breaklines=true,%
columns=fullflexible,keepspaces=true,%
basicstyle=\ttfamily\scriptsize]{code/srf2.lp}
%%%%%%%%%%%%%%%%%%%%%%%%%%%%%%%%%
\lstinputlisting[float=tb,caption={%
  根付き全域森問題の発展符号化},%
captionpos=b,frame=single,label=code:srf3.lp,%
xrightmargin=1zw,% 
xleftmargin=1zw,% 
numbersep=5pt,%
numbers=left,%
breaklines=true,%
columns=fullflexible,keepspaces=true,%
basicstyle=\ttfamily\scriptsize]{code/srf3.lp}
%%%%%%%%%%%%%%%%%%%%%%%%%%%%%%%%%

\subsection{トポロジ制約のASP符号化}
\textbf{基本符号化.}
根付き全域森問題のASP符号化をコード\ref{code:srf1.lp}に示す.
2行目のルールは,各辺(\code{X},\code{Y})に対して,
解の候補となるアトム\code{inForest(X,Y)}を導入する.
この\code{inForest(X,Y)}は,
辺(\code{X},\code{Y})が根付き全域森に含まれることを意味する.
%
各ノードの到達可能性は5~7行目のルールで表される.
アトム\code{reached(X,R)}は,ノード\code{X}は根
\code{R}から到達可能であることを意味する.
5行目のルールは,各根ノードは自分自身から到達可能であることを表す.
6~7行目のルールは,ノード\code{Y}が根ノード\code{R}から到達可能であり,
かつ,辺(\code{X},\code{Y})が全域森に含まれるならば,
ノード\code{X}も\code{R}から到達可能であることを表す.

非閉路制約は10~13行目のルールで表される.
このルールは,各連結成分のノード数と辺数の差が1になること(木の性質)を,
ASP の重み付き個数制約を使って表している.
%
根付き連結制約は16~17行目のルールで表される.
16行目のルールは,各ノードは少なくとも1つの根から到達可能であることを
表している(\textit{at-least-one}制約).
17行目のルールは,各ノードは高々1つの根から到達可能である
ことを表している(\textit{at-most-one}制約).
これら2つの一貫性制約により,各ノードはちょうど1つの根から到達可能であ
ることが強制される.

%%%%%%%%%%%%%%%%%%%%%%%%%%%%%%%%%
\textbf{改良符号化.}
基本符号化は,根付き全域森問題の制約をASPのルール7個で簡潔に表現できる.
しかし,根付き連結制約を表す\textit{at-most-one}制約の基礎化後のルール
数は,根ノード数の2乗に比例するため,大規模な問題に対する求解性能が低
下する可能性がある.
%
この問題を解決するために考案した改良符号化をコード\ref{code:srf2.lp}に
示す.基本符号化との違いは,根付き連結制約をASPの個数制約で表している
点である(16行目).
グラフのノード数を$n$,根ノード数を$r$として,
根付き連結制約の基礎化後のルール数を比較すると,
基本符号化が$n(1+{}_{r}C_{2})$個なのに対し,
改良符号化は$n$個と少なく抑えることができる.
これにより,大規模な問題に対する有効性が期待できる.
 
%%%%%%%%%%%%%%%%%%%%%%%%%%%%%%%%%
\textbf{発展符号化.}
非閉路制約を別の方法で符号化するために,根付き全域森問題を有向グラフに拡張し,
各ノードの入次数に関する制約を用いて符号化を行った.考案した発展符号化を
コード\ref{code:srf3.lp}に示す.
9行目のルールでは,各根ノードについて,入次数は0であることを強制している.
10行目のルールでは,根ノード以外の各ノードについて,入次数が1であることを保証している.
その他のルールは改良符号化と同じである.

%%%%%%%%%%%%%%%%%%%%%%%%%%%%%%%%%
\subsection{電流制約のASP符号化}

%%%%%%%%%%%%%%%%%%%%%%%%%%%%%%%%%
\lstinputlisting[float=tb,caption={%
  電流制約の符号化},%
captionpos=b,frame=single,label=code:electrical.lp,%
xrightmargin=1zw,% 
xleftmargin=1zw,% 
numbersep=5pt,%
numbers=left,%
breaklines=true,%
columns=fullflexible,keepspaces=true,%
basicstyle=\ttfamily\scriptsize]{code/electrical.lp}
%%%%%%%%%%%%%%%%%%%%%%%%%%%%%%%%%

電流制約のASP符号化をコード\ref{code:electrical.lp}に示す.
3行目のルールで,根付き全域森問題の解となるアトム\code{inForest(X,Y)}をもとに,
配電網において,スイッチが閉じて区間が接続されていることを表すアトム\code{connected(SW,S,T)}を
導入する.このアトムは,区間\code{S}からスイッチ\code{SW}によって,
もう1つの区間\code{T}に接続していることを表す.
6行目のルールで,配電網問題の解を表すアトム\code{closed_switch(SW)}を導入する.
この\code{closed_switch(SW)}は,スイッチ\code{SW}が閉じたスイッチであることを意味する.

ジャンクションノード内での上下流の関係は,8,9行目のルールで表される.
アトム\code{entrance_section(S,X)}は,区間\code{S}がノード\code{X}内で
最も上流にあることを意味する.8行目のルールで変電所と直接つながっている区間が上流にあることを
表している.9行目のルールで,ジャンクションノード内の区間が,スイッチによって外部のノードから
接続されているならば上流であることを意味している.

また供給経路内の上下関係は,アトム\code{downstream(S,T)}で
表される.この\code{downstream(S,T)}は,区間\code{S}の下流に区間\code{T}が接続されている
ことを意味する.12行目のルールで,接続されている2つの区間\code{(S,T)}がそのまま上下関係である
ことを意味している.13行目のルールでは,\code{entrance_section(S,X)}が属するノード\code{X}内
にある他の区間\code{T}が下流にあることを意味している.

15,16行目は,変電所と区間の供給関係を表している.アトム\code{suppliable(T,R)}は,
区間\code{T}が変電所\code{R}から供給を受けていることを意味する.
15行目のルールでは,各変電所は自身から供給を受けることを表す.16行目のルールでは,
区間\code{S}が変電所\code{R}から供給され,かつ,区間\code{T}が\code{S}の下流にある
ならば,\code{T}も\code{R}から供給されることを表す.

18行目のルールで,各変電所\code{R}について,供給する各区間\code{S}の負荷\code{I}の総和が
入力として与えられる\code{max_current}以下でなければならないことを表す.

%%% Local Variables:
%%% mode: japanese-latex
%%% TeX-master: "paper"
%%% End:
