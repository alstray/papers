\section{配電網問題のASP符号化}\label{chap:encode}

第~\ref{chap:intro}節の図~\ref{fig:arch}に示したように,
提案解法では,まず与えられた問題インスタンスを ASP のファクト形式に変
換した後,そのファクトと配電網問題を解くための ASP 符号化を結合した上
で,高速 ASP システム{\clingo}を用いて解を求める.
%
本節では,まず\ref{chap:fact}節で配電網問題インスタンスのファクト形式
について述べる.
\ref{chap:prepro}節で前処理として補助アトムを導入したのち,
\ref{chap:topology}節でトポロジ制約のASP符号化として,
基本符号化,改良符号化,有向符号化の3つを提案する.
最後に,\ref{chap:electrical}節で電流制約のASP符号化を示す.

%%%%%%%%%%%%%%%%%%%%%%%%%%%%%%%%%%%%%%%%%%%%%%%%%%%%%%%%%%%%%%%%%
\subsection{配電網問題インスタンスのファクト形式}\label{chap:fact}

%%%%%%%%%%%%%%%%%%%%%%%%%%%%%%%%%
\lstinputlisting[float=tb,caption={%
  配電網問題(図~\ref{fig:test-input})のファクト表現},%
captionpos=b,frame=single,label=code:test.lp,%
xrightmargin=1zw,% 
xleftmargin=1zw,% 
numbersep=5pt,%
numbers=none,%
breaklines=true,%
columns=fullflexible,keepspaces=true,%
basicstyle=\ttfamily\scriptsize]{code/test.lp}
%%%%%%%%%%%%%%%%%%%%%%%%%%%%%%%%%

図~\ref{fig:test-input}で示した
配電網問題の例の ASP ファクト表現をコード\ref{code:test.lp}に示す.
このファクト表現は
DNETに公開されているYAML形式の問題インスタンスを,ほぼ一対一に ASP の
ファクトに変換したものである.
%
アトム\code{dnet_node}は,図~\ref{fig:test-input}の四角ノード($\Box$)
に対応し,セクションとスイッチの結合関係を表している.
例えば,1行目の
\code{dnet_node(1, ("s_a"; "s_1"; "s_15"))}は,
赤色の四角ノードに対応し,
3つのセクション
\code{s_a}, \code{s_1}, \code{s_15}
が結合していることを表す.
アトム中のセミコロン(\code{;})は ASP の略記法であり,実行時には
\code{dnet_node(1, "s_a")},
\code{dnet_node(1, "s_1")},
\code{dnet_node(1, "s_15")}
の3つのアトムに展開される.
%
4行目の\code{dnet_node(4, ("s_1"; "sw_1"))}は,
セクション\code{s_1}とスイッチ\code{sw_1}
が結合していることを表す.

アトム\code{load/2}は,各セクションに与えられる電流の値を表す.例えば,
\code{load("s_a", 16)}は,セクション\code{s_a}は$16$Aの電流をもつことを意味する.
%
アトム\code{substation/1}は変電所を,アトム\code{switch/1}はスイッチをそれぞれ
表している.


%%%%%%%%%%%%%%%%%%%%%%%%%%%%%%%%%
\subsection{前処理:補助アトムの導入}\label{chap:prepro}

%%%%%%%%%%%%%%%%%%%%%%%%%%%%%%%%%
\lstinputlisting[float=tb,caption={%
  補助アトムの導入},%
captionpos=b,frame=single,label=code:prepro.lp,%
xrightmargin=1zw,% 
xleftmargin=1zw,% 
numbersep=5pt,%
numbers=left,%
breaklines=true,%
columns=fullflexible,keepspaces=true,%
basicstyle=\ttfamily\scriptsize]{code/prepro.lp}
%%%%%%%%%%%%%%%%%%%%%%%%%%%%%%%%%

配電網問題のASP符号化を簡潔に記述するために,前処理として
補助アトムを導入する.用いるASP符号化をコード\ref{code:prepro.lp}に示す.

1行目のルールは,あるノード\code{X}に対して,スイッチノードであることを表す
アトム\code{swt_node(X)}を導入する.この\code{swt_node(X)}は,
ノード\code{X}で,セクションとスイッチが結合していることを意味する.
%
反対に,2行目のルールは,あるノード\code{X}がジャンクションノードであること
表すアトム\code{jct_node(X)}を導入する.この\code{jct_node(X)}は,
ノード\code{X}で,スイッチと結合していないことを意味する.

4行目のルールは,セクション\code{S}がノード\code{X}に出現することを意味する
アトム\code{section(S,X)}を導入する.
また,5行目でセクションを表すアトム\code{section/1}を導入する.
%
7,8行目のルールでは,各セクション\code{S}に対して,スイッチノード
またはジャンクションノードに属することを意味するアトム
\code{swt_node(S)},\code{jct_node(S)}を導入する.
%
10行目のルールで,任意の2つのセクション\code{(S,T)}に対して,
あるスイッチ\code{SW}が.それらと結合していることを表すアトム
\code{switch(SW,S,T)}を導入する.


\vskip 3em


14行目からのルールはトポロジ制約(根付き全域森問題)のASP符号化の入力
となるアトムを生成するためのルールである.
14,15行目のルールは,根付き全域森問題のノードとセクションの
対応関係を表すアトム\code{node/2}を導入する.
アトム\code{node(X,S)}は,セクション\code{S}は根付き全域森問題の
ノード\code{X}に対応することを意味する.
14行目のルールは,各ジャンクションノード\code{X}について,それに
含まれるセクション\code{S}はそのまま対応することを表す.
15行目のルールでは,スイッチノードにのみ含まれるセクション\code{S}
について,ノード番号の小さいものに対応することを意味する.

18行目からのアトム\code{root/1}は根ノードを,アトム\code{node/1}はノードを,
アトム\code{edge/2}は辺をそれぞれ表している.
18行目のルールは,各ノード\code{X}について,それに属する\code{S}が
変電所であるならば,\code{X}が根ノードであることを意味する.
%
19行目のルールは,\code{node/2}から\code{node/1}が成り立つ.
%
20行目のルールは,ノード\code{X}に含まれるセクション\code{S}と,
ノード\code{Y}に含まれるセクション\code{T}が,あるスイッチ\code{SW}
で結合するならば,それら2つのノードは辺で結ばれることを意味する.


\subsection{トポロジ制約のASP符号化}\label{chap:topology}

%%%%%%%%%%%%%%%%%%%%%%%%%%%%%%%%%
\lstinputlisting[float=tb,caption={%
  基本符号化},%
captionpos=b,frame=single,label=code:srf1.lp,%
xrightmargin=1zw,% 
xleftmargin=1zw,% 
numbersep=5pt,%
numbers=left,%
breaklines=true,%
columns=fullflexible,keepspaces=true,%
basicstyle=\ttfamily\scriptsize]{code/srf1.lp}
%%%%%%%%%%%%%%%%%%%%%%%%%%%%%%%%%
\lstinputlisting[float=tb,caption={%
  改良符号化},%
captionpos=b,frame=single,label=code:srf2.lp,%
xrightmargin=1zw,% 
xleftmargin=1zw,% 
numbersep=5pt,%
numbers=left,%
breaklines=true,%
columns=fullflexible,keepspaces=true,%
basicstyle=\ttfamily\scriptsize]{code/srf2.lp}
%%%%%%%%%%%%%%%%%%%%%%%%%%%%%%%%%
\lstinputlisting[float=tb,caption={%
  有向符号化},%
captionpos=b,frame=single,label=code:srf3.lp,%
xrightmargin=1zw,% 
xleftmargin=1zw,% 
numbersep=5pt,%
numbers=left,%
breaklines=true,%
columns=fullflexible,keepspaces=true,%
basicstyle=\ttfamily\scriptsize]{code/srf3.lp}
%%%%%%%%%%%%%%%%%%%%%%%%%%%%%%%%%

トポロジ制約のASP符号化は第~\ref{chap:problem}節で示したように,
根付き全域森の制約を表すASP符号化として表現することが出来る.

\textbf{基本符号化.}
基本符号化をコード\ref{code:srf1.lp}に示す.
2行目のルールは,各辺(\code{X},\code{Y})に対して,
解の候補となるアトム\code{inForest(X,Y)}を導入する.
この\code{inForest(X,Y)}は,
辺(\code{X},\code{Y})が根付き全域森に含まれることを意味する.
%
各ノードの到達可能性は5~7行目のルールで表される.
アトム\code{reached(X,R)}は,ノード\code{X}は根
\code{R}から到達可能であることを意味する.
5行目のルールは,各根ノードは自分自身から到達可能であることを表す.
6~7行目のルールは,ノード\code{Y}が根ノード\code{R}から到達可能であり,
かつ,辺(\code{X},\code{Y})が根付き全域森に含まれるならば,
ノード\code{X}も\code{R}から到達可能であることを表す.

非閉路制約は10~13行目のルールで表される.
このルールは,各連結成分のノード数と辺数の差が1になること(木の性質)を,
ASP の重み付き個数制約を使って表している.
%
根付き連結制約は16~17行目のルールで表される.
16行目のルールは,各ノードは少なくとも1つの根から到達可能であることを
表している(\textit{at-least-one}制約).
17行目のルールは,各ノードは高々1つの根から到達可能である
ことを表している(\textit{at-most-one}制約).
これら2つの一貫性制約により,各ノードはちょうど1つの根から到達可能であ
ることが強制される.

%%%%%%%%%%%%%%%%%%%%%%%%%%%%%%%%%
\textbf{改良符号化.}
% 基本符号化は,根付き全域森問題の制約をASPのルール7個で簡潔に表現できる.
% しかし,根付き連結制約を表す\textit{at-most-one}制約の基礎化後のルール
% 数は,根ノード数の2乗に比例するため,大規模な問題に対する求解性能が低
% 下する可能性がある.
%
%この問題を解決するために考案した
改良符号化をコード\ref{code:srf2.lp}に示す.
基本符号化との違いは,根付き連結制約をASPの個数制約で表している点である(16行目).
グラフのノード数を$n$,根ノード数を$r$として,
根付き連結制約の基礎化後のルール数を比較すると,
基本符号化が$n(1+{}_{r}C_{2})$個なのに対し,
改良符号化は$n$個と少なく抑えることができる.
% これにより,大規模な問題に対する有効性が期待できる.
 
%%%%%%%%%%%%%%%%%%%%%%%%%%%%%%%%%
\textbf{有向符号化.}
非閉路制約を別の方法で符号化するために,与えられるグラフを有向グラフに拡張し,
各ノードの入次数に関する制約を用いて符号化を行った.
有向符号化をコード\ref{code:srf3.lp}に示す.
2行目で,解の候補となるアトムを導入するルールを有向辺に拡張する.
9行目のルールは,各根ノードについて,入次数は0であることを強制している.
10行目のルールは,根ノード以外の各ノードについて,入次数が1であることを保証している.
そのほかのルールは改良符号化と同じである.

%%%%%%%%%%%%%%%%%%%%%%%%%%%%%%%%%
\subsection{電流制約のASP符号化}\label{chap:electrical}

%%%%%%%%%%%%%%%%%%%%%%%%%%%%%%%%%
\lstinputlisting[float=tb,caption={%
  電流制約のASP符号化},%
captionpos=b,frame=single,label=code:electrical.lp,%
xrightmargin=1zw,% 
xleftmargin=1zw,% 
numbersep=5pt,%
numbers=left,%
breaklines=true,%
columns=fullflexible,keepspaces=true,%
basicstyle=\ttfamily\scriptsize]{code/electrical.lp}
%%%%%%%%%%%%%%%%%%%%%%%%%%%%%%%%%

電流制約のASP符号化をコード\ref{code:electrical.lp}に示す.
この符号化は,第~\ref{chap:exp}節で示すトポロジ制約のASP符号化の
実行実験で最も良い性能を示した有向符号化に基づき考案したASP符号化である.
%
1,4行目のルールは,配電網問題の実行可能解である閉じたスイッチを表すための
ルールである.閉じたスイッチの集合$X$は,アトム\code{closed_switch/1}で表される.
1行目のルールで,
トポロジ制約の解となるアトム\code{inForest(X,Y)}が成り立つとき,
ノード\code{X}に含まれるセクション\code{S}と,
ノード\code{Y}に含まれるセクション\code{T}を結合するスイッチ\code{SW}に
よって,それら2つのセクションが接続されることを意味するアトム
\code{connected(SW,S,T)}を導入する.
4行目のルールで,\code{connected/3}が成り立つ任意のスイッチ\code{SW}が
閉じていることを表すアトム\code{closed_switch(SW)}を導入する.

ジャンクションノード内のセクションの上下関係を表すために
補助アトム\code{entrance_section(S,X)}を導入する.
この\code{entrance_section(S,X)}は,セクション\code{S}が
ジャンクションノード\code{X}内で最も上流にあることを意味する.
6行目のルールで,各変電所\code{S}がノード\code{X}内で最も上流にあることを
表す.
7行目のルールで,各ジャンクションノード\code{X}に属するセクション\code{S}について,
\code{S}が他のノードに属するセクションから接続されるならば,\code{S}が最も上流であることを
表す.

任意の2つのセクションの上下関係は,アトム\code{downstream(S,T)}で表される.
この\code{downstream(S,T)}は,セクション\code{S}の次にセクション\code{T}
を経由することを意味する.
10行目のルールは,任意の2つのセクション\code{(S,T)}が,この順序で接続されている
ならば,\code{downstream(S,T)}が成り立つことを意味する.
13行目のルールは,ノード\code{X}内で最も上流にあるセクション\code{S}と,
同じノード\code{X}に属する他のセクション\code{T}について,
\code{downstream(S,T)}が成り立つことを意味する.

15,16行目は,各セクションについてどの変電所から供給を受けるかを
表すルールである.アトム\code{suppliable(S,R)}は,セクション\code{S}が
変電所\code{R}から供給を受けることを意味する補助アトムである.
15行目のルールで,各変電所は自身から供給を受けることを表す.
16行目のルールでは,
セクション\code{T}が変電所\code{R}から供給を受け,かつ,
\code{T}の次にセクション\code{S}を経由するならば,
\code{S}も\code{R}から供給を受けることを表す.

18行目のルールは,電流制約に対応するルールである.ルール中の\code{max_current}は
入力である$J_i^{max}$に対応する.
各変電所\code{R}に対して,\code{R}が供給する各セクション\code{S}の
電流\code{I}の総和が\code{max_current}以下であることを保証する.


%%% Local Variables:
%%% mode: japanese-latex
%%% TeX-master: "paper"
%%% End:
