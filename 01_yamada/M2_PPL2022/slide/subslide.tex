%%%% 補助スライド
\appendix
\backupbegin

\begin{frame}{~}
 \centering
 - 補足用 -
\end{frame}
%%%%%%%%%%%%%%%%%%%%%%%%%%%%%%%%%%%%%%%%%%%%%%%%%%
%% 解法の流れ
%%%%%%%%%%%%%%%%%%%%%%%%%%%%%%%%%%%%%%%%%%%%%%%%%%
\begin{frame}{ASPを用いた配電網遷移問題の解法の流れ}
 \centering\vfill
 \scalebox{0.6}{\hspace{-1zw}%%%%%%%%%%%%%%%%%%%%%%%%%%%%%%%%%%%%%%%%%%%%%%%%%%
%% ASPで問題を解く流れの図
%%%%%%%%%%%%%%%%%%%%%%%%%%%%%%%%%%%%%%%%%%%%%%%%%%
\begin{tikzpicture}

 \def\nodehspace{1cm}
 \def\nodevspace{0.5cm}

 \tikzset{block/.style={rectangle, very thick, text centered, draw=black, fill=cyan!10,
 text width=3cm, minimum height=1.5cm}};

 \tikzset{alertblock/.style={rectangle, ultra thick, draw=red, fill=red!10,
 text centered, text width=3cm, minimum height=1.5cm}};

 \tikzset{structureblock/.style={rectangle, very thick, draw=blue!60, fill=white,
 text centered, text width=3cm, minimum height=1.5cm}};

 \node[block](ins){配電網遷移問題\\インスタンス};
 \node[block, densely dotted, right=0.6\nodehspace of ins] (conv){変換器};
 \node[block, right=0.6\nodehspace of conv] (fact){ASPファクト};
 \node[alertblock, below=\nodevspace of fact](encode){\alert{\bf ASP符号化\\(論理プログラム)}};
 \node[structureblock, right=1.3\nodehspace of fact](sys1){ASPシステム};
 \node[block, right=\nodehspace of sys1] (ans){配電網遷移問題\\の解};

 \node[rectangle, draw=blue!60, ultra thick, minimum width=3.6cm, minimum height=4cm, fill=blue!10,
 below=-1.8cm of sys1, label=above:\structure{\bf \Large 提案ソルバー}](solver){};

 \node[structureblock, right=1.3\nodehspace of fact](sys){ASPシステム};
 \node[structureblock, below=\nodevspace of sys](recon){\structure{\bf 配電網遷移問題\\アルゴリズム}};

 \foreach \u / \v / \name in {ins/conv/,conv/fact/,fact/sys/,sys/ans/}
 \draw [very thick,->] (\u) to node[above]{\name} (\v);
 
 \draw [very thick,->] (sys.225) -- (recon.135);
 \draw [very thick,->] (recon.45) -- (sys.315);

 \coordinate [right=\nodevspace of encode] (a);
 \coordinate [above=2.0\nodehspace of a] (b);
 \draw [very thick] (encode) -- (a) -- (b);
 
\end{tikzpicture}}
 \vfill
 \begin{enumerate}
  \item 問題インスタンスをASPのファクト形式に変換する.
  \item ASPファクトと配電網遷移問題を解くASP符号化を入力として,
        ASPシステムを用いて解集合を計算する.
  \item 解集合を解釈して配電網遷移問題の解を得る.
 \end{enumerate}
\end{frame}
%%%%%%%%%%%%%%%%%%%%%%%%%%%%%%%%%%%%%%%%%%%%%%%%%%
%% ASPの構文
%%%%%%%%%%%%%%%%%%%%%%%%%%%%%%%%%%%%%%%%%%%%%%%%%%
\begin{frame}{ASPの構文}
  \renewcommand{\thefootnote}{\fnsymbol{footnote}}
  \setcounter{footnote}{1}
  \begin{alertblock}{}\centering
    ASPの言語は論理プログラムをベースとしている~\footnotemark.
  \end{alertblock}\vfill
  \begin{itemize}
  \item \structure{\bf 論理プログラム}とは,以下の\structure{\bf ルール}の有限集合である.
    \begin{center}
      \begin{minipage}[c]{0.7\textwidth}
        \begin{block}{}\centering \vskip -1em
         \begin{align*}
         \underbrace{a_0}_{\text{ヘッド}} \quad\text{\code{:-}}\quad
         \underbrace{a_1 \text{\code{,}}\ldots\text{\code{,}} a_m \text{\code{,}}%
         \ \text{\code{not}}~ a_{m+1} \text{\code{,}}\ldots\text{\code{,}} \text{\code{not}}~a_n
         \text{\code{.}}}_{\text{ボディ}}
         \end{align*}
        \end{block}        
      \end{minipage}
   \end{center}\vfill
    $0 \leq m \leq n$ であり,各 $a_i$ はアトム,
    \code{not}は\structure{\bf デフォルトの否定},\\
    ``\code{,}''は連言(AND)を表す.
        % ``\code{:-}''の左辺を\structure{\bf ヘッド},
		% 右辺を\structure{\bf ボディ}と呼ぶ.
  \item \alert{\bf 直感的な意味}は,
    「$a_1,\ldots,a_m$がすべて成り立ち,
    $a_{m+1},\ldots,a_n$のそれぞれが成り立たないならば,
    $a_0$が成り立つ」である.
  \end{itemize}
  \footnotetext{本発表では標準論理プログラムを単に論理プログラムと呼ぶ.}
\end{frame}
%%%%%%%%%%%%%%%%%%%%%%%%%%%%%%%%%%%%%%%%%%%%%%%%%%
%% ASPの拡張構文
%%%%%%%%%%%%%%%%%%%%%%%%%%%%%%%%%%%%%%%%%%%%%%%%%%
\begin{frame}{ASPの構文}
\begin{itemize}
   \item ボディが空のルールを\structure{\bf ファクト}と呼び,``\code{:-}''は省略できる.
         \begin{align*}
          \underbrace{a_0}_{\text{ヘッド}}\!\text{\code{.}}
         \end{align*}\vfill
  \item ヘッドが空のルールを\structure{\bf 一貫性制約}と呼ぶ.
        \begin{align*}
         \text{\code{:-}}\quad\underbrace{a_1 \text{\code{,}}\ldots\text{\code{,}} a_m \text{\code{,}}%
         \ \text{\code{not}}~ a_{m+1} \text{\code{,}}\ldots\text{\code{,}} \text{\code{not}}~a_n
         \text{\code{.}}}_{\text{ボディ}}
        \end{align*}
        ボディのリテラル($a_i$あるいは\ \code{not} $a_i$)の連言(AND)が成り立たないことを表す.
        % 例えば,\redunderline{`` \code{:-} $a_1$\code{,} \code{not}~$a_{2}$ ''} は,\\
        % 「$a_1$が成り立つならば,$a_2$が成り立つ」を意味する.
\end{itemize}
\end{frame}
%%%%%%%%%%%%%%%%%%%%%%%%%%%%%%%%%%%%%%%%%%%%%%%%%%
%% ASPの拡張構文
%%%%%%%%%%%%%%%%%%%%%%%%%%%%%%%%%%%%%%%%%%%%%%%%%%
\begin{frame}{ASPの拡張構文}
\begin{alertblock}{}\centering
  組合せ問題を解くための便利な構文が用意されている.
\end{alertblock}
\begin{itemize}
 \item \structure{\bf 選択子}
   \begin{center}
     \code{\{}$a_1$\code{;}\ldots\code{;}$a_n$\code{\}}
   \end{center}
   アトム集合 $\{a_1,\dots,a_n\}$
   の任意の部分集合が成り立つことを意味する.
       \vfill
 \item \structure{\bf 個数制約}
   \begin{center}
     $lb$\ \code{\{}$a_1$\code{;}\ldots\code{;}$a_n$\code{\}}\ $ub$
   \end{center}
   $a_1,\dots,a_n$ のうち,
   $lb$個以上,$ub$個以下が成り立つことを意味する.
 % \item \structure{\bf 重み付き個数制約}
 %   \begin{center}
 %     $lb$ \code{\#sum\{} $w_1$\code{:}$a_1$\code{;}\ldots\code{;}$w_n$\code{:}$a_n$ \code{\}} $ub$
 %   \end{center}
 %   $a_1,\dots,a_n$のうち,
 %   成り立つアトムの重み和が$lb$以上,$ub$以下になることを意味する.
\end{itemize}
\end{frame} 
%%%%%%%%%%%%%%%%%%%%%%%%%%%%%%%%%%%%%%%%%%%%%%%%%%
%% 辺の数の表
%%%%%%%%%%%%%%%%%%%%%%%%%%%%%%%%%%%%%%%%%%%%%%%%%%
\begin{frame}{実験結果: 解けた問題数による比較}
 
\begin{textblock*}{\linewidth}(10pt, 30pt)
\begin{table}[t]
 \begin{tabular}[t]{ccccc}
 \rowcolor[RGB]{0,96,0}
 \color{white} 辺の範囲 & \color{white}問題数 & 
		 \color{white}基本符号化 & \color{white}改良符号化 & \color{white}有向符号化\\
 %%%%%%%%
 \rowcolor[RGB]{230,239,230}
 ~~~~\;\:1 ~ 1,000 & 30 & \alert{30} & \alert{30} & \alert{30} \\
 \rowcolor[RGB]{196,230,196}
 1,001 ~ 4,000 & 20 & \alert{20} & \alert{20} & \alert{20} \\
 \rowcolor[RGB]{230,239,230}
 4,001 ~ 7,000 & 11 & 9 & 10 & \alert{11} \\
 \rowcolor[RGB]{196,230,196}
 ~\:7,001 ~ 10,000 & 8 & 4 & 6 &\alert{7} \\
 \rowcolor[RGB]{230,239,230}
 10,001 ~ 20,000 & 9 & 2 & 5 &\alert{9} \\
 \rowcolor[RGB]{196,230,196}
 20,001 ~ 30,000 & 2 & 1 & \alert{2} & \alert{2} \\
 \rowcolor[RGB]{230,239,230}
 30,001 ~ 40,000 & 1 & 0 & 0 & \alert{1} \\
 \rowcolor[RGB]{196,230,196}
 40,001 ~ 50,000 & 4 & 0 & 2 & \alert{4} \\
 %%%%%%%% 合計
 \noalign{\hrule height 0.5pt}
 \rowcolor[RGB]{230,239,230}
 計 & 85 & 66 & 75 & \alert{84} \\
 
\end{tabular}

\end{table}\vfill

\begin{itemize}
 \item 有向符号化は,ベンチマーク問題85問中,\textbf{84問}を解いている.
 \item 大規模な問題に対しても有向符号化は,優位性を示した.
 \item 有向符号化で解けなかったグラフは,\textit{leighton graph}と呼ばれる
       グラフから生成したものであった.
\end{itemize}\vfill
\end{textblock*}
\end{frame}
%%%%%%%%%%%%%%%%%%%%%%%%%%%%%%%%%%%%%%%%%%%%%%%%%%
%% 電流制約
%%%%%%%%%%%%%%%%%%%%%%%%%%%%%%%%%%%%%%%%%%%%%%%%%%
\begin{frame}{電流制約}
\begin{block}{電流制約}\small
 \centering
 \vskip -1em
 \begin{align*}
  J_i = \displaystyle\sum_{j\in S_i^{down}} I_j, \quad J_i \leq J^{max} 
  \quad (\forall s_{i}\in S)
 \end{align*}\vskip -1em
\begin{tabular}{ll}
 $S$ & セクションの集合 \\
 $s_i^{down}$ & セクション$i$より下流にあるセクション \\
 $I_i$ & セクション$i$の負荷電流 \\
 $J_i$ & セクション$i$に流れる電流 \\
 $J^{max}$ & 電流の許容範囲 (入力)
\end{tabular}
\end{block}\vfill
 \begin{exampleblock}{電流の計算例}
  \centering
  %%%%%%%%%%%%%%%%%%%%%%%%%%%%%%%%%%%%%%%%%%%%%%%%%%
% 電気制約の例
%%%%%%%%%%%%%%%%%%%%%%%%%%%%%%%%%%%%%%%%%%%%%%%%%%

\begin{tikzpicture}[scale=0.5]

 % 設定
 \tikzset{node/.style={rectangle, draw=black,fill=white}}

 \definecolor{edge1}{RGB}{191,0,0}
 \definecolor{node1}{RGB}{249,200,200}
 \definecolor{edge3}{RGB}{38,38,134}
 \definecolor{node3}{RGB}{200,200,249}

 % 補助線
 % \draw [help lines,blue] (0,0) grid (20,6);

 % node %
 \node[circle, ultra thick, draw=edge1, fill=node1,minimum size=1cm](1){};
 \node[node, thick, fill=node1, draw=edge1, right=2.5cm of 1] (2){};
 \node[node, thick, fill=node1, draw=edge1, right=3cm of 2] (3){};
 \node[node, thick, fill=node1, draw=edge1, right=2.5cm of 3] (4){};

 % 変電所 %
 \begin{scope}[scale=1.5]
 \draw [ultra thick, draw=edge1] (0,0) circle [radius=0.225cm] node[minimum size=0.5cm](root1){};
 \draw [ultra thick, draw=edge1] (0.225,0)--(0.35,0)--(0.35,0.35);
 \draw [ultra thick, draw=edge1] (-0.225,0)--(-0.35,0)--(-0.35,-0.35);
 \draw [ultra thick, draw=edge1] (0,0.225)--(0,0.35);
 \draw [ultra thick, draw=edge1] (0,-0.225)--(0,-0.35);
 \draw [ultra thick, draw=edge1] [domain=-0.284:-0.159] plot(\x,\x);
 \draw [ultra thick, draw=edge1] [domain=0.159:0.284] plot(\x,\x);
 \draw [ultra thick, draw=edge1] [domain=-0.284:-0.159] plot(\x,-\x);
 \draw [ultra thick, draw=edge1] [domain=0.159:0.284] plot(\x,-\x);
 \end{scope}

 \draw [line width=3.5pt, edge1] (1) -- %
 node[above, font=\Large, label=below:\color{black}{$I_i\colon\quad$30A}]
 {\textbf{$J_i\colon\quad$\!\!60A}}(5,0) -- (2);
 
 \draw [line width=2.5pt, edge1] (2) -- %
 node[above, font=\Large, label=below:\color{black}{20A}] {\textbf{30A}}(11,0) -- (3);

 \draw [line width=1.5pt, edge1] (3) -- %
 node[above, font=\large, label=below:\color{black}{10A}] {\textbf{10A}}(17,0) -- (4);

\end{tikzpicture}

%%%%%%%%%%%%%%%%%%%%%%%%%%%%%%%%%%%%%%%%%%%%%%%%%%%%%%%%%%
%%% Local Variables:
%%% mode: japanese-latex
%%% TeX-master: ``slide''
%%% End:

 \end{exampleblock}\vfill
 \begin{itemize}
  \item 電流が許容範囲を超えると電線が焼き切れる事故につながる.
 \end{itemize}\vfill
\end{frame}
%%%%%%%%%%%%%%%%%%%%%%%%%%%%%%%%%%%%%%%%%%%%%%%%%%
%% 電圧制約
%%%%%%%%%%%%%%%%%%%%%%%%%%%%%%%%%%%%%%%%%%%%%%%%%%
\begin{frame}{電圧制約}
\begin{block}{電圧制約}\small
 \centering
 \vskip -1em
 \begin{align*}
  V_i = V_0 - \displaystyle\sum_{s_j\in S_i^{up}\cup \{s_i\}} Z_j 
  \left[
  \displaystyle\sum_{s_k\in S_j^{down}} I_k + \frac{I_j}{2}
  \right], \quad V_i, \geq V^{min} \quad (\forall s_{i}\in S)
 \end{align*}\vskip -5pt
\begin{tabular}{ll}
 %$S$ & セクションの集合 \\
 $s_i^{up}$ & セクション$i$より上流にあるセクション \\
 $s_i^{down}$ & セクション$i$より下流にあるセクション \\
 $I_i$ & セクション$i$の電流 \\
 $Z_i$ & セクション$i$のインピーダンス \\
 $V_i$ & セクション$i$における電圧 \\
 $V^{min}$ & 電圧の許容範囲 (入力)
\end{tabular}
\end{block}\vfill
 \begin{itemize}
  \item 電圧が許容範囲を下回ると,電化製品などが適切に動作できない恐れがある.
  \item 純粋なASPのみで扱うことが困難な制約である.
 \end{itemize}\vfill
\end{frame}
\backupend

%%% Local Variables:
%%% mode: japanese-latex
%%% TeX-master: "slide.tex"
%%% End:
