\documentclass[dvipdfmx,11pt]{beamer}

\usepackage[deluxe]{otf} 
\usepackage{txfonts}
\renewcommand{\kanjifamilydefault}{\gtdefault}
\usepackage{amssymb,amsmath}
\usepackage{hyperref}
\usepackage[absolute,overlay]{textpos}
\usepackage{comment}
\usepackage{colortbl}
\usepackage{graphicx}
\usepackage{tikz}
\usetikzlibrary{positioning}
\usetikzlibrary{shadows}
\usetikzlibrary{calc}
\usepackage{listings}
\usepackage{plistings}
\usepackage{multicol}
\usepackage{multirow}
\usepackage{caption}
\def\lstlistingname{コード}
\lstset{escapeinside=||}
\newcommand{\code}[1]{\lstinline[basicstyle=\ttfamily]{#1}}
\newcommand{\lw}[1]{\smash{\lower-5.ex\hbox{#1}}}
\newcommand{\redunderline}[1]{\textcolor{red}{\underline{\textcolor{black}{#1}}}}
%%\usetheme{Frankfurt}
\usetheme{Warsaw}
\setbeamertemplate{navigation symbols}{} %スライドのボタン?(右下のやつ)を消す
\setbeamersize{text margin left=1.5em,text margin right=1.5em} % 余白なくすやつ

% footer setting %
\makeatother
\setbeamertemplate{footline}
{
  \leavevmode%
  \hbox{%
  \begin{beamercolorbox}[wd=.4\paperwidth,ht=2.25ex,dp=1ex,center]{author in head/foot}%
    \usebeamerfont{author in head/foot}\insertshortauthor
  \end{beamercolorbox}%
  \begin{beamercolorbox}[wd=.6\paperwidth,ht=2.25ex,dp=1ex,center]{title in head/foot}%
    \usebeamerfont{title in head/foot}\hspace*{1ex} \insertshorttitle\hspace*{3em}
    \textbf{ \insertframenumber{} / \inserttotalframenumber } \hspace*{1ex}
  \end{beamercolorbox}}%
  \vskip0pt%
}
\makeatletter

% exclude apprendix slides from framenumber %
\newcommand{\backupbegin}{
   \newcounter{framenumberappendix}
   \setcounter{framenumberappendix}{\value{framenumber}}
}
\newcommand{\backupend}{
   \addtocounter{framenumberappendix}{-\value{framenumber}}
   \addtocounter{framenumber}{\value{framenumberappendix}} 
}

\lstset{
 basicstyle=\ttfamily\color{black},
 keepspaces=true,
 escapechar=|,
 columns=[l]{fullflexible},
 commentstyle={\color{red}},
 stringstyle={\color{blue}}}

\title{解集合プログラミングを用いた\\配電網問題の解法}
\author[山田 健太郎,湊 真一,田村 直之,番原 睦則]
{山田 健太郎$^1$,湊 真一$^2$,田村 直之$^3$,番原 睦則$^1$}
\date{第24回プログラミングおよびプログラミング言語ワークショップ}
\institute{1.名古屋大学 大学院情報学研究科\\
2.京都大学 大学院情報学研究科\\
3.神戸大学 情報基盤センター
}

%#################################################
%# 本文 ##########################################
%#################################################
\begin{document}

%%%%%%%%%%%%%%%%%%%%%%%%%%%%%%%%%%%%%%%%%%%%%%%%%%
%% タイトル 
%%%%%%%%%%%%%%%%%%%%%%%%%%%%%%%%%%%%%%%%%%%%%%%%%%
\begin{frame}{}
  \titlepage
\end{frame}

%%%%%%%%%%%%%%%%%%%%%%%%%%%%%%%%%%%%%%%%%%%%%%%%%%
% 配電網
%%%%%%%%%%%%%%%%%%%%%%%%%%%%%%%%%%%%%%%%%%%%%%%%%%
\begin{frame}{配電網問題}
  \begin{alertblock}{}\centering
    求解困難な組合せ最適化問題の一種
  \end{alertblock}
  \vfill
  \begin{itemize}
  \item \alert{\bf 配電網}とは,変電所と,一般家庭や工場を繋ぐ電力供給
    経路のネットワークである.
  \item  配電網の構成技術はスマートグリッドや,災害時の停電復旧
         などを支える重要な基盤技術として期待されている.
  \item \alert{\bf 配電網問題}とは,
    \begin{itemize}
    \item \structure{\bf トポロジ制約}と\structure{\bf 電気制約}を満たしつつ,
    \item 損失電力を最小にするスイッチの開閉状態を求めることが目的.
    \end{itemize}
  \item これまで,メタヒューリスティクス等の解法が提案されている.
  \item 厳密解法として,フロンティア法を用いた解法が提案されている.
    \begin{itemize}
    \item 実用規模の配電網問題(\structure{\textsf{\bf fukui-tepco},\textbf{スイッチ数468個}})の
      最適解を求めることに成功~[井上ほか '12].
    \end{itemize}
  \end{itemize}
\end{frame}
%%%%%%%%%%%%%%%%%%%%%%%%%%%%%%%%%%%%%%%%%%%%%%%%%%
%% 解の遷移問題
%%%%%%%%%%%%%%%%%%%%%%%%%%%%%%%%%%%%%%%%%%%%%%%%%%
\begin{frame}{配電網遷移問題}
 \begin{alertblock}{配電網遷移問題}
  配電網問題とその2つの実行可能解が与えられたとき,
  一方の解から他方の解へ,\alert{\bf 遷移制約}を満たしつつ,
  実行可能解のみを経由して到達できるかどうかを判定する問題.
  \begin{itemize}
  \item 各ステップ$t$で変更可能なスイッチを$d$個に制限.(\textbf{遷移制約})
  \item 本研究では,到達可能であればその最短経路を求めることが目的.
  %\item 最短ステップ長での辺の変更手順を求めることが目的となる.
  \end{itemize}
 \end{alertblock}
 \begin{itemize}
  \item 配電網の構成制御における災害時の停電復旧などへの応用が狙い.
  \item 近年,理論計算機科学の分野を中心に急速に発展している
        \structure{\bf 組合せ遷移問題}の一種.
 \end{itemize}
 \begin{alertblock}{}\centering
  しかし現状では,配電網遷移問題を効率的に解くソルバーの \\
  \alert{\bf 実装技術は確立されていない}.
 \end{alertblock}
\end{frame}
%%%%%%%%%%%%%%%%%%%%%%%%%%%%%%%%%%%%%%%%%%%%%%%%%%
%% 遷移問題の例
%%%%%%%%%%%%%%%%%%%%%%%%%%%%%%%%%%%%%%%%%%%%%%%%%%
\begin{frame}{配電網遷移問題の例}
  \renewcommand{\thefootnote}{\fnsymbol{footnote}}
  \setcounter{footnote}{1}
  \begin{columns}
    \begin{column}{0.45\textwidth}\centering
      \begin{exampleblock}{スタート状態}
    \centering
    \scalebox{0.35}{\input{tikz/tikz-test-core-start}}
      \end{exampleblock}
    \end{column}
    \begin{column}{0.05\textwidth}\centering
      $\Rightarrow$
    \end{column}
    \begin{column}{0.45\textwidth}\centering
      \begin{exampleblock}{ゴール状態}
        \centering
        \scalebox{0.35}{\begin{tikzpicture}

 % setting
 \tikzset{customer/.style={rectangle,thick,draw=black,minimum size=0.5cm}}
 \tikzset{on_switch/.style={rectangle,fill=black}}
 \tikzset{off_switch/.style={rectangle,draw=black,fill=white}}
 
 \tikzset{node distance =1cm};

 % substation (x, y, label)
 \newcommand{\substation}[3]{
 \draw [very thick] (#1,#2) circle [radius=0.225cm] node[draw=none,minimum size=1cm](#3){};
 \draw [very thick] (#1+0.225,#2)--(#1+0.35,#2)--(#1+0.35,#2+0.3);
 \draw [very thick] (#1-0.225,#2)--(#1-0.35,#2)--(#1-0.35,#2-0.3);
 \draw [very thick] (#1,#2+0.225)--(#1,#2+0.35);
 \draw [very thick] (#1,#2-0.225)--(#1,#2-0.35);
 \draw [very thick] [domain=-0.284:-0.159] plot(\x+#1,\x+#2);
 \draw [very thick] [domain=0.159:0.284] plot(\x+#1,\x+#2);
 \draw [very thick] [domain=-0.284:-0.159] plot(\x+#1,-\x+#2);
 \draw [very thick] [domain=0.159:0.284] plot(\x+#1,-\x+#2);
 }

 %switch node (position, label, cap)
 %% right switch
 \newcommand{\swnodeR}[4]{
 \coordinate[#1] (#2);
 \node[#1,customer] (#2){#4};
 \node[circle, draw=black, text width=0.2cm, 
 right=0cm of #2, scale=0.3, thick] {};
 \node[right=0cm of #2,scale=0.3, minimum size=0.8cm] (#3){};
 }
 %% left switch
 \newcommand{\swnodeL}[4]{
 %\coordinate[#1] (#2);
 \node[#1,customer] (#2){#4};
 \node[circle, draw=black, fill=white, text width=0.2cm, 
 left=0cm of #2, scale=0.3, thick] (#3){};
 }
 % above switch
 \newcommand{\swnodeA}[4]{
 \coordinate[#1] (#2);
 \node[#1,customer] (#2){#4};
 \node[circle, draw=black, text width=0.2cm, 
 above=0cm of #2, scale=0.3, thick] (#3){};
 }
 % below switch
 \newcommand{\swnodeB}[4]{
 \coordinate[#1] (#2);
 \node[#1,customer] (#2){#4};
 \node[circle, draw=black, text width=0.2cm, 
 below=0cm of #2, scale=0.3, thick] {};
 \node[below=0cm of #2,scale=0.3,minimum size=0.8cm] (#3){};
 }
 
 \substation{0}{0}{sub};
 
 % root1
 \node[customer,fill=purple!60,below =4.5cm of sub] (root1) { };
 \swnodeL{left =of root1,fill=purple!60}{node1}{sw1}{ };
 
 \swnodeR{left=of node1,fill=purple!60}{node2}{sw2}{ };
 \node[customer,left=of node2,fill=purple!60] (junc1){ };
 \swnodeL{left =of junc1,fill=purple!60}{node3}{sw3}{ };
 \swnodeA{above= of junc1,fill=purple!60}{node4}{sw4}{ }

 \swnodeR{left=of node3,fill=purple!60}{node5}{sw5}{ };
 \swnodeA{above =of node5,fill=purple!60}{node6}{sw6}{ };

 \swnodeB{above =of node6,fill=purple!60}{node7}{sw7}{ };
 \swnodeA{above =of node7,fill=purple!60}{node29}{sw29}{ };

 \swnodeB{above =of node29,fill=cyan!80}{node30}{sw30}{ };
 
 \swnodeB{above =of node4,fill=purple!60}{node8}{sw8}{ };
 \swnodeA{above =of node8,fill=purple!60}{node31}{sw31}{ };
 
 \swnodeB{above =of node31,fill=cyan!80}{node32}{sw32}{ };

 \swnodeR{right =of root1,fill=purple!60}{node17}{sw17}{ };

 \swnodeL{right =of node17,fill=purple!60}{node18}{sw18}{ };

 % root2
 \node[customer,fill=cyan!80,above=4.5cm of sub, fill=cyan!80] (root2) { };
 \swnodeL{left =of root2,fill=cyan!80}{node9}{sw9}{ };

 \swnodeR{left=of node9,fill=cyan!80}{node10}{sw10}{ };
 \node[customer,left=of node10,fill=cyan!80] (junc2){ };
 \swnodeL{left =of junc2,fill=cyan!80}{node11}{sw11}{ };
 \swnodeB{below =of junc2,fill=cyan!80}{node12}{sw12}{ };
 
 \swnodeR{left =of node11,fill=cyan!80}{node13}{sw13}{ };
 \swnodeB{below =of node13,fill=cyan!80}{node14}{sw14}{ };
 
 \swnodeA{below =of node14,fill=cyan!80}{node15}{sw15}{ };

 \swnodeA{below =of node12,fill=cyan!80}{node16}{sw16}{ };

 \swnodeR{right =of root2,fill=cyan!80}{node22}{sw22}{ };

 \swnodeL{right =of node22,fill=cyan!80}{node23}{sw23}{ };
 
 % root3
 \node[customer,fill=cyan!80,right=5.2cm of sub,fill=yellow!80] (root3) { };
 \swnodeB{below =1.4of root3,fill=yellow!80}{node24}{sw24}{ };
 \swnodeA{above =1.4of root3,fill=yellow!80}{node25}{sw25}{ };

 \swnodeA{below =of node24,fill=purple!60}{node19}{sw19}{ };
 \swnodeL{below =1.3of node19,fill=purple!60}{node20}{sw20}{ };
 
 \swnodeR{left =of node20,fill=purple!60}{node21}{sw21}{ };

 \swnodeB{above =of node25,fill=cyan!80}{node26}{sw26}{ };
 \swnodeL{above =1.3of node26,fill=cyan!80}{node27}{sw27}{ };

 \swnodeR{left =of node27,fill=cyan!80}{node28}{sw28}{ };
 
 % sections
 \foreach \v / \u / \t in {root1/sub/$s_a$,root1/node1/$s_1$,node2/junc1/$s_2$, %
 junc1/node3/$s_3$,junc1/node4/$s_4$,node5/node6/$s_5$,node7/node29/$s_6$,node30/node15/$s_7$, %
 sub/root2/$s_b$,root2/node9/$s_8$,node10/junc2/$s_9$,junc2/node11/$s_{10}$,node12/junc2/$s_{11}$, %
 node14/node13/$s_{12}$,node8/node31/$s_{13}$,node32/node16/$s_{14}$,node17/root1/$s_{15}$, %
 node22/root2/$s_{16}$,root3/sub/$s_c$,node24/root3/$s_{17}$,root3/node25/$s_{18}$, %
 node20/node19/$s_{19}$,node26/node27/$s_{20}$,node21/node18/$s_{21}$,node28/node23/$s_{22}$} %
 \draw[thick] (\v) --  node[auto=right]{\t} (\u);

 % switches
 %% horizontal
 \foreach \v / \u / \t in {sw1/sw2/$sw_{1}$,sw3/sw5/$sw_{2}$,sw9/sw10/$sw_{11}$,
 sw11/sw13/$sw_{12}$,sw18/sw17/$sw_{13}$,sw20/sw21/$sw_{14}$,sw23/sw22/$sw_{15}$,
 sw27/sw28/$sw_{16}$}
 \draw[very thick] (\v) -- node[below=0.2of \v]{\t} (\u.center);
 
 % \foreach \v / \u in {}
 % \draw[very thick] (\v) -- (\u.45);
 %% vertical
 \foreach \v / \u / \t in {sw4/sw8/$sw_{4}$,sw6/sw7/$sw_{3}$,sw15/sw14/$sw_{7}$,%
 sw16/sw12/$sw_{8}$}
 \draw[very thick] (\v) -- node[auto=below]{\t~~~~~~~~~~} (\u.center);

 \foreach \v / \u in {sw29/sw30,sw31/sw32,sw25/sw26,sw19/sw24}
 \draw[very thick] (\v) -- (\u.-30);

 \coordinate[above=0.5of node7](A);
 \coordinate[above=0.3of node26](B);
 \coordinate[above=0.8of node20](C);
 \draw[ultra thick, draw=blue, densely dotted] (A) circle[x radius=0.8,y radius=2.2];
 \draw[ultra thick, draw=red, densely dotted] (B) circle[x radius=0.8,y radius=1.8];
 \draw[ultra thick, draw=teal, densely dotted] (C) circle[x radius=0.8,y radius=1.8];

\end{tikzpicture}

%%%%%%%%%%%%%%%%%%%%%%%%%%%%%%%%%%%%%%%%%%%%%%%%%%%%%%%%%%
%%% Local Variables:
%%% mode: japanese-latex
%%% TeX-master: paper.tex
%%% End:
}
      \end{exampleblock}
    \end{column}
  \end{columns}
 \vfill
 \begin{itemize}
  \item セクション数:25個,スイッチ数:16個,変電所数:3個\footnote{%
        変電所に直接つながるセクションの数}%
  \item 各ステップで変更可能なスイッチを2個に制限.(\textbf{遷移制約})
  \item スタート状態からゴール状態へ\structure{\bf 3ステップで到達可能}.
 \end{itemize}
\end{frame}
%%%%%%%%%%%%%%%%%%%%%%%%%%%%%%%%%%%%%%%%%%%%%%%%%%
%% ASP
%%%%%%%%%%%%%%%%%%%%%%%%%%%%%%%%%%%%%%%%%%%%%%%%%%
\begin{frame}{解集合プログラミング(Answer Set Programming; ASP)}
\vfill
 \begin{itemize}
  \item \structure{\bf ASPの言語}は一階論理に基づく知識表現言語の一種である.
  \item \structure{\bf ASPシステム}は論理プログラムから安定モデル意味
        論~[Gelfond and Lifschitz '88]に基づく解集合を計算するシステムである.
  \item 近年,SATソルバーの実装技術を応用した高速ASPシステムが実現され,
        システム検証,プランニング,システム生物学など様々な分野への応用が
        拡大している.
 \end{itemize}
 \vfill
 \begin{alertblock}{配電網遷移問題に対してASP技術を用いる利点}
  \begin{itemize}
   \item ASP言語の高い表現力を活かし,組合せ問題を\textbf{簡潔に記述可能}
         \begin{itemize}
          \item \alert{\bf 組合せ遷移問題への拡張も容易}
         \end{itemize}
   \item マルチショットASP解法により,ステップ長を増やしながら,組合せ遷移問題の
         \alert{\bf 到達可能性を効率的に検査可能}
         \begin{itemize}
          \item ASPシステムを複数回起動するオーバヘッドを回避可能
          \item 同様の探索失敗を避けるために獲得した学習節を再利用可能
         \end{itemize}
  \end{itemize}
 \end{alertblock} \vfill
\end{frame}
%%%%%%%%%%%%%%%%%%%%%%%%%%%%%%%%%%%%%%%%%%%%%%%%%%
%% 研究目的
%%%%%%%%%%%%%%%%%%%%%%%%%%%%%%%%%%%%%%%%%%%%%%%%%%
\begin{frame}{研究目的}
  \begin{alertblock}{目的}
   ASP技術を活用した大規模な配電網遷移問題を効率良く解くシステムを構築
   する.
  \end{alertblock}
  \vfill
 \begin{block}{研究内容}
  \begin{enumerate}
   \item \structure{\bf 配電網問題のASP符号化の考案}
         \begin{itemize}
          \item トポロジ制約のASP符号化として,\textbf{基本符号化},\textbf{改良符号化},
                \\ \alert{\bf 有向符号化} の3種類を考案.
          \item 電気制約として,電流制約のASP符号化を考案.
         \end{itemize}
   \item \structure{\bf 配電網遷移問題のASP符号化の考案}
         \begin{itemize}
          \item \textbf{シングルショット符号化}と\alert{\bf マルチショット符号化}の2種類を考案.
          \item 配電網問題のASP符号化の自然な拡張である.
         \end{itemize}
   \item \structure{\bf 実用規模の問題を含むベンチマークによる評価実験}
         \begin{itemize}
          \item 実用規模の問題である\textsf{fukui-tepco}をもとに,
                配電網遷移問題ベンチマークを1000問作成.
          \item \textbf{マルチショット符号化}は,
                \alert{\bf 約3.8倍の高速化}を実現.
         \end{itemize}
  \end{enumerate}
 \end{block}
\end{frame}
%%%%%%%%%%%%%%%%%%%%%%%%%%%%%%%%%%%%%%%%%%%%%%%%%%
%% トポロジ制約
%%%%%%%%%%%%%%%%%%%%%%%%%%%%%%%%%%%%%%%%%%%%%%%%%%
\begin{frame}{~}
 \LARGE \centering
 \structure{\bf 配電網問題}
\end{frame}
%
\begin{frame}[noframenumbering]{配電網問題のトポロジ制約}
 \begin{alertblock}{}
  トポロジ制約を満たす配電網構成は,グラフと根と呼ばれる特別なノードから,
  \alert{\bf 根付き全域森}を求める部分グラフ探索問題に帰着できる.
 \end{alertblock}
 \vfill
 \begin{block}{根付き全域森 (Spanning Rooted Forest) [川原・湊 '12]}
  グラフ$G=(V,E)$と,
  \textbf{根}と呼ばれる$V$上のノードが与えられたとき,
  $G$上の根付き全域森とは,以下の条件を満たす$G$の部分グラフ$G'=(V,E'),\ E' \subseteq E$である.
  \begin{enumerate}
   \item $G'$はサイクルを持たない. (\alert{\bf 非閉路制約})
   \item $G'$の各連結成分は,ちょうど1つの根を含む. (\alert{\bf 根付き連結制約})
  \end{enumerate}
 \end{block}
\end{frame}
%%%%%%%%%%%%%%%%%%%%%%%%%%%%%%%%%%%%%%%%%%%%%%%%%%
%% トポロジ制約の例
%%%%%%%%%%%%%%%%%%%%%%%%%%%%%%%%%%%%%%%%%%%%%%%%%%
\begin{frame}{配電網問題のトポロジ制約}
  \renewcommand{\thefootnote}{\fnsymbol{footnote}}
  \setcounter{footnote}{1}
  \begin{columns}
    \begin{column}{0.45\textwidth}\centering
      \begin{exampleblock}{配電網問題の解}
    \centering
    \scalebox{0.3}{\input{tikz/tikz-test-output}}
      \end{exampleblock}
    \end{column}
    \begin{column}{0.45\textwidth}\centering
      \begin{exampleblock}{根付き全域森}
        \centering
       \scalebox{0.5}{\hspace{2zw}%%%%%%%%%%%%%%%%%%%%%%%%%%%%%%%%%%%%%%%%%%%%%%%%%%
% 根付き全域森 (第2章で使う)
%%%%%%%%%%%%%%%%%%%%%%%%%%%%%%%%%%%%%%%%%%%%%%%%%%

\begin{tikzpicture}[x=1.5cm,y=1.5cm,scale=0.7]

 % 設定
 \tikzset{root/.style={circle,draw=black,fill=gray!30,minimum size=1cm}}
 \tikzset{node/.style={circle,draw=black,minimum size=1cm}}
 
 % 補助線
 % \draw [help lines,blue,step=2cm] (-3,0) grid (3,-3);


 \node[node,fill=purple!60,label=above:$r_1$] (r1){$1$};

 \node[node,fill=purple!60,left=of r1] (n1){$5$};
 \node[node,fill=purple!60,left=of n1] (n2){$4$};
 \node[node,fill=cyan!80,above=of n2] (n3){$6$};
 \node[node,fill=purple!60,above=of n1] (n4){$8$};
 \node[node,fill=cyan!80,above=of n3] (n5){$8$};
 \node[node,fill=cyan!80,above=of n4] (n6){$9$};
 \node[node,fill=cyan!80,above=of n5] (n7){$10$};
 \node[node,fill=cyan!80,above=of n6] (n8){$11$};

 \node[node,fill=cyan!80,right=of n8,label=above:$r_2$] (r2){$2$};
 \node[node,fill=cyan!80,right=of r2] (n9){$14$};
 \node[node,fill=yellow!80,right=of n9] (n10){$15$};
 \node[node,fill=purple!60,right=of r1] (n11){$12$};
 \node[node,fill=yellow!80,right=of n11] (n12){$13$};
 
 \node[node,fill=yellow!80,label=right:$r_3$](r3) at ($(n10)!0.5!(n12)$) {$3$};
 
 \foreach \v / \u in {r1/n1,n1/n2,n1/n4,n3/n5,n5/n7,n6/n8,n7/n8,n8/r2,
 r2/n9,r1/n11,r3/n10,r3/n12}
 \draw[thick] (\v) -- (\u);

\end{tikzpicture}

%%%%%%%%%%%%%%%%%%%%%%%%%%%%%%%%%%%%%%%%%%%%%%%%%%%%%%%%%%
%%% Local Variables:
%%% mode: japanese-latex
%%% TeX-master: paper.tex
%%% End:
}
      \end{exampleblock}
    \end{column}
  \end{columns}
 \vfill
  \begin{itemize}
   \item \structure{\bf 停電}(変電所と結ばれないセクション)
   \item \structure{\bf 短絡}(供給経路上のループ,複数の変電所と結ばれるセクション)
  \item \structure{\bf 配電網とグラフの対応}
	 \begin{center}
      \vskip -0.5em
      \begin{minipage}[c]{0.7\textwidth}
	   \begin{block}{}
		\centering
		\begin{tabular}{c|ccc}
		配電網 & セクション & スイッチ & 変電所 \\
		\hline
		グラフ & ノード & 辺 & 根
		\end{tabular}
	   \end{block}
      \end{minipage}
        \end{center}\vfill
  \end{itemize}
\end{frame}
%%%%%%%%%%%%%%%%%%%%%%%%%%%%%%%%%%%%%%%%%%%%%%%%%%
%% 配電網問題 ASP符号化
%%%%%%%%%%%%%%%%%%%%%%%%%%%%%%%%%%%%%%%%%%%%%%%%%%
\begin{frame}{トポロジ制約のASP符号化}
  \renewcommand{\thefootnote}{\fnsymbol{footnote}}
  \setcounter{footnote}{1}
\begin{block}{}
 \centering
 トポロジ制約に関して,\\
 基本符号化,改良符号化,有向符号化の3種類のASP符号化を考案
\end{block}
 \vfill
 \begin{itemize}
  \item \textbf{基本符号化}は,根付き連結制約を\textit{at-least-one}制約と,
        \textit{at-most-one}制約を用いて表現する基本的な符号化である.
  \item \textbf{改良符号化}は,根付き連結制約をASPの個数制約を用いることで,
        基礎化後のルール数を少なく抑えるように工夫した符号化である.
  \item \textbf{有向符号化}は,無向グラフの各辺$u-v$に対して,2つの弧$u\rightarrow v$と
        $v\rightarrow u$を対応させることで有向グラフ化して解く符号化であり,
        非閉路制約をノードの入次数の制約で簡潔に表現できる.
 \end{itemize}
\end{frame}
%%%%%%%%%%%%%%%%%%%%%%%%%%%%%%%%%%%%%%%%%%%%%%%%%%
%% ASP ファクト
%%%%%%%%%%%%%%%%%%%%%%%%%%%%%%%%%%%%%%%%%%%%%%%%%%
\begin{frame}[fragile]{グラフ表現のASPファクト形式}
\begin{figure}[t]
 \centering
 \scalebox{0.45}{%%%%%%%%%%%%%%%%%%%%%%%%%%%%%%%%%%%%%%%%%%%%%%%%%%
% 根付き全域森 (第2章で使う)
%%%%%%%%%%%%%%%%%%%%%%%%%%%%%%%%%%%%%%%%%%%%%%%%%%

\begin{tikzpicture}[x=1.5cm,y=1.5cm,scale=0.7]

 % 設定
 \tikzset{root/.style={circle,draw=black,fill=gray!30,minimum size=1cm}}
 \tikzset{node/.style={circle,draw=black,minimum size=1cm}}
 
 % 補助線
 % \draw [help lines,blue,step=2cm] (-3,0) grid (3,-3);


 \node[node,fill=black!40,label=above:$r_1$] (r1){$1$};

 \node[node,left=of r1] (n1){$5$};
 \node[node,left=of n1] (n2){$4$};
 \node[node,above=of n2] (n3){$6$};
 \node[node,above=of n1] (n4){$7$};
 \node[node,above=of n3] (n5){$8$};
 \node[node,above=of n4] (n6){$9$};
 \node[node,above=of n5] (n7){$10$};
 \node[node,above=of n6] (n8){$11$};

 \node[node,fill=black!10,right=of n8,label=above:$r_2$] (r2){$2$};
 \node[node,right=of r2] (n9){$14$};
 \node[node,right=of n9] (n10){$15$};
 \node[node,right=of r1] (n11){$12$};
 \node[node,right=of n11] (n12){$13$};
 
 \node[node,pattern=north east lines,label=right:$r_3$](r3) at ($(n10)!0.5!(n12)$) {$3$};
 
 \foreach \v / \u in {r1/n1,n1/n2,n2/n3,n1/n4,n3/n5,n4/n6,n5/n7,n6/n8,n7/n8,n8/r2,
 r2/n9,n9/n10,r1/n11,n11/n12,r3/n10,r3/n12}
 \draw[thick] (\v) -- (\u);

\end{tikzpicture}

%%%%%%%%%%%%%%%%%%%%%%%%%%%%%%%%%%%%%%%%%%%%%%%%%%%%%%%%%%
%%% Local Variables:
%%% mode: japanese-latex
%%% TeX-master: paper.tex
%%% End:
}
\end{figure}
\begin{exampleblock}{}
\begin{lstlisting}
node(1..15).

edge(1,5). edge(1,12). edge(2,11). edge(2,14). 
edge(3,15). edge(3,13). edge(4,5). edge(4,6). 
edge(5,7). edge(6,8). edge(7,9). edge(8,10).
edge(9,11). edge(10,11). edge(12,13).

root(1). root(2). root(3).
\end{lstlisting}
\end{exampleblock}
\end{frame}
%%%%%%%%%%%%%%%%%%%%%%%%%%%%%%%%%%%%%%%%%%%%%%%%%%
%% 有向符号化
%%%%%%%%%%%%%%%%%%%%%%%%%%%%%%%%%%%%%%%%%%%%%%%%%%
\begin{frame}[fragile]{有向符号化のASPコード}
\begin{exampleblock}{}
\begin{lstlisting}
(1) { inForest(X,Y); inForest(Y,X) } 1 :- edge(X,Y).

(2) :- root(R), inForest(_,R).
(3) :- node(X), not root(X), not 1 { inForest(_,X) } 1.

(4) reached(R,R) :- root(R).
(5) reached(X,R) :- reached(Y,R), inForest(Y,X).

(6) :- node(X), not 1 { reached(X,R) } 1.
\end{lstlisting}
\end{exampleblock}\vskip 0.5em
\begin{itemize}
 \only<1>{
 \item (1)のルールで,与えられた無向グラフを有向グラフ化する.
 \item アトム\code{inForest(X,Y)}は,辺\code{(X,Y)}が根付き全域森に含まれることを意味する.
       }
       %
 \only<2>{
 \item (2)--(3)のルールは,非閉路制約を表す.
 \item (2)は,各根\code{R}について,入次数が0であることを表す.
 \item (3)は,根ではない各ノード\code{X}について,入次数がちょうど1であることを表す.
       }
       %
 \only<3>{
 \item (4)--(5)は,到達可能性を表す.
 \item アトム\code{reached(X,R)}は,ノード\code{X}が根\code{R}から到達可能であることを意味する.
 \item (6)は,根付き連結制約を表す.
       }
\end{itemize}
\end{frame}
%%%%%%%%%%%%%%%%%%%%%%%%%%%%%%%%%%%%%%%%%%%%%%%%%%
%% 実験概要--配電網問題
%%%%%%%%%%%%%%%%%%%%%%%%%%%%%%%%%%%%%%%%%%%%%%%%%%
\begin{frame}{実験概要}
  \renewcommand{\thefootnote}{\fnsymbol{footnote}}
  \setcounter{footnote}{1}
  \begin{itemize}
  \item \structure{\bf 比較するASP符号化:}
    \begin{itemize}
     \item 基本符号化
     \item 改良符号化
     \item 有向符号化
    \end{itemize}
  \item \structure{\bf ベンチマーク問題:} 全85問
    \begin{itemize}
    \item DNET\footnote{https://github.com/takemaru/dnet}%
      で公開されている配電網問題 3問 \\ (トポロジ制約のみ,スイッチ数:
      16個,36個,468個)
    \item \textit{Graph Coloring and its Generalizations}
      \footnote{https://mat.tepper.cmu.edu/COLOR04/}で公開されている \\
      グラフ彩色問題をベースに,独自に生成した 82問 
      \footnote{各問題に対し,全ノードのうち1/5個をランダムに変電所として与えた.}\\
      (20 $\leq$ 辺数 $\leq$ 49,629)
    \end{itemize}
  \item \structure{\bf ASPシステム:} \textit{clingo-5.4.0} $+$ \textit{trendy}
  \item \structure{\bf 制限時間:} 3600秒/問
  \item \structure{\bf 実験環境:} Mac mini,3.2GHz Intel Core i7,64GBメモリ
  \end{itemize}
\end{frame}
%%%%%%%%%%%%%%%%%%%%%%%%%%%%%%%%%%%%%%%%%%%%%%%%%%
%% カクタスプロット
%%%%%%%%%%%%%%%%%%%%%%%%%%%%%%%%%%%%%%%%%%%%%%%%%%
\begin{frame}{実験結果:カクタスプロット}
 \begin{figure}[h]
  \centering
  \includegraphics[scale=0.25]{fig/cactus_hq.png}
 \end{figure}

\begin{itemize}
 \item 有向符号化は,他の符号化と比較して,より多くの問題(84/85問)を高速に解いている.
\end{itemize}\vfill
\end{frame}
%%%%%%%%%%%%%%%%%%%%%%%%%%%%%%%%%%%%%%%%%%%%%%%%%%
%% 配電網遷移問題
%%%%%%%%%%%%%%%%%%%%%%%%%%%%%%%%%%%%%%%%%%%%%%%%%%
\begin{frame}{~}
 \LARGE \centering
 \structure{\bf 配電網遷移問題}
\end{frame}
%
\begin{frame}[noframenumbering]{配電網遷移問題の定式化}
  \begin{itemize}
  \item 配電網問題の変数集合
    $\boldsymbol{x} = \{x_1,x_2,\ldots,x_n\}$
    に対して,ステップ$t~\geq 0$での各変数の値を表す変数集合
    $\boldsymbol{x}^{t} = \{x_1^t,x_2^t,\ldots,x_n^t\}$を導入.
  \item スタート状態から$\ell$ステップ遷移した後の各変数の値
    $\boldsymbol{x}^{\ell}$が,ゴール状態を満足するかを判定するため,
    論理式$\varphi_{\ell}$を構成する.
  \end{itemize}
  \begin{block}{}\centering\vskip-1em
  \begin{align*}
  \varphi_{\ell} &= S(\boldsymbol{x}^0)  & S: \textrm{スタート状態を表す論理式} \\
  &\land \bigwedge_{t=0}^{\ell} C(\boldsymbol{x}^t) & C: \textrm{トポロジ制約,電流制約を表す論理式} \\
  &\land \bigwedge_{t=1}^{\ell} T(\boldsymbol{x}^{t-1},\boldsymbol{x}^{t}) 
   & T: \textrm{遷移制約を表す論理式} \\
  &\land G(\boldsymbol{x}^\ell)  & G: \textrm{ゴール状態を表す論理式}
  \end{align*}
  \end{block}
  \begin{itemize}
  \item {$\varphi_{\ell}$}が充足可能の場合,
    ステップ長$\ell$の到達可能な遷移系列が存在する
    ことを意味する.
  \end{itemize}
\end{frame}
%%%%%%%%%%%%%%%%%%%%%%%%%%%%%%%%%%%%%%%%%%%%%%%%%%
%% 提案アプローチ
%%%%%%%%%%%%%%%%%%%%%%%%%%%%%%%%%%%%%%%%%%%%%%%%%%
\begin{frame}{ASPを用いた配電網遷移問題の解法} 
  \begin{alertblock}{}\centering
    配電網遷移問題に対して,制限された長さの遷移系列
    \begin{align*}
    \varphi_{\ell} = S(\boldsymbol{x}^0)  
    \land \bigwedge_{t=0}^{\ell} C(\boldsymbol{x}^t) 
    \land \bigwedge_{t=1}^{\ell} T(\boldsymbol{x}^{t-1},\boldsymbol{x}^{t}) 
    \land G(\boldsymbol{x}^\ell)  
        \end{align*}
    を論理プログラムとして表現し,ASP システムを用いて\\
   実行することにより,到達可能性の検査を行う.
  \end{alertblock}\vfill
  \begin{itemize}
   \item $\varphi_{\ell}$が\structure{\bf 充足可能}の場合,
         ステップ長$\ell$の\structure{\bf 到達可能}な遷移系列が存在.
   \item $\varphi_{\ell}$が\structure{\bf 充足不能}の場合,
         ステップ長$\ell$では\structure{\bf 到達不能}.
   \item 到達不能の場合,$\ell$を増加させた論理プログラムを再構成し,
         繰り返し ASP システムを実行.
  \end{itemize}
\end{frame}
%%%%%%%%%%%%%%%%%%%%%%%%%%%%%%%%%%%%%%%%%%%%%%%%%%
%% 提案アプローチ
%%%%%%%%%%%%%%%%%%%%%%%%%%%%%%%%%%%%%%%%%%%%%%%%%%
\begin{frame}{配電網遷移問題のASP符号化}
% \begin{alertblock}{}
%  \centering
%  配電網遷移問題を解くASP符号化を2種類考案
% \end{alertblock}
% \vfill
   \begin{block}{シングルショット符号化}
    \begin{itemize}
     \item $\varphi_{\ell}$をそのまま1つの論理プログラムとして記述.
     \item 配電網問題のASP符号化の自然な拡張.
    \end{itemize}
   \end{block}
     \begin{itemize}
      \item ステップ長$\ell$を増加させながら,
            $\varphi_{\ell}$を繰り返し構成し解く.
      %\item 長所: 実装が単純である.
      \item 短所: 学習節が再利用できない.
      \item 短所: ASPシステムを毎回起動するオーバーヘッドが大きい.
     \end{itemize}
    \begin{alertblock}{マルチショット符号化}
     \begin{itemize}
      \item $\varphi_{\ell}$を,$S(\boldsymbol{x}^{0})$を表す\code{base}部,
            $C(\boldsymbol{x}^{t})$,$T(\boldsymbol{x}^{t-1},\boldsymbol{x}^{t})$
            を表す\code{step(t)}部,
            $G(\boldsymbol{x}^{t})$を表す\code{check(t)}部に分けて記述.
     \end{itemize}
    \end{alertblock}
     \begin{itemize}
      \item % $S(\boldsymbol{x}^{0})$, $C(\boldsymbol{x}^{t})$,
            % $T(\boldsymbol{x}^{t-1},\boldsymbol{x}^{t})$, $G(\boldsymbol{x}^{t})$
            % を動的に追加・削除しながら,
            $\varphi_{\ell}$をインクリメンタルに構成しながら解くことが可能.
      \item 長所: 学習節の再利用が可能.ASPシステムの起動は1回のみ.
      \item 短所: 現状では,デバックしにくい.
     \end{itemize} 
\end{frame}
%%%%%%%%%%%%%%%%%%%%%%%%%%%%%%%%%%%%%%%%%%%%%%%%%%
%% 実験内容--遷移問題
%%%%%%%%%%%%%%%%%%%%%%%%%%%%%%%%%%%%%%%%%%%%%%%%%%
\begin{frame}{実験概要}
  \renewcommand{\thefootnote}{\fnsymbol{footnote}}
  \setcounter{footnote}{1}
 提案するASP符号化の性能の評価実験を行った.
  \vfill
  \begin{itemize}
  \item \structure{\bf 比較するASP符号化:}
    \begin{itemize}
    \item シングルショット符号化
    \item マルチショット符号化
    \end{itemize}
  \item \structure{\bf ベンチマーク問題:} 全1000問
    \begin{itemize}
    \item DNET \footnote{https://github.com/takemaru/dnet}
      で公開されている実用規模の配電網問題 (\structure{\bf fukui-tepco},
      スイッチ数 468,変電所の数 72,許容電流 300A)をベース
    \item 実行可能解の中から,スタート状態を10個,ゴール状態を100個を
          ランダムに抽出し,それらを組み合わせて生成
    \end{itemize}
  \item \structure{\bf ASPシステム:} \textit{clingo-5.4.0} $+$ \textit{trendy}
   \item \structure{\bf 制限時間:} 10分/問
  \item \structure{\bf 実験環境:} Mac mini,3.2GHz Intel Core i7,64GBメモリ
  \end{itemize}
\end{frame}
%%%%%%%%%%%%%%%%%%%%%%%%%%%%%%%%%%%%%%%%%%%%%%%%%%
%% 実験結果--遷移問題
%%%%%%%%%%%%%%%%%%%%%%%%%%%%%%%%%%%%%%%%%%%%%%%%%%
\begin{frame}{実験結果:平均CPU時間の比較}
 \vfill
 \centering
 \vskip -2ex
 \scalebox{0.8}{\begin{tabular}{ccrrr}
 \rowcolor[RGB]{0,96,0}
\color{white}最短ステップ長 & \color{white}問題数 
     & \multicolumn{1}{c}{\color{white}シングルショット} 
         & \multicolumn{1}{c}{\color{white}マルチショット} 
             & \multicolumn{1}{c}{\color{white}シングル/マルチ} \\
 \rowcolor[RGB]{230,239,230}
1 & 6 & 1.677 & 1.035 & 1.620 \\
 \rowcolor[RGB]{196,230,196}
2 & 62 & 3.507 & 1.608 & 2.180 \\
 \rowcolor[RGB]{230,239,230}
3 & 189 & 6.089 & 2.155 & 2.826 \\
 \rowcolor[RGB]{196,230,196}
4 & 312 & 9.294 & 2.734 & 3.399 \\
 \rowcolor[RGB]{230,239,230}
5 & 280 & 13.338 & 3.361 & 3.968 \\
 \rowcolor[RGB]{196,230,196}
6 & 130 & 18.303 & 4.165 & 4.394 \\
 \rowcolor[RGB]{230,239,230}
7 & 21 & 24.483 & 5.086 & 4.814 \\
\noalign{\hrule height 0.5pt}
 \rowcolor[RGB]{196,230,196}
計 & 1000 & 76.691 & 20.114 & 3.807 \\
\end{tabular}

}
 \vfill
\begin{itemize}
 \item 1000問全ての到達可能性を判定でき,全て到達可能であった.
 \item 今回生成した問題のうち,最長で最短ステップ数は7であった.
 \item マルチショットは,シングルショットと比較して,全ての
	   問題をより高速に解いており,\alert{\bf 平均で3.8倍の高速化}を実現している.
\end{itemize}
\end{frame}
%%%%%%%%%%%%%%%%%%%%%%%%%%%%%%%%%%%%%%%%%%%%%%%%%%
%% まとめ
%%%%%%%%%%%%%%%%%%%%%%%%%%%%%%%%%%%%%%%%%%%%%%%%%%
\begin{frame}{まとめと今後の課題}
 \begin{alertblock}{}
  \centering
  配電網遷移問題に対して,ASPを用いた解法を提案した.
 \end{alertblock}
 \vfill
  \begin{enumerate}
   \item \structure{\bf 配電網問題のASP符号化の考案}
         \begin{itemize}
          \item トポロジ制約のASP符号化として,\textbf{基本符号化},\textbf{改良符号化},
                \\ \alert{\bf 有向符号化} の3種類を考案.
          \item 電気制約として,電流制約のASP符号化を考案.
         \end{itemize}
   \item \structure{\bf 配電網遷移問題のASP符号化の考案}
         \begin{itemize}
          \item \textbf{シングルショット符号化}と\alert{\bf マルチショット符号化}の2種類を考案.
          \item 配電網問題のASP符号化の自然な拡張である.
         \end{itemize}
   \item \structure{\bf 実用規模の問題を含むベンチマークによる評価実験}
         \begin{itemize}
          \item 実用規模の問題である\textsf{fukui-tepco}をもとに,
                配電網遷移問題ベンチマークを1000問作成.
          \item \textbf{マルチショット符号化}は,
                \alert{\bf 約3.8倍の高速化}を実現.
         \end{itemize}
  \end{enumerate}
 \vfill
 \begin{exampleblock}{今後の課題}
\begin{itemize}
 \item 電流制約だけでなく電圧制約も含む配電網遷移問題への拡張
 \item 完全な問題は非線形な制約を含むため,ASP Modulo Theoriesを用いた解法を検討
\end{itemize}
 \end{exampleblock}
\end{frame}

%###########################################################
%##### 補助スライド ########################################
%###########################################################

%%%% 補助スライド
\appendix
\backupbegin

\begin{frame}{~}
 \centering
 - 補足用 -
\end{frame} 

%%%%%%%%%%%%%%%%%%%%%%%%%%%%%%%%%%%%%%%%%%%%%%%%%%
%% 電気制約
%%%%%%%%%%%%%%%%%%%%%%%%%%%%%%%%%%%%%%%%%%%%%%%%%%
\begin{frame}{補足 : 電気制約}
 \begin{itemize}
  \item \alert{電気制約}は,送電する電流$\cdot$電圧の適正範囲を保証する制約.
  \begin{itemize}
   \item 供給経路の各区間で許容電流を超えない.
   \item 電気抵抗による電圧降下が許容範囲を超えない.
   \item etc.
  \end{itemize}
  \item 電流と電圧が影響し合う\structure{実数ドメイン上の制約}によって表される.
		% \begin{itemize}
		%  		 \item 送電システム上の条件など.
		% \end{itemize}
  \item 実数ドメイン上の制約は,純粋なASPのみで扱うのは\alert{困難}.
		\begin{itemize}
		 \item 緩和問題として,変電所から供給できる家庭の数に上限をつける.
		 \item ASPMT技術により,ASPで得られた解について,
			   背景理論ソルバーと連携して実数ドメイン上の制約を調べる.
		\end{itemize}
 \end{itemize}
\end{frame}

%%%%%%%%%%%%%%%%%%%%%%%%%%%%%%%%%%%%%%%%%%%%%%%%%%
%% 基礎化
%%%%%%%%%%%%%%%%%%%%%%%%%%%%%%%%%%%%%%%%%%%%%%%%%%
\begin{frame}{補足 : ASPシステム}
 
 \vspace{-0.5cm}

 \begin{figure}[htbp]
  \centering
  %%%%%%%%%%%%%%%%%%%%%%%%%%%%%%%%%%%%%%%%%%%%%%%%%%
%% 基礎化の流れの図
%%%%%%%%%%%%%%%%%%%%%%%%%%%%%%%%%%%%%%%%%%%%%%%%%%
\begin{tikzpicture}

 \definecolor{edge}{RGB}{38,38,134}
 \definecolor{node}{RGB}{220,220,249}

 \definecolor{alert_edge}{RGB}{191,0,0}
 \definecolor{alert_node}{RGB}{249,200,200}

 \definecolor{ex_edge}{RGB}{0,96,0}
 \definecolor{ex_node}{RGB}{230,239,230}

 \def\nodespace{2.4cm}

 \tikzset{block/.style={rectangle, thick, draw=edge, fill=node, text width=3cm, 
 text centered, rounded corners, text width=2cm, minimum height=1.5cm}};

 \tikzset{alertblock/.style={rectangle, thick, draw=alert_edge, fill=alert_node, 
 text width=3cm, text centered, rounded corners, text width=1.5cm, minimum height=1.2cm}};

 \node[block](ikkai){一階ASP\\プログラム};

 \node[rectangle,rounded corners, thick, draw=ex_edge, fill=ex_node, 
 right=0.22*\nodespace of ikkai, minimum width=6cm, minimum height=3cm, 
 text centered, label=ASPシステム](sys){};

 \node[block, right=\nodespace of ikkai](meidai){命題ASP\\プログラム};
 \node[block, right=\nodespace of meidai](ASP){解集合};

 \node[right=0.6*\nodespace of ikkai, text width=1.5cm, 
 text centered, text=red, anchor=south](){基礎化\\ソルバー};
 \node[right=0.4*\nodespace of meidai, text width=1.5cm, 
 text centered, text=red, anchor=south](){解集合\\ソルバー};

 
 \foreach \u / \v / \n in {ikkai/meidai,meidai/ASP}
 \draw [thick,->] (\u) to (\v);

\end{tikzpicture}
 \end{figure}

 \vspace{-0.5cm}

 \begin{exampleblock}{}
  \begin{enumerate}
   \item 一階ASPプログラムを基礎化ソルバーによって,
		 命題ASPプログラムに\alert{基礎化}する.
   \item 命題ASPプログラムについて,SAT技術を応用した解集合ソルバーが解集合を探索する.
  \end{enumerate}
 \end{exampleblock}

\end{frame}
%%%%%%%%%%%%%%%%%%%%%%%%%%%%%%%%%%%%%%%%%%%%%%%%%%
%% ASPの構文
%%%%%%%%%%%%%%%%%%%%%%%%%%%%%%%%%%%%%%%%%%%%%%%%%%
\begin{frame}{ASPの構文}
  \begin{alertblock}{}\centering
    ASPの言語は論理プログラムをベースとしている~\footnotemark.
  \end{alertblock}
  \begin{itemize}
  \item \structure{\bf 論理プログラム}とは,以下の\structure{\bf ルール}の有限集合である.
    \begin{center}
      \begin{minipage}[c]{0.7\textwidth}
        \begin{block}{}\centering
          $a_0$\quad\code{:-}\quad$a_1$\code{,}\ldots\code{,}$a_m$\code{,}
          \ \code{not}~$a_{m+1}$\code{,}\ldots\code{,} \code{not}~$a_n$\code{.}
        \end{block}        
      \end{minipage}
   \end{center}\vfill
    $0 \leq m \leq n$ であり,各 $a_i$ はアトム,
    \code{not}は\structure{\bf デフォルトの否定},\\
    ``\code{,}''は連言(AND)を表す.``\code{:-}''の左辺を\structure{\bf ヘッド},
		右辺を\structure{\bf ボディ}と呼ぶ.
  \item \alert{\bf 直感的な意味}は,
    「$a_1,\ldots,a_m$がすべて成り立ち,
    $a_{m+1},\ldots,a_n$のそれぞれが成り立たないならば,
    $a_0$が成り立つ」である.
  \item ボディが空のルールを\structure{\bf ファクト}と呼び,``\code{:-}''は省略できる.
  \item ヘッドが空のルールを\structure{\bf 一貫性制約}と呼ぶ.例えば,\hspace{-1ex}
    ``\code{:-} $a_1$\code{,} \code{not}~$a_{2}$''は,
    「$a_1$が成り立つならば,$a_2$が成り立つ」を意味する.
  \end{itemize}
  \footnotetext{本発表では標準論理プログラムを単に論理プログラムと呼ぶ.}
\end{frame}
%%%%%%%%%%%%%%%%%%%%%%%%%%%%%%%%%%%%%%%%%%%%%%%%%%
%% ASPの拡張構文
%%%%%%%%%%%%%%%%%%%%%%%%%%%%%%%%%%%%%%%%%%%%%%%%%%
\begin{frame}{ASPの拡張構文}
\begin{alertblock}{}\centering
  組合せ問題を解くための便利な構文が用意されている.
\end{alertblock}
\begin{itemize}
 \item \structure{\bf 選択子}
   \begin{center}
     \code{\{}$a_1$\code{;}\ldots\code{;}$a_n$\code{\}}
   \end{center}
   アトム集合 $\{a_1,\dots,a_n\}$
   の任意の部分集合が成り立つことを意味する.
 \item \structure{\bf 個数制約}
   \begin{center}
     $lb$\ \code{\{}$a_1$\code{;}\ldots\code{;}$a_n$\code{\}}\ $ub$
   \end{center}
   $a_1,\dots,a_n$ のうち,
   $lb$個以上,$ub$個以下が成り立つことを意味する.
 \item \structure{\bf 重み付き個数制約}
   \begin{center}
     $lb$ \code{\#sum\{} $w_1$\code{:}$a_1$\code{;}\ldots\code{;}$w_n$\code{:}$a_n$ \code{\}} $ub$
   \end{center}
   $a_1,\dots,a_n$のうち,
   成り立つアトムの重み和が$lb$以上,$ub$以下になることを意味する.
\end{itemize}
\end{frame}
%%%%%%%%%%%%%%%%%%%%%%%%%%%%%%%%%%%%%%%%%%%%%%%%%%
%% 改良符号化 (到達可能性)
%%%%%%%%%%%%%%%%%%%%%%%%%%%%%%%%%%%%%%%%%%%%%%%%%%
\begin{frame}[fragile]{改良符号化: 到達可能性}
\begin{exampleblock}{}\small
\begin{lstlisting}
(1) { inForest(X,Y) } :- edge(X,Y).
\end{lstlisting}
\end{exampleblock}
\begin{itemize}
 \item (1) 各辺\code{(X,Y)について},根付き全域森に含まれること意味する \\
	  アトム\code{inForest(X,Y)}を導入する.
\end{itemize}
\begin{exampleblock}{}\small
\begin{lstlisting}
(2) reached(R,R) :- root(R).
(3) reached(X,R) :- reached(Y,R), inForest(Y,X).
(4) reached(X,R) :- reached(Y,R), inForest(X,Y).
\end{lstlisting}
\end{exampleblock}
\vfill
\begin{itemize}
\item アトム\code{reached(X,R)}は,ノード\code{X}が根ノード\code{R}から到達可能であることを意味する.
%\item (2) 各根ノード\code{R}について,自分自身から到達可能であることを表す.
\item (3) ノード\code{Y}が根ノード\code{R}から到達可能かつ,辺\code{(Y,X)}が根付き全域森に含まれるならば,
	  ノード\code{X}も同じ根ノード\code{R}から到達可能であることを表す.
\end{itemize}
\end{frame}
%%%%%%%%%%%%%%%%%%%%%%%%%%%%%%%%%%%%%%%%%%%%%%%%%%
%% 改良符号化 (根付き連結制約)
%%%%%%%%%%%%%%%%%%%%%%%%%%%%%%%%%%%%%%%%%%%%%%%%%%
\begin{frame}[fragile]{改良符号化: 根付き連結制約}
\begin{exampleblock}{}\small
\begin{lstlisting}
(5) :- node(X), not 1 { reached(X,R) } 1.
\end{lstlisting}
\end{exampleblock}
\vfill
\begin{itemize}
\item (5) 各ノード\code{X}について,ちょうど1つの根からのみ到達可能であることを意味する.
\end{itemize}
\end{frame}
%%%%%%%%%%%%%%%%%%%%%%%%%%%%%%%%%%%%%%%%%%%%%%%%%%
%% 改良符号化 (非閉路制約)
%%%%%%%%%%%%%%%%%%%%%%%%%%%%%%%%%%%%%%%%%%%%%%%%%%
\begin{frame}[fragile]{改良符号化: 非閉路制約}
\begin{minipage}[c]{1.01\textwidth}
\begin{exampleblock}{}\small
\begin{lstlisting}
(6) :- root(R),
       not 1 #sum{ 1,X:reached(X,R) ;
                  -1,X,Y:inForest(X,Y),reached(X,R),reached(Y,R)
                 } 1.
\end{lstlisting}
\end{exampleblock}
\end{minipage}
\vfill
\begin{itemize}
\item (6) 各連結成分の\structure{\bf ノード数と辺数の差が1}になることを意味する.
\item 各連結成分が\structure{\bf 木の性質}を満たすことにより,サイクルを持たない
	  ことを保証する.
\end{itemize}
\end{frame}
%%%%%%%%%%%%%%%%%%%%%%%%%%%%%%%%%%%%%%%%%%%%%%%%%%
%% ルール数の比較
%%%%%%%%%%%%%%%%%%%%%%%%%%%%%%%%%%%%%%%%%%%%%%%%%%
\begin{frame}{基礎化後のルール数}
  \begin{itemize}
  \item グラフのノード数を$|V|$,根ノードの数を$|R|$とする.
  \end{itemize}
  \begin{table}[t]
    \centering
    %%%%%%%%%%%%%%%%%%%%%%%%%%%%%%%%%%%%%%%%%%%%%%%%%%%%%%%%%%%%%%%%
\chapter{ハミルトン閉路問題および関連問題のASP符号化}\label{chap:proposal}
%%%%%%%%%%%%%%%%%%%%%%%%%%%%%%%%%%%%%%%%%%%%%%%%%%%%%%%%%%%%%%%% 

%%%%
\begin{figure}[h]
  \centering
  \thicklines
  \setlength{\unitlength}{1.2pt}
  \small\footnotesize\scriptsize
  \begin{picture}(280,57)(4,-10)
    \put(  0, 20){\dashbox(50,24){\shortstack{HCP問題\\インスタンス}}}
    \put( 60, 20){\framebox(50,24){変換器}}
    \put(120, 20){\dashbox(50,24){\shortstack{ASPファクト}}}
    \put(120,-10){\dashbox(50,24){\shortstack{ASP符号化\\(論理プログラム)}}}
    \put(180, 20){\framebox(50,24){ASPシステム}}
    \put(240, 20){\dashbox(50,24){\shortstack{HCP問題\\の解}}}
    \put( 50, 32){\vector(1,0){10}}
    \put(110, 32){\vector(1,0){10}}
    \put(170, 32){\vector(1,0){10}}
    \put(230, 32){\vector(1,0){10}}
    \put(170, +2){\line(1,0){4}}
    \put(174, +2){\line(0,1){30}}
  \end{picture}  
\caption{ASP を用いたハミルトン閉路問題(HCP)の解法}
\label{fig:arch}
\end{figure}
%%%%

%\begin{figure}[tbp]
\tikz{
  %1ノード目
  \path[draw=black, fill=blue!20, rounded corners=5pt]%線の設定
  node[at={(0.75,0.75)}] {問題}%文字を入れる
  (0,0) --(1.5,0) --(1.5,1.5) --(0,1.5) --cycle;%外周
  %2ノード目
  \path[draw=black, fill=blue!20, rounded corners=5pt, shift={(3,0)}]
  node[at={(0.75,0.75)}] {
    \begin{tabular}{c}
      ASP\\
      ファクト
    \end{tabular}
  }
  (0,0) --(1.5,0) --(1.5,1.5) --(0,1.5) --cycle;
  %3ノード目文字が複数行
  \path[draw=black, fill=green!20, rounded corners=5pt, shift={(6,0)}]
  node[at={(0.75,0.75)}] {
    \begin{tabular}{c}
      ASP\\
      システム
    \end{tabular}
  }
  (0,0) --(1.5,0) --(1.5,1.5) --(0,1.5) --cycle;
  %4ノード目文字が複数行
  \path[draw=black, fill=blue!20, rounded corners=5pt, shift={(9,0)}]
  node[at={(0.75,0.75)}] {解集合}
  (0,0) --(1.5,0) --(1.5,1.5) --(0,1.5) --cycle;
  %5ノード目文字が複数行
  \path[draw=black, fill=red!20, rounded corners=5pt, shift={(3,-3)}]
  node[at={(0.75,0.75)}] {
    \begin{tabular}{c}
      ASP\\
      符号化
    \end{tabular}
  }
  (0,0) --(1.5,0) --(1.5,1.5) --(0,1.5) --cycle;
  \draw[arrows=->] (1.5,0.75) --(3.0,0.75);
  \draw[arrows=->,shift={(3,0)}] (1.5,0.75) --(3.0,0.75);
  \draw[arrows=->,shift={(6,0)}] (1.5,0.75) --(3.0,0.75);
  \draw[arrows=->] (4.5,-2.25) --(6.0,0.5);
}
\caption{ASPを用いた解法}
\label{aspmethod}
\end{figure}


ASP を用いたハミルトン閉路問題および関連問題の解法について述べる.
図~\ref{fig:arch}に,解法の流れを示す.
与えられたハミルトン閉路問題は ASP ファクトに変換され,
ハミルトン閉路問題を解く ASP 符号化と結合され,
ASP システムによって解が計算される.
本論文では,ASP システムとして{\clingo}を用いる.

%%%%%%%%%%%%%%%%%%%%%%%%%%%%%%%%%%%%%%%%%%%%%%%%%%%%%%%%%%%%%%%%%%%%%%%
\section{ASPファクト形式}
%%%%%%%%%%%%%%%%%%%%%%%%%%%%%%%%%%%%%%%%%%%%%%%%%%%%%%%%%%%%%%%%%%%%%%%

%%%%%%%%%%%%%%%%%%%%%%%%%%%%%%
\begin{figure}[t]
\begin{center}
\begin{tikzpicture}
  %ノード1  
  \draw(4,2) circle (0.5)
  node[at={(4.1,2.1)}] {
    \begin{tabular}{c}
      1
    \end{tabular}
  };
  %ノード2  
  \draw(4,0) circle (0.5)
  node[at={(4.1,0.1)}] {
    \begin{tabular}{c}
      2
    \end{tabular}
  };
  %ノード3  
  \draw(6,2) circle (0.5)
  node[at={(6.1,2.1)}] {
    \begin{tabular}{c}
      3
    \end{tabular}
  };
  %ノード4  
  \draw(6,0) circle (0.5)
  node[at={(6.1,0.1)}] {
    \begin{tabular}{c}
      4
    \end{tabular}
  };
  %ノード5  
  \draw(8,2) circle (0.5)
  node[at={(8.1,2.1)}] {
    \begin{tabular}{c}
      5
    \end{tabular}
  };
  %ノード6  
  \draw(8,0) circle (0.5)
  node[at={(8.1,0.1)}] {
    \begin{tabular}{c}
      6
    \end{tabular}
  };
\draw(4,0.5) --(4,1.5);
\draw(6,0.5) --(6,1.5);
\draw(8,0.5) --(8,1.5);
\draw(4.5,0) --(5.5,0);
\draw(4.5,2) --(5.5,2);
\draw(6.5,0) --(7.5,0);
\draw(6.5,2) --(7.5,2);
\end{tikzpicture}

\caption{入力となる重み付き無向グラフの例}
\label{graphexample}
\end{center}
\end{figure}
%%%%%%%%%%%%%%%%%%%%%%%%%%%%%%

%%%%%%%%%%%%%%%%%%%%%%%%%%%%%%
\lstinputlisting[float=t,caption={%
図~\ref{graphexample}のASPファクト表現},%
captionpos=b,frame=single,label=code:graph_example.lp,%
numbers=none,%
breaklines=true,%
columns=fullflexible,keepspaces=true,%
basicstyle=\ttfamily\scriptsize]{code/graph_example.lp}
%%%%%%%%%%%%%%%%%%%%%%%%%%%%%%


本節では,最短ハミルトン閉路問題の例にとって,
入力となる重み付き無向グラフ(図~\ref{graphexample})の
ASP ファクト形式について説明する.
%
このグラフは,頂点数が6,辺の数が7であり,辺に付けられた値は距離を表す.
コード~\ref{code:graph_example.lp}に,ASPファクト形式を示す.
%
アトム\code{node/1}は頂点,\code{edge/2}は辺,\code{cost/3}は距離を表す.
例えば,\code{cost(1,2,3)}は,辺\code{edge(1,2)}の距離が3であることを
表している.

%%%%%%%%%%%%%%%%%%%%%%%%%%%%%%%%%%%%%%%%%%%%%%%%%%%%%%%%%%%%%%%%%%%%%%%
\section{ハミルトン閉路問題の ASP 符号化}\label{hamiltonianasp}
%%%%%%%%%%%%%%%%%%%%%%%%%%%%%%%%%%%%%%%%%%%%%%%%%%%%%%%%%%%%%%%%%%%%%%%

ハミルトン閉路問題は,与えられたグラフの全頂点をちょうど一度ずつ通る閉
路(ハミルトン閉路)が存在するかどうかを判定する問題である.
$G=(V,E)$にハミルトン閉路が存在する必要十分条件は,
以下の2つの制約を満たす部分グラフ$G'=(V,E')$が存在することである.

\begin{itemize}
\item $G'$の各頂点の次数が2 (次数制約)
\item $G'$が連結である (連結制約)
\end{itemize}

本論文では,前者を\textbf{次数制約},後者を\textbf{連結制約}と呼ぶ.
ハミルトン路問題は,ハミルトン閉路問題から始点と終点が一致するという閉
路の条件を取り除いたものである.
ハミルトン路問題では,次数制約は以下のように変わる.

\begin{itemize}
\item 始点と終点の次数が1,他の頂点の次数が2
\end{itemize}

以下では,ハミルトン閉路問題に対する3つの ASP 符号化
\textsf{undirected},\textsf{directed},\textsf{acyclicity}
を提案する.

%%%%%%%%%%%%%%%%%%%%%%%%%%%%%%%%%%%%%%%%%%%%%%%%%%%%%%%%%%%%%%%%%%%%%%%
\subsection{\textsf{undirected}符号化}
%%%%%%%%%%%%%%%%%%%%%%%%%%%%%%%%%%%%%%%%%%%%%%%%%%%%%%%%%%%%%%%%%%%%%%%

%%%%%%%%%%%%%%%%%%%%%%%%%%%%%%
\lstinputlisting[float=t,caption={%
\textsf{undirected}符号化},%
captionpos=b,frame=single,label=code:hamilton1.lp,%
numbers=left,%
breaklines=true,%
columns=fullflexible,keepspaces=true,%
basicstyle=\ttfamily\footnotesize]{code/hamilton1.lp}
%%%%%%%%%%%%%%%%%%%%%%%%%%%%%%

\textsf{undirected}符号化は,ハミルトン閉路問題の次数制約と連結制約を,
ASP の一貫性制約で表した基本的な符号化である.
コード~\ref{code:hamilton1.lp}に,\textsf{undirected}符号化を示す.
この符号化は,ハミルトン閉路問題とハミルトン路問題の両方に対応している.
符号化中の\code{s}は始点の頂点番号,\code{t}は終点の頂点番号を表し,こ
れらは実行時に与えられる.
ここでは,ハミルトン閉路問題(\code{s}=\code{t})の場合について説明する.

\begin{itemize}
\item 1行目のルールは,各辺\code{edge(X,Y)}に対して,その辺がハミルト
  ン閉路に含まれるかどうかを意味するアトム\code{in(X,Y)}を選択子を用い
  て導入している.
\item 次数制約は3行目のルールで表される.このルールは,
  各頂点\code{node(X)}に対して,その次数の和が2に等しいことを個数制約
  を使って表している.
\item 連結制約は11行目のルールで表される.
ある頂点\code{X}が始点\code{s}から到達可能であることを意味する補助アト
ム\code{reached(X)}を導入する.
8行目のルールは,始点\code{s}が到達可能あることを表している.
9行目のルールは,各辺\code{X}--\code{Y}に対して,その辺がハミルトン閉
路に含まれ(\code{in(X,Y)}),かつ,頂点\code{X}が始点から到
達可能であれば(\code{reached(X)}),\code{Y}も到達可能であることを表している.
10行目は9行目と同様であるが,辺\code{Y}--\code{X}の場合を表している.
11行目のルールは,各頂点\code{node(X)}が始点から到達可能でなければな
らないことを一貫性制約を使って表している.
\end{itemize}

%%%%%%%%%%%%%%%%%%%%%%%%%%%%%%%%%%%%%%%%%%%%%%%%%%%%%%%%%%%%%%%%%%%%%%%
\subsection{\textsf{directed}符号化}
%%%%%%%%%%%%%%%%%%%%%%%%%%%%%%%%%%%%%%%%%%%%%%%%%%%%%%%%%%%%%%%%%%%%%%%

%%%%%%%%%%%%%%%%%%%%%%%%%%%%%%
\lstinputlisting[float=t,caption={%
\textsf{directed}符号化},%
captionpos=b,frame=single,label=code:hamilton2.lp,%
numbers=left,%
breaklines=true,%
columns=fullflexible,keepspaces=true,%
basicstyle=\ttfamily\footnotesize]{code/hamilton2.lp}
%%%%%%%%%%%%%%%%%%%%%%%%%%%%%%

\textsf{directed}符号化は,\textsf{undirected}符号化をベースに,
与えられた無向グラフの各辺$u-v$に対して,2つの弧$u\rightarrow v$と
$v\rightarrow u$を対応させることで有向グラフ化して解く符号化である.
コード~\ref{code:hamilton2.lp}に,\textsf{directed}符号化を示す.
前節と同様に,ハミルトン閉路問題(\code{s}=\code{t})の場合について説明する.

\begin{itemize}
\item 1行目では,無向グラフの有向グラフ化を行う.
  与えられた無向グラフの各辺\code{edge(X,Y)}に対して,
  2つの弧\code{edge(X,Y)},\code{edge(Y,X)}を導入した.
\item 2行目のルールは,各弧\code{edge(X,Y)}に対して,その弧がハミルト
  ン閉路に含まれるかどうかを意味するアトム\code{in(X,Y)}を選択子を用い
  て導入している.
\item 次数制約は4,5行目のルールで表される.
  4行目では,各頂点\code{node(X)}に対して,
  その出次数が1に等しいことを個数制約を使って表している.
  5行目では,入次数について4行目と同様の制約を表す.
\item 連結制約は15行目のルールで表される.
  ある頂点\code{X}が始点\code{s}から到達可能であることを意味する
  補助アトム\code{reached(X)}を導入する.
  13行目のルールは,始点\code{s}が到達可能あることを表している.
  14行目のルールは,各弧\code{X}--\code{Y}に対して,その弧がハミルトン閉路
  に含まれ(\code{in(X,Y)}),かつ,頂点\code{X}が始点から
  到達可能であれば(\code{reached(X)}),\code{Y}も到達可能であることを表している.
  15行目のルールは,各頂点\code{node(X)}が始点から到達可能でなければ
  ならないことを一貫性制約を使って表している.
\item 18行目のルールは,解についての対称性を除去する.
  与えられた無向グラフ上の各ハミルトン閉路に対して,
  それを変換した有向グラフ上のハミルトン閉路は対称な2つが存在する.
  これによる解の重複を防ぐために,18行目のルールは,各弧\code{s}--\code{X},
  \code{Y}--\code{s}がハミルトン閉路に含まれるならば(\code{in(s,X),in(Y,s)}),
  \code{X < Y}でなければならないことを,一貫性制約を用いて表している
\end{itemize}

%%%%%%%%%%%%%%%%%%%%%%%%%%%%%%%%%%%%%%%%%%%%%%%%%%%%%%%%%%%%%%%%%%%%%%%
\subsection{\textsf{acyclicity}符号化}
%%%%%%%%%%%%%%%%%%%%%%%%%%%%%%%%%%%%%%%%%%%%%%%%%%%%%%%%%%%%%%%%%%%%%%%

%%%%%%%%%%%%%%%%%%%%%%%%%%%%%%
\lstinputlisting[float=t,caption={%
\textsf{acyclicity}符号化},%
captionpos=b,frame=single,label=code:hamilton3.lp,%
numbers=left,%
breaklines=true,%
columns=fullflexible,keepspaces=true,%
basicstyle=\ttfamily\footnotesize]{code/hamilton3.lp}
%%%%%%%%%%%%%%%%%%%%%%%%%%%%%%

\textsf{acyclicity}符号化は,\textsf{directed}符号化をベースに,
連結の制約に代わる部分閉路禁止制約を組込み非閉路制約で表現した符号化である.
コード~\ref{code:hamilton3.lp}に,\textsf{acyclicity}符号化を示す.
前節と同様に,ハミルトン閉路問題(\code{s}=\code{t})の場合について説明する.

\begin{itemize}
\item 1行目では,無向グラフの有向グラフ化を行う.
  与えられた無向グラフの各辺\code{edge(X,Y)}に対して,
  2つの弧\code{edge(X,Y)},\code{edge(Y,X)}を導入した.
\item 2行目のルールは,各弧\code{edge(X,Y)}に対して,その弧がハミルト
  ン閉路に含まれるかどうかを意味するアトム\code{in(X,Y)}を選択子を用い
  て導入している.
\item 次数制約は4,5行目のルールで表される.
  4行目では,各頂点\code{node(X)}に対して,
  その出次数が1に等しいことを個数制約を使って表している.
  5行目では,入次数について4行目と同様の制約を表す.
\item 部分閉路禁止制約は14行目のルールで表される.
  このルールは,始点でない各頂点\code{X},\code{Y}について,
  弧\code{X}--\code{Y}がハミルトン閉路に含まれるならば(\code{in(X,Y)}),
  そのような弧の集合をもつグラフが閉路をもたないことを,\code{#edge}宣言を用いて表す.
  ようするに,始点(終点)を含まないような閉路を禁止している.
\item 17行目のルールは,解についての対称性を除去する.
  与えられた無向グラフ上の各ハミルトン閉路に対して,
  それを変換した有向グラフ上のハミルトン閉路は対称な2つが存在する.
  これによる解の重複を防ぐために,17行目のルールは,各弧\code{s}--\code{X},
  \code{Y}--\code{s}がハミルトン閉路に含まれるならば(\code{in(s,X),in(Y,s)}),
  \code{X < Y}でなければならないことを,一貫性制約を用いて表している
\end{itemize}

%%%%%%%%%%%%%%%%%%%%%%%%%%%%%%%%%%%%%%%%%%%%%%%%%%%%%%%%%%%%%%%%%%%%%%% 
\section{最短ハミルトン閉路問題のASP符号化}\label{minexpl}
%%%%%%%%%%%%%%%%%%%%%%%%%%%%%%%%%%%%%%%%%%%%%%%%%%%%%%%%%%%%%%%%%%%%%%% 

%% %%%%%%%%%%%%%%%%%%%%%%%%%%%%%%
%% \lstinputlisting[caption =  最適化,label = minimize]{code/obj_minimize.lp}
%% %%%%%%%%%%%%%%%%%%%%%%%%%%%%%%

%%%%%%%%%%%%%%%%%%%%%%%%%%%%%%
\lstinputlisting[float=t,caption={%
最小化},%
captionpos=b,frame=single,label=code:obj_minimize.lp,%
numbers=left,%
breaklines=true,%
columns=fullflexible,keepspaces=true,%
basicstyle=\ttfamily\footnotesize]{code/obj_minimize.lp}
%%%%%%%%%%%%%%%%%%%%%%%%%%%%%%

最短ハミルトン閉路問題の目的関数は,
ハミルトン閉路を構成する各辺の距離の総和である.
コード\ref{code:obj_minimize.lp}は,
その目的関数の最小化を表す.
このコードは,各辺\code{edge(X,Y)}に対して,その辺がハミルトン閉路に
含まれ(\code{in(X,Y)}),その距離が\code{C}である時に(\code{cost(X,Y,C)}),
\code{C}の総和の最小化を,最小化関数を用いて表している.
.
%%%%%%%%%%%%%%%%%%%%%%%%%%%%%%
\lstinputlisting[float=t,caption={%
重み付き無向グラフの有向グラフ化},%
captionpos=b,frame=single,label=code:cost_both.lp,%
numbers=left,%
breaklines=true,%
columns=fullflexible,keepspaces=true,%
basicstyle=\ttfamily\footnotesize]{code/cost_both.lp}
%%%%%%%%%%%%%%%%%%%%%%%%%%%%%%

符号化directed,acyclicityについては,
与えられた無向グラフの各辺\code{edge(X,Y)}に対して,
2つの弧\code{edge(X,Y)},\code{edge(Y,X)}を導入した.
各辺の距離もこれに対応させるために,コード\ref{code:cost_both.lp}
を追加した.
このルールは,各辺\code{X}--\code{Y}の距離を表す\code{cost(X,Y,C)}について,
\code{cost(Y,X,C)}を導入する.
これにより,与えられた無向グラフの各辺\code{edge(X,Y)}の重み\code{C}が
2つの弧\code{edge(X,Y)},\code{edge(Y,X)}にも付与された.

%%%%%%%%%%%%%%%%%%%%%%%%%%%%%%%%%%%%%%%%%%%%%%%%%%%%%%%%%%%%%%%%%%%%%%% 
\section{コスト制約付きハミルトン閉路のASP符号化}
%%%%%%%%%%%%%%%%%%%%%%%%%%%%%%%%%%%%%%%%%%%%%%%%%%%%%%%%%%%%%%%%%%%%%%% 

%%%%%%%%%%%%%%%%%%%%%%%%%%%%%%
\lstinputlisting[float=t,caption={%
コスト制約},%
captionpos=b,frame=single,label=code:cost_constraint.lp,%
numbers=left,%
breaklines=true,%
columns=fullflexible,keepspaces=true,%
basicstyle=\ttfamily\footnotesize]{code/cost_constraint.lp}
%%%%%%%%%%%%%%%%%%%%%%%%%%%%%%

コスト制約付きハミルトン閉路問題は
ハミルトン閉路問題に,距離の総和が所与の閾値以下 (または以上) であること
を制約条件として付加した問題である.
コード\ref{code:const_constraing.lp}のルールは,その制約を表す.
ルール中の\code{c}は閾値を表し,これは実行時に与えられる.
このルールは,各辺\code{edge(X,Y)}に対して,その辺がハミルトン閉路に
含まれ(\code{in(X,Y)}),その距離が\code{C}である時に(\code{cost(X,Y,C)}),
\code{C}の総和が\code{c}以下でなければならないことを,
一貫性制約と重み付き個数制約を用いて表す.

また,\ref{minexpl}と同様に,
符号化directed,acyclicityについては,
アトム\code{cost}についても有向グラフ化に
対応させるためにコード\ref{code:cost_both.lp}を追加した.
%%%%%%%%%%%%%%%%%%%%%%%%%%%%%%%%%%%%%%%%%%%%%%%%%%%%%%%%%%%%%%%%%%%%%%%

%%% Local Variables:
%%% mode: latex
%%% TeX-master: "paper"
%%% End:

  \end{table}
\end{frame}

%%%%%%%%%%%%%%%%%%%%%%%%%%%%%%%%%%%%%%%%%%%%%%%%%%
%% ASPのコード
%%%%%%%%%%%%%%%%%%%%%%%%%%%%%%%%%%%%%%%%%%%%%%%%%%
\begin{frame}[fragile]{補足 : 根付き全域森 基本符号化}
%%%%%%%%%%%%%%%%%%%%%%%%%%%%%%%%% 
\lstinputlisting[frame=single,label=code:roop,%
xleftmargin=1zw,%
xrightmargin=1zw,%
numbersep=5pt,%
numbers=left,%
breaklines=true,%
columns=fullflexible,keepspaces=true,%
basicstyle=\ttfamily\scriptsize]{code/srf1.lp}
%%%%%%%%%%%%%%%%%%%%%%%%%%%%%%%%%
\end{frame}

\begin{frame}[fragile]{補足 : 遷移問題 シングルショット符号化}

\begin{multicols*}{2}
%%%%%%%%%%%%%%%%%%%%%%%%%%%%%%%%% 
\lstinputlisting[frame=single,label=code:roop,%
xleftmargin=1zw,%
xrightmargin=1zw,%
numbersep=5pt,%
numbers=left,%
breaklines=true,%
columns=fullflexible,keepspaces=true,%
basicstyle=\ttfamily\tiny]{code/trans-const.lp}
%%%%%%%%%%%%%%%%%%%%%%%%%%%%%%%%%
\end{multicols*}
\end{frame}

\begin{frame}[fragile]{補足 : 遷移問題 マルチショット符号化}
\begin{multicols*}{2}
%%%%%%%%%%%%%%%%%%%%%%%%%%%%%%%%%
\lstinputlisting[frame=single,label=code:incmode,% 
xleftmargin=1zw,%
xrightmargin=1zw,%
numbersep=5pt,%
numbers=left,%
breaklines=true,%
columns=fullflexible,keepspaces=true,%
basicstyle=\ttfamily\tiny]{code/dnet-trans.lp}
%%%%%%%%%%%%%%%%%%%%%%%%%%%%%%%%% 
\end{multicols*}
\end{frame}

\backupend


\end{document}
%%% Local Variables:
%%% mode: japanese-latex
%%% TeX-master: t
%%% End:

