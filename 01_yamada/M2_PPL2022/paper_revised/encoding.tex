\section{配電網問題のASP符号化}\label{chap:encode}

第~\ref{chap:intro}節の図~\ref{fig:arch}に示したように,
提案解法では,まず与えられた問題インスタンスを ASP のファクト形式に変
換した後,そのファクトと配電網問題を解くための ASP 符号化を結合した上
で,高速 ASP システム{\clingo}を用いて解を求める.
%
本節では,まず\ref{chap:fact}節で配電網問題インスタンスのファクト形式
について述べる.
\ref{chap:prepro}節で補助アトムを導入する.
\ref{chap:topology}節でトポロジ制約のASP符号化として,
基本符号化,改良符号化,有向符号化の3つを提案する.
最後に,\ref{chap:electrical}節で電流制約のASP符号化を示す.

%%%%%%%%%%%%%%%%%%%%%%%%%%%%%%%%%%%%%%%%%%%%%%%%%%%%%%%%%%%%%%%%%
\subsection{配電網問題インスタンスのファクト形式}\label{chap:fact}

%%%%%%%%%%%%%%%%%%%%%%%%%%%%%%%%%
\lstinputlisting[float=tb,caption={%
  配電網問題(図~\ref{fig:test-input})のファクト表現},%
captionpos=b,frame=single,label=code:test.lp,%
xrightmargin=1zw,% 
xleftmargin=1zw,% 
numbersep=5pt,%
numbers=none,%
breaklines=true,%
columns=fullflexible,keepspaces=true,%
basicstyle=\ttfamily\scriptsize]{code/test.lp}
%%%%%%%%%%%%%%%%%%%%%%%%%%%%%%%%%

図~\ref{fig:test-input}で示した
配電網問題の例の ASP ファクト表現をコード\ref{code:test.lp}に示す.
このファクト表現は
DNETに公開されているYAML形式の問題インスタンスを,ほぼ一対一に ASP の
ファクトに変換したものである.
%
アトム\code{dnet_node}は,図~\ref{fig:test-input}の四角ノード($\Box$)
に対応し,セクションとスイッチの結合関係を表している.
例えば,1行目の
\code{dnet_node(1, ("s_a"; "s_1"; "s_15"))}は,
赤色の四角ノードに対応し,
3つのセクション
\code{s_a}, \code{s_1}, \code{s_15}
が結合していることを表す.
アトム中のセミコロン(\code{;})は ASP の略記法であり,実行時には
\code{dnet_node(1, "s_a")},
\code{dnet_node(1, "s_1")},
\code{dnet_node(1, "s_15")}
の3つのアトムに展開される.
%
4行目の\code{dnet_node(4, ("s_1"; "sw_1"))}は,
セクション\code{s_1}とスイッチ\code{sw_1}
が結合していることを表す.
%
アトム\code{load/2}は,各セクションに与えられる電流の値を表す.例えば,
\code{load("s_a", 16)}は,セクション\code{s_a}は$16$Aの電流をもつことを意味する.
%
アトム\code{substation/1}は変電所を,アトム\code{switch/1}はスイッチをそれぞれ
表している.


%%%%%%%%%%%%%%%%%%%%%%%%%%%%%%%%%
\subsection{補助アトムの導入}\label{chap:prepro}

%%%%%%%%%%%%%%%%%%%%%%%%%%%%%%%%%
\lstinputlisting[float=tb,caption={%
  補助アトムの導入},%
captionpos=b,frame=single,label=code:prepro.lp,%
xrightmargin=1zw,% 
xleftmargin=1zw,% 
numbersep=5pt,%
numbers=left,%
breaklines=true,%
columns=fullflexible,keepspaces=true,%
basicstyle=\ttfamily\scriptsize]{code/prepro.lp}
%%%%%%%%%%%%%%%%%%%%%%%%%%%%%%%%%

配電網(遷移)問題の制約を簡潔に記述するために,
コード\ref{code:prepro.lp}のような補助アトムを導入する.
\begin{itemize}
\item 1行目の\code{swt_node(X)}は,ノード\code{X}がスイッチをも
  つことを意味する.
\item 2行目の\code{jct_node(X)}は,
  ノード\code{X}がスイッチをもたないことを意味する.
  この特殊なノードをジャンクションと呼ぶ.
  例えば,図~\ref{fig:test-input}の色の付いた四角ノードは
  ジャンクションである.
\item 4行目の\code{section(S,X)}は,
  セクション\code{S}がノード\code{X}に属することを意味する.
  5行目の\code{section(S)}は,\code{S}がセクションであることを意味する.
\item 8行目の\code{jct_section(S)}は,セクション\code{S}がジャンクショ
  ンに属することを意味する.
  逆に,7行目の\code{swt_section(S)}は,
  セクション\code{S}が
  ジャンクションではないスイッチを含むノードに属することを意味する.
\item 10行目の\code{switch(SW,S,T)}は,
  セクション\code{S}と\code{T}の間にスイッチ\code{SW}があることを意味
  する.
\item 17--19行目の
\code{root/1},
\code{node/1},
\code{edge/2}
は,根付き全域森(トポロジ制約を満たす配電網構成)を求めるための入力グラ
フを表すアトムであり,それぞれ,根ノード,ノード,辺を表す.
% 14行目からのルールは
% トポロジ制約(根付き全域森問題)のASP符号化の入力
% となるアトムを生成するためのルールである.
% 14,15行目のルールは,根付き全域森問題のノードとセクションの
% 対応関係を表すアトム\code{node/2}を導入する.
% アトム\code{node(X,S)}は,セクション\code{S}は根付き全域森問題の
% ノード\code{X}に対応することを意味する.
% 14行目のルールは,各ジャンクションノード\code{X}について,それに
% 含まれるセクション\code{S}はそのまま対応することを表す.
% \item 
% 15行目のルールでは,スイッチノードにのみ含まれるセクション\code{S}
% について,ノード番号の小さいものに対応することを意味する.
% \item 
% 18行目からのアトム\code{root/1}は根ノードを,アトム\code{node/1}はノードを,
% アトム\code{edge/2}は辺をそれぞれ表している.
% 18行目のルールは,各ノード\code{X}について,それに属する\code{S}が
% 変電所であるならば,\code{X}が根ノードであることを意味する.
% \item 
% 19行目のルールは,\code{node/2}から\code{node/1}が成り立つ.
% \item 

% 20行目のルールは,ノード\code{X}に含まれるセクション\code{S}と,
% ノード\code{Y}に含まれるセクション\code{T}が,あるスイッチ\code{SW}
% で結合するならば,それら2つのノードは辺で結ばれることを意味する.
\end{itemize}


\subsection{トポロジ制約のASP符号化}\label{chap:topology}

%%%%%%%%%%%%%%%%%%%%%%%%%%%%%%%%%
\lstinputlisting[float=tb,caption={%
  トポロジ制約の基本符号化},%
captionpos=b,frame=single,label=code:srf1.lp,%
xrightmargin=1zw,% 
xleftmargin=1zw,% 
numbersep=5pt,%
numbers=left,%
breaklines=true,%
columns=fullflexible,keepspaces=true,%
basicstyle=\ttfamily\scriptsize]{code/srf1.lp}
%%%%%%%%%%%%%%%%%%%%%%%%%%%%%%%%%
\lstinputlisting[float=tb,caption={%
  トポロジ制約の改良符号化},%
captionpos=b,frame=single,label=code:srf2.lp,%
xrightmargin=1zw,% 
xleftmargin=1zw,% 
numbersep=5pt,%
numbers=left,%
breaklines=true,%
columns=fullflexible,keepspaces=true,%
basicstyle=\ttfamily\scriptsize]{code/srf2.lp}
%%%%%%%%%%%%%%%%%%%%%%%%%%%%%%%%%
\lstinputlisting[float=tb,caption={%
  トポロジ制約の有向符号化},%
captionpos=b,frame=single,label=code:srf3.lp,%
xrightmargin=1zw,% 
xleftmargin=1zw,% 
numbersep=5pt,%
numbers=left,%
breaklines=true,%
columns=fullflexible,keepspaces=true,%
basicstyle=\ttfamily\scriptsize]{code/srf3.lp}
%%%%%%%%%%%%%%%%%%%%%%%%%%%%%%%%%

第~\ref{chap:problem}節で述べたように,
トポロジ制約を満たす配電網構成は,配電網に対応するグラフから,
根付き全域森を求める問題に帰着される.

\textbf{基本符号化}をコード\ref{code:srf1.lp}に示す.
2行目のルールは,各辺(\code{X},\code{Y})に対して,
解の候補となる\code{inForest(X,Y)}を導入する.
この\code{inForest(X,Y)}は,
辺(\code{X},\code{Y})が根付き全域森に含まれることを意味する.
%
各ノードの到達可能性は5--7行目のルールで表される.
\code{reached(X,R)}は,ノード\code{X}は根
\code{R}から到達可能であることを意味する.
5行目のルールは,各根ノードは自分自身から到達可能であることを表す.
6--7行目のルールは,ノード\code{Y}が根ノード\code{R}から到達可能であり,
かつ,辺(\code{X},\code{Y})が根付き全域森に含まれるならば,
ノード\code{X}も\code{R}から到達可能であることを表す.

非閉路制約は10--13行目のルールで表される.
このルールは,各連結成分のノード数と辺数の差が1になること(木の性質)を,
ASP の重み付き個数制約を使って表している.
%
根付き連結制約は16--17行目のルールで表される.
16行目のルールは,各ノードは少なくとも1つの根から到達可能であることを
表している(\textit{at-least-one}制約).
17行目のルールは,各ノードは高々1つの根から到達可能である
ことを表している(\textit{at-most-one}制約).
これら2つの一貫性制約により,各ノードはちょうど1つの根から到達可能であ
ることが強制される.

%%%%%%%%%%%%%%%%%%%%%%%%%%%%%%%%%

% 基本符号化は,根付き全域森問題の制約をASPのルール7個で簡潔に表現できる.
% しかし,根付き連結制約を表す\textit{at-most-one}制約の基礎化後のルール
% 数は,根ノード数の2乗に比例するため,大規模な問題に対する求解性能が低
% 下する可能性がある.
%
%この問題を解決するために考案した
\textbf{改良符号化}をコード\ref{code:srf2.lp}に示す.
基本符号化との違いは,根付き連結制約をASPの個数制約で表している点である(16行目).
グラフのノード数を$n$,根ノード数を$r$として,
根付き連結制約の基礎化後のルール数を比較すると,
基本符号化が$n(1+{}_{r}C_{2})$個なのに対し,
改良符号化は$n$個と少なく抑えることができる.
% これにより,大規模な問題に対する有効性が期待できる.
 
%%%%%%%%%%%%%%%%%%%%%%%%%%%%%%%%%
\textbf{有向符号化}をコード\ref{code:srf3.lp}に示す.
有向符号化は,各辺(\code{X},\code{Y})に対して,
2つの弧\code{inForest(X,Y)}と\code{inForest(Y,X)}
を対応させる(2行目)ことで有向グラフ化して解く符号化である.
この有向グラフ化により,非閉路制約を入次数を用いて表現できる.
9行目のルールは,各根ノードについて,入次数は0であることを強制している.
10行目のルールは,根ノード以外の各ノードについて,入次数が1であることを保証している.
他の部分は改良符号化と同じである.

%%%%%%%%%%%%%%%%%%%%%%%%%%%%%%%%%
\subsection{電流制約のASP符号化}\label{chap:electrical}

%%%%%%%%%%%%%%%%%%%%%%%%%%%%%%%%%
\lstinputlisting[float=tb,caption={%
  電流制約のASP符号化},%
captionpos=b,frame=single,label=code:electrical.lp,%
xrightmargin=1zw,% 
xleftmargin=1zw,% 
numbersep=5pt,%
numbers=left,%
breaklines=true,%
columns=fullflexible,keepspaces=true,%
basicstyle=\ttfamily\scriptsize]{code/electrical.lp}
%%%%%%%%%%%%%%%%%%%%%%%%%%%%%%%%%

電流制約のASP符号化をコード\ref{code:electrical.lp}に示す.
この符号化は,トポロジ制約の符号化として,
後述の実行実験で最も良い性能を示した有向符号化を使うことを前提としている.
% 
1--2行目は,根付き全域森に含まれる辺から,配電網における閉じ
たスイッチを求めるためのルールである.
1行目のルールは,\code{inForest(X,Y)}が成り立つとき,
セクション\code{S}と\code{T}がそれぞれ
ノード\code{X}と\code{Y}に属し,
\code{S}と\code{T}の間にスイッチ\code{SW}があれば,
\code{connected(SW,S,T)}を生成する.
すなわち,\code{connected(SW,S,T)}は,
セクション\code{S}と\code{T}の間のスイッチ\code{SW}が閉じていることを
意味する.
なお,供給経路上で\code{S}は\code{T}の親となることに注意が必要である.
2行目の\code{closed_switch/1}は閉じたスイッチを表す.
%
4--5行目は,同一ジャンクション内のセクションの親子関係を表すためのルー
ルである.
\code{entrance_section(S,X)}は,セクション\code{S}が
ジャンクション\code{X}内で,供給経路上で最も変電所に近いことを意味する.
4行目のルールは,各変電所\code{S}がノード\code{X}内で最も上流にあることを表す.
5行目のルールは,各ジャンクション\code{X}に属するセクション\code{S}に
ついて,\code{connected(_,_,S)}が成り立つならば,
\code{S}が最も上流であることを表す.
%
7--8行目の\code{downstream/2}は,任意の2つのセクションの親子関係を表す.
% 7行目のルールは,任意の2つのセクション\code{S}と\code{T}について,
% \code{connected(_,S,T)}が成り立つならば,\code{downstream(S,T)}を生成する.
% 8行目のルールは,ノード\code{X}内で最も上流にあるセクション\code{S}と,
% 同じノード\code{X}に属する他のセクション\code{T}について,
% \code{downstream(S,T)}を生成する.
%
各セクションへ電力が供給されるかどうかは10--11行目のルールで表される.
\code{suppliable(S,R)}は,セクション\code{S}は変電所\code{R}
から電力供給を受けることを意味する.
10行目のルールは,各変電所は自分自身から電力供給を受けることを表す.
11行目のルールは,セクション\code{T}が変電所\code{R}から
電力供給を受け,かつ,\code{downstream(T,S)}が成り立つならば,
% \code{T}の次にセクション\code{S}を経由するならば,
セクション\code{S}も\code{R}から電力供給を受けることを表す.
%
13行目のルールは,電流制約に対応するルールである.ルール中の\code{max_current}は
入力である$J^{max}$に対応する.
各変電所\code{R}に対して,\code{R}が供給する各セクション\code{S}の
電流\code{I}の総和が\code{max_current}以下であることを保証する.

%%% Local Variables:
%%% mode: japanese-latex
%%% TeX-master: "paper"
%%% End:
