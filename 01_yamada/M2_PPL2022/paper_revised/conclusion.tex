\section{おわりに}\label{chap:conc}

本論文では,電気制約として電流制約のみを考慮した配電網問題および配電網
遷移問題に対して,解集合プログラミング(ASP)を用いた解法を提案した.
配電網(遷移)問題に対する ASP を用いた研究は,著者らの知る限り,本論文
がはじめてである.
提案解法の特長と本論文の貢献について,以下にまとめる.

\begin{description}
\item[表現力:]
  配電網問題を解くための ASP 符号化を考案した.
  ASP 言語の高い表現力を生かし,
  配電網問題の制約を簡潔に記述できることを確認した.
  特に,有向符号化は,無向グラフの各辺$u-v$に対して,2つの弧
  $u\rightarrow v$と$v\rightarrow u$を対応させることで有向グラフ化して
  解く符号化であり,非閉路制約を簡潔に表現できる点が特長である.
\item[拡張性:]
  配電網遷移問題に対して,マルチショット ASP 解法を利用した符号化を提案した.
  この符号化は,配電網問題の ASP 符号化の自然な拡張となっている.
  マルチショット ASP 解法を利用することにより,
  ASP システムが同様の探索失敗を避けるために獲得した学習節を
  (部分的に)保持することで,無駄な探索を避けることができる点が特長である.
\item[効率性:]
  DNET (Power Distribution Network Evaluation Tool)
  に公開されている配電網問題(全3問)と,
  Graph Coloring and its Generalizations
  に公開されているグラフを基に独自に生成したトポロジ制約のみの配電網問
  題(全82問)を用いて実行実験を行なった.
  その結果,有向符号化は,他の2つの符号化と比較して,より多くの問題を
  より高速に解くことができ,その優位性を確認できた.
  %
  配電網遷移問題については,実用規模の問題({\sf fukui-tepco})に対して,
  実行可能解のペアをランダムに選び,合計 1000 問の配電網遷移問題を生成
  し,実行実験を行なった.その結果,すべての問題の到達可能性を判定する
  ことができ,得られた最短ステップ長の最大値は7であった.
\end{description}

今後の課題としては,電流制約と電圧制約を含む完全な配電網問題への拡張が
挙げられる.しかし,完全な問題は非線形な制約を含むため,標準的な ASP
言語では記述できない.この問題を解決するために,近年研究開発が進められ
ている背景理論付き ASP (ASP Modulo Theories~\cite{DBLP:conf/iclp/GebserKKOSW16}) 
を用いた解法の実現可能性について調査を進める.

%%% Local Variables:
%%% mode: japanese-latex
%%% TeX-master: "paper"
%%% End:
