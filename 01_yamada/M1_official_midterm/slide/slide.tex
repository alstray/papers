\documentclass[dvipdfmx,11pt]{beamer}

\usepackage[deluxe]{otf} 
\usepackage{txfonts}
\renewcommand{\kanjifamilydefault}{\gtdefault}
\usepackage{amssymb,amsmath}
\usepackage{hyperref}
\usepackage[absolute,overlay]{textpos}
\usepackage{comment}
\usepackage{colortbl}
\usepackage{graphicx}
\usepackage{tikz}
\usetikzlibrary{positioning}
\usetikzlibrary{shadows}
\usepackage{listings}
\usepackage{plistings}
\usepackage{multicol}
\def\lstlistingname{コード}
\newcommand{\code}[1]{\lstinline[basicstyle=\ttfamily]{#1}}
\newcommand{\lw}[1]{\smash{\lower-5.ex\hbox{#1}}}
\newcommand{\redunderline}[1]{\textcolor{red}{\underline{\textcolor{black}{#1}}}}
%%\usetheme{Frankfurt}
\usetheme{Warsaw}
\setbeamertemplate{navigation symbols}{} %スライドのボタン?(右下のやつ)を消す
\setbeamersize{text margin left=1.5em,text margin right=1.5em} % 余白なくすやつ

% footer setting %
\makeatother
\setbeamertemplate{footline}
{
  \leavevmode%
  \hbox{%
  \begin{beamercolorbox}[wd=.4\paperwidth,ht=2.25ex,dp=1ex,center]{author in head/foot}%
    \usebeamerfont{author in head/foot}\insertshortauthor
  \end{beamercolorbox}%
  \begin{beamercolorbox}[wd=.6\paperwidth,ht=2.25ex,dp=1ex,center]{title in head/foot}%
    \usebeamerfont{title in head/foot}\hspace*{1ex} \insertshorttitle\hspace*{3em}
    \textbf{ \insertframenumber{} / \inserttotalframenumber } \hspace*{1ex}
  \end{beamercolorbox}}%
  \vskip0pt%
}
\makeatletter

% exclude apprendix slides from framenumber %
\newcommand{\backupbegin}{
   \newcounter{framenumberappendix}
   \setcounter{framenumberappendix}{\value{framenumber}}
}
\newcommand{\backupend}{
   \addtocounter{framenumberappendix}{-\value{framenumber}}
   \addtocounter{framenumber}{\value{framenumberappendix}} 
}

\lstset{
 basicstyle=\ttfamily\color{black},
 keepspaces=true,
 escapechar=|,
 columns=[l]{fullflexible},
 commentstyle={\color{red}},
 stringstyle={\color{blue}}}

\title{解集合プログラミングを用いた\\配電網問題の解法}
\author[山田 健太郎]{山田 健太郎}
\date{2021年2月24日\\前期課程中間発表}
\institute{番原研究室}

%#################################################
%# 本文 ##########################################
%#################################################
\begin{document}

%%%%%%%%%%%%%%%%%%%%%%%%%%%%%%%%%%%%%%%%%%%%%%%%%%
%% タイトル 
%%%%%%%%%%%%%%%%%%%%%%%%%%%%%%%%%%%%%%%%%%%%%%%%%%
\begin{frame}{}
  \titlepage
\end{frame}

%%%%%%%%%%%%%%%%%%%%%%%%%%%%%%%%%%%%%%%%%%%%%%%%%%
% 配電網
%%%%%%%%%%%%%%%%%%%%%%%%%%%%%%%%%%%%%%%%%%%%%%%%%%
\begin{frame}{配電網問題}
  \begin{alertblock}{}\centering
    求解困難な組合せ最適化問題の一種
  \end{alertblock}
  \vfill
  \begin{itemize}
  \item \alert{\bf 配電網}とは,変電所と,一般家庭や工場を繋ぐ電力供給
    経路のネットワークである.
  \item  配電網の構成技術はスマートグリッドや,災害時の障害箇所の迂回
    構成などを支える重要な基盤技術として期待されている.
  \item \alert{\bf 配電網問題}とは,
    \begin{itemize}
    \item \structure{\bf トポロジ制約}と\structure{\bf 電気制約}を満たしつつ,
    \item 損失電力を最小にするスイッチの開閉状態を求めることが目的.
    \end{itemize}
  \item これまで,メタヒューリスティクス等の解法が提案されている.
  \item 厳密解法としては,フロンティア法を用いた解法が提案されており
    \begin{itemize}
    \item 実用規模の配電網問題(\structure{\textbf{スイッチ数468個}})の
      最適解を求めることに成功~[井上ほか '12].
    \end{itemize}
  \end{itemize}
\end{frame}
%%%%%%%%%%%%%%%%%%%%%%%%%%%%%%%%%%%%%%%%%%%%%%%%%%
%% ASP
%%%%%%%%%%%%%%%%%%%%%%%%%%%%%%%%%%%%%%%%%%%%%%%%%%
\begin{frame}{解集合プログラミング(Answer Set Programming; ASP)}
 \begin{itemize}
  \item \structure{\bf ASPの言語}は一階論理に基づく知識表現言語の一種である.
  \item \structure{\bf ASPシステム}は論理プログラムから安定モデル意味
        論~[Gelfond and Lifschitz '88]に基づく解集合を計算するシステムである.
  \item 近年,SATソルバーの実装技術を応用した高速ASPシステムが実現され,
        システム検証,プランニング,システム生物学など様々な分野への応用が
        拡大している.
 \end{itemize}
 \vfill
 \begin{alertblock}{配電網問題に対してASP技術を用いる利点}
  \begin{itemize}
   \item ASP言語の高い表現力を生かし,各種制約を\textbf{簡潔に記述可能}
   \item マルチショットASP解法により,
         ある配電網構成(スタート状態)から他の配電網構成(ゴール状態)
         へのスイッチの切替手順を求める\textbf{遷移問題への拡張が容易}
   \item 背景理論つきASPにより,様々な\textbf{背景理論ソルバーと連携可能}
   %\item 解の最適性を保証でき,最適解の列挙も可能
  \end{itemize}
 \end{alertblock}
\end{frame}
%%%%%%%%%%%%%%%%%%%%%%%%%%%%%%%%%%%%%%%%%%%%%%%%%% 
%% 根付き全域森問題
%%%%%%%%%%%%%%%%%%%%%%%%%%%%%%%%%%%%%%%%%%%%%%%%%%
\begin{frame}{根付き全域森問題}
 \begin{alertblock}{}
  トポロジ制約のみの配電網問題は,グラフと根と呼ばれる特別なノードから,
  \alert{\bf 根付き全域森}を求める部分グラフ探索問題に帰着できる.
 \end{alertblock}
 \vfill
 \begin{block}{根付き全域森 (Spanning Rooted Forest) [川原・湊 '12]}
  グラフ$G=(V,E)$と,
  \textbf{根}と呼ばれる$V$上のノードが与えられたとき,
  $G$上の根付き全域森とは,以下の条件を満たす$G$の部分グラフ$G'=(V,E'),\ E' \subseteq E$である.
  \begin{enumerate}
   \item $G'$はサイクルを持たない. (\alert{\bf 非閉路制約})
   \item $G'$の各連結成分は,ちょうど1つの根を含む. (\alert{\bf 根付き連結制約})
  \end{enumerate}
 \end{block}
 \vfill
 \begin{columns}[onlytextwidth]
  \begin{column}{0.45\textwidth}\centering
   \begin{exampleblock}{入力例}
	\centering
	\scalebox{0.8}{%%%%%%%%%%%%%%%%%%%%%%%%%%%%%%%%%%%%%%%%%%%%%%%%%%
% 根付き全域森の例
%%%%%%%%%%%%%%%%%%%%%%%%%%%%%%%%%%%%%%%%%%%%%%%%%%

\begin{tikzpicture}[scale=0.5]

 % 設定
 \tikzset{node/.style={circle,draw=black,fill=white}}

 \definecolor{edge1}{RGB}{191,0,0}
 \definecolor{node1}{RGB}{249,200,200}
 \definecolor{edge3}{RGB}{38,38,134}
 \definecolor{node3}{RGB}{200,200,249}

 % 補助線
 % \draw [help lines,blue] (0,0) grid (20,6);

 % 入力されるグラフ
 % node %
 \node[circle, ultra thick,draw=edge1,fill=node1] (in1) {1};
 \node[node,right= of in1] (in2){2};
 \node[circle, ultra thick, draw=edge3,fill=node3, right=of in2](in3){3};
 \node[node,below= of in1] (in4){4};
 \node[node,below= of in2] (in5){5};
 \node[node,below= of in3] (in6){6};

 % 辺
 \foreach \u / \v in {in1/in2,in2/in3,in1/in4,in2/in5,in3/in6,in4/in5,in5/in6}
 \draw (\u) -- (\v);

\end{tikzpicture}

%%%%%%%%%%%%%%%%%%%%%%%%%%%%%%%%%%%%%%%%%%%%%%%%%%%%%%%%%%
%%% Local Variables:
%%% mode: japanese-latex
%%% TeX-master: ``slide''
%%% End:
}
   \end{exampleblock}
  \end{column}
  \begin{column}{0.05\textwidth}\centering
   \\
   $\Rightarrow$
  \end{column}
  \begin{column}{0.45\textwidth}\centering
   \begin{exampleblock}{解の例}
	\centering
	\scalebox{0.8}{%%%%%%%%%%%%%%%%%%%%%%%%%%%%%%%%%%%%%%%%%%%%%%%%%%
% 根付き全域森の例
%%%%%%%%%%%%%%%%%%%%%%%%%%%%%%%%%%%%%%%%%%%%%%%%%%

\begin{tikzpicture}[scale=0.5]

 % 設定
 \tikzset{node/.style={circle,draw=black,fill=white}}

 \definecolor{edge1}{RGB}{191,0,0}
 \definecolor{node1}{RGB}{249,200,200}
 \definecolor{edge3}{RGB}{38,38,134}
 \definecolor{node3}{RGB}{200,200,249}

 % 補助線
 % \draw [help lines,blue] (0,0) grid (20,6);

 % node %
 \node[circle, ultra thick, draw=edge1, fill=node1](out1){1};
 \node[node, fill=node1, right=of out1] (out2){2};
 \node[circle, ultra thick, draw=edge3,fill=node3, right=of out2](out3){3};
 \node[node, fill=node1, below=of out1] (out4){4};
 \node[node, fill=node3, below=of out2] (out5){5};
 \node[node, fill=node3, below=of out3] (out6){6};

 \foreach \u / \v in {out1/out2,out1/out4}
 \draw [very thick, edge1] (\u) -- (\v);

 \foreach \u / \v in {out3/out6,out5/out6}
 \draw [very thick, edge3](\u) -- (\v);

\end{tikzpicture}

%%%%%%%%%%%%%%%%%%%%%%%%%%%%%%%%%%%%%%%%%%%%%%%%%%%%%%%%%%
%%% Local Variables:
%%% mode: japanese-latex
%%% TeX-master: ``slide''
%%% End:
}
   \end{exampleblock}
  \end{column}
 \end{columns}
 \vfill
\end{frame}
%%%%%%%%%%%%%%%%%%%%%%%%%%%%%%%%%%%%%%%%%%%%%%%%%%
%% 研究目的
%%%%%%%%%%%%%%%%%%%%%%%%%%%%%%%%%%%%%%%%%%%%%%%%%%
\begin{frame}{研究目的}
  \begin{alertblock}{目的}\centering
    ASP技術を活用して,大規模な配電網問題を効率良く解くシステムを構築
    する.
  \end{alertblock}
  \vfill
 \begin{block}{研究内容}
  \begin{itemize}
   \item 配電網問題のASP符号化
   \item 電気制約の効率的な実装
   \item 配電網問題の遷移問題への拡張
  \end{itemize}
 \end{block}
 \vfill
  \begin{exampleblock}{研究成果}
    \begin{enumerate}
    \item \structure{\bf 配電網問題のASP符号化(卒業研究)}
      \begin{itemize}
	   \item 根付き全域森問題を解く2種類のASP符号化を考案.
      % \item 基本符号化
      % \item 改良符号化
      \end{itemize}
    \item 根付き全域森問題のある解(スタート状態)から他の解(ゴール状態)
      への辺の切替手順を求める\structure{\bf 解の遷移問題への拡張}
      % \begin{itemize}
      % \item シングルショット符号化
      % \item マルチショット符号化
      % \end{itemize}
	  % \item \structure{\bf 実用規模の問題,および,より大規模な問題を用いて評価}
    \end{enumerate}
  \end{exampleblock}
\end{frame}
%%%%%%%%%%%%%%%%%%%%%%%%%%%%%%%%%%%%%%%%%%%%%%%%%%
%% 提案手法
%%%%%%%%%%%%%%%%%%%%%%%%%%%%%%%%%%%%%%%%%%%%%%%%%%
\begin{frame}{提案手法 (卒業研究)}
   %\scalebox{0.9}{\centering\begin{figure*}[t]
  \centering
  \thicklines
  \setlength{\unitlength}{1.28pt}
  \small
  \begin{picture}(280,57)(4,-10)
    \put( -35, 20){\dashbox(70,24){\shortstack{組合せ最適化問題\\のインスタンス}}}
    \put( 45, 20){\framebox(50,24){変換器}}
    \put(105, 20){\dashbox(70,24){\shortstack{ASPファクト}}}
    \put(105,-10){\dashbox(70,24){\shortstack{ASP符号化\\(論理プログラム)}}}
    \put(185,-10){\framebox(60,54){}}
    \put(189, 25){\framebox(52,12){ASPソルバー}}
    \put(190, -5){\framebox(50,12){LNPS}}
    % \put(180, 20){\framebox(50,24){ASPシステム}}
    \put(255, 20){\dashbox(70,24){\shortstack{組合せ最適化問題\\の最適解}}}
    \put(  35, 32){\vector(1,0){10}}
    \put(  95, 32){\vector(1,0){10}}
    \put(175, 32){\vector(1,0){10}}
    \put(245, 32){\vector(1,0){10}}
    \put(175, +2){\line(1,0){4}}
    \put(179, +2){\line(0,1){30}}
    \put(205,  7){\vector(0,1){17}}
    \put(225, 24){\vector(0,-1){17}}
    \put(190, 48){提案ソルバー}
  \end{picture}  
\caption{提案ソルバー\textit{asprior}の構成}
\label{fig:arch}
\end{figure*}

%%% Local Variables: 
%%% mode: latex
%%% TeX-master: "paper"
%%% End: 
}
   \begin{block}{根付き全域森問題の2種類のASP符号化を考案}
     \begin{itemize}
     \item \alert{\bf 基本符号化}
       \begin{itemize}
       \item 根付き全域森問題の制約を,\textbf{ASPのルール7個}で簡潔に記述
       \end{itemize}
     \item \alert{\bf 改良符号化}
       \begin{itemize}
       \item ASPシステムは,変数を含む論理プログラムを,変数を含まない
         論理プログラムに\textbf{基礎化}したのち解集合を計算する.
       \item 根付き連結制約をASPの個数制約で表現することにより,
         \textbf{基礎化後のルール数を少なく抑える}よう工夫されている.
       \item これにより,改良符号化は大規模な問題への有効性が期待できる.
       \end{itemize}
     \end{itemize}
   \end{block}
 \begin{itemize}
%  \renewcommand{\thefootnote}{\fnsymbol{footnote}}
 % \setcounter{footnote}{1}
  \item \textit{DNET}\footnote{https://github.com/takemaru/dnet},
        および,\textit{Graph Coloring and its Generalizations}
        \footnote{https://mat.tepper.cmu.edu/COLOR04/}%
        の問題をベースに独自に生成した配電網問題を用いて評価実験を行った.
  \item 結果として,改良符号化は,基本符号化と比較して,より多くの問題を高速に解いた.
 \end{itemize}
\end{frame}
%%%%%%%%%%%%%%%%%%%%%%%%%%%%%%%%%%%%%%%%%%%%%%%%%%
%% 解の遷移問題
%%%%%%%%%%%%%%%%%%%%%%%%%%%%%%%%%%%%%%%%%%%%%%%%%%
\begin{frame}{遷移問題への拡張}
\begin{alertblock}{根付き全域森遷移問題}
  根付き全域森問題とその2つの実行可能解が与えられたとき,
  ある解から他のもう一つの解へ根付き全域森の制約を満たしながら移る
  ``解の遷移問題''.
  \begin{itemize}
  \item 各ステップ$t$で変更可能な辺の数を$d$個以下に制限し(\textbf{遷移制約})
  \item 最短ステップ長での辺の変更手順を求めることが目的となる.
  \end{itemize}
\end{alertblock}
\vfill  
\begin{itemize}
\item ある配電網構成(スタート状態)から,他の配電網構成(ゴール状態)への
  スイッチの切替手順を求める問題に対応する.
\item 配電網における障害時の復旧予測への応用が期待できる.
\end{itemize} 
\end{frame}
%%%%%%%%%%%%%%%%%%%%%%%%%%%%%%%%%%%%%%%%%%%%%%%%%%
%% 提案アプローチ
%%%%%%%%%%%%%%%%%%%%%%%%%%%%%%%%%%%%%%%%%%%%%%%%%%
\begin{frame}{提案手法}
 \begin{itemize}
  \item 根付き全域森遷移問題のASP符号化を2種類考案した.
 \end{itemize}
 \begin{block}{シングルショット符号化 (改良符号化の自然な拡張)}
    \begin{itemize}
    \item 与えられたステップ長$t$の解が存在するかを判定する.
      % \begin{itemize}
      % \item 与えられたステップ長$t$の解が存在するかを判定する.
      % \item 各アトムにステップ長を表す項\code{T}を追加.例) \code{inForest(X,Y,T)}
      % \item スタート状態,ゴール状態,遷移制約に関するルールを追加.
      % \end{itemize}
    \item 解が見つかるまで,ステップ長$t$を増やしながら,複数の問題を
      繰り返し解く必要がある.
    \item 各問題中の制約の大部分は共通であるため,
      \textbf{同一の探索空間を何度も調べる}ことになり,
      \textbf{求解効率が低下}するという問題点がある.
  \end{itemize}
 \end{block}
 \vfill
 \begin{alertblock}{マルチショット符号化}
   \begin{itemize}
   \item ASPシステム\textit{clingo}のマルチショット解法ライブラリを使用.
   \item 同様の探索失敗を避けるために獲得した学習節を保持することによって,
		 \textbf{無駄な探索を行うことなく},制約を追加した論理プログラムを
		 連続的に解くことができる.
  \end{itemize}
 \end{alertblock}
\end{frame}
%%%%%%%%%%%%%%%%%%%%%%%%%%%%%%%%%%%%%%%%%%%%%%%%%%
%% 実験内容--遷移問題
%%%%%%%%%%%%%%%%%%%%%%%%%%%%%%%%%%%%%%%%%%%%%%%%%%
\begin{frame}{実験概要}
  \renewcommand{\thefootnote}{\fnsymbol{footnote}}
  \setcounter{footnote}{1}
  提案手法の有効性を評価するために,以下の実験を行った.
  \vfill
  \begin{itemize}
  \item \structure{\bf 比較するASP符号化:}
    \begin{itemize}
    \item シングルショット符号化
    \item マルチショット符号化
    \end{itemize}
  \item \structure{\bf ベンチマーク問題:} 全30問
    \begin{itemize}
    \item DNET\footnote{https://github.com/takemaru/dnet}
      で公開されている実用規模の配電網問題(\textsf{fukui-tepco},
      スイッチ数 468,変電所の数 72)をベース
    \item 実行可能解の中から,スタート状態を5つ,ゴール状態を6つをラン
      ダムに選択して使用
    \end{itemize}
  \item \structure{\bf ASPシステム:} \textit{clingo-5.4.0} $+$ \textit{trendy}
  \item \structure{\bf 実験環境:} Mac mini,3.2GHz Intel Core i7,64GBメモリ
  \end{itemize}
\end{frame}
%%%%%%%%%%%%%%%%%%%%%%%%%%%%%%%%%%%%%%%%%%%%%%%%%%
%% 実験結果--遷移問題
%%%%%%%%%%%%%%%%%%%%%%%%%%%%%%%%%%%%%%%%%%%%%%%%%%
\begin{frame}{実験結果:CPU時間の比較(1/2)}
 \centering
 \scalebox{0.8}{\begin{tabular}{ccrrr}  
 \rowcolor[RGB]{0,96,0}
  \color{white}問題名 &
  \multicolumn{1}{c}{\color{white}ステップ長$t$} & 
  \multicolumn{1}{c}{\color{white}シングルショット} & 
  \multicolumn{1}{c}{\color{white}マルチショット} & 
  \multicolumn{1}{c}{\color{white}シングル/マルチ} \\
 \rowcolor[RGB]{230,239,230}%%
s1\_g60 & 8 & 50.098 & \alert{25.659} & 1.952~ \\
 \rowcolor[RGB]{196,230,196}%
s1\_g70 & 8 & 47.177 & \alert{24.981} & 1.889~ \\
 \rowcolor[RGB]{230,239,230}%%
s1\_g80 & 8 & 43.193 & \alert{15.852} & 2.725~ \\
 \rowcolor[RGB]{196,230,196}%
s1\_g90 & 8 & 48.984 & \alert{22.983} & 2.131~ \\
 \rowcolor[RGB]{230,239,230}%%
s1\_g100 & 6 & 20.964 & \alert{4.709} & 4.452~ \\
 \rowcolor[RGB]{196,230,196}%
s10\_g60 & 6 & 21.947 & \alert{5.132} & 4.277~ \\
 \rowcolor[RGB]{230,239,230}%%
s10\_g70 & 8 & 43.840 & \alert{16.290} & 2.691~ \\
 \rowcolor[RGB]{196,230,196}%
s10\_g80 & 6 & 21.156 & \alert{4.887} & 4.329~ \\
 \rowcolor[RGB]{230,239,230}%%
s10\_g90 & 6 & 21.276 & \alert{5.324} & 3.996~ \\
 \rowcolor[RGB]{196,230,196}%
s10\_g100 & 8 & 45.704 & \alert{17.697} & 2.583~ \\
 \rowcolor[RGB]{230,239,230}%%
s20\_g60 & 6 & 21.202 & \alert{4.973} & 4.263~ \\
 \rowcolor[RGB]{196,230,196}%
s20\_g70 & 4 & 9.890 & \alert{2.699} & 3.664~ \\
 \rowcolor[RGB]{230,239,230}%%
s20\_g80 & 8 & 48.473 & \alert{16.335} & 2.967~ \\
 \rowcolor[RGB]{196,230,196}%
s20\_g90 & 10 & 107.938 & \alert{64.441} & 1.675~ \\
 \rowcolor[RGB]{230,239,230}%%
s20\_g100 & 8 & 48.473 & \alert{17.894} & 2.709~ \\
 \rowcolor[RGB]{196,230,196}%
s30\_g60 & 6 & 21.189 & \alert{5.287} & 4.008~ \\
 \rowcolor[RGB]{230,239,230}%%
s30\_g70 & 4 & 9.901 & \alert{2.735} & 3.620~ \\
 \rowcolor[RGB]{196,230,196}%
s30\_g80 & 6 & 21.884 & \alert{5.223} & 4.196~ \\
 \rowcolor[RGB]{230,239,230}%%
s30\_g90 & 4 & 9.979 & \alert{2.658} & 3.754~ \\
 \rowcolor[RGB]{196,230,196}%
s30\_g100 & 8 & 50.344 & \alert{18.845} & 2.671~ \\
\end{tabular}
}
\end{frame}

\begin{frame}{実験結果:CPU時間の比較(2/2)}
 \centering
 \vskip -2ex
 \scalebox{0.8}{\begin{tabular}{ccrrr}  
 \rowcolor[RGB]{0,96,0}
  \color{white}問題名 &
  \multicolumn{1}{c}{\color{white}ステップ長$t$} & 
  \multicolumn{1}{c}{\color{white}シングルショット} & 
  \multicolumn{1}{c}{\color{white}マルチショット} & 
  \multicolumn{1}{c}{\color{white}シングル/マルチ} \\
 \rowcolor[RGB]{230,239,230}
 s40\_g60 & 8 & 44.637 & \alert{14.254} & 3.132~ \\
 \rowcolor[RGB]{196,230,196}%
s40\_g70 & 6 & 21.334 & \alert{4.795} & 4.449~ \\
 \rowcolor[RGB]{230,239,230}
s40\_g80 & 8 & 45.202 & \alert{14.099} & 3.206~ \\
 \rowcolor[RGB]{196,230,196}%
s40\_g90 & 6 & 21.710 & \alert{5.119} & 4.241~ \\
 \rowcolor[RGB]{230,239,230}
s40\_g100 & 6 & 21.299 & \alert{5.666} & 3.759~ \\
  \rowcolor[RGB]{196,230,196}%
s50\_g60 & 4 & 10.021 & \alert{2.726} & 3.676~ \\
\rowcolor[RGB]{230,239,230}
s50\_g70 & 6 & 21.291 & \alert{4.718} & 4.513~ \\
 \rowcolor[RGB]{196,230,196}%
s50\_g80 & 6 & 21.163 & \alert{6.503} & 3.254~ \\
 \rowcolor[RGB]{230,239,230}
s50\_g90 & 10 & 108.299 & \alert{65.352} & 1.657~ \\
 \rowcolor[RGB]{196,230,196}%
s50\_g100 & 4 & 9.934 & \alert{2.700} & 3.679~ \\
 \noalign{\hrule height 0.5pt} \rowcolor[RGB]{230,239,230}
\multicolumn{2}{c}{平均} & 34.617 & \alert{13.685} & 3.337~ \\
\end{tabular}}
 
 \vskip 1em
\begin{itemize}
 \item マルチショット符号化は,シングルショット符号化と比較して,全ての
	   問題をより高速に解いており,\alert{\bf 平均で3.3倍の高速化}を実現している.
 %\item 最短ステップ長$t=10$の問題も解けることが確認できた.
\end{itemize}
\end{frame}
%%%%%%%%%%%%%%%%%%%%%%%%%%%%%%%%%%%%%%%%%%%%%%%%%%
%% まとめ
%%%%%%%%%%%%%%%%%%%%%%%%%%%%%%%%%%%%%%%%%%%%%%%%%%
\begin{frame}{まとめと今後の課題}
 \begin{enumerate}
  \item \structure{\bf 配電網問題のASP符号化(卒業研究)}
        \begin{itemize}
         \item 根付き全域森問題を解く2種類のASP符号化を提案.
         \item 基礎化後のルール数を抑えることで,より多くの問題を高速に解けることを確認.
        \end{itemize}
  \item \structure{\bf 遷移問題への拡張}
        \begin{itemize}
         \item 根付き全域森問題のある解(スタート状態)から他の解(ゴール状態)への辺の切替手順
               を求める遷移問題へ拡張.
         \item 根付き全域森遷移問題を解く2種類のASP符号化を提案.
         \item 遷移問題へのマルチショットASP解法の有効性を確認.
        \end{itemize}
 \end{enumerate}
 \vfill
 \begin{alertblock}{今後の課題}
  \begin{itemize}
   \item 電気制約の実装
         \begin{itemize}
          \item 背景理論付きASP(ASP Modulo Theories)の活用.
         \end{itemize}
   \item 根付き全域森問題のASP符号化の改良
  \end{itemize}
 \end{alertblock}
\end{frame}
%%%%%%%%%%%%%%%%%%%%%%%%%%%%%%%%%%%%%%%%%%%%%%%%%%
%% 研究業績
%%%%%%%%%%%%%%%%%%%%%%%%%%%%%%%%%%%%%%%%%%%%%%%%%%
\begin{frame}{研究業績}
 \begin{itemize}
  \item \structure{\bf 受賞}
  \begin{itemize}
   \item \alert{\bf 学生奨励賞受賞}\\
         山田健太郎, 湊真一, 番原睦則. 解集合プログラミングを用いた配電網問題の解法に関する一考察. 日本ソフトウェア科学会第37回大会講演論文集, 3-L, 日本ソフトウェア科学会, 2020年9月8日.
  \end{itemize}
  \item \structure{\bf 発表}
  \begin{itemize}
   \setlength{\itemsep}{3pt} % 項目間
   \item 解集合プログラミングを用いた配電網問題の解法に関する一考察.\\
         2020年度日本ソフトウェア科学会第37回大会. 
   \item 解集合プログラミングを用いた配電網問題の解法に関する一考察. \\
         NII共同研究「巨大な制約充足問題を解くためのSAT型CSPソルバーの研究開発」.
   \item 解集合プログラミングを用いた配電網問題の解法. \\
         基盤(A)「離散構造処理系に基づく列挙と最適化の統合的技法の研究」プロジェクト近況報告&自由討論会.
  \end{itemize}
 \end{itemize}
\end{frame}
%###########################################################
%##### 補助スライド ########################################
%###########################################################

%%%%% 補助スライド
\appendix
\backupbegin

\begin{frame}{~}
 \centering
 - 補足用 -
\end{frame} 

%%%%%%%%%%%%%%%%%%%%%%%%%%%%%%%%%%%%%%%%%%%%%%%%%%
%% 電気制約
%%%%%%%%%%%%%%%%%%%%%%%%%%%%%%%%%%%%%%%%%%%%%%%%%%
\begin{frame}{補足 : 電気制約}
 \begin{itemize}
  \item \alert{電気制約}は,送電する電流$\cdot$電圧の適正範囲を保証する制約.
  \begin{itemize}
   \item 供給経路の各区間で許容電流を超えない.
   \item 電気抵抗による電圧降下が許容範囲を超えない.
   \item etc.
  \end{itemize}
  \item 電流と電圧が影響し合う\structure{実数ドメイン上の制約}によって表される.
		% \begin{itemize}
		%  		 \item 送電システム上の条件など.
		% \end{itemize}
  \item 実数ドメイン上の制約は,純粋なASPのみで扱うのは\alert{困難}.
		\begin{itemize}
		 \item 緩和問題として,変電所から供給できる家庭の数に上限をつける.
		 \item ASPMT技術により,ASPで得られた解について,
			   背景理論ソルバーと連携して実数ドメイン上の制約を調べる.
		\end{itemize}
 \end{itemize}
\end{frame}

%%%%%%%%%%%%%%%%%%%%%%%%%%%%%%%%%%%%%%%%%%%%%%%%%%
%% 基礎化
%%%%%%%%%%%%%%%%%%%%%%%%%%%%%%%%%%%%%%%%%%%%%%%%%%
\begin{frame}{補足 : ASPシステム}
 
 \vspace{-0.5cm}

 \begin{figure}[htbp]
  \centering
  %%%%%%%%%%%%%%%%%%%%%%%%%%%%%%%%%%%%%%%%%%%%%%%%%%
%% 基礎化の流れの図
%%%%%%%%%%%%%%%%%%%%%%%%%%%%%%%%%%%%%%%%%%%%%%%%%%
\begin{tikzpicture}

 \definecolor{edge}{RGB}{38,38,134}
 \definecolor{node}{RGB}{220,220,249}

 \definecolor{alert_edge}{RGB}{191,0,0}
 \definecolor{alert_node}{RGB}{249,200,200}

 \definecolor{ex_edge}{RGB}{0,96,0}
 \definecolor{ex_node}{RGB}{230,239,230}

 \def\nodespace{2.4cm}

 \tikzset{block/.style={rectangle, thick, draw=edge, fill=node, text width=3cm, 
 text centered, rounded corners, text width=2cm, minimum height=1.5cm}};

 \tikzset{alertblock/.style={rectangle, thick, draw=alert_edge, fill=alert_node, 
 text width=3cm, text centered, rounded corners, text width=1.5cm, minimum height=1.2cm}};

 \node[block](ikkai){一階ASP\\プログラム};

 \node[rectangle,rounded corners, thick, draw=ex_edge, fill=ex_node, 
 right=0.22*\nodespace of ikkai, minimum width=6cm, minimum height=3cm, 
 text centered, label=ASPシステム](sys){};

 \node[block, right=\nodespace of ikkai](meidai){命題ASP\\プログラム};
 \node[block, right=\nodespace of meidai](ASP){解集合};

 \node[right=0.6*\nodespace of ikkai, text width=1.5cm, 
 text centered, text=red, anchor=south](){基礎化\\ソルバー};
 \node[right=0.4*\nodespace of meidai, text width=1.5cm, 
 text centered, text=red, anchor=south](){解集合\\ソルバー};

 
 \foreach \u / \v / \n in {ikkai/meidai,meidai/ASP}
 \draw [thick,->] (\u) to (\v);

\end{tikzpicture}
 \end{figure}

 \vspace{-0.5cm}

 \begin{exampleblock}{}
  \begin{enumerate}
   \item 一階ASPプログラムを基礎化ソルバーによって,
		 命題ASPプログラムに\alert{基礎化}する.
   \item 命題ASPプログラムについて,SAT技術を応用した解集合ソルバーが解集合を探索する.
  \end{enumerate}
 \end{exampleblock}

\end{frame}
%%%%%%%%%%%%%%%%%%%%%%%%%%%%%%%%%%%%%%%%%%%%%%%%%%
%% ASPの構文
%%%%%%%%%%%%%%%%%%%%%%%%%%%%%%%%%%%%%%%%%%%%%%%%%%
\begin{frame}{ASPの構文}
  \begin{alertblock}{}\centering
    ASPの言語は論理プログラムをベースとしている~\footnotemark.
  \end{alertblock}
  \begin{itemize}
  \item \structure{\bf 論理プログラム}とは,以下の\structure{\bf ルール}の有限集合である.
    \begin{center}
      \begin{minipage}[c]{0.7\textwidth}
        \begin{block}{}\centering
          $a_0$\quad\code{:-}\quad$a_1$\code{,}\ldots\code{,}$a_m$\code{,}
          \ \code{not}~$a_{m+1}$\code{,}\ldots\code{,} \code{not}~$a_n$\code{.}
        \end{block}        
      \end{minipage}
   \end{center}\vfill
    $0 \leq m \leq n$ であり,各 $a_i$ はアトム,
    \code{not}は\structure{\bf デフォルトの否定},\\
    ``\code{,}''は連言(AND)を表す.``\code{:-}''の左辺を\structure{\bf ヘッド},
		右辺を\structure{\bf ボディ}と呼ぶ.
  \item \alert{\bf 直感的な意味}は,
    「$a_1,\ldots,a_m$がすべて成り立ち,
    $a_{m+1},\ldots,a_n$のそれぞれが成り立たないならば,
    $a_0$が成り立つ」である.
  \item ボディが空のルールを\structure{\bf ファクト}と呼び,``\code{:-}''は省略できる.
  \item ヘッドが空のルールを\structure{\bf 一貫性制約}と呼ぶ.例えば,\hspace{-1ex}
    ``\code{:-} $a_1$\code{,} \code{not}~$a_{2}$''は,
    「$a_1$が成り立つならば,$a_2$が成り立つ」を意味する.
  \end{itemize}
  \footnotetext{本発表では標準論理プログラムを単に論理プログラムと呼ぶ.}
\end{frame}
%%%%%%%%%%%%%%%%%%%%%%%%%%%%%%%%%%%%%%%%%%%%%%%%%%
%% ASPの拡張構文
%%%%%%%%%%%%%%%%%%%%%%%%%%%%%%%%%%%%%%%%%%%%%%%%%%
\begin{frame}{ASPの拡張構文}
\begin{alertblock}{}\centering
  組合せ問題を解くための便利な構文が用意されている.
\end{alertblock}
\begin{itemize}
 \item \structure{\bf 選択子}
   \begin{center}
     \code{\{}$a_1$\code{;}\ldots\code{;}$a_n$\code{\}}
   \end{center}
   アトム集合 $\{a_1,\dots,a_n\}$
   の任意の部分集合が成り立つことを意味する.
 \item \structure{\bf 個数制約}
   \begin{center}
     $lb$\ \code{\{}$a_1$\code{;}\ldots\code{;}$a_n$\code{\}}\ $ub$
   \end{center}
   $a_1,\dots,a_n$ のうち,
   $lb$個以上,$ub$個以下が成り立つことを意味する.
 \item \structure{\bf 重み付き個数制約}
   \begin{center}
     $lb$ \code{\#sum\{} $w_1$\code{:}$a_1$\code{;}\ldots\code{;}$w_n$\code{:}$a_n$ \code{\}} $ub$
   \end{center}
   $a_1,\dots,a_n$のうち,
   成り立つアトムの重み和が$lb$以上,$ub$以下になることを意味する.
\end{itemize}
\end{frame}
%%%%%%%%%%%%%%%%%%%%%%%%%%%%%%%%%%%%%%%%%%%%%%%%%%
%% 改良符号化 (到達可能性)
%%%%%%%%%%%%%%%%%%%%%%%%%%%%%%%%%%%%%%%%%%%%%%%%%%
\begin{frame}[fragile]{改良符号化: 到達可能性}
\begin{exampleblock}{}\small
\begin{lstlisting}
(1) { inForest(X,Y) } :- edge(X,Y).
\end{lstlisting}
\end{exampleblock}
\begin{itemize}
 \item (1) 各辺\code{(X,Y)について},根付き全域森に含まれること意味する \\
	  アトム\code{inForest(X,Y)}を導入する.
\end{itemize}
\begin{exampleblock}{}\small
\begin{lstlisting}
(2) reached(R,R) :- root(R).
(3) reached(X,R) :- reached(Y,R), inForest(Y,X).
(4) reached(X,R) :- reached(Y,R), inForest(X,Y).
\end{lstlisting}
\end{exampleblock}
\vfill
\begin{itemize}
\item アトム\code{reached(X,R)}は,ノード\code{X}が根ノード\code{R}から到達可能であることを意味する.
%\item (2) 各根ノード\code{R}について,自分自身から到達可能であることを表す.
\item (3) ノード\code{Y}が根ノード\code{R}から到達可能かつ,辺\code{(Y,X)}が根付き全域森に含まれるならば,
	  ノード\code{X}も同じ根ノード\code{R}から到達可能であることを表す.
\end{itemize}
\end{frame}
%%%%%%%%%%%%%%%%%%%%%%%%%%%%%%%%%%%%%%%%%%%%%%%%%%
%% 改良符号化 (根付き連結制約)
%%%%%%%%%%%%%%%%%%%%%%%%%%%%%%%%%%%%%%%%%%%%%%%%%%
\begin{frame}[fragile]{改良符号化: 根付き連結制約}
\begin{exampleblock}{}\small
\begin{lstlisting}
(5) :- node(X), not 1 { reached(X,R) } 1.
\end{lstlisting}
\end{exampleblock}
\vfill
\begin{itemize}
\item (5) 各ノード\code{X}について,ちょうど1つの根からのみ到達可能であることを意味する.
\end{itemize}
\end{frame}
%%%%%%%%%%%%%%%%%%%%%%%%%%%%%%%%%%%%%%%%%%%%%%%%%%
%% 改良符号化 (非閉路制約)
%%%%%%%%%%%%%%%%%%%%%%%%%%%%%%%%%%%%%%%%%%%%%%%%%%
\begin{frame}[fragile]{改良符号化: 非閉路制約}
\begin{minipage}[c]{1.01\textwidth}
\begin{exampleblock}{}\small
\begin{lstlisting}
(6) :- root(R),
       not 1 #sum{ 1,X:reached(X,R) ;
                  -1,X,Y:inForest(X,Y),reached(X,R),reached(Y,R)
                 } 1.
\end{lstlisting}
\end{exampleblock}
\end{minipage}
\vfill
\begin{itemize}
\item (6) 各連結成分の\structure{\bf ノード数と辺数の差が1}になることを意味する.
\item 各連結成分が\structure{\bf 木の性質}を満たすことにより,サイクルを持たない
	  ことを保証する.
\end{itemize}
\end{frame}
%%%%%%%%%%%%%%%%%%%%%%%%%%%%%%%%%%%%%%%%%%%%%%%%%%
%% ルール数の比較
%%%%%%%%%%%%%%%%%%%%%%%%%%%%%%%%%%%%%%%%%%%%%%%%%%
\begin{frame}{基礎化後のルール数}
  \begin{itemize}
  \item グラフのノード数を$|V|$,根ノードの数を$|R|$とする.
  \end{itemize}
  \begin{table}[t]
    \centering
    %%%%%%%%%%%%%%%%%%%%%%%%%%%%%%%%%%%%%%%%%%%%%%%%%%%%%%%%%%%%%%%%
\chapter{ハミルトン閉路問題および関連問題のASP符号化}\label{chap:proposal}
%%%%%%%%%%%%%%%%%%%%%%%%%%%%%%%%%%%%%%%%%%%%%%%%%%%%%%%%%%%%%%%% 

%%%%
\begin{figure}[h]
  \centering
  \thicklines
  \setlength{\unitlength}{1.2pt}
  \small\footnotesize\scriptsize
  \begin{picture}(280,57)(4,-10)
    \put(  0, 20){\dashbox(50,24){\shortstack{HCP問題\\インスタンス}}}
    \put( 60, 20){\framebox(50,24){変換器}}
    \put(120, 20){\dashbox(50,24){\shortstack{ASPファクト}}}
    \put(120,-10){\dashbox(50,24){\shortstack{ASP符号化\\(論理プログラム)}}}
    \put(180, 20){\framebox(50,24){ASPシステム}}
    \put(240, 20){\dashbox(50,24){\shortstack{HCP問題\\の解}}}
    \put( 50, 32){\vector(1,0){10}}
    \put(110, 32){\vector(1,0){10}}
    \put(170, 32){\vector(1,0){10}}
    \put(230, 32){\vector(1,0){10}}
    \put(170, +2){\line(1,0){4}}
    \put(174, +2){\line(0,1){30}}
  \end{picture}  
\caption{ASP を用いたハミルトン閉路問題(HCP)の解法}
\label{fig:arch}
\end{figure}
%%%%

%\begin{figure}[tbp]
\tikz{
  %1ノード目
  \path[draw=black, fill=blue!20, rounded corners=5pt]%線の設定
  node[at={(0.75,0.75)}] {問題}%文字を入れる
  (0,0) --(1.5,0) --(1.5,1.5) --(0,1.5) --cycle;%外周
  %2ノード目
  \path[draw=black, fill=blue!20, rounded corners=5pt, shift={(3,0)}]
  node[at={(0.75,0.75)}] {
    \begin{tabular}{c}
      ASP\\
      ファクト
    \end{tabular}
  }
  (0,0) --(1.5,0) --(1.5,1.5) --(0,1.5) --cycle;
  %3ノード目文字が複数行
  \path[draw=black, fill=green!20, rounded corners=5pt, shift={(6,0)}]
  node[at={(0.75,0.75)}] {
    \begin{tabular}{c}
      ASP\\
      システム
    \end{tabular}
  }
  (0,0) --(1.5,0) --(1.5,1.5) --(0,1.5) --cycle;
  %4ノード目文字が複数行
  \path[draw=black, fill=blue!20, rounded corners=5pt, shift={(9,0)}]
  node[at={(0.75,0.75)}] {解集合}
  (0,0) --(1.5,0) --(1.5,1.5) --(0,1.5) --cycle;
  %5ノード目文字が複数行
  \path[draw=black, fill=red!20, rounded corners=5pt, shift={(3,-3)}]
  node[at={(0.75,0.75)}] {
    \begin{tabular}{c}
      ASP\\
      符号化
    \end{tabular}
  }
  (0,0) --(1.5,0) --(1.5,1.5) --(0,1.5) --cycle;
  \draw[arrows=->] (1.5,0.75) --(3.0,0.75);
  \draw[arrows=->,shift={(3,0)}] (1.5,0.75) --(3.0,0.75);
  \draw[arrows=->,shift={(6,0)}] (1.5,0.75) --(3.0,0.75);
  \draw[arrows=->] (4.5,-2.25) --(6.0,0.5);
}
\caption{ASPを用いた解法}
\label{aspmethod}
\end{figure}


ASP を用いたハミルトン閉路問題および関連問題の解法について述べる.
図~\ref{fig:arch}に,解法の流れを示す.
与えられたハミルトン閉路問題は ASP ファクトに変換され,
ハミルトン閉路問題を解く ASP 符号化と結合され,
ASP システムによって解が計算される.
本論文では,ASP システムとして{\clingo}を用いる.

%%%%%%%%%%%%%%%%%%%%%%%%%%%%%%%%%%%%%%%%%%%%%%%%%%%%%%%%%%%%%%%%%%%%%%%
\section{ASPファクト形式}
%%%%%%%%%%%%%%%%%%%%%%%%%%%%%%%%%%%%%%%%%%%%%%%%%%%%%%%%%%%%%%%%%%%%%%%

%%%%%%%%%%%%%%%%%%%%%%%%%%%%%%
\begin{figure}[t]
\begin{center}
\begin{tikzpicture}
  %ノード1  
  \draw(4,2) circle (0.5)
  node[at={(4.1,2.1)}] {
    \begin{tabular}{c}
      1
    \end{tabular}
  };
  %ノード2  
  \draw(4,0) circle (0.5)
  node[at={(4.1,0.1)}] {
    \begin{tabular}{c}
      2
    \end{tabular}
  };
  %ノード3  
  \draw(6,2) circle (0.5)
  node[at={(6.1,2.1)}] {
    \begin{tabular}{c}
      3
    \end{tabular}
  };
  %ノード4  
  \draw(6,0) circle (0.5)
  node[at={(6.1,0.1)}] {
    \begin{tabular}{c}
      4
    \end{tabular}
  };
  %ノード5  
  \draw(8,2) circle (0.5)
  node[at={(8.1,2.1)}] {
    \begin{tabular}{c}
      5
    \end{tabular}
  };
  %ノード6  
  \draw(8,0) circle (0.5)
  node[at={(8.1,0.1)}] {
    \begin{tabular}{c}
      6
    \end{tabular}
  };
\draw(4,0.5) --(4,1.5);
\draw(6,0.5) --(6,1.5);
\draw(8,0.5) --(8,1.5);
\draw(4.5,0) --(5.5,0);
\draw(4.5,2) --(5.5,2);
\draw(6.5,0) --(7.5,0);
\draw(6.5,2) --(7.5,2);
\end{tikzpicture}

\caption{入力となる重み付き無向グラフの例}
\label{graphexample}
\end{center}
\end{figure}
%%%%%%%%%%%%%%%%%%%%%%%%%%%%%%

%%%%%%%%%%%%%%%%%%%%%%%%%%%%%%
\lstinputlisting[float=t,caption={%
図~\ref{graphexample}のASPファクト表現},%
captionpos=b,frame=single,label=code:graph_example.lp,%
numbers=none,%
breaklines=true,%
columns=fullflexible,keepspaces=true,%
basicstyle=\ttfamily\scriptsize]{code/graph_example.lp}
%%%%%%%%%%%%%%%%%%%%%%%%%%%%%%


本節では,最短ハミルトン閉路問題の例にとって,
入力となる重み付き無向グラフ(図~\ref{graphexample})の
ASP ファクト形式について説明する.
%
このグラフは,頂点数が6,辺の数が7であり,辺に付けられた値は距離を表す.
コード~\ref{code:graph_example.lp}に,ASPファクト形式を示す.
%
アトム\code{node/1}は頂点,\code{edge/2}は辺,\code{cost/3}は距離を表す.
例えば,\code{cost(1,2,3)}は,辺\code{edge(1,2)}の距離が3であることを
表している.

%%%%%%%%%%%%%%%%%%%%%%%%%%%%%%%%%%%%%%%%%%%%%%%%%%%%%%%%%%%%%%%%%%%%%%%
\section{ハミルトン閉路問題の ASP 符号化}\label{hamiltonianasp}
%%%%%%%%%%%%%%%%%%%%%%%%%%%%%%%%%%%%%%%%%%%%%%%%%%%%%%%%%%%%%%%%%%%%%%%

ハミルトン閉路問題は,与えられたグラフの全頂点をちょうど一度ずつ通る閉
路(ハミルトン閉路)が存在するかどうかを判定する問題である.
$G=(V,E)$にハミルトン閉路が存在する必要十分条件は,
以下の2つの制約を満たす部分グラフ$G'=(V,E')$が存在することである.

\begin{itemize}
\item $G'$の各頂点の次数が2 (次数制約)
\item $G'$が連結である (連結制約)
\end{itemize}

本論文では,前者を\textbf{次数制約},後者を\textbf{連結制約}と呼ぶ.
ハミルトン路問題は,ハミルトン閉路問題から始点と終点が一致するという閉
路の条件を取り除いたものである.
ハミルトン路問題では,次数制約は以下のように変わる.

\begin{itemize}
\item 始点と終点の次数が1,他の頂点の次数が2
\end{itemize}

以下では,ハミルトン閉路問題に対する3つの ASP 符号化
\textsf{undirected},\textsf{directed},\textsf{acyclicity}
を提案する.

%%%%%%%%%%%%%%%%%%%%%%%%%%%%%%%%%%%%%%%%%%%%%%%%%%%%%%%%%%%%%%%%%%%%%%%
\subsection{\textsf{undirected}符号化}
%%%%%%%%%%%%%%%%%%%%%%%%%%%%%%%%%%%%%%%%%%%%%%%%%%%%%%%%%%%%%%%%%%%%%%%

%%%%%%%%%%%%%%%%%%%%%%%%%%%%%%
\lstinputlisting[float=t,caption={%
\textsf{undirected}符号化},%
captionpos=b,frame=single,label=code:hamilton1.lp,%
numbers=left,%
breaklines=true,%
columns=fullflexible,keepspaces=true,%
basicstyle=\ttfamily\footnotesize]{code/hamilton1.lp}
%%%%%%%%%%%%%%%%%%%%%%%%%%%%%%

\textsf{undirected}符号化は,ハミルトン閉路問題の次数制約と連結制約を,
ASP の一貫性制約で表した基本的な符号化である.
コード~\ref{code:hamilton1.lp}に,\textsf{undirected}符号化を示す.
この符号化は,ハミルトン閉路問題とハミルトン路問題の両方に対応している.
符号化中の\code{s}は始点の頂点番号,\code{t}は終点の頂点番号を表し,こ
れらは実行時に与えられる.
ここでは,ハミルトン閉路問題(\code{s}=\code{t})の場合について説明する.

\begin{itemize}
\item 1行目のルールは,各辺\code{edge(X,Y)}に対して,その辺がハミルト
  ン閉路に含まれるかどうかを意味するアトム\code{in(X,Y)}を選択子を用い
  て導入している.
\item 次数制約は3行目のルールで表される.このルールは,
  各頂点\code{node(X)}に対して,その次数の和が2に等しいことを個数制約
  を使って表している.
\item 連結制約は11行目のルールで表される.
ある頂点\code{X}が始点\code{s}から到達可能であることを意味する補助アト
ム\code{reached(X)}を導入する.
8行目のルールは,始点\code{s}が到達可能あることを表している.
9行目のルールは,各辺\code{X}--\code{Y}に対して,その辺がハミルトン閉
路に含まれ(\code{in(X,Y)}),かつ,頂点\code{X}が始点から到
達可能であれば(\code{reached(X)}),\code{Y}も到達可能であることを表している.
10行目は9行目と同様であるが,辺\code{Y}--\code{X}の場合を表している.
11行目のルールは,各頂点\code{node(X)}が始点から到達可能でなければな
らないことを一貫性制約を使って表している.
\end{itemize}

%%%%%%%%%%%%%%%%%%%%%%%%%%%%%%%%%%%%%%%%%%%%%%%%%%%%%%%%%%%%%%%%%%%%%%%
\subsection{\textsf{directed}符号化}
%%%%%%%%%%%%%%%%%%%%%%%%%%%%%%%%%%%%%%%%%%%%%%%%%%%%%%%%%%%%%%%%%%%%%%%

%%%%%%%%%%%%%%%%%%%%%%%%%%%%%%
\lstinputlisting[float=t,caption={%
\textsf{directed}符号化},%
captionpos=b,frame=single,label=code:hamilton2.lp,%
numbers=left,%
breaklines=true,%
columns=fullflexible,keepspaces=true,%
basicstyle=\ttfamily\footnotesize]{code/hamilton2.lp}
%%%%%%%%%%%%%%%%%%%%%%%%%%%%%%

\textsf{directed}符号化は,\textsf{undirected}符号化をベースに,
与えられた無向グラフの各辺$u-v$に対して,2つの弧$u\rightarrow v$と
$v\rightarrow u$を対応させることで有向グラフ化して解く符号化である.
コード~\ref{code:hamilton2.lp}に,\textsf{directed}符号化を示す.
前節と同様に,ハミルトン閉路問題(\code{s}=\code{t})の場合について説明する.

\begin{itemize}
\item 1行目では,無向グラフの有向グラフ化を行う.
  与えられた無向グラフの各辺\code{edge(X,Y)}に対して,
  2つの弧\code{edge(X,Y)},\code{edge(Y,X)}を導入した.
\item 2行目のルールは,各弧\code{edge(X,Y)}に対して,その弧がハミルト
  ン閉路に含まれるかどうかを意味するアトム\code{in(X,Y)}を選択子を用い
  て導入している.
\item 次数制約は4,5行目のルールで表される.
  4行目では,各頂点\code{node(X)}に対して,
  その出次数が1に等しいことを個数制約を使って表している.
  5行目では,入次数について4行目と同様の制約を表す.
\item 連結制約は15行目のルールで表される.
  ある頂点\code{X}が始点\code{s}から到達可能であることを意味する
  補助アトム\code{reached(X)}を導入する.
  13行目のルールは,始点\code{s}が到達可能あることを表している.
  14行目のルールは,各弧\code{X}--\code{Y}に対して,その弧がハミルトン閉路
  に含まれ(\code{in(X,Y)}),かつ,頂点\code{X}が始点から
  到達可能であれば(\code{reached(X)}),\code{Y}も到達可能であることを表している.
  15行目のルールは,各頂点\code{node(X)}が始点から到達可能でなければ
  ならないことを一貫性制約を使って表している.
\item 18行目のルールは,解についての対称性を除去する.
  与えられた無向グラフ上の各ハミルトン閉路に対して,
  それを変換した有向グラフ上のハミルトン閉路は対称な2つが存在する.
  これによる解の重複を防ぐために,18行目のルールは,各弧\code{s}--\code{X},
  \code{Y}--\code{s}がハミルトン閉路に含まれるならば(\code{in(s,X),in(Y,s)}),
  \code{X < Y}でなければならないことを,一貫性制約を用いて表している
\end{itemize}

%%%%%%%%%%%%%%%%%%%%%%%%%%%%%%%%%%%%%%%%%%%%%%%%%%%%%%%%%%%%%%%%%%%%%%%
\subsection{\textsf{acyclicity}符号化}
%%%%%%%%%%%%%%%%%%%%%%%%%%%%%%%%%%%%%%%%%%%%%%%%%%%%%%%%%%%%%%%%%%%%%%%

%%%%%%%%%%%%%%%%%%%%%%%%%%%%%%
\lstinputlisting[float=t,caption={%
\textsf{acyclicity}符号化},%
captionpos=b,frame=single,label=code:hamilton3.lp,%
numbers=left,%
breaklines=true,%
columns=fullflexible,keepspaces=true,%
basicstyle=\ttfamily\footnotesize]{code/hamilton3.lp}
%%%%%%%%%%%%%%%%%%%%%%%%%%%%%%

\textsf{acyclicity}符号化は,\textsf{directed}符号化をベースに,
連結の制約に代わる部分閉路禁止制約を組込み非閉路制約で表現した符号化である.
コード~\ref{code:hamilton3.lp}に,\textsf{acyclicity}符号化を示す.
前節と同様に,ハミルトン閉路問題(\code{s}=\code{t})の場合について説明する.

\begin{itemize}
\item 1行目では,無向グラフの有向グラフ化を行う.
  与えられた無向グラフの各辺\code{edge(X,Y)}に対して,
  2つの弧\code{edge(X,Y)},\code{edge(Y,X)}を導入した.
\item 2行目のルールは,各弧\code{edge(X,Y)}に対して,その弧がハミルト
  ン閉路に含まれるかどうかを意味するアトム\code{in(X,Y)}を選択子を用い
  て導入している.
\item 次数制約は4,5行目のルールで表される.
  4行目では,各頂点\code{node(X)}に対して,
  その出次数が1に等しいことを個数制約を使って表している.
  5行目では,入次数について4行目と同様の制約を表す.
\item 部分閉路禁止制約は14行目のルールで表される.
  このルールは,始点でない各頂点\code{X},\code{Y}について,
  弧\code{X}--\code{Y}がハミルトン閉路に含まれるならば(\code{in(X,Y)}),
  そのような弧の集合をもつグラフが閉路をもたないことを,\code{#edge}宣言を用いて表す.
  ようするに,始点(終点)を含まないような閉路を禁止している.
\item 17行目のルールは,解についての対称性を除去する.
  与えられた無向グラフ上の各ハミルトン閉路に対して,
  それを変換した有向グラフ上のハミルトン閉路は対称な2つが存在する.
  これによる解の重複を防ぐために,17行目のルールは,各弧\code{s}--\code{X},
  \code{Y}--\code{s}がハミルトン閉路に含まれるならば(\code{in(s,X),in(Y,s)}),
  \code{X < Y}でなければならないことを,一貫性制約を用いて表している
\end{itemize}

%%%%%%%%%%%%%%%%%%%%%%%%%%%%%%%%%%%%%%%%%%%%%%%%%%%%%%%%%%%%%%%%%%%%%%% 
\section{最短ハミルトン閉路問題のASP符号化}\label{minexpl}
%%%%%%%%%%%%%%%%%%%%%%%%%%%%%%%%%%%%%%%%%%%%%%%%%%%%%%%%%%%%%%%%%%%%%%% 

%% %%%%%%%%%%%%%%%%%%%%%%%%%%%%%%
%% \lstinputlisting[caption =  最適化,label = minimize]{code/obj_minimize.lp}
%% %%%%%%%%%%%%%%%%%%%%%%%%%%%%%%

%%%%%%%%%%%%%%%%%%%%%%%%%%%%%%
\lstinputlisting[float=t,caption={%
最小化},%
captionpos=b,frame=single,label=code:obj_minimize.lp,%
numbers=left,%
breaklines=true,%
columns=fullflexible,keepspaces=true,%
basicstyle=\ttfamily\footnotesize]{code/obj_minimize.lp}
%%%%%%%%%%%%%%%%%%%%%%%%%%%%%%

最短ハミルトン閉路問題の目的関数は,
ハミルトン閉路を構成する各辺の距離の総和である.
コード\ref{code:obj_minimize.lp}は,
その目的関数の最小化を表す.
このコードは,各辺\code{edge(X,Y)}に対して,その辺がハミルトン閉路に
含まれ(\code{in(X,Y)}),その距離が\code{C}である時に(\code{cost(X,Y,C)}),
\code{C}の総和の最小化を,最小化関数を用いて表している.
.
%%%%%%%%%%%%%%%%%%%%%%%%%%%%%%
\lstinputlisting[float=t,caption={%
重み付き無向グラフの有向グラフ化},%
captionpos=b,frame=single,label=code:cost_both.lp,%
numbers=left,%
breaklines=true,%
columns=fullflexible,keepspaces=true,%
basicstyle=\ttfamily\footnotesize]{code/cost_both.lp}
%%%%%%%%%%%%%%%%%%%%%%%%%%%%%%

符号化directed,acyclicityについては,
与えられた無向グラフの各辺\code{edge(X,Y)}に対して,
2つの弧\code{edge(X,Y)},\code{edge(Y,X)}を導入した.
各辺の距離もこれに対応させるために,コード\ref{code:cost_both.lp}
を追加した.
このルールは,各辺\code{X}--\code{Y}の距離を表す\code{cost(X,Y,C)}について,
\code{cost(Y,X,C)}を導入する.
これにより,与えられた無向グラフの各辺\code{edge(X,Y)}の重み\code{C}が
2つの弧\code{edge(X,Y)},\code{edge(Y,X)}にも付与された.

%%%%%%%%%%%%%%%%%%%%%%%%%%%%%%%%%%%%%%%%%%%%%%%%%%%%%%%%%%%%%%%%%%%%%%% 
\section{コスト制約付きハミルトン閉路のASP符号化}
%%%%%%%%%%%%%%%%%%%%%%%%%%%%%%%%%%%%%%%%%%%%%%%%%%%%%%%%%%%%%%%%%%%%%%% 

%%%%%%%%%%%%%%%%%%%%%%%%%%%%%%
\lstinputlisting[float=t,caption={%
コスト制約},%
captionpos=b,frame=single,label=code:cost_constraint.lp,%
numbers=left,%
breaklines=true,%
columns=fullflexible,keepspaces=true,%
basicstyle=\ttfamily\footnotesize]{code/cost_constraint.lp}
%%%%%%%%%%%%%%%%%%%%%%%%%%%%%%

コスト制約付きハミルトン閉路問題は
ハミルトン閉路問題に,距離の総和が所与の閾値以下 (または以上) であること
を制約条件として付加した問題である.
コード\ref{code:const_constraing.lp}のルールは,その制約を表す.
ルール中の\code{c}は閾値を表し,これは実行時に与えられる.
このルールは,各辺\code{edge(X,Y)}に対して,その辺がハミルトン閉路に
含まれ(\code{in(X,Y)}),その距離が\code{C}である時に(\code{cost(X,Y,C)}),
\code{C}の総和が\code{c}以下でなければならないことを,
一貫性制約と重み付き個数制約を用いて表す.

また,\ref{minexpl}と同様に,
符号化directed,acyclicityについては,
アトム\code{cost}についても有向グラフ化に
対応させるためにコード\ref{code:cost_both.lp}を追加した.
%%%%%%%%%%%%%%%%%%%%%%%%%%%%%%%%%%%%%%%%%%%%%%%%%%%%%%%%%%%%%%%%%%%%%%%

%%% Local Variables:
%%% mode: latex
%%% TeX-master: "paper"
%%% End:

  \end{table}
\end{frame}

%%%%%%%%%%%%%%%%%%%%%%%%%%%%%%%%%%%%%%%%%%%%%%%%%%
%% ASPのコード
%%%%%%%%%%%%%%%%%%%%%%%%%%%%%%%%%%%%%%%%%%%%%%%%%%
\begin{frame}[fragile]{補足 : 根付き全域森 基本符号化}
%%%%%%%%%%%%%%%%%%%%%%%%%%%%%%%%% 
\lstinputlisting[frame=single,label=code:roop,%
xleftmargin=1zw,%
xrightmargin=1zw,%
numbersep=5pt,%
numbers=left,%
breaklines=true,%
columns=fullflexible,keepspaces=true,%
basicstyle=\ttfamily\scriptsize]{code/srf1.lp}
%%%%%%%%%%%%%%%%%%%%%%%%%%%%%%%%%
\end{frame}

\begin{frame}[fragile]{補足 : 遷移問題 シングルショット符号化}

\begin{multicols*}{2}
%%%%%%%%%%%%%%%%%%%%%%%%%%%%%%%%% 
\lstinputlisting[frame=single,label=code:roop,%
xleftmargin=1zw,%
xrightmargin=1zw,%
numbersep=5pt,%
numbers=left,%
breaklines=true,%
columns=fullflexible,keepspaces=true,%
basicstyle=\ttfamily\tiny]{code/trans-const.lp}
%%%%%%%%%%%%%%%%%%%%%%%%%%%%%%%%%
\end{multicols*}
\end{frame}

\begin{frame}[fragile]{補足 : 遷移問題 マルチショット符号化}
\begin{multicols*}{2}
%%%%%%%%%%%%%%%%%%%%%%%%%%%%%%%%%
\lstinputlisting[frame=single,label=code:incmode,% 
xleftmargin=1zw,%
xrightmargin=1zw,%
numbersep=5pt,%
numbers=left,%
breaklines=true,%
columns=fullflexible,keepspaces=true,%
basicstyle=\ttfamily\tiny]{code/dnet-trans.lp}
%%%%%%%%%%%%%%%%%%%%%%%%%%%%%%%%% 
\end{multicols*}
\end{frame}

\backupend


\end{document}
%%% Local Variables:
%%% mode: japanese-latex
%%% TeX-master: t
%%% End:

