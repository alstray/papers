\documentclass[a4j,10pt,dvipdfmx]{jarticle}
\usepackage{multicol}
%%%%% 余白の設定 %%%%%%
\setlength{\textheight}{\paperheight}
\setlength{\topmargin}{-15.4truemm}
\addtolength{\topmargin}{-\headheight}
\addtolength{\topmargin}{-\headsep}
\addtolength{\textheight}{-20truemm}
\setlength{\textwidth}{\paperwidth}
\setlength{\oddsidemargin}{-10.4truemm}
\setlength{\evensidemargin}{\oddsidemargin}
\addtolength{\textwidth}{-30truemm}
%%%%% 行間の設定 %%%%%
\renewcommand{\baselinestretch}{.95}
\newcommand{\mysection}[1]{\vspace{-15pt}\section{#1}} %sectionの上部の隙間を詰める.

\pagestyle{empty}

%%%%% 題目 %%%%%
\title{解集合プログラミングを用いた配電網問題の解法}
%%%%% 氏名 %%%%%
\author{山田 健太郎(番原研究室)}

\date{2021年2月24日}
%\date{}
\begin{document}
\maketitle
\thispagestyle{empty}
\begin{multicols}{1}
%%%%% 本文ここから %%%%%

\section{研究概要}

\textbf{配電網問題}は,求解困難な組合せ最適化問題の一種である.
%一般に配電網は,変電所と家庭や工場を結ぶ電力供給経路のネットワークを指す.
配電網の構成技術は,スマートグリッドや災害時の障害箇所の迂回構成など
を支える重要な基盤技術として期待されている.
配電網問題は,供給経路に関する\textbf{トポロジ制約}と,配電システムに関する
\textbf{電気制約}を満たした上で,損失電力を最小にするスイッチの開閉状態を
求める問題である.
厳密解法として,フロンティア法による解法が提案されており,実用規模の配電網問題に
おいて最適解を発見している\cite{Minato:dnet:ZDD}.

\textbf{解集合プログラミング}
(Answer Set Programming; ASP\cite{inoue08:jssst})は,
論理プログラミングから派生したプログラミングパラダイムである.
ASP言語は,一階論理に基づく知識表現言語の一種であり,論理プログラムはASPの
ルールの有限集合である.ASPシステムは論理プログラムから安定モデル意味論に
基づく解集合を計算するシステムである.近年,SATソルバーの技術を応用した高速な
ASPシステムが確立され,制約充足問題,時間割問題,
システム検証など様々な分野への実用的応用が急速に拡大している.

\textbf{本研究の目的}は,ASP技術を活用した配電網問題ソルバーの開発である.
そのための研究課題として,トポロジ制約のASP符号化の考案,インクリメンタル
解法を活用した遷移問題への拡張,さらには,ASPのみでは解くことが困難とされる
電気制約の実装に取り組む.実用規模の問題を含む配電網問題に提案する手法
を適用して評価し,ASP技術による利点や実用性を明らかにする.


%%%%%% 研究成果 %%%%%%
\mysection{研究成果}

\textbf{トポロジ制約に関する研究(卒業研究).}
%
トポロジ制約を満たす配電網構成は,グラフと根と呼ばれる特別なノードから
\textbf{根付き全域森}\cite{Minato:dnet:netuki}
を求める根付き全域森問題に帰着することができる.
この問題に対し,基本符号化と改良符号化の2種類のASP符号化を考案した.
特に改良符号化は,根付き全域森の連結制約をASPの個数制約で表現しており,
基礎化後のルール数を抑えるよう工夫されており,大規模な問題に対して有効性
が期待できる.提案手法の有効性を評価するために,配電網問題及び,グラフ彩色
問題を基に作成した実用規模の問題を含む問題集を用いて評価実験を行ったところ,
改良符号化は基本符号化よりも多くの問題を解いており,大規模な問題への優位性
を示した.

% また改良符号化での工夫に加えて,問題を有向グラフとすることでノードの入次数の
% 制約によって,根付き全域森の非閉路制約を表現する符号化も新たに考案した.
% 同様の評価実験を行ったところ,求解までのリスタート数が大幅に減少し,問題85問中
% 84問を解く結果を示した.


\textbf{解の遷移問題に関する研究.}
%
根付き全域森問題とその2つの実行可能解が与えられたとき,
ある解から他のもう一つの解へ根付き全域森の制約を満たしながら移る解の遷移問題として,
問題の拡張を行った.この問題に対し,根付き全域森問題の改良符号化の自然な拡張である
シングルショット符号化と,ASPシステム\textit{clingo}のインクリメンタル解法ライブラリ
を用いた\textbf{マルチショット符号化}を考案した.
マルチショット符号化は,同様の探索失敗で獲得した学習節を保持するため,大規模な問題を
効率的に解くことができる.評価実験では,各ステップ間で変更可能な辺の数を1つに制限し,
最短ステップ長での変更手順を求めた.結果として,マルチショット符号化は,シングルショット
符号化と比較して,平均で3.3倍の求解時間の高速化を実現した.また,最短ステップ長が10の問題も
解けることを確認した.

%%%%%% まとめ %%%%%%
\mysection{まとめ}

本稿では,解集合プログラミングを用いた配電網問題の解法に関して,研究概要と
これまでの研究成果を示した.

今後の課題として,トポロジ制約のASP符号化の改良や,電気制約を背景理論付き
ASP(ASP Modulo Theories)を用いて実装することが挙げられる.


%%%%%% 業績 %%%%%%
\mysection{研究業績}

\begin{itemize}
 \setlength{\parskip}{0pt} % 段落間
 \setlength{\itemsep}{2pt} % 項目間
 \small
 \item 発表.\textbf{学生奨励賞受賞.}「解集合プログラミングを用いた配電網問題の解法に関する一考察」.
	   2020年度日本ソフトウェア科学会第37回大会. 
 \item 発表.「解集合プログラミングを用いた配電網問題の解法に関する一考察」.
	   NII共同研究「巨大な制約充足問題を解くためのSAT型CSPソルバーの研究開発」.
 \item 発表.「解集合プログラミングを用いた配電網問題の解法」.
	   基盤(A)「離散構造処理系に基づく列挙と最適化の統合的技法の研究」プロジェクト近況報告&自由討論会.
\end{itemize}


%%%%% 参考文献 %%%%%
% {
% \small
% \setlength{\itemsep}{-2pt} 
%  \bibliographystyle{junsrt}
%  \bibliography{bachelor,aisat}    % 参考文献リスト
%}

% 著者の省略のため手書き
\begin{thebibliography}{10}
\small
\setlength{\itemsep}{-2pt} % 項目間
\bibitem{Minato:dnet:ZDD} 井上武 ほか. フロンティア法による電力網構成制御. 
		 オペレーションズ・リサーチ, Vol. 57, No. 11, pp. 610-615, nov 2012.
\bibitem{Minato:dnet:netuki} 川原純, 湊真一. グラフ列挙索引化技法の種々の問題への適用. 
		 オペレーションズ・リサーチ, Vol. 57, No. 11, pp. 604-609, nov 2012. 
\bibitem{inoue08:jssst} 井上克巳, 坂間千秋: 論理プログラミングから解集合プログラミングへ, 
		 コンピュータソフトウェア, Vol. 25, No. 3(2008), pp. 20-32.
\end{thebibliography}

%%%%% 本文ここまで %%%%%
\end{multicols}
\vfill
\noindent
{\gt コメント欄}
{\footnotesize
(本資料をそのまま発表者に返却します.コメント欄以外にもコメントを書いていただいてもかまいません.)}
\\
\fbox{\begin{minipage}{\textwidth}\noindent\\\\\end{minipage}}	
\end{document}
