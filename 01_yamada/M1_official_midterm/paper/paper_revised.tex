\documentclass[a4j,10pt,dvipdfmx]{jarticle}
\usepackage{multicol}
%%%%% 余白の設定 %%%%%%
\setlength{\textheight}{\paperheight}
\setlength{\topmargin}{-15.4truemm}
\addtolength{\topmargin}{-\headheight}
\addtolength{\topmargin}{-\headsep}
\addtolength{\textheight}{-20truemm}
\setlength{\textwidth}{\paperwidth}
\setlength{\oddsidemargin}{-10.4truemm}
\setlength{\evensidemargin}{\oddsidemargin}
\addtolength{\textwidth}{-30truemm}
%%%%% 行間の設定 %%%%%
\renewcommand{\baselinestretch}{.95}
\newcommand{\mysection}[1]{\vspace{-15pt}\section{#1}} %sectionの上部の隙間を詰める.

\pagestyle{empty}

%%%%% 題目 %%%%%
\title{解集合プログラミングを用いた配電網問題の解法}
%%%%% 氏名 %%%%%
\author{山田 健太郎(番原研究室)}

\date{2021年2月24日}
%\date{}
\begin{document}
\maketitle
\thispagestyle{empty}
\begin{multicols}{1}

%%%%%%%%%%%%%%%%%%%%%%%%%%%%%%%%%%%%%%%%%%%%%%%%%%%%%%%%%%%%%%% 
\section{研究概要}
%%%%%%%%%%%%%%%%%%%%%%%%%%%%%%%%%%%%%%%%%%%%%%%%%%%%%%%%%%%%%%%
\textbf{配電網}は変電所と家庭や工場を繋ぐ電力供給ネットワークである.
配電網の構成技術はスマートグリッドや,災害時の障害箇所の迂回構成などを
支える重要な基盤技術である.
\textbf{配電網問題は},供給経路に関するトポロジ制約と,電流・電圧に関
する電気制約を満たしつつ,電力の損失を最小にするスイッチの開閉状態を求
める問題である.
配電網問題は求解困難な組合せ最適化問題の一種であり,これまでフロンティ
ア法を用いた解法等が提案されている~\cite{Minato:dnet:ZDD}.

\textbf{解集合プログラミング}
(ASP)は,
% (ASP~\cite{GelfondL88})は,
論理プログラミングから派生したプログラミングパラダイムである.
ASP 言語は,一階論理に基づく知識表現言語の一種である.
ASP システムは安定モデル意味論に基づく解集合を計算するシステムである.
近年,SAT 技術を応用した高速な ASP システムが確立され,
システム検証,システム生物学など様々な分野への実用的応用が急速に拡大し
ている.

\textbf{本研究の目的}は,ASP 技術を活用して,大規模な配電網問題を効率
良く解くシステムを構築することである.
配電網問題の ASP 符号化,
電気制約の効率的な実装,
配電網問題の遷移問題への拡張
を中心に研究開発を進める.
実用規模のベンチマーク問題を使って手法およびシステムを評価し,
ASP の特長を活かした配電網の構成技術の利点・実用性を明らかにする.

%%%%%%%%%%%%%%%%%%%%%%%%%%%%%%%%%%%%%%%%%%%%%%%%%%%%%%%%%%%%%%%
\section{研究成果}
%%%%%%%%%%%%%%%%%%%%%%%%%%%%%%%%%%%%%%%%%%%%%%%%%%%%%%%%%%%%%%%
\textbf{配電網問題の ASP 符号化 (卒業研究).}
%
トポロジ制約のみの配電網問題は,グラフと根と呼ばれる特別なノードから
根付き全域森を求める部分グラフ探索問題(以下,根付き全域森問題と呼ぶ)
に帰着できる.
この根付き全域森問題に対し,基本符号化と改良符号化の2種類の ASP 符号化
を考案した.
特に,改良符号化は,根付き全域森の連結制約を ASP の個数制約で表現する
ことにより,基礎化後のルール数を抑えるよう工夫されている.
%これにより,大規模な問題に対して有効性が期待できる.

提案手法の有効性を評価するために,配電網問題のベンチマーク問題集,
および,Graph Coloring and its Generalizations で公開されているグラフ
彩色問題をベースに独自に生成した配電網問題を用いて評価実験を行った.
その結果,
改良符号化は,基本符号化と比較して,より多くの問題を高速に解いた.
% また,改良符号化は,辺の数が 40,000 を超えるような問題も解けており,
% 大規模な問題に対する有効性が確認できた.

\textbf{遷移問題への拡張.}
配電網問題の解の遷移問題とは,
配電網問題とその二つの実行可能解が与えられたとき,一方の実行可能解
から他方の実行可能解へ,変更可能なスイッチの数が$k$個以下という遷移制
約を満たしつつ,実行可能解のみを経由して到達できるかを判定する問題である.
この問題は,配電網の構成制御における障害時のスイッチの切替手順を求める
問題に応用できる.
% 配電網問題の解の遷移問題は,トポロジ制約のみの場合,
% 根付き全域森問題の解の遷移問題に帰着できる.

この遷移問題に対して,
改良符号化の自然な拡張である
シングルショット符号化と,
ASPシステム\textit{clingo}のインクリメンタル解法ライブラリを用いた
\textbf{マルチショット符号化}を考案した.
マルチショット符号化は,
ASP システムが同様の探索失敗を避けるために獲得した学習節を保持すること
で,各遷移過程において無駄な探索を行わないように工夫されている.

提案手法の有効性を評価するために,
DNET で公開されている実用規模の配電網問題
とその実行可能解を用いて評価実験を行った.
その結果,
マルチショット符号化は,シングルショット符号化と比較して,すべての問題
をより高速に解いており,平均で3.3 倍の高速化を実現した.

%%%%%%%%%%%%%%%%%%%%%%%%%%%%%%%%%%%%%%%%%%%%%%%%%%%%%%%%%%%%%%%
\section{まとめと今後の課題}
%%%%%%%%%%%%%%%%%%%%%%%%%%%%%%%%%%%%%%%%%%%%%%%%%%%%%%%%%%%%%%%
本稿では,解集合プログラミングを用いた配電網問題の解法に関して,研究概
要とこれまでの研究成果を示した.
%
今後の課題として,トポロジ制約のASP符号化の改良や,電気制約を背景理論付き
ASP(ASP Modulo Theories)を用いて実装することが挙げられる.

%%%%%%%%%%%%%%%%%%%%%%%%%%%%%%%%%%%%%%%%%%%%%%%%%%%%%%%%%%%%%%%%%%%%
\section{受賞}
%%%%%%%%%%%%%%%%%%%%%%%%%%%%%%%%%%%%%%%%%%%%%%%%%%%%%%%%%%%%%%%%%%%%
\noindent\textbf{学生奨励賞}\\
山田健太郎, 湊真一, 番原睦則.
解集合プログラミングを用いた配電網問題の解法に関する一考察. 
日本ソフトウェア科学会第37回大会講演論文集,
2020年9月9日. 

% %%%%%% 業績 %%%%%%
% \section{研究業績}

% \begin{itemize}
%  \setlength{\parskip}{0pt} % 段落間
%  \setlength{\itemsep}{2pt} % 項目間
%  \small
%  \item 発表.\textbf{学生奨励賞受賞.}「解集合プログラミングを用いた配電網問題の解法に関する一考察」.
% 	   2020年度日本ソフトウェア科学会第37回大会. 
%  % \item 発表.「解集合プログラミングを用いた配電網問題の解法に関する一考察」.
%  %           NII共同研究「巨大な制約充足問題を解くためのSAT型CSPソルバーの研究開発」.
%  % \item 発表.「解集合プログラミングを用いた配電網問題の解法」.
%  %           基盤(A)「離散構造処理系に基づく列挙と最適化の統合的技法の研究」プロジェクト近況報告&自由討論会.
% \end{itemize}


%%%%% 参考文献 %%%%%
% {
% \small
% \setlength{\itemsep}{-2pt} 
%  \bibliographystyle{junsrt}
%  \bibliography{bachelor,aisat}    % 参考文献リスト
%}

% 著者の省略のため手書き
\begin{thebibliography}{10}
\small
\setlength{\itemsep}{-2pt} % 項目間
\bibitem{Minato:dnet:ZDD}
  井上武, 高野圭司, 渡辺喬之, 川原純, 吉仲亮, 岸本章宏, 津田宏治, 湊真一, 林泰弘.
  フロンティア法による電力網構成制御,
  オペレーションズ・リサーチ, Vol.57, No.11(2012), pp.610--615.
% \bibitem{GelfondL88}
%   Michael Gelfond and Vladimir Lifschitz.
%   The Stable Model Semantics for Logic Programming.
%   {\it In Proceedings of the Fifth International Conference and Symposium on Logic Programming}, 
%   MIT Press, pp.1070--1080, 1988.
\end{thebibliography}

%%%%% 本文ここまで %%%%%
\end{multicols}
\vfill
\noindent
{\gt コメント欄}
{\footnotesize
(本資料をそのまま発表者に返却します.コメント欄以外にもコメントを書いていただいてもかまいません.)}
\\
\fbox{\begin{minipage}{\textwidth}\noindent\\\\\end{minipage}}	
\end{document}

%%% Local Variables:
%%% mode: japanese-latex
%%% TeX-master: t
%%% End:
