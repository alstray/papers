\documentclass{abst}

\begin{document}

%%%%%%%%%%%%%%%%%%%%%%%%%%%%%%%%%%%%%%%%%%%%%%%%%%%%%%%%%%%%%%%%%%%
\年度{令和3年度}
\研究室名{番原}
\氏名{加藤聖人}
\卒論題目{%
  解集合プログラミングを用いたクイーン支配問題の解法に関する考察}
%%%%%%%%%%%%%%%%%%%%%%%%%%%%%%%%%%%%%%%%%%%%%%%%%%%%%%%%%%%%%%%%%%%
\卒論要旨{%

% 無向グラフ$G=(V,E)$の頂点の部分集合$S\subset V$に対して,
% 任意の頂点$u \in V\setminus S$にも辺$(u,v) \in E$が存在し,
% $v \in S$を満たすとき,
% $S$を$G$の\textbf{支配集合 (dominating set)}という.
% 支配集合の要素数をサイズという.
% サイズが最小の支配集合を最小支配集合という.
% 最小支配集合のサイズを$G$の
% \textbf{支配数}といい$\gamma(G)$で表す.
% % 
% \textbf{支配集合問題}は,
% グラフ$G$と正の整数$k$が与えられたとき,
% サイズ$k$の$G$の支配集合が存在するかどうかを判定する問題である.
% 支配集合問題は\textbf{NP完全}に属する問題である~\cite{Jhonson79}.
% 支配集合問題は,スケジューリング,配置問題など多くの現実の問題に応用さ
% れている~\cite{Haynes98,Haynes98Advanced}
% % グラフの彩色や超グラフの被覆に関係することも知られている.

% \textbf{クイーン支配問題(Queen Domination Problem; QDP)}は,
% $n \times n$のクイーングラフ$Q_{n}$と正の整数$k$が与えられたとき,
% サイズが$k$である$Q_{n}$の支配集合が存在するかどうかを判定する支配集合
% 問題の一種である.
% \textbf{クイーングラフ}とは,
% $n\times n$のチェス盤について各マスを頂点とし,
% クイーンが移動できるマス同士が辺で結ばれているグラフである.
% クイーングラフの支配数は,1862年に文献~\cite{Jaenisch62}で
% $\gamma(Q_8)=5$が示されてから研究されており,
% その後,$n=3,11$を除いて$n \leq 132$に対して,
% $\lceil n/2 \rceil \leq \gamma(Q_n) \leq \lceil n/2 \rceil +1$
% であることが証明されている~\cite{Ostergard01}.
% また,THE ON-LINE ENCYCLOPEDIA OF INTEGER SEQUENCES には,
% $1\leq n\leq 25$に対する$\gamma(Q_n)$
% が掲載されている~\footnote{\url{https://oeis.org/A075458}}.

% \textbf{解集合プログラミング(Answer Set Programming; ASP}\cite{%
%   Baral03:cambridge,%
%   Gelfond88:iclp,%
%   Inoue08:jssst,%
%   Niemela99:amai})
% は,論理プログラミングから派生した比較的新しいプログラミングパラダイムである.
% ASP言語は,一階論理に基づく知識表現言語の一種であり,
% 論理プログラムは ASP のルールの有限集合である.
% ASP システムは論理プログラムから安定モデル意味論に基づく解集合を計算するシステムである.
% 近年,SAT技術を応用した高速 ASP システムが開発され,スケジューリング,
% プランニング,システム生物学,システム検証,制約充足問題,
% 制約最適化問題など様々な分野への実用的応用が急速に拡大している.
% クイーン支配問題に対して ASP を用いる利点としては,
% ASP 言語の高い表現力を活かしてクイーン支配問題の制約を簡潔に記述できる
% 点が挙げられる.

% 本論文では,解集合プログラミング(ASP)を用いたクイーン支配問題の解法について述べる.
% クイーン支配問題に対して,基本符号化,改良符号化,部分和符号化の3種類
% の ASP 符号化を考案した.
% %
% 基本符号化は,チェス盤の各マスに1つ以上のクイーンが移動できることを,
% ASP の一貫性制約を用いて表現した符号化である.
% %
% 改良符号化は,各行,各列,各対角線(右上がりと右下がりの2種類)に対して
% クイーンが配置されているかどうかを表す補助アトムを導入し,
% 基本符号化で複数回出現する制約式をまとめた符号化である.
% %
% 部分和符号化は,各行,各列,各対角線に対して配置されているクイーンの個
% 数を表す補助アトムを導入し,それぞれの和がサイズ$k$に一致することをASP
% の個数制約を用いて表した符号化である.

% 考案した符号化の有効性を評価するために,
% $n$次のクイーン支配問題 $(1\leq n\leq 20)$について,最も難しい
% サイズ$\gamma(Q_n)$と$\gamma(Q_n)-1$の2種類を用いて実行実験を行なった.
% その結果,部分和符号化は,$\gamma(Q_{16})$まで解くことができ,
% 他の符号化と比較して,その優位性を確認できた.

}

\end{document}
