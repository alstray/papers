\documentclass[dvipdfmx,10pt]{beamer}
%%%% Packages %%%%%
\usepackage{bxdpx-beamer}% dvipdfmxなので必要
\usepackage{graphicx}
\usepackage{color}
\usepackage{skak}
% \usepackage{amsmath,amssymb,amsthm}
% \usepackage{multirow}
% \usepackage{url}
% \usepackage{tikz}
% \usepackage{alltt}
% \usepackage{bm}
% \usepackage{listings,jlisting}
% \usepackage{listings}
% \lstset{
%  basicstyle=\ttfamily\scriptsize,
%  keepspaces=true,
%  escapechar=|,
%  columns=[l]{fullflexible}
% }

%%%% Fonts %%%%%
\renewcommand{\kanjifamilydefault}{\gtdefault}
% \usepackage{otf} % otfパッケージ
\usepackage{tikz}
\usetikzlibrary{matrix}
\usetikzlibrary{shadows}
\usetikzlibrary{positioning}
\usepackage[deluxe]{otf} 
\usepackage{txfonts} % 数式・英文ローマン体を Lxfont にする
% \usepackage[T1]{fontenc} % 8bit フォント
% \usepackage{minijs}
% \usepackage{textcomp} % 欧文フォントの追加
% \usepackage[utf8]{inputenc} % 文字コードをUTF-8

%%%% Beamer %%%%%
\usetheme{Warsaw}
%\useinnertheme{rectangles}
%\useoutertheme{smoothbars}
%\setbeamercolor{enumerate}{fg=white, bg=black}
%\setbeamertemplate{frametitle}[default][center]
\setbeamertemplate{navigation symbols}{} % 右下のアイコンを消す
\useoutertheme{shadow}                 % 箱に影をつける
\usefonttheme{professionalfonts}       % 数式の文字を通常の LaTeX と同じにする
% \setbeamercovered{transparent} % 好みに応じてどうぞ
\setbeamertemplate{footline}[frame number]
\setbeamercolor{page number in head/foot}{fg=black} % ページ数を表示する
% \setbeamerfont{footline}{size=\normalsize,series=\bfseries}
% \setbeamerfont{footline}{size=\scriptsize,series=\mdseries}
% \setbeamercolor{footline}{fg=black,bg=black}
% \setbeamertemplate{blocks}[rounded][shadow=true]
\setbeamertemplate{items}[ball]
% exclude apprendix slides from framenumber %
\newcommand{\backupbegin}{
   \newcounter{framenumberappendix}
   \setcounter{framenumberappendix}{\value{framenumber}}
}
\newcommand{\backupend}{
   \addtocounter{framenumberappendix}{-\value{framenumber}}
   \addtocounter{framenumber}{\value{framenumberappendix}} 
}

% \setbeamertemplate{enumerate items}[default]
% \setbeamerfont{alerted text}{series=\bfseries}
\begin{document}
\title{解集合プログラミングを用いた\\クイーン支配問題の解法に関する考察}
\author{101830080 \quad 加藤 聖人}
\date{2021年度 卒業研究発表会 \\ 2022年2月18日}
\institute{番原研究室}

%
%表紙
%

\begin{frame}\frametitle{}
 \titlepage
\end{frame}

%
%支配集合問題について
%

\begin{frame}\frametitle{支配集合問題}
 \begin{block}{支配集合}
%   無向グラフ$G=(V,E)$の頂点の部分集合$S\subset V$に対して,
% 任意の頂点$u \in V\setminus S$にも辺$(u,v) \in E$が存在し,
% $v \in S$を満たすとき,$S$を$G$の\structure{支配集合}という.
無向グラフ$G=(V,E)$の頂点の部分集合$S\subset V$と,
その隣接頂点の集合との和集合が$V$と一致するとき,
$S$を$G$の\structure{支配集合}という.

  \begin{itemize}
   \item 支配集合の要素数を\structure{サイズ}という.
   \item サイズが最小の支配集合をグラフ$G$の\structure{最小支配集合}という.
   \item 最小支配集合のサイズをグラフ$G$の\alert{支配数}といい,
$\gamma(G)$で表す.
  \end{itemize}
 \end{block}
 \begin{block}{支配集合問題}
  グラフ$G$と正の整数$k$が与えられたとき,サイズが$k$の$G$の
支配集合が存在するかどうかを判定する問題を\structure{支配集合問題}という.
  \begin{itemize}
   \item 支配集合問題はNP完全であることが知られている.
   \item 支配集合問題は,スケジューリング,電波塔配置問題など多くの現実の
	 問題に応用されている.
  \end{itemize}
 \end{block}
\end{frame}

%
%クイーン支配問題
%

\begin{frame}\frametitle{クイーン支配問題(Queen Domination Problem; QDP)}
 \begin{alertblock}{} \centering
    本発表では,支配集合問題のインスタンスの一種である\\
    \alert{クイーン支配問題}を対象とする.
  \end{alertblock}
 \begin{block}{}
  サイズ$n\times n$のクイーングラフ$Q_n$と正の整数$k$が与えられたとき,
  サイズ$k$の$Q_n$の支配集合が存在するかどうかを
  判定する問題を\structure{クイーン支配問題}という.
  \begin{itemize}
   \item クイーングラフ$Q_n$は$n\times n$のチェス盤
	 について各マスを頂点とし,クイーンが移動できる
	 マス同士が辺で結ばれているグラフである.
   \item つまり,$n \times n$の盤面に$k$個のクイーンを
	 置いたとき,クイーンを移動させて全てのマスに
	 アタックできるかを判定する.
  \end{itemize}
 \end{block}
\end{frame}
 
 
%
%クイーン支配問題の例
%

\begin{frame}{クイーン支配問題の例:$Q_5$}
  \begin{exampleblock}{}
  \begin{center}
   \scalebox{1.3}{
   \begin{tikzpicture}
 \draw[gray] (-1.25,-1.25)--(-1.25,1.25);
 \draw[gray] (-0.75,-1.25)--(-0.75,1.25);
 \draw[gray] (-0.25,-1.25)--(-0.25,1.25); 
 \draw[gray] (0.25,-1.25)--(0.25,1.25); 
 \draw[gray] (0.75,-1.25)--(0.75,1.25); 
 \draw[gray] (1.25,-1.25)--(1.25,1.25); 
 \draw[gray] (-1.25,-1.25)--(1.25,-1.25); 
 \draw[gray] (-1.25,-0.75)--(1.25,-0.75); 
 \draw[gray] (-1.25,-0.25)--(1.25,-0.25); 
 \draw[gray] (-1.25,0.25)--(1.25,0.25); 
 \draw[gray] (-1.25,0.75)--(1.25,0.75); 
 \draw[gray] (-1.25,1.25)--(1.25,1.25);
 \foreach \x in {-1,-0.5,0,0.5,1}
  \foreach \y in {-1,-0.5,0,0.5,1} 
  \fill (\x,\y) circle (0.03);
 \draw[magenta] (-1.25,1)--(1.25,1);
 \draw[magenta] (-1,1.25)--(-1,-1.25);
 \draw[magenta] (-1.25,1.25)--(1.25,-1.25);
 \draw[magenta] (-1.25,0.75)--(-0.75,1.25);
 \draw[blue] (-1.25,-1)--(1.25,-1);
 \draw[blue] (-0.5,-1.25)--(-0.5,1.25);
 \draw[blue] (-0.25,-1.25)--(-1.25,-0.25);
 \draw[blue] (-0.74,-1.25)--(1.25,0.74);
 \draw[yellow] (1,1.25)--(1,-1.25);
 \draw[yellow] (-1.25,0.5)--(1.25,0.5);
 \draw[yellow] (0.25,1.25)--(1.25,0.25);
 \draw[yellow] (-0.76,-1.25)--(1.25,0.76);
 \fill[magenta] (-1,1) circle (0.03);
 \fill[yellow] (1,0.5) circle (0.03);
 \fill[blue] (-0.5,-1) circle (0.03);
  \matrix[matrix of nodes,nodes={inner sep=0pt,text width=0.5cm,
align=center,minimum height=0.43cm}]{
 \textcolor{magenta}{\symqueen} & \quad & \quad & \quad & \quad \\
 \quad & \quad & \quad & \quad & \textcolor{yellow}{\symqueen} \\
 \quad & \quad & \quad & \quad & \quad \\
 \quad & \quad & \quad & \quad & \quad \\
 \quad & \textcolor{blue}{\symqueen} & \quad & \quad & \quad \\};
\end{tikzpicture}
 

   }
  \end{center}
 \end{exampleblock}
 上の図は$Q_5$の最小支配集合の例である.
 \begin{itemize}
  \item 3個のクイーンを置いたとき,クイーンを
	移動させてすべてのマスにアタック可能である.
  \item 2個以下のクイーンを置いたとき,クイーンを移動させて
	全てのマスにアタックすることは不可能である.
  \item したがって,$\gamma(Q_{5})=3$となる.
 \end{itemize}
\end{frame}


%
%クイーン支配問題の支配数等に関する背景知識
%

\begin{frame}{クイーン支配問題の支配数}
  \begin{itemize}
    \item クイーン支配問題の支配数は,1862年に文献[Jaenisch,1862]で
	    $\gamma(Q_8)=5$が示されてから研究されている.
    \item $n=3,11$を除いた$n \leq 132$で $\lceil n/2 \rceil 
	    \leq \gamma(Q_{n}) \leq \lceil n/2 \rceil +1$
	 であることが証明されている[\"{O}sterg{\aa}rd,Weakley,2001].
%    \item THE ON-LINE ENCYCLOPEDIA OF INTEGER SEQUENCES には,
%    $1\leq n\leq 25$に対する$\gamma(Q_n)$
%    が掲載されている~\footnote{\url{https://oeis.org/A075458}}.
  \end{itemize}
 \begin{exampleblock}{$Q_{n}$の支配数$(1 \leq n \leq 20)$}
  \centering
  \begin{tabular}{c|c||c|c||c|c||c|c}%\hline
    $n$ & $\gamma(Q_{n})$ & $n$ & $\gamma(Q_{n})$ &$n$ & $\gamma(Q_{n})$ &$n$ & $\gamma(Q_{n})$ \\ \hline
    1 &1 &6 &3 &11 &5 &16 &9 \\ %\hline
    2 &1 &7 &4 &12 &6 &17 &9 \\ %\hline
    3 &1 &8 &5 &13 &7 &18 &9 \\ %\hline
    4 &2 &9 &5 &14 &8 &19 &10 \\ %\hline
    5 &3 &10 &5 &15 &9 &20 &11 \\ %\hline
  \end{tabular}
 \end{exampleblock}
\end{frame}

%
%ASPについて
%

\begin{frame}\frametitle{解集合プログラミング(Answer Set Programming; ASP)}
 \begin{itemize}
  \item \structure{ASP言語}は一階論理に基づいた知識表現言語の一種である.
  \item \structure{論理プログラム}は,ASP のルールの有限集合である.
  \item \structure{ASPシステム}は論理プログラムから
	安定モデル意味論[Gelfond and Lifschitz '88]に基づく
	解集合を計算するシステムである.
  \item 近年ではSAT技術を応用した高速ASPシステムが実現され,
	システム検証,プランニング,システム生物学など様々な
	分野への応用が拡大している.
 \end{itemize}
 \begin{alertblock}{クイーン支配問題に対してASPを用いる利点}
  \begin{itemize}
   \item ASP言語の高い表現力を活かし,クイーン支配問題の制約を簡潔に記述可能.
   % \item 個数制約を用いて,部分和を表す制約を簡潔に記述可能.
   % \item 高速な解列挙が可能.
  \end{itemize}
 \end{alertblock}
\end{frame}
 
%
%研究目的と研究内容
%

\begin{frame}\frametitle{研究目的}
 \begin{alertblock}{研究目的}
  ASP技術を活用し,支配集合問題を効率よく解くソルバーの実現.
 \end{alertblock}
 \begin{block}{研究内容}
  \begin{enumerate}
   \item クイーン支配問題を解く,3種類のASP符号化を考案.
	 \begin{itemize}
	  \item 基本符号化,改良符号化,部分和符号化.
	  \item 先行研究[山本,2021]で提案された制約モデルを参考に考案.
	 \end{itemize}
   \item $n$次のクイーン支配問題と既知の支配数を用いて,3種のASP符号化の
	 評価実験を行なった.
  \end{enumerate}
 \end{block}
\end{frame}

%
%符号化3つ
%

\begin{frame}{提案する符号化}
 % \begin{block}{クイーン支配問題の表現}
 %  与えられたクイーングラフ$Q_n$上に$k$個のクイーンを配置したとき,
 %  以下の制約を満たすならばクイーングラフ$Q_{n}$にサイズが$k$の
 %  支配集合が存在する.
 %  \begin{itemize}
 %   \item $Q_n$上のどのマスにも,1つ以上のクイーンが移動
 % 	 できなければならない(\alert{被覆制約})
 %  \end{itemize}
 % \end{block}
 \begin{block}{}\centering
   クイーン支配問題を解く3つのASP符号化を提案した.
 \end{block}
 \begin{enumerate}
   \item \alert{基本符号化}
   \begin{itemize}
     \item クイーン支配問題の制約を,ASP の一貫性制約を
      用いて簡潔に表現.
   \end{itemize}
   \item \alert{改良符号化}
   \begin{itemize}
      \item 各行,各列,各対角線に対して,
        クイーンが配置されているかどうかを表す補助アトム
        を導入する.
      \item 基本符号化で複数回出現する制約式をまとめている.
   \end{itemize}
   \item \alert{部分和符号化}
   \begin{itemize}
      \item 各行,各列,各対角線に対して,
        クイーンの個数を表す補助アトムを導入する.
      \item 行,列,対角線のそれぞれの和が
        サイズ$k$に一致することを,ASPの
        個数制約を用いて表現している.
   \end{itemize}
 \end{enumerate}
\end{frame}

%
%実験内容
%
 
\begin{frame}\frametitle{実験概要}
 \begin{block}{}
  提案した3種の符号化について,$n$次のクイーン支配問題
  と既知の支配数を用いて,支配集合の有無を判定した.
 \end{block}
 \begin{itemize}
  \item \structure{比較するASP符号化:}
	\begin{itemize}
	 \item 基本符号化
	 \item 改良符号化
	 \item 部分和符号化
	\end{itemize}
  \item \structure{対象とする問題:}
	\begin{itemize}
	 \item クイーングラフ$Q_{n} \qquad (1 \leq n \leq 20)$
	 \item $k=\gamma(Q_{n})$\quad (SAT),$k=\gamma(Q_{n})-1$\quad (UNSAT)
	\end{itemize}
  \item \structure{使用ASPソルバ:} \textit{clingo-5.5.0}
  \item \structure{実験環境:} Mac mini, 3.2GHz 6コア Intel Core i7, 64GBメモリ
  \item \structure{制限CPU時間:} 3600 (sec)
 \end{itemize}
\end{frame}

%
%実験結果(基本1,改良1,部分和2)
%

\begin{frame}\frametitle{実験結果}
 \begin{block}{}
  解の有無の判定に要したCPU時間.
 \end{block}
 \begin{columns}
  \begin{column}{0.50\textwidth}
   \begin{table}[htbp]
    \caption{SATの実験結果}
    \scalebox{0.7}{
     \centering 
 \begin{tabular}{c|c|r|r|r} \hline
  $n$ & $k$ & 基本 & 改良 & 部分和 \\ \hline
  10 & 4 & 14.400 & 21.232 & 3.005 \\
  11 & 4 & 54.674 & 123.488 & 32.540 \\
  12 & 5 & T.O. & 56.468 & 2.147 \\
  13 & 6 & T.O. & 1281.823 & 22.412 \\  
  14 & 7 & 2797.041 & 772.276 & 10.044 \\   
  15 & 8 & T.O. & T.O. & 275.683 \\  
  16 & 8 & T.O. & T.O. & \textbf{734.878} \\
  17 & 8 & T.O. & T.O. & T.O. \\
  18 & 8 & T.O. & T.O. & T.O. \\
  19 & 9 & T.O. & T.O. & T.O. \\
  20 & 10 & T.O. & T.O. & T.O. \\ \hline
 \end{tabular}}
   \end{table}
  \end{column}
  \begin{column}{0.50\textwidth}
   \begin{table}[htbp]
    \caption{UNSATの実験結果}
    \scalebox{0.7}{
     \centering 
 \begin{tabular}{c|c|r|r|r} %\hline
  $n$ & $k$ & 基本 & 改良 & 部分和 \\ \hline
  10 & 4 & 10.280 & 8.929 & \alert{2.612} \\
  11 & 4 & 28.184 & 24.537 & \alert{3.427} \\
  12 & 5 & 2706.092 & 2673.241 & \alert{337.127} \\
  13 & 6 & T.O. & T.O. & T.O. \\  
  14 & 7 & T.O. & T.O. & T.O. \\   
  15 & 8 & T.O. & T.O. & T.O. \\  
  16 & 8 & T.O. & T.O. & T.O. \\
  17 & 8 & T.O. & T.O. & T.O. \\
  18 & 8 & T.O. & T.O. & T.O. \\
  19 & 9 & T.O. & T.O. & T.O. \\
  20 & 10 & T.O. & T.O. & T.O. \\ %\hline
 \end{tabular}}
   \end{table}
  \end{column}
 \end{columns} 
 \begin{itemize}
  \item SATの問題では,\structure{部分和符号化}が$n=16$まで解き,
	その優位性を確認できた.
  \item UNSATの問題では,解けた問題数に差はなかったが,解なしが得られる
	までにかかった時間は\structure{部分和符号化}が最も早かった.
 \end{itemize}
\end{frame}

%
%まとめと今後の課題
%

\begin{frame}\frametitle{まとめ}
  \begin{itemize}
   \item \structure{クイーン支配問題を解く3種類のASP符号化を考案.}
	 \begin{itemize}
	  \item ASPの高い表現力を活かし,クイーン支配問題を簡潔に
		記述することができた.
	 \end{itemize}
   \item \structure{既知の支配数を用いた評価実験.}
	 \begin{itemize}
	  \item SATの問題では唯一$n=16$まで解くなど,部分和符号化の
		優位性を確認できた.
	  \item UNSATの問題では解を得るまでのCPU時間の観点から,
		部分和符号化の優位性を確認できた.
	 \end{itemize}
  \end{itemize}
 \begin{alertblock}{今後の課題}
  \begin{itemize}
   \item 支配集合問題をより高速に解くASP符号化の考案.
   \item 遷移問題への拡張.
  \end{itemize}
 \end{alertblock}
\end{frame}

%
%付録
%

%%%% 補助スライド
\appendix
\backupbegin

\begin{frame}{~}
 \centering
 - 補足用 -
\end{frame} 

%%%%%%%%%%%%%%%%%%%%%%%%%%%%%%%%%%%%%%%%%%%%%%%%%%
%% 電気制約
%%%%%%%%%%%%%%%%%%%%%%%%%%%%%%%%%%%%%%%%%%%%%%%%%%
\begin{frame}{補足 : 電気制約}
 \begin{itemize}
  \item \alert{電気制約}は,送電する電流$\cdot$電圧の適正範囲を保証する制約.
  \begin{itemize}
   \item 供給経路の各区間で許容電流を超えない.
   \item 電気抵抗による電圧降下が許容範囲を超えない.
   \item etc.
  \end{itemize}
  \item 電流と電圧が影響し合う\structure{実数ドメイン上の制約}によって表される.
		% \begin{itemize}
		%  		 \item 送電システム上の条件など.
		% \end{itemize}
  \item 実数ドメイン上の制約は,純粋なASPのみで扱うのは\alert{困難}.
		\begin{itemize}
		 \item 緩和問題として,変電所から供給できる家庭の数に上限をつける.
		 \item ASPMT技術により,ASPで得られた解について,
			   背景理論ソルバーと連携して実数ドメイン上の制約を調べる.
		\end{itemize}
 \end{itemize}
\end{frame}

%%%%%%%%%%%%%%%%%%%%%%%%%%%%%%%%%%%%%%%%%%%%%%%%%%
%% 基礎化
%%%%%%%%%%%%%%%%%%%%%%%%%%%%%%%%%%%%%%%%%%%%%%%%%%
\begin{frame}{補足 : ASPシステム}
 
 \vspace{-0.5cm}

 \begin{figure}[htbp]
  \centering
  %%%%%%%%%%%%%%%%%%%%%%%%%%%%%%%%%%%%%%%%%%%%%%%%%%
%% 基礎化の流れの図
%%%%%%%%%%%%%%%%%%%%%%%%%%%%%%%%%%%%%%%%%%%%%%%%%%
\begin{tikzpicture}

 \definecolor{edge}{RGB}{38,38,134}
 \definecolor{node}{RGB}{220,220,249}

 \definecolor{alert_edge}{RGB}{191,0,0}
 \definecolor{alert_node}{RGB}{249,200,200}

 \definecolor{ex_edge}{RGB}{0,96,0}
 \definecolor{ex_node}{RGB}{230,239,230}

 \def\nodespace{2.4cm}

 \tikzset{block/.style={rectangle, thick, draw=edge, fill=node, text width=3cm, 
 text centered, rounded corners, text width=2cm, minimum height=1.5cm}};

 \tikzset{alertblock/.style={rectangle, thick, draw=alert_edge, fill=alert_node, 
 text width=3cm, text centered, rounded corners, text width=1.5cm, minimum height=1.2cm}};

 \node[block](ikkai){一階ASP\\プログラム};

 \node[rectangle,rounded corners, thick, draw=ex_edge, fill=ex_node, 
 right=0.22*\nodespace of ikkai, minimum width=6cm, minimum height=3cm, 
 text centered, label=ASPシステム](sys){};

 \node[block, right=\nodespace of ikkai](meidai){命題ASP\\プログラム};
 \node[block, right=\nodespace of meidai](ASP){解集合};

 \node[right=0.6*\nodespace of ikkai, text width=1.5cm, 
 text centered, text=red, anchor=south](){基礎化\\ソルバー};
 \node[right=0.4*\nodespace of meidai, text width=1.5cm, 
 text centered, text=red, anchor=south](){解集合\\ソルバー};

 
 \foreach \u / \v / \n in {ikkai/meidai,meidai/ASP}
 \draw [thick,->] (\u) to (\v);

\end{tikzpicture}
 \end{figure}

 \vspace{-0.5cm}

 \begin{exampleblock}{}
  \begin{enumerate}
   \item 一階ASPプログラムを基礎化ソルバーによって,
		 命題ASPプログラムに\alert{基礎化}する.
   \item 命題ASPプログラムについて,SAT技術を応用した解集合ソルバーが解集合を探索する.
  \end{enumerate}
 \end{exampleblock}

\end{frame}
%%%%%%%%%%%%%%%%%%%%%%%%%%%%%%%%%%%%%%%%%%%%%%%%%%
%% ASPの構文
%%%%%%%%%%%%%%%%%%%%%%%%%%%%%%%%%%%%%%%%%%%%%%%%%%
\begin{frame}{ASPの構文}
  \begin{alertblock}{}\centering
    ASPの言語は論理プログラムをベースとしている~\footnotemark.
  \end{alertblock}
  \begin{itemize}
  \item \structure{\bf 論理プログラム}とは,以下の\structure{\bf ルール}の有限集合である.
    \begin{center}
      \begin{minipage}[c]{0.7\textwidth}
        \begin{block}{}\centering
          $a_0$\quad\code{:-}\quad$a_1$\code{,}\ldots\code{,}$a_m$\code{,}
          \ \code{not}~$a_{m+1}$\code{,}\ldots\code{,} \code{not}~$a_n$\code{.}
        \end{block}        
      \end{minipage}
   \end{center}\vfill
    $0 \leq m \leq n$ であり,各 $a_i$ はアトム,
    \code{not}は\structure{\bf デフォルトの否定},\\
    ``\code{,}''は連言(AND)を表す.``\code{:-}''の左辺を\structure{\bf ヘッド},
		右辺を\structure{\bf ボディ}と呼ぶ.
  \item \alert{\bf 直感的な意味}は,
    「$a_1,\ldots,a_m$がすべて成り立ち,
    $a_{m+1},\ldots,a_n$のそれぞれが成り立たないならば,
    $a_0$が成り立つ」である.
  \item ボディが空のルールを\structure{\bf ファクト}と呼び,``\code{:-}''は省略できる.
  \item ヘッドが空のルールを\structure{\bf 一貫性制約}と呼ぶ.例えば,\hspace{-1ex}
    ``\code{:-} $a_1$\code{,} \code{not}~$a_{2}$''は,
    「$a_1$が成り立つならば,$a_2$が成り立つ」を意味する.
  \end{itemize}
  \footnotetext{本発表では標準論理プログラムを単に論理プログラムと呼ぶ.}
\end{frame}
%%%%%%%%%%%%%%%%%%%%%%%%%%%%%%%%%%%%%%%%%%%%%%%%%%
%% ASPの拡張構文
%%%%%%%%%%%%%%%%%%%%%%%%%%%%%%%%%%%%%%%%%%%%%%%%%%
\begin{frame}{ASPの拡張構文}
\begin{alertblock}{}\centering
  組合せ問題を解くための便利な構文が用意されている.
\end{alertblock}
\begin{itemize}
 \item \structure{\bf 選択子}
   \begin{center}
     \code{\{}$a_1$\code{;}\ldots\code{;}$a_n$\code{\}}
   \end{center}
   アトム集合 $\{a_1,\dots,a_n\}$
   の任意の部分集合が成り立つことを意味する.
 \item \structure{\bf 個数制約}
   \begin{center}
     $lb$\ \code{\{}$a_1$\code{;}\ldots\code{;}$a_n$\code{\}}\ $ub$
   \end{center}
   $a_1,\dots,a_n$ のうち,
   $lb$個以上,$ub$個以下が成り立つことを意味する.
 \item \structure{\bf 重み付き個数制約}
   \begin{center}
     $lb$ \code{\#sum\{} $w_1$\code{:}$a_1$\code{;}\ldots\code{;}$w_n$\code{:}$a_n$ \code{\}} $ub$
   \end{center}
   $a_1,\dots,a_n$のうち,
   成り立つアトムの重み和が$lb$以上,$ub$以下になることを意味する.
\end{itemize}
\end{frame}
%%%%%%%%%%%%%%%%%%%%%%%%%%%%%%%%%%%%%%%%%%%%%%%%%%
%% 改良符号化 (到達可能性)
%%%%%%%%%%%%%%%%%%%%%%%%%%%%%%%%%%%%%%%%%%%%%%%%%%
\begin{frame}[fragile]{改良符号化: 到達可能性}
\begin{exampleblock}{}\small
\begin{lstlisting}
(1) { inForest(X,Y) } :- edge(X,Y).
\end{lstlisting}
\end{exampleblock}
\begin{itemize}
 \item (1) 各辺\code{(X,Y)について},根付き全域森に含まれること意味する \\
	  アトム\code{inForest(X,Y)}を導入する.
\end{itemize}
\begin{exampleblock}{}\small
\begin{lstlisting}
(2) reached(R,R) :- root(R).
(3) reached(X,R) :- reached(Y,R), inForest(Y,X).
(4) reached(X,R) :- reached(Y,R), inForest(X,Y).
\end{lstlisting}
\end{exampleblock}
\vfill
\begin{itemize}
\item アトム\code{reached(X,R)}は,ノード\code{X}が根ノード\code{R}から到達可能であることを意味する.
%\item (2) 各根ノード\code{R}について,自分自身から到達可能であることを表す.
\item (3) ノード\code{Y}が根ノード\code{R}から到達可能かつ,辺\code{(Y,X)}が根付き全域森に含まれるならば,
	  ノード\code{X}も同じ根ノード\code{R}から到達可能であることを表す.
\end{itemize}
\end{frame}
%%%%%%%%%%%%%%%%%%%%%%%%%%%%%%%%%%%%%%%%%%%%%%%%%%
%% 改良符号化 (根付き連結制約)
%%%%%%%%%%%%%%%%%%%%%%%%%%%%%%%%%%%%%%%%%%%%%%%%%%
\begin{frame}[fragile]{改良符号化: 根付き連結制約}
\begin{exampleblock}{}\small
\begin{lstlisting}
(5) :- node(X), not 1 { reached(X,R) } 1.
\end{lstlisting}
\end{exampleblock}
\vfill
\begin{itemize}
\item (5) 各ノード\code{X}について,ちょうど1つの根からのみ到達可能であることを意味する.
\end{itemize}
\end{frame}
%%%%%%%%%%%%%%%%%%%%%%%%%%%%%%%%%%%%%%%%%%%%%%%%%%
%% 改良符号化 (非閉路制約)
%%%%%%%%%%%%%%%%%%%%%%%%%%%%%%%%%%%%%%%%%%%%%%%%%%
\begin{frame}[fragile]{改良符号化: 非閉路制約}
\begin{minipage}[c]{1.01\textwidth}
\begin{exampleblock}{}\small
\begin{lstlisting}
(6) :- root(R),
       not 1 #sum{ 1,X:reached(X,R) ;
                  -1,X,Y:inForest(X,Y),reached(X,R),reached(Y,R)
                 } 1.
\end{lstlisting}
\end{exampleblock}
\end{minipage}
\vfill
\begin{itemize}
\item (6) 各連結成分の\structure{\bf ノード数と辺数の差が1}になることを意味する.
\item 各連結成分が\structure{\bf 木の性質}を満たすことにより,サイクルを持たない
	  ことを保証する.
\end{itemize}
\end{frame}
%%%%%%%%%%%%%%%%%%%%%%%%%%%%%%%%%%%%%%%%%%%%%%%%%%
%% ルール数の比較
%%%%%%%%%%%%%%%%%%%%%%%%%%%%%%%%%%%%%%%%%%%%%%%%%%
\begin{frame}{基礎化後のルール数}
  \begin{itemize}
  \item グラフのノード数を$|V|$,根ノードの数を$|R|$とする.
  \end{itemize}
  \begin{table}[t]
    \centering
    %%%%%%%%%%%%%%%%%%%%%%%%%%%%%%%%%%%%%%%%%%%%%%%%%%%%%%%%%%%%%%%%
\chapter{ハミルトン閉路問題および関連問題のASP符号化}\label{chap:proposal}
%%%%%%%%%%%%%%%%%%%%%%%%%%%%%%%%%%%%%%%%%%%%%%%%%%%%%%%%%%%%%%%% 

%%%%
\begin{figure}[h]
  \centering
  \thicklines
  \setlength{\unitlength}{1.2pt}
  \small\footnotesize\scriptsize
  \begin{picture}(280,57)(4,-10)
    \put(  0, 20){\dashbox(50,24){\shortstack{HCP問題\\インスタンス}}}
    \put( 60, 20){\framebox(50,24){変換器}}
    \put(120, 20){\dashbox(50,24){\shortstack{ASPファクト}}}
    \put(120,-10){\dashbox(50,24){\shortstack{ASP符号化\\(論理プログラム)}}}
    \put(180, 20){\framebox(50,24){ASPシステム}}
    \put(240, 20){\dashbox(50,24){\shortstack{HCP問題\\の解}}}
    \put( 50, 32){\vector(1,0){10}}
    \put(110, 32){\vector(1,0){10}}
    \put(170, 32){\vector(1,0){10}}
    \put(230, 32){\vector(1,0){10}}
    \put(170, +2){\line(1,0){4}}
    \put(174, +2){\line(0,1){30}}
  \end{picture}  
\caption{ASP を用いたハミルトン閉路問題(HCP)の解法}
\label{fig:arch}
\end{figure}
%%%%

%\begin{figure}[tbp]
\tikz{
  %1ノード目
  \path[draw=black, fill=blue!20, rounded corners=5pt]%線の設定
  node[at={(0.75,0.75)}] {問題}%文字を入れる
  (0,0) --(1.5,0) --(1.5,1.5) --(0,1.5) --cycle;%外周
  %2ノード目
  \path[draw=black, fill=blue!20, rounded corners=5pt, shift={(3,0)}]
  node[at={(0.75,0.75)}] {
    \begin{tabular}{c}
      ASP\\
      ファクト
    \end{tabular}
  }
  (0,0) --(1.5,0) --(1.5,1.5) --(0,1.5) --cycle;
  %3ノード目文字が複数行
  \path[draw=black, fill=green!20, rounded corners=5pt, shift={(6,0)}]
  node[at={(0.75,0.75)}] {
    \begin{tabular}{c}
      ASP\\
      システム
    \end{tabular}
  }
  (0,0) --(1.5,0) --(1.5,1.5) --(0,1.5) --cycle;
  %4ノード目文字が複数行
  \path[draw=black, fill=blue!20, rounded corners=5pt, shift={(9,0)}]
  node[at={(0.75,0.75)}] {解集合}
  (0,0) --(1.5,0) --(1.5,1.5) --(0,1.5) --cycle;
  %5ノード目文字が複数行
  \path[draw=black, fill=red!20, rounded corners=5pt, shift={(3,-3)}]
  node[at={(0.75,0.75)}] {
    \begin{tabular}{c}
      ASP\\
      符号化
    \end{tabular}
  }
  (0,0) --(1.5,0) --(1.5,1.5) --(0,1.5) --cycle;
  \draw[arrows=->] (1.5,0.75) --(3.0,0.75);
  \draw[arrows=->,shift={(3,0)}] (1.5,0.75) --(3.0,0.75);
  \draw[arrows=->,shift={(6,0)}] (1.5,0.75) --(3.0,0.75);
  \draw[arrows=->] (4.5,-2.25) --(6.0,0.5);
}
\caption{ASPを用いた解法}
\label{aspmethod}
\end{figure}


ASP を用いたハミルトン閉路問題および関連問題の解法について述べる.
図~\ref{fig:arch}に,解法の流れを示す.
与えられたハミルトン閉路問題は ASP ファクトに変換され,
ハミルトン閉路問題を解く ASP 符号化と結合され,
ASP システムによって解が計算される.
本論文では,ASP システムとして{\clingo}を用いる.

%%%%%%%%%%%%%%%%%%%%%%%%%%%%%%%%%%%%%%%%%%%%%%%%%%%%%%%%%%%%%%%%%%%%%%%
\section{ASPファクト形式}
%%%%%%%%%%%%%%%%%%%%%%%%%%%%%%%%%%%%%%%%%%%%%%%%%%%%%%%%%%%%%%%%%%%%%%%

%%%%%%%%%%%%%%%%%%%%%%%%%%%%%%
\begin{figure}[t]
\begin{center}
\begin{tikzpicture}
  %ノード1  
  \draw(4,2) circle (0.5)
  node[at={(4.1,2.1)}] {
    \begin{tabular}{c}
      1
    \end{tabular}
  };
  %ノード2  
  \draw(4,0) circle (0.5)
  node[at={(4.1,0.1)}] {
    \begin{tabular}{c}
      2
    \end{tabular}
  };
  %ノード3  
  \draw(6,2) circle (0.5)
  node[at={(6.1,2.1)}] {
    \begin{tabular}{c}
      3
    \end{tabular}
  };
  %ノード4  
  \draw(6,0) circle (0.5)
  node[at={(6.1,0.1)}] {
    \begin{tabular}{c}
      4
    \end{tabular}
  };
  %ノード5  
  \draw(8,2) circle (0.5)
  node[at={(8.1,2.1)}] {
    \begin{tabular}{c}
      5
    \end{tabular}
  };
  %ノード6  
  \draw(8,0) circle (0.5)
  node[at={(8.1,0.1)}] {
    \begin{tabular}{c}
      6
    \end{tabular}
  };
\draw(4,0.5) --(4,1.5);
\draw(6,0.5) --(6,1.5);
\draw(8,0.5) --(8,1.5);
\draw(4.5,0) --(5.5,0);
\draw(4.5,2) --(5.5,2);
\draw(6.5,0) --(7.5,0);
\draw(6.5,2) --(7.5,2);
\end{tikzpicture}

\caption{入力となる重み付き無向グラフの例}
\label{graphexample}
\end{center}
\end{figure}
%%%%%%%%%%%%%%%%%%%%%%%%%%%%%%

%%%%%%%%%%%%%%%%%%%%%%%%%%%%%%
\lstinputlisting[float=t,caption={%
図~\ref{graphexample}のASPファクト表現},%
captionpos=b,frame=single,label=code:graph_example.lp,%
numbers=none,%
breaklines=true,%
columns=fullflexible,keepspaces=true,%
basicstyle=\ttfamily\scriptsize]{code/graph_example.lp}
%%%%%%%%%%%%%%%%%%%%%%%%%%%%%%


本節では,最短ハミルトン閉路問題の例にとって,
入力となる重み付き無向グラフ(図~\ref{graphexample})の
ASP ファクト形式について説明する.
%
このグラフは,頂点数が6,辺の数が7であり,辺に付けられた値は距離を表す.
コード~\ref{code:graph_example.lp}に,ASPファクト形式を示す.
%
アトム\code{node/1}は頂点,\code{edge/2}は辺,\code{cost/3}は距離を表す.
例えば,\code{cost(1,2,3)}は,辺\code{edge(1,2)}の距離が3であることを
表している.

%%%%%%%%%%%%%%%%%%%%%%%%%%%%%%%%%%%%%%%%%%%%%%%%%%%%%%%%%%%%%%%%%%%%%%%
\section{ハミルトン閉路問題の ASP 符号化}\label{hamiltonianasp}
%%%%%%%%%%%%%%%%%%%%%%%%%%%%%%%%%%%%%%%%%%%%%%%%%%%%%%%%%%%%%%%%%%%%%%%

ハミルトン閉路問題は,与えられたグラフの全頂点をちょうど一度ずつ通る閉
路(ハミルトン閉路)が存在するかどうかを判定する問題である.
$G=(V,E)$にハミルトン閉路が存在する必要十分条件は,
以下の2つの制約を満たす部分グラフ$G'=(V,E')$が存在することである.

\begin{itemize}
\item $G'$の各頂点の次数が2 (次数制約)
\item $G'$が連結である (連結制約)
\end{itemize}

本論文では,前者を\textbf{次数制約},後者を\textbf{連結制約}と呼ぶ.
ハミルトン路問題は,ハミルトン閉路問題から始点と終点が一致するという閉
路の条件を取り除いたものである.
ハミルトン路問題では,次数制約は以下のように変わる.

\begin{itemize}
\item 始点と終点の次数が1,他の頂点の次数が2
\end{itemize}

以下では,ハミルトン閉路問題に対する3つの ASP 符号化
\textsf{undirected},\textsf{directed},\textsf{acyclicity}
を提案する.

%%%%%%%%%%%%%%%%%%%%%%%%%%%%%%%%%%%%%%%%%%%%%%%%%%%%%%%%%%%%%%%%%%%%%%%
\subsection{\textsf{undirected}符号化}
%%%%%%%%%%%%%%%%%%%%%%%%%%%%%%%%%%%%%%%%%%%%%%%%%%%%%%%%%%%%%%%%%%%%%%%

%%%%%%%%%%%%%%%%%%%%%%%%%%%%%%
\lstinputlisting[float=t,caption={%
\textsf{undirected}符号化},%
captionpos=b,frame=single,label=code:hamilton1.lp,%
numbers=left,%
breaklines=true,%
columns=fullflexible,keepspaces=true,%
basicstyle=\ttfamily\footnotesize]{code/hamilton1.lp}
%%%%%%%%%%%%%%%%%%%%%%%%%%%%%%

\textsf{undirected}符号化は,ハミルトン閉路問題の次数制約と連結制約を,
ASP の一貫性制約で表した基本的な符号化である.
コード~\ref{code:hamilton1.lp}に,\textsf{undirected}符号化を示す.
この符号化は,ハミルトン閉路問題とハミルトン路問題の両方に対応している.
符号化中の\code{s}は始点の頂点番号,\code{t}は終点の頂点番号を表し,こ
れらは実行時に与えられる.
ここでは,ハミルトン閉路問題(\code{s}=\code{t})の場合について説明する.

\begin{itemize}
\item 1行目のルールは,各辺\code{edge(X,Y)}に対して,その辺がハミルト
  ン閉路に含まれるかどうかを意味するアトム\code{in(X,Y)}を選択子を用い
  て導入している.
\item 次数制約は3行目のルールで表される.このルールは,
  各頂点\code{node(X)}に対して,その次数の和が2に等しいことを個数制約
  を使って表している.
\item 連結制約は11行目のルールで表される.
ある頂点\code{X}が始点\code{s}から到達可能であることを意味する補助アト
ム\code{reached(X)}を導入する.
8行目のルールは,始点\code{s}が到達可能あることを表している.
9行目のルールは,各辺\code{X}--\code{Y}に対して,その辺がハミルトン閉
路に含まれ(\code{in(X,Y)}),かつ,頂点\code{X}が始点から到
達可能であれば(\code{reached(X)}),\code{Y}も到達可能であることを表している.
10行目は9行目と同様であるが,辺\code{Y}--\code{X}の場合を表している.
11行目のルールは,各頂点\code{node(X)}が始点から到達可能でなければな
らないことを一貫性制約を使って表している.
\end{itemize}

%%%%%%%%%%%%%%%%%%%%%%%%%%%%%%%%%%%%%%%%%%%%%%%%%%%%%%%%%%%%%%%%%%%%%%%
\subsection{\textsf{directed}符号化}
%%%%%%%%%%%%%%%%%%%%%%%%%%%%%%%%%%%%%%%%%%%%%%%%%%%%%%%%%%%%%%%%%%%%%%%

%%%%%%%%%%%%%%%%%%%%%%%%%%%%%%
\lstinputlisting[float=t,caption={%
\textsf{directed}符号化},%
captionpos=b,frame=single,label=code:hamilton2.lp,%
numbers=left,%
breaklines=true,%
columns=fullflexible,keepspaces=true,%
basicstyle=\ttfamily\footnotesize]{code/hamilton2.lp}
%%%%%%%%%%%%%%%%%%%%%%%%%%%%%%

\textsf{directed}符号化は,\textsf{undirected}符号化をベースに,
与えられた無向グラフの各辺$u-v$に対して,2つの弧$u\rightarrow v$と
$v\rightarrow u$を対応させることで有向グラフ化して解く符号化である.
コード~\ref{code:hamilton2.lp}に,\textsf{directed}符号化を示す.
前節と同様に,ハミルトン閉路問題(\code{s}=\code{t})の場合について説明する.

\begin{itemize}
\item 1行目では,無向グラフの有向グラフ化を行う.
  与えられた無向グラフの各辺\code{edge(X,Y)}に対して,
  2つの弧\code{edge(X,Y)},\code{edge(Y,X)}を導入した.
\item 2行目のルールは,各弧\code{edge(X,Y)}に対して,その弧がハミルト
  ン閉路に含まれるかどうかを意味するアトム\code{in(X,Y)}を選択子を用い
  て導入している.
\item 次数制約は4,5行目のルールで表される.
  4行目では,各頂点\code{node(X)}に対して,
  その出次数が1に等しいことを個数制約を使って表している.
  5行目では,入次数について4行目と同様の制約を表す.
\item 連結制約は15行目のルールで表される.
  ある頂点\code{X}が始点\code{s}から到達可能であることを意味する
  補助アトム\code{reached(X)}を導入する.
  13行目のルールは,始点\code{s}が到達可能あることを表している.
  14行目のルールは,各弧\code{X}--\code{Y}に対して,その弧がハミルトン閉路
  に含まれ(\code{in(X,Y)}),かつ,頂点\code{X}が始点から
  到達可能であれば(\code{reached(X)}),\code{Y}も到達可能であることを表している.
  15行目のルールは,各頂点\code{node(X)}が始点から到達可能でなければ
  ならないことを一貫性制約を使って表している.
\item 18行目のルールは,解についての対称性を除去する.
  与えられた無向グラフ上の各ハミルトン閉路に対して,
  それを変換した有向グラフ上のハミルトン閉路は対称な2つが存在する.
  これによる解の重複を防ぐために,18行目のルールは,各弧\code{s}--\code{X},
  \code{Y}--\code{s}がハミルトン閉路に含まれるならば(\code{in(s,X),in(Y,s)}),
  \code{X < Y}でなければならないことを,一貫性制約を用いて表している
\end{itemize}

%%%%%%%%%%%%%%%%%%%%%%%%%%%%%%%%%%%%%%%%%%%%%%%%%%%%%%%%%%%%%%%%%%%%%%%
\subsection{\textsf{acyclicity}符号化}
%%%%%%%%%%%%%%%%%%%%%%%%%%%%%%%%%%%%%%%%%%%%%%%%%%%%%%%%%%%%%%%%%%%%%%%

%%%%%%%%%%%%%%%%%%%%%%%%%%%%%%
\lstinputlisting[float=t,caption={%
\textsf{acyclicity}符号化},%
captionpos=b,frame=single,label=code:hamilton3.lp,%
numbers=left,%
breaklines=true,%
columns=fullflexible,keepspaces=true,%
basicstyle=\ttfamily\footnotesize]{code/hamilton3.lp}
%%%%%%%%%%%%%%%%%%%%%%%%%%%%%%

\textsf{acyclicity}符号化は,\textsf{directed}符号化をベースに,
連結の制約に代わる部分閉路禁止制約を組込み非閉路制約で表現した符号化である.
コード~\ref{code:hamilton3.lp}に,\textsf{acyclicity}符号化を示す.
前節と同様に,ハミルトン閉路問題(\code{s}=\code{t})の場合について説明する.

\begin{itemize}
\item 1行目では,無向グラフの有向グラフ化を行う.
  与えられた無向グラフの各辺\code{edge(X,Y)}に対して,
  2つの弧\code{edge(X,Y)},\code{edge(Y,X)}を導入した.
\item 2行目のルールは,各弧\code{edge(X,Y)}に対して,その弧がハミルト
  ン閉路に含まれるかどうかを意味するアトム\code{in(X,Y)}を選択子を用い
  て導入している.
\item 次数制約は4,5行目のルールで表される.
  4行目では,各頂点\code{node(X)}に対して,
  その出次数が1に等しいことを個数制約を使って表している.
  5行目では,入次数について4行目と同様の制約を表す.
\item 部分閉路禁止制約は14行目のルールで表される.
  このルールは,始点でない各頂点\code{X},\code{Y}について,
  弧\code{X}--\code{Y}がハミルトン閉路に含まれるならば(\code{in(X,Y)}),
  そのような弧の集合をもつグラフが閉路をもたないことを,\code{#edge}宣言を用いて表す.
  ようするに,始点(終点)を含まないような閉路を禁止している.
\item 17行目のルールは,解についての対称性を除去する.
  与えられた無向グラフ上の各ハミルトン閉路に対して,
  それを変換した有向グラフ上のハミルトン閉路は対称な2つが存在する.
  これによる解の重複を防ぐために,17行目のルールは,各弧\code{s}--\code{X},
  \code{Y}--\code{s}がハミルトン閉路に含まれるならば(\code{in(s,X),in(Y,s)}),
  \code{X < Y}でなければならないことを,一貫性制約を用いて表している
\end{itemize}

%%%%%%%%%%%%%%%%%%%%%%%%%%%%%%%%%%%%%%%%%%%%%%%%%%%%%%%%%%%%%%%%%%%%%%% 
\section{最短ハミルトン閉路問題のASP符号化}\label{minexpl}
%%%%%%%%%%%%%%%%%%%%%%%%%%%%%%%%%%%%%%%%%%%%%%%%%%%%%%%%%%%%%%%%%%%%%%% 

%% %%%%%%%%%%%%%%%%%%%%%%%%%%%%%%
%% \lstinputlisting[caption =  最適化,label = minimize]{code/obj_minimize.lp}
%% %%%%%%%%%%%%%%%%%%%%%%%%%%%%%%

%%%%%%%%%%%%%%%%%%%%%%%%%%%%%%
\lstinputlisting[float=t,caption={%
最小化},%
captionpos=b,frame=single,label=code:obj_minimize.lp,%
numbers=left,%
breaklines=true,%
columns=fullflexible,keepspaces=true,%
basicstyle=\ttfamily\footnotesize]{code/obj_minimize.lp}
%%%%%%%%%%%%%%%%%%%%%%%%%%%%%%

最短ハミルトン閉路問題の目的関数は,
ハミルトン閉路を構成する各辺の距離の総和である.
コード\ref{code:obj_minimize.lp}は,
その目的関数の最小化を表す.
このコードは,各辺\code{edge(X,Y)}に対して,その辺がハミルトン閉路に
含まれ(\code{in(X,Y)}),その距離が\code{C}である時に(\code{cost(X,Y,C)}),
\code{C}の総和の最小化を,最小化関数を用いて表している.
.
%%%%%%%%%%%%%%%%%%%%%%%%%%%%%%
\lstinputlisting[float=t,caption={%
重み付き無向グラフの有向グラフ化},%
captionpos=b,frame=single,label=code:cost_both.lp,%
numbers=left,%
breaklines=true,%
columns=fullflexible,keepspaces=true,%
basicstyle=\ttfamily\footnotesize]{code/cost_both.lp}
%%%%%%%%%%%%%%%%%%%%%%%%%%%%%%

符号化directed,acyclicityについては,
与えられた無向グラフの各辺\code{edge(X,Y)}に対して,
2つの弧\code{edge(X,Y)},\code{edge(Y,X)}を導入した.
各辺の距離もこれに対応させるために,コード\ref{code:cost_both.lp}
を追加した.
このルールは,各辺\code{X}--\code{Y}の距離を表す\code{cost(X,Y,C)}について,
\code{cost(Y,X,C)}を導入する.
これにより,与えられた無向グラフの各辺\code{edge(X,Y)}の重み\code{C}が
2つの弧\code{edge(X,Y)},\code{edge(Y,X)}にも付与された.

%%%%%%%%%%%%%%%%%%%%%%%%%%%%%%%%%%%%%%%%%%%%%%%%%%%%%%%%%%%%%%%%%%%%%%% 
\section{コスト制約付きハミルトン閉路のASP符号化}
%%%%%%%%%%%%%%%%%%%%%%%%%%%%%%%%%%%%%%%%%%%%%%%%%%%%%%%%%%%%%%%%%%%%%%% 

%%%%%%%%%%%%%%%%%%%%%%%%%%%%%%
\lstinputlisting[float=t,caption={%
コスト制約},%
captionpos=b,frame=single,label=code:cost_constraint.lp,%
numbers=left,%
breaklines=true,%
columns=fullflexible,keepspaces=true,%
basicstyle=\ttfamily\footnotesize]{code/cost_constraint.lp}
%%%%%%%%%%%%%%%%%%%%%%%%%%%%%%

コスト制約付きハミルトン閉路問題は
ハミルトン閉路問題に,距離の総和が所与の閾値以下 (または以上) であること
を制約条件として付加した問題である.
コード\ref{code:const_constraing.lp}のルールは,その制約を表す.
ルール中の\code{c}は閾値を表し,これは実行時に与えられる.
このルールは,各辺\code{edge(X,Y)}に対して,その辺がハミルトン閉路に
含まれ(\code{in(X,Y)}),その距離が\code{C}である時に(\code{cost(X,Y,C)}),
\code{C}の総和が\code{c}以下でなければならないことを,
一貫性制約と重み付き個数制約を用いて表す.

また,\ref{minexpl}と同様に,
符号化directed,acyclicityについては,
アトム\code{cost}についても有向グラフ化に
対応させるためにコード\ref{code:cost_both.lp}を追加した.
%%%%%%%%%%%%%%%%%%%%%%%%%%%%%%%%%%%%%%%%%%%%%%%%%%%%%%%%%%%%%%%%%%%%%%%

%%% Local Variables:
%%% mode: latex
%%% TeX-master: "paper"
%%% End:

  \end{table}
\end{frame}

%%%%%%%%%%%%%%%%%%%%%%%%%%%%%%%%%%%%%%%%%%%%%%%%%%
%% ASPのコード
%%%%%%%%%%%%%%%%%%%%%%%%%%%%%%%%%%%%%%%%%%%%%%%%%%
\begin{frame}[fragile]{補足 : 根付き全域森 基本符号化}
%%%%%%%%%%%%%%%%%%%%%%%%%%%%%%%%% 
\lstinputlisting[frame=single,label=code:roop,%
xleftmargin=1zw,%
xrightmargin=1zw,%
numbersep=5pt,%
numbers=left,%
breaklines=true,%
columns=fullflexible,keepspaces=true,%
basicstyle=\ttfamily\scriptsize]{code/srf1.lp}
%%%%%%%%%%%%%%%%%%%%%%%%%%%%%%%%%
\end{frame}

\begin{frame}[fragile]{補足 : 遷移問題 シングルショット符号化}

\begin{multicols*}{2}
%%%%%%%%%%%%%%%%%%%%%%%%%%%%%%%%% 
\lstinputlisting[frame=single,label=code:roop,%
xleftmargin=1zw,%
xrightmargin=1zw,%
numbersep=5pt,%
numbers=left,%
breaklines=true,%
columns=fullflexible,keepspaces=true,%
basicstyle=\ttfamily\tiny]{code/trans-const.lp}
%%%%%%%%%%%%%%%%%%%%%%%%%%%%%%%%%
\end{multicols*}
\end{frame}

\begin{frame}[fragile]{補足 : 遷移問題 マルチショット符号化}
\begin{multicols*}{2}
%%%%%%%%%%%%%%%%%%%%%%%%%%%%%%%%%
\lstinputlisting[frame=single,label=code:incmode,% 
xleftmargin=1zw,%
xrightmargin=1zw,%
numbersep=5pt,%
numbers=left,%
breaklines=true,%
columns=fullflexible,keepspaces=true,%
basicstyle=\ttfamily\tiny]{code/dnet-trans.lp}
%%%%%%%%%%%%%%%%%%%%%%%%%%%%%%%%% 
\end{multicols*}
\end{frame}

\backupend


\end{document}