\documentclass[dvipdfmx,10pt]{beamer}
%%%% Packages %%%%%
\usepackage{bxdpx-beamer}% dvipdfmxなので必要
\usepackage{graphicx}
\usepackage{color}
\usepackage{skak}
% \usepackage{amsmath,amssymb,amsthm}
% \usepackage{multirow}
% \usepackage{url}
% \usepackage{tikz}
% \usepackage{alltt}
% \usepackage{bm}
% \usepackage{listings,jlisting}
% \usepackage{listings}
% \lstset{
%  basicstyle=\ttfamily\scriptsize,
%  keepspaces=true,
%  escapechar=|,
%  columns=[l]{fullflexible}
% }

%%%% Fonts %%%%%
\renewcommand{\kanjifamilydefault}{\gtdefault}
% \usepackage{otf} % otfパッケージ
\usepackage{tikz}
\usetikzlibrary{matrix}
\usetikzlibrary{shadows}
\usetikzlibrary{positioning}
\usepackage[deluxe]{otf} 
\usepackage{txfonts} % 数式・英文ローマン体を Lxfont にする
% \usepackage[T1]{fontenc} % 8bit フォント
% \usepackage{minijs}
% \usepackage{textcomp} % 欧文フォントの追加
% \usepackage[utf8]{inputenc} % 文字コードをUTF-8

%%%% Beamer %%%%%
\usetheme{Warsaw}
%\useinnertheme{rectangles}
%\useoutertheme{smoothbars}
%\setbeamercolor{enumerate}{fg=white, bg=black}
%\setbeamertemplate{frametitle}[default][center]
\setbeamertemplate{navigation symbols}{} % 右下のアイコンを消す
\useoutertheme{shadow}                 % 箱に影をつける
\usefonttheme{professionalfonts}       % 数式の文字を通常の LaTeX と同じにする
% \setbeamercovered{transparent} % 好みに応じてどうぞ
\setbeamertemplate{footline}[frame number]
\setbeamercolor{page number in head/foot}{fg=black} % ページ数を表示する
% \setbeamerfont{footline}{size=\normalsize,series=\bfseries}
% \setbeamerfont{footline}{size=\scriptsize,series=\mdseries}
% \setbeamercolor{footline}{fg=black,bg=black}
% \setbeamertemplate{blocks}[rounded][shadow=true]
\setbeamertemplate{items}[ball]
% exclude apprendix slides from framenumber %
\newcommand{\backupbegin}{
   \newcounter{framenumberappendix}
   \setcounter{framenumberappendix}{\value{framenumber}}
}
\newcommand{\backupend}{
   \addtocounter{framenumberappendix}{-\value{framenumber}}
   \addtocounter{framenumber}{\value{framenumberappendix}} 
}

% \setbeamertemplate{enumerate items}[default]
% \setbeamerfont{alerted text}{series=\bfseries}
\begin{document}
\title{解集合プログラミングを用いた\\クイーン支配問題の解法に関する考察}
\author{101830080 \quad 加藤 聖人}
\date{2021年度 卒業研究発表会 \\ 2022年2月18日}
\institute{番原研究室}

%
%表紙
%

\begin{frame}\frametitle{}
 \titlepage
\end{frame}

%
%支配集合問題について
%

\begin{frame}\frametitle{支配集合問題}
 \begin{block}{支配集合}
  無向グラフ$G=(V,E)$の頂点の部分集合$S\subset V$に対して,
任意の頂点$u \in V\setminus S$にも辺$(u,v) \in E$が存在し,
$v \in S$を満たすとき,$S$を$G$の\structure{支配集合}という.
  \begin{itemize}
   \item 支配集合の要素数を\structure{サイズ}という.
   \item サイズが最小の支配集合をグラフ$G$の\structure{最小支配集合}という.
   \item 最小支配集合のサイズをグラフ$G$の\alert{支配数}といい,
$\gamma(G)$で表す.
  \end{itemize}
 \end{block}
 \begin{block}{支配集合問題}
  グラフ$G$と正の整数$k$が与えられたとき,サイズが$k$の$G$の
支配集合が存在するかどうかを判定する問題である.
  \begin{itemize}
   \item NP完全であることが知られている.
   \item スケジューリング,電波塔配置問題など多くの現実の
	 問題に応用されている.
  \end{itemize}
 \end{block}
\end{frame}

%
%クイーン支配問題
%

\begin{frame}\frametitle{クイーン支配問題(Queen Domination Problem; QDP)}
  \begin{alertblock}{} \centering
    本発表では,支配集合問題のインスタンスの一種である\\
    \alert{クイーン支配問題}を対象とする.
  \end{alertblock}
 \begin{block}{クイーン支配問題}
  サイズ$n\times n$のクイーングラフ$Q_n$と正の整数$k$が与えられたとき,
  サイズ$k$の$Q_n$の支配集合が存在するかどうかを
  判定する問題である.
  \begin{itemize}
   \item クイーングラフ$Q_n$は$n\times n$のチェス盤
	 について各マスを頂点とし,クイーンが移動できる
	 マス同士が辺で結ばれているグラフである.
   \item $n \times n$の盤面に$k$個のクイーンを
	 置いたとき,クイーンを移動させて全てのマスに
	 アタックできるかを判定する問題に等しい.
  \end{itemize}
 \end{block}
\end{frame}
 
 
%
%クイーン支配問題の例
%

\begin{frame}{クイーン支配問題の例}
  \begin{exampleblock}{$Q_{5}$の最小支配集合}
  \begin{center}
   \scalebox{1.3}{
   \begin{tikzpicture}
 \draw[lightgray] (-1.25,-1.25)--(-1.25,1.25);
 \draw[lightgray] (-0.75,-1.25)--(-0.75,1.25);
 \draw[lightgray] (-0.25,-1.25)--(-0.25,1.25); 
 \draw[lightgray] (0.25,-1.25)--(0.25,1.25); 
 \draw[lightgray] (0.75,-1.25)--(0.75,1.25); 
 \draw[lightgray] (1.25,-1.25)--(1.25,1.25); 
 \draw[lightgray] (-1.25,-1.25)--(1.25,-1.25); 
 \draw[lightgray] (-1.25,-0.75)--(1.25,-0.75); 
 \draw[lightgray] (-1.25,-0.25)--(1.25,-0.25); 
 \draw[lightgray] (-1.25,0.25)--(1.25,0.25); 
 \draw[lightgray] (-1.25,0.75)--(1.25,0.75); 
 \draw[lightgray] (-1.25,1.25)--(1.25,1.25); 
 \draw (-1.25,1)--(1.25,1);
 \draw (-1,1.25)--(-1,-1.25);
 \draw (-1.25,1.25)--(1.25,-1.25);
 \draw (-1.25,0.75)--(-0.75,1.25);
 \draw (-1.25,-1)--(1.25,-1);
 \draw (-0.5,-1.25)--(-0.5,1.25);
 \draw (-0.25,-1.25)--(-1.25,-0.25);
 \draw (-0.75,-1.25)--(1.25,0.75);
 \draw (1,1.25)--(1,-1.25);
 \draw (-1.25,0.5)--(1.25,0.5);
 \draw (0.25,1.25)--(1.25,0.25);
\foreach \x in {-1,-0.5,0,0.5,1}
  \foreach \y in {-1,-0.5,0,0.5,1} 
  \fill (\x,\y) circle (0.03);
  \matrix[matrix of nodes,nodes={inner sep=0pt,text width=0.5cm,
align=center,minimum height=0.43cm}]{
 \symqueen & \quad & \quad & \quad & \quad \\
 \quad & \quad & \quad & \quad & \symqueen \\
 \quad & \quad & \quad & \quad & \quad \\
 \quad & \quad & \quad & \quad & \quad \\
 \quad & \symqueen & \quad & \quad & \quad \\};
\end{tikzpicture}
 

   }
  \end{center}
 \end{exampleblock}
 %上の図は$Q_5$の最小支配集合の例である.
 \begin{itemize}
  \item 3個のクイーンを置いたとき,クイーンを
	移動させてすべてのマスにアタック可能である.
  \item 2個以下のクイーンを置いたとき,クイーンを移動させて
	全てのマスにアタックすることは不可能である.
  \item したがって,$\gamma(Q_{5})=3$となる.
 \end{itemize}
\end{frame}


%
%クイーン支配問題の支配数とかに関する背景知識
%

\begin{frame}{クイーン支配問題の支配数}
  \begin{itemize}
    \item クイーン支配問題の支配数は,1862年に文献[Jaenisch,1862]で
	    $\gamma(Q_8)=5$が示されてから研究されている.
    \item $n=3,11$を除いた$n \leq 132$で $\lceil n/2 \rceil 
	    \leq \gamma(Q_{n}) \leq \lceil n/2 \rceil +1$
	 であることが証明されている[\"{O}sterg{\aa}rd,Weakley,2001].
%    \item THE ON-LINE ENCYCLOPEDIA OF INTEGER SEQUENCES には,
%    $1\leq n\leq 25$に対する$\gamma(Q_n)$
%    が掲載されている~\footnote{\url{https://oeis.org/A075458}}.
  \end{itemize}
 \begin{exampleblock}{$Q_{n}$の支配数$(1 \leq n \leq 20)$}
  \centering
  %  \begin{itemize}
%   \item クイーン支配問題の支配数は,1862年に文献[Jaenisch,1862]で
%	 $\gamma(Q_8)=5$が示されてから研究されている.
%   \item $n=3,11$を除いた$n \leq 132$で $\lceil n/2 \rceil 
%	 \leq \gamma(Q_{n}) \leq \lceil n/2 \rceil +1$
%	 であることが証明されている[\"{O}sterg{\aa}rd,Weakley,2001].
%   \item $1\leq n \leq 20$のクイーングラフの支配数は以下のとおりである.
%  \end{itemize}

  \begin{tabular}{c|c||c|c||c|c||c|c}%\hline
    $n$ & $\gamma(Q_{n})$ & $n$ & $\gamma(Q_{n})$ &$n$ & $\gamma(Q_{n})$ &$n$ & $\gamma(Q_{n})$ \\ \hline
    1 &1 &6 &3 &11 &5 &16 &9 \\ %\hline
    2 &1 &7 &4 &12 &6 &17 &9 \\ %\hline
    3 &1 &8 &5 &13 &7 &18 &9 \\ %\hline
    4 &2 &9 &5 &14 &8 &19 &10 \\ %\hline
    5 &3 &10 &5 &15 &9 &20 &11 \\ %\hline
  \end{tabular}
 \end{exampleblock}
\end{frame}

%
%ASPについて
%

\begin{frame}\frametitle{解集合プログラミング(Answer Set Programming; ASP)}
 \begin{itemize}
  \item \structure{ASP言語}は一階論理に基づいた知識表現言語の一種である.
  \item \structure{論理プログラム}は,ASP のルールの有限集合である.
  \item \structure{ASPシステム}は論理プログラムから
	安定モデル意味論[Gelfond and Lifschitz '88]に基づく
	解集合を計算するシステムである.
  \item 近年ではSAT技術を応用した高速ASPシステムが実現され,
	システム検証,プランニング,システム生物学など様々な
	分野への応用が拡大している.
 \end{itemize}
 \begin{alertblock}{クイーン支配問題に対してASPを用いる利点}
  \begin{itemize}
   \item ASP言語の高い表現力を活かし,クイーン支配問題の制約を簡潔に記述可能.
   % \item 個数制約を用いて,部分和を表す制約を簡潔に記述可能.
   % \item 高速な解列挙が可能.
  \end{itemize}
 \end{alertblock}
\end{frame}
 
%
%研究目的と研究内容
%

\begin{frame}\frametitle{研究目的}
 \begin{alertblock}{研究目的}\centering
  ASP技術を活用した,支配集合問題を効率よく解くソルバーの実現
 \end{alertblock}
 \begin{block}{研究内容}
  \begin{enumerate}
   \item クイーン支配問題を解く,3種類のASP符号化を考案.
	 \begin{itemize}
	  \item 基本符号化,改良符号化,部分和符号化.
	  \item 先行研究[山本,2021]で提案された制約モデルを参考に考案.
	 \end{itemize}
   \item $n$次のクイーン支配問題と既知の支配数を用いて,3種のASP符号化の
	 評価実験を行なった.
  \end{enumerate}
 \end{block}
\end{frame}

%
%符号化3つ
%

\begin{frame}{提案する符号化}
 % \begin{block}{クイーン支配問題の表現}
 %  与えられたクイーングラフ$Q_n$上に$k$個のクイーンを配置したとき,
 %  以下の制約を満たすならばクイーングラフ$Q_{n}$にサイズが$k$の
 %  支配集合が存在する.
 %  \begin{itemize}
 %   \item $Q_n$上のどのマスにも,1つ以上のクイーンが移動
 % 	 できなければならない(\alert{被覆制約})
 %  \end{itemize}
 % \end{block}
 %\begin{itemize}
 % \item クイーン支配問題を解く3つのASP符号化を提案する.
 %\end{itemize}
 \begin{block}{}\centering
   クイーン支配問題を解く3つのASP符号化を提案した.
 \end{block}
 \begin{enumerate}
   \item \alert{基本符号化}
   \begin{itemize}
     \item クイーン支配問題の制約を,ASP の一貫性制約を
      用いて簡潔に表現.
   \end{itemize}
   \item \alert{改良符号化}
   \begin{itemize}
      \item 各行,各列,各対角線に対して,
        クイーンが配置されているかどうかを表す補助アトム
        を導入する.
      \item 基本符号化で複数回出現する制約式をまとめている.
   \end{itemize}
   \item \alert{部分和符号化}
   \begin{itemize}
      \item 各行,各列,各対角線に対して,
        クイーンの個数を表す補助アトムを導入する.
      \item 行,列,対角線のそれぞれの和が
        サイズ$k$に一致することを,ASPの
        個数制約を用いて表現している.
   \end{itemize}
 \end{enumerate}
% \begin{itemize}
%  \item \alert{基本符号化}
%	\begin{itemize}
%	 \item クイーン支配問題の制約をASP の一貫性制約を
%	       用いて表現した基本的な符号化
%	\end{itemize}
%  \item \alert{改良符号化} 
%	\begin{itemize}
%	 \item 各行,列,対角線方向に対してクイーンが配置
%	       されているかどうかを表す補助アトムを導入し,
%	       基本符号化で複数回出現する制約式をまとめた符号化
%	\end{itemize}
%  \item \alert{部分和符号化}
%	\begin{itemize}
%	 \item 各行,各列,各対角線に対して配置されている
%	       クイーンの個数を表す補助アトムを導入し,それぞれの
%	       和がサイズ$k$に一致することをASPの個数制約を
%	       用いて表した符号化
%	\end{itemize}
% \end{itemize}
\end{frame}

%
%実験内容
%
 
\begin{frame}\frametitle{実験概要}
 \begin{block}{}
  提案した3種の符号化について,$n$次のクイーン支配問題
  と既知の支配数を用いて,支配集合の有無を判定した.
 \end{block}
 \begin{itemize}
  \item \structure{比較するASP符号化:}
	\begin{itemize}
	 \item 基本符号化
	 \item 改良符号化
	 \item 部分和符号化
	\end{itemize}
  \item \structure{対象とする問題:}
	\begin{itemize}
	 \item クイーングラフ$Q_{n} \qquad (1 \leq n \leq 20)$
	 \item $k=\gamma(Q_{n})$\quad (SAT),$k=\gamma(Q_{n})-1$\quad (UNSAT)
	\end{itemize}
  \item \structure{使用ASPソルバ:} \textit{clingo-5.5.0}
  \item \structure{実験環境:} Mac mini, 3.2GHz 6コア Intel Core i7, 64GBメモリ
  \item \structure{制限CPU時間:} 3600 (sec)
 \end{itemize}
\end{frame}

%
%実験結果(基本1,改良1,部分和2)
%

\begin{frame}\frametitle{実験結果: CPU 時間}
% \begin{block}{}
%  解の有無の判定に要したCPU時間.
% \end{block}
\begin{block}{}
  \begin{columns}
    \begin{column}{0.50\textwidth}
      \centering
      SAT の実験結果\\
      \vspace{4pt}
      \scalebox{0.7}{
         \centering 
 \begin{tabular}{c|c|r|r|r} %\hline
  $n$ & $k$ & 基本 & 改良 & 部分和 \\ \hline
  10 & 5 & 14.400 & 21.232 & 3.005 \\
  11 & 5 & 54.674 & 123.488 & 32.540 \\
  12 & 6 & T.O. & 56.468 & 2.147 \\
  13 & 7 & T.O. & 1281.823 & 22.412 \\  
  14 & 8 & 2797.041 & 772.276 & 10.044 \\   
  15 & 9 & T.O. & T.O. & 275.683 \\  
  16 & 9 & T.O. & T.O. & \alert{\textbf{734.878}} \\
  17 & 9 & T.O. & T.O. & T.O. \\
  18 & 9 & T.O. & T.O. & T.O. \\
  19 & 10 & T.O. & T.O. & T.O. \\
  20 & 11 & T.O. & T.O. & T.O. \\ %\hline
 \end{tabular}}
    \end{column}
    \begin{column}{0.50\textwidth}
      \centering
      UNSAT の実験結果\\
      \vspace{4pt}
      \scalebox{0.7}{
         \centering 
 \begin{tabular}{c|c|r|r|r} %\hline
  $n$ & $k$ & 基本 & 改良 & 部分和 \\ \hline
  10 & 4 & 10.280 & 8.929 & 2.612 \\
  11 & 4 & 28.184 & 24.537 & 3.427 \\
  12 & 5 & 2706.092 & 2673.241 & \alert{\textbf{337.127}} \\
  13 & 6 & T.O. & T.O. & T.O. \\  
  14 & 7 & T.O. & T.O. & T.O. \\   
  15 & 8 & T.O. & T.O. & T.O. \\  
  16 & 8 & T.O. & T.O. & T.O. \\
  17 & 8 & T.O. & T.O. & T.O. \\
  18 & 8 & T.O. & T.O. & T.O. \\
  19 & 9 & T.O. & T.O. & T.O. \\
  20 & 10 & T.O. & T.O. & T.O. \\ %\hline
 \end{tabular}}
    \end{column}
  \end{columns}
\end{block}
% \begin{columns}
%  \begin{column}{0.50\textwidth}
%   \begin{table}[htbp]
%    \caption{SATの実験結果}
%    \scalebox{0.7}{
%     \centering 
 \begin{tabular}{c|c|r|r|r} %\hline
  $n$ & $k$ & 基本 & 改良 & 部分和 \\ \hline
  10 & 5 & 14.400 & 21.232 & 3.005 \\
  11 & 5 & 54.674 & 123.488 & 32.540 \\
  12 & 6 & T.O. & 56.468 & 2.147 \\
  13 & 7 & T.O. & 1281.823 & 22.412 \\  
  14 & 8 & 2797.041 & 772.276 & 10.044 \\   
  15 & 9 & T.O. & T.O. & 275.683 \\  
  16 & 9 & T.O. & T.O. & \alert{\textbf{734.878}} \\
  17 & 9 & T.O. & T.O. & T.O. \\
  18 & 9 & T.O. & T.O. & T.O. \\
  19 & 10 & T.O. & T.O. & T.O. \\
  20 & 11 & T.O. & T.O. & T.O. \\ %\hline
 \end{tabular}}
%   \end{table}
%  \end{column}
%  \begin{column}{0.50\textwidth}
%   \begin{table}[htbp]
%    \caption{UNSATの実験結果}
%    \scalebox{0.7}{
%     \centering 
 \begin{tabular}{c|c|r|r|r} %\hline
  $n$ & $k$ & 基本 & 改良 & 部分和 \\ \hline
  10 & 4 & 10.280 & 8.929 & 2.612 \\
  11 & 4 & 28.184 & 24.537 & 3.427 \\
  12 & 5 & 2706.092 & 2673.241 & \alert{\textbf{337.127}} \\
  13 & 6 & T.O. & T.O. & T.O. \\  
  14 & 7 & T.O. & T.O. & T.O. \\   
  15 & 8 & T.O. & T.O. & T.O. \\  
  16 & 8 & T.O. & T.O. & T.O. \\
  17 & 8 & T.O. & T.O. & T.O. \\
  18 & 8 & T.O. & T.O. & T.O. \\
  19 & 9 & T.O. & T.O. & T.O. \\
  20 & 10 & T.O. & T.O. & T.O. \\ %\hline
 \end{tabular}}
%   \end{table}
%  \end{column}
% \end{columns} 
 \begin{itemize}
  \item SATの問題では,\structure{部分和符号化}が$n=16$まで解き,
	その優位性を確認できた.
  \item UNSATの問題では,解けた問題数に差はなかったが,解なしが得られる
	までにかかった時間は\structure{部分和符号化}が最も早かった.
 \end{itemize}
\end{frame}

%
%まとめと今後の課題
%

\begin{frame}\frametitle{まとめ}
  \begin{itemize}
   \item \structure{クイーン支配問題を解く3種類のASP符号化を考案.}
	 \begin{itemize}
	  \item ASPの高い表現力を活かし,クイーン支配問題を簡潔に
		記述できた.
%    \item 最も簡潔な符号化では,ルール数は3個であった.
	 \end{itemize}
   \item \structure{既知の支配数を用いた評価実験.}
	 \begin{itemize}
	  \item SATの問題では唯一$n=16$まで解くなど,部分和符号化の
		優位性を確認できた.
	  \item UNSATの問題では解を得るまでのCPU時間の観点から,
		部分和符号化の優位性を確認できた.
	 \end{itemize}
  \end{itemize}
 \begin{alertblock}{今後の課題}
  \begin{itemize}
   \item クイーン支配問題をより高速に解くASP符号化の考案.
   \item 遷移問題への拡張.
  \end{itemize}
 \end{alertblock}
\end{frame}

%
%付録
%

%%%% 補助スライド

\begin{frame}{~}
 \centering
 - 補足用 -
\end{frame} 

\begin{frame}{補足 : スマートグリッド}
 \begin{itemize}
  \item \structure{スマートグリッド}とは,電力の供給側,需要側において双方向の
		やり取りを可能にする次世代の\structure{賢い}電力網である.
  \item 従来と違い,通信技術の発達により,使用状況などを
		リアルタイムに把握することが可能となった.
  \item その時に応じた最適な配電網を構成し,制御するといったことが考えられている.
		\begin{itemize}
		 \item 電力需要の変化による,配電ロスの少ない構成.
		 \item 自然エネルギーによる発電量の変動を補う構成.
		\end{itemize}
  \item ASP言語の表現力や拡張性が,こうした条件の追加に活用できる可能性がある.
 \end{itemize}
\end{frame}

%%%%%%%%%%%%%%%%%%%%%%%%%%%%%%%%%%%%%%%%%%%%%%%%%%
%% 電気制約
%%%%%%%%%%%%%%%%%%%%%%%%%%%%%%%%%%%%%%%%%%%%%%%%%%
\begin{frame}{補足 : 電気制約}
 \begin{itemize}
  \item \alert{電気制約}は,送電する電流$\cdot$電圧の適正範囲を保証する制約.
  \begin{itemize}
   \item 供給経路の各区間で許容電流を超えない.
   \item 電気抵抗による電圧降下が許容範囲を超えない.
   \item etc.
  \end{itemize}
  \item 電流と電圧が影響し合う\structure{実数ドメイン上の制約}によって表される.
		% \begin{itemize}
		%  		 \item 送電システム上の条件など.
		% \end{itemize}
  \item 実数ドメイン上の制約は,純粋なASPのみで扱うのは\alert{困難}.
		\begin{itemize}
		 \item 緩和問題として,変電所から供給できる家庭の数に上限をつける.
		 \item ASPMT技術により,ASPで得られた解について,
			   背景理論ソルバーと連携して実数ドメイン上の制約を調べる.
		\end{itemize}
 \end{itemize}
\end{frame}


%%%%%%%%%%%%%%%%%%%%%%%%%%%%%%%%%%%%%%%%%%%%%%%%%%
%% 基礎化
%%%%%%%%%%%%%%%%%%%%%%%%%%%%%%%%%%%%%%%%%%%%%%%%%%
\begin{frame}{補足 : ASPシステム}
 
 \vspace{-0.5cm}

 \begin{figure}[htbp]
  \centering
  %%%%%%%%%%%%%%%%%%%%%%%%%%%%%%%%%%%%%%%%%%%%%%%%%%
%% 基礎化の流れの図
%%%%%%%%%%%%%%%%%%%%%%%%%%%%%%%%%%%%%%%%%%%%%%%%%%
\begin{tikzpicture}

 \definecolor{edge}{RGB}{38,38,134}
 \definecolor{node}{RGB}{220,220,249}

 \definecolor{alert_edge}{RGB}{191,0,0}
 \definecolor{alert_node}{RGB}{249,200,200}

 \definecolor{ex_edge}{RGB}{0,96,0}
 \definecolor{ex_node}{RGB}{230,239,230}

 \def\nodespace{2.4cm}

 \tikzset{block/.style={rectangle, thick, draw=edge, fill=node, text width=3cm, 
 text centered, rounded corners, text width=2cm, minimum height=1.5cm}};

 \tikzset{alertblock/.style={rectangle, thick, draw=alert_edge, fill=alert_node, 
 text width=3cm, text centered, rounded corners, text width=1.5cm, minimum height=1.2cm}};

 \node[block](ikkai){一階ASP\\プログラム};

 \node[rectangle,rounded corners, thick, draw=ex_edge, fill=ex_node, 
 right=0.22*\nodespace of ikkai, minimum width=6cm, minimum height=3cm, 
 text centered, label=ASPシステム](sys){};

 \node[block, right=\nodespace of ikkai](meidai){命題ASP\\プログラム};
 \node[block, right=\nodespace of meidai](ASP){解集合};

 \node[right=0.6*\nodespace of ikkai, text width=1.5cm, 
 text centered, text=red, anchor=south](){基礎化\\ソルバー};
 \node[right=0.4*\nodespace of meidai, text width=1.5cm, 
 text centered, text=red, anchor=south](){解集合\\ソルバー};

 
 \foreach \u / \v / \n in {ikkai/meidai,meidai/ASP}
 \draw [thick,->] (\u) to (\v);

\end{tikzpicture}
 \end{figure}

 \vspace{-0.5cm}

 \begin{exampleblock}{}
  \begin{enumerate}
   \item 一階ASPプログラムを基礎化ソルバーによって,
		 命題ASPプログラムに\alert{基礎化}する.
   \item 命題ASPプログラムについて,SAT技術を応用した解集合ソルバーが解集合を探索する.
  \end{enumerate}
 \end{exampleblock}

\end{frame}


%%%%%%%%%%%%%%%%%%%%%%%%%%%%%%%%%%%%%%%%%%%%%%%%%%
%% ASPのコード
%%%%%%%%%%%%%%%%%%%%%%%%%%%%%%%%%%%%%%%%%%%%%%%%%%
\begin{frame}[fragile]{補足 : 基本符号化のASPプログラム}
 \begin{exampleblock}{}
  \begin{center}
   %%%%%%%%%%%%%%%%%%%%%%%%%%%%%%%%%
   \lstinputlisting[numbers=left,%
   basicstyle=\ttfamily\tiny]{code/srf1.lp}
   %%%%%%%%%%%%%%%%%%%%%%%%%%%%%%%%% 
  \end{center}
 \end{exampleblock}
\end{frame}

\begin{frame}[fragile]{補足 : 改良符号化のASPプログラム}

 \begin{exampleblock}{}
  \begin{center}
   %%%%%%%%%%%%%%%%%%%%%%%%%%%%%%%%%
   \lstinputlisting[numbers=left,%
   basicstyle=\ttfamily\tiny]{code/srf2.lp}
   %%%%%%%%%%%%%%%%%%%%%%%%%%%%%%%%% 
  \end{center}
 \end{exampleblock}

\end{frame}



\end{document}