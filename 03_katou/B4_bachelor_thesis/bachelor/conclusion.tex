%%%%%%%%%%%%%%%%%%%%%%%%%%%%%%%%%%%%%%%%%%%%%%%%%%%%%%%%%% 
\chapter{結論}\label{chap:conclusion}
%%%%%%%%%%%%%%%%%%%%%%%%%%%%%%%%%%%%%%%%%%%%%%%%%%%%%%%%%%

本論文では,解集合プログラミング(ASP)を用いたクイーン支配問題の解法について述べた.

SATソルバーを用いた既存研究~\cite{yamamoto21}で考案された
基本モデル,改良モデル,部分和モデルを,
クイーン支配問題を解くASP符号化としてそれぞれ2種類ずつ提案した.
基本モデルは,各マスに対してその隣接頂点に
1つ以上クイーンが配置されていなければならないことを
一貫性制約を用いて記述した符号化である.
改良モデルは,各マスに対してその隣接頂点に
1つ以上クイーンが配置されていなければならないことを
一貫性制約を用いて記述した符号化である.
部分和モデルは,各行,各列,各対角線方向に置かれている
クイーンの個数を表す補助整数変数を導入し,
行方向,列方向,右上がり対角線方向,右下がり対角線方向のクイーンの和が
サイズ$k$に一致する制約を追加した符号化である.

考案した符号化の有効性を評価するために,
サイズ$10 \times 10$から$20 \times 20$のクイーングラフと
既知の支配数を用いて実行実験を行なった.
その結果,部分和モデルがSAT,UNSATのどちらにおいても
基本モデル,改良モデルと比較してより高速に解くことに成功し,
その優位性を確認できた.

今後の課題としては,SATソルバーを用いた既存研究~\cite{yamamoto21}での
実験結果と比較すると実験結果があまり良くないため,
より高速にクイーン支配問題を解くことができるASP符号化の提案と,
遷移問題への拡張を考えている.

%%% Local Variables:
%%% mode: latex
%%% TeX-master: "paper"
%%% End:
