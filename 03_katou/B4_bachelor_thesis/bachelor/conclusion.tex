%%%%%%%%%%%%%%%%%%%%%%%%%%%%%%%%%%%%%%%%%%%%%%%%%%%%%%%%%% 
\chapter{結論}\label{chap:conclusion}
%%%%%%%%%%%%%%%%%%%%%%%%%%%%%%%%%%%%%%%%%%%%%%%%%%%%%%%%%%

本論文では,解集合プログラミング(ASP)を用いたクイーン支配問題の解法について述べた.

SATソルバーを用いた既存研究~\cite{yamamoto21}で考案された基本モデル,改良モデル,部分和モデルを,クイーン支配問題を解くASP符号化としてそれぞれ2種類ずつ提案した.
基本モデルは,各マスに対してその隣接頂点にひとつ以上クイーンが配置されていなければならないことを一貫性制約を用いて記述した符号化である.
改良モデルは,各マスに対してその隣接頂点にひとつ以上クイーンが配置されていなければならないことを一貫性制約を用いて記述した符号化である.
部分和モデルは,各行,各列,各対角線方向に置かれているクイーンの個数を表す補助整数変数を導入し,行方向,列方向,右上がり対角線方向,右下がり対角線方向のクイーンの和がサイズ$k$に一致する制約を追加した符号化である.

考案した符号化の有効性を評価するために,サイズ$10 \times 10$から$20 \times 20$のクイーングラフと既知の支配数を用いて実行実験を行なった.
その結果,部分和モデルがSAT,UNSATのどちらにおいても基本モデル,改良モデルと比較してより高速に解くことに成功し,その優位性を確認できた.

今後の課題としては,SATソルバーを用いた既存研究~\cite{yamamoto21}での実験結果と比較すると実験結果があまり良くないため,より高速にクイーン支配問題を解くことができるASP符号化の提案と,遷移問題への拡張を考えている.

%%% Local Variables:
%%% mode: latex
%%% TeX-master: "paper"
%%% End:
