%%%%%%%%%%%%%%%%%%%%%%%%%%%%%%%%%%%%%%%%%%%%%%%%%%%%%%%%%% 
\chapter{結論}\label{chap:conclusion}
%%%%%%%%%%%%%%%%%%%%%%%%%%%%%%%%%%%%%%%%%%%%%%%%%%%%%%%%%%

本論文では,解集合プログラミング(ASP)を用いたクイーン支配問題の解法について述べた.
%
クイーン支配問題に対して,基本符号化,改良符号化,部分和符号化の3種類
の ASP 符号化を考案した.
%
基本符号化は,チェス盤の各マスに1つ以上のクイーンが移動できることを,
ASP の一貫性制約を用いて表現した符号化である.
%
改良符号化は,各行,各列,各対角線(右上がりと右下がりの2種類)に対して
クイーンが配置されているかどうかを表す補助アトムを導入し,
基本符号化で複数回出現する制約式をまとめた符号化である.
%
部分和符号化は,各行,各列,各対角線に対して配置されているクイーンの個
数を表す補助アトムを導入し,それぞれの和がサイズ$k$に一致することをASP
の個数制約を用いて表した符号化である.

$n$次のクイーン支配問題 $(1\leq n\leq 20)$について,最も難しい
サイズ$\gamma(Q_n)$と$\gamma(Q_n)-1$の2種類を用いて実行実験を行なった.
その結果,部分和符号化が$\gamma(Q_{16})$まで解くことができ,他の符号化と比較して
その優位性を確認できた.

今後の課題としては,SATソルバーを用いた先行研究~\cite{yamamoto21}での
実験結果と比較すると提案した符号化の性能があまり良くないため,
より高速にクイーン支配問題を解くことができるASP符号化の提案と,
クイーン支配問題の一方の実行可能解から他方の実行可能解へ,
遷移制約を満たしつつ実行可能解のみを経由しながら遷移できるかどうかを判定する
遷移問題への拡張を考えている.