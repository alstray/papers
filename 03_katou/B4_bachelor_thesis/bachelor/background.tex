%%%%%%%%%%%%%%%%%%%%%%%%%%%%%%%%%%%%%%%%%%%%%%%%%%%%%%%%%% 
\chapter{クイーン支配問題}\label{chap:background}
%%%%%%%%%%%%%%%%%%%%%%%%%%%%%%%%%%%%%%%%%%%%%%%%%%%%%%%%%% 

\section{支配集合問題}
グラフ$G=(V,E)$の頂点の部分集合$S\subset V$とその隣接頂点の集合との和集合が$V$と一致するとき,$S$を$G$の\textbf{支配集合}という.\par
支配集合$S$の要素数を\textbf{要素数}という.
 \begin{itemize}
  \item サイズが最小の支配集合を\textbf{最小支配集合}という.
  \item 最小支配集合のサイズをグラフ$G$の\textbf{支配数}と呼ぶ.本論文ではグラフ$G$の支配数を$\gamma(G)$で表す.
 \end{itemize}
  グラフ$G$と正の整数$k$が与えられたとき,サイズが$k$の$G$の支配集合が存在するかどうかを判定する問題を\textbf{支配集合問題}という.
\section{クイーン支配問題}
$n\times n$のチェス盤について各マスを頂点とし,クイーンが移動できるマス同士が辺で結ばれているグラフを\textbf{クイーングラフ}という.サイズ$n\times n$のクイーングラフを$Q_n$で表す.\par
例えば,サイズ3のチェス盤とクイーングラフ$Q_3$の対応関係は以下のようになる.
% \begin{figure}[tb]
%   \centering
%   \includegraphics[width=0.6\linewidth]{fig/q_graph.png}
%   \caption{$Q_8$の最小支配集合の例}
%   \label{ex:queengraph}
% \end{figure}
クイーングラフ$Q_n$と正の整数$k$が与えられたとき,サイズ$k$の$Q_n$の支配集合が存在するかどうか判定する問題を\textbf{クイーン支配問題}という.\par
図\ref{ex:queengraph}は,$Q_8$の最小支配集合の例である.5個のクイーンを置いたとき,クイーンを移動させて全てのマスにアタック可能である.また,4個以下のクイーンを置いたとき,クイーンを移動させて全てのマスにアタックすることは不可能である.
% \begin{figure}[tb]
%   \centering
%   \includegraphics[width=0.6\linewidth]{fig/ex_queen.png}
%   \caption{$Q_8$の最小支配集合の例}
%   \label{ex:queengraph}
% \end{figure}



%%% Local Variables:
%%% mode: latex
%%% TeX-master: "paper"
%%% End:
