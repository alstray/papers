%%%%%%%%%%%%%%%%%%%%%%%%%%%%%%%%%%%%%%%%%%%%%%%%%%%%%%%%%% 
\chapter{序論}\label{chap:introduction}
\pagenumbering{arabic}
%%%%%%%%%%%%%%%%%%%%%%%%%%%%%%%%%%%%%%%%%%%%%%%%%%%%%%%%%% 

\textbf{クイーン支配問題(Queen Domination Problem; QDP)}は,
$n \times n$のクイーングラフ$Q_{n}$と正の整数$k$が与えられたとき,
サイズが$k$である$Q_{n}$の支配集合が存在するかどうかを判定する支配集合問題の一種である.
支配集合問題はNP完全に属する問題であり,スケジューリングなど多くの問題に応用されている.
特に,サイズが最小の支配集合を求める最小支配集合問題は,電波塔の最適な配置といった現実の問題への実用的応用が考えられる.


\textbf{解集合プログラミング(Answer Set Programming; ASP}\cite{%
  Baral03:cambridge,%
  Gelfond88:iclp,%
  Inoue08:jssst,%
  Niemela99:amai})
は,論理プログラミングから派生した比較的新しいプログラミングパラダイムである.
ASP言語は,一階論理に基づく知識表現言語の一種であり,
論理プログラムは ASP のルールの有限集合である.
ASP システムは論理プログラムから安定モデル意味論に基づく解集合を計算するシステムである.
近年,SAT技術を応用した高速 ASP システムが開発され,スケジューリング,
プランニング,システム生物学,システム検証,制約充足問題,
制約最適化問題など様々な分野への実用的応用が急速に拡大している.
クイーン支配問題に対して ASP を用いる利点として,
ASP 言語の高い表現力が挙げられる.重み付き個数制約や選択子といった構文が存在し,
複雑な問題の制約を簡潔に記述することが可能である.

本論文では,解集合プログラミング(ASP)を用いたクイーン支配問題の解法について述べる.

SATソルバーを用いた既存研究~\cite{yamamoto21}で考案された基本モデル,
改良モデル,部分和モデルの3つのASP符号化を,
クイーン支配問題を解く符号化として,それぞれ2つ提案する.

基本モデルは,各マスに対してその隣接頂点に1つ以上クイーンが配置されていなければならないことを,
一貫性制約を用いて記述した符号化である.
改良モデルは,各行,各列,各対角線方向に対して
クイーンが配置されているかを表す補助ブール変数を導入し,
基本モデルで複数回出現する部分式をまとめた符号化である.
部分和モデルは,各行,各列,各対角線方向に置かれている
クイーンの個数を表す補助整数変数を導入し,行方向,列方向,右上がり対角線方向,
右下がり対角線方向のクイーンの和がサイズ$k$に一致する制約を追加した符号化である.

考案した符号化の有効性を評価するために,
サイズ$10 \times 10$から$20 \times 20$のクイーングラフと
既知の支配数を用いて実行実験を行なった.
その結果,部分和モデルがSAT,UNSATのどちらにおいても基本モデル,改良モデルと
比較してより高速に解くことに成功し,その優位性を確認できた.

本論文の構成は以下の通りである.
まず,第~\ref{chap:background}章では,本論文が対象とする
クイーン支配問題について述べる.
第~\ref{chap:asp}章では,解集合プログラミングの言語構文,
システム,プログラム例を説明する.
第~\ref{chap:constraint}章では,SATソルバーを用いた既存研究~\cite{yamamoto21}
で提案された制約モデルの説明を行う.
第~\ref{chap:proposal}章では,第~\ref{chap:constraint}章で
説明した制約モデルのASP符号化を提案する.
第~\ref{chap:experiment}章では,提案した符号化に対して
行なった実行実験について述べる.
第~\ref{chap:conclusion}章で本論文の結論を述べる.

%%% Local Variables:
%%% mode: latex
%%% TeX-master: "paper"
%%% End:
