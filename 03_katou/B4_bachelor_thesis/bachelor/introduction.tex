%%%%%%%%%%%%%%%%%%%%%%%%%%%%%%%%%%%%%%%%%%%%%%%%%%%%%%%%%% 
\chapter{序論}\label{chap:introduction}
\pagenumbering{arabic}
%%%%%%%%%%%%%%%%%%%%%%%%%%%%%%%%%%%%%%%%%%%%%%%%%%%%%%%%%% 

\textbf{クイーン支配問題(Queen Domination Problem; QDP)}は,$n \times n$のクイーングラフ$Q_{n}$と正の整数$k$が与えられたとき,サイズが$k$である$Q_{n}$の支配集合が存在するかどうか判定する問題である.

\textbf{解集合プログラミング(Answer Set Programming; ASP}\cite{%
  Baral03:cambridge,%
  Gelfond88:iclp,%
  Inoue08:jssst,%
  Niemela99:amai})
は,論理プログラミングから派生した比較的新しいプログラミングパラダイムである.
ASP言語は,一階論理に基づく知識表現言語の一種であり,
論理プログラムは ASP のルールの有限集合である.
ASP システムは論理プログラムから安定モデル意味論に基づく解集合を計算す
るシステムである.
近年,SAT技術を応用した高速 ASP システムが開発され,スケジューリング,
プランニング,システム生物学,システム検証,制約充足問題,
制約最適化問題など様々な分野への実用的応用が急速に拡大している.
ハミルトン閉路問題およびその関連問題に対して ASP を用いる利点としては,
ASP 言語の高い表現力,
充足不能コアに基づく最適化,
インクリメンタルASP解法,
組込み非閉路制約,
高速な解列挙
などが挙げられる.

本論文では,解集合プログラミング(ASP)を用いたクイーン支配問題の解法について述べる.

SATソルバーを用いた既存研究~\cite{yamamoto21}で考案された基本モデル,改良モデル,部分和モデルの3つを,クイーン支配問題を解く符号化として,それぞれ2つずつASP符号化を提案する.

基本モデルは,各マスに対してその隣接頂点にひとつ以上クイーンが配置されていなければならないことを一貫性制約を用いて記述した符号化である.改良モデルは,各行,各列,各対角線方向に対してクイーンが配置されているかを表す補助ブール変数を導入し,基本モデルで複数回出現する部分式をまとめた符号化である.部分和モデルは,各行,各列,各対角線方向に置かれているクイーンの個数を表す補助整数変数を導入し,行方向,列方向,右上がり対角線方向,右下がり対角線方向のクイーンの和がサイズ$k$に一致する制約を追加した符号化である.

考案した符号化の有効性を評価するために,サイズ$10 \times 10$から$20 \times 20$のクイーングラフと既知の支配数を用いて実行実験を行なった.
その結果,部分和モデルがSAT,UNSATのどちらにおいても基本モデル,改良モデルと比較してより高速に解くことに成功し,その優位性を確認できた.

本論文の構成は以下の通りである.
まず,第~\ref{chap:background}章では,本論文が対象とするクイーン支配問題について述べる.
第~\ref{chap:asp}章では,解集合プログラミングの言語構文,システム,プログラム例を説明する.
第~\ref{chap:constraint}章では,SATソルバーを用いた既存研究~\cite{yamamoto21}で提案された制約モデルについての説明を行う.
第~\ref{chap:proposal}章では,第~\ref{chap:constraint}章で説明した制約モデルのASP符号化を提案する.
第~\ref{chap:experiment}章では,提案した符号化に対して行なった実行実験について述べる.
第~\ref{chap:conclusion}章で本論文の結論を述べる.

%%% Local Variables:
%%% mode: latex
%%% TeX-master: "paper"
%%% End:
