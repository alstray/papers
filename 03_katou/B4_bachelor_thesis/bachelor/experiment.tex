%%%%%%%%%%%%%%%%%%%%%%%%%%%%%%%%%%%%%%%%%%%%%%%%%%%%%%%%%% 
\chapter{実行実験}\label{chap:experiment}
%%%%%%%%%%%%%%%%%%%%%%%%%%%%%%%%%%%%%%%%%%%%%%%%%%%%%%%%%% 
本章では,前章で提案した3つのASP符号化である,基本モデル,改良モデル,部分和モデルの性能を評価するために実行実験を行なった.実験の概要は以下の通りである.
\begin{itemize}
 \item  $10 \leq n \leq 20$
 \item $ k = \gamma(Q_(n))$: SAT,$\gamma(Q_(n))-1$: UNSATの2種類
 \item 使用ASPソルバ: \textit{clingo-5.5.0}
       \begin{itemize}
	\item \textit{configuration}は\textit{handy}を使用.
       \end{itemize}
 \item 実行環境: Mac mini, 3.2GHz 6コア Intel Core i7, 64GBメモリ
 \item 制限CPU時間: 3600 (sec)
\end{itemize}

この実験に対して,それぞれのモデルの実験結果を示す.
\begin{table}[ht]
 \caption{$k=\gamma(Q_n)$:SATの実験結果}
 \label{tb:exSAT}
 \centering
 \begin{tabular}{|c|c|r|r|r|r|r|r|} \hline
  $n$ & $k$ & 基本1 & 基本2 & 改良1 & 改良2 & 部分和1 & 部分和2 \\ \hline
  10 & 5 & 14.400 & 29.120 & 21.232 & 10.814 & 4.852 & \textbf{3.005} \\
  11 & 5 & 54.674 & 618.276 & 123.488 & 448.659 & \textbf{5.414} & 32.540 \\
  12 & 6 & T.O. & T.O. & 56.468 & T.O. & 437.161 & \textbf{2.147} \\
  13 & 7 & T.O. & T.O. & 1281.823 & T.O. & \textbf{0.093} & 22.412 \\
  14 & 8 & 2797.041 & T.O. & 772.276 & T.O. & 594.860 & \textbf{10.044}\\
  15 & 9 & T.O. & T.O. & T.O. & T.O. & \textbf{0.073} & 275.683 \\
  16 & 9 & T.O. & T.O. & T.O. & T.O. & T.O. & \textbf{734.878} \\
  17 & 9 & T.O. & T.O. & T.O. & T.O. & T.O. & T.O. \\
  18 & 9 & T.O. & T.O. & T.O. & T.O. & T.O. & T.O. \\
  19 & 10 & T.O. & T.O. & T.O. & T.O. & T.O. & T.O. \\
  20 & 11 & T.O. & T.O. & T.O. & T.O. & T.O. & T.O. \\ \hline
 \end{tabular}
\end{table}

\begin{table}[ht]
 \caption{$k=\gamma(Q_n)-1$:UNSATの実験結果}
 \label{tb:exUNSAT}
 \centering 
 \begin{tabular}{|c|c|r|r|r|r|r|r|} \hline
  $n$ & $k$ & 基本1 & 基本2 & 改良1 & 改良2 & 部分和1 & 部分和2 \\ \hline
  10 & 4 & 10.280 & 10.144 & 8.929 & \textbf{1.538} & 2.561 & 2.612 \\
  11 & 4 & 28.184 & 28.043 & 24.537 & \textbf{2.894} & 3.236 & 3.427 \\
  12 & 5 & 2706.092 & 2643.216 & 2673.241 & 2718.137 & 355.526 & \textbf{337.127} \\
  13 & 6 & T.O. & T.O. & T.O. & T.O. & T.O. & T.O. \\  
  14 & 7 & T.O. & T.O. & T.O. & T.O. & T.O. & T.O. \\   
  15 & 8 & T.O. & T.O. & T.O. & T.O. & T.O. & T.O. \\  
  16 & 8 & T.O. & T.O. & T.O. & T.O. & T.O. & T.O. \\
  17 & 8 & T.O. & T.O. & T.O. & T.O. & T.O. & T.O. \\
  18 & 8 & T.O. & T.O. & T.O. & T.O. & T.O. & T.O. \\
  19 & 9 & T.O. & T.O. & T.O. & T.O. & T.O. & T.O. \\
  20 & 10 & T.O. & T.O. & T.O. & T.O. & T.O. & T.O. \\ \hline
 \end{tabular}
\end{table}

SATの問題においては,$n$が奇数のとき部分和モデル1が,$n$が偶数のとき部分和モデル2が最も良い性能を示した.特に部分和モデル2は,唯一$n=16$の問題を解いた.各モデルでは,基本モデルでは基本モデル1が基本モデル2よりも良い性能を示し,改良モデルでは改良モデル1のほうが改良モデル2に比べ良い性能を示した.

UNSATの問題においては,どのモデルも$n=12$までしか解くことができなかった.$n=10,11$のときは改良モデル2が,$n=12$では部分和モデルが良い性能を示したことがわかった.どのモデルにおいても,モデル1と2では大きな差異は見られなかった.
%%% Local Variables:
%%% mode: latex
%%% TeX-master: "paper"
%%% End:
