%%%%%%%%%%%%%%%%%%%%%%%%%%%%%%%%%%%%%%%%%%%%%%%%%%%%%%%%%% 
\chapter{実行実験}\label{chap:experiment}
%%%%%%%%%%%%%%%%%%%%%%%%%%%%%%%%%%%%%%%%%%%%%%%%%%%%%%%%%% 
本章では,前章で提案した3つのASP符号化である,基本モデル,改良モデル,部分和モデルの性能を評価するために実行実験を行なった.実験の概要は以下の通りである.
\begin{itemize}
 \item  $10 \leq n \leq 20$
 \item $ k = \gamma(n)$: SAT,$\gamma(n)-1$: UNSATの2種類
 \item 使用ASPソルバ: \textit{clingo-5.5.0}
       \begin{itemize}
	\item \textit{configuration}は\textit{trendy, handy}を使用.
       \end{itemize}
 \item 実行環境: Mac mini, 3.2GHz 6コア Intel Core i7, 64GBメモリ
 \item 制限CPU時間: 3600 (sec)
\end{itemize}

この実験に対して,それぞれのモデルとの実験結果を示す.
\begin{table}[ht]
 \caption{$k=\gamma(n)$:SATの実験結果(trendy)}
 \label{tb:exSAT}
 \centering
 \begin{tabular}{|c|c|r|r|r|} \hline
  $n$ & $k$ & 基本モデル1 & 基本モデル2 & 改良モデル1 \\ \hline
  10 & 5 & 8.467 & 5.470 & 6.292 \\
  11 & 5 & 189.473 & 595.193 & 602.542 \\
  12 & 6 & 1349.260 & 1335.134 & T.O. \\
  13 & 7 & T.O. & T.O. & T.O. \\
  14 & 8 & T.O. & T.O. & T.O. \\
  15 & 9 & T.O. & T.O. & T.O. \\
  16 & 9 & T.O. & T.O. & T.O. \\
  17 & 9 & T.O. & T.O. & T.O. \\
  18 & 9 & T.O. & T.O. & T.O. \\
  19 & 10 & T.O. & T.O. & T.O. \\
  20 & 11 & T.O. & T.O. & T.O. \\ \hline \hline
  $n$ & $k$ & 改良モデル2 & 部分和モデル1 & 部分和モデル2 \\ \hline
  10 & 5 & 5.312 & 3.369 & 0.185 \\
  11 & 5 & 212.930 & 51.558 & 10.281 \\
  12 & 6 & 2929.741 & 32.588 & 7.491 \\
  13 & 7 & T.O. & 0.098 & 122.809 \\
  14 & 8 & T.O. & 1028.098 & 146.556 \\
  15 & 9 & T.O. & 0.069 & 94.653 \\
  16 & 9 & T.O. & T.O. & 2277.852 \\
  17 & 9 & T.O. & T.O. & T.O. \\
  18 & 9 & T.O. & T.O. & T.O. \\
  19 & 10 & T.O. & T.O. & T.O.  \\
  20 & 11 & T.O. & T.O. & T.O. \\ \hline
 \end{tabular}
\end{table}

\begin{table}[ht]
 \caption{$k=\gamma(n)$:SATの実験結果(handy)}
 \label{tb:exSAT}
 \centering
 \begin{tabular}{|c|c|r|r|r|} \hline
  n & k & 基本モデル1 & 基本モデル2 & 改良モデル1 \\ \hline
  10 & 5 & 14.400 & 29.120 & 21.232 \\
  11 & 5 & 54.674 & 618.276 & 123.488 \\
  12 & 6 & T.O. & T.O. & 56.468 \\
  13 & 7 & T.O. & T.O. & 1281.823 \\
  14 & 8 & 2797.041 & T.O. & 772.276 \\
  15 & 9 & T.O. & T.O. & T.O. \\
  16 & 9 & T.O. & T.O. & T.O. \\
  17 & 9 & T.O. & T.O. & T.O. \\
  18 & 9 & T.O. & T.O. & T.O. \\
  19 & 10 & T.O. & T.O. & T.O. \\
  20 & 11 & T.O. & T.O. & T.O. \\ \hline \hline
  $n$ & $k$ & 改良モデル2 & 部分和モデル1 & 部分和モデル2 \\ \hline
  10 & 5 & 10.814 & 4.852 & 3.005 \\
  11 & 5 & 448.659 & 5.414 & 32.540 \\
  12 & 6 & T.O. & 437.161 & 2.147 \\
  13 & 7 & T.O. & 0.093 & 22.412 \\
  14 & 8 & T.O. & 594.860 & 10.044\\
  15 & 9 & T.O. & 0.073 & 275.683 \\
  16 & 9 & T.O. & T.O. & 734.878 \\
  17 & 9 & T.O. & T.O. & T.O. \\
  18 & 9 & T.O. & T.O. & T.O. \\
  19 & 10 & T.O. & T.O. & T.O. \\
  20 & 11 & T.O. & T.O. & T.O. \\ \hline
 \end{tabular}
\end{table}

\begin{table}[ht]
 \caption{$k=\gamma(n)-1$:UNSATの実験結果(trendy)}
 \label{tb:exUNSAT}
 \centering
 \begin{tabular}{|c|c|r|r|r|} \hline
  $n$ & $k$ & 基本モデル1 & 基本モデル2 & 改良モデル1 \\ \hline
  10 & 4 & 6.520 & 6.974 & 8.519 \\
  11 & 4 & 16.468 & 18.341 & 18.169 \\
  12 & 5 & 3071.506 & 2993.934 & 3097.029 \\
  13 & 6 & T.O. & T.O. & T.O. \\  
  14 & 7 & T.O. & T.O. & T.O. \\ 
  15 & 8 & T.O. & T.O. & T.O. \\  
  16 & 8 & T.O. & T.O. & T.O. \\  
  17 & 8 & T.O. & T.O. & T.O. \\  
  18 & 8 & T.O. & T.O. & T.O. \\  
  19 & 9 & T.O. & T.O. & T.O. \\  
  20 & 10 & T.O. & T.O. & T.O. \\ \hline \hline
  $n$ & $k$ & 改良モデル2 & 部分和モデル1 & 部分和モデル2 \\ \hline
  10 & 4 & 1.036 & 1.849 & 1.538 \\
  11 & 4 & 2.122 & 1.819 & 1.694 \\
  12 & 5 & 3012.484 & 129.567 & 187.554 \\
  13 & 6 & T.O. & T.O. & T.O. \\  
  14 & 7 & T.O. & T.O. & T.O. \\ 
  15 & 8 & T.O. & T.O. & T.O. \\  
  16 & 8 & T.O. & T.O. & T.O. \\  
  17 & 8 & T.O. & T.O. & T.O. \\  
  18 & 8 & T.O. & T.O. & T.O. \\  
  19 & 9 & T.O. & T.O. & T.O. \\  
  20 & 10 & T.O. & T.O. & T.O. \\ \hline
 \end{tabular}
\end{table}

\begin{table}[ht]
 \caption{$k=\gamma(n)-1$:UNSATの実験結果(handy)}
 \label{tb:exUNSAT}
 \centering
 \begin{tabular}{|c|c|r|r|r|} \hline
  $n$ & $k$ & 基本モデル1 & 基本モデル2 & 改良モデル1 \\ \hline
  10 & 4 & 10.280 & 10.144 & 8.929 \\
  11 & 4 & 28.184 & 28.043 & 24.537 \\
  12 & 5 & 2706.092 & 2643.216 & 2673.241 \\
  13 & 6 & T.O. & T.O. & T.O. \\  
  14 & 7 & T.O. & T.O. & T.O. \\ 
  15 & 8 & T.O. & T.O. & T.O. \\  
  16 & 8 & T.O. & T.O. & T.O. \\  
  17 & 8 & T.O. & T.O. & T.O. \\  
  18 & 8 & T.O. & T.O. & T.O. \\  
  19 & 9 & T.O. & T.O. & T.O. \\  
  20 & 10 & T.O. & T.O. & T.O. \\ \hline \hline
  $n$ & $k$ & 改良モデル2 & 部分和モデル1 & 部分和モデル2 \\ \hline
  10 & 4 & 1.538 & 2.561 & 2.612 \\
  11 & 4 & 2.894 & 3.236 & 3.427 \\
  12 & 5 & 2718.137 & 355.526 & 337.127 \\
  13 & 6 & T.O. & T.O. & T.O. \\  
  14 & 7 & T.O. & T.O. & T.O. \\ 
  15 & 8 & T.O. & T.O. & T.O. \\  
  16 & 8 & T.O. & T.O. & T.O. \\  
  17 & 8 & T.O. & T.O. & T.O. \\  
  18 & 8 & T.O. & T.O. & T.O. \\  
  19 & 9 & T.O. & T.O. & T.O. \\  
  20 & 10 & T.O. & T.O. & T.O. \\ \hline
 \end{tabular}
\end{table}
%%% Local Variables:
%%% mode: latex
%%% TeX-master: "paper"
%%% End:
