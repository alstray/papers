%%%%%%%%%%%%%%%%%%%%%%%%%%%%%%%%%%%%%%%%%%%%%%%%%%%%%%%%%% 
\chapter{実行実験}\label{chap:experiment}
%%%%%%%%%%%%%%%%%%%%%%%%%%%%%%%%%%%%%%%%%%%%%%%%%%%%%%%%%% 
本章では,前章で提案した3つのASP符号化である,
基本符号化,改良符号化,部分和符号化の性能を
評価するために実行実験を行なった.
実験の概要は以下の通りである.
\begin{itemize}
 \item ベンチマーク問題
 \begin{itemize}
      \item グラフ: $Q_{n}$ $10 \le n \le 20$ 
      \item サイズ: $ k = \gamma(Q_n)$ (SAT),$\gamma(Q_n)-1$ (UNSAT)%の2種類
 \end{itemize}
 %\item  $10 \leq n \leq 20$
 %\item $ k = \gamma(Q_(n))$: SAT,$\gamma(Q_(n))-1$: UNSATの2種類
 \item 使用ASPソルバ: \textit{clingo-5.5.0}
       \begin{itemize}
	\item \textit{configuration}は\textit{handy}を使用.
       \end{itemize}
 \item 実行環境: Mac mini, 3.2GHz 6コア Intel Core i7, 64GBメモリ
 \item 制限CPU時間: 3600 (sec)
\end{itemize}

この実験に対して,それぞれの符号化の実験結果を表~\ref{tb:exSAT},~\ref{tb:exUNSAT}に示す.なお,表中のT.O.は制限CPU時間内に解が得られなかったことを表す.
\begin{table}[ht]
 \caption{$k=\gamma(Q_n)$: SATの実験結果}
 \label{tb:exSAT}
 \centering
 \begin{tabular}{c|c|r|r|r|r|r|r} \hline
  $n$ & $k$ & 基本1 & 基本2 & 改良1 & 改良2 & 部分和1 & 部分和2 \\ \hline
  10 & 5 & 14.400 & 29.120 & 21.232 & 10.814 & 4.852 & \textbf{3.005} \\
  11 & 5 & 54.674 & 618.276 & 123.488 & 448.659 & \textbf{5.414} & 32.540 \\
  12 & 6 & T.O. & T.O. & 56.468 & T.O. & 437.161 & \textbf{2.147} \\
  13 & 7 & T.O. & T.O. & 1281.823 & T.O. & \textbf{0.093} & 22.412 \\
  14 & 8 & 2797.041 & T.O. & 772.276 & T.O. & 594.860 & \textbf{10.044}\\
  15 & 9 & T.O. & T.O. & T.O. & T.O. & \textbf{0.073} & 275.683 \\
  16 & 9 & T.O. & T.O. & T.O. & T.O. & T.O. & \textbf{734.878} \\
  17 & 9 & T.O. & T.O. & T.O. & T.O. & T.O. & T.O. \\
  18 & 9 & T.O. & T.O. & T.O. & T.O. & T.O. & T.O. \\
  19 & 10 & T.O. & T.O. & T.O. & T.O. & T.O. & T.O. \\
  20 & 11 & T.O. & T.O. & T.O. & T.O. & T.O. & T.O. \\ \hline
 \end{tabular}
\end{table}

\begin{table}[ht]
 \caption{$k=\gamma(Q_n)-1$: UNSATの実験結果}
 \label{tb:exUNSAT}
 \centering 
 \begin{tabular}{c|c|r|r|r|r|r|r} \hline
  $n$ & $k$ & 基本1 & 基本2 & 改良1 & 改良2 & 部分和1 & 部分和2 \\ \hline
  10 & 4 & 10.280 & 10.144 & 8.929 & \textbf{1.538} & 2.561 & 2.612 \\
  11 & 4 & 28.184 & 28.043 & 24.537 & \textbf{2.894} & 3.236 & 3.427 \\
  12 & 5 & 2706.092 & 2643.216 & 2673.241 & 2718.137 & 355.526 & \textbf{337.127} \\
  13 & 6 & T.O. & T.O. & T.O. & T.O. & T.O. & T.O. \\  
  14 & 7 & T.O. & T.O. & T.O. & T.O. & T.O. & T.O. \\   
  15 & 8 & T.O. & T.O. & T.O. & T.O. & T.O. & T.O. \\  
  16 & 8 & T.O. & T.O. & T.O. & T.O. & T.O. & T.O. \\
  17 & 8 & T.O. & T.O. & T.O. & T.O. & T.O. & T.O. \\
  18 & 8 & T.O. & T.O. & T.O. & T.O. & T.O. & T.O. \\
  19 & 9 & T.O. & T.O. & T.O. & T.O. & T.O. & T.O. \\
  20 & 10 & T.O. & T.O. & T.O. & T.O. & T.O. & T.O. \\ \hline
 \end{tabular}
\end{table}
\newpage
% さらに,比較として先行研究~\cite{yamamoto21}で得られたSATソルバーの実験結果を表~\ref{tb:exyamamotoSAT},~\ref{tb:exyamamotoUNSAT}に示す.なお,先行研究~\cite{yamamoto21}で行なった実験の概要は以下の通りである.今回は比較のためSAT型CSPソルバの\textit{Sugar}を用いた$10 \le n \le 20$の実験結果を示す.
% \begin{itemize}
%  \item ベンチマーク問題
%  \begin{itemize}
%       \item グラフ: $Q_{n}$ $10 \le n \le 25$ 
%       \item サイズ: $ k = \gamma(Q_n)$ (SAT),$\gamma(Q_n)-1$ (UNSAT)%の2種類
%  \end{itemize}
%  \item 使用CSPソルバ
%        \begin{itemize}
% 	\item \textit{Sugar2.3.3}+\textit{GlueMiniSat2.8.8}
% 	\item \textit{AbsCon19-06}
% 	\item \textit{Choco4.10.6}
%        \end{itemize}
%  \item 実行環境: 3.40GHz CPU,32GB メモリ
%  \item 制限CPU時間: 5400 (sec)
% \end{itemize}
% \begin{table}[ht]
%  \caption{先行研究\cite{yamamoto21}における$k=\gamma(Q_n)$:SATの実験結果}
%  \label{tb:exyamamotoSAT}
%  \centering
%  \begin{tabular}{c|c|r|r|r} \hline
%   $n$ & $k$ & 基本符号化 & 改良符号化 & 部分和符号化 \\ \hline
%   10 & 5 & 2.45 & 18.64 & 1.25 \\ \hline
%   11 & 5 & 9.14 & 2.73 & 1.68 \\ \hline
%   12 & 6 & 515.21 & 48.06 & 4.30 \\ \hline
%   13 & 7 & 224.87 & 126.73 & 1.29 \\ \hline
%   14 & 8 & 1727.85 & T.O. & 1.50 \\ \hline
%   15 & 9 & T.O. & 670.57 & 1.86 \\ \hline
%   16 & 9 & T.O. & T.O. & 5.04 \\ \hline
%   17 & 9 & T.O. & T.O. & 3.49 \\ \hline
%   18 & 9 & T.O. & T.O. & 209.20 \\ \hline
%   19 & 10 & T.O. & T.O. & T.O. \\ \hline
%   20 & 11 & T.O. & T.O. & 63.61 \\ \hline
%  \end{tabular}
% \end{table}
% \begin{table}[ht]
%  \caption{先行研究\cite{yamamoto21}における$k=\gamma(Q_n)-1$:UNSATの実験結果}
%  \label{tb:exyamamotoUNSAT}
%  \centering
%  \begin{tabular}{c|c|r|r|r} \hline
%   $n$ & $k$ & 基本符号化 & 改良符号化 & 部分和符号化 \\ \hline
%   10 & 4 & 4.32 & 3.41 & 0.87 \\ \hline
%   11 & 4 & 5.89 & 5.15 & 0.99 \\ \hline
%   12 & 5 & 538.60 & 201.16 & 1.48 \\ \hline
%   13 & 6 & T.O. & T.O. & 5.67 \\ \hline
%   14 & 7 & T.O. & T.O. & 48.53 \\ \hline
%   15 & 8 & T.O. & T.O. & 701.09 \\ \hline
%   16 & 8 & T.O. & T.O. & 325.69 \\ \hline
%   17 & 8 & T.O. & T.O. & 212.22 \\ \hline
%   18 & 8 & T.O. & T.O. & 121.99 \\ \hline
%   19 & 9 & T.O. & T.O. & 924.61 \\ \hline
%   20 & 10 & T.O. & T.O. & 524.44 \\ \hline
%  \end{tabular}
% \end{table}
\newpage
SATの問題においては,$n$が奇数のとき部分和符号化1が,
$n$が偶数のとき部分和符号化2が最も良い性能を示した.
特に部分和符号化2は,唯一$n=16$の問題を解いた.
各符号化では,基本符号化では基本符号化1が基本符号化2よりも良い性能を示し,
改良符号化では改良符号化1のほうが改良符号化2に比べ良い性能を示した.

UNSATの問題においては,どの符号化も$n=12$までしか解くことができなかった.
$n=10,11$のときは改良符号化2が,$n=12$では部分和符号化が良い性能を示したことがわかった.
どの符号化においても,符号化1と2では大きな差異は見られなかった.

先行研究~\cite{yamamoto21}の実験結果と比較すると,部分和符号化における実験結果が先行研究~\cite{yamamoto21}では$\gamma(Q_{20})$まで解けているのに対して,今回提案した符号化は$\gamma(Q_{16})$までしか解けていないため,より性能の良いASP符号化を再度考え直していく必要がある.
%%% Local Variables:
%%% mode: latex
%%% TeX-master: "paper"
%%% End:
