%%%%%%%%%%%%%%%%%%%%%%%%%%%%%%%%%%%%%%%%%%%%%%%%%%%%%%%%%% 
\chapter*{概要}
\pagenumbering{roman}
%%%%%%%%%%%%%%%%%%%%%%%%%%%%%%%%%%%%%%%%%%%%%%%%%%%%%%%%%% 

本論文では,解集合プログラミング(ASP)を用いたクイーン支配問題の解法について述べる.
クイーン支配問題に対して,基本符号化,改良符号化,
部分和符号化の3種類の ASP 符号化を考案した.
%
基本符号化は,チェス盤の各マスに1つ以上のクイーンが移動できることを,
ASP の一貫性制約を用いて表現した符号化である.
%
改良符号化は,各行,各列,各対角線(右上がりと右下がりの2種類)に対して
クイーンが配置されているかどうかを表す補助アトムを導入し,
基本符号化で複数回出現する制約式をまとめた符号化である.
%
部分和符号化は,各行,各列,各対角線に対して配置されているクイーンの個
数を表す補助アトムを導入し,それぞれの和がサイズ$k$に一致することをASP
の個数制約を用いて表した符号化である.
%
考案した符号化の有効性を評価するために,
$n$次のクイーン支配問題 $(1\leq n\leq 20)$について,最も難しい
サイズ$\gamma(Q_n)$と$\gamma(Q_n)-1$の2種類を用いて実行実験を行なった.
その結果,部分和符号化は,$\gamma(Q_{16})$まで解くことができ,
他の符号化と比較して,その優位性を確認できた.


% 表紙を情報工学コース用のスタイルにするために,作成した{\tt jbachelor.sty}が
% 必要である.
% TexStudioなどの便利なTex統合環境を利用するために,lualtexを使うとよい.
% platex と lualatex を切り替えるためには,このファイルの先頭を編集してlualatex用
% のltjbook.clsを使うようにする.
% \begin{verbatim}
% %%% for platex
% \documentclass[a4paper,12pt]{jbook}
% %%% for lualatex
% %\documentclass[a4paper,12pt]{ltjbook}
% \end{verbatim}

%%% Local Variables:
%%% mode: japanese-latex
%%% TeX-master: "paper"
%%% End: