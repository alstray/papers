%%%%%%%%%%%%%%%%%%%%%%%%%%%%%%%%%%%%%%%%%%%%%%%%%%%%%%%%%% 
\chapter{クイーン支配問題のASP符号化}\label{chap:proposal}
%%%%%%%%%%%%%%%%%%%%%%%%%%%%%%%%%%%%%%%%%%%%%%%%%%%%%%%%%% 

\begin{figure}[h]
  \centering
  \thicklines
  \setlength{\unitlength}{1.2pt}
  \small\footnotesize\scriptsize
  \begin{picture}(280,57)(4,-10)
    \put(  0, 20){\dashbox(50,24){\shortstack{QDP問題\\インスタンス}}}
    \put( 60, 20){\framebox(50,24){変換器}}
    \put(120, 20){\dashbox(50,24){\shortstack{ASPファクト}}}
    \put(120,-10){\dashbox(50,24){\shortstack{ASP符号化\\(論理プログラム)}}}
    \put(180, 20){\framebox(50,24){ASPシステム}}
    \put(240, 20){\dashbox(50,24){\shortstack{QDP問題\\の解}}}
    \put( 50, 32){\vector(1,0){10}}
    \put(110, 32){\vector(1,0){10}}
    \put(170, 32){\vector(1,0){10}}
    \put(230, 32){\vector(1,0){10}}
    \put(170, +2){\line(1,0){4}}
    \put(174, +2){\line(0,1){30}}
  \end{picture}  
\caption{ASP を用いたクイーン支配問題(QDP)の解法}
\label{fig:arch}
\end{figure}
ASP を用いたクイーン支配問題の解法について述べる.
図~\ref{fig:arch}に,解法の流れを示す.
与えられたクイーン支配問題はASPファクトに変換される.
ASPファクトはクイーン支配問題を解くASP符号化と結合され,
ASPシステムによって解が計算される.
本論文では,ASPシステムとして{\clingo}を用いる.

クイーン支配問題は,与えられたクイーングラフ$Q_n$と正の整数$k$に対して,
サイズが$k$である$Q_n$の支配集合が存在するかどうかを判定する問題である.
以下の制約を満たすとき,クイーングラフ$Q_{n}$にサイズが$k$の支配集合が存在する.
\begin{itemize}
 \item クイーングラフ$Q_n$上のどのマスにも,
  1つ以上のクイーンが移動できなければならない.
\end{itemize}
本論文では,この制約を\textbf{被覆制約}と呼ぶ.

本章では,第~\ref{chap:constraint}章で説明した基本制約モデル,改良制約モデル,
部分和制約モデルのASP符号化について,それぞれ2種類ずつ提案する.
以降の符号化において,アトム\code{n/1}は$Q_n$の列番号,もしくは行番号を,
\code{u/1}は右上がり対角線の番号を,\code{d/1}は右下がり対角線の番号を表す.
アトム\code{q/2}はクイーンの有無を表す.

\newpage

\section{基本制約モデルのASP符号化}
基本制約モデルのASP符号化は,被覆制約をASPの一貫性制約で記述した基本的な符号化である.

\lstinputlisting[float=ht,caption={%
基本モデルのASP符号化1},%
captionpos=b,frame=single,label=code:qdp1_asp.lp,%
numbers=none,%
breaklines=true,%
columns=fullflexible,keepspaces=true,%
basicstyle=\ttfamily\scriptsize]{code/qdp1_asp.lp}

基本モデルの符号化1は,
\begin{itemize}
 \item 5--8行目で,マス\code{(A,B)}にクイーンが置かれているならば,
  マス\code{(A,B)}からクイーンが移動できるマスはすべて被覆制約を満たすということ,
 \item 10行目で,各マスに対する被覆制約
\end{itemize}
を表現している.

%\newpage
\lstinputlisting[float=ht,caption={%
基本モデルのASP符号化2},%
captionpos=b,frame=single,label=code:qdp1_sat.lp,%
numbers=none,%
breaklines=true,%
columns=fullflexible,keepspaces=true,%
basicstyle=\ttfamily\scriptsize]{code/qdp1_sat.lp}


基本モデルの符号化2は,5行目で選択子と個数制約を用いて
各マスに対する被覆制約を表現している.
\newpage
\section{改良制約モデルのASP符号化}
\lstinputlisting[float=ht,caption={%
 改良モデルのASP符号化1},%
captionpos=b,frame=single,label=code:qdp2_asp.lp,%
numbers=none,%
breaklines=true,%
columns=fullflexible,keepspaces=true,%
basicstyle=\ttfamily\scriptsize]{code/qdp2_asp.lp}

改良制約モデルのASP符号化1では,基本制約モデルにおいて複数回
出現する部分式をまとめるための補助アトムと制約を追加する.

具体的には,
\begin{itemize}
  \item 5--8行目でマス\code{(X,Y)}にクイーンが置かれていたら
    その行,列,対角線方向はクイーンが移動できるということ
  \item 10行目で一貫性制約を用いて,各マスに対する被覆制約
\end{itemize}
を記述している.

\lstinputlisting[float=ht,caption={%
 改良モデルのASP符号化2},%
captionpos=b,frame=single,label=code:qdp2_sat.lp,%
numbers=none,%
breaklines=true,%
columns=fullflexible,keepspaces=true,%
basicstyle=\ttfamily\scriptsize]{code/qdp2_sat.lp}

\newpage
改良モデルのASP符号化2では,
\begin{itemize}
 \item 7--10行目で各行,各列,各対角線方向に変数\code{attackable}を定義
 \item 12--15行目でその行,列,対角線方向が\code{attackable}ならば
  クイーンが1つ以上配置されていること
 \item 17行目で一貫性制約を用いて,各マスに対する被覆制約
\end{itemize}
を記述している.

\newpage
\section{部分和モデルのASP符号化}

\lstinputlisting[float=ht,caption={%
 部分和モデルのASP符号化1},%
captionpos=b,frame=single,label=code:qdp3_cc.lp,%
numbers=none,%
breaklines=true,%
columns=fullflexible,keepspaces=true,%
basicstyle=\ttfamily\scriptsize]{code/qdp3_cc.lp}

部分和モデルのASP符号化1では,
\begin{itemize}
 \item 7--15行目で各行,各列,各対角線上のクイーンの個数を表す補助整数変数の導入
 \item 18--21行目で各行,列,対角線のクイーンの総数が$k$でなければならない制, 
 \item 24行目で重み付き個数制約を用いて
	$$M+N+P+Q>0$$
  という形で各マスに対する被覆制約
\end{itemize}
を表現している.

\newpage
\lstinputlisting[float=ht,caption={%
 部分和モデルのASP符号化2},%
captionpos=b,frame=single,label=code:qdp3_ic.lp,%
numbers=none,%
breaklines=true,%
columns=fullflexible,keepspaces=true,%
basicstyle=\ttfamily\scriptsize]{code/qdp3_ic.lp}

部分和モデルのASP符号化2は,7行目から21行目までは部分和モデルのASP符号化1と同じである.
24行目から29行目では,一貫性制約を用いて
$$M>0 \vee N>0 \vee P>0 \vee Q>0$$
という形で各マスに対する被覆制約を表現している.

%%% Local Variables:
%%% mode: latex
%%% TeX-master: "paper"
%%% End:
