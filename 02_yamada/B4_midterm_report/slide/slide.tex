\documentclass[dvipdfmx,11pt]{beamer}

%全体設定
%\AtBeginDvi{\special{pdf:tounicode 90ms-RKSJ-UCS2}}

\usepackage{bxdpx-beamer}% dvipdfmxなので必要
\usepackage{pxjahyper}
\usepackage{minijs}
\usepackage{otf}
\usepackage{amssymb,amsmath}
\usepackage{hyperref}
\usepackage[absolute,overlay]{textpos}
\usepackage{comment}
\usepackage{colortbl}
\usepackage{graphicx}
\usepackage{tikz}
\usetikzlibrary{positioning}
\usetikzlibrary{shadows}
\usepackage{listings}
%\usepackage{plistings}
\def\lstlistingname{コード}
\usetheme{Copenhagen}
\setbeamertemplate{navigation symbols}{} %スライドのボタン?(右下のやつ)を消す
%\setbeamertemplate{footline}[short title]
\setbeamersize{text margin left=1.5em,text margin right=1.5em} % 余白なくすやつ

% footer setting %
\makeatother
\setbeamertemplate{footline}
{
  \leavevmode
  \hbox{
  \begin{beamercolorbox}[wd=.4\paperwidth,ht=2.25ex,dp=1ex,center]{author in head/foot}
    \usebeamerfont{author in head/foot}\insertshortauthor
  \end{beamercolorbox}
  \begin{beamercolorbox}[wd=.6\paperwidth,ht=2.25ex,dp=1ex,center]{title in head/foot}
    \usebeamerfont{title in head/foot}\hspace*{1ex} \insertshorttitle\hspace*{3em}
    \textbf{ \insertframenumber{} / \inserttotalframenumber } \hspace*{1ex}
  \end{beamercolorbox}}
  \vskip0pt
}
\makeatletter

% exclude apprendix slides from framenumber %
\newcommand{\backupbegin}{
   \newcounter{framenumberappendix}
   \setcounter{framenumberappendix}{\value{framenumber}}
}
\newcommand{\backupend}{
   \addtocounter{framenumberappendix}{-\value{framenumber}}
   \addtocounter{framenumber}{\value{framenumberappendix}} 
}

\renewcommand{\kanjifamilydefault}{\gtdefault}
%\usetheme{Madrid}

\title[ASPを用いた組合せ遷移問題の解法に関する考察]{解集合プログラミングを用いた\\組合せ遷移問題の解法に関する考察}
\author{山田 悠也}
\date{2020年度番原研中間発表会\\2020年12月11日}
\institute{番原研究室}

%% テンプレ 
\begin{comment}

%%%%%%%%%%%%%%%%%%%%%%%%%%%%%%%%%%%%%%%%%%%%%%%%%%
%% タイトル
%%%%%%%%%%%%%%%%%%%%%%%%%%%%%%%%%%%%%%%%%%%%%%%%%%
\begin{frame}\frametitle{}
\end{frame}

\end{comment}

%###########################################################
%# 本文 ####################################################
%###########################################################
\begin{document}

%%%%%%%%%%%%%%%%%%%%%%%%%%%%%%%%%%%%%%%%%%%%%%%%%%
%% タイトル 
%%%%%%%%%%%%%%%%%%%%%%%%%%%%%%%%%%%%%%%%%%%%%%%%%%
\begin{frame}\frametitle{}
  \titlepage
\end{frame}

%%%%%%%%%%%%%%%%%%%%%%%%%%%%%%%%%%%%%%%%%%%%%%%%%%
%% 組合せ遷移問題
%%%%%%%%%%%%%%%%%%%%%%%%%%%%%%%%%%%%%%%%%%%%%%%%%%
\begin{frame}\frametitle{組合せ遷移問題(Combinatorial Reconfiguration)}

  \begin{itemize}
    \item \alert{組合せ遷移問題}とは, 組合せ判定問題の解空間グラフ上における初期状態から目標状態への経路の存在についての判定問題である.
    \item \alert{解空間グラフ}とは,
    \begin{itemize}
      \item 頂点は, 組合せ判定問題の実行可能解となる.
      \item 辺は, \structure{隣接関係}を満たす2つの頂点間に存在する.
    \end{itemize} 
    \item 既存の組合せ判定問題の多くを, 組合せ遷移問題に拡張できる.
    \begin{itemize}
      \item \structure{グラフ点彩色問題}, 充足可能性判定問題, 独立点集合問題など
    \end{itemize}
    \item \structure{持続可能なシステム}へと応用することが可能.
    \begin{itemize}
      \item リスト点彩色問題は無線ネットワークの周波数割当てに応用可能.
      \item これを遷移問題に拡張することで, 正常な動作を保ちながら別の割当てへと変化させることが可能となる.
    \end{itemize}
    \item \structure{\textbf{PSPACE完全}}である問題も多数存在.
  \end{itemize}

  \begin{alertblock}{}
    \centering
    グラフ点彩色問題を遷移問題へと拡張したものは,\\色数によってPSPACE完全であることが知られている.
  \end{alertblock}

\end{frame}

%%%%%%%%%%%%%%%%%%%%%%%%%%%%%%%%%%%%%%%%%%%%%%%%%%
%% グラフ点彩色問題
%%%%%%%%%%%%%%%%%%%%%%%%%%%%%%%%%%%%%%%%%%%%%%%%%%
\begin{frame}\frametitle{グラフ点彩色問題}
    
  \begin{block}{グラフ点彩色問題の定義}
    与えられたグラフ$G=(V, E)$と色数$k$に対して, 以下の制約を満たす解が存在するかを判定する問題.
    \begin{itemize}
      \item 各頂点は1つの色で塗られる.
      \item $(u, v) \in E$である$u, v \in V$について, $u$と$v$は異なる色で塗られる.
    \end{itemize}
  \end{block}
  
  \begin{exampleblock}{グラフ点彩色問題の例($k=4$)}
    \begin{columns}
      \begin{column}{1.0\textwidth}
        \centering
        \begin{tikzpicture}
 \draw (0,0)--(8,0);
 \draw (0,1)--(8,1);
 \draw (0,2)--(8,2);
 \draw (0,3)--(8,3);
 \draw (0,4)--(8,4);
 \draw (0,5)--(8,5);
 \draw (0,6)--(8,6);
 \draw (0,7)--(8,7);
 \draw (0,8)--(8,8);
 \draw (0,0)--(0,8);
 \draw (1,0)--(1,8);
 \draw (2,0)--(2,8);
 \draw (3,0)--(3,8);
 \draw (4,0)--(4,8);
 \draw (5,0)--(5,8);
 \draw (6,0)--(6,8);
 \draw (7,0)--(7,8);
 \draw (8,0)--(8,8);
 \draw (0,0)--(0,8);
 %\fill[red] (4.5,7.5) circle (0.3);
 \draw[red] (4.5,0.5)--(4.5,7.5);
 \draw[red] (0.5,7.5)--(7.5,7.5);
 \draw[red] (0.5,3.5)--(4.5,7.5);
 \draw[red] (7.5,4.5)--(4.5,7.5);
 %\fill[cyan] (6.5,6.5) circle (0.3);
 \draw[cyan] (0.5,6.5)--(7.5,6.5);
 \draw[cyan] (6.5,0.5)--(6.5,7.5);
 \draw[cyan] (0.5,0.5)--(7.5,7.5);
 \draw[cyan] (5.5,7.5)--(7.5,5.5);
 %\fill[violet] (0.5,3.5) circle (0.3);
 \draw[violet] (0.5,3.5)--(7.5,3.5);
 \draw[violet] (0.5,0.5)--(0.5,7.5);
 \draw[violet] (0.5,3.5)--(4.5,7.5);
 \draw[violet] (0.5,3.5)--(3.5,0.5);
 %\fill[teal] (3.5,2.5) circle (0.3);
 \draw[teal] (0.5,2.5)--(7.5,2.5);
 \draw[teal] (3.5,0.5)--(3.5,7.5);
 \draw[teal] (1.5,0.5)--(7.5,6.5);
 \draw[teal] (0.5,5.5)--(5.5,0.5);
 %\fill[orange] (6.5,0.5) circle (0.3);
 \draw[orange] (0.5,0.5)--(7.5,0.5);
 \draw[orange] (6.5,0.5)--(6.5,7.5);
 \draw[orange] (6.5,0.5)--(0.5,6.5);
 \draw[orange] (6.5,0.5)--(7.5,1.5);
 \fill[red] (4.5,7.5) \symqueen ;
 \fill[cyan] (6.5,6.5) circle (0.35);
 \fill[violet] (0.5,3.5) circle (0.35);
 \fill[teal] (3.5,2.5) circle (0.35);
 \fill[orange] (6.5,0.5) circle (0.35);
\end{tikzpicture}
      \end{column}
    \end{columns}
  \end{exampleblock}
  

\end{frame}

%%%%%%%%%%%%%%%%%%%%%%%%%%%%%%%%%%%%%%%%%%%%%%%%%%
%% 点彩色遷移問題
%%%%%%%%%%%%%%%%%%%%%%%%%%%%%%%%%%%%%%%%%%%%%%%%%%

\begin{frame}\frametitle{点彩色遷移問題}
  \begin{itemize}
    \item \alert{点彩色遷移問題}とは, グラフ点彩色問題を遷移問題に拡張したもの.
  \end{itemize}

  \begin{block}{点彩色遷移問題の定義}
    \begin{itemize}
      \item 問題の入力は, グラフ\structure{$G=(V, E)$}, 色数\structure{$k$}, 及び$G$の$k$-彩色の2つの実行可能解\structure{$\alpha$}(初期状態)と\structure{$\beta$}(目標状態).
      \item 隣接関係は, ある1つの頂点のみ色が異なるような実行可能解.
      \item \structure{$\alpha$から$\beta$への経路の存在についての判定問題.}
    \end{itemize}
  \end{block}

  \begin{exampleblock}{点彩色遷移問題の例}
    \begin{columns}
      \begin{column}{0.3\textwidth}
        \centering
        \begin{tikzpicture}
 \draw (0,0)--(8,0);
 \draw (0,1)--(8,1);
 \draw (0,2)--(8,2);
 \draw (0,3)--(8,3);
 \draw (0,4)--(8,4);
 \draw (0,5)--(8,5);
 \draw (0,6)--(8,6);
 \draw (0,7)--(8,7);
 \draw (0,8)--(8,8);
 \draw (0,0)--(0,8);
 \draw (1,0)--(1,8);
 \draw (2,0)--(2,8);
 \draw (3,0)--(3,8);
 \draw (4,0)--(4,8);
 \draw (5,0)--(5,8);
 \draw (6,0)--(6,8);
 \draw (7,0)--(7,8);
 \draw (8,0)--(8,8);
 \draw (0,0)--(0,8);
 %\fill[red] (4.5,7.5) circle (0.3);
 \draw[red] (4.5,0.5)--(4.5,7.5);
 \draw[red] (0.5,7.5)--(7.5,7.5);
 \draw[red] (0.5,3.5)--(4.5,7.5);
 \draw[red] (7.5,4.5)--(4.5,7.5);
 %\fill[cyan] (6.5,6.5) circle (0.3);
 \draw[cyan] (0.5,6.5)--(7.5,6.5);
 \draw[cyan] (6.5,0.5)--(6.5,7.5);
 \draw[cyan] (0.5,0.5)--(7.5,7.5);
 \draw[cyan] (5.5,7.5)--(7.5,5.5);
 %\fill[violet] (0.5,3.5) circle (0.3);
 \draw[violet] (0.5,3.5)--(7.5,3.5);
 \draw[violet] (0.5,0.5)--(0.5,7.5);
 \draw[violet] (0.5,3.5)--(4.5,7.5);
 \draw[violet] (0.5,3.5)--(3.5,0.5);
 %\fill[teal] (3.5,2.5) circle (0.3);
 \draw[teal] (0.5,2.5)--(7.5,2.5);
 \draw[teal] (3.5,0.5)--(3.5,7.5);
 \draw[teal] (1.5,0.5)--(7.5,6.5);
 \draw[teal] (0.5,5.5)--(5.5,0.5);
 %\fill[orange] (6.5,0.5) circle (0.3);
 \draw[orange] (0.5,0.5)--(7.5,0.5);
 \draw[orange] (6.5,0.5)--(6.5,7.5);
 \draw[orange] (6.5,0.5)--(0.5,6.5);
 \draw[orange] (6.5,0.5)--(7.5,1.5);
 \fill[red] (4.5,7.5) \symqueen ;
 \fill[cyan] (6.5,6.5) circle (0.35);
 \fill[violet] (0.5,3.5) circle (0.35);
 \fill[teal] (3.5,2.5) circle (0.35);
 \fill[orange] (6.5,0.5) circle (0.35);
\end{tikzpicture}
        $\alpha$(ステップ0)
      \end{column}
      \begin{column}{0.05\textwidth}
        \textbf{$\longrightarrow$}
      \end{column}
      \begin{column}[]{0.3\textwidth}
        \centering
        %%%%%%%%%%%%%%%%%%%%%%%%%%%%%%%%%%%%%%%%%%%%%%%%%%
% 実行例(t=0) (第6章で使う)
%%%%%%%%%%%%%%%%%%%%%%%%%%%%%%%%%%%%%%%%%%%%%%%%%%

\begin{tikzpicture}[scale=0.6]

  % 設定
  \tikzset{node/.style={circle,draw=black}}
 
  \definecolor{col_r}{RGB}{230,0,18}
  %\definecolor{col_b}{RGB}{0,104,183}
  \definecolor{col_b}{RGB}{51,51,179}
  \definecolor{col_y}{RGB}{255,251,0}
  \definecolor{col_g}{RGB}{0,96,0}
 
  % 補助線
  % \draw [help lines,blue] (0,0) grid (20,6);
 
  % node %
  \node[node, fill=col_y!70] (node1){\textbf{1}};
  \node[node, fill=col_b!70, right=of node1] (node2){\textbf{2}};
  \node[node, fill=col_y!70, below=of node1] (node3){\textbf{3}};
  \node[node, fill=col_g!70, below=of node2] (node4){\textbf{4}};
 
  \foreach \u / \v in {node1/node2, node2/node3, node2/node4, node3/node4}
  \draw (\u) -- (\v);
 \end{tikzpicture}
 
 %%%%%%%%%%%%%%%%%%%%%%%%%%%%%%%%%%%%%%%%%%%%%%%%%%%%%%%%%%
 %%% Local Variables:
 %%% mode: japanese-latex
 %%% TeX-master: paper.tex
 %%% End:
 
        (ステップ1)
      \end{column}
      \begin{column}{0.05\textwidth}
        \textbf{$\longrightarrow$}
      \end{column}
      \begin{column}{0.3\textwidth}
        \centering
        \begin{tikzpicture}
 \draw (0,0)--(2.5,0);
 \draw (0,0.5)--(2.5,0.5);
 \draw (0,1.0)--(2.5,1.0);
 \draw (0,1.5)--(2.5,1.5);
 \draw (0,2.0)--(2.5,2.0);
 \draw (0,2.5)--(2.5,2.5);
 \draw (0,0)--(0,2.5);
 \draw (0.5,0)--(0.5,2.5);
 \draw (1.0,0)--(1.0,2.5);
 \draw (1.5,0)--(1.5,2.5);
 \draw (2.0,0)--(2.0,2.5);
 \draw (2.5,0)--(2.5,2.5);
 \node (X) at (0.25,0.25) {X};
 \draw [orange] (0.25,0.25) circle[radius = 0.24];
 \node (X) at (0.25,0.75) {X};
 \node (X) at (0.25,1.25) {X};
 \node (X) at (0.25,1.75) {X};
 \draw [blue] (0.25,1.75) circle[radius = 0.24];
 \node (X) at (0.25,2.25) {X};
 \draw [red] (0.25,2.25) circle[radius = 0.24];
 \node (X) at (0.75,2.25) {X};
 \node (X) at (0.75,0.75) {X};
 \node (X) at (0.75,1.25) {X};
 \node (X) at (0.75,1.75) {X};
 \node (X) at (1.25,0.75) {X};
 \node (X) at (1.25,1.25) {X};
 \node (X) at (1.75,0.25) {X};
 \node (X) at (1.75,0.75) {X};
 \node (X) at (1.75,1.25) {\color{purple}X};
 \node (X) at (1.75,1.75) {X};
 \node (X) at (2.25,0.25) {X};
 \node (X) at (2.25,0.75) {\color{purple}X};
 \node (X) at (2.25,1.25) {\color{purple}X};
 \node (X) at (2.25,2.25) {X};
 \node (A) at (3.30,2.25) {\color{red}$r_{1} = 1$};
 \node (B) at (3.30,1.75) {\color{blue}$r_{2} = 1$};
 \node (M) at (3.30,1.25) {\color{purple}$r_{3} = 0$};
 \node (M) at (3.30,0.75) {\color{purple}$r_{4} = 0$};
 \node (C) at (3.30,0.25) {\color{orange}$r_{5} = 1$};
 \node (D) at (3.50,-0.50) {$\sum\limits_{i=1}^{5}r_{i}=3$};
\end{tikzpicture}
        $\beta$(ステップ2)
      \end{column}
    \end{columns}
  \end{exampleblock}
  
\end{frame}

%%%%%%%%%%%%%%%%%%%%%%%%%%%%%%%%%%%%%%%%%%%%%%%%%%
%% 点彩色遷移問題
%%%%%%%%%%%%%%%%%%%%%%%%%%%%%%%%%%%%%%%%%%%%%%%%%%

\begin{frame}\frametitle{点彩色遷移問題の性質}

  \begin{itemize}
    \item 色数$k$によって問題の性質が異なることが知られている.
    \begin{itemize}
      \item \structure{$k=2$}のとき, グラフGは2部グラフであり\structure{自明}.
      \item \structure{$k=3$}のとき, \structure{クラスP}に属する.~[L. Cerecedaほか '08]
      \item \structure{$k \ge 4$}のとき, 一般に\structure{\textbf{PSPACE完全}}となる.~[Paul Bonsmaほか '09]
    \end{itemize}

    \item グラフの形に制限を加えることにより, 多項式時間で解決可能となるものが存在することがわかっている.[Paul Bonsmaほか '09]
    \begin{itemize}
      \item 平面グラフであり, かつ$4 \le k \le 6$でないとき.
      \item 2部平面グラフであり, かつ$k=4$でないとき.
    \end{itemize}

    \item 一方で, \alert{$k \ge 4$における汎用的かつ効率的なアルゴリズムは見つかっていない}.

  \end{itemize}

\end{frame}

%%%%%%%%%%%%%%%%%%%%%%%%%%%%%%%%%%%%%%%%%%%%%%%%%%
%% ASP
%%%%%%%%%%%%%%%%%%%%%%%%%%%%%%%%%%%%%%%%%%%%%%%%%%

\begin{frame}\frametitle{解集合プログラミング(Answer Set Programming; ASP)}

  \begin{itemize}
    \item ASP言語は一階論理に基づいた知識表現言語の一種である.
    \item ASPシステムは安定モデル理論~[Gelfond and Lifschitz '88] に基づいた知識表現言語の一種である.
    \item 近年, SAT技術の応用により高速なASPシステムが実現し, システム検証やプランニングなどの様々な分野での実用が拡大している.
  \end{itemize}

  \begin{alertblock}{組合せ遷移問題に対してASPを用いる利点}
    \begin{itemize}
      \item ASPの高い表現力により, 記号上の制約を簡潔に記述できる.
      \item ASPシステムの1つである\textit{clingo}のインクリメンタルモードを用いることにより, 段階的な解の探索を効率よく実行できる.
    \end{itemize}
  \end{alertblock}
  
\end{frame}

%%%%%%%%%%%%%%%%%%%%%%%%%%%%%%%%%%%%%%%%%%%%%%%%%%
%% 研究目的
%%%%%%%%%%%%%%%%%%%%%%%%%%%%%%%%%%%%%%%%%%%%%%%%%%

\begin{frame}\frametitle{研究目的}
  \begin{alertblock}{目的}
    ASPを利用し, PSPACE完全な問題の1つである$k \ge 4$の点彩色遷移問題を高速で解ける符号化を提案する.
  \end{alertblock}

  \begin{block}{研究内容}
    \begin{enumerate}
      \item 点彩色遷移問題を解く, 3種類の符号化の提案.
      \item 点彩色遷移問題のベンチマークの生成に使用可能なグラフの調査.
      \item (ベンチマークの生成.)
      \item (作成したベンチマークを用いた, 各種符号化の評価実験.)
    \end{enumerate}
  \end{block}

\end{frame}

%%%%%%%%%%%%%%%%%%%%%%%%%%%%%%%%%%%%%%%%%%%%%%%%%%
%% 符号化
%%%%%%%%%%%%%%%%%%%%%%%%%%%%%%%%%%%%%%%%%%%%%%%%%%

\begin{frame}\frametitle{符号化(1/2)}

  \begin{itemize}
    \item グラフ$G$, 色数$k$, Gのk-彩色の2つの実行可能解$\alpha$, $\beta$に加え, \structure{ステップ数$t$}を与える.
    \begin{itemize}
      \item $t$回の遷移で$\alpha$から$\beta$へ遷移可能かの判定問題へと変更.
      \item $t$を遷移回数の上限値まで大きくしていくことで, 最終的に経路の存在について判定可能. 
    \end{itemize}
    \item これに対し, \structure{隣接関係の制約に対する記述が異なる\textbf{3}種類}のASP符号化を提案した. 
    \begin{itemize}
      \item グラフ点彩色に関わる制約は同じものを用いている.
      \item 各符号化で独自に追加するアトム\footnote{ASP上の正リテラルを指す.}以外は, 同じものを用いている.
    \end{itemize}
  \end{itemize}


  
\end{frame}

%%%%%%%%%%%%%%%%%%%%%%%%%%%%%%%%%%%%%%%%%%%%%%%%%%
%% 符号化
%%%%%%%%%%%%%%%%%%%%%%%%%%%%%%%%%%%%%%%%%%%%%%%%%%

\begin{frame}\frametitle{符号化(2/2)}

  \begin{itemize}
    \item $|T|$をステップ数$t$, $|C|$を色数$k$, $|V|$をグラフの頂点数とする.
    
  \end{itemize}

  \begin{table}[t]
    \centering
    \begin{tabular}{l|p{8cm}}
  符号化名 & 遷移制約 \\\hline
  origin & 任意の二つの頂点に対し違反する組合せを列挙することで表現  \\ \hline
  changed & 遷移制約を ASP の個数制約を用いて表現\\\hline
  unchanged & 「各遷移で色が変化しない頂点は$|V|-1$個」という制約を,
         ASPの個数制約を用いて表現
\end{tabular}
  \end{table}

  \begin{itemize}
    \item \structure{changed(X, T)}は, ステップ$T-1$とステップ$T$において頂点Xの色が異なることを意味する.
    \item \structure{unchanged(X, T)}は, ステップ$T-1$とステップ$T$において頂点Xの色が同じであることを意味する.
    \item アトムを追加することにより, ルール数を削減することができた.
  \end{itemize}
  
\end{frame}

%%%%%%%%%%%%%%%%%%%%%%%%%%%%%%%%%%%%%%%%%%%%%%%%%%
%% ベンチマーク
%%%%%%%%%%%%%%%%%%%%%%%%%%%%%%%%%%%%%%%%%%%%%%%%%%

\begin{frame}\frametitle{ベンチマーク}

  \begin{itemize}
    \item 現時点で組合せ遷移問題は理論面の研究が主流であり, ベンチマークの整備が必要.
    \item 実験においてステップ$t$を与えるとき, その上限値が必要となる.
    \item ステップ$t$の上限値は, グラフ$G$を$k$彩色するときの実行可能解の数と等しい.
  \end{itemize}

  従って, 全解列挙が可能な($G, k$)からベンチマークを生成する必要がある.
  
\end{frame}

%%%%%%%%%%%%%%%%%%%%%%%%%%%%%%%%%%%%%%%%%%%%%%%%%%
%% 実験環境
%%%%%%%%%%%%%%%%%%%%%%%%%%%%%%%%%%%%%%%%%%%%%%%%%%

\begin{frame}\frametitle{グラフの調査}
  全解列挙可能なグラフと色数の組合せを得るため, 以下の調査を行った.
  \begin{itemize}
    \item \structure{使用するグラフ}: 全44個
    \begin{itemize}
      \item \textit{Graph Coloring and its Generalizations}
      \footnote{https://mat.tepper.cmu.edu/COLOR04/}で公開されている, \textit{Graph Coloring Instances}に属するグラフを使用.
      \item そのうち\structure{彩色数}が判明しているもの.[Tamuraほか '09]
    \end{itemize}
    \item \structure{色数}: 各グラフの彩色数

    \item \structure{ASPシステム}: \textit{clingo-5.4.0}
      \begin{itemize}
        \item \textit{configuration}は\textit{crafty, tweety}を使用.
      \end{itemize}
    \item \structure{制限時間}: 3600秒/問
    \item \structure{環境}: Mac mini, 3.2GHz 6コア Intel Core i7, 64GB メモリ
  \end{itemize}


  \begin{alertblock}{}
    結果, 44個中\structure{\textbf{9}}個のグラフにおいて彩色数での全解列挙が可能であった.
    \begin{itemize}
      \item 最小で\structure{120個}, 最大で\structure{約28億個}の解を持つことが判明した.
    \end{itemize}
  \end{alertblock}
  
  
\end{frame}

%%%%%%%%%%%%%%%%%%%%%%%%%%%%%%%%%%%%%%%%%%%%%%%%%%
%% ベンチマークの生成
%%%%%%%%%%%%%%%%%%%%%%%%%%%%%%%%%%%%%%%%%%%%%%%%%%

\begin{frame}\frametitle{今後の方針}

  \begin{enumerate}
    \item ベンチマークの生成 
    \begin{itemize}
      \item 実行可能解の総数が小さいグラフについては, \structure{全ての解}を出力.
      \item 実行可能解の総数が100万個を超えるグラフについては, \structure{10万個の解のみ}を出力.
      \item 出力された解からランダムで2つの解を抽出しベンチマークを生成.
      \item 各グラフから5問ずつ, \structure{計45問}のベンチマークを生成.
    \end{itemize}

    \item 符号化の評価実験
    \begin{itemize}
      \item 生成したベンチマークを用い, 3種の符号化の評価実験を行う.
    \end{itemize}
  \end{enumerate}
  

\end{frame}

%%%%%%%%%%%%%%%%%%%%%%%%%%%%%%%%%%%%%%%%%%%%%%%%%%
%% 今後の課題
%%%%%%%%%%%%%%%%%%%%%%%%%%%%%%%%%%%%%%%%%%%%%%%%%%

\begin{frame}\frametitle{まとめと今後の課題}

  \begin{block}{まとめ}
    \begin{itemize}
      \item 点彩色遷移問題の隣接関係に対して, 3種類の符号化を提案した.
      \begin{itemize}
        \item アトムを追加することで, ルール数を大幅に削減することができた.
      \end{itemize}
      \item ベンチマークの生成に使用可能なグラフについて調査した.
      \begin{itemize}
        \item 解空間グラフの大きさ(実行可能解の総数)が異なる9つのグラフを使用可能であることが判明した.
      \end{itemize}
    \end{itemize}
  \end{block}
  
  \begin{alertblock}{今後の課題}
    \begin{itemize}
      \item ベンチマークの生成.
      \item 生成したベンチマークを用いた, 各種符号化の評価実験.
      \item 評価実験で優れていた符号化を, インクリメンタルな解法へと拡張.
    \end{itemize}
  \end{alertblock}

\end{frame}

%###########################################################
%##### 補助スライド ########################################
%###########################################################

%%%% 補助スライド

\begin{frame}{~}
 \centering
 - 補足用 -
\end{frame} 

\begin{frame}{補足 : スマートグリッド}
 \begin{itemize}
  \item \structure{スマートグリッド}とは,電力の供給側,需要側において双方向の
		やり取りを可能にする次世代の\structure{賢い}電力網である.
  \item 従来と違い,通信技術の発達により,使用状況などを
		リアルタイムに把握することが可能となった.
  \item その時に応じた最適な配電網を構成し,制御するといったことが考えられている.
		\begin{itemize}
		 \item 電力需要の変化による,配電ロスの少ない構成.
		 \item 自然エネルギーによる発電量の変動を補う構成.
		\end{itemize}
  \item ASP言語の表現力や拡張性が,こうした条件の追加に活用できる可能性がある.
 \end{itemize}
\end{frame}

%%%%%%%%%%%%%%%%%%%%%%%%%%%%%%%%%%%%%%%%%%%%%%%%%%
%% 電気制約
%%%%%%%%%%%%%%%%%%%%%%%%%%%%%%%%%%%%%%%%%%%%%%%%%%
\begin{frame}{補足 : 電気制約}
 \begin{itemize}
  \item \alert{電気制約}は,送電する電流$\cdot$電圧の適正範囲を保証する制約.
  \begin{itemize}
   \item 供給経路の各区間で許容電流を超えない.
   \item 電気抵抗による電圧降下が許容範囲を超えない.
   \item etc.
  \end{itemize}
  \item 電流と電圧が影響し合う\structure{実数ドメイン上の制約}によって表される.
		% \begin{itemize}
		%  		 \item 送電システム上の条件など.
		% \end{itemize}
  \item 実数ドメイン上の制約は,純粋なASPのみで扱うのは\alert{困難}.
		\begin{itemize}
		 \item 緩和問題として,変電所から供給できる家庭の数に上限をつける.
		 \item ASPMT技術により,ASPで得られた解について,
			   背景理論ソルバーと連携して実数ドメイン上の制約を調べる.
		\end{itemize}
 \end{itemize}
\end{frame}


%%%%%%%%%%%%%%%%%%%%%%%%%%%%%%%%%%%%%%%%%%%%%%%%%%
%% 基礎化
%%%%%%%%%%%%%%%%%%%%%%%%%%%%%%%%%%%%%%%%%%%%%%%%%%
\begin{frame}{補足 : ASPシステム}
 
 \vspace{-0.5cm}

 \begin{figure}[htbp]
  \centering
  %%%%%%%%%%%%%%%%%%%%%%%%%%%%%%%%%%%%%%%%%%%%%%%%%%
%% 基礎化の流れの図
%%%%%%%%%%%%%%%%%%%%%%%%%%%%%%%%%%%%%%%%%%%%%%%%%%
\begin{tikzpicture}

 \definecolor{edge}{RGB}{38,38,134}
 \definecolor{node}{RGB}{220,220,249}

 \definecolor{alert_edge}{RGB}{191,0,0}
 \definecolor{alert_node}{RGB}{249,200,200}

 \definecolor{ex_edge}{RGB}{0,96,0}
 \definecolor{ex_node}{RGB}{230,239,230}

 \def\nodespace{2.4cm}

 \tikzset{block/.style={rectangle, thick, draw=edge, fill=node, text width=3cm, 
 text centered, rounded corners, text width=2cm, minimum height=1.5cm}};

 \tikzset{alertblock/.style={rectangle, thick, draw=alert_edge, fill=alert_node, 
 text width=3cm, text centered, rounded corners, text width=1.5cm, minimum height=1.2cm}};

 \node[block](ikkai){一階ASP\\プログラム};

 \node[rectangle,rounded corners, thick, draw=ex_edge, fill=ex_node, 
 right=0.22*\nodespace of ikkai, minimum width=6cm, minimum height=3cm, 
 text centered, label=ASPシステム](sys){};

 \node[block, right=\nodespace of ikkai](meidai){命題ASP\\プログラム};
 \node[block, right=\nodespace of meidai](ASP){解集合};

 \node[right=0.6*\nodespace of ikkai, text width=1.5cm, 
 text centered, text=red, anchor=south](){基礎化\\ソルバー};
 \node[right=0.4*\nodespace of meidai, text width=1.5cm, 
 text centered, text=red, anchor=south](){解集合\\ソルバー};

 
 \foreach \u / \v / \n in {ikkai/meidai,meidai/ASP}
 \draw [thick,->] (\u) to (\v);

\end{tikzpicture}
 \end{figure}

 \vspace{-0.5cm}

 \begin{exampleblock}{}
  \begin{enumerate}
   \item 一階ASPプログラムを基礎化ソルバーによって,
		 命題ASPプログラムに\alert{基礎化}する.
   \item 命題ASPプログラムについて,SAT技術を応用した解集合ソルバーが解集合を探索する.
  \end{enumerate}
 \end{exampleblock}

\end{frame}


%%%%%%%%%%%%%%%%%%%%%%%%%%%%%%%%%%%%%%%%%%%%%%%%%%
%% ASPのコード
%%%%%%%%%%%%%%%%%%%%%%%%%%%%%%%%%%%%%%%%%%%%%%%%%%
\begin{frame}[fragile]{補足 : 基本符号化のASPプログラム}
 \begin{exampleblock}{}
  \begin{center}
   %%%%%%%%%%%%%%%%%%%%%%%%%%%%%%%%%
   \lstinputlisting[numbers=left,%
   basicstyle=\ttfamily\tiny]{code/srf1.lp}
   %%%%%%%%%%%%%%%%%%%%%%%%%%%%%%%%% 
  \end{center}
 \end{exampleblock}
\end{frame}

\begin{frame}[fragile]{補足 : 改良符号化のASPプログラム}

 \begin{exampleblock}{}
  \begin{center}
   %%%%%%%%%%%%%%%%%%%%%%%%%%%%%%%%%
   \lstinputlisting[numbers=left,%
   basicstyle=\ttfamily\tiny]{code/srf2.lp}
   %%%%%%%%%%%%%%%%%%%%%%%%%%%%%%%%% 
  \end{center}
 \end{exampleblock}

\end{frame}



\end{document}