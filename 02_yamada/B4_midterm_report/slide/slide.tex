\documentclass[dvipdfmx,11pt]{beamer}

%全体設定
%\AtBeginDvi{\special{pdf:tounicode 90ms-RKSJ-UCS2}}

\usepackage{bxdpx-beamer}% dvipdfmxなので必要
\usepackage{pxjahyper}
\usepackage{minijs}
\usepackage{otf}
\usepackage{amssymb,amsmath}
\usepackage{hyperref}
\usepackage[absolute,overlay]{textpos}
\usepackage{comment}
\usepackage{colortbl}
\usepackage{graphicx}
\usepackage{tikz}
\usetikzlibrary{positioning}
\usetikzlibrary{shadows}
\usepackage{listings}
\usepackage{plistings}
\def\lstlistingname{コード}

\renewcommand{\kanjifamilydefault}{\gtdefault}
%\usetheme{Madrid}
\usetheme{Copenhagen}
\setbeamertemplate{navigation symbols}{} %スライドのボタン?(右下のやつ)を消す
%\setbeamertemplate{footline}[short title]
\setbeamersize{text margin left=1.5em,text margin right=1.5em} % 余白なくすやつ

\title[ASPを用いた組合せ遷移問題の解法に関する考察]{解集合プログラミングを用いた\\組合せ遷移問題の解法に関する考察}
\author{山田 悠也}
\date{中間発表会\\2020年12月11日}
\institute{番原研究室}

%% テンプレ 
\begin{comment}

%%%%%%%%%%%%%%%%%%%%%%%%%%%%%%%%%%%%%%%%%%%%%%%%%%
%% タイトル
%%%%%%%%%%%%%%%%%%%%%%%%%%%%%%%%%%%%%%%%%%%%%%%%%%
\begin{frame}\frametitle{}
\end{frame}

\end{comment}

%###########################################################
%# 本文 ####################################################
%###########################################################
\begin{document}

%%%%%%%%%%%%%%%%%%%%%%%%%%%%%%%%%%%%%%%%%%%%%%%%%%
%% タイトル 
%%%%%%%%%%%%%%%%%%%%%%%%%%%%%%%%%%%%%%%%%%%%%%%%%%
\begin{frame}\frametitle{}
  \titlepage
\end{frame}

%%%%%%%%%%%%%%%%%%%%%%%%%%%%%%%%%%%%%%%%%%%%%%%%%%
%% 組合せ遷移問題
%%%%%%%%%%%%%%%%%%%%%%%%%%%%%%%%%%%%%%%%%%%%%%%%%%
\begin{frame}\frametitle{組合せ遷移問題(Combinatorial Reconfiguration)}

  \begin{itemize}
    \item \alert{組合せ遷移問題}とは, 決定問題の解空間グラフ上における初期状態から目標状態への経路の存在についての判定問題である.
    \item \alert{解空間グラフ}とは,
    \begin{itemize}
      \item 頂点は, 決定問題の実行可能解となる.
      \item 辺は, \structure{隣接関係}を満たす2つの頂点間に存在する.
    \end{itemize} 
    \item 既存の決定問題の多くを, 組合せ遷移問題に拡張できる.
    \begin{itemize}
      \item \structure{グラフ点彩色問題}, 充足可能性判定問題, 独立点集合問題など
    \end{itemize}
    \item \structure{\textbf{PSPACE完全}}である問題も多数存在.
  \end{itemize}

\end{frame}

%%%%%%%%%%%%%%%%%%%%%%%%%%%%%%%%%%%%%%%%%%%%%%%%%%
%% グラフ点彩色問題
%%%%%%%%%%%%%%%%%%%%%%%%%%%%%%%%%%%%%%%%%%%%%%%%%%
\begin{frame}\frametitle{グラフ点彩色問題}
    
  \begin{block}{グラフ点彩色問題の定義}
    与えられたグラフ$G=(V, E)$と色数$k$に対して, 以下の制約を満たす解が存在するかを判定する問題.
    \begin{itemize}
      \item 各頂点は1つの色で塗られる.
      \item $(u, v) \in E$である$u, v \in V$について, $u$と$v$は異なる色で塗られる.
    \end{itemize}
  \end{block}
  
  \begin{exampleblock}{点彩色遷移問題の例($k=4$)}
    \begin{columns}
      \begin{column}{1.0\textwidth}
        \centering
        %%%%%%%%%%%%%%%%%%%%%%%%%%%%%%%%%%%%%%%%%%%%%%%%%%
% 実行例(t=0) (第6章で使う)
%%%%%%%%%%%%%%%%%%%%%%%%%%%%%%%%%%%%%%%%%%%%%%%%%%

\begin{tikzpicture}[scale=0.6]

  % 設定
  \tikzset{node/.style={circle,draw=black}}
 
  \definecolor{col_r}{RGB}{230,0,18}
  %\definecolor{col_b}{RGB}{0,104,183}
  \definecolor{col_b}{RGB}{51,51,179}
  \definecolor{col_y}{RGB}{255,251,0}
  \definecolor{col_g}{RGB}{0,96,0}
 
  % 補助線
  % \draw [help lines,blue] (0,0) grid (20,6);
 
  % node %
  \node[node, fill=col_r!70] (node1){\textbf{1}};
  \node[node, fill=col_b!70, right=of node1] (node2){\textbf{2}};
  \node[node, fill=col_y!70, below=of node1] (node3){\textbf{3}};
  \node[node, fill=col_g!70, below=of node2] (node4){\textbf{4}};
 
  \foreach \u / \v in {node1/node2, node2/node3, node2/node4, node3/node4}
  \draw (\u) -- (\v);
 \end{tikzpicture}
 
 %%%%%%%%%%%%%%%%%%%%%%%%%%%%%%%%%%%%%%%%%%%%%%%%%%%%%%%%%%
 %%% Local Variables:
 %%% mode: japanese-latex
 %%% TeX-master: paper.tex
 %%% End:
 
      \end{column}
    \end{columns}
  \end{exampleblock}
  

\end{frame}

%%%%%%%%%%%%%%%%%%%%%%%%%%%%%%%%%%%%%%%%%%%%%%%%%%
%% 点彩色遷移問題
%%%%%%%%%%%%%%%%%%%%%%%%%%%%%%%%%%%%%%%%%%%%%%%%%%

\begin{frame}\frametitle{点彩色遷移問題(1/2)}
  \begin{itemize}
    \item \alert{点彩色遷移問題}とは, グラフ点彩色問題を遷移問題に拡張したものである.
  \end{itemize}

  \begin{block}{点彩色問題の定義}
    \begin{itemize}
      \item 問題の入力は, グラフ$G=(V, E)$, 色数$k$, 及びこれらのグラフ点彩色問題の2つの実行可能解$\alpha$(初期状態)と$\beta$(目標状態).
      \item 隣接関係は, ある1つの頂点のみ色が異なるような実行可能解.
      \item $\alpha$から$\beta$への経路の存在についての判定問題.
    \end{itemize}
  \end{block}

  \begin{exampleblock}{点彩色遷移問題の例}
    \begin{columns}
      \begin{column}{0.45\textwidth}
        \centering
        %%%%%%%%%%%%%%%%%%%%%%%%%%%%%%%%%%%%%%%%%%%%%%%%%%
% 実行例(t=0) (第6章で使う)
%%%%%%%%%%%%%%%%%%%%%%%%%%%%%%%%%%%%%%%%%%%%%%%%%%

\begin{tikzpicture}[scale=0.6]

  % 設定
  \tikzset{node/.style={circle,draw=black}}
 
  \definecolor{col_r}{RGB}{230,0,18}
  %\definecolor{col_b}{RGB}{0,104,183}
  \definecolor{col_b}{RGB}{51,51,179}
  \definecolor{col_y}{RGB}{255,251,0}
  \definecolor{col_g}{RGB}{0,96,0}
 
  % 補助線
  % \draw [help lines,blue] (0,0) grid (20,6);
 
  % node %
  \node[node, fill=col_r!70] (node1){\textbf{1}};
  \node[node, fill=col_b!70, right=of node1] (node2){\textbf{2}};
  \node[node, fill=col_y!70, below=of node1] (node3){\textbf{3}};
  \node[node, fill=col_g!70, below=of node2] (node4){\textbf{4}};
 
  \foreach \u / \v in {node1/node2, node2/node3, node2/node4, node3/node4}
  \draw (\u) -- (\v);
 \end{tikzpicture}
 
 %%%%%%%%%%%%%%%%%%%%%%%%%%%%%%%%%%%%%%%%%%%%%%%%%%%%%%%%%%
 %%% Local Variables:
 %%% mode: japanese-latex
 %%% TeX-master: paper.tex
 %%% End:
 
      \end{column}
      \begin{column}{0.45\textwidth}
        \centering
        \input{tikz/graph3.tex}
      \end{column}
    \end{columns}
  \end{exampleblock}
  
\end{frame}

%%%%%%%%%%%%%%%%%%%%%%%%%%%%%%%%%%%%%%%%%%%%%%%%%%
%% 点彩色遷移問題
%%%%%%%%%%%%%%%%%%%%%%%%%%%%%%%%%%%%%%%%%%%%%%%%%%

\begin{frame}\frametitle{点彩色遷移問題(2/2)}

  \begin{itemize}
    \item 色数$k$によって問題の性質が異なることが知られている.
    \begin{itemize}
      \item \structure{$k=2$}のとき, グラフGは2部グラフであり\structure{自明}.
      \item \structure{$k=3$}のとき, \structure{クラスP}に属する.[ L. Cerecedaほか '08]
      \item \structure{$k \ge 4$}のとき, 一般に\structure{\textbf{PSPACE完全}}となる.[Paul Bonsmaほか '09]
    \end{itemize}

    \item グラフの形に制限を加えることにより, 多項式時間で解決可能となるものが存在することがわかっている.[Paul Bonsmaほか '09]
    \begin{itemize}
      \item 平面グラフであり, かつ$4 \le k \le 6$でないとき.
      \item 2部平面グラフであり, かつ$k=4$でないとき.
    \end{itemize}

  \end{itemize}

\end{frame}

%%%%%%%%%%%%%%%%%%%%%%%%%%%%%%%%%%%%%%%%%%%%%%%%%%
%% ASP
%%%%%%%%%%%%%%%%%%%%%%%%%%%%%%%%%%%%%%%%%%%%%%%%%%

\begin{frame}\frametitle{解集合プログラミング}
  
\end{frame}

%%%%%%%%%%%%%%%%%%%%%%%%%%%%%%%%%%%%%%%%%%%%%%%%%%
%% 研究目的
%%%%%%%%%%%%%%%%%%%%%%%%%%%%%%%%%%%%%%%%%%%%%%%%%%

\begin{frame}\frametitle{研究目的}
  \begin{alertblock}{目的}
    ASPを利用し, PSPACE完全な問題の1つである$k \ge 4$の点彩色遷移問題を高速で解ける符号化を提案する.
  \end{alertblock}

  \begin{block}{研究内容} % 予定も含む
    \begin{enumerate}
      \item 点彩色遷移問題を解く, 3種類の符号化の提案.
      \item 点彩色遷移問題のベンチマークの生成.
      \item 作成したベンチマークを用いた, 各種符号化の評価実験.
    \end{enumerate}
  \end{block}

\end{frame}

%%%%%%%%%%%%%%%%%%%%%%%%%%%%%%%%%%%%%%%%%%%%%%%%%%
%% 符号化
%%%%%%%%%%%%%%%%%%%%%%%%%%%%%%%%%%%%%%%%%%%%%%%%%%

\begin{frame}\frametitle{符号化}
  
\end{frame}

%%%%%%%%%%%%%%%%%%%%%%%%%%%%%%%%%%%%%%%%%%%%%%%%%%
%% ベンチマーク
%%%%%%%%%%%%%%%%%%%%%%%%%%%%%%%%%%%%%%%%%%%%%%%%%%

\begin{frame}\frametitle{ベンチマーク}

  \begin{itemize}
    \item 現時点で組合せ遷移問題は理論面の研究が主流であり, ベンチマークの整備が必要.
    \item 実験においてステップ$t$を与えるとき, その上限値が必要となる.
    \item ステップ$t$の最大値は, グラフ$G$を$k$彩色するときの実行可能解の数と等しい.
  \end{itemize}

  従って, 全解列挙が可能な($G, k$)からベンチマークを生成する必要がある.
  
\end{frame}

%%%%%%%%%%%%%%%%%%%%%%%%%%%%%%%%%%%%%%%%%%%%%%%%%%
%% 実験環境
%%%%%%%%%%%%%%%%%%%%%%%%%%%%%%%%%%%%%%%%%%%%%%%%%%

\begin{frame}\frametitle{実験概要}
  ベンチマークの生成に適したグラフを得るため, 以下の実験を行った.
  \begin{itemize}
    \item \structure{使用するグラフ}: 全44個
    \begin{itemize}
      \item \textit{Graph Coloring and its Generalizations}
      \footnote{https://mat.tepper.cmu.edu/COLOR04/}で公開されているグラフを使用.
      \item \textit{Graph Coloring Instances}に属するグラフのうち, \structure{彩色数}が判明しているもの.[Tamuraほか '09]
    \end{itemize}
    \item \structure{色数}: 各グラフの彩色数

    \item \structure{ASPシステム}: \textit{clingo-5.4.0}
      \begin{itemize}
        \item \textit{configuration}は\textit{crafty, tweety}を使用.
      \end{itemize}
    \item \structure{制限時間}: 3600秒/問
    \item \structure{実験環境}: Mac mini, 
  \end{itemize}
  
\end{frame}

%%%%%%%%%%%%%%%%%%%%%%%%%%%%%%%%%%%%%%%%%%%%%%%%%%
%% 実験結果
%%%%%%%%%%%%%%%%%%%%%%%%%%%%%%%%%%%%%%%%%%%%%%%%%%

\begin{frame}\frametitle{実験結果}
  44個のグラフのうち, \structure{9個}のグラフにおいて彩色数での全解列挙が可能であった.
  
  \begin{table}[t]
    \centering
    \begin{tabular}{l|rrr} 
  & 到達可能 & 到達不能 & 合計 \\ \hline
  origin & 11 & 0 & 11 \\
  changed & 11 & 10 & 21 \\
  unchanged & 11 & 10 & 21 \\
\end{tabular}
  \end{table}
\end{frame}



%%%%%%%%%%%%%%%%%%%%%%%%%%%%%%%%%%%%%%%%%%%%%%%%%%
%% ベンチマークの生成
%%%%%%%%%%%%%%%%%%%%%%%%%%%%%%%%%%%%%%%%%%%%%%%%%%

\begin{frame}\frametitle{ベンチマークの生成}

  \begin{itemize}
    \item 実行可能解の総数が小さいグラフについては, \structure{全ての解}を出力.
    \item 実行可能解の総数が100万個を超えるグラフについては, \structure{10万個の解のみ}を出力.
    \item 出力された解からランダムで2つの解を抽出しベンチマークを生成.
    \item 各グラフから5問ずつ, \structure{計45問}のベンチマークを生成.
  \end{itemize}

\end{frame}

%%%%%%%%%%%%%%%%%%%%%%%%%%%%%%%%%%%%%%%%%%%%%%%%%%
%% 今後の課題
%%%%%%%%%%%%%%%%%%%%%%%%%%%%%%%%%%%%%%%%%%%%%%%%%%

\begin{frame}\frametitle{まとめと今後の課題}

  \begin{block}{まとめ}
    \begin{itemize}
      \item 点彩色遷移問題の隣接関係に対して, 3種類の符号化を提案した.
      \item 符号化の評価実験に用いるベンチマークを作成した.
      \begin{itemize}
        \item 頂点数, 辺数, 解空間グラフの大きさ(実行可能解の総数)が異なる9つのグラフを用いた.
      \end{itemize}
    \end{itemize}
  \end{block}
  
  \begin{alertblock}{今後の課題}
    \begin{itemize}
      \item 生成したベンチマークを用いた, 各種符号化の評価実験.
      \item 評価実験で優れていた符号化を, インクリメンタルな解法へと拡張.
    \end{itemize}
  \end{alertblock}

\end{frame}

%%%%%%%%%%%%%%%%%%%%%%%%%%%%%%%%%%%%%%%%%%%%%%%%%%
%% まとめ
%%%%%%%%%%%%%%%%%%%%%%%%%%%%%%%%%%%%%%%%%%%%%%%%%%
\end{document}