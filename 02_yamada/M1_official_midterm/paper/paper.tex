\documentclass[a4j,10pt,dvipdfmx]{jarticle}
\usepackage{multicol}
\usepackage{listings}
\usepackage{graphicx}
\usepackage{color}
\usepackage{tikz}
\usepackage{here}

\newcommand{\code}[1]{\lstinline[basicstyle=\ttfamily]{#1}}

%%%%% 余白の設定 %%%%%%
\setlength{\textheight}{\paperheight}
\setlength{\topmargin}{-15.4truemm}
\addtolength{\topmargin}{-\headheight}
\addtolength{\topmargin}{-\headsep}
\addtolength{\textheight}{-20truemm}
\setlength{\textwidth}{\paperwidth}
\setlength{\oddsidemargin}{-10.4truemm}
\setlength{\evensidemargin}{\oddsidemargin}
\addtolength{\textwidth}{-30truemm}
%%%%% 行間の設定 %%%%%
\renewcommand{\baselinestretch}{.95}

\pagestyle{empty}

%%%%% 題目 %%%%%
\title{解集合プログラミングを用いた組合せ遷移ソルバーの設計と実装}
%%%%% 氏名 %%%%%
\author{山田 悠也(番原研究室)}

\date{}
\begin{document}
\maketitle
\thispagestyle{empty}

\begin{figure}[H]
  \centering
  \setlength{\unitlength}{1.5pt}
\small
\thicklines
%    \thicklines
  \setlength{\unitlength}{1.28pt}
  \small
  \begin{picture}(280,57)(4,-10)
    \put(  0, 20){\dashbox(50,24){\shortstack{根付き全域森\\問題}}}
    \put( 60, 20){\framebox(50,24){変換器}}
    \put(120, 20){\dashbox(50,24){\shortstack{ASPファクト}}}
    \put(120,-10){\alert{\bf\dashbox(50,24){\scriptsize{\shortstack{ASP符号化\\(論理プログラム)}}}}}
    \put(180, 20){\framebox(50,24){ASPシステム}}
    \put(240, 20){\dashbox(50,24){\shortstack{根付き全域森\\問題の解}}}
    \put( 50, 32){\vector(1,0){10}}
    \put(110, 32){\vector(1,0){10}}
    \put(170, 32){\vector(1,0){10}}
    \put(230, 32){\vector(1,0){10}}
    \put(170, +2){\line(1,0){4}}
    \put(174, +2){\line(0,1){30}}
  \end{picture}  

\begin{picture}(280,57)(4,-10)
  \put(  0, 20){\dashbox(50,24){\shortstack{組合せ遷移問題\\のインスタンス}}}
  \put( 60, 20){\framebox(50,24){変換器}}
  \put(120, 20){\dashbox(50,24){\shortstack{ASPファクト}}}
  \put(120,-10){\dashbox(50,24){\shortstack{論理プログラム\\(ASP符号化)}}}
  \put(180,-10){\framebox(50,54){}}
  \put(185, 27){\framebox(40,12){ASP システム}}
  \put(185, -5){\bf\framebox(40,18){\shortstack{有界組合せ遷移\\アルゴリズム}}}
  \put(240, 20){\dashbox(50,24){\shortstack{組合せ遷移問題\\の解}}}
  \put( 50, 32){\vector(1,0){10}}
  \put(110, 32){\vector(1,0){10}}
  \put(170, 32){\vector(1,0){10}}
  \put(230, 32){\vector(1,0){10}}
  \put(170, +2){\line(1,0){4}}
  \put(174, +2){\line(0,1){30}}
  \put(195, 13){\vector(0,1){14}}
  \put(215, 27){\vector(0,-1){14}}
  \put(188, 50){提案ソルバー}
\end{picture} 
  \caption{提案ソルバーの構成図}
  \label{fig:bcr}
\end{figure}

\begin{multicols}{1}
%%%%% 本文ここから %%%%%

\section{研究概要}
\textbf{組合せ遷移問題}とは,
基となる組合せ問題とその2つの実行可能解が与えられたとき,
一方の実行可能解から他方の実行可能解へ,遷移制約を満たしつつ,
実行可能解のみを経由して到達できるかどうかを判定する問題である.
基の組合せ問題が NP 完全であるとき,その遷移問題の
多くは PSPACE 完全になることが知られている.
2022年からは組合せ遷移問題を対象とした国際競技会も開催され,
理論的な研究だけでなく,実践的な研究も盛んに行われることが期待されている.
なかでも,組合せ遷移問題を解く実用的な汎用ソルバーの研究開発は
重要な課題である.

%\textbf{解集合プログラミング}(ASP)は,
%論理プログラミングから派生した宣言的プログラミングパラダイムである.
%ASP 言語は一階論理に基づく知識表現言語の一種である.
%ASP システムは安定モデル意味論に基づく解集合を計算するシステムである.
%
%近年,SAT の技術を応用した高速な ASP システムが開発され,
%プランニング,
%有界モデル検査,
%システム生物学など様々
%な分野への実用的応用が急速に拡大している
%
%組合せ遷移問題に対して ASP を用いる利点として,
%ASP 言語の高い表現力により,記号制約を簡潔に記述できる点,
%インクリメンタル ASP 解法により,同一な解空間を繰り返し
%探索することを防げる点などがあげられる.

\textbf{本研究の目的}は,解集合プログラミング(ASP)技術を利用し,
効率的な組合せ遷移ソルバーを実現することである.
%有界モデル検査やプランニングの技術を取り入れ,
%遷移系列の長さを制限した部分問題を繰り返し解くことで
%到達可能性を判定する.
有界モデル検査やプランニングの技術を応用し,
ソルバーの開発や,各種組合せ遷移問題を解く ASP 符号化を中心に研究開発
を進める.国際競技会等のベンチマーク問題集を用いて提案手法やソルバーを
評価し,ASPに基づく組合せ遷移の有用性・実用性を明らかにする.

\section{研究成果}
\textbf{$k$彩色遷移問題の ASP 符号化 (卒業研究).}
% $k$彩色遷移問題~\cite{core:gcp:BonsmaC09}は,
% グラフ点彩色問題を基とする代表的な組合せ遷移問題である.
$k$彩色遷移問題は,グラフ点彩色問題を基とする代表的な組合せ遷移問題である.
一般に$k \ge 4$において PSPACE 完全であることが知られている.
$k$彩色遷移問題に対し,遷移制約の表現方法が異なる3種類のASP符号化
%として
%\code{origin}符号化,\code{changed}符号化,
%\code{unchanged}符号化
を提案した.
これらの符号化を評価するために,独自に作成した
$k$彩色遷移問題のベンチマーク問題90問を用いて
評価実験を行った.
%その結果,特に\code{unchanged}符号化の
%有効性を確認できた.

\textbf{有界組合せ遷移の提案.} 
有界組合せ遷移は,組合せ遷移問題に対して,制限された長さの (すなわち有
界の) 遷移系列を求める判定問題を記号的に表現し,その判定問題 を ASP シ
ステム等の汎用ソルバーで実行することにより,到達可能性の検査を行う手法である.
この手法は,遷移系列が見つかるまで,その長さの制限を大きくしつつ繰り返す.
遷移系列が見つかれば基の組合せ遷移問題も到達可能となる.
一般に到達不能は判定できないが,遷移系列の長さの
上限値が既知である場合は判定可能となる.
% 長さの上限値としては,基の組合せ問題における解の
% 総数などがあげられる.
%
% 本研究では,有界組合せ遷移を
% 4種類の制約を表す部分論理式から構成される
% 論理式として定式化した.

\textbf{ソルバーの開発.}
有界組合せ遷移に基づくソルバーを開発した.
図\ref{fig:bcr}にソルバーの構成図を示す.
有界組合せ遷移アルゴリズムとして,
2種類の実装方法を考案した.
特に,マルチショット ASP 解法を利用した実装方法は,
無駄な解空間の探索や前処理のオーバーヘッドを避けるように工夫されている
点が特長である.
有界組合せ遷移の有効性を評価するために,
$k$彩色遷移問題(90問)を用いた評価実験を行った.
その結果,マルチショット ASP 解法を利用した実装方法の有効性が確認でき
た.

\section{まとめと今後の課題}
本稿では,解集合プログラミングを用いた
組合せ遷移ソルバーの設計と実装に関する
研究概要と研究成果を述べた.
今後の課題として,国際組合せ遷移競技会
のベンチマーク問題として採用されている独立集合遷移問題
への適用,およびソルバーの高速化があげられる.

\section{研究業績}
\begin{itemize}
\item 山田悠也,湊真一,番原睦則.
  解集合プログラミングに基づく組合せ遷移ソルバーの実装方式に関する考察.
  日本ソフトウェア科学会第38回大会講演論文集,
  46-L,日本ソフトウェア科学会,2021年9月3日.
  {\bf (学生奨励賞)}
\end{itemize}

% %%%%% 参考文献 %%%%%
% \begin{thebibliography}{10}

% %\bibitem{core:Itho2011}
% %Takehiro Ito, Erik D. Demaine, Nicholas J. A. Harvey,
% %Christos H. Papadimitriou, Martha Sideri,
% %Ryuhei Uehara, Yushi Uno.
% %On the complexity of reconfiguration problems,
% %Theoretical Computer Science, Vol.412, No.12-14(2011),
% %pp.1054--1065.
% %
% \bibitem{core:gcp:BonsmaC09}
% Paul S. Bonsma, Luis Cereceda.
% Finding Paths between graph colourings: PSPACE-completeness and superpolynomial distances,
% Theoretical Computer Science, Vol.410, No.50(2009), pp.5215--5226.

% \end{thebibliography}

%%%%% 本文ここまで %%%%%
%%% 末尾のコメント欄と注意事項を削除しないこと. %%%
\end{multicols}
\vfill
\noindent
{\gt コメント欄}
{\footnotesize
(本資料をそのまま発表者に返却します.コメント欄以外にもコメントを書いていただいてもかまいません.)}
\\
\fbox{\begin{minipage}{\textwidth}\noindent\\\\\end{minipage}}	
\end{document}
