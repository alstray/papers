\documentclass[a4j,10pt]{jarticle}
\usepackage{multicol}
\usepackage{listings}

\newcommand{\code}[1]{\lstinline[basicstyle=\ttfamily]{#1}}

%%%%% 余白の設定 %%%%%%
\setlength{\textheight}{\paperheight}
\setlength{\topmargin}{-15.4truemm}
\addtolength{\topmargin}{-\headheight}
\addtolength{\topmargin}{-\headsep}
\addtolength{\textheight}{-20truemm}
\setlength{\textwidth}{\paperwidth}
\setlength{\oddsidemargin}{-10.4truemm}
\setlength{\evensidemargin}{\oddsidemargin}
\addtolength{\textwidth}{-30truemm}
%%%%% 行間の設定 %%%%%
\renewcommand{\baselinestretch}{.95}

\pagestyle{empty}

%%%%% 題目 %%%%%
\title{解集合プログラミングを用いた組合せ遷移ソルバーに関する研究}
%%%%% 氏名 %%%%%
\author{山田 悠也(番原研究室)}

\date{}
\begin{document}
\maketitle
\thispagestyle{empty}
\begin{multicols}{1}
%%%%% 本文ここから %%%%%

\section{研究概要}
\textbf{組合せ遷移問題}とは,
基となる組合せ問題とその2つの実行可能解が与えられたとき,
一方の実行可能解から他方の実行可能解へ,遷移制約を満たしつつ,
実行可能解のみを経由して到達できるかどうかを判定する問題である.
基の組合せ問題が NP 完全であるとき,その遷移問題の
多くは PSPACE 完全になることが知られている.
2022年からは組合せ遷移問題を対象とした国際競技会も開催され,
理論的な研究だけでなく実践的な研究も盛んに行われている.
組合せ遷移問題を解く実用的な汎用ソルバーの開発は
重要な課題である.

\textbf{解集合プログラミング}(ASP)は,
論理プログラミングから派生した宣言的プログラミングパラダイムである.
ASP 言語は一階論理に基づく知識表現言語の一種である.
ASP システムは安定モデル意味論に基づく解集合を計算するシステムである.
%近年,SAT の技術を応用した高速な ASP システムが開発され,
%プランニング,
%有界モデル検査,
%システム生物学など様々
%な分野への実用的応用が急速に拡大している
組合せ遷移問題に対して ASP を用いる利点として,
ASP 言語の高い表現力により,記号制約を簡潔に記述できる点,
インクリメンタル ASP 解法により,同一な解空間を繰り返し
探索することを防げる点などがあげられる.

\textbf{本研究の目的}は,ASP システムを利用し,
組合せ遷移問題を効率よく解く汎用ソルバーを実現することである.
%有界モデル検査やプランニングの技術を取り入れ,
%遷移系列の長さを制限した部分問題を繰り返し解くことで
%到達可能性を判定する.
有界モデル検査の技術を取り入れ,
ソルバーの開発や,組合せ遷移問題のインスタンスの ASP 符号化を中心に
研究開発をすすめる.
独自に生成したベンチマーク問題を用い提案手法やソルバーを評価する.

\section{研究成果}
\textbf{$k$彩色遷移問題の ASP 符号化 (卒業研究).}
$k$彩色遷移問題は,グラフ点彩色問題を基とする代表的な組合せ遷移問題である.
一般に$k \ge 4$において PSPACE 完全であることが知られている.
$k$彩色遷移問題に対し,遷移制約の異なる3種類の符号化として
\code{origin}符号化,\code{changed}符号化,
\code{unchanged}符号化
を提案した.
特に\code{changed}符号化と\code{unchanged}符号化は,
ASP の個数制約を用いることで基礎化後のルール数が少なくなる
ように工夫されている.

3種類の符号化を評価するために,独自に作成した
$k$彩色遷移問題のベンチマーク問題90問を用いて
評価実験を行った.
その結果,特に\code{unchanged}符号化の
有効性を確認できた.

\textbf{有界組合せ遷移の提案.} 

\textbf{ソルバーの開発.}

\section{まとめ}


%\section{受賞}

%%%%% 参考文献 %%%%%
\begin{thebibliography}{10}

\bibitem{cube} ...

\end{thebibliography}

%%%%% 本文ここまで %%%%%
%%% 末尾のコメント欄と注意事項を削除しないこと. %%%
\end{multicols}
\vfill
\noindent
{\gt コメント欄}
{\footnotesize
(本資料をそのまま発表者に返却します.コメント欄以外にもコメントを書いていただいてもかまいません.)}
\\
\fbox{\begin{minipage}{\textwidth}\noindent\\\\\end{minipage}}	
\end{document}
