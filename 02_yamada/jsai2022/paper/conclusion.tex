\section{おわりに}\label{sec:conclusion}

本稿では,解集合プログラミング(ASP) を用いた 独立集合遷移問題(ISRP) の
解法について述べた.
ISRP を解くための基本的な符号化に加え,求解を効率化するためのヒント制
約と探索ヒューリスティックスを考案した.

提案手法の有効性を評価するために,CoRe Challenge 2022 国際競技会で使用
されたベンチマーク問題集のうち,
第1弾ベンチマーク問題集(計11問)を用いた実行実験を行った.
その結果,すべての問題インスタンスの到達可能性を判定することができた.
最短経路を求める shortest については,
到達可能と判定できた7問すべてに対して,その最短経路長を求めることができた.
ループのない最長経路を求める longest については,
到達可能と判定できた7問中4問に対して,その最長経路長を求めることができた.
ISRP の最適値探索において,
独立集合がなるべく極大になるように変数選択および値割当てを行う探索ヒュー
リスティックス(\code{h})の有効性を確認できた.
その一方で,ループを禁止する制約 (\code{noloop}) については改善の必要
があることがわかった.

今後の課題は,CoRe Challenge 2022 の第2弾,第3弾ベンチマーク問題
集を用いた評価実験である.他の課題としては,
ループを禁止する制約の改良,
token sliding に基づく独立集合遷移問題への拡張などが挙げられる.

%%% Local Variables:
%%% mode: japanese-latex
%%% TeX-master: "paper"
%%% End:
