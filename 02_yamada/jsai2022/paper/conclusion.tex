\section{おわりに}\label{sec:conclusion}
本稿では,ASP を用いた ISRP の解法について述べた.
ISRP の符号化に加え,求解を効率化することを期待した
5個のヒント制約を考案した.
また,longest を解くときに必要となる,
遷移系列に同じ実行可能解が高々1回しか出現しない制約も
一つ考案した.

提案手法に対して,CoRe Challenge 2022のベンチマーク問題のうち,
第1弾ベンチマーク問題11問を利用した実行実験を行った.
その結果,shortest ではすべての問題で最短路を求めることができた.
また,ヒント h の優位性を確認できた.
longest では到達可能な7問中4問について最長路を求めることができた.
一方,一部のインスタンスでは最短路すら求めることができない符号化も見られた.
これはnoloop の処理によるオーバーヘッドが大きいことと,
ヒント h と noloop の相性がよくないことの2点が原因として
考えられる.

今後の課題として,第2弾,第3弾ベンチマークを利用した追加の
実験があげられる.
また,noloop の改良も課題の一つである.