\section{はじめに}\label{sec:introduction}

\textbf{組合せ遷移} (Combinatorial Reconfiguration)とは,
基となる組合せ問題とその二つの実行可能解が与えられたとき,
一方から他方へ遷移制約を満たしつつ,
実行可能解のみを経由して到達可能かを判定する問題である.
入力として与えられる二つの実行可能解の片方を
スタート状態,もう一方をゴール状態と呼ぶ.
組合せ遷移が対象とする問題にはいくつかの種類が存在する.
exsistent は経路の存在を判定する問題であり,
経路が存在する場合は具体的な解を出力する.
shortest と longest は最適化問題である.
shortest では最短の経路を求める.
また longest では,同じ実行可能解を経由しない最長の経路を
求める.

組合せ遷移は,2010年代から理論計算機科学の分野での
研究が行われており,
特に計算複雑性に関する研究が多数存在する.%~\cite{core:Heuvel13}
その一方で,実践的な研究はなされていなかった.
しかし今年度からは国際組合せ遷移競技会
(CoRe Challenge 2022)\footnote{\url{https://core-challenge.github.io/2022/}}
も開催され,
今後ソルバー開発を始めとした研究がさかんに行われることが予想される.
%実用的な汎用ソルバーの開発は重要な課題の一つである.

\textbf{独立集合問題} (Independent Set Problem; ISP)は,
与えられたグラフ$G$と自然数$k$に対して,$G$が要素数$k$以上の
独立集合をもつかを判定し,もつ場合には独立集合を求める問題である.
ISP を基とする組合せ遷移問題を
\textbf{独立集合遷移問題} (Independent Set Reconfiguration Problem; ISRP
\cite{core:ItoDHPSUU11})という.
ISRP は,スタート状態の独立集合に含まれるすべての頂点上に
トークンを一つ配置し,各トークンが独立集合の制約を満たすように
トークンを移動させる問題である.
遷移制約は複数提案されている.
代表的なものが\textbf{Token Jump} (TJ) モデルである.
TJ モデルにおける遷移制約は,
「1回の遷移でちょうど一つのトークンが移動する」
となる.

ISRPは代表的な組合せ遷移問題の一つであり,
PSPACE 完全であることが知られている~\cite{core:ItoDHPSUU11}.
また,CoRe Challenge 2022 のベンチマーク問題にも採用されている.

\textbf{解集合プログラミング} (Answer Set Programming; ASP
\cite{Gelfond88:iclp})は,
論理プログラミングから派生した宣言的プログラミングパラダイムである.
ASP 言語は一階論理に基づく知識表現言語の一種である.
ASP システムは安定モデル意味論に基づく解集合を計算するシステムである.
近年,SAT の技術を応用した高速な ASP システムが開発され,
プランニングや有界モデル検査など様々な
分野への実用的応用が急速に拡大している.

組合せ遷移問題に対して ASP を用いる利点として,
ASP 言語の高い表現力により,記号制約を簡潔に記述できる点,
マルチショット ASP 解法により,同一な解空間を繰り返し
探索することを防げる点,
変数選択ヒューリスティックを手動で調整できる点
などがあげられる.

\textbf{有界組合せ遷移}~\cite{Yamada21:jssst}は.
組合せ遷移問題に対して,制限された長さの (すなわち有
界の) 遷移系列を求める判定問題を記号的に表現し,その判定問題 を ASP シ
ステム等の汎用ソルバーで実行することにより,到達可能性の検査を行う手法である.
有界組合せ遷移において遷移系列の長さを$\ell$に制限したとき,
解くべき問題は式\eqref{BoCoRe:phi}に示す$\varphi_{\ell}$として表される.
%
\begin{equation}
  \varphi_{\ell} = S(\bm{x}^0)
  \land \bigwedge_{t=0}^{\ell} C(\bm{x}^t) 
  \land \bigwedge_{t=1}^{\ell} T(\bm{x}^{t-1},\bm{x}^{t})
  \land G(\bm{x}^\ell) \label{BoCoRe:phi}
\end{equation}
%\begin{align}
%  \varphi_{\ell} &= S(\bm{x}^0) \\
%  &\land \bigwedge_{t=0}^{\ell} C(\bm{x}^t) 
%  \land \bigwedge_{t=1}^{\ell} T(\bm{x}^{t-1},\bm{x}^{t}) \label{BoCoRe:phi}\\
%  &\land G(\bm{x}^\ell)
%\end{align}
%
論理式$S(\bm{x}^0)$はスタート制約であり,遷移系列の最初の状態が,
スタート状態として与えられる実行可能解と一致することを表す.
論理式$C(\bm{x}^t)$は組合せ制約であり,
各ステップ$t$で基の組合せ問題の制約を満たすことを表す.
論理式$T(\bm{x}^{t-1},\bm{x}^{t})$は遷移制約であり,
各ステップ$t-1$と$t$の間で遷移制約を満たすことを表す.
論理式$G(\bm{x}^\ell)$はゴール制約であり,遷移系列の最後の状態が,
ゴール状態として与えられる実行可能解と一致することを表す.

本稿では,ASP を用いた ISRP の解法について述べる.
本研究では,ISRP を解くときに必要となる ASP 符号化に加え,
求解に必須ではないが,探索効率を上げることを期待した
ヒント制約を5個考案した.
また,longest を解く場合に必要となる,
遷移系列に同じ実行可能解が高々1回しか出現しない
ことを表す制約を1個考案した.
提案手法を評価するにあたり,CoRe Challenge 2022のベンチマーク問題のうち,
第1弾ベンチマーク問題11問を利用し,shortest と longest の実行実験を行った.
ソルバーには,先行研究~\cite{Yamada21:jssst}で有効性が示された,
マルチショット ASP 解法に基づいたソルバーを利用した.
実験の結果,shortest では,到達可能性の判定,および到達可能な場合の最短路を
求めることができた.
またlongest では,到達可能であった7問中4問について最長路を求めることができた.