\section{実行実験}\label{sec:experiment}

%%%%%%%%%%%%%%%%%%%%%%%%%%%%%%%%%%%%
\begin{table}[t]
  \centering
  \caption{ISRP インスタンスの基となる独立集合問題}
  \vskip 1em
  \begin{tabular}{lrrr|r}
    \begin{tabular}{lrrr|r}
  グラフ名 & 頂点数 & 辺数 & 彩色数 & 実行可能解の総数 \\ \hline
  1-FullIns\_3 & 30 & 100 & 4 & 50,693,280 \\ 
  le450\_5a & 450 & 5,714 & 5 & 3,840 \\ 
  le450\_5c & 450 & 9,803 & 5 & 120 \\ 
  le450\_5d & 450 & 9,757 & 5 & 960 \\ 
  myciel3 & 11 & 20 & 4 & 12,480 \\ 
  myciel4 & 23 & 71 & 5 & 2,845,658,400 \\ 
  queen5\_5 & 25 & 160 & 5 & 240 \\  
  queen6\_6 & 36 & 290 & 7 & 100,800 \\ 
  queen7\_7 & 49 & 476 & 7 & 20,160 \\
\end{tabular}
  \end{tabular}
  \label{tab:graph}
\end{table}

\begin{table*}[t]
  \centering
  \caption{ISRP shortest の実験結果: 求解に要した CPU時間 (秒)}
  \vskip 1em
  \begin{tabular}{lr||lr|rrrrr}
    インスタンス名 & サイズ & 到達可能性 & 遷移長 & \code{nohint} & \code{h} & \code{d1h} & \code{d2ht2} & \code{allhint} \\ \hline
\code{grid004x004_01} & 7 & UNREACHABLE & 19 & 0.064 & 0.086 & 0.067 & 0.148 & \textcolor{red}{0.058} \\ 
\code{grid004x004_02} & 7 & UNREACHABLE & 19 & 0.061 & 0.203 & 0.064 & 0.327 & \textcolor{red}{0.060} \\ 
\code{grid004x004_03} & 6 & REACHABLE & 7 & \textcolor{red}{0.050} & 0.057 & 0.052 & 0.058 & 0.055 \\ 
\code{grid004x004_04} & 6 & REACHABLE & 8 & 0.057 & \textcolor{red}{0.053} & 0.058 & 0.054 & 0.060 \\ 
\code{grid004x004_05} & 7 & UNREACHABLE & 19 & 0.065 & 0.062 & 0.065 & 0.320 & \textcolor{red}{0.058} \\ 
\code{hc-power-11_01} & 9 & REACHABLE & 21 & 0.162 & 0.185 & 0.207 & 0.091 & 0.098 \\ 
\code{hc-power-12_01} & 15 & REACHABLE & 69 & 180.123 & \textcolor{red}{17.410} & 17.491 & 36.278 & 71.051 \\ 
\code{hc-square-01_01} & 6 & REACHABLE & 12 & \textcolor{red}{0.055} & 0.060 & 0.061 & 0.058 & 0.061 \\ 
\code{hc-square-02_01} & 10 & REACHABLE & 30 & 0.818 & 0.586 & 0.594 & \textcolor{red}{0.403} & 0.432 \\ 
\code{hc-toyno-01_01} & 2 & UNREACHABLE & 5 & 0.050 & 0.049 & \textcolor{red}{0.048} & 0.049 & 0.049 \\ 
\code{hc-toyyes-01_01} & 3 & REACHABLE & 3 & 0.049 & \textcolor{red}{0.046} & 0.047 & 0.050 & 0.047 \\ 

  \end{tabular}
  \label{tab:1st_shortest}
\end{table*}

\begin{table*}[t]
  \centering
  \caption{ISRP longest の実験結果: 得られた最適値と最良値}
  \vskip 1em
  \begin{tabular}{l|rrrrrrrr}
    インスタンス名 & \code{nohint} & \code{d1} & \code{d1d2} 
  & \code{d1d2h} & \code{d1h} & \code{d2} & \code{d2h} & \code{h} \\ \hline
\code{grid004x004_03} & 109 & 109 & 109 & 111 & 110 & 109
  & 111 & \textcolor{red}{112} \\ 
\code{grid004x004_04} & 108 & 107 & 109 & \textcolor{red}{110}
  & \textcolor{red}{110} & 109 & \textcolor{red}{110} & \textcolor{red}{110} \\ 
\code{hc-power-11_01} & \textcolor{red}{31*} & \textcolor{red}{31*}
  & \textcolor{red}{31*} & \textcolor{red}{31*} & \textcolor{red}{31*}
  & \textcolor{red}{31*} & \textcolor{red}{31*} & \textcolor{red}{31*} \\ 
\code{hc-power-12_01} & \textcolor{red}{91} & 89 & 85 & - & - & 85 & - & - \\ 
\code{hc-square-01_01} & \textcolor{red}{14*} & \textcolor{red}{14*}
  & \textcolor{red}{14*} & \textcolor{red}{14*} & \textcolor{red}{14*}
  & \textcolor{red}{14*} & \textcolor{red}{14*} & \textcolor{red}{14*} \\ 
\code{hc-square-02_01} & 68 & \textcolor{red}{74*} & \textcolor{red}{74*}
  & \textcolor{red}{74*} & \textcolor{red}{74*} & 72 & \textcolor{red}{74*}
  & \textcolor{red}{74*} \\ 
\code{hc-toyyes-01_01} & \textcolor{red}{7*} & \textcolor{red}{7*}
  & \textcolor{red}{7*} & \textcolor{red}{7*} & \textcolor{red}{7*}
  & \textcolor{red}{7*} & \textcolor{red}{7*} & \textcolor{red}{7*} \\ 

  \end{tabular}
  \label{tab:1st_longest}
\end{table*}
%%%%%%%%%%%%%%%%%%%%%%%%%%%%%%%%%%%%

提案する ASP 符号化の有効性を評価するために,2つの実験を行った.
最初に,ISRP の最短経路を求める shortest の実験について述べる.
%
符号化には,コード\ref{code:exact1.lp}に
ヒント制約と探索ヒューリスティックス (計5個)
を組み合わせた32通りを使用した.
ベンチマークには,CoRe Challenge 2022のベンチマーク問題集のうち,
第1弾ベンチマーク問題集(計11問)を使用した.
表\ref{tab:graph}に,各 ISRP インスタンスの基となる独立集合問題につい
てまとめたものを示す.
各列は左から順に,グラフ名,グラフの頂点数,辺数,
独立集合の要素数$k$,
事前に計算した実行可能解の総数となっている.
今回の実験では,ステップ数の上限値として,解の総数から1ひいた値を使用した.
% 
ソルバーには,ASP システム{\clingo}上に実装した
組合せ遷移ソルバー\textsf{recongo}~\cite{Yamada21:jssst}を用いた.
使用した{\clingo}のバージョンは5.5.1である.
1問あたりの制限時間は10分とした.
実験環境は,Mac OS, 3.3GHz 12コア Intel Xeon W, 96GB メモリである.

表\ref{tab:1st_shortest}に実験結果を示す.
各列は左から順に,インスタンス名,独立集合の要素数$k$,
到達可能性,
遷移長,各符号化における CPU 時間(秒)である.
遷移長の列には,到達可能(REACHABLE)なものについては最短経路の遷移長を,
到達不能(UNREACHABLE)なものについてはステップ数の上限値
(解の総数から1ひいた値と同じ)を記載している.
表\ref{tab:1st_shortest}には,
比較した32通りの符号化のうち,
コード\ref{code:exact1.lp}のみ (\code{nohint}),
ヒント制約と探索ヒューリスティックスをすべて利用したもの (\code{allhint})
に加えて,総 CPU 時間の短かった3つを記載している.
各インスタンスについて,最も短い CPU 時間を赤色で示している.
%
実験の結果,すべての問題に対して到達可能性を判定することができた.
また,到達可能と判定できた問題7問について,その最短経路を求めることができた.
最短経路長が最も大きかったのは,\code{hc-power-12_01}の69ステップであった.
特に,探索ヒューリスティックス(\code{h})を使用した符号化は,
\code{nohint} と比べて優れた性能を示しており,その有効性が確認できた.

つぎに,ISRP のループのない最長経路を求める longest の実験について述べる.
符号化には,コード\ref{code:exact1.lp}とループを禁止する
制約 \code{noloop} に,
ヒント制約(\code{d1}, \code{d2})と探索ヒューリスティックス(\code{h})を
組み合わせた8通りを使用した.
ベンチマークには,第1弾ベンチマーク問題11問のうち,shortest で到達可能
と判定できた7問を使用した.
1問あたりの制限時間は30分とした.
使用したソルバーおよび実験環境は,shortest の場合と同様である.

表\ref{tab:1st_longest}に実験結果を示す.
各列は左から順に,インスタンス名,各符号化で求められた最適値と最良値で
ある.`*' がついた値は最適値,すなわち最長経路長であることを示している.
各インスタンスについて,最も良い結果を赤色で示している.
'-' は1つも解が見つからなかったことを示す.
%
実験の結果,7問中4問で最適値を求めることができた.
その一方で,\code{hc-power-12_01}では最短経路長69の解が求まらない符号
化が存在した.
これは,\code{noloop} のコストが大きいことと,
探索ヒューリスティックス(\code{h})と \code{noloop} の相性がよくないこ
との2点が原因として考えられる.

%%% Local Variables:
%%% mode: japanese-latex
%%% TeX-master: "paper"
%%% End:
