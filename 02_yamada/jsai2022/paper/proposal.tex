\section{独立集合遷移問題の ASP 符号化}\label{sec:proposal}

%----------------------------------------
\lstinputlisting[float=t,caption={%
独立集合遷移問題(図~\ref{fig:ex_isrp})の ASP ファクト形式},%
captionpos=b,frame=single,label=code:isrp_fact.lp,%
numbers=none,%
breaklines=true,%
columns=fullflexible,keepspaces=true,%
xrightmargin=1zw,% 
xleftmargin=1zw,% 
basicstyle=\ttfamily\small]{code/isrp_fact.lp} 
%----------------------------------------

%----------------------------------------
\lstinputlisting[float=t,caption={%
独立集合遷移問題の ASP 符号化},%
captionpos=b,frame=single,label=code:exact1.lp,%
numbers=left,%
breaklines=true,%
columns=fullflexible,keepspaces=true,%
xrightmargin=1zw,% 
xleftmargin=1zw,% 
basicstyle=\ttfamily\small]{code/exact1_inc.lp} 
%----------------------------------------

提案解法では,まず与えられた問題インスタンスを ASP のファクト形式に変
換した後,ASP ファクトと ISRP を解くための ASP 符号化と結合した上で,
高速 ASP システム{\clingo}上に実装された
組合せ遷移ソルバー\textsf{recongo}を用いて解を求める.
%
コード~\ref{code:isrp_fact.lp}に
独立集合遷移問題の例(図~\ref{fig:ex_isrp})の ASP ファクト表現を示す.
%
アトム\code{n/1}はグラフの頂点数,\code{e/1}は辺数,
\code{k/1}は独立集合の要素数を表す.
アトム\code{node/1}は各頂点を,\code{edge/2}は各辺を表す.
\code{start/1}と\code{goal/1}はそれぞれスタート状態と
ゴール状態を表すアトムである.

コード\ref{code:exact1.lp}に独立集合遷移問題を解くための ASP 符号化を示す.
\code{#program}文は,一つのプログラムをいくつかの
サブプログラムに分けるための宣言子である.
コード\ref{code:exact1.lp}のプログラムは,
\code{base}, \code{step(t)}, \code{check(t)}の3つの
サブプログラムから構成されている.
アトム\code{in(X,t)}は,ステップ\code{t}で頂点\code{X}
にトークンが配置される(すなわち,頂点\code{X}が独立集合に含まれる)
ことを意味する.

最初に \code{base} には,ステップ\code{t=0}で満たすべき制約を記述する.
3行目のルールは,スタート状態とステップ\code{0}でのトークン配置が
一致することを表している.
次に \code{step(t)} には,
各ステップ \code{t} において満たすべき制約を記述する.
7行目のルールは,トークの個数 (すなわち,独立集合の要素数) が \code{K}
個であることを表す.
8行目のルールは,隣接する頂点 \code{X} と \code{Y} に同時にトークンが
置かれないことを表している.
10行目の \code{moved_from(X,t)}は,ステップ \code{t-1} から \code{t}
への遷移で,頂点 \code{X} にあったトークンが移動したことを意味する補助
アトムである.
11行目のルールは,各ステップで移動できる
トークンはただ一つという遷移制約を表している.
最後に,\code{check(t)} では,終了条件を記述する.
ここでは,15行目のルールでゴール状態を表している.
ステップを表す \code{t} がインクリメントされると,一つ前の不要になっ
た終了条件は,\code{query(t)} の真偽を動的に操作することにより無効化
される.

コード\ref{code:exact1.lp}のプログラムとは別に,
求解に必須ではないが,探索効率を上げることを目的としたヒント制約4つ
と探索ヒューリスティックス1つを考案した.
\begin{itemize}
\item \code{distance1 (d1)}:
  スタート状態から \code{t} ステップ後において,
  スタート状態ではトークンが配置されており,
  かつステップ \code{t} では配置されていない頂点の数は高々 \code{t} 個
  であることを表す制約である.
\item \code{distance2 (d2)}:
  ステップ \code{t} において,
  それより前の各ステップ \code{T} $\in\{$\code{0},\ldots,\code{t-1}$\}$
  に対して,ゴール状態ではトークンが配置されており,かつステップ \code{T}
  では配置されていない頂点の数は高々 \code{t-T} 個であることを表す制約
  である.
\item \code{token1 (t1)}:
  ステップ \code{t-1} で頂点 \code{X} から頂点 \code{Y}
  にトークンが移動したとき,
  ステップ \code{t} で任意の頂点 \code{Z} から頂点 \code{X}
  へトークンが移動することを禁止する制約である.
  この連続する2ステップの遷移は,ステップ \code{t-1} で
  頂点 \code{Z} から頂点 \code{Y} へトークンを移動させることで,
  1ステップで再現できることを利用している.
\item \code{token2 (t2)}:
  ステップ \code{t-1} で頂点 \code{Y} から頂点 \code{X}
  にトークンが移動したとき,
  ステップ \code{t} で頂点 \code{X} から任意の頂点 \code{Z}
  へトークンが移動することを禁止する制約である.
  この連続する2ステップの遷移は,ステップ \code{t-1} で
  頂点 \code{Y} から頂点 \code{Z} へトークンを移動させることで,
  1ステップで再現できることを利用している.
\item \code{heu (h)}:
  ASP システム{\clingo}の \code{#heuristic} 文を用いて,
  各ステップにおいて独立集合がなるべく極大になるように
  変数選択および値割当てを行う探索ヒューリスティックスである.
\end{itemize}

コード\ref{code:exact1.lp}の ASP 符号化,および,
ヒント制約とヒューリスティックスは,
ISRP の3つのバリエーション exsistent, shortest, longest
を解くために使用できる.
ただし,ヒント制約 \code{t1} と \code{t2} は冗長なトークンの動きを制限
するため,ループのない最長経路を求める longest には使用できない.
また,longest については,別途考案したループを禁止する制約を加える必要がある.

%%% Local Variables:
%%% mode: japanese-latex
%%% TeX-master: "paper"
%%% End:
