\section{独立集合遷移問題の ASP 符号化}\label{sec:proposal}

%----------------------------------------
\lstinputlisting[float=t,caption={%
マルチショット ASP 解法における独立集合遷移問題の ASP 符号化.},%
captionpos=b,frame=single,label=code:exact1.lp,%
numbers=none,%
breaklines=true,%
columns=fullflexible,keepspaces=true,%
xrightmargin=1zw,% 
xleftmargin=1zw,% 
basicstyle=\ttfamily\scriptsize]{code/exact1_inc.lp} 
%----------------------------------------

マルチショット ASP 解法における ISRP の ASP 符号化を
コード\ref{code:exact1.lp}に示す.
\code{#program}文は,一つのファイルをいくつかの
サブプログラムにわける記述である.
3行目のルールはスタート制約を表している.
7--8行目は組合せ制約を表している.
10--11行目は遷移制約を表している.
15行目はゴール制約を表している.

コード\ref{code:exact1.lp}とは別に,
求解に必須ではないが,探索効率を上げることを期待した
ヒント制約を5個考案した.
\begin{itemize}
  \item \code{distance1} (d1): スタート状態から$t$ステップ後において,
    スタート状態ではトークンが配置されており,かつステップ$t$では
    配置されていない頂点の数は高々$t$個であることを表す.
  \item \code{distance2} (d2): ゴール状態から$t$ステップ前において,
    ゴール状態ではトークンが配置されており,かつステップ$\ell-t$
    では配置されていない頂点の数は高々$t$個であることを表す.
  \item \code{token1} (t1): ステップ$t-1$で頂点\code{X}から頂点\code{Y}
    にトークンが移動したとき,ステップ$t$で任意の頂点\code{Z}から頂点\code{X}
    へトークンが移動することを禁止する制約である.
    この2ステップの遷移は,ステップ$t-1$で
    頂点\code{Z}から頂点\code{Y}へトークンを
    移動させることで,1ステップで再現できることを利用している.
    冗長な動きを制限するヒント制約であるため,longest では使用できない.
  \item \code{token2} (t2): ステップ$t-1$で頂点\code{X}から頂点\code{Y}
    にトークンが移動したとき,ステップ$t$で頂点\code{Y}から任意の頂点\code{Z}
    へトークンが移動することを禁止する制約である.
    この2ステップの遷移は,ステップ$t-1$で
    頂点\code{X}から頂点\code{Z}へトークンを移動させることで,
    1ステップで再現できることを利用している.
    冗長な動きを制限するヒント制約であるため,longest では使用できない.
  \item \code{heu} (h): ASP システムの{\clingo}では,変数選択
    ヒューリスティックを調整するための構文が用意されている.
    これを利用し,遷移系列の序盤では独立集合がなるべく極大になるようにしている.
\end{itemize}

longest では,遷移系列に同じ実行可能解が高々1回しか出現しない制約を加える
必要がある.
本研究では,「任意の2つのステップにおいて,
独立集合に含まれる頂点がすべて同じにはならない」
ことを表す制約を追加した.
