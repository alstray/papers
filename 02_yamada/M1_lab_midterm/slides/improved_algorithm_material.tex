\begin{frame}\frametitle{改良アルゴリズム}

  \begin{block}{改良アルゴリズムの手続き}
    \begin{enumerate}
    \item ステップ$t$を$t=0$,論理プログラム$\Psi$を$\Psi = \emptyset$とする.
    \item ファクト形式の問題インスタンス$I$を与える.
    \item ステップ$t$で追加する論理式を記憶するリスト$p=\emptyset$とする.
      \label{improved_solver:loop}
    \item $t>0$であれば,$Remove(\Psi, \enc{G(\bm{x}^{t-1})}$
      % ゴール状態の制約$G(\bm{x}^{t -1})$を表すルールを削除する.
    \item $t=0$であれば,$p = p \ \cup \ \enc{S(\bm{x}^0)}
      \ \cup \ \enc{I}$
      % スタート状態の制約$S(\bm{x}^0)$を表すルールを追加する.
    \item $p = p \ \cup \ \enc{C(\bm{x}^{t})} \ \cup \ 
      \enc{T(\bm{x}^{t-1},\bm{x}^{t})} \ \cup \ \enc{G(\bm{x}^{t})}$
      % グラフ点彩色問題の制約$C(\bm{x}^{t})$と遷移制約$T(\bm{x}^{t-1},\bm{x}^{t})$
      % を表すルールを追加する.
      % ゴール状態の制約$G(\bm{x}^{t})$を表すルールを追加する.
    \item $\Psi = Add(\Psi, p)$
    \item $\Psi$を ASP システムに与えて解く.
    \item ASP システムの出力が充足可能であれば終了する.
      充足不能であれば$t$の値を1増加させ,
      (\ref{improved_solver:loop})に戻り手続きを繰り返す.
      \begin{itemize}
      \item ただし,$t$がステップ長の上限値を超えたところで
        繰り返しを停止できる.
      \end{itemize} \label{improved_solver:end}
    \end{enumerate}
  \end{block}

  \begin{itemize}
  \item 改良アルゴリズムは,ASP システム{\clingo}の python インターフェー
    スを用いて簡潔に実装できる.
  \end{itemize}

\end{frame}