%%%% 補助スライド
\appendix
\backupbegin
%%%%%%%%%%%%%%%%%%%%%%%%%%%%%%%%%%%%%%%%%%%%%%%%%%%%%%%%%%%%%%%%%%%
\begin{frame}{}
  \begin{center}\Huge
    Appendix
  \end{center}
\end{frame}
%%%%%%%%%%%%%%%%%%%%%%%%%%%%%%%%%%%%%%%%%%%%%%%%%%%%%%%%%%%%%%%%%%%
\begin{frame}
  \frametitle{到達可能の判定に要したCPU時間}

  \begin{exampleblock}{}
    \centering
    \scalebox{0.75}{\begin{tabular}{lr|rr|rr}
  \multicolumn{2}{r|}{} & \multicolumn{2}{c|}{基本ソルバー} & \multicolumn{2}{c}{改良ソルバー} \\
  問題名 & ステップ長 & \code{changed} & \code{unchanged} & \code{changed} & \code{unchanged} \\ \hline
  \code{1-FullIns_3_col4_1} & 11 & 1.374 & 1.071 & 1.291 & 0.756 \\
  \code{myciel3_col4_1} & 11 & 0.179 & 0.148 & 0.065 & 0.065 \\
  \code{myciel3_col4_3} & 11 & 0.206 & 0.164 & 0.082 & 0.080 \\
  \code{myciel3_col4_4} & 13 & 0.486 & 0.445 & 0.303 & 0.583 \\
  \code{myciel3_col4_7} & 8 & 0.090 & 0.072 & 0.042 & 0.036 \\
  \code{myciel3_col4_8} & 9 & 0.119 & 0.094 & 0.048 & 0.044 \\
  \code{myciel4_col5_2} & 16 & 42.482 & 48.721 & 20.308 & 134.056 \\
  \code{myciel4_col5_3} & 14 & 17.098 & 21.211 & 5.286 & 80.848 \\
  \code{myciel4_col5_4} & 7 & 0.256 & 0.193 & 0.086 & 0.072 \\
  \code{myciel4_col5_6} & 10 & 1.525 & 1.391 & 1.052 & 1.558 \\
  \code{myciel4_col5_10} & 17 & 212.682 & 73.646 & 29.999 & 200.068 \\
\end{tabular}}
  \end{exampleblock}
  \begin{itemize}
    \item ステップ長については,最大で長さ17の問題が解けた.
    % \item 全体の結果とは異なる傾向が見られた.
    %   \begin{itemize}
    %     \item 十分な長さの遷移系列が最短の解となるようなベンチマークが必要である.
    %   \end{itemize}
  \end{itemize}
\end{frame}
%%%%%%%%%%%%%%%%%%%%%%%%%%%%%%%%%%%%%%%%%%%%%%%%%%%%%%%%%%%%%%%%%%% 
\begin{frame}
  \frametitle{基となるグラフごとの解けた問題数}
  
  \begin{exampleblock}{}
    \centering
    \scalebox{0.8}{\begin{tabular}{l|rrr|rrr}
  グラフ名 & 平均次数 & 色数$k$ & 実行可能解の総数 & 到達可能 & 到達不能 & 判定不能 \\ \hline
  1-FullIns\_3 & 6.67 & 4 & 50,693,280 & 1 & 0 & \structure{9} \\ 
  le450\_5a & 25.40 & 5 & 3,840 & 0 & 10 & 0 \\ 
  le450\_5c & 43.57 & 5 & 120 & 0 & 10 & 0 \\ 
  le450\_5d & 43.36 & 5 & 960 & 0 & 10 & 0 \\ 
  myciel3 & 3.64 & 4 & 12,480 & 5 & 0 & \structure{5} \\ 
  myciel4 & 6.17 & 5 & 2,845,658,400 & 5 & 0 & \structure{5} \\ 
  queen5\_5 & 12.80 & 5 & 240 & 0 & 10 & 0 \\  
  queen6\_6 & 16.11 & 7 & 100,800 & 0 & 10 & 0 \\ 
  queen7\_7 & 19.43 & 7 & 20,160 & 0 & 10 & 0 \\
\end{tabular}}
  \end{exampleblock}
  \begin{itemize}
    \item $\textrm{平均次数} = (\textrm{辺数}*2) / (\textrm{頂点数})$
    \item 色数$k$は彩色数に等しい.
    \item 難しい問題の特徴としては,基となる問題の実行可能解の総数が多いことや,
      平均次数と色数$k$の差が少ないことなどが考えられる.
      \begin{itemize}
        \item $k \ge \Delta(G) + 2$ならば到達可能である.
        \item $\Delta(G)$はグラフ$G$の最大次数である.
      \end{itemize}
  \end{itemize}
\end{frame}
%%%%%%%%%%%%%%%%%%%%%%%%%%%%%%%%%%%%%%%%%%%%%%%%%%%%%%%%%%%%%%%%%%% 
\begin{frame}[shrink]
  \frametitle{$k$彩色遷移問題を解く\code{unchanged} 符号化 {\small(基本ソルバー用)}}

\begin{columns}[t]
\begin{column}{0.95\linewidth}
\begin{exampleblock}{\code{unchanged.lp}}
\lstinputlisting[frame=none,numbers=left,basicstyle=\ttfamily\scriptsize]{code/gcrp_cc_unchanged.lp} 
\end{exampleblock}    
\end{column}
\end{columns}

% \begin{itemize}
%   \item 9--10行目: 遷移制約$T(\bm{x}^{t-1},\bm{x}^{t})$の表し方が,
%     \code{changed}符号化と異なる.
%   \item アトム\code{unchanged(X,T)}は,ステップ\code{T-1}とステップ\code{T}の間
%     で,頂点\code{X}の色が変化しなかったことを意味する.
%   \item 10行目: 各ステップ\code{T}において,色が変化しない頂点はちょうど
%     \code{N-1}個であることを表す.
%     \begin{itemize}
%       \item \code{N}はグラフの頂点数を表す.
%     \end{itemize}
% \end{itemize}

\end{frame}
%%%%%%%%%%%%%%%%%%%%%%%%%%%%%%%%%%%%%%%%%%%%%%%%%%%%%%%%%%%%%%%%%%% 
\begin{frame}[shrink]
  \frametitle{$k$彩色遷移問題を解く\code{unchanged} 符号化 {\small(改良ソルバー用)}}

\begin{columns}[t]
\begin{column}{0.9\linewidth}
\begin{exampleblock}{\code{unchanged_inc.lp}}
\lstinputlisting[frame=none,numbers=left,basicstyle=\ttfamily\scriptsize]{code/gcrp_cc_unchanged_inc.lp} 
\end{exampleblock}    
\end{column}
\end{columns}
\end{frame}
%%%%%%%%%%%%%%%%%%%%%%%%%%%%%%%%%%%%%%%%%%%%%%%%%%%%%%%%%%%%%%%%%%%
\begin{frame}\frametitle{$k$彩色遷移問題の性質}

  \begin{itemize}
    \item 色数$k$によって問題の性質が異なることが知られている.
    \begin{itemize}
      \item \structure{$k=2$}のとき,グラフGは2部グラフであり\structure{明らかに到達不能}~[Cereceda+ '08].
      \item \structure{$k=3$}のとき,\structure{クラスP}に属する~[Cereceda+ '08].
      \item \structure{$k \ge 4$}のとき,一般に\structure{\textbf{PSPACE完全}}となる~[Bonsma+ '09].
    \end{itemize}

    \item グラフの形に制限を加えることにより,多項式時間で解決可能となるものが存在することがわかっている~[Bonsma+ '09].
    \begin{itemize}
      \item 平面グラフであり,かつ$k \ge 7$のとき.
      \item 2部平面グラフであり,かつ$k \ge 5$のとき.
    \end{itemize}

  \end{itemize}
\end{frame}
%%%%%%%%%%%%%%%%%%%%%%%%%%%%%%%%%%%%%%%%%%%%%%%%%%%%%%%%%%%%%%%%%%%
\begin{frame}{ASP 言語の構文(1)}
  \begin{alertblock}{}\centering
    ASP言語は論理プログラムをベースとする\footnotemark[1].
  \end{alertblock}
\begin{itemize}
\item \structure{論理プログラム}とは以下の形式の\structure{ルール}
  の有限集合である.
  \[
    \underbrace{a_0}_{\textrm{ヘッド}}\ \texttt{:-}\
    \underbrace{a_1\texttt{,}\dots\texttt{,}a_m\texttt{,}
      \texttt{not}\ {a_{m+1}}\texttt{,}\dots\texttt{,}
      \texttt{not}\ {a_n}\texttt{.}}_{\textrm{ボディ}}
  \]
\item $0\leq m\leq n$ であり,各$a_i$はアトム,
  \texttt{not}は\structure{デフォルトの否定},
  ``\texttt{,}''は連言を表す.
\item \alert{\bf 直観的な意味}は「$a_1,\ldots,a_m$がすべて成り立ち,
  $a_{m+1},\ldots,a_n$のそれぞれが成り立たないならば,$a_0$が成り立つ」である.
\end{itemize}
 \footnotetext[1]{本発表では標準論理プログラムを単に論理プログラムと呼ぶ.}
\end{frame}
%%%%%%%%%%%%%%%%%%%%%%%%%%%%%%%%%%%%%%%%%%%%%%%%%%%%%%%%%%%%%%%%%%%
\begin{frame}{ASP 言語の構文(2)}
\begin{itemize}
\item ボディが空のルールは\structure{ファクト}と呼ばれ,
  ``\texttt{:-}''は省略できる.
  \[
    \underbrace{a_0}_{\textrm{ヘッド}}\texttt{.}
  \]
\item ヘッドが空のルールは\structure{一貫性制約}と呼ばれる.
  \[
    \texttt{:-}\
    \underbrace{a_1\texttt{,}\dots\texttt{,}a_m\texttt{,}
      \texttt{not}\ {a_{m+1}}\texttt{,}\dots\texttt{,}
      \texttt{not}\ {a_n}\texttt{.}}_{\textrm{ボディ}}
  \]
  例えば,\\[1em]
  \begin{tabular}[t]{l|l}
    \(\texttt{:-}\ a\texttt{.}\) &
   「$a$ではない」という禁止を表す.\\
    \(\texttt{:-}\ \texttt{not}\ a\texttt{.}\) &
   「$a$でなければならない」という強制を表す.\\
    \(\texttt{:-}\ \texttt{not}\ a_1\texttt{,} {a_{2}}\texttt{.}\)&
  「$a_2$ならば$a_1$」を表す.
  \end{tabular}
\end{itemize}
\end{frame}
%%%%%%%%%%%%%%%%%%%%%%%%%%%%%%%%%%%%%%%%%%%%%%%%%%%%%%%%%%%%%%%%%%% 
\begin{frame}[shrink]{拡張構文}
\begin{alertblock}{}\centering
  組合せ問題やグラフ問題を解くのに便利な構文が用意されている.
\end{alertblock}

\begin{itemize}
\item \structure{選択子}\\
  \begin{center}
   \code{\{}\(a_1\texttt{;}\dots\texttt{;}a_n\)\code{\}.}\\
  \end{center}
  アトム集合\(\{a_1,\dots,a_n\}\)の任意の部分集合が成り立つことを意味
  する.
\item \structure{個数制約}
  \begin{center}
   $lb$\ \code{\{}\(a_1\texttt{;}\dots\texttt{;}a_n\)\code{\}}\ $ub$\code{.}
  \end{center}
  $a_1,\dots,a_n$のうち,$lb$個以上$ub$個以下が成り立つことを意味する.
\end{itemize}
% \vfill
% 組合せ最適化問題を解くために,最小化関数 (\structure{\texttt{\#minimize}}) 
% と最大化関数 (\structure{\texttt{\#maximize}}) 等も用意されている.
\end{frame}
%%%%%%%%%%%%%%%%%%%%%%%%%%%%%%%%%%%%%%%%%%%%%%%%%%%%%%%%%%%%%%%%%%%
\backupend

%%% Local Variables:
%%% mode: japanese-latex
%%% TeX-master: "slides"
%%% End:
