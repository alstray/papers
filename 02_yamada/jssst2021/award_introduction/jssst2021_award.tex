% 以下の3行は変更しないこと.
\documentclass[T, dvipdfmx]{compsoft}
\taikai{2021}
\pagestyle {empty}

% ユーザが定義したマクロなどはここに置く.ただし学会誌のスタイルの
% 再定義は原則として避けること.

\begin{document}

% 論文のタイトル
\title{受賞者による受賞研究紹介}

% 著者
% 和文論文の場合,姓と名の間には半角スペースを入れ,
% 複数の著者の間は全角スペースで区切る
%
\author{山田 悠也}

\maketitle \thispagestyle {empty}

%%% 本文 %%%
\section{解集合プログラミングに基づく組合せ遷移ソルバーの実装方式に関する考察 (山田 悠也)}

組合せ遷移問題とは,基となる組合せ問題とその2つの実行可能解が与えられたとき,
一方の実行可能解から他方の実行可能解へ,遷移制約を満たしつつ,
実行可能解のみを経由して到達できるかどうかを判定する問題です.

組合せ遷移問題は,理論計算機科学の分野を中心に最近10年で急速に発展し,理論的な基盤が整備されつつあります.
この理由の一つは,組合せ遷移問題には PSPACE 完全な問題が多く存在する点にあるのでしょう.
今回の発表で例として用いた$k$彩色遷移問題も PSPACE 完全な問題の一つであり,
$k \ge 4$において PSPACE 完全であることが知られています.
一方で実社会での応用に目を向けると,配電網の切替えや無線の周波数割当ての切替えなど,
持続可能なシステムへの実用的応用が期待されています.
このように,組合せ遷移問題は大きな注目を集めつつありますが,汎用的なソルバーの開発はまさに始まったばかりです.
私たちの目的は,有界モデル検査や解集合プログラミングなどの既存の技術を取入れ,汎用ソルバーを開発することにあります.

本研究で開発しているソルバーは,組合せ遷移問題に対してステップ長$t$を導入し,
ステップ長を制限しながら何度も到達可能性を判定します.
さらに ASP システムの\textit{clingo}に備わるインクリメンタルな探索機能を用い,
システム上のオーバーヘッドを軽減しています.
これにより,判定可能な問題数が増えたことは大会で発表させていただいたとおりです.

今後の課題としては,ソルバーの高速化と問題記述例の蓄積を考えています.
問題記述例については,独立点集合問題の遷移問題の記述に取組んでいるところです.
高速化については,有界モデル検査やプランニングについての知識を深め,
さらに技術を取入れたいと考えています.

最後になりますが,共著者の皆様,そしてご指導,ご協力いただいた研究室の皆様に
心より感謝申し上げます.
今回の受賞を励みとし,より一層努力してまいります.

\end{document}
