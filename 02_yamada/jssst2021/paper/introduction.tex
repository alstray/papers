%%%%%%%%%%%%%%%%%%%%%%%%%%%%%%%%%%%%%%%%%%%%%%%%%%%%%%%%%% 
\section{はじめに}\label{chap:introduction}
%%%%%%%%%%%%%%%%%%%%%%%%%%%%%%%%%%%%%%%%%%%%%%%%%%%%%%%%%% 

\textbf{組合せ遷移問題}
(Combinatorial Reconfiguration Problems;
CoRe~\cite{core:ItoDHPSUU11,core:Nishimura18,core:Heuvel13})
とは,基となる組合せ問題とその2つの実行可能解が与えられたとき,
一方の実行可能解から他方の実行可能解へ,遷移制約を満たしつつ,
実行可能解のみを経由して到達できるかどうかを判定する問題である.
% \footnote{\url{http://www.ecei.tohoku.ac.jp/alg/coresurvey/}  

計算困難性について,基の組合せ問題が NP 完全であるとき,その遷移問題の
多くは PSPACE 完全になることが知られている.代表的なものとしては,
命題論理の充足可能性判定問題(SAT),
集合被覆問題,
独立点集合,
グラフ点彩色問題を基とする組合せ遷移問題などがある~\cite{%
  core:gcp:BonsmaC09,%
  core:gcp:CerecedaHJ11,%
  core:sat:GopalanKMP09,%
  core:ItoDHPSUU11%
}.

組合せ遷移問題の研究は,理論計算機科学の分野を中心に,最近10年で急速に
発展し,理論的な基盤が整備されつつある.
また,実社会においても,配電網の切替えなど持続的システムへの実用的応用
が期待されている.
しかしながら,一般の組合せ遷移問題を解く汎用ソルバーのアルゴリズム,
ヒューリスティックスなどの実践的な研究開発はまだ始まったばかりである.
著者らの知る限り,実用的な汎用ソルバーはまだ存在していない.

解集合プログラミング(Answer Set Programming; ASP~\cite{%
Baral03:cambridge,%
Gelfond88:iclp,%
Inoue08:jssst,%
Niemela99:amai})
は,論理プログラミングから派生した宣言的プログラミングパラダイムである.
ASP 言語は一階論理に基づく知識表現言語の一種である.
ASP システムは安定モデル意味論~\cite{Gelfond88:iclp}に基づく解集合を計算するシステムである.
近年,SAT の技術を応用した高速な ASP システムが開発され,
プランニング,
%スケジューリング,
有界モデル検査,
システム生物学など様々
な分野への実用的応用が急速に拡大している~\cite{Erdem16:AI}.

組合せ遷移問題に対して ASP を用いる利点としては,
ASP 言語の高い表現力により,記号制約を簡潔に記述できる点,
インクリメンタル ASP 解法により,学習節を再利用した
%遷移問題に対する
効率的な解探索が可能である点,
探索ヒューリスティックスを簡単にカスタマイズできる点,
などが挙げられる.

本稿では,解集合プログラミング(ASP)を用いた組合せ遷移問題の解法として,
有界組合せ遷移を提案し,そのソルバーの実装方法について述べる.
有界組合せ遷移は,SAT 技術を用いたモデル検査手法の1つである
有界モデル検査~\cite{%
  JSAI:BanbaraT10,DBLP:series/faia/Biere09,DBLP:conf/tacas/BiereCCZ99}%
を組合せ遷移に応用した解法といえる.

有界組合せ遷移は,組合せ遷移問題に対して,制限された長さの(すなわち
有界の)遷移系列を論理プログラムとして記号的に表現し,その論理プログラ
ムを ASP システムで実行することにより,到達可能性の検査を行う.
ASP システムが充足可能と判定した 場合,制限された長さの到達可能な遷移
系列が存在することを意味する.逆に充足不能と判定した場合,制限された長
さでは到達不能であることを意味する. この場合,遷移系列の長さを増加させ
た論理プログラムを構成し,再び ASP システムによる実行を繰り返す.
有界組合せ遷移は,到達可能な遷移系列を探すだけで,到達不能の証明は行わ
ない不完全な手続きであるが,検査すべき遷移系列の長さに上限が存在する場
合には完全な手続きとなる.

有界組合せ遷移の有効性を評価するために,
代表的な組合せ遷移問題の1つである$k$彩色遷移問題をベンチマークとして,
提案ソルバーの性能評価を行った.
その結果,インクリメンタル ASP 解法を用いた実装方法が
より多くの問題を高速に解くことができることを確認した.

%%% Local Variables:
%%% mode: latex
%%% TeX-master: "paper"
%%% End:
