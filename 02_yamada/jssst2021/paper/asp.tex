\section{解集合プログラミング} \label{chap:asp}

\textbf{解集合プログラミング} (Answer Set Programming; ASP~\cite{%
  Baral03:cambridge,%
  Gelfond88:iclp,%
  Inoue08:jssst,%
  Niemela99:amai})
は,演繹データベース,否定を含む論理プログラミング,
非単調推論,制約充足 (特に,SAT)を起源にもつ,
宣言的プログラミングパラダイムである.
ASP の言語は,一般拡張選言プログラムに基づいている.
本稿では説明の簡略化のため,そのサブクラスである
標準論理プログラムについて説明する.
以降,標準論理プログラムを単に論理プログラムとよぶ.

\textbf{論理プログラム}は,以下の形式をした\textbf{ルール}の有限集合である.
\[
  a_0\leftarrow a_1,\dots,a_m,\naf{a_{m+1}},\dots,\naf{a_n}
\]
ここで,
$0\leq m\leq n$ であり,
各$a_i$はアトム,
$\naf{}$はデフォルトの否定
\footnote{``失敗による否定''ともよばれる.述語論理で定義される否定($\neg$)とは意味が異なる.},
``$,$''は連言を表す.
$\leftarrow$の左側をヘッド,右側をボディとよぶ.
ルールの直観的な意味は,
「$a_1,\ldots,a_m$がすべて成り立ち,$a_{m+1},\ldots,a_n$のそれぞれが成
り立たないならば,$a_0$が成り立つ」である.
ボディが空のルール(すなわち\(a_0\leftarrow\))を\textbf{ファクト}とよび,
$\leftarrow$を省略してよい.

ヘッドが空のルールを\textbf{一貫性制約}とよぶ.
\[
  \leftarrow a_1,\dots,a_m,\naf{a_{m+1}},\dots,\naf{a_n}
\]

直観的には,\(\leftarrow a_1\)は,「$a_1$ではない」という禁止を表し,
\(\leftarrow \naf{a_1}\)は,「$a_1$でなければならない」という強制を表す.
また,
%\(\leftarrow a_1,a_2\)は,「$a_1$と$a_2$が両方同時に成り立つことはない」を意味し,
\(\leftarrow a_1, \naf{a_{2}}\)は,「$a_1$が成り立つならば,$a_2$が成り立つ」を意味する.

ASP 言語は,組合せ問題を解くための便利な拡張構文を備えている.
その代表的なものが\textbf{選択子}と\textbf{個数制約}である.
例えば,選択子\(\{a_1;\dots;a_n\}\)をファクトとして書くと,
「アトム集合\(\{a_1,\dots,a_n\}\)の任意の部分集合が成り立つ」を意味する.
個数制約は選択子の両端に選択可能な個数の上下限を付けたものである.
例えば,\(lb\ \{a_1;\dots;a_n\}\ ub \leftarrow Body\)と書くと,
「$Body$が成り立つならば,$a_1,\dots,a_n$のうち,$lb$個以上$ub$個以下
が成り立つ」を意味する.

\textbf{ASP システム}は,与えられた論理プログラムから,
安定モデル意味論~\cite{Gelfond88:iclp}
に基づく解集合を計算するソフトウェアである.
ASP システムの多くは,
変数を含む論理プログラムを変数を含まない論理プログラムに
\textbf{基礎化}したのち,基礎ソルバーを用いて解集合を計算する.
%近年,SAT ソルバー技術を応用した高速な ASP システムが開発されている.
本稿で使用する高速 ASP システム
{\clingo}~\footnote{\url{https://potassco.org/}}
は,基礎化のためのグラウンダー{\gringo}と基礎ソルバー{\clasp}を
シームレスに結合したものである.
以降で示す論理プログラムのソースコードはすべて{\clingo}言語で書か
れており,表記上の対応については表~\ref{tbl:map}の通りである.

%%%%%%%%%%%%%%%%%%%%%%%%%%%%%%%%%
\begin{table}[t]
  \centering
  \caption{論理プログラムとソースコードの対応}
  \tabcolsep = 2mm
  \begin{tabular}{l|*{5}{c}}\small
    論理プログラム &  $\leftarrow$ & $,$      & $;$      & $\sim$    \\\hline
    ソースコード   &  \code{:-}    & \code{,} & \code{;} & \code{not}
  \end{tabular}
  \label{tbl:map}
\end{table}
%%%%%%%%%%%%%%%%%%%%%%%%%%%%%%%%%
%%%%%%%%%%%%%%%%%%%%%%%%%%%%%%%%%
\lstinputlisting[float=t,caption={%
図~\ref{fig:graph}のグラフの ASP ファクト表現 (\code{graph.lp})},%
captionpos=b,frame=single,label=code:graph.lp,%
xrightmargin=1zw,% 
xleftmargin=1zw,% 
numbersep=5pt,%
numbers=none,%
breaklines=true,%
columns=fullflexible,keepspaces=true,%
basicstyle=\ttfamily\scriptsize]{code/graph.lp}
%%%%%%%%%%%%%%%%%%%%%%%%%%%%%%%%%
%%%%%%%%%%%%%%%%%%%%%%%%%%%%%%%%%
\lstinputlisting[float=t,caption={%
色数 \code{c} のグラフ点彩色問題を解く論理プログラム (\code{color.lp})},%
captionpos=b,frame=single,label=code:color.lp,%
xrightmargin=1zw,% 
xleftmargin=1zw,% 
numbersep=5pt,%
numbers=left,%
breaklines=true,%
columns=fullflexible,keepspaces=true,%
basicstyle=\ttfamily\scriptsize]{code/color.lp}
%%%%%%%%%%%%%%%%%%%%%%%%%%%%%%%%%
\lstinputlisting[float=t,caption={%
{\clingo}の実行例},%
captionpos=b,frame=single,label=code:color.log,%
numbers=none,%
breaklines=true,%
columns=fullflexible,keepspaces=true,%
basicstyle=\ttfamily\scriptsize]{code/color.log}
%%%%%%%%%%%%%%%%%%%%%%%%%%%%%%%%%
%%%%%%%%%%%%%%%%%%%%%%%%%%%%%%%%%
\lstinputlisting[float=t,caption={%
基礎化された論理プログラム},%
captionpos=b,frame=single,label=code:color_ground.lp,%
xrightmargin=1zw,% 
xleftmargin=1zw,% 
numbersep=5pt,%
numbers=left,%
breaklines=true,%
columns=fullflexible,keepspaces=true,%
basicstyle=\ttfamily\scriptsize]{code/color_ground.lp}
%%%%%%%%%%%%%%%%%%%%%%%%%%%%%%%%%

ASP を用いた基本的な問題解法は,
最初に解きたい問題を論理プログラムとして表現する.
つぎにASP システムを用いて論理プログラムの解集合を計算する.
最後に解集合を解釈してもとの問題の解を得る,
という3つのステップからなる.
以下,グラフ点彩色問題を例にとって,各ステップごとに説明する.

図~\ref{fig:graph}の無向グラフを,ASP のファクト形式で表したものをコー
ド~\ref{code:graph.lp}に示す.
アトム\code{node/1}は頂点を表し,
アトム\code{edge/2}は辺を表している.
ピリオド(``\code{.}'')はルールの終わりを表す終端記号である.

グラフ点彩色問題を解く論理プログラムを,
コード~\ref{code:color.lp}に示す.
コード中の\code{c}は色数を表す定数であり,実際の値は実行時に
{\clingo}のオプションとして与えられる.
1行目のアトム\code{col/1}は色を表し,
\code{col(1..c).}は\code{col(1).}, \code{col(2).}, \ldots,
\code{col(c).}と書くことに等しい.
%
2行目のルールは,頂点\code{X}が色\code{C}で塗られることを意
味するアトム\code{color(X,C)}を導入し,
個数制約を用いて「各頂点は1つの色で塗られる」という制約を表している.
% セミコロン(\code{:})は条件付きリテラ
% ルと呼ばれる拡張構文であり,このルールのヘッドは,
% \code{1 \{ color(X,r);color(X,b);color(X,g) \} 1}のように展開される.
3行目のルールは,一貫性制約と個数制約を使って「辺で結ばれた頂点\code{X}と
\code{Y}が同じ色\code{C}で塗られることはない」という制約を表している.

ASP システムは,
コード~\ref{code:graph.lp}のファクトと
コード~\ref{code:color.lp}の論理プログラム
から解集合を計算する.
コード~\ref{code:color.log}に{\clingo}の実行例を示す.
色\code{1}を赤,色\code{2}を青,色\code{3}を黄とすると,
得られた解集合\{
\code{color(1,1)},
\code{color(2,3)},
\code{color(3,1)},
\code{color(4,2)}\}
は,図~\ref{fig:graph}の右側の彩色を表す.
より詳しく言うと,
{\clingo}は,グラウンダー{\gringo}を用いて
コード~\ref{code:graph.lp}のファクトと
コード~\ref{code:color.lp}の論理プログラムを基礎化した後,
{\clasp}を用いて基礎化された論理プログラムの解集合を計算する.
基礎化されたルール集合をコード~\ref{code:color_ground.lp}に示す.
4--7行目が
コード~\ref{code:color.lp}の2行目に,
8--19行目が
コード~\ref{code:color.lp}の3行目に
対応している.

%%% Local Variables:
%%% mode: japanese-latex
%%% TeX-master: "paper"
%%% End:
