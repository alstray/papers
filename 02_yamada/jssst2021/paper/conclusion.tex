%%%%%%%%%%%%%%%%%%%%%%%%%%%%%%%%%%%%%%%%%%%%%%%%%%%%%%%%%% 
\section{結論} \label{chap:conclusion}
%%%%%%%%%%%%%%%%%%%%%%%%%%%%%%%%%%%%%%%%%%%%%%%%%%%%%%%%%%
本研究では,解集合プログラミングを用いた$k$彩色遷移問題の解法について述べた.
これにあたり,$k$彩色遷移問題の ASP 符号化を3種類提案した.
これらは基礎化後のルール数が異なり,ルール数を減らした符号化\code{vrc3}では
連結・非連結・部分的非連結のすべてで優位性を示した.
結果として,ルール数を少なくすることが有効であることが確認できた.

今後の課題として,より効率的な符号化の研究とベンチマーク環境の整備があげられる.
効率的な符号化の手法として,{\clingo}のインクリメンタルサーチモードを用いることを考えている.
インクリメンタルサーチでは,
遷移回数を$t-1$としたときの学習節を引き継いだ上で$t$以降を解くことができる.
本研究でもっとも有効性を示した符号化\code{vrc3}を基盤とし,
インクリメンタルサーチモードに対応させたい.

ベンチマーク環境の整備として,特に連結インスタンスの母数が少なかった.
このため,評価実験で比較的多くの連結インスタンスが存在したグラフ
myciel3やmyciel4を調査することを考えている.
また,彩色数ではなくより多い色数とすることで遷移制約が成立しやすくなると予想している.
このため,彩色数において全解数がもっとも少なかったグラフle450\_5cで調査を行うつもりである.


%%% Local Variables:
%%% mode: latex
%%% TeX-master: "paper"
%%% End:
