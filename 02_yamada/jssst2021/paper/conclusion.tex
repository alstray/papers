%%%%%%%%%%%%%%%%%%%%%%%%%%%%%%%%%%%%%%%%%%%%%%%%%%%%%%%%%% 
\section{おわりに}\label{chap:conclusion}
%%%%%%%%%%%%%%%%%%%%%%%%%%%%%%%%%%%%%%%%%%%%%%%%%%%%%%%%%%

本稿では,解集合プログラミング(ASP)を用いた組合せ遷移問題の解法として,
有界組合せ遷移を提案し,そのソルバーの実装方法について述べた.
提案手法の有効性を評価するために,
代表的な組合せ遷移問題の1つである$k$彩色遷移問題を用いた実験結果を行った.
その結果,インクリメンタル ASP 解法を用いた実装方式が
より多くの問題を高速に解くことができ,その有効性が確認できた.

今後の課題として,ソルバーの高速化と問題記述例の蓄積があげられる.


%%% Local Variables:
%%% mode: latex
%%% TeX-master: "paper"
%%% End:
