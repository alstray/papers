%%%%%%%%%%%%%%%%%%%%%%%%%%%%%%%%%%%%%%%%%%%%%%%%%%%%%%%%%% 
\section{おわりに}\label{chap:conclusion}
%%%%%%%%%%%%%%%%%%%%%%%%%%%%%%%%%%%%%%%%%%%%%%%%%%%%%%%%%%

本稿では,ASP に基づく組合せ遷移問題ソルバーの実装方式について述べた.
初めに,有界組合せ遷移問題を提案した.
これは,組合せ遷移問題の遷移系列の長さに制限を加えたものである.
次に,有界組合せ遷移問題に対応した2つの実装方式を提案した.
特に改良ソルバーはインクリメンタル ASP 解法を利用しており,
有効性が期待できる.
また,$k$彩色遷移問題を解く2つの符号化を提案した.
これらの符号化は,遷移制約の表現方法が異なる.

最後に,ソルバーの性能評価を行うための実験を行った.
ベンチマークには独自に作成した$k$彩色遷移問題(90問)
を利用した.
実験の結果,改良ソルバーが
より多くの問題を高速で解くことができ,有効性を示した.
また,基礎化後のルール数を抑えるようにした符号化が
良い性能を示した.

今後の課題として,改良ソルバーの高速化があげられる.
高速化の手法としては,アルゴリズムの刷新や
探索ヒューリスティクスの改良を考えている.
同時に,$k$彩色遷移問題などの個別のインスタンスの ASP 符号化の
蓄積・改良も考えている.
ASP 符号化の改良には,有効なヒント制約の追加が効果的だと考えている.
また,ベンチマークの増強も必要である.
特に到達可能なベンチマークの数が少ないため,
色数を彩色数より増やしたベンチマークを導入することを考えている.

%本稿では,解集合プログラミング(ASP)を用いた組合せ遷移問題の解法として,
%有界組合せ遷移を提案し,そのソルバーの実装方法について述べた.
%提案手法の有効性を評価するために,
%代表的な組合せ遷移問題の1つである$k$彩色遷移問題を用いた実験結果を行った.
%その結果,インクリメンタル ASP 解法を用いた実装方式が
%より多くの問題を高速に解くことができ,その有効性が確認できた.
%
%今後の課題として,ソルバーの高速化と問題記述例の蓄積があげられる.


%%% Local Variables:
%%% mode: latex
%%% TeX-master: "paper"
%%% End:
