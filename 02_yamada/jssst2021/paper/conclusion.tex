%%%%%%%%%%%%%%%%%%%%%%%%%%%%%%%%%%%%%%%%%%%%%%%%%%%%%%%%%% 
\section{おわりに}\label{chap:conclusion}
%%%%%%%%%%%%%%%%%%%%%%%%%%%%%%%%%%%%%%%%%%%%%%%%%%%%%%%%%%

本稿では,解集合プログラミング(ASP)を用いた組合せ遷移問題の解法として,
有界組合せ遷移を提案し,そのソルバーの実装方法について述べた.
有界組合せ遷移の有効性を評価するために,
代表的な組合せ遷移問題の1つである$k$彩色遷移問題をベンチマークとして,
提案ソルバーの性能評価を行った.
その結果,インクリメンタル ASP 解法を用いた改良ソルバーが
より多くの問題を高速に解くことができることを確認した.

組合せ遷移問題を解く汎用ソルバーの研究開発はまだ始まったばかりである.
今後の課題として,改良ソルバーの高速化があげられる.
高速化の手法としては,アルゴリズムの刷新や
探索ヒューリスティックスの改良を考えている.
%
集合被覆問題や独立点集合を基とする組合せ遷移問題,
15パズルなどの問題記述例の蓄積も重要な課題の一つである.
また,$k$彩色遷移問題の ASP 符号化の改良も考えている.

% また,ベンチマークの増強も必要である.
% 特に到達可能なベンチマークの数が少ないため,
% 色数を彩色数より増やしたベンチマークを導入することを考えている.

%本稿では,解集合プログラミング(ASP)を用いた組合せ遷移問題の解法として,
%有界組合せ遷移を提案し,そのソルバーの実装方法について述べた.
%提案手法の有効性を評価するために,
%代表的な組合せ遷移問題の1つである$k$彩色遷移問題を用いた実験結果を行った.
%その結果,インクリメンタル ASP 解法を用いた実装方式が
%より多くの問題を高速に解くことができ,その有効性が確認できた.
%
%今後の課題として,ソルバーの高速化と問題記述例の蓄積があげられる.


%%% Local Variables:
%%% mode: latex
%%% TeX-master: "paper"
%%% End:
