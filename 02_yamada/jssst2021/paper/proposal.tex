%%%%%%%%%%%%%%%%%%%%%%%%%%%%%%%%%%%%%%%%%%%%%%%%%%%%%%%%%% 
\chapter{$k$彩色遷移問題のASP符号化} \label{chap:proposal}
%%%%%%%%%%%%%%%%%%%%%%%%%%%%%%%%%%%%%%%%%%%%%%%%%%%%%%%%%% 

\section{ASP ファクト形式}
問題の入力として与えられるグラフ,色数,初期状態,目標状態のうち,
グラフ,初期状態,目標状態はASP ファクトとして与えられる.
図~\ref{fig:ex_graph}のグラフをファクト形式で表したものをコード~\ref{code:graph.lp}に示す.

\lstinputlisting[float=t,caption={%
図~\ref{fig:ex_graph}のファクト表現 (\code{graph.lp})},%
captionpos=b,frame=single,label=code:graph.lp,%
numbers=left,%
breaklines=true,%
columns=fullflexible,keepspaces=true,%
basicstyle=\ttfamily\scriptsize]{code/graph.lp}

1~2行目ではそれぞれ頂点の数と辺の数を宣言している.
4行目ではアトム\code{node}によって頂点1から頂点4を定義している.
6行目ではアトム\code{edge}によって辺を定義している.
\code{edge(X,Y)}は頂点\code{X}と頂点\code{Y}間に辺が存在することを意味する.

次に, 図~\ref{fig:recol_s},~\ref{fig:recol_g}の初期状態と目標状態をファクト形式で表したものを
コード~\ref{code:path.lp}に示す.

\lstinputlisting[float=t,caption={%
図~\ref{fig:recol_s},図~\ref{fig:recol_g}のファクト表現 (\code{path.lp})},%
captionpos=b,frame=single,label=code:path.lp,%
numbers=left,%
breaklines=true,%
columns=fullflexible,keepspaces=true,%
basicstyle=\ttfamily\scriptsize]{code/path.lp}

1行目ではアトム\code{start}によって初期状態を定義している.
\code{start(X,C)}は初期状態において頂点\code{X}が色\code{C}で塗られることを意味する.
2行目ではアトム\code{goal}によって目標状態を定義している.
\code{goal(X,C)}は目標状態において頂点\code{X}が色\code{C}で塗られることを意味する.

\section{ASP 符号化}
$k$彩色遷移問題の制約は主に基となった組合せ問題の制約と遷移制約からなる.
本研究では遷移制約の記述が異なる3種の符号化を提案した.
また,本来与えられる入力に加え遷移回数$t$を与えるようにした.
本来の問題では「グラフ$G$,色数$k$における初期状態から目標状態への経路の存在」を解いていたのに対し,
遷移回数を加えることで
「グラフ$G$,色数$k$における遷移回数$t$での初期状態から目標状態への経路の存在」を解くことになる.
ここで$t$の最大値を$T_{limit}$とする.
$T_{limit}$は遷移回数の最大値であり,
これは解空間グラフの頂点数$-1$,すなわち組合せ問題の全解数$-1$となる.
$t=1, 2, \dots T_{limit}$のすべてで経路が存在しないのであれば本来の問題でも経路は存在しない.
一方で,ある$t$で経路が存在するのであれば本来の問題でも経路は存在する.

初めに,基本となる\code{vrc1}符号化をコード~\ref{code:vrc1.lp}に示す.
\code{color(X,C,T)}は\code{T}回の遷移をした状態(以下ステップ\code{T})
において頂点\code{X}が色\code{C}で塗られることを意味する.
1行目では定数を用いて色\code{1}から色\code{c}までを定義している.
2行目では定数を用いてステップ\code{0}からステップ\code{t}までを定義している.
4~5行目では一貫性制約を用いて初期状態と目標状態を定義している.
4行目のルールは「\code{start(X,C)}ならば\code{color(X,C,0)}である」を意味する.
同様に,5行目のルールは「\code{goal(X,C)}ならば\code{color(X,C,t)}である」を意味する.
9~10行目は各ステップにおいてグラフ点彩色問題の定義を満たすためのルールを記述している.
9行目のルールは「各ステップにおいて,各頂点は一つの色で塗られる」を意味する.
10行目のルールは
「各ステップにおいて,辺で結ばれた頂点\code{X}と\code{Y}は同じ色で塗られない」を意味する.
14~15行目は遷移制約を表したルールである.
「ステップ\code{T-1}と\code{T}($1 \leq T \leq t$)において,
ある2つの頂点\code{X}と\code{Y}が変化することはない」
を意味する.

\lstinputlisting[float=t,caption={%
$k$彩色遷移問題の論理プログラム (\code{vrc1.lp})},%
captionpos=b,frame=single,label=code:vrc1.lp,%
numbers=left,%
breaklines=true,%
columns=fullflexible,keepspaces=true,%
basicstyle=\ttfamily\scriptsize]{code/vartex_recoloring1.lp}

次に,コード~\ref{code:vrc1.lp}に新しいアトムを加え,遷移制約に関するルールを変更した
\code{vrc2}符号化と\code{vrc3}符号化をそれぞれコード~\ref{code:vrc2.lp}とコード~\ref{code:vrc3.lp}に示す.
これらのコードは,1~13行目はコード~\ref{code:vrc1.lp}と同じである.

\lstinputlisting[float=t,caption={%
$k$彩色遷移問題の論理プログラム (\code{vrc2.lp})},%
captionpos=b,frame=single,label=code:vrc2.lp,%
numbers=left,%
breaklines=true,%
columns=fullflexible,keepspaces=true,%
basicstyle=\ttfamily\scriptsize]{code/vartex_recoloring2.lp}

\lstinputlisting[float=t,caption={%
$k$彩色遷移問題の論理プログラム (\code{vrc3.lp})},%
captionpos=b,frame=single,label=code:vrc3.lp,%
numbers=left,%
breaklines=true,%
columns=fullflexible,keepspaces=true,%
basicstyle=\ttfamily\scriptsize]{code/vartex_recoloring3.lp}

コード~\ref{code:vrc2.lp}ではアトム\code{changed}を使用している.
14~15行目が遷移制約を表すルールである.
\code{changed(X,T)}は
「ステップ\code{T-1}とステップ\code{T}において頂点\code{X}の色が異なる」を意味し,
14行目でそれを定義している.
15行目では一貫性制約と個数制約を用いて
「各ステップで色が変化する頂点は一つ」という制約を表している.

コード~\ref{code:vrc3.lp}ではアトム\code{unchanged}を使用している.
14~15行目が遷移制約を表すルールである.
\code{unchanged(X,T)}は
「ステップ\code{T-1}とステップ\code{T}において頂点\code{X}の色が同じ」を意味し,
14行目でそれを定義している.
15行目では一貫性制約と個数制約を用いて
「各ステップで色が変化しない頂点は\code{N-1}個」という制約を表している.
ここで,\code{N}はグラフ$G$の頂点数である.

各符号化における遷移制約を表す基礎化後のルール数を表~\ref{tab:rule}に示す.
基礎化とは,各アトムの変数に代入を行い命題変数とすることである.
基礎化後のルール数は基となるアトムの変数の組合せより求められる.
本研究で用いる符号化の場合,$($色(\code{C})の選択方法の数$)*($
頂点(\code{V})の選択方法の数$)*($遷移回数(\code{T})の選択方法の数$)$となる.
表~\ref{tab:rule}において,$|T|$は遷移回数,$|C|$は色数,$|V|$はグラフ$G$の頂点数を表す.
アトムを追加した\code{vrc2},\code{vrc3}ではルール数が減少していることがわかる.
特に\code{vrc3}はルール数が最少であり,
大規模な問題に対する有効性が期待できる.

\begin{table}[tb]
  \centering
  \caption{各符号化の遷移制約を表す基礎化後のルール数}
  \begingroup
\renewcommand{\arraystretch}{1.5}
\begin{tabular}{l|c}\hline
  符号化名 & 基礎化後のルール数(遷移制約) \\ \hline
  origin & $4|T|{|C|\choose 2}^{2}{|V|\choose 2}$ \\ \hline
  changed & $|T|\left(2{|C|\choose 2}|V| + 1\right)$ \\ \hline
  unchanged & $|T|(|CV| + 1)$ \\ \hline
\end{tabular}
\endgroup
  \label{tab:rule}
\end{table}

%%% Local Variables:
%%% mode: latex
%%% TeX-master: "paper"
%%% End:
