%%%%%%%%%%%%%%%%%%%%%%%%%%%%%%%%%%%%%%%%%%%%%%%%%%%%%%%%%% 
\section{ASPを用いた組合せ遷移問題の解法}\label{chap:proposal}
%%%%%%%%%%%%%%%%%%%%%%%%%%%%%%%%%%%%%%%%%%%%%%%%%%%%%%%%%% 

ASP 技術を用いた組合せ遷移問題の解法として,
\textbf{有界組合せ遷移}(Bounded Combinatorial Reconfiguration)
を提案し,そのソルバーの実装方法について述べる.

%%%%%%%%%%%%%%%%%%%%%%%%%%%%%%%%%
\subsection{有界組合せ遷移}
%%%%%%%%%%%%%%%%%%%%%%%%%%%%%%%%%

有界組合せ遷移では,組合せ遷移問題に対して,制限された長さの
遷移系列を論理プログラムとして記号的に表現し,その論理プログラ
ムを ASP システムで実行することにより,到達可能性の検査を行う.
ASP システムが充足可能と判定した場合,
制限された長さの到達可能な遷移系列が存在することを意味する.
逆に充足不能と判定した場合,制限された長さでは到達不能である
ことを意味する.この場合,遷移系列の長さを増加させた論理プログラム
を構成し,再び ASP システムによる実行を繰り返す.
有界組合せ遷移は,到達可能な遷移系列を探すだけで,
到達不能の証明は行わない不完全な手続きであるが,検査すべき遷移系列の長
さに上限が存在する場合には,完全な手続きとなる.
例えば,基の組合せ問題の直径 (任意の2つの実行可能解の最短経路のうちの
最大長)を上限とすればよい~\footnote{%
理論的には,問題の直径を超えるまで手続きを繰り返せばよいが,
実用的な問題ではしばしば直径が大きすぎ,現実的には適用できない場合もある.}.

組合せ遷移問題が与えられたとき,
基の組合せ問題の変数集合
$\bm{x} = \{x_1,x_2,\ldots,x_n\}$,
各遷移ステップ$t\geq 0$に対して,
ステップ$t$での各変数の値を表す変数集合
$\bm{x}^{t} = \{x_1^t,x_2^t,\ldots,x_n^t\}$
を導入する.
ここで,
スタート状態の制約を表す論理式を$S(\bm{x})$,
基の組合せ問題の制約を表す論理式を$C(\bm{x)}$,
遷移制約を表す論理式を$T(\bm{x},\bm{x}')$,
ゴール状態の制約を表す論理式を$G(\bm{x)}$とする.
スタート状態から$\ell$ステップ遷移した後の各変数の値
$\bm{x}^{\ell}$は,
\(
S(\bm{x}^0) \land 
\bigwedge_{t=0}^{\ell} C(\bm{x}^t) \land
\bigwedge_{t=1}^{\ell} T(\bm{x}^{t-1},\bm{x}^{t})
\)
を満たす.
そこで,ゴール状態の制約を満たすかどうか判定するため,
$G(\bm{x}^\ell)$を付け加えた論理式$\varphi_{\ell}$を構成する.
\begin{adjustvboxheight}
\begin{align*}
  \varphi_{\ell} &= S(\bm{x}^0) \nonumber\\
  &\land \bigwedge_{t=0}^{\ell} C(\bm{x}^t) 
  \land \bigwedge_{t=1}^{\ell} T(\bm{x}^{t-1},\bm{x}^{t}) \label{BoCoRe:phi}\\
  &\land G(\bm{x}^\ell) \nonumber
\end{align*}
\end{adjustvboxheight}
与えられたステップ長$\ell$に対し$\varphi_\ell$が充足可能であれば,
到達可能な遷移系列が得られる.

%%%%%%%%%%%%%%%%%%%%%%%%%%%%%%%%%
\subsection{有界組合せ遷移の基本ソルバー} \label{sec:based_solver}
%%%%%%%%%%%%%%%%%%%%%%%%%%%%%%%%%

ASP システムを用いて有界組合せ遷移を行う手続きは以下のようになる.
\begin{enumerate}
\item ステップ長$\ell$を$\ell=0$とする.
\item \label{BoCoRe:base:solver:2}
  $\varphi_\ell$を論理プログラムとして記述し,
  ファクト形式のCoRe問題インスタンスといっしょに,
  ASP システムに入力として与える.
\item ASP システムの出力結果が充足可能であれば終了する.
  充足不能であれば$\ell$の値を1増加させ,
  \ref{BoCoRe:base:solver:2}に戻り手続きを繰り返す.
  ただし,
  $\ell$が基の問題の直径を超えたところで繰り返しを停止できる.
\end{enumerate}
この手続は,与えられた組合せ遷移問題に対して,
到達可能な最短の遷移系列を探索する.
一般に,到達不能の証明は行えない不完全な手続きであるが,
検査すべきステップ長$\ell$に上限が存在する場合には完全な手続きとなり,
到達不能も判定できる.この手続きを,単純に実装したプログラムを
\textbf{基本ソルバー}と呼ぶことにする.

%%%%%%%%%%%%%%%%%%%%%%%%%%%%%%%%%
\lstinputlisting[float=t,caption={%
$k=4$彩色遷移問題のファクト表現 (\code{graph_reconfig.lp})},%
captionpos=b,frame=single,label=code:graph_reconfig.lp,%
xrightmargin=1zw,% 
xleftmargin=1zw,% 
numbersep=5pt,%
numbers=left,%
breaklines=true,%
columns=fullflexible,keepspaces=true,%
basicstyle=\ttfamily\scriptsize]{code/graph_reconfig.lp}
%%%%%%%%%%%%%%%%%%%%%%%%%%%%%%%%%

$k$彩色遷移問題を例にとって,基本ソルバーを用いた問題解法を説明する.
図~\ref{fig:gcrp}の$k=4$彩色遷移問題の ASP ファクト表現を
コード~\ref{code:graph_reconfig.lp}に示す.
コード~\ref{code:graph.lp}のグラフ点彩色問題のファクト表現と比較して,
スタート状態を表す
\code{start/2}と
ゴール状態を表す
\code{goal/2}が追加されている.
ここで,色\code{1}は赤,色\code{2}は青,色\code{3}は黄,
色\code{4}は緑とする.
例えば,\code{start(1,4).}は,スタート状態($t=0$)において頂点\code{1}
が色\code{4} (緑)で塗られることを表す.

コード\ref{code:gcrp_cc_changed.lp}に,
$k$=\code{c}彩色遷移問題を解く論理プログラムとして\code{changed}符号化を示す.
コード中の\code{c} は色数,\code{length} はステップ長を表す.
1行目のアトム\code{t/1}はステップ数を表す.
% \code{t(0..length).}は\code{t(0).}, \code{t(1).}, \ldots,
% \code{t(length).}に展開される.
アトム\code{color(X,C,T)}は,ステップ\code{T}において,頂点\code{X}が
色\code{C}で塗られることを意味する.
3行目はスタート状態の制約を表す.
5--7行目は基のグラフ点彩色問題の制約を表す.
9--11行目は遷移制約を表す.
9行目の補助アトム\code{changed(X,T)}は,
ステップ\code{T-1}と\code{T}の間で頂点\code{X}の色が
変化したことを意味する.
11行目のルールは,個数制約を用いて
「各ステップ\code{T}において色が変化する頂点はただ1つのみ」という制約
を表している.

コード\ref{code:gcrp_cc_unchanged.lp}に,遷移制約の表現方法が異なる
\code{unchanged}符号化を示す.
\code{changed}符号化との違いは,9--10行目のルールだけである.
9行目の補助アトム\code{unchanged(X,T)}は,
ステップ\code{T-1}と\code{T}の間で頂点\code{X}の色が
変化しないことを意味する.
10行目のルールは,個数制約を用いて
「各ステップ\code{T}において色が変化しない頂点は\code{N-1}個」
という制約を表している.
この\code{unchanged}符号化の特徴は,
\code{changed}符号化と比較して,
基礎化後のルール数が少なく抑えられる点である.

基本ソルバーは,
CoRe問題インスタンスのファクト表現と,CoRe問題を解く符号化を入力として,
到達可能性を検査する.
コード~\ref{code:graph_reconfig.log}に,
$k$彩色遷移問題に対する基本ソルバーの動作を示す.
この例より,
ステップ長\code{length}を0から3まで変化させ,{\clingo}を合計4回実行
することにより,最短の遷移系列が得られることがわかる.
\code{length=3}で得られた解集合は,図~\ref{fig:gcrp}に示した遷移系列に
対応している.

%%%%%%%%%%%%%%%%%%%%%%%%%%%%%%%%%
\lstinputlisting[float=t,caption={%
  $k=$\code{c} 彩色遷移問題を解く\code{changed}符号化 (\code{changed.lp}).
  定数\code{length} はステップ長を表す.
},%
captionpos=b,frame=single,label=code:gcrp_cc_changed.lp,%
xrightmargin=1zw,% 
xleftmargin=1zw,% 
numbersep=5pt,%
numbers=left,%
breaklines=true,%
columns=fullflexible,keepspaces=true,%
basicstyle=\ttfamily\scriptsize]{code/gcrp_cc_changed.lp}
%%%%%%%%%%%%%%%%%%%%%%%%%%%%%%%%%

%%%%%%%%%%%%%%%%%%%%%%%%%%%%%%%%%
\lstinputlisting[float=t,caption={%
  $k=$\code{c} 彩色遷移問題を解く\code{unchanged}符号化 (\code{unchanged.lp}).
  定数\code{length} はステップ長を表す.
},%
captionpos=b,frame=single,label=code:gcrp_cc_unchanged.lp,%
xrightmargin=1zw,% 
xleftmargin=1zw,% 
numbersep=5pt,%
numbers=left,%
breaklines=true,%
columns=fullflexible,keepspaces=true,%
basicstyle=\ttfamily\scriptsize]{code/gcrp_cc_unchanged.lp}
%%%%%%%%%%%%%%%%%%%%%%%%%%%%%%%%%

%%%%%%%%%%%%%%%%%%%%%%%%%%%%%%%%% 
\lstinputlisting[float=t,caption={%
{\clingo}の実行例},%
captionpos=b,frame=single,label=code:graph_reconfig.log,%
numbers=none,%
breaklines=true,%
columns=fullflexible,keepspaces=true,%
basicstyle=\ttfamily\scriptsize]{code/graph_reconfig.log}
%%%%%%%%%%%%%%%%%%%%%%%%%%%%%%%%%

%%%%%%%%%%%%%%%%%%%%%%%%%%%%%%%%%%%%%%%%%%%%%%%%%%%%%%%%%%%%%%%%%%
\subsection{ソルバーの改良} \label{sec:improved_solver}
%%%%%%%%%%%%%%%%%%%%%%%%%%%%%%%%%%%%%%%%%%%%%%%%%%%%%%%%%%%%%%%%%%

%%%%%%%%%%%%%%%%%%%%%%%%%%%%%%%%%
  \thicklines
  \setlength{\unitlength}{1.28pt}
  \small
  \begin{picture}(280,57)(4,-10)
    \put(  0, 20){\dashbox(50,24){\shortstack{根付き全域森\\問題}}}
    \put( 60, 20){\framebox(50,24){変換器}}
    \put(120, 20){\dashbox(50,24){\shortstack{ASPファクト}}}
    \put(120,-10){\alert{\bf\dashbox(50,24){\scriptsize{\shortstack{ASP符号化\\(論理プログラム)}}}}}
    \put(180, 20){\framebox(50,24){ASPシステム}}
    \put(240, 20){\dashbox(50,24){\shortstack{根付き全域森\\問題の解}}}
    \put( 50, 32){\vector(1,0){10}}
    \put(110, 32){\vector(1,0){10}}
    \put(170, 32){\vector(1,0){10}}
    \put(230, 32){\vector(1,0){10}}
    \put(170, +2){\line(1,0){4}}
    \put(174, +2){\line(0,1){30}}
  \end{picture}  

%%%%%%%%%%%%%%%%%%%%%%%%%%%%%%%%%

有界組合せ遷移は,
ステップ長$\ell$を増やしながら,複数の$\varphi_\ell$を
繰り返し解く必要がある.
しかし,各$\varphi_\ell$に含まれる制約の大部分は同じであるため,
実際に$\varphi_{\ell -1}$と$\varphi_{\ell}$の差をみてみると,
\[
  \varphi_{\ell -1} \oplus \varphi_{\ell} = 
    \{C(\bm{x}^{\ell}), T(\bm{x}^{\ell -1}, 
    \bm{x}^{\ell}), G(\bm{x}^{\ell -1}), G(\bm{x}^{\ell})\}
\]
であり,多くの制約は共通していることがわかる.
そのため,各$\varphi_\ell$に対して
毎回 ASP システムを実行する基本ソルバーは,
同一の探索空間を何度も調べることになり求解効率が低下する.
また,毎回$\varphi_\ell$全体を基礎化するオーバーヘッドも無視できない.

この問題を解決するために,有界モデル検査やプランニングの実装で使われる
インクリメンタル ASP 解法を用いる~\cite{GebserKKS19}.
この解法は,ルールを追加・削除した一連の論理プログラムを,
同様の探索失敗を防ぐために獲得した学習節を再利用しながら,
連続的に解くための手法である.

インクリメンタル ASP 解法を用いた改良ソルバーは,
ステップ長$\ell$を増やしながら,
段階的に$\varphi_\ell$を構成し,
獲得した学習節を再利用した効率のよい到達可能性の検査を可能とする.
改良ソルバーの構成を図~\ref{fig:arch}に示す.

% %%%%%%%%%%%%%%%%%%%%%%%%%%%%%%%%% 
% \lstinputlisting[float=t,caption={%
% {\clingo}の Python API を用いた有界組合せ遷移ソルバーの実装 (\code{core.lp})},%
% captionpos=b,frame=single,label=code:core.lp,%
% xrightmargin=1zw,% 
% xleftmargin=1zw,% 
% numbersep=5pt,%
% numbers=left,%
% breaklines=true,%
% columns=fullflexible,keepspaces=true,%
% basicstyle=\ttfamily\scriptsize]{code/core.lp}
%%%%%%%%%%%%%%%%%%%%%%%%%%%%%%%%% 

%%%%%%%%%%%%%%%%%%%%%%%%%%%%%%%%% 
\begin{figure*}[t]
\centering  
\begin{lstlisting}[
frame=single,
xrightmargin=1zw,% 
xleftmargin=1zw,% 
numbersep=5pt,%
numbers=left,%
breaklines=true,%
columns=fullflexible,keepspaces=true,%
basicstyle=\ttfamily\footnotesize]
#script (python) 
import clingo

def get(val, default):
    return val if val != None else default

def main(prg):
    core_stop  = get(prg.get_const("core_stop"), clingo.String("SAT"))
    core_max   = prg.get_const("core_max")

    step, ret = 0, None
    while ((core_max is None or step <= core_max.number) and
            (step == 0 or (core_stop.string == "SAT" and not ret.satisfiable) or
            (core_stop.string == "UNSAT" and not ret.unsatisfiable))):    
        parts = []
        if step > 0:
            prg.release_external(clingo.Function("query", [step-1]))
            prg.cleanup()
        else:
            parts.append(("base", []))
        parts.append(("step", [step]))
        parts.append(("check", [step]))
        prg.ground(parts)
        prg.assign_external(clingo.Function("query", [step]), True)
        ret, step = prg.solve(), step+1

    print('c step: %s' % str(step-1))
    if (step != 0 and
        (core_stop.string != "SAT" or ret.satisfiable) and
        (core_stop.string != "UNSAT" or ret.unsatisfiable)):
        print('s REACHABLE')
    else:
        print('s UNREACHABLE')

#end.
#program check(t).
#external query(t).
\end{lstlisting}
\caption{{\clingo}の Python API を用いた改良ソルバーの実装 (\code{core.lp})}
\label{code:core.lp}
\end{figure*}
%%%%%%%%%%%%%%%%%%%%%%%%%%%%%%%%% 

改良ソルバーにおける有界組合せ遷移を行う手続き
(図~\ref{fig:arch}の\code{core.lp})は以下のようになる.
\begin{enumerate}
  \item ASP システムを起動する.
  \item ファクト形式のCoRe 問題インスタンスと,
    $S$を表すサブプログラム,
    $C$と$T$を表すサブプログラム,
    $G$を表すサブプログラムから構成される論理プログラムを与える.
  \item ステップ長$\ell$を$\ell=0$とする.
  \item \label{BoCoRe:improved:solver:4}
    $\ell > 0$であれば,$G(\bm{x}^{\ell -1})$を表すルールを削除する.
  \item \label{BoCoRe:improved:solver:5}
    $\ell=0$であれば,$S(\bm{x}^0)$を表すルールを追加する.
  \item \label{BoCoRe:improved:solver:6}
    $C(\bm{x}^{\ell})$と$T(\bm{x}^{\ell -1}, \bm{x}^{\ell})$を表すルー
    ルを追加する.
  \item \label{BoCoRe:improved:solver:7}
    $G(\bm{x}^{\ell})$を表すルールを追加する.
  \item ASP システムを用いて到達可能性を検査する.
  \item ASP システムの出力結果が充足可能であれば終了する.
        充足不能であれば$\ell$の値を増加させ,
        \ref{BoCoRe:improved:solver:4}に戻り手続きを繰り返す.
        ただし,$\ell$が基の問題の直径を超えたところで繰り返しを停止できる.
  \item ASP システムを終了する.
\end{enumerate}
到達不能の証明については,\ref{sec:based_solver}節で述べた基本ソルバー
の手続きと同じである.

図~\ref{code:core.lp}に,
{\clingo}の Python インターフェースを用いて上記の手続きを実装した
改良ソルバーを示す.
%
変数\code{core_max}はステップ長$\ell$の上限値を表す.
\code{core_stop}はプログラムの停止条件表す変数であり,
基本的には\code{SAT}を指定する.
11行目で定義されている\code{step}は$\ell$に対応する.
% 図~\ref{code:core.lp}では,
% \code{step}が$t$の時に追加する制約を列挙してから,
% ひとまとめに基礎化している.
16--18行目は手続き\ref{BoCoRe:improved:solver:4}に対応している.
19--20行目は手続き\ref{BoCoRe:improved:solver:5}に対応している.
21行目は手続き\ref{BoCoRe:improved:solver:6}に対応している.
22行目は手続き\ref{BoCoRe:improved:solver:7}に対応している.
23行目でステップ数$\ell$で追加されるルールの基礎化を行い,
25行目の\code{prg.solve()}で到達可能性を検査している.

%%%%%%%%%%%%%%%%%%%%%%%%%%%%%%%%%%%%%%%
\lstinputlisting[float=t,caption={%
$k$=\code{c} 彩色遷移問題を解く\code{changed_inc} 符号化 (\code{changed_inc.lp})},%
captionpos=b,frame=single,label=code:gcrp_cc_changed_inc.lp,%
xrightmargin=1zw,% 
xleftmargin=1zw,% 
numbersep=5pt,%
numbers=left,%
breaklines=true,%
columns=fullflexible,keepspaces=true,%
basicstyle=\ttfamily\scriptsize]{code/gcrp_cc_changed_inc.lp}
%%%%%%%%%%%%%%%%%%%%%%%%%%%%%%%%%%%%%%%

%%%%%%%%%%%%%%%%%%%%%%%%%%%%%%%%%%%%%%%
\lstinputlisting[float=t,caption={%
$k$=\code{c} 彩色遷移問題を解く\code{unchanged_inc} 符号化 (\code{unchanged_inc.lp})},%
captionpos=b,frame=single,label=code:gcrp_cc_unchanged_inc.lp,%
xrightmargin=1zw,% 
xleftmargin=1zw,% 
numbersep=5pt,%
numbers=left,%
breaklines=true,%
columns=fullflexible,keepspaces=true,%
basicstyle=\ttfamily\scriptsize]{code/gcrp_cc_unchanged_inc.lp}
%%%%%%%%%%%%%%%%%%%%%%%%%%%%%%%%%%%%%%%

%%%%%%%%%%%%%%%%%%%%%%%%%%%%%%%%% 
\lstinputlisting[float=t,caption={%
提案ソルバーの実行例},%
captionpos=b,frame=single,label=code:graph_reconfig_inc.log,%
numbers=none,%
breaklines=true,%
columns=fullflexible,keepspaces=true,%
basicstyle=\ttfamily\scriptsize]{code/graph_reconfig_inc.log}
%%%%%%%%%%%%%%%%%%%%%%%%%%%%%%%%%

改良ソルバーの入力となる論理プログラムは,
\code{base}, \code{step(t)}, \code{check(t)}
の3つのパートから構成される.
\code{base} パートには,スタート状態の制約$S$を記述する.
\code{step(t)} パートには,各ステップ$0\leq$~\code{t}~$\leq \ell$
において満たすべき基の組合せ問題の制約$C$および遷移制約$T$を記述する.
\code{check(t)} パートには,ゴール状態の制約$G$を記述する.
%
コード~\ref{code:gcrp_cc_changed_inc.lp}と
コード~\ref{code:gcrp_cc_unchanged_inc.lp}に,
$k$=\code{c}彩色遷移問題を解く2種類の論理プログラムを示す.
これらは,それぞれ
コード~\ref{code:gcrp_cc_changed.lp}と
コード~\ref{code:gcrp_cc_unchanged.lp}
の論理プログラムに対応している.
改良ソルバーの実行例をコード~\ref{code:graph_reconfig_inc.log}に示す.
得られた解集合は,図~\ref{fig:gcrp}の遷移系列に対応している.

%%% Local Variables:
%%% mode: latex
%%% TeX-master: "paper"
%%% End:
