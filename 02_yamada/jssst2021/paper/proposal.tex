%%%%%%%%%%%%%%%%%%%%%%%%%%%%%%%%%%%%%%%%%%%%%%%%%%%%%%%%%% 
\section{解集合プログラミングに基づく組合せ遷移ソルバー} \label{chap:proposal}
%%%%%%%%%%%%%%%%%%%%%%%%%%%%%%%%%%%%%%%%%%%%%%%%%%%%%%%%%% 
  \thicklines
  \setlength{\unitlength}{1.28pt}
  \small
  \begin{picture}(280,57)(4,-10)
    \put(  0, 20){\dashbox(50,24){\shortstack{根付き全域森\\問題}}}
    \put( 60, 20){\framebox(50,24){変換器}}
    \put(120, 20){\dashbox(50,24){\shortstack{ASPファクト}}}
    \put(120,-10){\alert{\bf\dashbox(50,24){\scriptsize{\shortstack{ASP符号化\\(論理プログラム)}}}}}
    \put(180, 20){\framebox(50,24){ASPシステム}}
    \put(240, 20){\dashbox(50,24){\shortstack{根付き全域森\\問題の解}}}
    \put( 50, 32){\vector(1,0){10}}
    \put(110, 32){\vector(1,0){10}}
    \put(170, 32){\vector(1,0){10}}
    \put(230, 32){\vector(1,0){10}}
    \put(170, +2){\line(1,0){4}}
    \put(174, +2){\line(0,1){30}}
  \end{picture}  


本研究では,解集合プログラミングに基づく組合せ遷移ソルバーを2種類提案する.
提案するソルバーの構成図を図\ref{fig:arch}に示す.
それぞれのソルバーは組合せ遷移ソルバー内の仕組みが異なる.
提案手法では,組合せ遷移問題の基本的な入力である
基となる組合せ問題と二つの実行可能解に加え,ステップ長$l$も与える.
本来の問題では
「ある組合せ問題における,
与えられた二つの実行可能解の到達可能性」
を解くのに対し,ステップ長を加えた場合は
「ある組合せ問題における,
ステップ長$l$での与えられた二つの実行可能解の到達可能性」
を解くことになる.
基となる組合せ問題の実行可能解の総数を$M$とすると,
$0 \le l \le M-1$として問題を解くことで
元の組合せ遷移問題の到達可能性が求まる.
一つでも到達可能である$l$が存在すれば元の問題でも到達可能となる.
また,すべての$l$で到達不能であれば元の問題でも到達不能となる.

\subsection{基本ソルバー} \label{sec:based_solver}

提案する基本ソルバーは,組合せ遷移問題をASPファクト形式で表したファイルと
問題を解くための ASP 符号化,及び$M$が与えられたとき,
ステップ長$l$の値を変化させならがASPシステムを繰り返し起動し
問題を解き到達可能性を出力する.
組合せ遷移問題を解く符号化の例として,
$k$彩色遷移問題を解く2種類の符号化
\code{changed}と\code{unchanged}をそれぞれ
コード\ref{code:gcrp_cc_changed.lp}と
コード\ref{code:gcrp_cc_unchanged.lp}に示す.

%コード名にccを入れる?入れない?
\lstinputlisting[float=t,caption={%
基本ソルバーにおいて$k$彩色遷移問題を解く\code{changed}符号化 (\code{gcrp_cc_changed.lp})},%
captionpos=b,frame=single,label=code:gcrp_cc_changed.lp,%
xrightmargin=1zw,% 
xleftmargin=1zw,% 
numbersep=5pt,%
numbers=left,%
breaklines=true,%
columns=fullflexible,keepspaces=true,%
basicstyle=\ttfamily\scriptsize]{code/gcrp_cc_changed.lp}

\lstinputlisting[float=t,caption={%
基本ソルバーにおいて$k$彩色遷移問題を解く\code{unchanged}符号化 (\code{gcrp_cc_unchanged.lp})},%
captionpos=b,frame=single,label=code:gcrp_cc_unchanged.lp,%
xrightmargin=1zw,% 
xleftmargin=1zw,% 
numbersep=5pt,%
numbers=left,%
breaklines=true,%
columns=fullflexible,keepspaces=true,%
basicstyle=\ttfamily\scriptsize]{code/gcrp_cc_unchanged.lp}

コード\ref{code:gcrp_cc_changed.lp}において,
1行目では定数を用いて色\code{1}から色\code{c}までを定義している.
2行目では定数を用いてステップ\code{0}からステップ\code{l}までを定義している.
5行目では一貫性制約を用いて初期状態を定義しており,
「\code{start(X,C)}ならば\code{color(X,C,0)}である」を意味する.
8~9行目では,各ステップにおいてグラフ点彩色問題の制約を満たすことを,
基数制約を用いて定義している.
8行目は「各ステップにおいて,各頂点は一つの色で塗られる」
を意味する.
9行目は「各ステップにおいて,
辺で結ばれた二つの頂点で色\code{C}で塗られる頂点は一つ以下」
を意味する.
12~13行目では遷移制約を定義している.
12行目で定義している\code{changed(X, T)}は,
「ステップ\code{T-1}とステップ\code{T}で頂点\code{X}の色が
変化した」を意味する.
13行目では一貫性制約と個数制約を用いて
「各ステップで色が変化する頂点は一つ」という制約を表している.
16行目では一貫性制約を用いて目標状態を定義しており,
「\code{goal(X,C)}ならば\code{color(X,C,l)}である」を意味する.

コード\ref{code:gcrp_cc_unchanged.lp}は,
遷移制約の表し方がコード\ref{code:gcrp_cc_changed.lp}
と異なる.
12行目で定義している\code{unchanged(X, T)}は,
「ステップ\code{T-1}とステップ\code{T}で頂点\code{X}の色が
変化しなかった」を意味する.
13行目では一貫性制約と個数制約を用いて
「各ステップで色が変化しない頂点は\code{N-1}個」という制約を表している.
ここで\code{N}はグラフの頂点数である.

基本ソルバーでは,ASP システムの起動や
ルールの\textbf{基礎化}による
オーバーヘッドが大きいという問題がある.
基礎化とは,一階述語論理に代入を行い
命題論理へと変換することをいう.
コード\ref{code:gcrp_cc_changed.lp}と
コード\ref{code:gcrp_cc_unchanged.lp}において,
ある$l$で生成される節集合から
目標状態を定義する制約を除いたものを
$C(l)$とすると,
\begin{align}
  C(l) \: \in \: C(l') \: (l \: < \: l')
\end{align}
となるため,同じ節を何度も生成することになる.

\subsection{改良ソルバー}

\begin{figure}[tb]
  \centering
  \begin{tabular}{l}\hline
    \textbf{Algorythm} improved solver\\\hline
    %~1: input: a problem $P$ \\
    ~2: input: $stop$, $min$, $max$ \\
    ~3: launch ASP system $S$; \\
    ~4: $step := 0$; \\
    ~5: $ret := None$; \\
    ~6: \bf{repeat} \\
    ~7: \quad \quad generate a empty list $parts$; \\
    ~8: \quad \quad $append\_list(parts, "check", step)$; \\
    ~9: \quad \quad $append\_list(parts, "step", step)$; \\
    10: \quad \quad \textbf{if} $step > 0$ \textbf{then} \\
    11: \quad \quad \quad \quad $assign\_false(S, "query", step-1)$; \\
    12: \quad \quad \textbf{else} \\
    13: \quad \quad \quad \quad $append\_list(parts, "base", step)$; \\
    14: \quad \quad \textbf{end if} \\
    15: \quad \quad $ground(S, parts)$; \\
    16: \quad \quad $assign\_true(S, "query", step)$; \\
    17: \quad \quad $ret := solve(S)$; \\
    18: \quad \quad $step := step+1$; \\
    19: \textbf{until} $step > max$ or $ret == stop$ and $step \ge min$\\
    20: close $S$; \\
    21: \textbf{if} $ret == stop$ and $step \ge min$ \textbf{then} \\
    22: \quad \quad \textbf{return} REACHABLE \\
    23: \textbf{else} \\
    24: \quad \quad \textbf{return} UNREACHABLE \\
    25: \textbf{end if} \\ \hline
  \end{tabular}
  \caption{改良ソルバーのアルゴリズム}
  \label{algo:inc_solver}
\end{figure}

\lstinputlisting[float=tb,caption={%
\clingo のPythonAPIを用いた改良ソルバー (\code{core.lp})},%
captionpos=b,frame=single,label=code:core.lp,%
xrightmargin=1zw,% 
xleftmargin=1zw,% 
numbersep=5pt,%
numbers=left,%
breaklines=true,%
columns=fullflexible,keepspaces=true,%
basicstyle=\ttfamily\scriptsize]{code/core.lp}

\ref{sec:based_solver}節で述べた問題点を改善するために
改良ソルバーを提案した.
改良ソルバーでは \clingo のインクリメンタルサーチモードを用いる.
改良ソルバーのアルゴリズムを図\ref{algo:inc_solver}に示す.
\begin{enumerate}
  \item 終了条件と,ステップ数の下限と上限を与える.
  \item ASPシステムを起動する.
  \item $step$と$ret$の値を初期化する.
  \item 空のリスト$parts$を生成する.
  $parts$は基礎化するものを記憶するためのものである.
  \item 論理プログラム内において,
  \code{check}ブロックと\code{step}ブロックに
  割り振られたステップ数$step$のルールを$parts$に加える.

  \begin{itemize}
    \item \code{check}ブロックには,主に目標状態に関するルールが記述される.
    \item \code{step}ブロックには,主に各ステップで満たされるべきルールが記述される.
  \end{itemize}

  \item $step \ge 0$のとき,ステップ数$step-1$で基礎化された
  命題変数\code{query(step-1)}を偽とする.そうでないとき,
  \code{base}ブロックに割り振られたルールを$parts$に加える.
  \begin{itemize}
    \item \code{query}は,すでに生成されている前の段階時点における目標状態
    に関するルールを無効化するためのものである.
  \end{itemize}

  \begin{itemize}
    \item \code{base}ブロックには,
    ステップ数に関わらず成り立つルールが記述される.
    \item ファクト形式のファイルなどの,ブロックが明記されてないルールも
    \code{base}ブロックとみなされる.
  \end{itemize}

  \item $parts$内のルールについて,$step$を代入し基礎化する.
  \item ステップ数$step$で基礎化された命題変数\code{query(step)}を
  真とする.
  \item ASP システムにより問題を解く.
  \item 終了条件を満たすまで,ステップ4~9を繰り返す.
  \item 結果を出力する.
  \item ASPシステムを終了する.
\end{enumerate}
$min = 0, max = l_{ub}$とすることで,基本ソルバーと同様に
到達可能性を判定できる.

コード\ref{code:core.lp}は,\clingo の PythonAPI を用いて実装したものである.
ステップ1とステップ11に該当する箇所は省略している.
17~21行目がステップ6に,22行目がステップ7に,23行目がステップ8に,
それぞれ対応している.

\lstinputlisting[float=tb,caption={%
基本ソルバーにおいて$k$彩色遷移問題を解く\code{unchanged}符号化 (\code{gcrp_cc_unchanged_inc.lp})},%
captionpos=b,frame=single,label=code:gcrp_cc_unchanged_inc.lp,%
xrightmargin=1zw,% 
xleftmargin=1zw,% 
numbersep=5pt,%
numbers=left,%
breaklines=true,%
columns=fullflexible,keepspaces=true,%
basicstyle=\ttfamily\scriptsize]{code/gcrp_cc_unchanged_inc.lp}


組合せ遷移問題を解く ASP 符号化も,
改良ソルバーに対応させる必要がある.
コード\ref{code:gcrp_cc_unchanged.lp}の\code{unchanged}符号化
を対応させたものをコード\ref{code:gcrp_cc_unchanged_inc.lp}
に示す.
変更点は以下の通りである.
\begin{itemize}
  \item 図\ref{algo:inc_solver}内の$max$などに対応する
  値を宣言する記述を追加した.(1行目)
  \item ステップを表すファクト\code{t(0..l)}を削除した.
  \item \code{base}などのブロックを宣言する記述を追加した.
  \item 目標状態を表すルールに\code{query(t)}を追加した.
\end{itemize}

%%% Local Variables:
%%% mode: latex
%%% TeX-master: "paper"
%%% End:
