%%%%%%%%%%%%%%%%%%%%%%%%%%%%%%%%%%%%%%%%%%%%%%%%%%%%%%%%%% 
\section{結論} \label{chap:conclusion}
%%%%%%%%%%%%%%%%%%%%%%%%%%%%%%%%%%%%%%%%%%%%%%%%%%%%%%%%%%
本研究では,ASP に基づく組合せ遷移問題ソルバーについて述べた.
初めに,有界組合せ遷移問題を提案した.
これは,遷移系列の長さに制限を加えた組合せ遷移問題である.
次に,有界組合せ遷移問題に対応した2種類のソルバーを提案した.
特に改良ソルバーは ASP システムの{\clingo}に備わる
インクリメンタルサーチモードを利用しており,
有効性が期待できる.
また,$k$彩色遷移問題を解く2種類の符号化を提案した.
これらの符号化は,遷移制約の表現方法が異なる.

最後に,ソルバーの性能評価を行うための実験を行った.
ベンチマークには独自に作成した90問の$k$彩色遷移問題
を利用した.
実験の結果,改良ソルバーは基本ソルバーと比べて40問以上多く
解くことができた.
また,\code{changed}符号化と\code{unchanged}符号化を
比べると,\code{unchanged}符号化のほうが
良い性能を示した.
これは基礎化によって生成される節数が少ないためと考えられる.

今後の課題として,改良ソルバーの高速化があげられる.
高速化の手法としては,アルゴリズムの刷新や
変数選択ヒューリスティクスの改良を考えている.
同時に,$k$彩色遷移問題などの個別のインスタンスの符号化の
改良も考えている.
インスタンスの改良には,有効なヒント制約の追加などを考えている.
また,ベンチマークの増強も必要である.
特に到達可能なインスタンスの数が少ないため,
色数を増やしたベンチマークを導入することを考えている.