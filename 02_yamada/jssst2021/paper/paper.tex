% 以下の3行は変更しないこと.
\documentclass[T]{compsoft}
\taikai{2021}
\pagestyle {empty}

\usepackage [dvipdfmx] {graphicx}

% ユーザが定義したマクロなどはここに置く.ただし学会誌のスタイルの
% 再定義は原則として避けること.

\begin{document}

% 論文のタイトル
\title{解集合プログラミングに基づく組合せ遷移ソルバーの実装方式に関する考察}

% 著者
% 和文論文の場合,姓と名の間には半角スペースを入れ,
% 複数の著者の間は全角スペースで区切る
%
\author{山田 悠也 湊 真一 番原 睦則
%
% ここにタイトル英訳 (英文の場合は和訳) を書く.
%
\ejtitle{A Study on ASP-based Implementation of Combinatorial Reconfiguration Solver}
%
% ここに著者英文表記 (英文の場合は和文表記) および
% 所属 (和文および英文) を書く.
% 複数著者の所属はまとめてよい.
%
\shozoku{Yuya Yamada, Mutsunori Banbara}{名古屋大学 大学院情報学研究科}%
{Graduate School of Informatics, Nagoya University}}

% 和文アブストラクト
\Jabstract{%
組合せ遷移問題とは,基となる組合せ問題とその二つの実行可能解が与えられ
たとき,一方の実行可能解から他方の実行可能解へ,遷移制約を満たしつつ,
実行可能解のみを経由して到達できるかを判定する問題である.組合せ遷移問
題の多くは PSPACE 完全になることが知られている.
本発表では,解集合プログラミング(ASP)に基づく組合せ遷移ソルバーの実装
方式について述べる.提案ソルバーは,ASP ファクト形式で記述された問題イ
ンスタンス,組合せ遷移問題を解く ASP 符号化,遷移系列の長さを入力とし,
ASPソルバーを繰り返し呼び出すことにより,問題の到達可能性を判定する.
到達可能の場合は,解として遷移系列を出力する.提案方式の性能評価として,
代表的な組合せ遷移問題の一つであるグラフ点彩色問題の遷移問題をベンチマー
クに用いた実験結果を示す.
}

\maketitle \thispagestyle {empty}


\end{document}
