\scalebox{0.8}{
\begin{tabular}{l|c|c} \hline
  符号化名 & 特徴 & 遷移制約 \\ \hline
  vrc1 & 基本となる符号化 & 一貫性制約のみを用いて表現  \\ \hline
  vrc2 & アトムchangedを追加 & \begin{tabular}{c}
    「遷移で色が変わる頂点は一つのみ」 \\
    を個数制約も用いて表現
  \end{tabular}  \\ \hline
  vrc3 & アトムunchangedを追加 & \begin{tabular}{c}
    「遷移で色が変わらない頂点は \\ $|V|-1$個のみ」 \\ 
    を個数制約も用いて表現
  \end{tabular} \\ \hline
\end{tabular}
}