\documentclass[dvipdfmx,11pt]{beamer}

%全体設定
%\AtBeginDvi{\special{pdf:tounicode 90ms-RKSJ-UCS2}}

\usepackage{bxdpx-beamer}% dvipdfmxなので必要
\usepackage{pxjahyper}
\usepackage{minijs}
\usepackage{otf}
\usepackage{amssymb,amsmath}
\usepackage{hyperref}
\usepackage[absolute,overlay]{textpos}
\usepackage{comment}
\usepackage{colortbl}
\usepackage{graphicx}
\usepackage{tikz}
\usetikzlibrary{positioning}
\usetikzlibrary{shadows}
\usepackage{listings}
%\usepackage{plistings}
\def\lstlistingname{コード}
\usetheme{Copenhagen}
\setbeamertemplate{navigation symbols}{} %スライドのボタン?(右下のやつ)を消す
%\setbeamertemplate{footline}[short title]
\setbeamersize{text margin left=1.5em,text margin right=1.5em} % 余白なくすやつ

% footer setting %
\makeatother
\setbeamertemplate{footline}
{
  \leavevmode
  \hbox{
  \begin{beamercolorbox}[wd=.4\paperwidth,ht=2.25ex,dp=1ex,center]{author in head/foot}
    \usebeamerfont{author in head/foot}\insertshortauthor
  \end{beamercolorbox}
  \begin{beamercolorbox}[wd=.6\paperwidth,ht=2.25ex,dp=1ex,center]{title in head/foot}
    \usebeamerfont{title in head/foot}\hspace*{1ex} \insertshorttitle\hspace*{3em}
    \textbf{ \insertframenumber{} / \inserttotalframenumber } \hspace*{1ex}
  \end{beamercolorbox}}
  \vskip0pt
}
\makeatletter

% exclude apprendix slides from framenumber %
\newcommand{\backupbegin}{
   \newcounter{framenumberappendix}
   \setcounter{framenumberappendix}{\value{framenumber}}
}
\newcommand{\backupend}{
   \addtocounter{framenumberappendix}{-\value{framenumber}}
   \addtocounter{framenumber}{\value{framenumberappendix}} 
}

\renewcommand{\kanjifamilydefault}{\gtdefault}
%\usetheme{Madrid}

\title[ASPを用いた組合せ遷移問題の解法に関する考察]{解集合プログラミングを用いた\\組合せ遷移問題の解法に関する考察}
\author{山田~悠也}
\date{2020年度卒業研究発表会\\2021年2月19日}
\institute{番原研究室}

%% テンプレ 
\begin{comment}

%%%%%%%%%%%%%%%%%%%%%%%%%%%%%%%%%%%%%%%%%%%%%%%%%%
%% タイトル
%%%%%%%%%%%%%%%%%%%%%%%%%%%%%%%%%%%%%%%%%%%%%%%%%%
\begin{frame}\frametitle{}
\end{frame}

\end{comment}

%###########################################################
%# 本文 ####################################################
%###########################################################
\begin{document}

%%%%%%%%%%%%%%%%%%%%%%%%%%%%%%%%%%%%%%%%%%%%%%%%%%
%% タイトル 
%%%%%%%%%%%%%%%%%%%%%%%%%%%%%%%%%%%%%%%%%%%%%%%%%%
\begin{frame}\frametitle{}
  \titlepage
\end{frame}

%%%%%%%%%%%%%%%%%%%%%%%%%%%%%%%%%%%%%%%%%%%%%%%%%%
%% 組合せ遷移問題
%%%%%%%%%%%%%%%%%%%%%%%%%%%%%%%%%%%%%%%%%%%%%%%%%%
\begin{frame}\frametitle{組合せ遷移問題(Combinatorial Reconfiguration)}

  \begin{itemize}
    \item \alert{組合せ遷移問題}では,組合せ問題の解空間グラフ上における初期状態から目標状態への経路の存在を問う.
    \item \alert{解空間グラフ}とは,
    \begin{itemize}
      \item 頂点は組合せ問題の実行可能解となる.
      \item 辺は\structure{遷移制約}を満たす二つの頂点間に存在する.
    \end{itemize} 
    \item 既存の組合せ問題の多くを組合せ遷移問題に拡張できる.
    \begin{itemize}
      \item NP完全である組合せ問題を基とする問題の多くは,\alert{PSPACE完全}であることが知られている.
      \item \structure{グラフ点彩色問題}・充足可能性判定問題(SAT)・集合被覆問題など.
    \end{itemize}
    %\item \structure{持続可能なシステム}への応用が期待されている.
    %\item 実践的な研究は初期段階.
    \item \alert{$k$彩色遷移問題}は組合せ遷移問題の一つ.
    \begin{itemize}
      \item 基となる問題はグラフ点彩色問題.
      \item 色数$k \geq 4$のとき,PSPACE完全であることが知られている.~[Paul Bonsmaほか '09]
      \item \alert{汎用的かつ効率的なアルゴリズムは見つかっていない}.
    \end{itemize}
  \end{itemize}

\end{frame}

%%%%%%%%%%%%%%%%%%%%%%%%%%%%%%%%%%%%%%%%%%%%%%%%%%
%% k彩色遷移問題
%%%%%%%%%%%%%%%%%%%%%%%%%%%%%%%%%%%%%%%%%%%%%%%%%%

\begin{frame}\frametitle{k彩色遷移問題}

  \begin{block}{$k$彩色遷移問題の定義}
    \begin{itemize}
      \item 問題の入力は,グラフ\structure{$G=(V, E)$},色数\structure{$k$},及び$G$の$k$彩色の二つの実行可能解\structure{$\alpha$}(初期状態)と\structure{$\beta$}(目標状態).
      \item 隣接関係は,ある一つの頂点のみ色が異なるような実行可能解.
      \item \structure{$\alpha$から$\beta$への経路の存在についての判定問題.}
    \end{itemize}
  \end{block}

  \begin{exampleblock}{$k$彩色遷移問題の例}
    \begin{columns}
      \begin{column}{0.3\textwidth}
        \centering
        %%%%%%%%%%%%%%%%%%%%%%%%%%%%%%%%%%%%%%%%%%%%%%%%%%
% 実行例(t=0) (第6章で使う)
%%%%%%%%%%%%%%%%%%%%%%%%%%%%%%%%%%%%%%%%%%%%%%%%%%

\begin{tikzpicture}[scale=0.6]

  % 設定
  \tikzset{node/.style={circle,draw=black}}
 
  \definecolor{col_r}{RGB}{230,0,18}
  %\definecolor{col_b}{RGB}{0,104,183}
  \definecolor{col_b}{RGB}{51,51,179}
  \definecolor{col_y}{RGB}{255,251,0}
  \definecolor{col_g}{RGB}{0,96,0}
 
  % 補助線
  % \draw [help lines,blue] (0,0) grid (20,6);
 
  % node %
  \node[node, fill=col_r!70] (node1){\textbf{1}};
  \node[node, fill=col_b!70, right=of node1] (node2){\textbf{2}};
  \node[node, fill=col_y!70, below=of node1] (node3){\textbf{3}};
  \node[node, fill=col_g!70, below=of node2] (node4){\textbf{4}};
 
  \foreach \u / \v in {node1/node2, node2/node3, node2/node4, node3/node4}
  \draw (\u) -- (\v);
 \end{tikzpicture}
 
 %%%%%%%%%%%%%%%%%%%%%%%%%%%%%%%%%%%%%%%%%%%%%%%%%%%%%%%%%%
 %%% Local Variables:
 %%% mode: japanese-latex
 %%% TeX-master: paper.tex
 %%% End:
 
        $\alpha$ステップ0(初期状態)
      \end{column}
      \begin{column}{0.05\textwidth}
        \textbf{$\longrightarrow$}
      \end{column}
      \begin{column}[]{0.3\textwidth}
        \centering
        \begin{tikzpicture}

 \tikzset{node/.style={circle,draw=black}}

 \definecolor{col_r}{RGB}{230,0,18}
 \definecolor{col_b}{RGB}{51,51,179}
 \definecolor{col_y}{RGB}{255,251,0}
 \definecolor{col_g}{RGB}{0,96,0}

 \node[node, fill=col_r!70] (node1){\textbf{1}};
 \node[node, fill=col_g!70, right=of node1] (node2){\textbf{2}};
 \node[node, fill=col_b!70, below=of node1] (node3){\textbf{3}};
 \node[node, fill=col_y!70, below=of node2] (node4){\textbf{4}};

 \foreach \u / \v in {node1/node2, node1/node3, node1/node4, node2/node3, node2/node4, node3/node4}
 \draw (\u) -- (\v);

\end{tikzpicture}
        ステップ1
      \end{column}
      \begin{column}{0.05\textwidth}
        \textbf{$\longrightarrow$}
      \end{column}
      \begin{column}{0.3\textwidth}
        \centering
        \begin{tikzpicture}
 \draw (0,0)--(2.5,0);
 \draw (0,0.5)--(2.5,0.5);
 \draw (0,1.0)--(2.5,1.0);
 \draw (0,1.5)--(2.5,1.5);
 \draw (0,2.0)--(2.5,2.0);
 \draw (0,2.5)--(2.5,2.5);
 \draw (0,0)--(0,2.5);
 \draw (0.5,0)--(0.5,2.5);
 \draw (1.0,0)--(1.0,2.5);
 \draw (1.5,0)--(1.5,2.5);
 \draw (2.0,0)--(2.0,2.5);
 \draw (2.5,0)--(2.5,2.5);
 \node (X) at (0.25,0.25) {X};
 \draw [orange] (0.25,0.25) circle[radius = 0.24];
 \node (X) at (0.25,0.75) {X};
 \node (X) at (0.25,1.25) {X};
 \node (X) at (0.25,1.75) {X};
 \draw [blue] (0.25,1.75) circle[radius = 0.24];
 \node (X) at (0.25,2.25) {X};
 \draw [red] (0.25,2.25) circle[radius = 0.24];
 \node (X) at (0.75,2.25) {X};
 \node (X) at (0.75,0.75) {X};
 \node (X) at (0.75,1.25) {X};
 \node (X) at (0.75,1.75) {X};
 \node (X) at (1.25,0.75) {X};
 \node (X) at (1.25,1.25) {X};
 \node (X) at (1.75,0.25) {X};
 \node (X) at (1.75,0.75) {X};
 \node (X) at (1.75,1.25) {\color{purple}X};
 \node (X) at (1.75,1.75) {X};
 \node (X) at (2.25,0.25) {X};
 \node (X) at (2.25,0.75) {\color{purple}X};
 \node (X) at (2.25,1.25) {\color{purple}X};
 \node (X) at (2.25,2.25) {X};
 \node (A) at (3.30,2.25) {\color{red}$r_{1} = 1$};
 \node (B) at (3.30,1.75) {\color{blue}$r_{2} = 1$};
 \node (M) at (3.30,1.25) {\color{purple}$r_{3} = 0$};
 \node (M) at (3.30,0.75) {\color{purple}$r_{4} = 0$};
 \node (C) at (3.30,0.25) {\color{orange}$r_{5} = 1$};
 \node (D) at (3.50,-0.50) {$\sum\limits_{i=1}^{5}r_{i}=3$};
\end{tikzpicture}
        ステップ2(終了状態)
      \end{column}
    \end{columns}
  \end{exampleblock}
  
\end{frame}

%%%%%%%%%%%%%%%%%%%%%%%%%%%%%%%%%%%%%%%%%%%%%%%%%%
%% ASP
%%%%%%%%%%%%%%%%%%%%%%%%%%%%%%%%%%%%%%%%%%%%%%%%%%

\begin{frame}\frametitle{解集合プログラミング(Answer Set Programming; ASP)}

  \begin{itemize}
    \item ASP言語は一階論理に基づいた知識表現言語の一種である.
    \item ASPシステムは安定モデル理論~[Gelfond and Lifschitz '88] に基づいた知識表現言語の一種である.
    \item 近年, SAT技術の応用により高速なASPシステムが実現し, システム検証やプランニングなどの様々な分野での実用が拡大している.
  \end{itemize}

  \begin{alertblock}{組合せ遷移問題に対してASPを用いる利点}
    \begin{itemize}
      \item ASPの高い表現力により, 記号上の制約を簡潔に記述できる.
      \item ASPシステムの1つである\textit{clingo}のインクリメンタルモードを用いることにより, 段階的な解の探索を効率よく実行できる.
    \end{itemize}
  \end{alertblock}
  
\end{frame}

%%%%%%%%%%%%%%%%%%%%%%%%%%%%%%%%%%%%%%%%%%%%%%%%%%
%% 研究目的
%%%%%%%%%%%%%%%%%%%%%%%%%%%%%%%%%%%%%%%%%%%%%%%%%%

\begin{frame}\frametitle{研究目的}
  \begin{alertblock}{目的}
    ASPを利用し, PSPACE完全な問題の1つである$k \ge 4$の$k$彩色遷移問題を高速で解ける符号化を提案する.
  \end{alertblock}

  \begin{block}{研究内容}
    \begin{enumerate}
      \item $k$彩色遷移問題を解く,3種類の符号化の提案.
      \item ベンチマークの環境の整備.
      \item 作成したベンチマークを用いた,各種符号化の評価実験.
    \end{enumerate}
  \end{block}

\end{frame}

%%%%%%%%%%%%%%%%%%%%%%%%%%%%%%%%%%%%%%%%%%%%%%%%%%
%% 符号化
%%%%%%%%%%%%%%%%%%%%%%%%%%%%%%%%%%%%%%%%%%%%%%%%%%

\begin{frame}\frametitle{符号化(1/2)}

  \begin{itemize}
    \item グラフ$G$, 色数$k$, Gのk-彩色の2つの実行可能解$\alpha$, $\beta$に加え, \structure{ステップ数$t$}を与える.
    \begin{itemize}
      \item $t$回の遷移で$\alpha$から$\beta$へ遷移可能かの判定問題へと変更.
      \item $t$を遷移回数の上限値まで大きくしていくことで, 最終的に経路の存在について判定可能. 
    \end{itemize}
    \item これに対し, \structure{隣接関係の制約に対する記述が異なる\textbf{3}種類}のASP符号化を提案した. 
    \begin{itemize}
      \item グラフ点彩色に関わる制約は同じものを用いている.
      \item 各符号化で独自に追加するアトム\footnote{ASP上の正リテラルを指す.}以外は, 同じものを用いている.
    \end{itemize}
  \end{itemize}


  
\end{frame}

%%%%%%%%%%%%%%%%%%%%%%%%%%%%%%%%%%%%%%%%%%%%%%%%%%
%% 符号化
%%%%%%%%%%%%%%%%%%%%%%%%%%%%%%%%%%%%%%%%%%%%%%%%%%

\begin{frame}\frametitle{符号化(2/2)}

  \begin{itemize}
    \item $|T|$をステップ数$t$, $|C|$を色数$k$, $|V|$をグラフの頂点数とする.
    
  \end{itemize}

  \begin{table}[t]
    \centering
    \begin{tabular}[]{|c|c|c|} \hline
  符号化 & 特徴 & \begin{tabular}{c} 隣接関係を記述する \\ ルール数 \end{tabular} \\ \hline
  符号化1 & \begin{tabular}{l} すべての符号化において \\ 共通するアトムのみを使用 \end{tabular} & \alert{$4{|C|\choose 2}^{2}{|V|\choose 2}|T|$}  \\ \hline
  符号化2 & \begin{tabular}{l} 独自のアトム \\ \alert{changed(X, T)}の追加 \end{tabular} & \alert{$2{|C|\choose 2}|VT| + 1$} \\ \hline
  符号化3 & \begin{tabular}{l} 独自のアトム \\ \alert{unchanged(X, T)}の追加 \end{tabular} & \alert{$|CVT| + 1$} \\ \hline
\end{tabular}
  \end{table}

  \begin{itemize}
    \item \structure{changed(X, T)}は, ステップ$T-1$とステップ$T$において頂点Xの色が異なることを意味する.
    \item \structure{unchanged(X, T)}は, ステップ$T-1$とステップ$T$において頂点Xの色が同じであることを意味する.
    \item アトムを追加することにより, ルール数を削減することができた.
  \end{itemize}
  
\end{frame}

%%%%%%%%%%%%%%%%%%%%%%%%%%%%%%%%%%%%%%%%%%%%%%%%%%
%% ベンチマーク
%%%%%%%%%%%%%%%%%%%%%%%%%%%%%%%%%%%%%%%%%%%%%%%%%%

\begin{frame}\frametitle{ベンチマーク}

  \begin{itemize}
    \item 現時点で組合せ遷移問題は理論面の研究が主流であり, ベンチマークの整備が必要.
    \item 実験においてステップ$t$を与えるとき, その上限値が必要となる.
    \item ステップ$t$の上限値は, グラフ$G$を$k$彩色するときの実行可能解の数と等しい.
  \end{itemize}

  従って, 全解列挙が可能な($G, k$)からベンチマークを生成する必要がある.
  
\end{frame}

%%%%%%%%%%%%%%%%%%%%%%%%%%%%%%%%%%%%%%%%%%%%%%%%%%
%% 実験環境
%%%%%%%%%%%%%%%%%%%%%%%%%%%%%%%%%%%%%%%%%%%%%%%%%%

\begin{frame}\frametitle{グラフの調査}
  全解列挙可能なグラフと色数の組合せを得るため, 以下の調査を行った.
  \begin{itemize}
    \item \structure{使用するグラフ}: 全44個
    \begin{itemize}
      \item \textit{Graph Coloring and its Generalizations}
      \footnote{https://mat.tepper.cmu.edu/COLOR04/}で公開されている, \textit{Graph Coloring Instances}に属するグラフを使用.
      \item そのうち\structure{彩色数}が判明しているもの.[Tamuraほか '09]
    \end{itemize}
    \item \structure{色数}: 各グラフの彩色数

    \item \structure{ASPシステム}: \textit{clingo-5.4.0}
      \begin{itemize}
        \item \textit{configuration}は\textit{crafty, tweety}を使用.
      \end{itemize}
    \item \structure{制限時間}: 3600秒/問
    \item \structure{環境}: Mac mini, 3.2GHz 6コア Intel Core i7, 64GB メモリ
  \end{itemize}


  \begin{alertblock}{}
    結果, 44個中\structure{\textbf{9}}個のグラフにおいて彩色数での全解列挙が可能であった.
    \begin{itemize}
      \item 最小で\structure{120個}, 最大で\structure{約28億個}の解を持つことが判明した.
    \end{itemize}
  \end{alertblock}
  
  
\end{frame}

%%%%%%%%%%%%%%%%%%%%%%%%%%%%%%%%%%%%%%%%%%%%%%%%%%
%% ベンチマークの生成
%%%%%%%%%%%%%%%%%%%%%%%%%%%%%%%%%%%%%%%%%%%%%%%%%%

\begin{frame}\frametitle{今後の方針}

  \begin{enumerate}
    \item ベンチマークの生成 
    \begin{itemize}
      \item 実行可能解の総数が小さいグラフについては, \structure{全ての解}を出力.
      \item 実行可能解の総数が100万個を超えるグラフについては, \structure{10万個の解のみ}を出力.
      \item 出力された解からランダムで2つの解を抽出しベンチマークを生成.
      \item 各グラフから5問ずつ, \structure{計45問}のベンチマークを生成.
    \end{itemize}

    \item 符号化の評価実験
    \begin{itemize}
      \item 生成したベンチマークを用い, 3種の符号化の評価実験を行う.
    \end{itemize}
  \end{enumerate}
  

\end{frame}

%%%%%%%%%%%%%%%%%%%%%%%%%%%%%%%%%%%%%%%%%%%%%%%%%%
%% 今後の課題
%%%%%%%%%%%%%%%%%%%%%%%%%%%%%%%%%%%%%%%%%%%%%%%%%%

\begin{frame}\frametitle{まとめと今後の課題}

  \begin{block}{まとめ}
    \begin{itemize}
      \item $k$彩色遷移問題の隣接関係に対して, 3種類の符号化を提案した.
      \begin{itemize}
        \item アトムを追加することで, ルール数を大幅に削減することができた.
      \end{itemize}
      \item ベンチマークの生成に使用可能なグラフについて調査した.
      \begin{itemize}
        \item 解空間グラフの大きさ(実行可能解の総数)が異なる9つのグラフを使用可能であることが判明した.
      \end{itemize}
    \end{itemize}
  \end{block}
  
  \begin{alertblock}{今後の課題}
    \begin{itemize}
      \item ベンチマークの生成.
      \item 生成したベンチマークを用いた, 各種符号化の評価実験.
      \item 評価実験で優れていた符号化を, インクリメンタルな解法へと拡張.
    \end{itemize}
  \end{alertblock}

\end{frame}

%###########################################################
%##### 補助スライド ########################################
%###########################################################

%%%% 補助スライド
\appendix
\backupbegin

\begin{frame}{~}
 \centering
 - 補足用 -
\end{frame} 

%%%%%%%%%%%%%%%%%%%%%%%%%%%%%%%%%%%%%%%%%%%%%%%%%%
%% 電気制約
%%%%%%%%%%%%%%%%%%%%%%%%%%%%%%%%%%%%%%%%%%%%%%%%%%
\begin{frame}{補足 : 電気制約}
 \begin{itemize}
  \item \alert{電気制約}は,送電する電流$\cdot$電圧の適正範囲を保証する制約.
  \begin{itemize}
   \item 供給経路の各区間で許容電流を超えない.
   \item 電気抵抗による電圧降下が許容範囲を超えない.
   \item etc.
  \end{itemize}
  \item 電流と電圧が影響し合う\structure{実数ドメイン上の制約}によって表される.
		% \begin{itemize}
		%  		 \item 送電システム上の条件など.
		% \end{itemize}
  \item 実数ドメイン上の制約は,純粋なASPのみで扱うのは\alert{困難}.
		\begin{itemize}
		 \item 緩和問題として,変電所から供給できる家庭の数に上限をつける.
		 \item ASPMT技術により,ASPで得られた解について,
			   背景理論ソルバーと連携して実数ドメイン上の制約を調べる.
		\end{itemize}
 \end{itemize}
\end{frame}

%%%%%%%%%%%%%%%%%%%%%%%%%%%%%%%%%%%%%%%%%%%%%%%%%%
%% 基礎化
%%%%%%%%%%%%%%%%%%%%%%%%%%%%%%%%%%%%%%%%%%%%%%%%%%
\begin{frame}{補足 : ASPシステム}
 
 \vspace{-0.5cm}

 \begin{figure}[htbp]
  \centering
  %%%%%%%%%%%%%%%%%%%%%%%%%%%%%%%%%%%%%%%%%%%%%%%%%%
%% 基礎化の流れの図
%%%%%%%%%%%%%%%%%%%%%%%%%%%%%%%%%%%%%%%%%%%%%%%%%%
\begin{tikzpicture}

 \definecolor{edge}{RGB}{38,38,134}
 \definecolor{node}{RGB}{220,220,249}

 \definecolor{alert_edge}{RGB}{191,0,0}
 \definecolor{alert_node}{RGB}{249,200,200}

 \definecolor{ex_edge}{RGB}{0,96,0}
 \definecolor{ex_node}{RGB}{230,239,230}

 \def\nodespace{2.4cm}

 \tikzset{block/.style={rectangle, thick, draw=edge, fill=node, text width=3cm, 
 text centered, rounded corners, text width=2cm, minimum height=1.5cm}};

 \tikzset{alertblock/.style={rectangle, thick, draw=alert_edge, fill=alert_node, 
 text width=3cm, text centered, rounded corners, text width=1.5cm, minimum height=1.2cm}};

 \node[block](ikkai){一階ASP\\プログラム};

 \node[rectangle,rounded corners, thick, draw=ex_edge, fill=ex_node, 
 right=0.22*\nodespace of ikkai, minimum width=6cm, minimum height=3cm, 
 text centered, label=ASPシステム](sys){};

 \node[block, right=\nodespace of ikkai](meidai){命題ASP\\プログラム};
 \node[block, right=\nodespace of meidai](ASP){解集合};

 \node[right=0.6*\nodespace of ikkai, text width=1.5cm, 
 text centered, text=red, anchor=south](){基礎化\\ソルバー};
 \node[right=0.4*\nodespace of meidai, text width=1.5cm, 
 text centered, text=red, anchor=south](){解集合\\ソルバー};

 
 \foreach \u / \v / \n in {ikkai/meidai,meidai/ASP}
 \draw [thick,->] (\u) to (\v);

\end{tikzpicture}
 \end{figure}

 \vspace{-0.5cm}

 \begin{exampleblock}{}
  \begin{enumerate}
   \item 一階ASPプログラムを基礎化ソルバーによって,
		 命題ASPプログラムに\alert{基礎化}する.
   \item 命題ASPプログラムについて,SAT技術を応用した解集合ソルバーが解集合を探索する.
  \end{enumerate}
 \end{exampleblock}

\end{frame}
%%%%%%%%%%%%%%%%%%%%%%%%%%%%%%%%%%%%%%%%%%%%%%%%%%
%% ASPの構文
%%%%%%%%%%%%%%%%%%%%%%%%%%%%%%%%%%%%%%%%%%%%%%%%%%
\begin{frame}{ASPの構文}
  \begin{alertblock}{}\centering
    ASPの言語は論理プログラムをベースとしている~\footnotemark.
  \end{alertblock}
  \begin{itemize}
  \item \structure{\bf 論理プログラム}とは,以下の\structure{\bf ルール}の有限集合である.
    \begin{center}
      \begin{minipage}[c]{0.7\textwidth}
        \begin{block}{}\centering
          $a_0$\quad\code{:-}\quad$a_1$\code{,}\ldots\code{,}$a_m$\code{,}
          \ \code{not}~$a_{m+1}$\code{,}\ldots\code{,} \code{not}~$a_n$\code{.}
        \end{block}        
      \end{minipage}
   \end{center}\vfill
    $0 \leq m \leq n$ であり,各 $a_i$ はアトム,
    \code{not}は\structure{\bf デフォルトの否定},\\
    ``\code{,}''は連言(AND)を表す.``\code{:-}''の左辺を\structure{\bf ヘッド},
		右辺を\structure{\bf ボディ}と呼ぶ.
  \item \alert{\bf 直感的な意味}は,
    「$a_1,\ldots,a_m$がすべて成り立ち,
    $a_{m+1},\ldots,a_n$のそれぞれが成り立たないならば,
    $a_0$が成り立つ」である.
  \item ボディが空のルールを\structure{\bf ファクト}と呼び,``\code{:-}''は省略できる.
  \item ヘッドが空のルールを\structure{\bf 一貫性制約}と呼ぶ.例えば,\hspace{-1ex}
    ``\code{:-} $a_1$\code{,} \code{not}~$a_{2}$''は,
    「$a_1$が成り立つならば,$a_2$が成り立つ」を意味する.
  \end{itemize}
  \footnotetext{本発表では標準論理プログラムを単に論理プログラムと呼ぶ.}
\end{frame}
%%%%%%%%%%%%%%%%%%%%%%%%%%%%%%%%%%%%%%%%%%%%%%%%%%
%% ASPの拡張構文
%%%%%%%%%%%%%%%%%%%%%%%%%%%%%%%%%%%%%%%%%%%%%%%%%%
\begin{frame}{ASPの拡張構文}
\begin{alertblock}{}\centering
  組合せ問題を解くための便利な構文が用意されている.
\end{alertblock}
\begin{itemize}
 \item \structure{\bf 選択子}
   \begin{center}
     \code{\{}$a_1$\code{;}\ldots\code{;}$a_n$\code{\}}
   \end{center}
   アトム集合 $\{a_1,\dots,a_n\}$
   の任意の部分集合が成り立つことを意味する.
 \item \structure{\bf 個数制約}
   \begin{center}
     $lb$\ \code{\{}$a_1$\code{;}\ldots\code{;}$a_n$\code{\}}\ $ub$
   \end{center}
   $a_1,\dots,a_n$ のうち,
   $lb$個以上,$ub$個以下が成り立つことを意味する.
 \item \structure{\bf 重み付き個数制約}
   \begin{center}
     $lb$ \code{\#sum\{} $w_1$\code{:}$a_1$\code{;}\ldots\code{;}$w_n$\code{:}$a_n$ \code{\}} $ub$
   \end{center}
   $a_1,\dots,a_n$のうち,
   成り立つアトムの重み和が$lb$以上,$ub$以下になることを意味する.
\end{itemize}
\end{frame}
%%%%%%%%%%%%%%%%%%%%%%%%%%%%%%%%%%%%%%%%%%%%%%%%%%
%% 改良符号化 (到達可能性)
%%%%%%%%%%%%%%%%%%%%%%%%%%%%%%%%%%%%%%%%%%%%%%%%%%
\begin{frame}[fragile]{改良符号化: 到達可能性}
\begin{exampleblock}{}\small
\begin{lstlisting}
(1) { inForest(X,Y) } :- edge(X,Y).
\end{lstlisting}
\end{exampleblock}
\begin{itemize}
 \item (1) 各辺\code{(X,Y)について},根付き全域森に含まれること意味する \\
	  アトム\code{inForest(X,Y)}を導入する.
\end{itemize}
\begin{exampleblock}{}\small
\begin{lstlisting}
(2) reached(R,R) :- root(R).
(3) reached(X,R) :- reached(Y,R), inForest(Y,X).
(4) reached(X,R) :- reached(Y,R), inForest(X,Y).
\end{lstlisting}
\end{exampleblock}
\vfill
\begin{itemize}
\item アトム\code{reached(X,R)}は,ノード\code{X}が根ノード\code{R}から到達可能であることを意味する.
%\item (2) 各根ノード\code{R}について,自分自身から到達可能であることを表す.
\item (3) ノード\code{Y}が根ノード\code{R}から到達可能かつ,辺\code{(Y,X)}が根付き全域森に含まれるならば,
	  ノード\code{X}も同じ根ノード\code{R}から到達可能であることを表す.
\end{itemize}
\end{frame}
%%%%%%%%%%%%%%%%%%%%%%%%%%%%%%%%%%%%%%%%%%%%%%%%%%
%% 改良符号化 (根付き連結制約)
%%%%%%%%%%%%%%%%%%%%%%%%%%%%%%%%%%%%%%%%%%%%%%%%%%
\begin{frame}[fragile]{改良符号化: 根付き連結制約}
\begin{exampleblock}{}\small
\begin{lstlisting}
(5) :- node(X), not 1 { reached(X,R) } 1.
\end{lstlisting}
\end{exampleblock}
\vfill
\begin{itemize}
\item (5) 各ノード\code{X}について,ちょうど1つの根からのみ到達可能であることを意味する.
\end{itemize}
\end{frame}
%%%%%%%%%%%%%%%%%%%%%%%%%%%%%%%%%%%%%%%%%%%%%%%%%%
%% 改良符号化 (非閉路制約)
%%%%%%%%%%%%%%%%%%%%%%%%%%%%%%%%%%%%%%%%%%%%%%%%%%
\begin{frame}[fragile]{改良符号化: 非閉路制約}
\begin{minipage}[c]{1.01\textwidth}
\begin{exampleblock}{}\small
\begin{lstlisting}
(6) :- root(R),
       not 1 #sum{ 1,X:reached(X,R) ;
                  -1,X,Y:inForest(X,Y),reached(X,R),reached(Y,R)
                 } 1.
\end{lstlisting}
\end{exampleblock}
\end{minipage}
\vfill
\begin{itemize}
\item (6) 各連結成分の\structure{\bf ノード数と辺数の差が1}になることを意味する.
\item 各連結成分が\structure{\bf 木の性質}を満たすことにより,サイクルを持たない
	  ことを保証する.
\end{itemize}
\end{frame}
%%%%%%%%%%%%%%%%%%%%%%%%%%%%%%%%%%%%%%%%%%%%%%%%%%
%% ルール数の比較
%%%%%%%%%%%%%%%%%%%%%%%%%%%%%%%%%%%%%%%%%%%%%%%%%%
\begin{frame}{基礎化後のルール数}
  \begin{itemize}
  \item グラフのノード数を$|V|$,根ノードの数を$|R|$とする.
  \end{itemize}
  \begin{table}[t]
    \centering
    %%%%%%%%%%%%%%%%%%%%%%%%%%%%%%%%%%%%%%%%%%%%%%%%%%%%%%%%%%%%%%%%
\chapter{ハミルトン閉路問題および関連問題のASP符号化}\label{chap:proposal}
%%%%%%%%%%%%%%%%%%%%%%%%%%%%%%%%%%%%%%%%%%%%%%%%%%%%%%%%%%%%%%%% 

%%%%
\begin{figure}[h]
  \centering
  \thicklines
  \setlength{\unitlength}{1.2pt}
  \small\footnotesize\scriptsize
  \begin{picture}(280,57)(4,-10)
    \put(  0, 20){\dashbox(50,24){\shortstack{HCP問題\\インスタンス}}}
    \put( 60, 20){\framebox(50,24){変換器}}
    \put(120, 20){\dashbox(50,24){\shortstack{ASPファクト}}}
    \put(120,-10){\dashbox(50,24){\shortstack{ASP符号化\\(論理プログラム)}}}
    \put(180, 20){\framebox(50,24){ASPシステム}}
    \put(240, 20){\dashbox(50,24){\shortstack{HCP問題\\の解}}}
    \put( 50, 32){\vector(1,0){10}}
    \put(110, 32){\vector(1,0){10}}
    \put(170, 32){\vector(1,0){10}}
    \put(230, 32){\vector(1,0){10}}
    \put(170, +2){\line(1,0){4}}
    \put(174, +2){\line(0,1){30}}
  \end{picture}  
\caption{ASP を用いたハミルトン閉路問題(HCP)の解法}
\label{fig:arch}
\end{figure}
%%%%

%\begin{figure}[tbp]
\tikz{
  %1ノード目
  \path[draw=black, fill=blue!20, rounded corners=5pt]%線の設定
  node[at={(0.75,0.75)}] {問題}%文字を入れる
  (0,0) --(1.5,0) --(1.5,1.5) --(0,1.5) --cycle;%外周
  %2ノード目
  \path[draw=black, fill=blue!20, rounded corners=5pt, shift={(3,0)}]
  node[at={(0.75,0.75)}] {
    \begin{tabular}{c}
      ASP\\
      ファクト
    \end{tabular}
  }
  (0,0) --(1.5,0) --(1.5,1.5) --(0,1.5) --cycle;
  %3ノード目文字が複数行
  \path[draw=black, fill=green!20, rounded corners=5pt, shift={(6,0)}]
  node[at={(0.75,0.75)}] {
    \begin{tabular}{c}
      ASP\\
      システム
    \end{tabular}
  }
  (0,0) --(1.5,0) --(1.5,1.5) --(0,1.5) --cycle;
  %4ノード目文字が複数行
  \path[draw=black, fill=blue!20, rounded corners=5pt, shift={(9,0)}]
  node[at={(0.75,0.75)}] {解集合}
  (0,0) --(1.5,0) --(1.5,1.5) --(0,1.5) --cycle;
  %5ノード目文字が複数行
  \path[draw=black, fill=red!20, rounded corners=5pt, shift={(3,-3)}]
  node[at={(0.75,0.75)}] {
    \begin{tabular}{c}
      ASP\\
      符号化
    \end{tabular}
  }
  (0,0) --(1.5,0) --(1.5,1.5) --(0,1.5) --cycle;
  \draw[arrows=->] (1.5,0.75) --(3.0,0.75);
  \draw[arrows=->,shift={(3,0)}] (1.5,0.75) --(3.0,0.75);
  \draw[arrows=->,shift={(6,0)}] (1.5,0.75) --(3.0,0.75);
  \draw[arrows=->] (4.5,-2.25) --(6.0,0.5);
}
\caption{ASPを用いた解法}
\label{aspmethod}
\end{figure}


ASP を用いたハミルトン閉路問題および関連問題の解法について述べる.
図~\ref{fig:arch}に,解法の流れを示す.
与えられたハミルトン閉路問題は ASP ファクトに変換され,
ハミルトン閉路問題を解く ASP 符号化と結合され,
ASP システムによって解が計算される.
本論文では,ASP システムとして{\clingo}を用いる.

%%%%%%%%%%%%%%%%%%%%%%%%%%%%%%%%%%%%%%%%%%%%%%%%%%%%%%%%%%%%%%%%%%%%%%%
\section{ASPファクト形式}
%%%%%%%%%%%%%%%%%%%%%%%%%%%%%%%%%%%%%%%%%%%%%%%%%%%%%%%%%%%%%%%%%%%%%%%

%%%%%%%%%%%%%%%%%%%%%%%%%%%%%%
\begin{figure}[t]
\begin{center}
\begin{tikzpicture}
  %ノード1  
  \draw(4,2) circle (0.5)
  node[at={(4.1,2.1)}] {
    \begin{tabular}{c}
      1
    \end{tabular}
  };
  %ノード2  
  \draw(4,0) circle (0.5)
  node[at={(4.1,0.1)}] {
    \begin{tabular}{c}
      2
    \end{tabular}
  };
  %ノード3  
  \draw(6,2) circle (0.5)
  node[at={(6.1,2.1)}] {
    \begin{tabular}{c}
      3
    \end{tabular}
  };
  %ノード4  
  \draw(6,0) circle (0.5)
  node[at={(6.1,0.1)}] {
    \begin{tabular}{c}
      4
    \end{tabular}
  };
  %ノード5  
  \draw(8,2) circle (0.5)
  node[at={(8.1,2.1)}] {
    \begin{tabular}{c}
      5
    \end{tabular}
  };
  %ノード6  
  \draw(8,0) circle (0.5)
  node[at={(8.1,0.1)}] {
    \begin{tabular}{c}
      6
    \end{tabular}
  };
\draw(4,0.5) --(4,1.5);
\draw(6,0.5) --(6,1.5);
\draw(8,0.5) --(8,1.5);
\draw(4.5,0) --(5.5,0);
\draw(4.5,2) --(5.5,2);
\draw(6.5,0) --(7.5,0);
\draw(6.5,2) --(7.5,2);
\end{tikzpicture}

\caption{入力となる重み付き無向グラフの例}
\label{graphexample}
\end{center}
\end{figure}
%%%%%%%%%%%%%%%%%%%%%%%%%%%%%%

%%%%%%%%%%%%%%%%%%%%%%%%%%%%%%
\lstinputlisting[float=t,caption={%
図~\ref{graphexample}のASPファクト表現},%
captionpos=b,frame=single,label=code:graph_example.lp,%
numbers=none,%
breaklines=true,%
columns=fullflexible,keepspaces=true,%
basicstyle=\ttfamily\scriptsize]{code/graph_example.lp}
%%%%%%%%%%%%%%%%%%%%%%%%%%%%%%


本節では,最短ハミルトン閉路問題の例にとって,
入力となる重み付き無向グラフ(図~\ref{graphexample})の
ASP ファクト形式について説明する.
%
このグラフは,頂点数が6,辺の数が7であり,辺に付けられた値は距離を表す.
コード~\ref{code:graph_example.lp}に,ASPファクト形式を示す.
%
アトム\code{node/1}は頂点,\code{edge/2}は辺,\code{cost/3}は距離を表す.
例えば,\code{cost(1,2,3)}は,辺\code{edge(1,2)}の距離が3であることを
表している.

%%%%%%%%%%%%%%%%%%%%%%%%%%%%%%%%%%%%%%%%%%%%%%%%%%%%%%%%%%%%%%%%%%%%%%%
\section{ハミルトン閉路問題の ASP 符号化}\label{hamiltonianasp}
%%%%%%%%%%%%%%%%%%%%%%%%%%%%%%%%%%%%%%%%%%%%%%%%%%%%%%%%%%%%%%%%%%%%%%%

ハミルトン閉路問題は,与えられたグラフの全頂点をちょうど一度ずつ通る閉
路(ハミルトン閉路)が存在するかどうかを判定する問題である.
$G=(V,E)$にハミルトン閉路が存在する必要十分条件は,
以下の2つの制約を満たす部分グラフ$G'=(V,E')$が存在することである.

\begin{itemize}
\item $G'$の各頂点の次数が2 (次数制約)
\item $G'$が連結である (連結制約)
\end{itemize}

本論文では,前者を\textbf{次数制約},後者を\textbf{連結制約}と呼ぶ.
ハミルトン路問題は,ハミルトン閉路問題から始点と終点が一致するという閉
路の条件を取り除いたものである.
ハミルトン路問題では,次数制約は以下のように変わる.

\begin{itemize}
\item 始点と終点の次数が1,他の頂点の次数が2
\end{itemize}

以下では,ハミルトン閉路問題に対する3つの ASP 符号化
\textsf{undirected},\textsf{directed},\textsf{acyclicity}
を提案する.

%%%%%%%%%%%%%%%%%%%%%%%%%%%%%%%%%%%%%%%%%%%%%%%%%%%%%%%%%%%%%%%%%%%%%%%
\subsection{\textsf{undirected}符号化}
%%%%%%%%%%%%%%%%%%%%%%%%%%%%%%%%%%%%%%%%%%%%%%%%%%%%%%%%%%%%%%%%%%%%%%%

%%%%%%%%%%%%%%%%%%%%%%%%%%%%%%
\lstinputlisting[float=t,caption={%
\textsf{undirected}符号化},%
captionpos=b,frame=single,label=code:hamilton1.lp,%
numbers=left,%
breaklines=true,%
columns=fullflexible,keepspaces=true,%
basicstyle=\ttfamily\footnotesize]{code/hamilton1.lp}
%%%%%%%%%%%%%%%%%%%%%%%%%%%%%%

\textsf{undirected}符号化は,ハミルトン閉路問題の次数制約と連結制約を,
ASP の一貫性制約で表した基本的な符号化である.
コード~\ref{code:hamilton1.lp}に,\textsf{undirected}符号化を示す.
この符号化は,ハミルトン閉路問題とハミルトン路問題の両方に対応している.
符号化中の\code{s}は始点の頂点番号,\code{t}は終点の頂点番号を表し,こ
れらは実行時に与えられる.
ここでは,ハミルトン閉路問題(\code{s}=\code{t})の場合について説明する.

\begin{itemize}
\item 1行目のルールは,各辺\code{edge(X,Y)}に対して,その辺がハミルト
  ン閉路に含まれるかどうかを意味するアトム\code{in(X,Y)}を選択子を用い
  て導入している.
\item 次数制約は3行目のルールで表される.このルールは,
  各頂点\code{node(X)}に対して,その次数の和が2に等しいことを個数制約
  を使って表している.
\item 連結制約は11行目のルールで表される.
ある頂点\code{X}が始点\code{s}から到達可能であることを意味する補助アト
ム\code{reached(X)}を導入する.
8行目のルールは,始点\code{s}が到達可能あることを表している.
9行目のルールは,各辺\code{X}--\code{Y}に対して,その辺がハミルトン閉
路に含まれ(\code{in(X,Y)}),かつ,頂点\code{X}が始点から到
達可能であれば(\code{reached(X)}),\code{Y}も到達可能であることを表している.
10行目は9行目と同様であるが,辺\code{Y}--\code{X}の場合を表している.
11行目のルールは,各頂点\code{node(X)}が始点から到達可能でなければな
らないことを一貫性制約を使って表している.
\end{itemize}

%%%%%%%%%%%%%%%%%%%%%%%%%%%%%%%%%%%%%%%%%%%%%%%%%%%%%%%%%%%%%%%%%%%%%%%
\subsection{\textsf{directed}符号化}
%%%%%%%%%%%%%%%%%%%%%%%%%%%%%%%%%%%%%%%%%%%%%%%%%%%%%%%%%%%%%%%%%%%%%%%

%%%%%%%%%%%%%%%%%%%%%%%%%%%%%%
\lstinputlisting[float=t,caption={%
\textsf{directed}符号化},%
captionpos=b,frame=single,label=code:hamilton2.lp,%
numbers=left,%
breaklines=true,%
columns=fullflexible,keepspaces=true,%
basicstyle=\ttfamily\footnotesize]{code/hamilton2.lp}
%%%%%%%%%%%%%%%%%%%%%%%%%%%%%%

\textsf{directed}符号化は,\textsf{undirected}符号化をベースに,
与えられた無向グラフの各辺$u-v$に対して,2つの弧$u\rightarrow v$と
$v\rightarrow u$を対応させることで有向グラフ化して解く符号化である.
コード~\ref{code:hamilton2.lp}に,\textsf{directed}符号化を示す.
前節と同様に,ハミルトン閉路問題(\code{s}=\code{t})の場合について説明する.

\begin{itemize}
\item 1行目では,無向グラフの有向グラフ化を行う.
  与えられた無向グラフの各辺\code{edge(X,Y)}に対して,
  2つの弧\code{edge(X,Y)},\code{edge(Y,X)}を導入した.
\item 2行目のルールは,各弧\code{edge(X,Y)}に対して,その弧がハミルト
  ン閉路に含まれるかどうかを意味するアトム\code{in(X,Y)}を選択子を用い
  て導入している.
\item 次数制約は4,5行目のルールで表される.
  4行目では,各頂点\code{node(X)}に対して,
  その出次数が1に等しいことを個数制約を使って表している.
  5行目では,入次数について4行目と同様の制約を表す.
\item 連結制約は15行目のルールで表される.
  ある頂点\code{X}が始点\code{s}から到達可能であることを意味する
  補助アトム\code{reached(X)}を導入する.
  13行目のルールは,始点\code{s}が到達可能あることを表している.
  14行目のルールは,各弧\code{X}--\code{Y}に対して,その弧がハミルトン閉路
  に含まれ(\code{in(X,Y)}),かつ,頂点\code{X}が始点から
  到達可能であれば(\code{reached(X)}),\code{Y}も到達可能であることを表している.
  15行目のルールは,各頂点\code{node(X)}が始点から到達可能でなければ
  ならないことを一貫性制約を使って表している.
\item 18行目のルールは,解についての対称性を除去する.
  与えられた無向グラフ上の各ハミルトン閉路に対して,
  それを変換した有向グラフ上のハミルトン閉路は対称な2つが存在する.
  これによる解の重複を防ぐために,18行目のルールは,各弧\code{s}--\code{X},
  \code{Y}--\code{s}がハミルトン閉路に含まれるならば(\code{in(s,X),in(Y,s)}),
  \code{X < Y}でなければならないことを,一貫性制約を用いて表している
\end{itemize}

%%%%%%%%%%%%%%%%%%%%%%%%%%%%%%%%%%%%%%%%%%%%%%%%%%%%%%%%%%%%%%%%%%%%%%%
\subsection{\textsf{acyclicity}符号化}
%%%%%%%%%%%%%%%%%%%%%%%%%%%%%%%%%%%%%%%%%%%%%%%%%%%%%%%%%%%%%%%%%%%%%%%

%%%%%%%%%%%%%%%%%%%%%%%%%%%%%%
\lstinputlisting[float=t,caption={%
\textsf{acyclicity}符号化},%
captionpos=b,frame=single,label=code:hamilton3.lp,%
numbers=left,%
breaklines=true,%
columns=fullflexible,keepspaces=true,%
basicstyle=\ttfamily\footnotesize]{code/hamilton3.lp}
%%%%%%%%%%%%%%%%%%%%%%%%%%%%%%

\textsf{acyclicity}符号化は,\textsf{directed}符号化をベースに,
連結の制約に代わる部分閉路禁止制約を組込み非閉路制約で表現した符号化である.
コード~\ref{code:hamilton3.lp}に,\textsf{acyclicity}符号化を示す.
前節と同様に,ハミルトン閉路問題(\code{s}=\code{t})の場合について説明する.

\begin{itemize}
\item 1行目では,無向グラフの有向グラフ化を行う.
  与えられた無向グラフの各辺\code{edge(X,Y)}に対して,
  2つの弧\code{edge(X,Y)},\code{edge(Y,X)}を導入した.
\item 2行目のルールは,各弧\code{edge(X,Y)}に対して,その弧がハミルト
  ン閉路に含まれるかどうかを意味するアトム\code{in(X,Y)}を選択子を用い
  て導入している.
\item 次数制約は4,5行目のルールで表される.
  4行目では,各頂点\code{node(X)}に対して,
  その出次数が1に等しいことを個数制約を使って表している.
  5行目では,入次数について4行目と同様の制約を表す.
\item 部分閉路禁止制約は14行目のルールで表される.
  このルールは,始点でない各頂点\code{X},\code{Y}について,
  弧\code{X}--\code{Y}がハミルトン閉路に含まれるならば(\code{in(X,Y)}),
  そのような弧の集合をもつグラフが閉路をもたないことを,\code{#edge}宣言を用いて表す.
  ようするに,始点(終点)を含まないような閉路を禁止している.
\item 17行目のルールは,解についての対称性を除去する.
  与えられた無向グラフ上の各ハミルトン閉路に対して,
  それを変換した有向グラフ上のハミルトン閉路は対称な2つが存在する.
  これによる解の重複を防ぐために,17行目のルールは,各弧\code{s}--\code{X},
  \code{Y}--\code{s}がハミルトン閉路に含まれるならば(\code{in(s,X),in(Y,s)}),
  \code{X < Y}でなければならないことを,一貫性制約を用いて表している
\end{itemize}

%%%%%%%%%%%%%%%%%%%%%%%%%%%%%%%%%%%%%%%%%%%%%%%%%%%%%%%%%%%%%%%%%%%%%%% 
\section{最短ハミルトン閉路問題のASP符号化}\label{minexpl}
%%%%%%%%%%%%%%%%%%%%%%%%%%%%%%%%%%%%%%%%%%%%%%%%%%%%%%%%%%%%%%%%%%%%%%% 

%% %%%%%%%%%%%%%%%%%%%%%%%%%%%%%%
%% \lstinputlisting[caption =  最適化,label = minimize]{code/obj_minimize.lp}
%% %%%%%%%%%%%%%%%%%%%%%%%%%%%%%%

%%%%%%%%%%%%%%%%%%%%%%%%%%%%%%
\lstinputlisting[float=t,caption={%
最小化},%
captionpos=b,frame=single,label=code:obj_minimize.lp,%
numbers=left,%
breaklines=true,%
columns=fullflexible,keepspaces=true,%
basicstyle=\ttfamily\footnotesize]{code/obj_minimize.lp}
%%%%%%%%%%%%%%%%%%%%%%%%%%%%%%

最短ハミルトン閉路問題の目的関数は,
ハミルトン閉路を構成する各辺の距離の総和である.
コード\ref{code:obj_minimize.lp}は,
その目的関数の最小化を表す.
このコードは,各辺\code{edge(X,Y)}に対して,その辺がハミルトン閉路に
含まれ(\code{in(X,Y)}),その距離が\code{C}である時に(\code{cost(X,Y,C)}),
\code{C}の総和の最小化を,最小化関数を用いて表している.
.
%%%%%%%%%%%%%%%%%%%%%%%%%%%%%%
\lstinputlisting[float=t,caption={%
重み付き無向グラフの有向グラフ化},%
captionpos=b,frame=single,label=code:cost_both.lp,%
numbers=left,%
breaklines=true,%
columns=fullflexible,keepspaces=true,%
basicstyle=\ttfamily\footnotesize]{code/cost_both.lp}
%%%%%%%%%%%%%%%%%%%%%%%%%%%%%%

符号化directed,acyclicityについては,
与えられた無向グラフの各辺\code{edge(X,Y)}に対して,
2つの弧\code{edge(X,Y)},\code{edge(Y,X)}を導入した.
各辺の距離もこれに対応させるために,コード\ref{code:cost_both.lp}
を追加した.
このルールは,各辺\code{X}--\code{Y}の距離を表す\code{cost(X,Y,C)}について,
\code{cost(Y,X,C)}を導入する.
これにより,与えられた無向グラフの各辺\code{edge(X,Y)}の重み\code{C}が
2つの弧\code{edge(X,Y)},\code{edge(Y,X)}にも付与された.

%%%%%%%%%%%%%%%%%%%%%%%%%%%%%%%%%%%%%%%%%%%%%%%%%%%%%%%%%%%%%%%%%%%%%%% 
\section{コスト制約付きハミルトン閉路のASP符号化}
%%%%%%%%%%%%%%%%%%%%%%%%%%%%%%%%%%%%%%%%%%%%%%%%%%%%%%%%%%%%%%%%%%%%%%% 

%%%%%%%%%%%%%%%%%%%%%%%%%%%%%%
\lstinputlisting[float=t,caption={%
コスト制約},%
captionpos=b,frame=single,label=code:cost_constraint.lp,%
numbers=left,%
breaklines=true,%
columns=fullflexible,keepspaces=true,%
basicstyle=\ttfamily\footnotesize]{code/cost_constraint.lp}
%%%%%%%%%%%%%%%%%%%%%%%%%%%%%%

コスト制約付きハミルトン閉路問題は
ハミルトン閉路問題に,距離の総和が所与の閾値以下 (または以上) であること
を制約条件として付加した問題である.
コード\ref{code:const_constraing.lp}のルールは,その制約を表す.
ルール中の\code{c}は閾値を表し,これは実行時に与えられる.
このルールは,各辺\code{edge(X,Y)}に対して,その辺がハミルトン閉路に
含まれ(\code{in(X,Y)}),その距離が\code{C}である時に(\code{cost(X,Y,C)}),
\code{C}の総和が\code{c}以下でなければならないことを,
一貫性制約と重み付き個数制約を用いて表す.

また,\ref{minexpl}と同様に,
符号化directed,acyclicityについては,
アトム\code{cost}についても有向グラフ化に
対応させるためにコード\ref{code:cost_both.lp}を追加した.
%%%%%%%%%%%%%%%%%%%%%%%%%%%%%%%%%%%%%%%%%%%%%%%%%%%%%%%%%%%%%%%%%%%%%%%

%%% Local Variables:
%%% mode: latex
%%% TeX-master: "paper"
%%% End:

  \end{table}
\end{frame}

%%%%%%%%%%%%%%%%%%%%%%%%%%%%%%%%%%%%%%%%%%%%%%%%%%
%% ASPのコード
%%%%%%%%%%%%%%%%%%%%%%%%%%%%%%%%%%%%%%%%%%%%%%%%%%
\begin{frame}[fragile]{補足 : 根付き全域森 基本符号化}
%%%%%%%%%%%%%%%%%%%%%%%%%%%%%%%%% 
\lstinputlisting[frame=single,label=code:roop,%
xleftmargin=1zw,%
xrightmargin=1zw,%
numbersep=5pt,%
numbers=left,%
breaklines=true,%
columns=fullflexible,keepspaces=true,%
basicstyle=\ttfamily\scriptsize]{code/srf1.lp}
%%%%%%%%%%%%%%%%%%%%%%%%%%%%%%%%%
\end{frame}

\begin{frame}[fragile]{補足 : 遷移問題 シングルショット符号化}

\begin{multicols*}{2}
%%%%%%%%%%%%%%%%%%%%%%%%%%%%%%%%% 
\lstinputlisting[frame=single,label=code:roop,%
xleftmargin=1zw,%
xrightmargin=1zw,%
numbersep=5pt,%
numbers=left,%
breaklines=true,%
columns=fullflexible,keepspaces=true,%
basicstyle=\ttfamily\tiny]{code/trans-const.lp}
%%%%%%%%%%%%%%%%%%%%%%%%%%%%%%%%%
\end{multicols*}
\end{frame}

\begin{frame}[fragile]{補足 : 遷移問題 マルチショット符号化}
\begin{multicols*}{2}
%%%%%%%%%%%%%%%%%%%%%%%%%%%%%%%%%
\lstinputlisting[frame=single,label=code:incmode,% 
xleftmargin=1zw,%
xrightmargin=1zw,%
numbersep=5pt,%
numbers=left,%
breaklines=true,%
columns=fullflexible,keepspaces=true,%
basicstyle=\ttfamily\tiny]{code/dnet-trans.lp}
%%%%%%%%%%%%%%%%%%%%%%%%%%%%%%%%% 
\end{multicols*}
\end{frame}

\backupend


\end{document}