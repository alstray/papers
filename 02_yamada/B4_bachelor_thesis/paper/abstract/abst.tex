\documentclass{abst}

\begin{document}

%%%%%%%%%%%%%%%%%%%%%%%%%%%%%%%%%%%%%%%%%%%%%%%%%%%%%%%%%%%%%%%%%%%
\研究室名{番原}
\氏名{山~~田~~~悠~~也}
\卒論題目{%
  解集合プログラミングを用いた組合せ遷移問題の解法に関する考察}
%%%%%%%%%%%%%%%%%%%%%%%%%%%%%%%%%%%%%%%%%%%%%%%%%%%%%%%%%%%%%%%%%%%
\卒論要旨{%
\vskip .5em
\textbf{組合せ遷移問題 (Combinatorial Reconfiguration Problems)}とは,
ある組合せ問題とその2つの実行可能解が与えられたとき,一方の実行可能解
から他方の実行可能解へ,遷移制約を満たしつつ,到達できるかを判定する問
題である.
従来の組合せ問題が実行可能解が存在するかを判定するのに対し,
組合せ遷移問題では,実行可能解が形成する解空間グラフにおいて,2つの頂
点間の連結性や到達可能性を判定することが目的となる.
計算困難性については,基の組合せ問題が NP 完全であるとき,その遷移
問題の多くは,PSPACE 完全になることが知られている.

\vskip .5em
組合せ遷移問題の代表例としては,
命題論理の充足可能性判定(SAT)の遷移問題,
グラフ点彩色の遷移問題,
支配集合の遷移問題,
独立点集合の遷移問題
などが挙げられる.
グラフ点彩色の遷移問題は\textbf{$k$彩色遷移問題}とも呼ばれ,
色数$k$のグラフ点彩色問題と2つの彩色が与えられたとき,
一方の彩色から他方の彩色へ,各遷移過程において色が変化する頂点はただ1
つという遷移制約を満たしつつ,到達できるかを判定する問題である.
一般に,$k\ge 4$においてPSPACE 完全であることが示されている.

\vskip .5em
組合せ遷移問題の研究は,理論計算機科学の分野を中心に,2000年以降急
速に研究が発展し,理論的な基盤が整備されつつある.
しかしながら,組合せ遷移問題のアルゴリズム開発等の実践的な研究はまだ始
まったばかりであり,組合せ遷移問題を解く汎用ソルバーの研究開発は重要な
研究課題の一つである.

\vskip .5em
\textbf{解集合プログラミング(Answer Set Programing; ASP)}は,
論理プログラミングから派生した比較的新しいプログラミングパラダイムである.
ASP 言語は,一階論理に基づく知識表現言語の一種である.
論理プログラムは ASP のルールの有限集合である.
ASP システムは論理プログラムから安定モデル意味論に基づく解集合を計算す
るシステムである.
近年,SAT ソルバーの技術を応用した高速な ASP システムが開発され,プラ
ンニング,スケジューリング,有界モデル検査,システム生物学など様々な分
野への実用的応用が急速に拡大している.
% 山田さんへ:マルチショット ASP 解法を書こうと思ったけど,まだ実装し
% てないので書かないことにしました.

\vskip .5em
本論文では,解集合プログラミング(ASP)を用いた
$k$彩色遷移問題の解法について述べる.
$k$彩色遷移問題を解く3種類のASP符号化
\textsf{vrc1},\textsf{vrc2},\textsf{vrc3}
を考案した.
これらの符号化は,$k$彩色遷移問題の制約を7〜8つのASPルールで
簡潔に記述できる.
特に,\textsf{vrc3}符号化は,他の符号化と比較して,
基礎化後の遷移制約のためのルール数を少なく抑えるように工夫されており,
大規模な問題に対する有効性が期待できる.

\vskip .5em
考案した符号化を評価するにあたり,
Graph Coloring and its Generalization
で公開されてるグラフ点彩色問題のインスタンスを基に,
新しく$k$彩色遷移問題のベンチマーク問題(90個)を生成し,評価実験を行った.
その結果,すべての符号化で90個中11個について,到達可能であることを判定できた.
さらに,\textsf{vrc3}符号化は,多くの問題で判定に要した CPU 時間が短く,
その優位性が確認できた.

}

\end{document}

%%% Local Variables:
%%% mode: latex
%%% TeX-master: t
%%% End:
