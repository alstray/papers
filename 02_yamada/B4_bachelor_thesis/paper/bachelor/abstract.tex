%%%%%%%%%%%%%%%%%%%%%%%%%%%%%%%%%%%%%%%%%%%%%%%%%%%%%%%%%% 
\chapter*{概要}
\pagenumbering{roman}
%%%%%%%%%%%%%%%%%%%%%%%%%%%%%%%%%%%%%%%%%%%%%%%%%%%%%%%%%% 

本論文では,解集合プログラミングを用いた組合せ遷移問題の解法について述べる.
組合せ遷移問題とは,ある組合せ問題とその二つの実行可能解が与えられたとき,
一方の実行可能解から他方の実行可能解へ,
遷移制約を満たしつつ,実行可能解のみを経由して到達できるかを判定する問題である.

$k$彩色遷移問題はグラフ点彩色問題を基とした組合せ遷移問題であり,
色数$k$のグラフ点彩色問題と二つの彩色が与えられたとき,
一方の彩色から他方の彩色へ,各遷移過程において色が変化する頂点はただ一つ
という遷移制約を満たしつつ,到達できるかを判定する問題である.
一般に,$k \ge 4$において PSPACE 完全であることが知られている.

解集合プログラミング(ASP)は,論理プログラムから派生したプログラミングパラダイムである.
ASP 言語は一階論理に基づいた知識表現言語の一種である.
論理プログラムはルールの有限集合である.
ASP システムは解集合を計算するシステムである.

本論文では,解集合プログラミング(ASP)を用いた$k$彩色遷移問題の解法について述べる.
本研究では問題の入力に遷移回数$t$を加え,「遷移回数$t$での経路の存在」を解く.
まず,$k$彩色遷移問題を解く3種類の ASP 符号化,
\code{vrc1},\code{vrc2},\code{vrc3}を提案した.
特に\code{vrc3}では基礎化後のルール数を少なく抑えているため,
大規模な問題に対する有効性が期待できる.

提案した符号化を評価するにあたり,
独自に生成した90問のベンチマークを使用し評価実験を行った.
その結果,すべての符号化で90問中11問で到達可能であることを判定できた.
また,\code{vrc2}符号化と\code{vrc3}符号化で
90問中10問では到達不能であることが判定できた.
さらに,\textsf{vrc3}符号化は,多くの問題で判定に要した CPU 時間が短く,
その優位性が確認できた.

%%% Local Variables:
%%% mode: japanese-latex
%%% TeX-master: "paper"
%%% End: