%%%%%%%%%%%%%%%%%%%%%%%%%%%%%%%%%%%%%%%%%%%%%%%%%%%%%%%%%% 
\chapter{序論} \label{chap:introduction}
\pagenumbering{arabic}
%%%%%%%%%%%%%%%%%%%%%%%%%%%%%%%%%%%%%%%%%%%%%%%%%%%%%%%%%% 

%近年ソフトウェアやハードウェア技術の進歩により,
%効率的なアルゴリズムが存在しないような問題であっても社会での実用が見込める程度の時間で解けるようになってきた.
%NP完全の一つでもある充足可能性問題はその一例である.

%一方で現実にはより計算が困難とされる問題が存在する.
\textbf{組合せ遷移問題} (\textbf{Combinatorial Reconfiguration Problems}
~\cite{Ito18:tohoku})とは,
ある組合せ問題とその二つの実行可能解が与えられたとき,一方の実行可能解
から他方の実行可能解へ,\textbf{遷移制約}を満たしつつ,
実行可能解のみを経由して到達できるかを判定する問題である.
組合せ問題が実行可能解が存在するかを判定するのに対し,
組合せ遷移問題では,実行可能解が形成する\textbf{解空間グラフ}において,二つの頂
点間の到達可能性を判定することが目的となる.
計算困難性については,組合せ遷移問題はクラス PSPACE に属する問題群である.
基の組合せ問題が NP 完全であるとき,その遷移
問題の多くは,PSPACE 完全になることが知られている.
PSPACE 完全である問題には充足可能性判定問題(SAT),
集合被覆問題,グラフ点彩色問題などを基とする問題が存在する.

組合せ遷移問題の研究は,理論計算機科学の分野を中心に,2000年以降急
速に研究が発展し,理論的な基盤が整備されつつある.
また社会的には,持続可能なシステムへの応用が期待されている.
しかしながら,組合せ遷移問題のアルゴリズム開発等の実践的な研究はまだ始
まったばかりであり,組合せ遷移問題を解く汎用ソルバーの研究開発は重要な
研究課題の一つである.

\textbf{$k$彩色遷移問題}はグラフ点彩色問題を基とした組合せ遷移問題であり,
色数$k$のグラフ点彩色問題と二つの彩色が与えられたとき,
一方の彩色から他方の彩色へ,各遷移過程において色が変化する頂点はただ一つ
という遷移制約を満たしつつ,到達できるかを判定する問題である.
一般に$k \ge 4$において PSPACE 完全であることがわかっている.
$k$彩色遷移問題では,グラフの形や色数に制限を加えることで多項式時間で解ける問題が存在すること
がわかっているが,汎用的かつ効率的なアルゴリズムは見つかっていない.

解集合プログラミング(Answer Set Programming; ASP~\cite{%
Baral03:cambridge,%
Gelfond88:iclp,%
Inoue08:jssst,%
Niemela99:amai})
は,論理プログラミングから派生した比較的新しいプログラミングパラダイムであり,
一般拡張選言プログラムをベースとしている.
ASP 言語は一階論理に基づいた知識表現言語の一種である.
論理プログラムはルールの有限集合である.
ASP システムは安定モデル理論~\cite{Gelfond88:iclp}に基づく解集合を計算するシステムである.
近年,SAT ソルバーの技術を応用した高速な ASP システムが開発され,プラ
ンニング,スケジューリング,有界モデル検査,システム生物学など様々な分
野への実用的応用が急速に拡大している~\cite{Erdem16:AI}.
ASP システムの特長として豊富な表現力があげられる.
個数制約や選択子といった構文が存在し,
複雑な問題の制約を簡潔に記述することが可能である.
%また{\clingo}には\textbf{インクリメンタルモード}という機能が存在する.
%インクリメンタルモードは効率的に段階的な解の探索が可能であり,
%組合せ遷移問題を解くのに適していると考えられる.

本論文では,解集合プログラミング(ASP)を用いた$k$彩色遷移問題の解法について述べる.
本研究では問題の入力に遷移回数$t$を加え,「遷移回数$t$での経路の存在」を解く.
この時,考えられるすべての$t$において解くことで元の問題を解くことが可能である.
本研究ではグラフと二つの実行可能解を ASP ファクトとして表現した.
また,$k$彩色遷移問題を解く3種類の ASP 符号化,
\code{vrc1},\code{vrc2},\code{vrc3}を提案した.
これらの符号化は,$k$彩色遷移問題の制約を7〜8個の ASP ルールで
簡潔に記述でき,遷移制約の表現に差異が存在する.
\code{vrc1}は基本となる符号化である.
\code{vrc2},\code{vrc3}では\code{vrc1}に独自のアトムを追加し,
ASP システムの個数制約を用いて遷移制約を表している.
特に\code{vrc3}では基礎化後のルール数を抑えており,大規模な問題に対する有効性が期待できる.

提案した3種の符号化を評価するにあたり,ベンチマーク環境を整備した.
遷移回数$t$の最大値は解空間グラフの頂点数$-1$に等しい.
従って,実行可能解の総数を求められるグラフと色数の組合せが必要となる.
本研究では,Graph Coloring and its Generalization~\footnote{https://mat.tepper.cmu.edu/COLOR04/}
で公開されているグラフのうち,彩色数が判明しており~\cite{DBLP:journals/constraints/TamuraTKB09},
かつ彩色数における全解数を求めることができたグラフ9個を使用し,
各グラフから10問ずつ,計90問のベンチマークを生成し評価実験を行った.
その結果,すべての符号化で90個中11個について,到達可能であることを判定できた.
また,\code{vrc2}符号化と\code{vrc3}符号化で
90問中10問では到達不能であることが判定できた.
さらに,\textsf{vrc3}符号化は,多くの問題で判定に要した CPU 時間が短く,
その優位性が確認できた.

本論文の構成は,\ref{chap:background}章で組合せ遷移問題と$k$彩色遷移問題の
背景や定義について述べ,\ref{chap:asp}章で解集合プログラミングについて説明する.
\ref{chap:proposal}章では$k$彩色遷移問題を解くASP符号化,及び考案したプログラムについて示し,
\ref{chap:experiment}章で考案したプログラムの評価実験についての説明と考察を行う.
最後に\ref{chap:conclusion}章で本論文について総括する.

\begin{comment}
近年ソフトウェアやハードウェア技術の進歩により,
効率的なアルゴリズムが存在しないような問題であっても社会での実用が見込める程度の時間で解けるようになってきた.
NP完全の一つでもある充足可能性問題はその一例である.

一方で現実にはより計算が困難とされる問題が存在する.
\textbf{組合せ遷移問題}はクラスPSPACEに属する問題群であり,
クラスNPなどに属する探索問題を基としている.
社会的には持続可能なシステムへの応用が期待されている.

組合せ遷移問題では基となる探索問題の実行可能解を頂点とした\textbf{解空間グラフ}上での,
二つの実行可能解,つまり二つの頂点間の経路の存在について問う.
このとき解空間グラフの辺は\textbf{隣接関係}により定義され,隣接関係は探索問題ごとに異なる.

\textbf{$k$彩色遷移問題}は組合せ遷移問題の一つであり,
グラフ点彩色問題を基としている.
問題の入力はグラフ$G$,色数$k$,及び$G$と$k$をグラフ点彩色問題としたときの二つの実行可能解である.
隣接関係は$G$を$k$彩色したときに一つの頂点のみ色が異なるような実行可能解である.

現在,一般に$k \ge 4$においてPSPACE完全であることがわかっている.
$k$彩色遷移問題では,グラフの形や色数に制限を加えることで多項式時間で解ける問題があること
がわかっているが,汎用的かつ効率的なアルゴリズムは見つかっていない.

解集合プログラミング(ASP; \cite{%
Baral03:cambridge,%
Gelfond88:iclp,%
Niemela99:amai,%
Inoue08:jssst})
は,一般拡張宣言プログラムをベースとしている.
ASP言語は一階論理に基づいた知識表現言語の一種であり,
論理プログラムはルールの有限集合である.
ASPシステムは安定モデル理論~\cite{Gelfond88:iclp}に基づいており,
近年のSATソルバーの発達により高速なものが実現されている.
現在,システム検証やプランニングといった様々な分野において実用化が進んでいる~\cite{Erdem16:AI}.

本論文では,解集合プログラミングを用いた$k$彩色遷移問題の解法について述べる.
本研究では問題の入力に遷移回数$t$を加え,「$t$回で遷移可能な経路が存在するか」を解くものとする.
この場合,考えられるすべての$t$において解くことで元の問題を解くことが可能である.
本研究ではグラフ$G$と二つの実行可能解をASPファクトとして表現した.
また,$k$彩色遷移問題を解く3種類のASP符号化,
\code{vrc1},\code{vrc2},\code{vrc3}を提案した.
これらの符号化は隣接関係に関するルールが異なる.
\code{vrc1}は基本的な符号化である.
\code{vrc2}は\code{vrc1}に\code{changed}というアトムを加え,
ASPの個数制約により隣接関係を表している.
\code{vrc3}は\code{vrc1}に\code{unchanged}というアトムを加え,
個数制約により隣接関係を表している.
アトムを追加することにより,基礎化後の隣接関係に関するルール数を減らすことに成功した.

提案した符号化を評価するために,Graph Coloring and its Generalization\footnote{https://mat.tepper.cmu.edu/COLOR04/}
で公開されてるグラフのうち,彩色数が判明しており,
かつ彩色数における全解数を求めることができたグラフ9個から生成したベンチマーク90問を用いて評価実験を行った.

本論文の構成は,\ref{chap:background}章で組合せ遷移問題と$k$彩色遷移問題について
背景や定義について述べ,\ref{chap:asp}章で解集合プログラミングについて説明する.
\ref{chap:proposal}章では$k$彩色遷移問題を解くASP符号化,及び考案したプログラムについて示し,
\ref{chap:experiment}章で考案したプログラムの評価実験についての説明と考察を行う.
最後に\ref{chap:conclusion}章で本論文について総括する.
\end{comment}

%%% Local Variables:
%%% mode: latex
%%% TeX-master: "paper"
%%% End:
