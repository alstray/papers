%%%%%%%%%%%%%%%%%%%%%%%%%%%%%%%%%%%%%%%%%%%%%%%%%%%%%%%%%% 
\chapter{序論} \label{chap:introduction}
\pagenumbering{arabic}
%%%%%%%%%%%%%%%%%%%%%%%%%%%%%%%%%%%%%%%%%%%%%%%%%%%%%%%%%% 

近年ソフトウェアやハードウェア技術の進歩により,
効率的なアルゴリズムが存在しないような問題であっても社会での実用が見込める程度の時間で解けるようになってきた.
NP完全の一つでもある充足可能性問題はその一例である.

一方で現実にはより計算が困難とされる問題が存在する.
\textbf{組合せ遷移問題}はクラスPSPACEに属する問題群であり,
クラスNPなどに属する探索問題を基としている.
社会的には持続可能なシステムへの応用が期待されている.

組合せ遷移問題では基となる探索問題の実行可能解を頂点とした\textbf{解空間グラフ}上での,
二つの実行可能解,つまり二つの頂点間の経路の存在について問う.
このとき解空間グラフの辺は\textbf{隣接関係}により定義され,隣接関係は探索問題ごとに異なる.

\textbf{$k$彩色遷移問題}は組合せ遷移問題の一つであり,
グラフ点彩色問題を基としている.
問題の入力はグラフ$G$,色数$k$,及び$G$と$k$をグラフ点彩色問題としたときの二つの実行可能解である.
隣接関係はグラフを$k$彩色したときに一つの頂点のみ色が異なるような実行可能解である.

現在,一般に$k \ge 4$においてPSPACE完全であることがわかっている.
$k$彩色遷移問題では,グラフの形や色数に制限を加えることで多項式時間で解ける問題があること
がわかっているが,汎用的かつ効率的なアルゴリズムは見つかっていない.

解集合プログラミング(ASP; \cite{%
Baral03:cambridge,%
Gelfond88:iclp,%
Niemela99:amai,%
Inoue08:jssst})
は,一般拡張宣言プログラムをベースとしている.
ASP言語は一階論理に基づいた知識表現言語の一種でり,
論理プログラムはルールの有限集合である.
ASPシステムは安定モデル理論~\cite{Gelfond88:iclp}に基づいており,
近年のSATソルバーの発達により高速なものが実現されている.
現在,システム検証やプランニングといった様々な分野において実用化が進んでいる~\cite{Erdem16:AI}.

本論文では,解集合プログラミングを用いた$k$彩色遷移問題の解放について述べる.
本研究では問題の入力に遷移回数$t$を加え,「$t$回で遷移可能な経路が存在するか」を解くものとする.
この場合,考えられるすべての$t$において解くことで元の問題を解くことが可能である.
本研究ではグラフ$G$と二つの実行可能解をASPファクトとして表現した.
また,$k$彩色遷移問題を解くASP符号化を3種類提案した.
これらの符号化は隣接関係に関するルールに差異が存在する.

提案した符号化を評価するために,Graph Coloring and its Generalization\footnote{https://mat.tepper.cmu.edu/COLOR04/}
で公開されてるグラフのうち,彩色数が判明しており,
かつ彩色数における全解数を求めることができたグラフ9個から生成したベンチマーク90問を用いて評価実験を行った.

本論文の構成は,\ref{chap:background}章で組合せ遷移問題と$k$彩色遷移問題について
背景や定義について述べ,\ref{chap:asp}章で解集合プログラミングについて説明する.
\ref{chap:proposal}章では$k$彩色遷移問題を解くASP符号化,及び考案したプログラムについて示し,
\ref{chap:experiment}章で考案したプログラムの評価実験についての説明と考察を行う.
最後に\ref{chap:conclusion}章で本論文について総括する.

%%% Local Variables:
%%% mode: latex
%%% TeX-master: "paper"
%%% End:
