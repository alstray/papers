%%%%%%%%%%%%%%%%%%%%%%%%%%%%%%%%%%%%%%%%%%%%%%%%%%%%%%%%%% 
\chapter{序論} \label{chap:introduction}
\pagenumbering{arabic}
%%%%%%%%%%%%%%%%%%%%%%%%%%%%%%%%%%%%%%%%%%%%%%%%%%%%%%%%%% 

%近年ソフトウェアやハードウェア技術の進歩により,
%効率的なアルゴリズムが存在しないような問題であっても社会での実用が見込める程度の時間で解けるようになってきた.
%NP完全の一つでもある充足可能性問題はその一例である.

%一方で現実にはより計算が困難とされる問題が存在する.
\textbf{組合せ遷移問題} (Combinatorial Reconfiguration Problems)は,基となる探索問題から生成される
\textbf{解空間グラフ}上での二つの頂点間の経路の存在について問う.
解空間グラフの頂点は,基となる探索問題の実行可能解である.
また解空間グラフの辺は\textbf{隣接関係}により定義され,
隣接関係は探索問題ごとに異なる.
組合せ遷移問題はクラスPSPACEに属する問題群である.
基となる問題には充足可能性判定問題(SAT),集合被覆問題,グラフ点彩色問題などが存在し,
\textbf{PSPACE完全}であるものも存在する.
社会的には持続可能なシステムへの応用が期待されている.
しかし,現状では理論面の研究が主流であり,
汎用的なソルバーは存在していない.

\textbf{$k$彩色遷移問題}は組合せ遷移問題の一つであり,
グラフ点彩色問題を基としている.
問題の入力はグラフ$G$,色数$k$,
及び$G$と$k$をグラフ点彩色問題としたときの二つの実行可能解である.
隣接関係は$G$を$k$彩色したときに一つの頂点のみ色が異なるような実行可能解である.
現在,一般に$k \ge 4$においてPSPACE完全であることがわかっている.
$k$彩色遷移問題では,グラフの形や色数に制限を加えることで多項式時間で解ける問題があること
がわかっているが,汎用的かつ効率的なアルゴリズムは見つかっていない.

解集合プログラミング(Answer Set Programming; ASP~\cite{%
Baral03:cambridge,%
Gelfond88:iclp,%
Inoue08:jssst,%
Niemela99:amai})
は,一般拡張宣言プログラムをベースとしている.
ASP言語は一階論理に基づいた知識表現言語の一種であり,
論理プログラムはルールの有限集合である.
ASPシステムは安定モデル理論~\cite{Gelfond88:iclp}に基づいており,
近年のSATソルバーの発達により高速なものが実現されている.
特に{\clingo}~\footnote{\url{https://potassco.org/}}が知られている.
現在,システム検証やプランニングといった様々な分野において実用化が進んでいる~\cite{Erdem16:AI}.

ASPシステムの特徴として豊富な表現力があげられる.
例として個数制約や選択肢といった構文が存在し,
問題の制約を簡潔に記述することが可能である.
また{\clingo}には\textbf{インクリメンタルモード}という機能が存在する.
インクリメンタルモードは効率的に段階的な解の探索が可能であり,
組合せ遷移問題を解くのに適していると考えられる.

本論文では,解集合プログラミングを用いた$k$彩色遷移問題の解法について述べる.
本研究では問題の入力に遷移回数$t$を加え,「遷移回数$t$での経路の存在」を解くものとする.
この時,考えられるすべての$t$において解くことで元の問題を解くことが可能である.
本研究ではグラフ$G$と二つの実行可能解をASPファクトとして表現した.
また,$k$彩色遷移問題を解く3種類のASP符号化,
\code{vrc1},\code{vrc2},\code{vrc3}を提案した.
\code{vrc2},\code{vrc3}では独自のアトムを追加することにより,
基礎化後の隣接関係に関するルール数を減らすことに成功した.
最もルール数が少ない\code{vrc3}では,効率よく解くことが期待される.

提案した3種の符号化を評価するにあたり,ベンチマーク環境を整備した.
遷移回数$t$の最大値は解空間グラフの頂点数$-1$に等しい.
従って,実行可能解の総数を求められるグラフと色数の組合せが必要となる.
本研究では,Graph Coloring and its Generalization~\footnote{https://mat.tepper.cmu.edu/COLOR04/}
で公開されてるグラフのうち,彩色数が判明しており~\cite{DBLP:journals/constraints/TamuraTKB09},
かつ彩色数における全解数を求めることができたグラフ9個を使用した.
各グラフから10問ずつ,計90問のベンチマークを生成し評価実験を行った.
結果,経路の存在の有無に関わらず\code{vrc3}が最もいい性能を示した.

本論文の構成は,\ref{chap:background}章で組合せ遷移問題と$k$彩色遷移問題について
背景や定義について述べ,\ref{chap:asp}章で解集合プログラミングについて説明する.
\ref{chap:proposal}章では$k$彩色遷移問題を解くASP符号化,及び考案したプログラムについて示し,
\ref{chap:experiment}章で考案したプログラムの評価実験についての説明と考察を行う.
最後に\ref{chap:conclusion}章で本論文について総括する.

\begin{comment}
近年ソフトウェアやハードウェア技術の進歩により,
効率的なアルゴリズムが存在しないような問題であっても社会での実用が見込める程度の時間で解けるようになってきた.
NP完全の一つでもある充足可能性問題はその一例である.

一方で現実にはより計算が困難とされる問題が存在する.
\textbf{組合せ遷移問題}はクラスPSPACEに属する問題群であり,
クラスNPなどに属する探索問題を基としている.
社会的には持続可能なシステムへの応用が期待されている.

組合せ遷移問題では基となる探索問題の実行可能解を頂点とした\textbf{解空間グラフ}上での,
二つの実行可能解,つまり二つの頂点間の経路の存在について問う.
このとき解空間グラフの辺は\textbf{隣接関係}により定義され,隣接関係は探索問題ごとに異なる.

\textbf{$k$彩色遷移問題}は組合せ遷移問題の一つであり,
グラフ点彩色問題を基としている.
問題の入力はグラフ$G$,色数$k$,及び$G$と$k$をグラフ点彩色問題としたときの二つの実行可能解である.
隣接関係は$G$を$k$彩色したときに一つの頂点のみ色が異なるような実行可能解である.

現在,一般に$k \ge 4$においてPSPACE完全であることがわかっている.
$k$彩色遷移問題では,グラフの形や色数に制限を加えることで多項式時間で解ける問題があること
がわかっているが,汎用的かつ効率的なアルゴリズムは見つかっていない.

解集合プログラミング(ASP; \cite{%
Baral03:cambridge,%
Gelfond88:iclp,%
Niemela99:amai,%
Inoue08:jssst})
は,一般拡張宣言プログラムをベースとしている.
ASP言語は一階論理に基づいた知識表現言語の一種であり,
論理プログラムはルールの有限集合である.
ASPシステムは安定モデル理論~\cite{Gelfond88:iclp}に基づいており,
近年のSATソルバーの発達により高速なものが実現されている.
現在,システム検証やプランニングといった様々な分野において実用化が進んでいる~\cite{Erdem16:AI}.

本論文では,解集合プログラミングを用いた$k$彩色遷移問題の解法について述べる.
本研究では問題の入力に遷移回数$t$を加え,「$t$回で遷移可能な経路が存在するか」を解くものとする.
この場合,考えられるすべての$t$において解くことで元の問題を解くことが可能である.
本研究ではグラフ$G$と二つの実行可能解をASPファクトとして表現した.
また,$k$彩色遷移問題を解く3種類のASP符号化,
\code{vrc1},\code{vrc2},\code{vrc3}を提案した.
これらの符号化は隣接関係に関するルールが異なる.
\code{vrc1}は基本的な符号化である.
\code{vrc2}は\code{vrc1}に\code{changed}というアトムを加え,
ASPの個数制約により隣接関係を表している.
\code{vrc3}は\code{vrc1}に\code{unchanged}というアトムを加え,
個数制約により隣接関係を表している.
アトムを追加することにより,基礎化後の隣接関係に関するルール数を減らすことに成功した.

提案した符号化を評価するために,Graph Coloring and its Generalization\footnote{https://mat.tepper.cmu.edu/COLOR04/}
で公開されてるグラフのうち,彩色数が判明しており,
かつ彩色数における全解数を求めることができたグラフ9個から生成したベンチマーク90問を用いて評価実験を行った.

本論文の構成は,\ref{chap:background}章で組合せ遷移問題と$k$彩色遷移問題について
背景や定義について述べ,\ref{chap:asp}章で解集合プログラミングについて説明する.
\ref{chap:proposal}章では$k$彩色遷移問題を解くASP符号化,及び考案したプログラムについて示し,
\ref{chap:experiment}章で考案したプログラムの評価実験についての説明と考察を行う.
最後に\ref{chap:conclusion}章で本論文について総括する.
\end{comment}

%%% Local Variables:
%%% mode: latex
%%% TeX-master: "paper"
%%% End:
