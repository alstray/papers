%%%%%%%%%%%%%%%%%%%%%%%%%%%%%%%%%%%%%%%%%%%%%%%%%%%%%%%%%% 
\chapter{結論} \label{chap:conclusion}
%%%%%%%%%%%%%%%%%%%%%%%%%%%%%%%%%%%%%%%%%%%%%%%%%%%%%%%%%%
本研究では,解集合プログラミングを用いた$k$彩色遷移問題の解放について述べた.
これにあたり,$k$彩色遷移問題のASP符号化を3種類提案した.
これらは基礎化後のルール数が異なり,ルール数を減らした符号化\code{vrc3}では
他に比べ部分的非連結な問題において長い遷移回数まで確かめることができた.
結果として,ルール数を少なくすることが有効であることが確認できた.

今後の課題として,より効率的な符号化の研究とベンチマーク環境の整備があげられる.
効率的な符号化では,{\clingo}のインクリメンタルサーチモードを用いることを考えている.
インクリメンタルサーチでは,
遷移回数を$t-1$としたときの学習節を引き継いだ上で$t$以降を解くことができる.
本研究でもっとも有効性を示した符号化\code{vrc3}を基盤とし,
インクリメンタルサーチモードに対応させたい.

ベンチマークについて,特に連結インスタンスの母数が少なかった.
このため,評価実験で比較的多くの連結インスタンスが存在したグラフmyciel3やmyciel4を調査することを考えている.
また,彩色数ではなくより多い色数とすることで隣接関係が成立しやすくなると予想している.
これについては,彩色数において全解数がもっとも少なかっったグラフle450\_5cで調査を行うつもりである.


%%% Local Variables:
%%% mode: latex
%%% TeX-master: "paper"
%%% End:
