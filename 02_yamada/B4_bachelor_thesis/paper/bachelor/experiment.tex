%%%%%%%%%%%%%%%%%%%%%%%%%%%%%%%%%%%%%%%%%%%%%%%%%%%%%%%%%% 
\chapter{実行実験} \label{chap:experiment}
%%%%%%%%%%%%%%%%%%%%%%%%%%%%%%%%%%%%%%%%%%%%%%%%%%%%%%%%%% 

\section{ベンチマーク}
\ref{chap:proposal}章で提案した3種の符号化を評価するにあたり,ベンチマークを作成した.
評価には,ステップ数の最大値となる全解数を求めることが可能な
グラフ$G$と色数$k$の組合せが必要となる.
そこで初めに,ベンチマークに使用可能なグラフと色数について調査を行った.
調査するグラフとして,
Graph Coloring and its Generalization\footnote{https://mat.tepper.cmu.edu/COLOR04/}
で公開されているグラフ彩色問題127問のうち,
\textbf{彩色数}が判明している44問~\cite{DBLP:journals/constraints/TamuraTKB09}を用いた.
彩色数とは, グラフを彩色可能な色数の最小値である.

44問のグラフに対して,色数を彩色数としたときに全解数を求めることが可能か調査を行った.
%また,色数を彩色数-1としたときにUNSATになるかについても調査した.
調査にあたり,各グラフを ASP ファクト形式にしたうえで{\clingo}を使用した.
グラフ点彩色問題を解くのに用いた論理プログラムをコード~\ref{code:gcp.lp}に示す.
コード~\ref{code:gcp.lp}は,コード~\ref{code:color.lp}からグラフに関するファクトを除いたものである.
それぞれの制限時間は1時間とした.
{\clingo}のバージョンは5.4.0,オプションは\textsl{tweety}と\textsl{crafty}を用いた.
調査環境はMac Mini,3.2GHz 6コア Intel Core i7,64GBメモリである.

\lstinputlisting[float=t,caption={%
グラフ彩色問題の論理プログラム (\code{gcp.lp})},%
captionpos=b,frame=single,label=code:gcp.lp,%
numbers=left,%
breaklines=true,%
columns=fullflexible,keepspaces=true,%
basicstyle=\ttfamily\scriptsize]{code/gcp.lp}

\begin{table}[tb]
  \centering
  \caption{彩色数における全解数が判明したグラフ}
  \begin{tabular}{lrrr|r}
  グラフ名 & 頂点数 & 辺数 & 彩色数 & 実行可能解の総数 \\ \hline
  1-FullIns\_3 & 30 & 100 & 4 & 50,693,280 \\ 
  le450\_5a & 450 & 5,714 & 5 & 3,840 \\ 
  le450\_5c & 450 & 9,803 & 5 & 120 \\ 
  le450\_5d & 450 & 9,757 & 5 & 960 \\ 
  myciel3 & 11 & 20 & 4 & 12,480 \\ 
  myciel4 & 23 & 71 & 5 & 2,845,658,400 \\ 
  queen5\_5 & 25 & 160 & 5 & 240 \\  
  queen6\_6 & 36 & 290 & 7 & 100,800 \\ 
  queen7\_7 & 49 & 476 & 7 & 20,160 \\
\end{tabular}
  \label{tab:bench_graph}
\end{table}

結果,9つのグラフにおいて全解数を求めることができた.
全解数が判明したものについて,表~\ref{tab:bench_graph}に示す.
全解数が100万個未満のグラフについては,全ての解からランダムで2個の解を抽出した.
一方,全解数が100万個以上存在するものについては出力する解の個数を10万個に制限し,
その中からランダムで2個の解を抽出した.
そして各グラフごとに10問ずつ,計90問のベンチマークを生成した.

\section{実行実験}
次に,3種の符号化の評価実験を行った.
ベンチマークは,前節で生成した90問を用いた.
制限時間は1時間とした.
ただし,各グラフ,色数,初期状態,目標状態に対して$t=1, 2, \dots$と増やしていくものに対する制限時間である.
{\clingo}のバージョンは5.4.0,オプションは\textsl{jumpy}を用いた.
調査環境はMac Mini,3.2GHz 6コア Intel Core i7,64GBメモリである.

得られた結果の分類と数を表~\ref{tab:result}に示す.
連結はある$t$で経路が存在したものである.
部分的非連結はある$t' < T_{limit}$までは経路が存在しないことが確かめられたものである.
非連結は$T_{limit}$までに経路が存在せず,その問題では経路が存在しないと確かめられたものである.

\begin{table}
  \centering
  \caption{結果ごとのインスタンス数}
  \begin{tabular}{l|rr|rr} 
  & \multicolumn{2}{c|}{基本ソルバー} & \multicolumn{2}{c}{改良ソルバー} \\
  & \code{changed} & \code{unchanged} & \code{changed} & \code{unchanged} \\ \hline
  解けた問題数(到達可能) & 11 & 11 & 11 & 11 \\
  解けた問題数(到達不能) & 10 & 10 & 56 & \alert{60} \\\hline
  平均 CPU 時間(秒) & 223.796 & 151.341 & 101.758 & \alert{59.095} \\
\end{tabular}
  \label{tab:result}
\end{table}

次に結果が連結であるものについて,かかった CPU 時間を各符号化ごとにまとめたものを表~\ref{tab:result_con}に示す.
各行はインスタンスごとに分けられている.
インスタンス名はグラフ名,色数,識別番号により構成される.
例えばインスタンス名が1-FullIns\_3\_col4\_1ならば,
グラフ名が1-FullIns\_3,色数が4,識別番号が1であることを意味する.
また,最終行は各符号化ごとにもっとも早く解いたインスタンスの合計を表している.
各列は左から順にインスタンス名,経路が存在した最小の遷移回数,
\code{vrc1}での CPU 時間,\code{vrc2}での CPU 時間,
\code{vrc3}での CPU 時間となっている.
このとき各 CPU 時間は,$t=1, 2, \dots $遷移回数での CPU 時間での合計となっている.
各インスタンに対して最速で解いたものを色付きで示している.

\begin{table}
  \centering
  \caption{連結インスタンスにおける経路発見までのCPU時間の合計}
  \scalebox{0.8}{
\begin{tabular}{lrr|rrrr} \hline
  インスタンス & 頂点数 & 辺数 & 遷移回数 & vrc1 & vrc2 & vrc3 \\ \hline
  1-FullIns\_3\_col4\_1 & 30 & 100 & 11 & 47.394 & 1.367 & \textcolor{red}{1.059} \\ 
  myciel3\_col4\_1 & 11 & 20 & 11 & 2.450 & 0.180 & \textcolor{red}{0.150} \\ 
  myciel3\_col4\_3 & 11 & 20 & 11 & 2.611 & 0.206 & \textcolor{red}{0.153} \\ 
  myciel3\_col4\_4 & 11 & 20 & 13 & 4.717 & 0.487 & \textcolor{red}{0.442} \\ 
  myciel3\_col4\_7 & 11 & 20 & 8 & 0.970 & 0.090 & \textcolor{red}{0.072} \\ 
  myciel3\_col4\_8 & 11 & 20 & 9 & 1.419 & 0.119 & \textcolor{red}{0.092} \\ 
  myciel4\_col5\_2 & 23 & 71 & 16 & 763.484 & \textcolor{red}{42.717} & 49.151 \\ 
  myciel4\_col5\_3 & 23 & 71 & 14 & 358.494 & \textcolor{red}{17.103} & 21.239 \\ 
  myciel4\_col5\_4 & 23 & 71 & 7 & 26.987 & 0.256 & \textcolor{red}{0.192} \\ 
  myciel4\_col5\_6 & 23 & 71 & 10 & 102.396 & 1.524 & \textcolor{red}{1.391} \\ 
  myciel4\_col5\_10 & 23 & 71 & 17 & 818.589 & 212.066 & \textcolor{red}{73.48} \\ \hline
  Sum &  &  &  & 0 & 2 & 9 \\ \hline
\end{tabular}
}
  \label{tab:result_con}
\end{table}

次に結果が部分的非連結であるものについて述べる.
比較の指標として,
各インスタンスで非連結であることが確かめられた遷移回数の最大値を用いる.
このとき,グラフと色数が同じであるインスタンスは,
初期状態と終了状態の組合せに関係なく同じ程度の数値を示す傾向が見られた.
従って,グラフと色数が同じであるインスタンスの最大値の平均を表~\ref{tab:ave_unknown}に示す.
また,加工前のデータは付録内の表~\ref{tab:result_unknown}に示す.

表~\ref{tab:ave_unknown}では,各行はグラフと色数の組合せを表している.
また,最終行は各符号化ごとにもっとも長い遷移回数まで確かめられたインスタンスの合計を表している.
各列は左から順にグラフ名と色数,部分的非連結であったインスタンス数,
vrc1で確かめられた最大値の平均値,vrc2で確かめられた最大値の平均値,
vrc3で確かめられた最大値の平均値となっている.

\begin{table}[htbp]
  \centering
  \caption{部分的非連結インスタンスにおける最長遷移回数の平均値}
  \begin{tabular}{|c|r|r|r|r|} \hline
  Instance & \multicolumn{1}{c|}{問題数} & \code{vrc1} & \code{vrc2} & \code{vrc3} \\ \hline
  1-FullIns\_3\_col4 & 9 &  & \textcolor{red}{} &  \\ \hline  
  le450\_5a\_col5 & 10 &  &  & \textcolor{red}{} \\ \hline
  le450\_5c\_col5 & 10 &  &  & \textcolor{red}{} \\ \hline
  le450\_5d\_col5 & 10 &  & 94 & \textcolor{red}{} \\ \hline
  myciel3\_col4 & 5 & \textcolor{red}{} &  &  \\ \hline
  myciel4\_col5 & 5 &  & \textcolor{red}{} &  \\ \hline
  queen5\_5\_col5 & 10 &  & \textcolor{red}{240} & \textcolor{red}{240} \\ \hline
  queen6\_6\_col7 & 10 &  &  & \textcolor{red}{} \\ \hline
  queen7\_7\_col7 & 10 &  &  & \textcolor{red}{} \\ \hline \hline
  Sum & 79 & 6 & 23 & 64 \\ \hline
\end{tabular}
  \label{tab:ave_unknown}
\end{table}

連結インスタンスと部分的非連結インスタンスの両方で\code{vrc3}$>$\code{vrc2}$>$
\code{vrc1}の順に優れた結果を出した.
ただし連結インスタンスにおいては1秒未満で解けたものも多く,
\code{vrc2}と\code{vrc3}で大きな差がついたのは3問にとどまった.
また,クイーングラフ問題を基としたグラフでは,
\code{vrc2}は著しく悪い性能を示した.

このような結果の要因の一つとして,表~\ref{tab:rule}で
示した基礎化後のルール数の差が考えられる.
{\clingo}ではルールを基礎化して充足可能性問題として解く.
ここで基礎化後のルール数は充足可能性問題における節数に関係してくる.
充足可能性問題において節数が多いことは,単位伝播や変数選択ヒューリスティックにおける
オーバーヘッドが大きくなることにつながる.
結果として基礎化後のルール数が多いほどオーバーヘッドも大きいことになり,
性能差が生じたと考えられる.

%\begin{longtable}{l|r|r|r}
  \caption{インスタンスごとの非連結が確かめられた遷移回数の最大値}
  \label{tab:result_unknown}
  \\ %注意
  \hline
    Instance & \code{vrc1} & \code{vrc2} & \code{vrc3} \\ \hline
  \endfirsthead
  \hline
  Instance & \code{vrc1} & \code{vrc2} & \code{vrc3} \\ \hline
  \endhead
  \endlastfoot
  \endfoot
  1-FullIns\_3\_col4\_2 & 50 & \textcolor{red}{125} & 102 \\ \hline
  1-FullIns\_3\_col4\_3 & 50 & \textcolor{red}{122} & 102 \\ \hline
  1-FullIns\_3\_col4\_4 & 50 & \textcolor{red}{125} & 105 \\ \hline
  1-FullIns\_3\_col4\_5 & 50 & \textcolor{red}{121} & 99 \\ \hline
  1-FullIns\_3\_col4\_6 & 50 & \textcolor{red}{124} & 101 \\ \hline
  1-FullIns\_3\_col4\_7 & 50 & \textcolor{red}{126} & 101 \\ \hline
  1-FullIns\_3\_col4\_8 & 50 & \textcolor{red}{124} & 102 \\ \hline
  1-FullIns\_3\_col4\_9 & 50 & \textcolor{red}{123} & 104 \\ \hline
  1-FullIns\_3\_col4\_10 & 40 & \textcolor{red}{53} & 50 \\ \hline
  le450\_5a\_col5\_1 & 6 & 86 & \textcolor{red}{115} \\ \hline
  le450\_5a\_col5\_2 & 6 & 86 & \textcolor{red}{122} \\ \hline
  le450\_5a\_col5\_3 & 6 & 86 & \textcolor{red}{116} \\ \hline
  le450\_5a\_col5\_4 & 5 & 87 & \textcolor{red}{122} \\ \hline
  le450\_5a\_col5\_5 & 5 & 86 & \textcolor{red}{124} \\ \hline
  le450\_5a\_col5\_6 & 5 & 88 & \textcolor{red}{122} \\ \hline
  le450\_5a\_col5\_7 & 5 & 86 & \textcolor{red}{125} \\ \hline
  le450\_5a\_col5\_8 & 5 & 86 & \textcolor{red}{114} \\ \hline
  le450\_5a\_col5\_9 & 6 & 86 & \textcolor{red}{124} \\ \hline
  le450\_5a\_col5\_10 & 6 & 87 & \textcolor{red}{126} \\ \hline
  le450\_5c\_col5\_1 & 5 & 97 & \textcolor{red}{104} \\ \hline
  le450\_5c\_col5\_2 & 6 & 97 & \textcolor{red}{103} \\ \hline
  le450\_5c\_col5\_3 & 6 & 98 & \textcolor{red}{103} \\ \hline
  le450\_5c\_col5\_4 & 5 & 97 & \textcolor{red}{103} \\ \hline
  le450\_5c\_col5\_5 & 6 & 98 & \textcolor{red}{103} \\ \hline
  le450\_5c\_col5\_6 & 5 & 98 & \textcolor{red}{104} \\ \hline
  le450\_5c\_col5\_7 & 6 & 98 & \textcolor{red}{103} \\ \hline
  le450\_5c\_col5\_8 & 6 & 97 & \textcolor{red}{103} \\ \hline
  le450\_5c\_col5\_9 & 6 & 97 & \textcolor{red}{103} \\ \hline
  le450\_5c\_col5\_10 & 6 & 97 & \textcolor{red}{103} \\ \hline
  le450\_5d\_col5\_1 & 5 & 94 & \textcolor{red}{102} \\ \hline
  le450\_5d\_col5\_2 & 5 & 94 & \textcolor{red}{102} \\ \hline
  le450\_5d\_col5\_3 & 6 & 94 & \textcolor{red}{102} \\ \hline
  le450\_5d\_col5\_4 & 6 & 94 & \textcolor{red}{102} \\ \hline
  le450\_5d\_col5\_5 & 6 & 94 & \textcolor{red}{102} \\ \hline
  le450\_5d\_col5\_6 & 5 & 94 & \textcolor{red}{102} \\ \hline
  le450\_5d\_col5\_7 & 6 & 94 & \textcolor{red}{102} \\ \hline
  le450\_5d\_col5\_8 & 6 & 94 & \textcolor{red}{103} \\ \hline
  le450\_5d\_col5\_9 & 6 & 94 & \textcolor{red}{102} \\ \hline
  le450\_5d\_col5\_10 & 6 & 94 & \textcolor{red}{103} \\ \hline
  myciel3\_col4\_2 & \textcolor{red}{44} & 36 & 32 \\ \hline
  myciel3\_col4\_5 & \textcolor{red}{43} & 36 & 32 \\ \hline
  myciel3\_col4\_6 & \textcolor{red}{43} & 36 & 31 \\ \hline
  myciel3\_col4\_9 & \textcolor{red}{42} & 36 & 31 \\ \hline
  myciel3\_col4\_10 & \textcolor{red}{42} & 35 & 32 \\ \hline
  myciel4\_col5\_1 & \textcolor{red}{17} & \textcolor{red}{17} & \textcolor{red}{17} \\ \hline
  myciel4\_col5\_5 & 18 & \textcolor{red}{19} & \textcolor{red}{19} \\ \hline
  myciel4\_col5\_7 & 18 & 18 & \textcolor{red}{19} \\ \hline
  myciel4\_col5\_8 & 17 & \textcolor{red}{19} & 18 \\ \hline
  myciel4\_col5\_9 & 17 & \textcolor{red}{18} & \textcolor{red}{18} \\ \hline
  queen5\_5\_col5\_1 & 109 & \textcolor{red}{240} & \textcolor{red}{240} \\ \hline
  queen5\_5\_col5\_2 & 108 & \textcolor{red}{240} & \textcolor{red}{240} \\ \hline
  queen5\_5\_col5\_3 & 108 & \textcolor{red}{240} & \textcolor{red}{240} \\ \hline
  queen5\_5\_col5\_4 & 108 & \textcolor{red}{240} & \textcolor{red}{240} \\ \hline
  queen5\_5\_col5\_5 & 105 & \textcolor{red}{240} & \textcolor{red}{240} \\ \hline
  queen5\_5\_col5\_6 & 107 & \textcolor{red}{240} & \textcolor{red}{240} \\ \hline
  queen5\_5\_col5\_7 & 108 & \textcolor{red}{240} & \textcolor{red}{240} \\ \hline
  queen5\_5\_col5\_8 & 107 & \textcolor{red}{240} & \textcolor{red}{240} \\ \hline
  queen5\_5\_col5\_9 & 107 & \textcolor{red}{240} & \textcolor{red}{240} \\ \hline
  queen5\_5\_col5\_10 & 107 & \textcolor{red}{240} & \textcolor{red}{240} \\ \hline
  queen6\_6\_col7\_1 & 32 & 4 & \textcolor{red}{499} \\ \hline
  queen6\_6\_col7\_2 & 31 & 4 & \textcolor{red}{499} \\ \hline
  queen6\_6\_col7\_3 & 29 & 5 & \textcolor{red}{499} \\ \hline
  queen6\_6\_col7\_4 & 29 & 4 & \textcolor{red}{500} \\ \hline
  queen6\_6\_col7\_5 & 30 & 4 & \textcolor{red}{499} \\ \hline
  queen6\_6\_col7\_6 & 30 & 4 & \textcolor{red}{499} \\ \hline
  queen6\_6\_col7\_7 & 29 & 4 & \textcolor{red}{500} \\ \hline
  queen6\_6\_col7\_8 & 31 & 4 & \textcolor{red}{500} \\ \hline
  queen6\_6\_col7\_9 & 30 & 4 & \textcolor{red}{500} \\ \hline
  queen6\_6\_col7\_10 & 30 & 4 & \textcolor{red}{499} \\ \hline
  queen7\_7\_col7\_1 & 26 & 6 & \textcolor{red}{66} \\ \hline
  queen7\_7\_col7\_2 & 26 & 4 & \textcolor{red}{89} \\ \hline
  queen7\_7\_col7\_3 & 25 & 4 & \textcolor{red}{263} \\ \hline
  queen7\_7\_col7\_4 & 26 & 5 & \textcolor{red}{122} \\ \hline
  queen7\_7\_col7\_5 & 25 & 4 & \textcolor{red}{388} \\ \hline
  queen7\_7\_col7\_6 & 27 & 4 & \textcolor{red}{389} \\ \hline
  queen7\_7\_col7\_7 & 26 & 4 & \textcolor{red}{322} \\ \hline
  queen7\_7\_col7\_8 & 26 & 5 & \textcolor{red}{111} \\ \hline
  queen7\_7\_col7\_9 & 26 & 4 & \textcolor{red}{141} \\ \hline
  queen7\_7\_col7\_10 & 26 & 4 & \textcolor{red}{188} \\ \hline \hline
  Sum & 6 & 23 & 64 \\ \hline
\end{longtable}

%%% Local Variables:
%%% mode: latex
%%% TeX-master: "paper"
%%% End:
