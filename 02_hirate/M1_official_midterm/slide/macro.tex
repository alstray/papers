\renewcommand{\bibname}{参考文献}

%%
%% Figure 環境中で Table 環境の見出しを表示・カウンタの操作に必要
%%
\makeatletter
\newcommand{\figcaption}[1]{\def\@captype{figure}\caption{#1}}
\newcommand{\tblcaption}[1]{\def\@captype{table}\caption{#1}}
%\newcommand{\code}[1]{\lstinline[basicstyle=\ttfamily]{#1}}
\makeatother


%listingsの設定
\lstset{float=t,
  frame=single,
  numbers=left,
  breaklines=true,
  columns=fullflexible,
  keepspaces=true,
  basicstyle=\ttfamily\footnotesize,
  moredelim=[is][\color{red}\bfseries]{<\#}{\#>},
}

%Tikzの設定(ベン図)
\usetikzlibrary{calc}
\usetikzlibrary{shapes}
\tikzset{
  set label/.style={fill=white,ellipse,inner sep=2}
}
\def\vennradius{2}
\def\vennratio{0.8}
\def\venncenterA{90:\vennratio*\vennradius}
\def\venncenterB{-150:\vennratio*\vennradius}
\def\venncenterC{-30:\vennratio*\vennradius}
\def\venncolorA{blue}
\def\venncolorB{red}
\def\venncolorC{green!70!black}
\def\venncircleA{(\venncenterA) circle [radius=\vennradius]}
\def\venncircleB{(\venncenterB) circle [radius=\vennradius]}
\def\venncircleC{(\venncenterC) circle [radius=\vennradius]}

%%% For ASP
\newcommand{\asap}{\textit{teaspoon}}
\newcommand{\clasp}{\textit{clasp}}
\newcommand{\gringo}{\textit{gringo}}
\newcommand{\clingo}{\textit{clingo}}
\newcommand{\dlv}{\textit{DLV}}
\newcommand{\wasp}{\textit{WASP}}
\newcommand{\code}[1]{\lstinline[basicstyle=\ttfamily]{#1}}
\newcommand{\naf}[1]{\ensuremath{{\sim\!\!{#1}}}}
\newcommand{\head}[1]{\ensuremath{\mathit{head}(#1)}}
\newcommand{\body}[1]{\ensuremath{\mathit{body}(#1)}}
%\newcommand{\atom}[1]{\ensuremath{\mathit{atom}(#1)}}
\newcommand{\poslits}[1]{\ensuremath{{#1}^+}}
\newcommand{\neglits}[1]{\ensuremath{{#1}^-}}
\newcommand{\pbody}[1]{\poslits{\body{#1}}}
\newcommand{\nbody}[1]{\neglits{\body{#1}}}
%\newcommand{\Cn}[1]{\ensuremath{\mathit{Cn}(#1)}}
\newcommand{\reduct}[2]{\ensuremath{#1^{#2}}}
\newcommand{\mcode}[1]{\ensuremath{{\mbox{\code{#1}}}}}
