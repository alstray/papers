%%%%%%%%%%%%%%%%%%%%%%%%%%%%%%%%%%%%%%%%%%%%%%%%%%%%%%%%%% 
\chapter{ハミルトン路・閉路問題}
%%%%%%%%%%%%%%%%%%%%%%%%%%%%%%%%%%%%%%%%%%%%%%%%%%%%%%%%%% 

本章では,研究にて対象とした無向グラフ上でのハミルトン路問題(HPP; Hamiltonian Path Problem),
ハミルトン閉路問題(HCP; Hamiltonian Cycle Problem)とその拡張問題の定義について説明する.

まず,本研究においてハミルトン路とハミルトン閉路は以下のように定義される.\cite{DAD:book}\\
\textbf{ハミルトン路・閉路}~~無向グラフ$G=(V,E)$について,全頂点を一度ずつ通る路をハミルトン路,全頂点を一度ずつ通る閉路をハミルトン閉路とする.

続いて,ハミルトン路問題を以下のように定義する.\\
\textbf{ハミルトン路問題}~~無向グラフ$G=(V,E)$と始点$s \in V$,終点$t \in V$が与えられた時に,
$G$が,$s$を始点とし$t$を終点とするハミルトン路を持つか否かを判定する問題.

最後に,ハミルトン閉路問題は以下のように定義される.\cite{DAD:book}\\
\textbf{ハミルトン閉路問題}~~無向グラフ$G=(V,E)$が与えられた時に,$G$がハミルトン閉路を持つか否かを判定する問題.

本研究にて提案するハミルトン路・閉路を導く3つの手法を比較するために行った実験にて,2つの拡張問題を用いた.
拡張問題は,\textbf{最短ハミルトン路・閉路問題}と\textbf{コスト制約付きハミルトン路問題}であり,
いずれも,辺に重みがつけられたグラフである重み付きグラフ上の問題である.
それぞれの定義を以下に示す.

最短ハミルトン路・閉路問題は,それぞれ次のように定義される.\\
\textbf{最短ハミルトン路問題}~~重み付きグラフ$G=(V,E)$と始点$s \in V$,終点$t \in V$が与えられた時に,
$G$が持つ$s$を始点とし$t$を終点とするハミルトン路の内,それが含む全ての辺$e \in E$の重みの総和を目的関数とする.
最短ハミルトン路問題はその目的関数値を最小化する問題である.\\
\textbf{最短ハミルトン閉路問題}~~重み付きグラフ$G=(V,E)$が与えられた時に,
$G$が持つハミルトン閉路の内,それが含む全ての辺$e \in E$の重みの総和を目的関数とする.
最短ハミルトン閉路問題はその目的関数値を最小化する問題である.

最後に,コスト制約付きハミルトン路問題を以下で定義する.\\
\textbf{コスト制約付きハミルトン路問題}~~重み付きグラフ$G=(V,E)$と始点$s \in V$,終点$t \in V$そして定数$n$が与えられた時に,
$G$が持つ$s$を始点とし$t$を終点とするハミルトン路の内,それが含む全ての辺$e \in E$の重みの総和を目的関数とする.
コスト制約付きハミルトン路問題はその目的関数値が$n$以下であるようなハミルトン路を全て求める問題である.\\
%%% Local Variables:
%%% mode: latex
%%% TeX-master: "paper"
%%% End:
