\begin{table}[t]\footnotesize
  \caption{最短ハミルトン閉路問題: 得られた目的関数の値}
  \label{min_table_tr}
  \tabcolsep = 2mm
  %\renewcommand{\arraystretch}{1.0}
  \vskip .5em
  \centering
  \begin{tabular}{lrr|rrr}\hline
     問題 & 頂点の数 & 辺の数 &\textsf{undirected} & \textsf{directed} & \textsf{acyclicity} \\
    \hline
    grid5 & 36 & 60 &\textcolor{red}{50,656*}&\textcolor{red}{50,656*}&\textcolor{red}{50,656*} \\
    grid6 & 49 & 84 &\textcolor{red}{68,656*}&\textcolor{red}{68,656*}&\textcolor{red}{68,656*} \\
    grid7 & 64 & 112 &\textcolor{red}{91,822*}&\textcolor{red}{91,822*}&\textcolor{red}{91,822*} \\
    grid8 & 81 & 144 &113,250&\textcolor{red}{112,916}&113,277 \\
    grid9 & 100 & 180 &\textcolor{red}{142,502}&143,326&143,660 \\
    grid10 & 121 & 220 &\textcolor{red}{172,703}&174,866&175,999 \\
    grid11 & 144 & 264 &\textcolor{red}{200,399}&204,456&200,638 \\
    grid12 & 169 & 312 &\textcolor{red}{231,278}&239,275&232,012 \\
    grid13 & 196 & 364 &\textcolor{red}{276,692}&276,926&276,899 \\
    grid14 & 225 & 420 &317,617&\textcolor{red}{317,144}&317,676 \\
    grid15 & 256 & 480 &\textcolor{red}{375,906}&376,809&376,210 \\
    grid16 & 289 & 544 &421,249&\textcolor{red}{419,737}&423,753 \\
    US48 & 48 & 105 &\textcolor{red}{11,698*}&\textcolor{red}{11,698*}&\textcolor{red}{11,698*} \\
    \hline
    最適値と最良値の数 &&& 10 & 7 & 4\\    \hline
  \end{tabular}
\end{table}
