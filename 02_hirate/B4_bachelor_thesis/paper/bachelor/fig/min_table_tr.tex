\begin{table}[t]\footnotesize
  \caption{最短ハミルトン閉路問題: 得られた目的関数の値}
  \label{min_table_tr}
  \tabcolsep = 2mm
  %\renewcommand{\arraystretch}{1.0}
  \vskip .5em
  \centering
  \begin{tabular}{l|rrr}\hline
     問題 & \textsf{undirected} & \textsf{directed} & \textsf{acyclicity} \\
    \hline
    grid5&50,656*&50,656*&50,656* \\
    grid6&68,656*&68,656*&68,656* \\
    grid7&91,822*&91,822*&91,822* \\
    grid8&113,250&\textcolor{red}{112,916}&113,277 \\
    grid9&\textcolor{red}{142,502}&143,326&143,660 \\
    grid10&\textcolor{red}{172,703}&174,866&175,999 \\
    grid11&\textcolor{red}{200,399}&204,456&200,638 \\
    grid12&\textcolor{red}{231,278}&239,275&232,012 \\
    grid13&\textcolor{red}{276,692}&276,926&276,899 \\
    grid14&317,617&\textcolor{red}{317,144}&317,676 \\
    grid15&\textcolor{red}{375,906}&376,809&376,210 \\
    grid16&421,249&\textcolor{red}{419,737}&423,753 \\
    US48&11,698*&11,698*&11,698* \\
    \hline
    最適値と最良値の数 & 10 & 7 & 4\\    \hline
  \end{tabular}
\end{table}
