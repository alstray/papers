%%%%%%%%%%%%%%%%%%%%%%%%%%%%%%%%%%%%%%%%%%%%%%%%%%
%% ASPで問題を解く流れの図
%%%%%%%%%%%%%%%%%%%%%%%%%%%%%%%%%%%%%%%%%%%%%%%%%%
\begin{tikzpicture}

 \definecolor{edge}{RGB}{38,38,134}
 \definecolor{node}{RGB}{220,220,249}

 \definecolor{alert_edge}{RGB}{191,0,0}
 \definecolor{alert_node}{RGB}{249,200,200}

 \def\nodehspace{1.5cm}

 \tikzset{block/.style={rectangle, thick, draw=edge, fill=node, text centered, 
 rounded corners, text width=1.5cm, minimum height=1.6cm,minimum width=1.5cm}};

 \tikzset{alertblock/.style={rectangle, thick, draw=alert_edge, fill=alert_node, 
 text centered, rounded corners, text width=1.5cm, minimum height=1.6cm}};

 \node[block](ins){HCP\\HPP};
 \node[block, right=\nodehspace of ins] (fact){ASP\\ファクト};
 \node[alertblock, below=of fact](encode){ASP\\符号化};
 \node[block, right=\nodehspace of fact](sys){ASP\\システム};
 \node[block, right=\nodehspace of sys] (ans){問題の解};

 \foreach \u / \v / \name in {ins/fact/変換,fact/sys/,encode/sys/,sys/ans/}
 \draw [thick,->] (\u) to node[above]{\name} (\v);

\end{tikzpicture}
