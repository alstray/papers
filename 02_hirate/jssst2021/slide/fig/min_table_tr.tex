\begin{table}[t]\footnotesize
  \tabcolsep = 2mm
  %\renewcommand{\arraystretch}{1.0}
  \vskip .5em
  \centering
  \begin{tabular}{lrr|rrr}
     問題名 & 頂点数 & 辺数 &\textsf{undirected} & \textsf{directed} & \textsf{acyclicity} \\
    \hline
    grid6 & 49 & 84 &\textbf{*68,656}&\textbf{*68,656}&\textbf{*68,656} \\
    grid8 & 81 & 144 &113,250&113,335&\textbf{113,186} \\
    grid10 & 121 & 220 &\textbf{172,703}&174,861&172,759 \\
    grid12 & 169 & 312 &\textbf{231,278}&236,211&233,657 \\
    grid14 & 225 & 420 &317,617&\textbf{316,347}&318,419 \\
    grid16 & 289 & 544 &\textbf{421,249}&421,696&422,963 \\
    US48 & 48 & 105 &\textbf{*11,698}&\textbf{*11,698}&\textbf{*11,698} \\
    \hline
    最適値または最良値の数 &&& \textbf{5} & 3 & 3\\
  \end{tabular}
  \vskip .5em
%  \caption{最短ハミルトン路問題: 得られた目的関数の値}
  \label{min_table_tr}
\end{table}
