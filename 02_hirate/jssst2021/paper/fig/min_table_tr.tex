\begin{table*}[t]\footnotesize
  \tabcolsep = 2mm
  %\renewcommand{\arraystretch}{1.0}
  \vskip .5em
  \centering
  \begin{tabular}{lrr|rrr}\hline
     問題 & 頂点の数 & 辺の数 &\textsf{undirected} & \textsf{directed} & \textsf{acyclicity} \\
    \hline
    grid6 & 49 & 84 &\textcolor{red}{68,656*}&\textcolor{red}{68,656*}&\textcolor{red}{68,656*} \\
    grid8 & 81 & 144 &113,250&113,335&\textcolor{red}{113,186} \\
    grid10 & 121 & 220 &\textcolor{red}{172,703}&174,861&172,759 \\
    grid12 & 169 & 312 &\textcolor{red}{231,278}&236,211&233,657 \\
    grid14 & 225 & 420 &317,617&\textcolor{red}{316,347}&318,419 \\
    grid16 & 289 & 544 &\textcolor{red}{421,249}&421,696&422,963 \\
    US48 & 48 & 105 &\textcolor{red}{11,698*}&\textcolor{red}{11,698*}&\textcolor{red}{11,698*} \\
    \hline
    最適値と最良値の数 &&& 5 & 3 & 3\\    \hline
  \end{tabular}
  \vskip .5em
  \caption{最短ハミルトン閉路問題: 得られた目的関数の値}
  \label{min_table_tr}
\end{table*}
