\begin{table*}[tb]\footnotesize
  \tabcolsep = 2mm
  %\renewcommand{\arraystretch}{1.0}
  \vskip .5em
  \centering
  \begin{tabular}{lr|rrr}
    \hline
    閾値(倍率)    &	解の総数 & \textsf{undirected} & \textsf{directed} & \textsf{acyclicity} \\
    \hline
    11698(1.00)   &	1      &\textcolor{red}{2.979} & 7.531 & 4.586	\\
    11814(1.01)   &	8      &5.587  & 15.322	& \textcolor{red}{5.250}	\\
    11931(1.02)   &	28     &\textcolor{red}{3.243}& 18.600	& 3.578	\\
    12282(1.05)   &	388    &10.003&19.818	& \textcolor{red}{6.296}	\\
    12867(1.10)   &	16,180  &16.548& 28.555	& \textcolor{red}{9.764}\\
    14037(1.20)   &	939,209 &48.262       &40.717	& \textcolor{red}{26.837}\\
    15207(1.30)   &	4,525,541&88.172      &55.276	& \textcolor{red}{42.037}\\
    16377(1.40)   &	6,702,964&99.154       &47.647	& \textcolor{red}{40.640}	\\
    17547(1.50)   &	6,876,526&95.390       &45.265	& \textcolor{red}{38.411}	\\
    18716(1.60)   &	6,876,928&98.937       &49.138	& \textcolor{red}{40.748}	\\
    \hline
    平均CPU時間 &   & 46.8275 & 32.7869  & \textcolor{red}{21.8147}\\\hline
%    Best    &   & 2 & 0 & \textcolor{red}{8} \\ \hline
  \end{tabular}
  \vskip .5em
  \caption{コスト制約付きハミルトン路問題: 解の全列挙に要した CPU 時間}
  \label{cost_table}
\end{table*}
