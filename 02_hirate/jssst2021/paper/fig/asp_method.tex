\begin{figure}[tp]
\tikz{
  %1ノード目
  \path[draw=black, fill=blue!20, rounded corners=5pt]%線の設定
  node[at={(0.75,0.75)}] {ハミルトン閉路問題(HCP)}%文字を入れる
  (0,0) --(1.5,0) --(1.5,1.5) --(0,1.5) --cycle;%外周
  %2ノード目
  \path[draw=black, fill=blue!20, rounded corners=5pt, shift={(3,0)}]
  node[at={(0.75,0.75)}] {
    \begin{tabular}{c}
      HCPのASPファクト
    \end{tabular}
  }
  (0,0) --(1.5,0) --(1.5,1.5) --(0,1.5) --cycle;
  %3ノード目文字が複数行
  \path[draw=black, fill=green!20, rounded corners=5pt, shift={(6,0)}]
  node[at={(0.75,0.75)}] {
    \begin{tabular}{c}
      {\clingo}
    \end{tabular}
  }
  (0,0) --(1.5,0) --(1.5,1.5) --(0,1.5) --cycle;
  %4ノード目文字が複数行
  \path[draw=black, fill=blue!20, rounded corners=5pt, shift={(9,0)}]
  node[at={(0.75,0.75)}] {解集合}
  (0,0) --(1.5,0) --(1.5,1.5) --(0,1.5) --cycle;
  %5ノード目文字が複数行
  \path[draw=black, fill=red!20, rounded corners=5pt, shift={(3,-3)}]
  node[at={(0.75,0.75)}] {
    \begin{tabular}{c}
      HCPを解くASP符号化
    \end{tabular}
  }
  (0,0) --(1.5,0) --(1.5,1.5) --(0,1.5) --cycle;
  \draw[arrows=->] (1.5,0.75) --(3.0,0.75);
  \draw[arrows=->,shift={(3,0)}] (1.5,0.75) --(3.0,0.75);
  \draw[arrows=->,shift={(6,0)}] (1.5,0.75) --(3.0,0.75);
  \draw[arrows=->] (4.5,-2.25) --(6.0,0.5);
}
\caption{ASP を用いたハミルトン閉路問題の解法}
\label{aspmethod}
\end{figure}
