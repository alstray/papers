%%%%%%%%%%%%%%%%%%%%%%%%%%%%%%%%%%%%%%%%%%%%%%%%%%%%%%%%%% 
\section{おわりに}\label{chap:conclusion}
%%%%%%%%%%%%%%%%%%%%%%%%%%%%%%%%%%%%%%%%%%%%%%%%%%%%%%%%%%

本稿では,解集合プログラミング(ASP)を用いた
ハミルトン閉路問題,
最短ハミルトン閉路問題,
コスト制約付きハミルトン閉路問題
の解法について述べた.
%
ハミルトン閉路問題を解くASP符号化として,
\textsf{undirected},
\textsf{directed},
\textsf{acyclicity}
の3つを考案し,ハミルトン閉路問題および関連問題の制約を,
ASPのルール10個程度で簡潔に表現できることを確認した.
%
考案した3つの符号化の有効性を評価するために,
Flinders Hamiltonian Cycle Project (FHCP) で公開されている
ハミルトン閉路問題の問題集(全1001問)を用いた実験を行った.
その結果,
\textsf{directed}符号化が,875 問と最も多くの問題を解き,
他の符号化と比較してその優位性が確認できた.

今後の課題としては,
ハミルトン閉路問題の緩和問題と
CEGAR (Counterexample Guided Abstraction Refinement)
を用いた解法[soh jelia, pos2021]を応用した提案手法の高速化,
巡回セールスマン問題への拡張などがある.

ハミルトン閉路問題は代表的な NP 完全問題であり,
また重要な工学的応用が数多く存在するため,
その解法については古くから盛んに研究されている.
本研究と関連の深い研究としては,
SATソルバーを用いた解法[Prestwich, VelevandGao, SOH]
整数計画ソルバーを用いた解法[沼田]がある.
これらを含む他のアプローチとの比較は,重要な今後の課題の一つである.


%%% Local Variables:
%%% mode: latex
%%% TeX-master: "paper"
%%% End:
