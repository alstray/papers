%%%%%%%%%%%%%%%%%%%%%%%%%%%%%%%%%%%%%%%%%%%%%%%%%%%%%%%%%% 
\section{おわりに}\label{chap:conclusion}
%%%%%%%%%%%%%%%%%%%%%%%%%%%%%%%%%%%%%%%%%%%%%%%%%%%%%%%%%%

本稿では,解集合プログラミング(ASP)を用いた
ハミルトン閉路問題,
最短ハミルトン閉路問題,
コスト制約付きハミルトン閉路問題
の解法について述べた.
%
ハミルトン閉路問題を解くASP符号化として,
\textsf{undirected},
\textsf{directed},
\textsf{acyclicity}
の3つを考案し,ハミルトン閉路問題および関連問題の制約を,
ASPのルール10個程度で簡潔に表現できることを確認した.
%
考案した3つの符号化の有効性を評価するために,
Flinders Hamiltonian Cycle Project (FHCP) で公開されている
ハミルトン閉路問題の問題集(全1001問)を用いた実験を行った.
その結果,
\textsf{directed}符号化が,875 問と最も多くの問題を解き,
他の符号化と比較してその優位性が確認できた.

本研究で使用したハミルトン閉路問題の問題集は,
2015年から2016年にかけて行われた 
FHCP Challenge と呼ばれる国際競技会~\cite{haythorpe19:fhcp}で使用され
たものである.
この競技会では,
整数計画ソルバー CPLEX を使用した手法~\cite{cohen17:1001graph}が 985
問を解き優勝した.
2位となった SAT ソルバーを使用した手法は 614 問を解い
た~\cite{andrew18:triple}.
3位から5位の解けた問題数はそれぞれ,488 問,464 問,385 問であった.
よって,875 問を解いた\textsf{directed}符号化は,
ASPのルール10個程度の簡潔なプログラムであるにもかかわらず,
実質2位の成績に相当する.

%%%%%%%%%%%%%%%%%%%%%%%%%%%%%%%%%%%%%%%%%%%%%%%%%%%%%%%%%
今後の課題としては,
ハミルトン閉路問題の緩和問題と
CEGAR (Counterexample Guided Abstraction Refinement)
を用いた解法~\cite{soh14:jelia2014,soh20:cegar}を応用した提案手法の高速化,
巡回セールスマン問題への拡張などが挙げられる.

ハミルトン閉路問題は代表的な NP 完全問題であり,
また重要な工学的応用が数多く存在するため,
その解法については古くから盛んに研究されている.
本研究と関連の深いものとしては,
SATソルバーを用いた解法~\cite{Prestwich03:DAM,VelevG09:relative,soh14:jelia2014},
整数計画ソルバーを用いた解法~\cite{numata11:tsp}がある.
これらや FHCP Challenge の上位ソルバー~\cite{cohen17:1001graph,andrew18:triple}
との比較も重要な今後の課題である.

%%% Local Variables:
%%% mode: latex
%%% TeX-master: "paper"
%%% End:
