%%%%%%%%%%%%%%%%%%%%%%%%%%%%%%%%%%%%%%%%%%%%%%%%%%%%%%%%%% 
\section{おわりに}\label{chap:conclusion}
%%%%%%%%%%%%%%%%%%%%%%%%%%%%%%%%%%%%%%%%%%%%%%%%%%%%%%%%%%

本稿では,解集合プログラミング(ASP)を用いた
ハミルトン閉路問題,
最短ハミルトン閉路問題,
コスト制約付きハミルトン閉路問題
の解法について述べた.
%
ハミルトン閉路問題を解くASP符号化として,
\textsf{undirected},
\textsf{directed},
\textsf{acyclicity}
の3つを考案し,ハミルトン閉路問題および関連問題の制約を,
ASPのルール10個程度で簡潔に表現できることを確認した.
%
考案した3つの符号化の有効性を評価するために,
Flinders Hamiltonian Cycle Project (FHCP) で公開されている
ハミルトン閉路問題の問題集(全1001問)を用いた実験を行った.
その結果,
\textsf{directed}符号化が,875 問と最も多くの問題を解き,
他の符号化と比較してその優位性が確認できた.

%%%%%%%%%%%%%%%%%%%%%%%%%%%%%%%%%%%%%%%%%%%%%%%%%%%%%%%%%
\textcolor{red}{
  この実験で使用した問題集は,
  2016年に FHCP Challenge と呼ばれる国際競技会で使用されたものである.
  この競技会では,
  1位となった D.~Coudert と N.~Cohen が
  整数計画ソルバー CPLEX を使用して 985 問\cite{cohen17:1001graph},
  2位となった A.~Johnson が 
  SATソルバーを使用して 614 問\cite{andrew18:triple},
  3位となった A.~Gharbi と U.~Syarif が 488 問,
  4位となった M.~Noisternig が 464 問,
  5位となった M.~Nurhafiz が 385 問をそれぞれ解いた\cite{haythorpe19:fhcp}.
  \textsf{directed}符号化は,
  ASPのルール10個程度の簡潔な記述であるにもかかわらず,
  実質的に二位の成績に相当した.
}

%%%%%%%%%%%%%%%%%%%%%%%%%%%%%%%%%%%%%%%%%%%%%%%%%%%%%%%%%
今後の課題としては,
ハミルトン閉路問題の緩和問題と
CEGAR (Counterexample Guided Abstraction Refinement)
を用いた解法~\cite{soh14:jelia2014,soh20:cegar}を応用した提案手法の高速化,
巡回セールスマン問題への拡張などが挙げられる.

ハミルトン閉路問題は代表的な NP 完全問題であり,
また重要な工学的応用が数多く存在するため,
その解法については古くから盛んに研究されている.
本研究と関連の深い研究としては,
SATソルバーを用いた解法~\cite{Prestwich03:DAM,VelevG09:relative,soh14:jelia2014},
整数計画ソルバーを用いた解法~\cite{numata11:tsp}がある.
これら
\textcolor{red}{
  や FHCP Challenge で 用いられた解法\cite{cohen17:1001graph,andrew18:triple}
}を含む他のアプローチとの比較も重要な今後の課題である.


%%% Local Variables:
%%% mode: latex
%%% TeX-master: "paper"
%%% End:
