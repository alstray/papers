%%%%%%%%%%%%%%%%%%%%%%%%%%%%%%%%%%%%%%%%%%%%%%%%%%%%%%%%%% 
\section{実行実験}\label{chap:experiment}
%%%%%%%%%%%%%%%%%%%%%%%%%%%%%%%%%%%%%%%%%%%%%%%%%%%%%%%%%% 

考案した3つの符号化
%\textsf{undirected},\textsf{directed},\textsf{acyclicity}
の性能を評価するために実行実験を行った.
実験に使用したベンチマーク問題は以下の通りである.

\begin{itemize}
\item \textbf{ハミルトン閉路問題 (1,001問)}\\
  Flinders Hamiltonian Cycle Project (FHCP) で公開されている
  HCPインスタンス(全1,001問)を使用した~\cite{haythorpe19:fhcp}.
%  \footnote{{\tt https://sites.flinders.edu.au/flinders-hamiltonian-cycle-project/}}
  この問題集は FHCP Challenge と呼ばれる国際競技会で使用された
  頂点数が 66個から9,528個の HCP インスタンスで構成されている.
  すべてのインスタンスはハミルトン閉路が存在する\textsf{SAT}インスタン
  スである.
\item \textbf{最短ハミルトン路問題 (7問)}\\
  図~\ref{fig:USmap}に示した米国本土48州の隣接グラフに基づく問題インス
  タンス(1問)と,
  $2N+1$次の正方グリッドグラフに基づく問題インスタンス
  (6問,$3\leq N\leq 8$)を使用した.
\item \textbf{コスト制約付きハミルトン路問題(全列挙)}\\
  図~\ref{fig:USmap}に示した米国本土48州の隣接グラフに基づく問題インス
  タンスを使用した.
  コード~\ref{code:cost_constraint.lp}に示したコスト制約の
  閾値\code{c}として,
  最短距離(11,698マイル)の$N$倍となる10個の値,
  $N=1.00,1.01,1.02,1.05,1.1,1.2,1.3,1.4,1.5,1.6$
  を使用した.
\end{itemize}

ASP システムには,{\clingo}のバージョン5.5.0 を使用し,
実験環境は Mac OS (Intel Core i7 3.2GHz 64GBメモリ)である.

%%%%%%%%%%%%%%%%%%%%%%%%%%%%%%%%%%%%%%%%%%%%%%%%%%%%%%%%%%
\subsection{ハミルトン閉路問題}
%%%%%%%%%%%%%%%%%%%%%%%%%%%%%%%%%%%%%%%%%%%%%%%%%%%%%%%%%%

%%%%%%%%%%%%%%%%%%%%%%%%%%%%%%%%%%%%%%%%%%%%%%%
\begin{table*}[t]\footnotesize
  \centering
% \tabcolsep = 0.8mm
% \renewcommand{\arraystretch}{1.2}
  \begin{tabular}{lr||r|r|r}
    頂点数 & 問題数 & \textsf{undirected} & \textsf{directed} & \textsf{acyclicity}\\
   \hline
    $\:\:\:\:\:\,\, 0 \leq |V| < 1000$  & 171   & 156   & \textbf{171}   & 155  \\
    $1000 \leq |V| < 2000$  & 165   & 120   & \textbf{158}   & 124  \\
    $2000 \leq |V| < 3000$  & 177   & 125   & \textbf{162}   & 73   \\
    $3000 \leq |V| < 4000$  & 185   & 104   & \textbf{148}   & 42   \\
    $4000 \leq |V| < 5000$  & 128   & 92    & \textbf{104}   & 28   \\
    $5000 \leq |V| < 6000$  & 80    & 63    & \textbf{68}    & 23   \\
    $6000 \leq |V| < 7000$  & 55    & 39    & \textbf{43}    & 21   \\
    $7000 \leq |V| < 8000$  & 28    & 12    & \textbf{14}    & 5    \\
    $8000 \leq |V| < 9000$  & 10    & 2     & \textbf{5}     & 1    \\
    $9000 \leq |V| < 10000$ & 2     & \textbf{2}     & \textbf{2}     & 1    \\
   \hline
    合計 & 1001 & 715   & \textbf{875}   & 473  
  \end{tabular}
  \vskip .5em
  \caption{ハミルトン閉路問題: 解けた問題数}
  \label{sat_table}
\end{table*}
%label{sat_table}
%%%%%%%%%%%%%%%%%%%%%%%%%%%%%%%%%%%%%%%%%%%%%%%

%%%%%%%%%%%%%%%%%%%%%%%%%%%%%%%%%%%%%%%%%%%%%%%
\begin{figure}[tb]
\begin{center}
  \includegraphics[width=0.8\linewidth]{fig/cactus_fhcp.png}
\caption{ハミルトン閉路問題: カクタスプロット}
\label{cactus}
\end{center}
\end{figure}
%%%%%%%%%%%%%%%%%%%%%%%%%%%%%%%%%%%%%%%%%%%%%%%

表~\ref{sat_table}に,ハミルトン閉路問題の解けた問題数を示す.
左から,HCPインスタンスの頂点数の範囲,問題数,
各符号化で解けた問題数を示している.
各範囲において,最もよい結果をボールド体で示している.
% 
解けた問題数は,
\textsf{undirected}符号化が 715 問,
\textsf{directed}符号化が 875 問,
\textsf{acyclicity}符号化が 473 問であり,
\textsf{directed}符号化がもっとも多くの問題を解いた.
\textsf{directed}符号化は,どの頂点数の範囲においても他の符号化と同じ
かより多くの問題を解いており,その優位性が確認できる.

図~\ref{cactus}に,カクタスプロットを示す.
縦軸は問題を解くのに要した CPU 時間(秒),横軸は解けた問題数を表す.
グラフが右によるほど多くの問題を解けたことを示し,
下によるほどより速く解けたことを示す.
図~\ref{cactus}より,\textsf{directed}符号化が,
他の2つの符号化と比較して,より多くの問題を高速に解いていることが確認できる.

今回の実験では,
ASPシステム{\clingo}のオプションは\textit{trendy}を使用し,
1問あたりの時間制限は30分とした.
また,\textsf{directed}と\textsf{acyclicity}符号化については,
個数制約の上限(コード\ref{code:hamilton2.lp}と\ref{code:hamilton3.lp}の1行目)
を省略したものを使用した.

%%%%%%%%%%%%%%%%%%%%%%%%%%%%%%%%%%%%%%%%%%%%%%%%%%%%%%%%%%
\subsection{最短ハミルトン路問題}
%%%%%%%%%%%%%%%%%%%%%%%%%%%%%%%%%%%%%%%%%%%%%%%%%%%%%%%%%%

%%%%%%%%%%%%%%%%%%%%%%%%%%%%%%%%%%%%%%%%%%%%%%%
\begin{table}[t]\footnotesize
  \caption{最短ハミルトン閉路問題: 得られた目的関数の値}
  \label{min_table_tr}
  \tabcolsep = 2mm
  %\renewcommand{\arraystretch}{1.0}
  \vskip .5em
  \centering
  \begin{tabular}{l|rrr}\hline
     問題 & \textsf{undirected} & \textsf{directed} & \textsf{acyclicity} \\
    \hline
    grid5&50,656*&50,656*&50,656* \\
    grid6&68,656*&68,656*&68,656* \\
    grid7&91,822*&91,822*&91,822* \\
    grid8&113,250&\textcolor{red}{112,916}&113,277 \\
    grid9&\textcolor{red}{142,502}&143,326&143,660 \\
    grid10&\textcolor{red}{172,703}&174,866&175,999 \\
    grid11&\textcolor{red}{200,399}&204,456&200,638 \\
    grid12&\textcolor{red}{231,278}&239,275&232,012 \\
    grid13&\textcolor{red}{276,692}&276,926&276,899 \\
    grid14&317,617&\textcolor{red}{317,144}&317,676 \\
    grid15&\textcolor{red}{375,906}&376,809&376,210 \\
    grid16&421,249&\textcolor{red}{419,737}&423,753 \\
    US48&11,698*&11,698*&11,698* \\
    \hline
    最適値と最良値の数 & 10 & 7 & 4\\    \hline
  \end{tabular}
\end{table}
%\label{min_table_tr}
%%%%%%%%%%%%%%%%%%%%%%%%%%%%%%%%%%%%%%%%%%%%%%%

表\ref{min_table_tr}に,最短ハミルトン路問題について,
各符号化で得られた目的関数の値を示す.
ASPシステム{\clingo}のオプションは\textit{trendy}を使用し,
1問あたりの時間制限は3時間とした.
% 
各問題ごとに,最も良かった値をボールド体で示している.
*マークは,最適値を表している.
最適値と最良値の数は,
\textsf{undirected}符号化が5問,
\textsf{directed}符号化が3問,
\textsf{acyclicity}符号化が3問であり,
\textsf{undirected}符号化が2問ではあるが,
他の符号化より精度の高い解を生成した.
grid6 と US48 については,3つの符号化すべてで最適値を得ることができた.
最適値を得るのに要した CPU 時間は,最も速かった\textsf{undirected}符号
化で,grid6 は 1.451秒,US48 は 6.68秒であった.

%%%%%%%%%%%%%%%%%%%%%%%%%%%%%%%%%%%%%%%%%%%%%%%%%%%%%%%%%%
\subsection{コスト制約付きハミルトン路問題}
%%%%%%%%%%%%%%%%%%%%%%%%%%%%%%%%%%%%%%%%%%%%%%%%%%%%%%%%%%

%%%%%%%%%%%%%%%%%%%%%%%%%%%%%%%%%%%%%%%%%%%%%%%
\begin{table}[htbp]
  \caption{実験結果3}
  \label{cost_table}
  \centering
  \begin{tabular}{|l|r|rrr|}
    \hline
    制約Cost値    &	Models & undirected & directed & acyclicity \\
    \hline
    11698   &	1      &\textcolor{red}{2.919} &10.020 & 4.355	\\
    11814   &	8      &5.458  &7.416	& \textcolor{red}{4.136}	\\
    11931   &	28     &\textcolor{red}{3.226}&10.317	& 4.799	\\
    12282   &	388    &\textcolor{red}{9.993}&15.787	& 10.715	\\
    12867   &	16180  &16.386       &23.406	& \textcolor{red}{10.819}\\
    14037   &	939209 &47.894       &41.515	& \textcolor{red}{24.655}\\
    15207   &	4525541&85.256       &56.953	& \textcolor{red}{41.217}\\
    16377   &	6702964&93.595       &51.991	& \textcolor{red}{41.301}	\\
    17547   &	6876526&91.750       &46.065	& \textcolor{red}{37.290}	\\
    18716   &	6876928&95.659       &45.416	& \textcolor{red}{37.905}	\\
    \hline
    Average &   & 45.2136 & 30.889  & \textcolor{red}{21.7192}\\
    Best    &   & 3 & 0 & \textcolor{red}{7} \\
    \hline
  \end{tabular}
\end{table}
%\label{cost_table}
%%%%%%%%%%%%%%%%%%%%%%%%%%%%%%%%%%%%%%%%%%%%%%%

表~\ref{cost_table}に,
図~\ref{fig:USmap}に示した米国本土48州の隣接グラフに基づく
コスト制約付きハミルトン路問題について,
各符号化が解の全列挙に要した CPU 時間(秒)を示す.
ASPシステム{\clingo}のオプションは\textit{crafty}を使用し,
1問あたりの時間制限を3時間とした.
表の1列目はコスト制約の閾値と最短距離からの倍率,2列目は解の総数を表している.
各閾値ごとに,最も良かった値をボールド体で示している.
表より,\textsf{acyclicity}符号化が,他の符号化と比較して,より多くの
問題を高速に解いていることがわかる.また,平均CPU時間も最も短い.

%%%%%%%%%%%%%%%%%%%%%%%%%%%%%%%%%%%%%%%%%%%%%%%%%%%%%%%%%%

%%% Local Variables:
%%% mode: latex
%%% TeX-master: "paper"
%%% End:
