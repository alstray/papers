%%%%%%%%%%%%%%%%%%%%%%%%%%%%%%%%%%%%%%%%%%%%%%%%%%%%%%%%%% 
\section{実行実験}\label{chap:experiment}
%%%%%%%%%%%%%%%%%%%%%%%%%%%%%%%%%%%%%%%%%%%%%%%%%%%%%%%%%% 

本章では,前章で提案した3つの符号化
\textsf{undirected},\textsf{directed},\textsf{acyclicity}
の性能を評価するために実行実験を行った.
%
実験に使用したベンチマーク問題集(計1008問)は,以下の通りである.
\begin{itemize}
\item \textsf{fhcp} (1001問)\\
  Jerzy Filar と Vladimir Ejov が主導する
  チームプロジェクト Flinders Hamiltonian Cycle Project
  \footnote{\url{https://sites.flinders.edu.au/flinders-hamiltonian-cycle-project/}}
  が提供するハミルトン閉路問題のグラフインスタンス.\cite{haythorpe19:fhcp}
\item \textsf{grid} (6問)\\
  $2N+1$次の正方グリッドグラフのインスタンス($3\leq N\leq 8$).
\item \textsf{usmap} (1問)\\
  図~\ref{fig:USmap}に示されたグラフ.
  D.~E~.Knuth の教科書
  The Art of Computer Programming~\cite{Knuth:TAOCP:SAT}
  に記載されている最短ハミルトン路問題の例.
\end{itemize}

使用した ASP システムは{\clingo}のバージョン5.5.0である.
実験環境は,Mac mini Intel Corei7 3.2GHz 64GBメモリである.

%%%%%%%%%%%%%%%%%%%%%%%%%%%%%%%%%%%%%%%%%%%%%%%%%%%%%%%%%%
\subsection{ハミルトン閉路問題の実験結果}
%%%%%%%%%%%%%%%%%%%%%%%%%%%%%%%%%%%%%%%%%%%%%%%%%%%%%%%%%%

%%%%%%%%%%%%%%%%%%%%%%%%%%%%%%%%%%%%%%%%%%%%%%%
\begin{table*}[t]\footnotesize
  \centering
% \tabcolsep = 0.8mm
% \renewcommand{\arraystretch}{1.2}
  \begin{tabular}{lr||r|r|r}
    頂点数 & 問題数 & \textsf{undirected} & \textsf{directed} & \textsf{acyclicity}\\
   \hline
    $\:\:\:\:\:\,\, 0 \leq |V| < 1000$  & 171   & 156   & \textbf{171}   & 155  \\
    $1000 \leq |V| < 2000$  & 165   & 120   & \textbf{158}   & 124  \\
    $2000 \leq |V| < 3000$  & 177   & 125   & \textbf{162}   & 73   \\
    $3000 \leq |V| < 4000$  & 185   & 104   & \textbf{148}   & 42   \\
    $4000 \leq |V| < 5000$  & 128   & 92    & \textbf{104}   & 28   \\
    $5000 \leq |V| < 6000$  & 80    & 63    & \textbf{68}    & 23   \\
    $6000 \leq |V| < 7000$  & 55    & 39    & \textbf{43}    & 21   \\
    $7000 \leq |V| < 8000$  & 28    & 12    & \textbf{14}    & 5    \\
    $8000 \leq |V| < 9000$  & 10    & 2     & \textbf{5}     & 1    \\
    $9000 \leq |V| < 10000$ & 2     & \textbf{2}     & \textbf{2}     & 1    \\
   \hline
    合計 & 1001 & 715   & \textbf{875}   & 473  
  \end{tabular}
  \vskip .5em
  \caption{ハミルトン閉路問題: 解けた問題数}
  \label{sat_table}
\end{table*}
%label{sat_table}
%%%%%%%%%%%%%%%%%%%%%%%%%%%%%%%%%%%%%%%%%%%%%%%

%%%%%%%%%%%%%%%%%%%%%%%%%%%%%%%%%%%%%%%%%%%%%%%
\begin{figure}[tb]
\begin{center}
  \includegraphics[width=0.8\linewidth]{fig/cactus_fhcp.png}
\caption{ハミルトン閉路問題: カクタスプロット}
\label{cactus}
\end{center}
\end{figure}
%%%%%%%%%%%%%%%%%%%%%%%%%%%%%%%%%%%%%%%%%%%%%%%


%--
本節では,ハミルトン閉路問題の実験結果について述べる.
{\clingo}のオプションは\textit{trendy}を使用し,一問あたりの時間制限を30分とした.
ベンチマーク問題は,\textsf{fhcp}の1001問である.

%--
表~\ref{sat_table}に,各符号化で解けた問題数を,問題の頂点数毎に示す.
左から,問題の頂点数,問題数,各符号化で解けた問題数を示している.
%
解けた問題数は,
\textsf{undirected}符号化が715問,
\textsf{directed}符号化が875問,
\textsf{acyclicity}符号化が473問であり,
\textsf{directed}がもっとも多くの問題を解いた.
\textsf{directed}は,どの頂点数においても同様に,
安定した性能の良さを示した.
図~\ref{cactus}に,カクタスプロットを示す.
縦軸は問題を解くのに要した CPU 時間,横軸は解けた問題数を表す.
グラフが右によるほど多くの問題を解けたことを示し,
下によるほどより速く解けたことを示す.
図~\ref{cactus}より,\textsf{directed}符号化が,
他の2つの符号化と比較して,より多くの問題を高速に解いていることが確認できた.
%% しかし,一部\textsf{undirected}符号化が\textsf{directed}符号化を下回る
%% 部分が確認できる.
%% 実際に,一部の問題では\textsf{undirected}符号化が
%% \textsf{directed}符号化よりも高速に解いていた.

%%%%%%%%%%%%%%%%%%%%%%%%%%%%%%%%%%%%%%%%%%%%%%%%%%%%%%%%%%
\subsection{最短ハミルトン閉路問題}
%%%%%%%%%%%%%%%%%%%%%%%%%%%%%%%%%%%%%%%%%%%%%%%%%%%%%%%%%%

%%%%%%%%%%%%%%%%%%%%%%%%%%%%%%%%%%%%%%%%%%%%%%%
\begin{table}[t]\footnotesize
  \caption{最短ハミルトン閉路問題: 得られた目的関数の値}
  \label{min_table_tr}
  \tabcolsep = 2mm
  %\renewcommand{\arraystretch}{1.0}
  \vskip .5em
  \centering
  \begin{tabular}{l|rrr}\hline
     問題 & \textsf{undirected} & \textsf{directed} & \textsf{acyclicity} \\
    \hline
    grid5&50,656*&50,656*&50,656* \\
    grid6&68,656*&68,656*&68,656* \\
    grid7&91,822*&91,822*&91,822* \\
    grid8&113,250&\textcolor{red}{112,916}&113,277 \\
    grid9&\textcolor{red}{142,502}&143,326&143,660 \\
    grid10&\textcolor{red}{172,703}&174,866&175,999 \\
    grid11&\textcolor{red}{200,399}&204,456&200,638 \\
    grid12&\textcolor{red}{231,278}&239,275&232,012 \\
    grid13&\textcolor{red}{276,692}&276,926&276,899 \\
    grid14&317,617&\textcolor{red}{317,144}&317,676 \\
    grid15&\textcolor{red}{375,906}&376,809&376,210 \\
    grid16&421,249&\textcolor{red}{419,737}&423,753 \\
    US48&11,698*&11,698*&11,698* \\
    \hline
    最適値と最良値の数 & 10 & 7 & 4\\    \hline
  \end{tabular}
\end{table}
%\label{min_table_tr}
%%%%%%%%%%%%%%%%%%%%%%%%%%%%%%%%%%%%%%%%%%%%%%%

%--
本節では,最短ハミルトン閉路問題の実験結果について述べる.
{\clingo}のオプションは\textit{trendy}を使用し,
一問あたりの時間制限を3時間とした.
ベンチマーク問題は,\textsf{grid}と\textsf{usmap}の合計13問である.

%--
表\ref{min_table_tr}に,各符号化で得られた最適値と最良値を示す.
各問題毎に,最も良かった値を赤字で示している.
*マークは,最適値を表している.
最適値と最良値の数は,
\textsf{undirected}符号化が10問,
\textsf{directed}符号化が7問,
\textsf{acyclicity}符号化が4問であり,
\textsf{undirected}符号化の優位性が確認できた.

%%%%%%%%%%%%%%%%%%%%%%%%%%%%%%%%%%%%%%%%%%%%%%%%%%%%%%%%%%
\subsection{コスト制約付きハミルトン路問題}
%%%%%%%%%%%%%%%%%%%%%%%%%%%%%%%%%%%%%%%%%%%%%%%%%%%%%%%%%%

%%%%%%%%%%%%%%%%%%%%%%%%%%%%%%%%%%%%%%%%%%%%%%%
\begin{table}[htbp]
  \caption{実験結果3}
  \label{cost_table}
  \centering
  \begin{tabular}{|l|r|rrr|}
    \hline
    制約Cost値    &	Models & undirected & directed & acyclicity \\
    \hline
    11698   &	1      &\textcolor{red}{2.919} &10.020 & 4.355	\\
    11814   &	8      &5.458  &7.416	& \textcolor{red}{4.136}	\\
    11931   &	28     &\textcolor{red}{3.226}&10.317	& 4.799	\\
    12282   &	388    &\textcolor{red}{9.993}&15.787	& 10.715	\\
    12867   &	16180  &16.386       &23.406	& \textcolor{red}{10.819}\\
    14037   &	939209 &47.894       &41.515	& \textcolor{red}{24.655}\\
    15207   &	4525541&85.256       &56.953	& \textcolor{red}{41.217}\\
    16377   &	6702964&93.595       &51.991	& \textcolor{red}{41.301}	\\
    17547   &	6876526&91.750       &46.065	& \textcolor{red}{37.290}	\\
    18716   &	6876928&95.659       &45.416	& \textcolor{red}{37.905}	\\
    \hline
    Average &   & 45.2136 & 30.889  & \textcolor{red}{21.7192}\\
    Best    &   & 3 & 0 & \textcolor{red}{7} \\
    \hline
  \end{tabular}
\end{table}
%\label{cost_table}
%%%%%%%%%%%%%%%%%%%%%%%%%%%%%%%%%%%%%%%%%%%%%%%

%--
本節では,第~\ref{chap:background}章でも説明した
コスト制約付きハミルトン路問題(全列挙)の実験結果について述べる.
{\clingo}のオプションは\textit{crafty}を使用し,
一問あたりの時間制限を3時間とした.
ベンチマーク問題は,D.~E~.Knuth の教科書
The Art of Computer Programming~\cite{Knuth:TAOCP:SAT}
に記載されているグラフを使用した(図~\ref{fig:USmap}参照).
このグラフは,米国本土48州の隣接関係を表しており,
頂点数は48,辺の数は105である.
この問題の最短距離は 11698 である(表~\ref{min_table_tr}参照).
今回の実験では,
コスト制約を最短距離のN倍以下
($N=1.00,1.01,1.02,1.05,1.1,1.2,1.3,1.4,1.5,1.6$)として,解の全列挙を
行った.

%--
表~\ref{cost_table}に,各符号化が解の全列挙に要した CPU 時間を示す.
表の1列目はコスト制約の閾値と最短距離からの倍率,2列目は解の総数を表している.
各閾値毎に,最も良かった値を赤字で示している.
表より,
\textsf{acyclicity}符号化が,他の符号化と比較して,より多くの問題を
高速に解いていることがわかる.また,平均CPU時間も最も短い.
%%%%%%%%%%%%%%%%%%%%%%%%%%%%%%%%%%%%%%%%%%%%%%%%%%%%%%%%%%

%%% Local Variables:
%%% mode: latex
%%% TeX-master: "paper"
%%% End:
