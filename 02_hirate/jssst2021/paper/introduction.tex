%%%%%%%%%%%%%%%%%%%%%%%%%%%%%%%%%%%%%%%%%%%%%%%%%%%%%%%%%% 
\section{はじめに}
%%%%%%%%%%%%%%%%%%%%%%%%%%%%%%%%%%%%%%%%%%%%%%%%%%%%%%%%%% 

\textbf{ハミルトン閉路問題} (Hamiltonian Cycle Problem; HCP) は,
与えられたグラフの全頂点をちょうど一度ずつ通る閉路が存在するかどうかを
判定する問題である\cite{hirata15:book}.
ハミルトン閉路問題は代表的な NP 完全問題である.
\textbf{ハミルトン路問題} (Hamiltonian Path Problem; HPP) は,
ハミルトン閉路問題から始点と終点が一致するという閉路の条件を取り除いた
ものである.
これらの問題は,重要な工学的応用が数多く存在するため,古くから盛んに研
究されている.
例えば,数理最適化の分野で有名な巡回セールスマン問題は,グラフの辺に距
離が付随しているとき,最短距離のハミルトン閉路を求める
\textbf{最短ハミルトン閉路問題}と考えることができる.
また,ごく最近では,距離の総和が所与の閾値以下(または以上)であることを
制約条件として付加した
\textbf{コスト制約付きハミルトン閉路問題}
の解を高速に全列挙するアルゴリズムも提案されている\cite{comp20:Minato}.
本研究では,これらの(無向グラフ上の)ハミルトン閉路問題およびその関連問
題を対象とする.

解集合プログラミング(Answer Set Programming; ASP\cite{%
  Baral03:cambridge,%
  Gelfond88:iclp,%
  Inoue08:jssst,%
  Niemela99:amai})
は,論理プログラミングから派生した宣言的問題解法のための
プログラミングパラダイムである.
ASP言語は,一階論理に基づく知識表現言語の一種である.
論理プログラムは ASP のルールの有限集合である.
ASP システムは論理プログラムから安定モデル意味論に基づく解集合を計算す
るシステムである.
近年,SAT技術を応用した高速 ASP システムが開発され,
スケジューリング,
プランニング,
システム生物学,
システム検証
%制約充足問題,
%制約最適化問題
など様々な分野への実用的応用が急速に拡大している.
%
ハミルトン閉路問題およびその関連問題に対して ASP を用いる利点としては,
ASP 言語の高い表現力,
高速な解列挙,
組込み非閉路制約,
充足不能コアに基づく最適化,
インクリメンタルASP解法
などが挙げられる.

本論文では,解集合プログラミング(ASP)を用いた
ハミルトン閉路問題,
最短ハミルトン閉路問題,
コスト制約付きハミルトン閉路問題
の解法について述べる.
%
ハミルトン閉路問題を解くASP符号化として,
\textsf{undirected},
\textsf{directed},
\textsf{acyclicity}
の3つを考案した.
\textsf{undirected}は,
ハミルトン閉路問題を次数制約と連結制約で簡潔に表現した符号化である.
\textsf{directed}は,
与えられた無向グラフの各辺$u-v$に対して,
2つの弧$u\rightarrow v$と$v\rightarrow u$を対応させることで有向グラフ
化して解く符号化である.
% 変換した有向グラフ上のハミルトン閉路は元の無向グラフ上のハミルトン閉路
% となり,また逆も成り立つ.
\textsf{acyclicity}は,\textsf{directed}符号化をベースに,
連結制約に代わる部分閉路禁止制約を,
ASP の組込み非閉路制約~\cite{bomanson16:acyclicity}で表現した符号化である.
最短ハミルトン閉路問題とコスト制約付きハミルトン閉路問題については,
考案した3つの符号化に目的関数とコスト制約をそれぞれ追加することで自然
に拡張できる.

\vskip 3em
{\huge 以下 TODO}
\vskip 3em



考案した符号化の有効性を評価するために,
既存のベンチマーク問題集(7種類,計517問)を用いて実行実験を行なった.
その結果,
ハミルトン閉路問題とコスト制約付きハミルトン閉路問題(解の全列挙)について,
\textsf{acyclicity}符号化が,
\textsf{undirected}と\textsf{directed}と比較して,
より多くの問題を高速に解くことに成功し,その優位性を確認できた.
また,最短ハミルトン閉路問題については,
\textsf{undirected}符号化が,他の符号化と比較して,より多くの問題で最
適値・最良値を求めることができた.

% 本論文の構成は以下の通りである.
% まず,第~\ref{chap:background}章では,本論文が対象とする無向グラフ上の
% ハミルトン閉路問題およびその関連問題について述べる.
% 第~\ref{chap:asp}章では,解集合プログラミングの言語構文,システム,プ
% ログラム例を説明する.
% 第~\ref{chap:proposal}章では,ハミルトン閉路問題および関連問題を解く
% ASP符号化について詳しく述べる.
% 第~\ref{chap:experiment}章では,既存のベンチマーク問題集を用いた実行実
% 験について述べる.
% 最後に,第~\ref{chap:conclusion}章で本論文の結論を述べる.

%%% Local Variables:
%%% mode: latex
%%% TeX-master: "paper"
%%% End:
