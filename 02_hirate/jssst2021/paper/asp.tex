%%%%%%%%%%%%%%%%%%%%%%%%%%%%%%%%%%%%%%%%%%%%%%%%%%%%%%%%%%%%%%%%%%%%%%%%%%%%%%%%%
\section{解集合プログラミング}\label{chap:asp}
%%%%%%%%%%%%%%%%%%%%%%%%%%%%%%%%%%%%%%%%%%%%%%%%%%%%%%%%%%%%%%%%%%%%%%%%%%%%%%%%%

解集合プログラミング(ASP; \cite{%
  Baral03:cambridge,%
  Gelfond88:iclp,%
  Inoue08:jssst,%
  Niemela99:amai})
の言語は,一般拡張選言プログラムに基づいている.
本稿では説明の簡略化のため,そのサブクラスである
標準論理プログラム(以降,論理プログラムと呼ぶ)について説明する.
以降,論理プログラムのソースコードはすべてASPシステム
{\clingo}~\footnote{\url{https://potassco.org/clingo/}}
の言語構文に従う.

\textbf{論理プログラム}は,以下の形式の\textbf{ルール}の有限集合である.
\[
  a_0\ \texttt{:-}\
  a_1\texttt{,}\dots\texttt{,}a_m\texttt{,}
  \texttt{not}\ {a_{m+1}}\texttt{,}\dots\texttt{,}\texttt{not}\ {a_n}\texttt{.}
\]
ここで,$0\leq m\leq n$ であり,各$a_i$はアトム,
\texttt{not}は\textbf{デフォルトの否定}
\footnote{\textbf{失敗による否定}とも呼ばれる.述語論理の否定($\neg$)とは意味が異なる.},
カンマ(``\texttt{,}'')は連言を表す.
ピリオド(``\code{.}'')はルールの終わりを表す終端記号である.
``\texttt{:-}''の左側を\textbf{ヘッド},右側を\textbf{ボディ}と呼ぶ.
ルールの直観的な意味は,
「$a_1,\ldots,a_m$がすべて成り立ち,$a_{m+1},\ldots,a_n$のそれぞれが成
り立たないならば,$a_0$が成り立つ」である.

ボディが空のルール(すなわち\(a_0\ \texttt{:-.}\))を\textbf{ファクト}と呼び,
``\texttt{:-}''を省略してもよい.
また,以下のようにヘッドが空のルールを\textbf{一貫性制約}と呼ぶ.
\[
  \texttt{:-} \ a_1\texttt{,}\dots\texttt{,}a_m\texttt{,}
  \texttt{not}\ {a_{m+1}}\texttt{,}\dots\texttt{,}\texttt{not}\ {a_n}\texttt{.}
\]
例えば,
%\(\leftarrow a_1,a_2\)は,「$a_1$と$a_2$が両方同時に成り立つことはない」を意味し,
\(\texttt{:-}\ \texttt{not}\ a_1\texttt{,} {a_{2}}\texttt{.}\)は,
「$a_2$が成り立つならば,$a_1$が成り立つ」を意味する.

ASP言語には,組合せ問題を解くための便利な拡張構文として,
\textbf{選択子}と\textbf{個数制約}が用意されている.
例えば,選択子\code{\{}\(a_1\texttt{;}\dots\texttt{;}a_n\)\code{\}}
をファクトとして書くと,
「アトム集合\(\{a_1,\dots,a_n\}\)の任意の部分集合が成り立つ」
を意味する.
個数制約は選択子の両端に選択可能な個数の上下限を付けたものである.
例えば,
\code{:- not}\ $lb$\ \code{\{}\(a_1\texttt{;}\dots\texttt{;}a_n\)\code{\}}\ $ub$\code{,}$b$\code{.}
と書くと,
「$b$が成り立つならば,$a_1,\dots,a_n$のうち,$lb$個以上$ub$個以下が成り立つ」を意味する.

また,組合せ最適化問題を解くために,最小化関数
(\code{\#minimize})と最大化関数(\code{\#maximize})等も用意されている.
さらに,最近では,グラフ問題を解くための便利な拡張構文として,
有向グラフの非閉路性を簡潔に記述できる
\code{\#edge}宣言が追加されている~\cite{bomanson16:acyclicity}.
例えば,\code{\#edge(X,Y): arc(X,Y).}は,\code{arc(X,Y)}を満たす有向辺
\code{X}~$\rightarrow$~\code{Y}を辺集合としてもつグラフが,
閉路をもたないことを保証する.

%%% Local Variables:
%%% mode: japanese-latex
%%% TeX-master: "paper"
%%% End:
