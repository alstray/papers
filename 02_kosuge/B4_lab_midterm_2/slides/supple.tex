


%%%%%%%%%%%%%%%%%%%%%%%%%%%%%%%%%%%%%%%%%%%%%%%%%%%%%%%%%%%%%
% %%%% 補助スライド
%%%%%%%%%%%%%%%%%%%%%%%%%%%%%%%%%%%%%%%%%%%%%%%%%%%%%%%%%%%%%

\appendix

\backupbegin

%%%%%%%%%%%%%%%%%%%%%%%%%%%%%%%%%%%%%%
% ~
%%%%%%%%%%%%%%%%%%%%%%%%%%%%%%%%%%%%%%
\begin{frame}
    \frametitle{~}
    \centering
    - 補足用 -
\end{frame}



%%%%%%%%%%%%%%%%%%%%%%%%%%%%%%%%%%%%%%
% at-least-one制約
%%%%%%%%%%%%%%%%%%%%%%%%%%%%%%%%%%%%%%
\begin{frame}
    \frametitle{at-least-one制約を用いたヒント制約(ALT1)}
    \vspace{-3mm}
    \begin{block}{}
        $alldfifferent(x_1,x_2,\ldots,x_n)$について, $x_i \in \{\ell, \ell+1,\ldots, u\}$かつ$u-\ell=n-1$であるときに以下の制約を追加する.\\
        \vspace{-3mm}
        $$\bigvee_{i=1}^n x_{ia}=1 \qquad (a \in \{\ell,\ldots, u\})$$
    \end{block}
    \begin{exampleblock}{at-least-one制約を用いたヒント制約の例}
        $alldifferent(x_1, x_2, x_3, x_4)$について, $x_i \in \{1, 2, 3, 4\}$であるときには以下の制約が追加される.
        \vspace{-3mm}
        \begin{eqnarray*}
            (x_{11}=1) \lor (x_{21}=1) \lor (x_{31}=1) \lor (x_{41}=1)\\
            (x_{12}=1) \lor (x_{22}=1) \lor (x_{32}=1) \lor (x_{42}=1)\\
            (x_{13}=1) \lor (x_{23}=1) \lor (x_{33}=1) \lor (x_{43}=1)\\
            (x_{14}=1) \lor (x_{24}=1) \lor (x_{34}=1) \lor (x_{44}=1)
        \end{eqnarray*}
    \end{exampleblock}
\end{frame}


%%%%%%%%%%%%%%%%%%%%%%%%%%%%%%%%%%%%%%
% kernel制約
%%%%%%%%%%%%%%%%%%%%%%%%%%%%%%%%%%%%%%
\begin{frame}
    \frametitle{kernel制約を用いたヒント制約(kernel)}
    MVEにおいて,頂点$x$が色$a$で彩色されないとき,$x$に隣接するある頂点が色$a$に彩色されることを表す制約を{\alldifferent}のヒント制約として追加する.
    % \begin{block}{}
    %     ある変数$x_i$について,それを含む{\alldifferent}制約の他の要素$x_j \, (j \neq i)$との関係を表す以下の制約を追加する.
    %     \begin{eqnarray*}
    %         
    %     \end{eqnarray*}
    % \end{block}
    \begin{exampleblock}{}
        $x_1$について
        $alldifferent(x_1,x_2,x_3),alldifferent(x_1,x_4,x_5),alldifferent(x_1,x_6,x_7)$\\ $(x_i \in \{1,2,3\})$があるとすると,以下の制約が追加される.
        {\fontsize{10pt}{0pt}\selectfont
        \begin{eqnarray*}
            (x_{11} = 1) \lor (x_{21} = 1) \lor (x_{31} = 1) \lor (x_{41} = 1) \lor (x_{51} = 1) \lor (x_{61} = 1) \lor (x_{71} = 1)\\
            (x_{12} = 1) \lor (x_{22} = 1) \lor (x_{32} = 1) \lor (x_{42} = 1) \lor (x_{52} = 1) \lor (x_{62} = 1) \lor (x_{72} = 1)\\
            (x_{13} = 1) \lor (x_{23} = 1) \lor (x_{33} = 1) \lor (x_{43} = 1) \lor (x_{53} = 1) \lor (x_{63} = 1) \lor (x_{73} = 1)\\
        \end{eqnarray*}
    }
    \end{exampleblock}
    この制約を追加することで,複数の{\alldifferent}制約について高速化を促すことが期待できる.
\end{frame}


%%%%%%%%%%%%%%%%%%%%%%%%%%%%%%%%%%%%%%
% 大野3
% %%%%%%%%%%%%%%%%%%%%%%%%%%%%%%%%%%%%%%
\begin{frame}
    \frametitle{大野3を用いた手法}
    \begin{block}{}
        $alldifferent(x_1,\dots,x_n)$について,$x_i \in \{\ell,\ell+1,\dots u\}$かつ$u-\ell > n-1$である時,
        新たな整数変数$y_j \, (j \in \{\ell \dots u\})$を導入し,以下の式をPBの式に加える
        \vspace{-3mm}
        \begin{eqnarray*}
            & x_{1j} + \dots + x_{nj}  =  y_j & (j=\ell \dots u)\\
            & y_{\ell} + \dots + y_u  =  n    &
        \end{eqnarray*}
    \end{block}
    \begin{exampleblock}{例}
        $alldifferent(x_1, x_2, x_3)$について, $x_i \in \{1,2,3,4\}$であるとき,以下のように分解する.
        \vspace{-3mm}
        \begin{eqnarray*}
            x_{11} + x_{21} + x_{31} \leq 1 & x_{11} + x_{21} + x_{31} = y_1 \\
            x_{12} + x_{22} + x_{32} \leq 1 & x_{12} + x_{22} + x_{32} = y_2 \\
            x_{13} + x_{23} + x_{33} \leq 1 & x_{13} + x_{23} + x_{33} = y_3 \\
            x_{14} + x_{24} + x_{34} \leq 1 & x_{14} + x_{24} + x_{34} = y_4 \\
                                            & y_1 + y_2 + y_3 + y_4 = 3
        \end{eqnarray*}
    \end{exampleblock}
\end{frame}



%%%%%%%%%%%%%%%%%%%%%%%%%%%%%%%%%%%%%%
% 大野4
% %%%%%%%%%%%%%%%%%%%%%%%%%%%%%%%%%%%%%%
\begin{frame}
    \frametitle{大野4を用いた手法}
    \begin{block}{}
        $alldifferent(x_1,\dots,x_n)$について,$x_i \in \{\ell,\ell+1,\dots u\}$かつ$u-\ell > n-1$である時,
        新たな01変数$x_{(n+1)j}\,(j \in \{\ell \dots u\})$を導入し,以下のように分解する.
        \vspace{-3mm}
        \begin{eqnarray*}
            & x_{1j} + \dots + x_{(n+1)j} = 1 & (j=\ell \dots u)\\
            & x_{(n+1)\ell} + \dots + x_{(n+1)u} = (u-\ell)-(n-1) &
        \end{eqnarray*}
    \end{block}
    \begin{exampleblock}{例}
        $alldifferent(x_1, x_2, x_3)$について, $x_i \in \{1,2,3,4\}$であるとき,以下のように分解する.
        \vspace{-3mm}
        \begin{eqnarray*}
            x_{11} + x_{21} + x_{31} + x_{41} = 1 \\
            x_{12} + x_{22} + x_{32} + x_{42} = 1 \\
            x_{13} + x_{23} + x_{33} + x_{43} = 1 \\
            x_{14} + x_{24} + x_{34} + x_{44} = 1 \\
            x_{41} + x_{42} + x_{43} + x_{44} = 1
        \end{eqnarray*}
    \end{exampleblock}
\end{frame}



%%%%%%%%%%%%%%%%%%%%%%%%%%%%%%%%%%%%%%
% 実験結果全体
%%%%%%%%%%%%%%%%%%%%%%%%%%%%%%%%%%%%%%
\begin{frame}
    \frametitle{実験結果全体}
    \begin{block}{}
        {\fontsize{5pt}{5pt}\selectfont  \begin{tabular}[c] {|c|c|c|c|c|c||r|r|r|r|r|}\hline
  model & 符号    & alldiff & PHP        & ALT1       & kernel     & N=8    & N=9      & N=10    & N=11     & N=12 \\
        &         &         &            &            &            & UNSAT  & UNSAT    & UNSAT   & SAT      & SAT  \\\hline
 0     & OE      & neq     &    &      &        & 20.120    & 1296.917  & 3017.705   & TO       & TO \\
 1     & OE      & neq     & \checkmark   &      &        & 0.118     & 479.376   & TO         & TO       & TO \\
 2     & OE      & neq     &    & \checkmark   &        & 0.021     & 4.256     & 58.701     & 203.959  & TO \\
 3     & OE      & neq     & \checkmark   & \checkmark    &        & 0.017     & 3.833     & 58.621     & 452.605  & TO \\
 4     & OE\textless=\textgreater DE & neq     &    &     &        & 11.532    & 1333.950  & TO         & TO       & TO \\
 5     & OE\textless=\textgreater DE & neq     & \checkmark   &     &        & 0.047     & 364.726   & TO         & TO       & TO \\
 6     & OE\textless=\textgreater DE & neq     &    & \checkmark    &        & 0.005     & 1.600     & 25.872     & 758.905  & TO \\
 7     & OE\textless=\textgreater DE & neq     & \checkmark   & \checkmark    &        & 0.006     & 1.571     & 24.978     & 311.325  & TO \\
 8     & OE\textless=\textgreater DE & PB      &    &      &        & 0.005     & 1.605     & 27.360     & 761.812  & TO \\
 9     & OE\textless=\textgreater DE & PB      & \checkmark   &      &        & 0.005     & 1.525     & 25.105     & 610.408  & TO \\
 10    & OE\textless=\textgreater DE & 大野3   &    &      &        & 0.006     & 0.884     & 21.967     & 446.034  & TO \\
 11    & OE\textless=\textgreater DE & 大野3   & \checkmark   &      &        & 0.006     & 1.195     & 22.950     & 81.861   & TO \\
 12    & OE\textless=\textgreater DE & 大野4   &    &      &        & 0.007     & 1.069     & 21.644     & 954.395  & TO \\
 13    & OE\textless=\textgreater DE & 大野4   & \checkmark   &      &        & 0.006     & 1.147     & 26.128     & 332.564  & TO \\
  14    & OE=\textgreater MVE & neq     &            &            &            & 13.351 & 5044.195 & TO      & TO       & TO   \\
  15    & OE=\textgreater MVE & neq     &            &            & \checkmark & 13.634 & 3504.006 & TO      & TO       & TO   \\
  16    & OE=\textgreater MVE & neq     &            & \checkmark &            & 0.010  & 2.530    & 511.445 & TO       & TO   \\
  17    & OE=\textgreater MVE & neq     &            & \checkmark & \checkmark & 0.013  & 3.810    & 256.044 & TO       & TO   \\
  18    & OE=\textgreater MVE & neq     & \checkmark &            &            & 0.047  & 452.829  & TO      & TO       & TO   \\
  19    & OE=\textgreater MVE & neq     & \checkmark &            & \checkmark & 0.074  & 397.605  & TO      & TO       & TO   \\
  20    & OE=\textgreater MVE & neq     & \checkmark & \checkmark &            & 0.009  & 1.800    & 889.400 & TO       & TO   \\
  21    & OE=\textgreater MVE & neq     & \checkmark & \checkmark & \checkmark & 0.010  & 3.687    & 894.358 & TO       & TO   \\
  22    & OE=\textgreater MVE & PB      &            &            &            & 0.015  & 2.935    & 227.920 & TO       & TO   \\
  23    & OE=\textgreater MVE & PB      &            &            & \checkmark & 0.018  & 3.293    & 510.718 & TO       & TO   \\
  24    & OE=\textgreater MVE & PB      & \checkmark &            &            & 0.009  & 3.828    & 290.225 & TO       & TO   \\
  25    & OE=\textgreater MVE & PB      & \checkmark &            & \checkmark & 0.012  & 3.101    & 316.683 & TO       & TO   \\
  26    & OE=\textgreater MVE & 大野3   &            &            &            & 0.012  & 1.869    & 130.062 & TO       & TO   \\
  27    & OE=\textgreater MVE & 大野3   &            &            & \checkmark & 0.014  & 2.297    & 170.116 & 3330.446 & TO   \\
  28    & OE=\textgreater MVE & 大野3   & \checkmark &            &            & 0.014  & 1.866    & 128.660 & 2373.458 & TO   \\
  29    & OE=\textgreater MVE & 大野3   & \checkmark &            & \checkmark & 0.015  & 2.050    & 144.217 & 4769.444 & TO   \\
  30    & OE=\textgreater MVE & 大野4   &            &            &            & 0.013  & 2.000    & 148.720 & 3169.771 & TO   \\
  31    & OE=\textgreater MVE & 大野4   &            &            & \checkmark & 0.015  & 2.302    & 226.185 & 2754.992 & TO   \\
  32    & OE=\textgreater MVE & 大野4   & \checkmark &            &            & 0.011  & 1.875    & 105.552 & TO       & TO   \\
  33    & OE=\textgreater MVE & 大野4   & \checkmark &            & \checkmark & 0.009  & 2.094    & 131.638 & TO       & TO   \\\hline
 \end{tabular}
}
    \end{block}
\end{frame}


%%%%%%%%%%%%%%%%%%%%%%%%%%%%%%%%%%%%%%
% 追加実験
%%%%%%%%%%%%%%%%%%%%%%%%%%%%%%%%%%%%%%
\begin{frame}
    \frametitle{色数を増やした実験結果全体}
    kernel制約の有効性を確認するためにクイーングラフ彩色問題の色数を1色増やして実験を行った\footnote{ALT1は{\alldifferent}制約の要素数とサイズが一致する時のみ追加されるため,今回は省略している}
    \begin{block}{}
        {\fontsize{5pt}{5pt}\selectfont  \begin{tabular}[c] {|c|c|c|c|c||r|r|r|r|r|}\hline
  model & 符号    & alldiff & PHP & kernel & N=8     & N=9      & N=10      & N=11      & N=12 \\
        &         &         &     &        & C=9     & C=10     & C=11      & C=12      & C=13 \\\hline
  0     & OE      & neq     &    &        & 1.053   & 2.280    & TO        & TO        & TO \\
  1     & OE      & neq     & \checkmark   &        & \alert{0.036}   & 10.663   & TO        & TO        & TO \\
  4     & OE\textless=\textgreater DE & neq     &    &        & 1.456   & 11.838   & TO        & TO        & TO \\
  6     & OE\textless=\textgreater DE & neq     & \checkmark   &        & 1.197   & 13.797   & 1042.342  & TO        & TO \\
  8     & OE\textless=\textgreater DE & PB      &    &        & 4.125   & 5.133    & 4712.599  & TO        & TO \\
  9     & OE\textless=\textgreater DE & PB      & \checkmark   &        & 1.855   & 9.640    & 5248.163  & TO        & TO \\
  10    & OE\textless=\textgreater DE & 大野3   &    &        & 0.502   & 18.096   & TO        & TO        & TO \\
  11    & OE\textless=\textgreater DE & 大野3   & \checkmark   &        & 1.936   & 1.958    & 1438.766  & TO        & TO \\
  12    & OE\textless=\textgreater DE & 大野4   &    &        & 0.404   & 24.845   & \alert{123.527}   & TO        & TO \\
  13    & OE\textless=\textgreater DE & 大野4   & \checkmark   &        & 1.457   & 19.329   & 2682.197  & TO        & TO \\
  14    & OE=\textgreater MVE & neq     &    &       & 0.091   & 6.319    & 542.155   & TO        & TO \\
  15    & OE=\textgreater MVE & neq     &    & \checkmark      & 0.426   & 11.204   & 148.194   & TO        & TO \\
  18    & OE=\textgreater MVE & neq     & \checkmark   &       & 3.698   & 9.898    & 2181.056  & TO        & TO \\
  19    & OE=\textgreater MVE & neq     & \checkmark   & \checkmark      & 0.135   & 4.444    & TO        & TO        & TO \\
  22    & OE=\textgreater MVE & PB      &    &       & 0.551   & \alert{0.079}    & TO        & TO        & TO \\
  23    & OE=\textgreater MVE & PB      &    & \checkmark      & 0.558   & 1.701    & 1557.683  & TO        & TO \\
  24    & OE=\textgreater MVE & PB      & \checkmark   &       & 5.756   & 6.394    & 1834.187  & TO        & TO \\
  25    & OE=\textgreater MVE & PB      & \checkmark   & \checkmark      & 0.157   & 11.333   & 1059.484  & TO        & TO \\
  26    & OE=\textgreater MVE & 大野3   &    &       & 1.711   & 59.162   & 4852.973  & TO        & TO \\
  27    & OE=\textgreater MVE & 大野3   &    & \checkmark      & 2.690   & 33.712   & TO        & TO        & TO \\
  28    & OE=\textgreater MVE & 大野3   & \checkmark   &       & 4.869   & 8.683    & TO        & TO        & TO \\
  29    & OE=\textgreater MVE & 大野3   & \checkmark   & \checkmark      & 6.598   & 137.804  & TO        & TO        & TO \\
  30    & OE=\textgreater MVE & 大野4   &    &       & 2.954   & 16.938   & 3903.732  & TO        & TO \\
  31    & OE=\textgreater MVE & 大野4   &    & \checkmark      & 3.754   & 111.038  & TO        & TO        & TO \\
  32    & OE=\textgreater MVE & 大野4   & \checkmark   &       & 5.778   & 10.363   & TO        & TO        & TO \\
  33    & OE=\textgreater MVE & 大野4   & \checkmark   & \checkmark      & 4.880   & 74.860   & TO        & TO        & TO \\\hline
  \end{tabular}
}
    \end{block}
\end{frame}


%%%%%%%%%%%%%%%%%%%%%%%%%%%%%%%%%%%%%%
% OEとMVEのチャネリング
%%%%%%%%%%%%%%%%%%%%%%%%%%%%%%%%%%%%%%
\begin{frame}
    \frametitle{OEとMVEのチャネリング}\small
    OEとMVEをチャネリングさせる手法は以下の通りである.
    \begin{block}{}
        $x \in \{\ell \dots u\}$について,$x$が$a (a \in \{\ell \dots u\})$であることを表す01変数$x_{a}$を用意し,以下の制約を追加する.
        \vspace{-3mm}
        \[
            (x = a) \Rightarrow (x_{a} = 1)
        \]
    \end{block}
    \begin{exampleblock}{例}
        $x \in \{1 \dots 4\}$であるときには,01変数$x_a$($a \in \{ 1 \dots 4\}$)と以下の制約が追加される.
        \vspace{-3mm}
        \begin{eqnarray*}
            (x = 1) \Rightarrow (x_1 = 1) \\
            (x = 2) \Rightarrow (x_2 = 1) \\
            (x = 3) \Rightarrow (x_3 = 1) \\
            (x = 4) \Rightarrow (x_4 = 1)
        \end{eqnarray*}
    \end{exampleblock}
    OEをMVEとチャネリングさせることで{\alldifferent}制約をPBや大野3,大野4で表現できる.\\
    また,ヒント制約としてkernel制約が使用できる.
\end{frame}





\backupend

%%% Local Variables:
%%% mode: japanese-latex
%%% TeX-master: "slide"
%%% End:
