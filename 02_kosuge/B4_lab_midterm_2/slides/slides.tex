\documentclass [dvipdfmx,11pt]{beamer}
\usepackage{bxdpx-beamer}
\usepackage{pxjahyper}
\usepackage{amsmath}
\usepackage{bm}
\usepackage{minijs}
\usepackage{tikz}
\usepackage{multicol}
\usepackage{amssymb}
%\usepackage{otf}
%\renewcommand{\kanjifamilydefault}{\gtdefault}
%%% Beamer
%\AtBeginDvi{\special{pdf:tounicode EUC-UCS2}}
%\usetheme{Madrid}
%\usetheme{Copenhagen}
\usetheme{Warsaw}
% \renewcommand{\kanjifamilydefault}{\gtdefault}
\usefonttheme{structurebold}
%\usefonttheme{professionalfonts}
\setbeamertemplate{blocks}[shadow=true,rounded]
% \setbeamercolor{structure}{fg=blue!60!black}
\setbeamercolor{structure}{fg=blue!50!black}
\setbeamercolor{item projected}{fg=black,bg=blue!20!white}
%\setbeamercolor{alerted text}{fg=red!80!black}
\setbeamercolor{alerted text}{fg=red!70!black}
\setbeamertemplate{navigation symbols}{}
\useoutertheme[subsection=false]{miniframes}
\setbeamertemplate{footline}[frame number]
%%% Tikz
\usetikzlibrary{intersections, calc, arrows}
\setbeamertemplate{navigation symbols}{}
\setbeamertemplate{itemize item}[circle]
\setbeamersize{text margin left=1.5em,text margin right=1.5em}
\setlength{\abovedisplayskip}{0pt} % 上部のマージン
\setlength{\belowdisplayskip}{0pt} % 下部のマージン
\setlength{\columnsep}{0pt}
%
%
%
% footer setting %
\makeatother
\setbeamertemplate{footline}
{
    \leavevmode%
    \hbox{%
        \begin{beamercolorbox}[wd=.4\paperwidth,ht=2.25ex,dp=1ex,center]{author in head/foot}%
            \usebeamerfont{author in head/foot}\insertshortauthor
        \end{beamercolorbox}%
        \begin{beamercolorbox}[wd=.6\paperwidth,ht=2.25ex,dp=1ex,center]{title in head/foot}%
            \usebeamerfont{title in head/foot}\hspace*{1ex} \insertshorttitle\hspace*{2em}
            \textbf{ \insertframenumber{} / \inserttotalframenumber } \hspace*{1ex}
    \end{beamercolorbox}}%
    \vskip0pt%
}
\makeatletter
% exclude apprendix slides from framenumber %
\newcommand{\backupbegin}{
    \newcounter{framenumberappendix}
    \setcounter{framenumberappendix}{\value{framenumber}}
}
\newcommand{\backupend}{
    \addtocounter{framenumberappendix}{-\value{framenumber}}
    \addtocounter{framenumber}{\value{framenumberappendix}}
}


%%%%%%%%%%%% my macro %%%%%%%%%%%%%%%%%
\newcommand{\alldifferent}{$alldifferent$}
%%%%%%%%%%%%%%%%%%%%%%%%%%%%%%%%%%%%%%%


\title{alldifferent制約のSAT符号化と\\クイーングラフ彩色問題への応用}
% \title{チャネリング制約を用いた\\ alldifferent 制約の SAT 符号化}
\author{小菅脩司}
\date{2021年度番原研中間発表会\\2021年12月3日}
\institute{番原研究室}
\begin{document}
\begin{frame} {}
    \titlepage
\end{frame}
%%%%%%%%%%%%%%%%%%%%%%%%%%%%%%%%%%%%%%



%%%%%%%%%%%%%%%%%%%%%%%%%%%%%%%%%%%%%%
% Sugar
%%%%%%%%%%%%%%%%%%%%%%%%%%%%%%%%%%%%%%
\begin{frame}
    \frametitle{Sugar}
    \begin{alertblock}{}
        \alert{\bf Sugar}は順序符号化(OE)というSAT符号化手法をベースとしたSAT型制約ソルバーである.
    \end{alertblock}
    \begin{itemize}
        \item Sugarの特徴は表現能力の高い述語論理で問題を簡潔に記述できる点や,解を求める外部のSATソルバーを切り替えることができる点である.
        % \item 制約充足問題や制約最適化問題,最大制約充足問題に対応できる.
        \item SugarはCSP Solver Competitionで好成績を残しており,特にAlldiff部門で1位にもなっている.
        \item OEは他の直接符号化(DE)や多値符号化(MVE)などに比べて平均的に良い性能を持つことが知られている.
        \item また,OEとDEを組み合わせることで,より高速に{\alldifferent}制約を解くことができるという研究もある.
    \end{itemize}
\end{frame}



%%%%%%%%%%%%%%%%%%%%%%%%%%%%%%%%%%%%%%
% alldifferent制約と制約充足問題
%%%%%%%%%%%%%%%%%%%%%%%%%%%%%%%%%%%%%%
\begin{frame}
    \frametitle{alldifferent制約と制約充足問題}
    \begin{alertblock}{}
        \bm{$$alldifferent(x_{1},x_{2},\ldots, x_{n})$$}
        {\alldifferent}制約は,整数上の変数$x_{i}$が互いに異なることを表す制約
        である.
    \end{alertblock}
    \begin{itemize}
        \item この制約は,
            $$\bigwedge_{1 \leq i < j \leq n} x_i \neq x_j$$
            を意味する.
        \item {\alldifferent}制約は,時間割問題,グラフ彩色問題,組合せデザイン
            など様々な制約充足問題に現れる.
        \item そのような問題に対し,{\alldifferent}制約を効率良く解くことは重要
            な研究課題である.
    \end{itemize}
\end{frame}



%%%%%%%%%%%%%%%%%%%%%%%%%%%%%%%%%%%%%%
% alldifferent制約が現れる制約充足問題の例
%%%%%%%%%%%%%%%%%%%%%%%%%%%%%%%%%%%%%%
\begin{frame}
    \frametitle{{\alldifferent}制約が現れる制約充足問題の例}
    \begin{block}{クイーングラフ彩色問題}
        $N$個ずつの$N$個のグループからなるクイーン (計$N^2$個) を,
        $N\times N$のチェス盤に,同じグループのクイーン同士が互いに取られ
        ないように配置する問題
    \end{block}
    \begin{exampleblock}{}\centering
        \begin{columns}
            \begin{column}{0.48\textwidth}\centering
                %左側のテキストや画像
                \includegraphics[width=3cm]{images/qgcp_5.jpg}
            \end{column}
            \begin{column}{0.48\textwidth}\centering
                %右側のテキストや画像
                \includegraphics[width=3cm]{images/qgcp_5_c.jpg}
            \end{column}
        \end{columns}
    \end{exampleblock}
    \begin{itemize}
        \item この問題は{\alldifferent}制約のみを用いて記述できる.
        \item Knuthの教科書 The Art of Computer Programming でも
            取り上げられている.
    \end{itemize}
\end{frame}



%%%%%%%%%%%%%%%%%%%%%%%%%%%%%%%%%%%%%%
% 研究概要
%%%%%%%%%%%%%%%%%%%%%%%%%%%%%%%%%%%%%%
\begin{frame}
    \frametitle{研究概要}
    \begin{alertblock}{研究目的}
        クイーングラフ彩色問題を題材とし,Sugarの{\alldifferent}制約を高速化するための
        符号化手法やヒント制約を実装し,比較評価する.
    \end{alertblock}
    \begin{block}{研究内容}
        \begin{enumerate}
            \item \structure{{\alldifferent}制約に対し,3種類の符号化法を実装}
                \vspace{-1mm}
                \begin{itemize}
                    \item {\footnotesize SugarのOEをそのまま用いた手法(OE) }
                    \item {\footnotesize \alert{\bf OEとDEをチャネリングさせた手法(OE{\textless=\textgreater}DE)} }
                    \item {\footnotesize OEとMVEをチャネリングさせた手法(OE{=\textgreater}MVE) }
                \end{itemize}
                \vspace{-3mm}
            \item \structure{{\alldifferent}制約に対し,4種類の表現方法を実装}
                \vspace{-5mm}
                \begin{multicols}{2}
                    \begin{itemize}
                        \item {\footnotesize not-equal制約に分解(neq) }
                        \item {\footnotesize \alert{\bf ブール基数制約で表す(PB)} }
                        \item {\footnotesize ~[大野,'19]の手法3(大野3) }
                        \item {\footnotesize ~[大野,'19]の手法4(大野4) }
                    \end{itemize}
                \end{multicols}
        % \begin{columns}
        %     \begin{column}{0.48\textwidth}\centering
        %         %左側のテキストや画像
        %             \begin{itemize}
        %                 \item {\footnotesize not-equal制約に分解(neq) }
        %                 \item {\footnotesize \alert{\bf ブール基数制約で表す(PB)} }
        %             \end{itemize}
        %     \end{column}
        %     \begin{column}{0.48\textwidth}\centering
        %         %右側のテキストや画像
        %             \begin{itemize}
        %                 \item {\footnotesize ~[大野,'19]の3つ目の手法(大野3) }
        %                 \item {\footnotesize ~[大野,'19]の4つ目の手法(大野4) }
        %             \end{itemize}
        %     \end{column}
        % \end{columns}
                \vspace{-5mm}
            \item \structure{{\alldifferent}制約に対し,3種類のヒント制約を実装}
                \vspace{-1mm}
                \begin{itemize}
                    \item {\footnotesize \alert{\bf 鳩の巣原理を用いたヒント制約(PHP)} }
                    \item {\footnotesize at-least-one制約を用いたヒント制約(ALT1) }
                    \item {\footnotesize kernel制約を用いたヒント制約(kernel) }
                \end{itemize}
                \vspace{-3mm}
            \item \structure{クイーングラフ彩色問題($5\leq N \leq 13$)を用いた評価実験}
        \end{enumerate}
    \end{block}
\end{frame}



%%%%%%%%%%%%%%%%%%%%%%%%%%%%%%%%%%%%%%
% OEとDEのチャネリング
%%%%%%%%%%%%%%%%%%%%%%%%%%%%%%%%%%%%%%
\begin{frame}
    \frametitle{OEとDEのチャネリング}
    OEとDEをチャネリングさせる手法は以下の通りである.
    \begin{block}{}
        $x \in \{\ell \dots u\}$について,$x$が$a (a \in \{\ell \dots u\})$であることを表す01変数$x_{a}$を用意し,以下の制約を追加する.
        \vspace{-3mm}
        \[
            (x = a) \Leftrightarrow (x_{a} = 1)
        \]
    \end{block}
    \begin{exampleblock}{例}
        $x \in \{1 \dots 4\}$であるときには,01変数$x_a$($a \in \{ 1 \dots 4\}$)と以下の制約が追加される.
        \vspace{-3mm}
        \begin{eqnarray*}
            (x = 1) \Leftrightarrow (x_1 = 1) \\
            (x = 2) \Leftrightarrow (x_2 = 1) \\
            (x = 3) \Leftrightarrow (x_3 = 1) \\
            (x = 4) \Leftrightarrow (x_4 = 1)
        \end{eqnarray*}
    \end{exampleblock}
    OEをDEとチャネリングさせることで{\alldifferent}制約をPBや大野3,大野4で表現できる.
\end{frame}


%%%%%%%%%%%%%%%%%%%%%%%%%%%%%%%%%%%%%%
% 疑似ブール制約
%%%%%%%%%%%%%%%%%%%%%%%%%%%%%%%%%%%%%%
% \begin{frame}
%     \frametitle{{\alldifferent}制約のブール基数制約を用いた表現方法(PB)}
%     {\alldifferent}制約の分解方法としてブール基数制約を用いる.
%     \begin{block}{}
%         $alldifferent(x_1,\dots,x_n)$について,$x_i \in \{\ell,\ell+1,\dots u\}$である時,以下のように分解する.
%         \vspace{-6mm}
%         \begin{eqnarray}
%             x_{il} + \dots + x_{iu} = 1 & (i=1,\dots,n) \\
%             x_{1j} + \dots + x_{nj} \leq 1 & (j=\ell,\dots,u)
%         \end{eqnarray}
%         \vspace{-3mm}
%         $u-\ell = n-1$である時,式(2)は以下の式に変更される.
%         \vspace{-1mm}
%         \begin{eqnarray}
%             x_{1j} + \dots + x_{nj} = 1 & (j=\ell,\dots,u)
%         \end{eqnarray}
%     \end{block}
%     \begin{exampleblock}{例}
%         $alldifferent(x_1, x_2, x_3)$について, $x_i \in \{1,2,3\}$であるとき,以下の制約に分解される.
%         \vspace{-3mm}
%         \begin{eqnarray*}
%             x_{11} + x_{12} + x_{13} = 1 & x_{11} + x_{21} + x_{31} = 1 \\
%             x_{21} + x_{22} + x_{23} = 1 & x_{12} + x_{22} + x_{32} = 1 \\
%             x_{31} + x_{32} + x_{33} = 1 & x_{13} + x_{23} + x_{33} = 1
%         \end{eqnarray*}
%     \end{exampleblock}
% \end{frame}
\begin{frame}
    \frametitle{{\alldifferent}制約のブール基数制約を用いた表現方法(PB)}
    {\alldifferent}制約の分解方法としてブール基数制約を用いる.
    \begin{block}{}
        $alldifferent(x_1,\dots,x_n)$について,$x_i \in \{\ell,\ell+1,\dots u\}$である時,$x_i$が$a \, (a \in \{\ell, \dots, u\})$であることを表す01変数$x_{ia}$を用いて,以下のように分解する.
        \vspace{-3mm}
        \begin{eqnarray*}
            \begin{cases}
                x_{1j} + \dots + x_{nj} = 1 \, (j=\ell,\dots,u)    & (if \; u-\ell = n-1) \\
                x_{1j} + \dots + x_{nj} \leq 1 \, (j=\ell,\dots,u) & (if \; u-\ell > n-1)
            \end{cases}
        \end{eqnarray*}
    \end{block}
    \begin{exampleblock}{例}
        $alldifferent(x_1, x_2, x_3)$について, $x_i \in \{1,2,3,4\}$であるとき,以下の制約に分解される.
        \vspace{-3mm}
        \begin{eqnarray*}
            x_{11} + x_{21} + x_{31} \leq 1 \\
            x_{12} + x_{22} + x_{32} \leq 1 \\
            x_{13} + x_{23} + x_{33} \leq 1 \\
            x_{14} + x_{24} + x_{34} \leq 1
        \end{eqnarray*}
    \end{exampleblock}
\end{frame}


%%%%%%%%%%%%%%%%%%%%%%%%%%%%%%%%%%%%%%
% 鳩の巣原理を用いたヒント制約(PHP)~[田島・田村,2008]
%%%%%%%%%%%%%%%%%%%%%%%%%%%%%%%%%%%%%%
\begin{frame}
    \frametitle{鳩の巣原理を用いたヒント制約(PHP)~[田島・田村,2008]}
    SAT符号化された{\alldifferent}制約に,鳩の巣原理を用いたヒントを加える
    と求解速度が向上することが知られている.
    \begin{block}{}
        $alldifferent(x_{1},\ldots,x_{n})$について,$x_i \in
        \{\ell,\ell+1,\ldots,u\}$であるとき,以下の2つの制約を追加する.
        \[
            \bigvee_{i=1}^{n}x_{i}\geq \ell+n-1 \qquad
            \bigvee_{i=1}^{n}x_{i}\leq u-n+1
        \]
    \end{block}
    \begin{exampleblock}{例}
        $alldifferent(x_1, x_2, x_3)$について, $x_i \in \{1,2,3\}$であるとき,以下の制約が追加される.
        \vspace{-3mm}
        \begin{eqnarray*}
            (x_1\geq 3) \lor (x_2 \geq 3) \lor (x_3 \geq 3)\\
            (x_1\leq 1) \lor (x_2 \leq 1) \lor (x_3 \leq 1)
        \end{eqnarray*}
    \end{exampleblock}
\end{frame}






%%%%%%%%%%%%%%%%%%%%%%%%%%%%%%%%%%%%%%
% 実験概要
%%%%%%%%%%%%%%%%%%%%%%%%%%%%%%%%%%%%%%
\begin{frame}
    \frametitle{実験概要}
    実装したSAT符号化及び{\alldifferent}制約の高速化手法を評価するために,以下の実験を行なった.
    \begin{itemize}
        \item \structure{比較に用いた実装(14個)}\footnote{予備実験で性能の悪かったOE={\textgreater}MVEモデル(計20個)は省略}\\
            % \vspace{-3mm}
            % \begin{block}{}\centering
            %     {\fontsize{4pt}{5pt}\selectfont  \begin{tabular}[c] {c|c|c|c|c|c|c|c|c}
   & 整数変数の & \multicolumn{4}{|c|}{alldifferent制約の分解} & PHP & ALT1 \\\cline{3-6}
   & 符号化法   & $\neq$分解 & PB & PB3 & PB4 & & &\\\hline\hline
  0     & OE                    & OE      &    &       &             &     &     & \\
  1     & OE                    & OE      &    &       &             & OE  &     & (Sugarと同じ)\\\hline
  2     & OE$\Leftrightarrow$DE & DE      &    &       &             &     &     & \\
  3     & OE$\Leftrightarrow$DE & DE      &    &       &             &     & DE  & [Gent+ '04]\\
  4     & OE$\Leftrightarrow$DE & DE      &    &       &             & OE  &     & \\
  5     & OE$\Leftrightarrow$DE & DE      &    &       &             & OE  & DE  & \\
  6     & OE$\Leftrightarrow$DE &         & OE &       &             &     &     & \\
  7     & OE$\Leftrightarrow$DE &         & OE &       &             & OE  &     & \\
  8     & OE$\Leftrightarrow$DE &         &    & OE    &             &     &     & \\
  9     & OE$\Leftrightarrow$DE &         &    & OE    &             & OE  &     & \\
  10    & OE$\Leftrightarrow$DE &         &    &       & OE          &     &     & \\
  11    & OE$\Leftrightarrow$DE &         &    &       & OE          & OE  &     &
 \end{tabular}
}
            % \end{block}
            \begin{itemize}
                \item OEをそのまま使う手法(4個)
                \item OEとDEをチャネリングさせる手法(10個)
            \end{itemize}
        \item \structure{ベンチマーク問題}
            \begin{itemize}
                \item $N$次クイーングラフ彩色問題 ($5\leq N\leq 13$)
                \item $N$次クイーングラフ彩色問題 ($5\leq N\leq 13$)の色数を1色増やした問題
            \end{itemize}
        \item \structure{SATソルバー}: Sugar ver.2.3.3 ,GlueMiniSat 2.2.10-193
        \item \structure{制限時間}: 1問あたり2時間
        \item \structure{実験環境}: Mac mini, 3.2GHz, 64GB メモリ
    \end{itemize}
\end{frame}



%%%%%%%%%%%%%%%%%%%%%%%%%%%%%%%%%%%%%%
% 実験結果
%%%%%%%%%%%%%%%%%%%%%%%%%%%%%%%%%%%%%%
\begin{frame}
    \frametitle{実験結果: 求解に要したCPU時間(秒)}
    \begin{block}{}\centering
        {\tiny \begin{tabular}{l|rrr} 
  & 到達可能 & 到達不能 & 合計 \\ \hline
  origin & 11 & 0 & 11 \\
  changed & 11 & 10 & 21 \\
  unchanged & 11 & 10 & 21 \\
\end{tabular}}
    \end{block}
    \begin{itemize}
        \item ヒントの有無で比較すると,ヒント有がより高速に解けていることから,ヒントが有効に働いていることがわかる.特にALT1のヒントが有効である.
        % \item 符号化方法を比較すると,$N=11$を解けていることから,OEをそのまま使うよりOEをDEとチャネリングさせてヒントを追加した方が性能が良い.
        \item UNSATで比較すると,DEとチャネリングさせALT1を使うか,PB・大野3・大野4を使うのが性能が良い
    \end{itemize}
\end{frame}



%%%%%%%%%%%%%%%%%%%%%%%%%%%%%%%%%%%%%%
% 実験結果2
%%%%%%%%%%%%%%%%%%%%%%%%%%%%%%%%%%%%%%
\begin{frame}
    \frametitle{実験結果2: 求解に要したCPU時間(秒)}
    {\alldifferent}制約の要素数と要素のサイズが等しくない場合について比較するために,色数を1色増やして実験を行った.\footnote{ALT1は{\alldifferent}制約の要素数とサイズが一致する時のみ追加されるため,今回は省略している}
    \begin{block}{}\centering
        {\tiny  \begin{tabular}[c] {|c|c|c|c||r|r|r|r|r|}\hline
  model & 符号    & alldiff & PHP &   N=8     & N=9      & N=10      & N=11      & N=12 \\
        &         &         &     &         C=9     & C=10     & C=11      & C=12      & C=13 \\\hline
  0     & OE      & neq     &    &         1.053   & 2.280    & TO        & TO        & TO \\
  1     & OE      & neq     & \checkmark   &         \alert{0.036}   & 10.663   & TO        & TO        & TO \\
  4     & OE\textless=\textgreater DE & neq     &    &         1.456   & 11.838   & TO        & TO        & TO \\
  5     & OE\textless=\textgreater DE & neq     & \checkmark           & 1.197   & 13.797   & 1042.342  & TO        & TO \\
  8     & OE\textless=\textgreater DE & PB      &    &         4.125   & 5.133    & 4712.599  & TO        & TO \\
  9     & OE\textless=\textgreater DE & PB      & \checkmark           & 1.855   & 9.640    & 5248.163  & TO        & TO \\
  10    & OE\textless=\textgreater DE & 大野3   &    &         0.502   & 18.096   & TO        & TO        & TO \\
  11    & OE\textless=\textgreater DE & 大野3   & \checkmark           & 1.936   & \alert{1.958}    & 1438.766  & TO        & TO \\
  12    & OE\textless=\textgreater DE & 大野4   &    &         0.404   & 24.845   & \alert{123.527}   & TO        & TO \\
  13    & OE\textless=\textgreater DE & 大野4   & \checkmark           & 1.457   & 19.329   & 2682.197  & TO        & TO \\\hline
  \end{tabular}
}
    \end{block}
    \begin{itemize}
        \item 符号化方法を比較すると,$N=10$を解けていることから,DEとチャネリングさせた方が性能が良いと言える.
    \end{itemize}
\end{frame}



%%%%%%%%%%%%%%%%%%%%%%%%%%%%%%%%%%%%%%
% まとめ
%%%%%%%%%%%%%%%%%%%%%%%%%%%%%%%%%%%%%%
\begin{frame}
    \frametitle{まとめ}
    \begin{enumerate}
        \item \structure{{\alldifferent}制約に対して,3種34個の実装方法を提案}
            \begin{itemize}
                \item OEをそのまま使う手法(4個)
                \item OEとDEをチャネリングさせる手法(10個)
                \item OEとMVEをチャネリングさせる手法(20個)
            \end{itemize}
        \item \structure{クイーングラフ彩色問題($5 \leq N \leq 13$)を用いた評価実験}
            \begin{itemize}
                \item {\alldifferent}制約を高速化するにはOEをDEとチャネリングさせるのが良い.
                \item ヒント制約としてはat-least-one制約が性能が良い
                \item OEとMVEをチャネリングさせたモデルは性能が悪かった.
                \item kernel制約の有効性は確認できなかった.
            \end{itemize}
    \end{enumerate}
    \begin{alertblock}{今後の課題}
        \begin{itemize}
            \item 色数を増やした場合の実験結果についてより詳しい調査を行う.
        \end{itemize}
    \end{alertblock}
\end{frame}



%%%%%%%%%%%%%%%%%%%%%%%%%%%%%%%%%%%%%%
% 12次クイーングラフ彩色問題の解
%%%%%%%%%%%%%%%%%%%%%%%%%%%%%%%%%%%%%%
\begin{frame}
    \frametitle{12次クイーングラフ彩色問題の解}
    \begin{columns}
        \begin{column}{0.4\textwidth}
            %左側のテキストや画像
            実験において$N=11$で性能の良かった上位3モデルを用いて,$N=12$を制限時間1週間で
            実験した結果を示す.
        \end{column}
        \begin{column}{0.6\textwidth}
            %右側のテキストや画像
            \begin{exampleblock}{}\centering
                \includegraphics[width=5cm]{images/qgcp_12.jpg}
            \end{exampleblock}
        \end{column}
    \end{columns}
    \begin{block}{}\centering
        {\tiny  \begin{tabular}[c] {|c|c|c|c|c||r|}\hline
 model & 符号    & alldiff & PHP & ALT1  & N=12 \\
       &         &         &     &       & SAT  \\\hline
 2     & OE      & neq     &     & \checkmark     & 123124.213  \\
 7     & OE{\textless=\textgreater}DE & neq     & \checkmark   & \checkmark     & TO   \\
 11     & OE{\textless=\textgreater}DE & 大野3   & \checkmark   &       & 142694.686 \\\hline
%  13    & OE{\textless=\textgreater}DE & 大野4   & \checkmark   &       & TO   \\\hline
 \end{tabular}
}
    \end{block}
\end{frame}
% %%%% 補助スライド


%%%%%%%%%%%%%%%%%%%%%%%%%%%%%%%%%%%%%%%%%%%%%%%%%%%%%%%%%%%%%
% %%%% 補助スライド
%%%%%%%%%%%%%%%%%%%%%%%%%%%%%%%%%%%%%%%%%%%%%%%%%%%%%%%%%%%%%

\appendix

\backupbegin

%%%%%%%%%%%%%%%%%%%%%%%%%%%%%%%%%%%%%%
% ~
%%%%%%%%%%%%%%%%%%%%%%%%%%%%%%%%%%%%%%
\begin{frame}
    \frametitle{~}
    \centering
    - 補足用 -
\end{frame}



%%%%%%%%%%%%%%%%%%%%%%%%%%%%%%%%%%%%%%
% 鳩の巣原理を用いたヒント制約(PHP)~[田島・田村,2008]
%%%%%%%%%%%%%%%%%%%%%%%%%%%%%%%%%%%%%%
\begin{frame}
    \frametitle{鳩の巣原理を用いたヒント制約(PHP)~[田島・田村,2008]}
    SAT符号化された{\alldifferent}制約に,鳩の巣原理を用いたヒントを加える
    と求解速度が向上することが知られている.
    \begin{block}{}
        $alldifferent(x_{1},\ldots,x_{n})$について,$x_i \in
        \{\ell,\ell+1,\ldots,u\}$であるとき,以下の2つの制約を追加する.
        \[
            \bigvee_{i=1}^{n}x_{i}\geq \ell+n-1 \qquad
            \bigvee_{i=1}^{n}x_{i}\leq u-n+1
        \]
    \end{block}
    \begin{exampleblock}{例}
        $alldifferent(x_1, x_2, x_3)$について, $x_i \in \{1,2,3\}$であるとき,以下の制約が追加される.
        \vspace{-3mm}
        \begin{eqnarray*}
            (x_1\geq 3) \lor (x_2 \geq 3) \lor (x_3 \geq 3)\\
            (x_1\leq 1) \lor (x_2 \leq 1) \lor (x_3 \leq 1)
        \end{eqnarray*}
    \end{exampleblock}
\end{frame}


%%%%%%%%%%%%%%%%%%%%%%%%%%%%%%%%%%%%%%
% at-least-one制約
%%%%%%%%%%%%%%%%%%%%%%%%%%%%%%%%%%%%%%
\begin{frame}
    \frametitle{at-least-one制約を用いたヒント制約(ALT1)}
    \vspace{-3mm}
    \begin{block}{}
        $alldfifferent(x_1,x_2,\ldots,x_n)$について, $x_i \in \{\ell, \ell+1,\ldots, u\}$かつ$u-\ell=n-1$であるときに以下の制約を追加する.\\
        \vspace{-3mm}
        $$\bigvee_{i=1}^n x_i=a \qquad (a \in \{\ell,\ldots, u\})$$
    \end{block}
    \begin{exampleblock}{at-least-one制約を用いたヒント制約の例}
        $alldifferent(x_1, x_2, x_3, x_4)$について, $x_i \in \{1, 2, 3, 4\}$であるときには以下の制約が追加される.
        \vspace{-3mm}
        \begin{eqnarray*}
            (x_1=1) \lor (x_2=1) \lor (x_3=1) \lor (x_4=1)\\
            (x_1=2) \lor (x_2=2) \lor (x_3=2) \lor (x_4=2)\\
            (x_1=3) \lor (x_2=3) \lor (x_3=3) \lor (x_4=3)\\
            (x_1=4) \lor (x_2=4) \lor (x_3=4) \lor (x_4=4)
        \end{eqnarray*}
    \end{exampleblock}
\end{frame}


%%%%%%%%%%%%%%%%%%%%%%%%%%%%%%%%%%%%%%
% alldifferent制約の擬似ブール符号化
%%%%%%%%%%%%%%%%%%%%%%%%%%%%%%%%%%%%%%
\begin{frame}
    \frametitle{{\alldiff}制約の擬似ブール符号化(PB)}
    {\alldiff}制約をブール基数制約で表現することができる.
    \begin{exampleblock}{}
        $x_i \in \{ 1 \dots d \}, n \geq d$ である $\Alldiff$に対して,$p_{ij}=1 \Llra x_i=j$である$n$行$d$列の0-1行列($p_{ij}$)を導入する.
        \vspace{-3mm}
        \begin{displaymath}
            \begin{array}{cccc}
             & & &
             \begin{array}{cccc}
                 1&2&\dots&d
             \end{array}\\
                (p_{ij})&=&
                \begin{array}{c}x_1\\ x_2\\ \vdots\\ x_n \end{array}&
                \left(
                    \begin{array}{cccc}
                        p_{11}&p_{12}&\dots&p_{1d}\\
                        p_{21}&p_{22}&\dots&p_{2d}\\
                        \vdots&\vdots&\ddots&\vdots\\
                        p_{n1}&p_{n2}&\dots&p_{nd}
                \end{array}\right)
            \end{array}
        \end{displaymath}
        \begin{itemize}
            \item 各$x_i$はちょうど一つの値をとる.
            \vspace{-3mm}
                % $$ \sum_{j=1}^{d} p_{ij}=1 \; (i \in \{1,2,\ldots,n\}) $$
            $$ p_{i1} + \ldots + p_{id} = 1 \; (i \in \{1,2,\ldots,n\})$$
            \item 各列について1となるのは高々1つである.
            \vspace{-3mm}
            $$ p_{1j} + \ldots + p_{nj} \leq 1 \; (j \in \{1,2,\ldots,d\})$$
                % $$ \sum_{i=1}^{n} p_{ij} \leq 1 \; (j \in \{l,l+1,\ldots,u\})$$
                % これは$n=d$の時には等号にできる
                % $$\sum_{i=1}^{n} p_{ij} = 1 \; (j \in \{l,l+1,\ldots,u\})$$
        \end{itemize}
    \end{exampleblock}
\end{frame}

% %%%%%%%%%%%%%%%%%%%%%%%%%%%%%%%%%%%%%%
% % PB3
% %%%%%%%%%%%%%%%%%%%%%%%%%%%%%%%%%%%%%%
% \begin{frame}
%     \frametitle{PB3}
%     {\alldiff}制約をブール基数制約
%     \begin{exampleblock}{}
%         $x_i \in \{ 1 \dots d \}, n \geq d$ である $\Alldiff$に対して,$p_{ij}=1 \Llra x_i=j$である$n$行$d$列の0-1行列($p_{ij}$)を導入する.
%         \vspace{-3mm}
%         \begin{displaymath}
%             \begin{array}{cccc}
%              & & &
%              \begin{array}{cccc}
%                  1&2&\dots&d
%              \end{array}\\
%                 (p_{ij})&=&
%                 \begin{array}{c}x_1\\ x_2\\ \vdots\\ x_n \end{array}&
%                 \left(
%                     \begin{array}{cccc}
%                         p_{11}&p_{12}&\dots&p_{1d}\\
%                         p_{21}&p_{22}&\dots&p_{2d}\\
%                         \vdots&\vdots&\ddots&\vdots\\
%                         p_{n1}&p_{n2}&\dots&p_{nd}
%                 \end{array}\right)
%             \end{array}
%         \end{displaymath}
%         \begin{itemize}
%             \item 各$x_i$はちょうど一つの値をとる.
%             \vspace{-3mm}
%                 % $$ \sum_{j=1}^{d} p_{ij}=1 \; (i \in \{1,2,\ldots,n\}) $$
%             $$ p_{i1} + \ldots + p_{id} = 1 \; (i \in \{1,2,\ldots,n\})$$
%             \item 各列について1となるのは高々1つである.
%             \vspace{-3mm}
%             $$ p_{1j} + \ldots + p_{nj} \leq 1 \; (j \in \{1,2,\ldots,d\})$$
%                 % $$ \sum_{i=1}^{n} p_{ij} \leq 1 \; (j \in \{l,l+1,\ldots,u\})$$
%                 % これは$n=d$の時には等号にできる
%                 % $$\sum_{i=1}^{n} p_{ij} = 1 \; (j \in \{l,l+1,\ldots,u\})$$
%         \end{itemize}
%     \end{exampleblock}
% \end{frame}
%
\backupend

%%% Local Variables:
%%% mode: japanese-latex
%%% TeX-master: "slide"
%%% End:


\end{document}
