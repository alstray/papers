%%%%%%%%%%%%%%%%%%%%%%%%%%%%%%%%%%%%%%%%%%%%%%%%%%%%%%%%%% 
\chapter{序論}
\pagenumbering{arabic}
%%%%%%%%%%%%%%%%%%%%%%%%%%%%%%%%%%%%%%%%%%%%%%%%%%%%%%%%%% 
命題論理の充足可能性判定問題(Boolean SATisfiability; SAT)\cite{JSAI:InoueT10}は,与えられ
た命題論理式の充足可能性を判定する問題である.
SAT は,人工知能および計算機工学における最も基本的な問題として,システ
ム検証,プランニング,スケジューリングなど,さまざまな分野に応用されている.
近年,SAT の解を求める SAT ソルバーの性能が大きく向上し,
SAT を拡張・発展させた問題を中心に,SAT 技術が大きな広がりを見せている:
MaxSAT,
擬似ブール制約(PB),
限量ブール式(QBF),
%モデル列挙,
モデル計数(\#SAT),
背景理論付き SAT.

なかでも,背景理論付き SAT (Satisfiability Modulo Theories; SMT) \cite{JSAI:IwanumaN10}は,
等号や算術,配列やリスト,ビットベクターなど様々な背景理論が扱えるよ
うに,SAT を拡張・発展させた技術である.
SAT では,これらの背景理論を命題論理で記述する必要があるが,
複雑な背景理論を表現することは困難な場合が多い.
SMT は,背景理論をより表現能力の高い述語論理で記述できるため,
問題を簡潔に記述することができる点が特長である.
近年,SAT ソルバー
を拡張した高速 SMT ソルバーが開発され,制約充足問題,プログラム検証な
どへの応用が活発に研究されている.

制約プログラミングの言語である制約充足問題\cite{JSAI:TamuraTB10}は,与えられた制約を満たす解
を探索する問題である.
人工知能分野で生じる多くの組合せ問題は,制約充足問題として定式化
できることが知られている.
制約充足問題では,グローバル制約を用いて,複数の変数に対する複雑な
制約を簡潔に表現できる点が特長の一つである.
代表的なグローバル制約$distinct(x_{1},x_{2},\ldots x_{n})$は,
$x_{i}$が互いに異なる値をとることを表す.
この$distinct$制約は,記述性の向上を目的として SMT ソルバー
にも取り入れられている.
しかしながら,制約充足問題に対する SMT ソルバーの求解性能は,
SAT 型制約ソルバーと比べて劣っているとの報告もあり,
$distinct$制約を含めその効率的な実装は重要な研究課題となっている.

本論文では,D.~E.~Knuth の教科書
The Art of Computer Programming\cite{Knuth:TAOCP:SAT}
でも取り上げられているクイーングラフ彩
色問題を題材とし,その SMT 符号化と$distinct$制約の高速化について述べる.
クイーングラフ彩色問題の SMT 符号化として,
クイーンの色を表す整数変数を用いた定式化(色変数モデル),
クイーンの位置を表す整数変数を用いた定式化(位置変数モデル),
0-1変数を用いた定式化(0-1変数モデル),
チャネリング制約を用いて色変数モデルと0-1変数モデルをハイブリッドした
定式化(ハイブリッドモデル)を実装した.
また,$distinct$制約の高速化として,
色変数モデルと位置変数モデルでは,
SAT型制約ソルバーで有効性が示されている鳩の巣原理等に基づくヒント制約を追加した.
0-1変数モデルとハイブリッドモデルでは,
$distinct$制約のPB符号化\cite{Ono19:ai}を応用し,探索空間の枝刈りを実装した.

実装した SMT 符号化と$distinct$制約の高速化手法の有効性を評価するために,
クイーングラフ彩色問題($5\leq N\leq 13$)を用いた実行実験を行なった.
その結果,色変数モデル+ヒント制約と位置変数モデル+ヒント制約が,$N=11$
まで解を求め,最も良い性能を示した.
これにより,$distinct$制約の高速化について,
SATでのヒント制約が SMT においても有効であることが確認できた.
その一方で,チャネリング制約を用いたハイブリッドモデルは,
SMTソルバーの場合,求解性能が低下することが確認された.


% %%%%%%%%%%%%%%%%%%%%%%%%%%%%%%%%%%%%%%%%%%%%%%%%
% %%
% %% SMTについて
% %%
% %%%%%%%%%%%%%%%%%%%%%%%%%%%%%%%%%%%%%%%%%%%%%%%%
% SMTは,命題論理よりも表現力の高い論理体系で記述された背景理論を,SAT技法で効果的に取り扱うことを目的としている.
% SATが命題論理を扱うのに対し,SMTでは述語論理を扱う.
%
% SMT ソルバーは,プログラム検証, スケジューリング, プランニングなどの場面に使用される.
% SMTソルバーを用いる利点としては,等号や算術,配列やリスト,ビットベクターといった背景理論を切り替えて取り扱うことができる点や,
% 高い表現力により制約を簡潔に記述することができる点などが挙げられる.
%
%
% %%%%%%%%%%%%%%%%%%%%%%%%%%%%%%%%%%%%%%%%%%%%%%%%
% %%
% %% distinct制約について
% %%
% %%%%%%%%%%%%%%%%%%%%%%%%%%%%%%%%%%%%%%%%%%%%%%%%
% $distinct$制約とは,与えられた要素が互いに異なるという制約である.
% この制約は時間割問題やグラフ彩色問題など様々な制約充足問題で扱われる.
% そのため,この制約をより効率よく解くことによって多くの人がその恩恵を得られる.
%
%
% %%%%%%%%%%%%%%%%%%%%%%%%%%%%%%%%%%%%%%%%%%%%%%%%
% %%
% %% クイーングラフ彩色問題とは
% %%
% %%%%%%%%%%%%%%%%%%%%%%%%%%%%%%%%%%%%%%%%%%%%%%%%
% クイーングラフ彩色問題とは,N×Nの大きさのチェス盤にN色のクイーンの駒を各N色ずつ互いに取り合わないように配置する問題である.
% つまり,チェス盤上での各行,各列,各右上がり対角線,各右下がり対角線がに配置されるクイーンの色が互いに異なるということである.
%
% クイーングラフ彩色問題の解き方としては,色変数モデルと位置変数モデルがあり,
% 色変数モデルでは配置されるクイーンの駒の色を変数として問題を解き,
% 位置変数モデルではある行に対してある色のクイーンが何列に配置されるのかを変数として問題を解く解き方である.
%
%
%
%
% %%%%%%%%%%%%%%%%%%%%%%%%%%%%%%%%%%%%%%%%%%%%%%%%
% %%
% %% 本論文で何をするのか
% %%
% %%%%%%%%%%%%%%%%%%%%%%%%%%%%%%%%%%%%%%%%%%%%%%%%
% 本論文では,SMTソルバーにおける$distinct$制約の高速化手法について述べる.
% まず,1つ目の手法として$distinct$制約にヒント制約を追加するものを2種類用意した.
% 1つは鳩の巣原理を用いたヒント制約で,もう1つはドメインサイズと要素数が等しい場合のヒント制約である.
% 次に,2つ目の手法として$distinct$制約をPB符号化し解くものを3種類用意した.
% 1つはat-most-one制約を用いた基本的な手法であり,残り2つはその基本的な手法にヒント制約を付け足したものである,
% 3つ目の手法は2つ目のPB符号化して解く手法にチャネリング制約を用いて$distinct$制約の部分だけPB符号化させて解く手法である.
% 加えたヒント制約というのは$distinct$制約において探索の際に枝刈りを行うもので,これにより性能の向上が期待できる.
%
% %%%%%%%%%%%%%%%%%%%%%%%%%%%%%%%%%%%%%%%%%%%%%%%%
% %%
% %% 実験内容について
% %%
% %%%%%%%%%%%%%%%%%%%%%%%%%%%%%%%%%%%%%%%%%%%%%%%%
% 提案した$distinct$制約の高速化手法の有効性を評価するために,
% クイーングラフ彩色問題のN=5から13について,色変数モデルと位置変数モデルで実行実験を行なった.
%
% ヒント制約を追加する高速化手法では,
% 追加した2種類のヒント制約ではどちらも単に$distinct$制約のみを用いたものよりも高速に解くことができた.
%
% PB符号化を用いた高速化手法では,
% 3種類の符号化方法の内,2種類で単に$distinct$制約のみを用いたものよりも高速に解くことができた.
%
% % チャネリング制約を使用しPB符号化を用いた高速化手法はSMTソルバーでは性能低下を引き起こす原因になることがわかった.
%
% このことから,既存のSATソルバーの高速化手法がSMTでも有効であることが確認できた.


%%%%%%%%%%%%%%%%%%%%%%%%%%%%%%%%%%%%%%%%%%%%%%%%
%%
%% 論文構成について
%%
%%%%%%%%%%%%%%%%%%%%%%%%%%%%%%%%%%%%%%%%%%%%%%%%
本論文の構成は以下の通りである.
第2章ではSMTについて,
第3章ではクイーングラフ彩色問題についてSMTでのプログラム例を示しながらモデル化方法について説明を行う.
第4章では用いたSMTソルバにおける$distinct$制約の高速化手法についてプログラムを示しながら説明し,
第5章では作成したプログラムの比較実験とその考察について述べる.

%%% Local Variables:
%%% mode: latex
%%% TeX-master: "paper"
%%% End:
