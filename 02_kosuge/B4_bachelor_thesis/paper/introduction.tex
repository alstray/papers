%%%%%%%%%%%%%%%%%%%%%%%%%%%%%%%%%%%%%%%%%%%%%%%%%%%%%%%%%% 
\chapter{序論}
\pagenumbering{arabic}
%%%%%%%%%%%%%%%%%%%%%%%%%%%%%%%%%%%%%%%%%%%%%%%%%%%%%%%%%% 


%%%%%%%%%%%%%%%%%%%%%%%%%%%%%%%%%%%%%%%%%%%%%%%%
%%
%% SMTについて
%%
%%%%%%%%%%%%%%%%%%%%%%%%%%%%%%%%%%%%%%%%%%%%%%%%
SMTは,命題論理よりも表現力の高い論理体系で記述された背景理論を,SAT技法で効果的に取り扱うことを目的としている.
SATが命題論理を扱うのに対し,SMTでは述語論理を扱う.
SMTソルバーを用いる利点としては,等号や算術,配列やリスト,ビットベクターといった背景理論を切り替えて取り扱うことができる点や,
高い表現力により制約を簡潔に記述することができる点などが挙げられる.


%%%%%%%%%%%%%%%%%%%%%%%%%%%%%%%%%%%%%%%%%%%%%%%%
%%
%% distinct制約について
%%
%%%%%%%%%%%%%%%%%%%%%%%%%%%%%%%%%%%%%%%%%%%%%%%%
$distinct$制約とは,与えられた要素が互いに異なるという制約である.
この制約は時間割問題やグラフ彩色問題など様々な制約充足問題で扱われる.
そのため,この制約をより効率よく解くことによって多くの人がその恩恵を得られる.


%%%%%%%%%%%%%%%%%%%%%%%%%%%%%%%%%%%%%%%%%%%%%%%%
%%
%% クイーングラフ彩色問題とは
%%
%%%%%%%%%%%%%%%%%%%%%%%%%%%%%%%%%%%%%%%%%%%%%%%%
クイーングラフ彩色問題とは,N×Nの大きさのチェス盤にN色のクイーンの駒を各N色ずつ互いに取り合わないように配置する問題である.
つまり,チェス盤上での各行,各列,各右上がり対角線,各右下がり対角線がに配置されるクイーンの色が互いに異なるということである.

クイーングラフ彩色問題の解き方としては,色変数モデルと位置変数モデルがあり,
色変数モデルでは配置されるクイーンの駒の色を変数として問題を解き,
位置変数モデルではある行に対してある色のクイーンが何列に配置されるのかを変数として問題を解く解き方である.




%%%%%%%%%%%%%%%%%%%%%%%%%%%%%%%%%%%%%%%%%%%%%%%%
%%
%% 本論文で何をするのか
%%
%%%%%%%%%%%%%%%%%%%%%%%%%%%%%%%%%%%%%%%%%%%%%%%%
本論文では,SMTソルバーにおける$distinct$制約の高速化手法について述べる.
まず,1つ目の手法として$distinct$制約にヒント制約を追加するものを2種類用意した.
1つは鳩の巣原理を用いたヒント制約で,もう1つはドメインサイズと要素数が等しい場合のヒント制約である.
次に,2つ目の手法として$distinct$制約をPB符号化し解くものを3種類用意した.
1つはat-most-one制約を用いた基本的な手法であり,残り2つはその基本的な手法にヒント制約を付け足したものである,
3つ目の手法は2つ目のPB符号化して解く手法にチャネリング制約を用いて$distinct$制約の部分だけPB符号化させて解く手法である.
加えたヒント制約というのは$distinct$制約において探索の際に枝刈りを行うもので,これにより性能の向上が期待できる.

%%%%%%%%%%%%%%%%%%%%%%%%%%%%%%%%%%%%%%%%%%%%%%%%
%%
%% 実験内容について
%%
%%%%%%%%%%%%%%%%%%%%%%%%%%%%%%%%%%%%%%%%%%%%%%%%
提案した$distinct$制約の高速化手法の有効性を評価するために,
クイーングラフ彩色問題のN=5から13について,色変数モデルと位置変数モデルで実行実験を行なった.

ヒント制約を追加する高速化手法では,
追加した2種類のヒント制約ではどちらも単に$distinct$制約のみを用いたものよりも高速に解くことができた.

PB符号化を用いた高速化手法では,
3種類の符号化方法の内,2種類で高速に解くことができた.

% チャネリング制約を使用しPB符号化を用いた高速化手法では

このことから,既存のSATソルバーの高速化手法がSMTでも有効であることが確認できた.


%%%%%%%%%%%%%%%%%%%%%%%%%%%%%%%%%%%%%%%%%%%%%%%%
%%
%% 論文構成について
%%
%%%%%%%%%%%%%%%%%%%%%%%%%%%%%%%%%%%%%%%%%%%%%%%%
本論文の構成は以下の通りである.
第2章ではSMTについて,
第3章ではクイーングラフ彩色問題についてSMTでのプログラム例を示しながらモデル化方法について説明を行う.
第4章では用いたSMTソルバにおける$distinct$制約の高速化手法についてプログラムを示しながら説明し,
第5章では作成したプログラムの比較実験とその考察について述べる.

%%% Local Variables:
%%% mode: latex
%%% TeX-master: "paper"
%%% End:
