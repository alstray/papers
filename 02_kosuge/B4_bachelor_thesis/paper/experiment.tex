%%%%%%%%%%%%%%%%%%%%%%%%%%%%%%%%%%%%%%%%%%%%%%%%%%%%%%%%%% 
\chapter{実行実験}
%%%%%%%%%%%%%%%%%%%%%%%%%%%%%%%%%%%%%%%%%%%%%%%%%%%%%%%%%% 

導入した高速化手法が有効であるかを評価するために,先に示したクイーングラフ彩色問題を用いて実験をし,その結果について比較評価を行う.

実験は,SMTソルバーのz3(ver. 4.8.9)を用いてクイーングラフ彩色問題の大きさN=5〜13について,1問あたりの制限時間を2時間として実験し,各実行CPU時間を計測した.

実験環境は,Mac mini,3.2GHz,64GBメモリである.

実行実験の結果をまとめたものを表\ref{tb:result}に表す.

上から1行目が問題の大きさ$N$と,その問題に解があるかを示している.
(satが解が存在していることを表し,unsatが解が存在しないことを表している)

左から1列目が用いたモデル化方法と高速化手法を示している.

COLとPOSがそれぞれ色変数モデルと位置変数モデルを用いることを表している.

H1とH2とH3がそれぞれ$distinct$制約に対するヒント制約1とヒント制約2とヒント制約1と2を合わせて使うことを表している.

PB1とPB2とPB3がそれぞれ$distinct$制約のPB符号化の符号化方法1と符号化方法2と符号化方法3を用いることを表している.

Cはチャネリング制約を用いることを表している.

表中の赤字は同じ大きさの問題の中で最も速く解けたことを示している.
(N=5,6については差が無かったため省略している)

表中の青文字については間違った解を導き出したことを表している.
\begin{table}[htb]
    \caption{実験結果}
{\tiny \begin{tabular}{l|rrr} 
  & 到達可能 & 到達不能 & 合計 \\ \hline
  origin & 11 & 0 & 11 \\
  changed & 11 & 10 & 21 \\
  unchanged & 11 & 10 & 21 \\
\end{tabular}\label{tb:result}}
\end{table}

実験結果より,単に$distinct$制約のみを用いたものはSMTでは色変数モデルよりも位置変数モデルの方が性能が良いということがわかる.
また,N=11まで解けていることから加えた\ref{sec:hint}のヒント制約はSMTにおいても有効性が確認できた.

\ref{}に示したPB符号化を行う手法は,符号化方法1,2は有効性を確認できたが,符号化方法3は有効性は確認できなかった.
また,実行した際の出力からこの方法ではSMTソルバーはSATソルバーのみを使用していることがわかった.
このことからこの手法は理論ソルバーを扱えるという利点を活かせていないと言える.

\ref{}に示したチャネリング制約を用いた手法は,
COL+PB2+C, COL+PB3+Cでは単に$distinct$制約を用いた場合よりも性能が低下しており,
COL+PB1+Cでは,N=11において間違った解を出力している.このことからz3ソルバーにおいてチャネリング制約を用いることは不適であると言える.


%%% Local Variables:
%%% mode: latex
%%% TeX-master: "paper"
%%% End:
