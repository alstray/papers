%%%%%%%%%%%%%%%%%%%%%%%%%%%%%%%%%%%%%%%%%%%%%%%%%%%%%%%%%%
\chapter{結論}
%%%%%%%%%%%%%%%%%%%%%%%%%%%%%%%%%%%%%%%%%%%%%%%%%%%%%%%%%%
% 本論文では,SMTの$distinct$制約について新たな手法を提案し,その応用としてクイーングラフ彩色問題の解を求めた.
% 提案した手法は3つあり,$distinct$制約にヒント制約を追加する手法とPB符号化を行う手法と,PB符号化にチャネリング制約を用いる手法である.
% ヒント制約を追加する手法は,鳩の巣原理を用いた制約とドメインサイズと要素数が等しい場合の制約の2つあり,そのどちらも早期に枝刈りを行うことで求解速度をあげることができる.
% PB符号化を行う手法は,基本的なものとヒントを加えたものの3種類を用意した.
% 実行実験では,クイーングラフ彩色問題のN=11をヒント制約を追加する手法を用いたものが解くことができ,単に$distinct$制約を用いたものよりも高速に解くことに成功した.
% これにより,SATソルバーにおいて有効な手法がSMTソルバーにおいても有効であることを確認した.
% また,z3ソルバーにおいて,チャネリング制約を用いると性能低下の原因になりうることを発見した.
% 今後の課題は,新たな$distinct$制約の符号化の提案とチャネリング制約を用いた際の性能低下の原因の解消である.

本論文では,SMTの$distinct$制約に高速化手法を用い,その評価を行うためにクイーングラフ彩色問題の解を求めた.
実装した高速化手法は3つあり,鳩の巣原理を用いたヒント制約の追加とat-most-one制約を用いたヒント制約の追加と$distinct$制約のPB符号化の改良手法である.
これらの手法は探索空間の枝刈りを行うことで求解速度を向上させる.
また,チャネリング制約を用いることで色変数モデルと位置変数モデルにおいても$distinct$制約のPB符号化を用いることができるハイブリッドモデルを実装した.
実行実験では,クイーングラフ彩色問題のN=11をヒント制約を追加する手法を用いたものが解くことができ,単に$distinct$制約を用いたものよりも高速に解くことに成功した.
これにより,SATソルバーにおいて有効な手法がSMTソルバーにおいても有効であることを確認した.
また,z3ソルバーにおいて,チャネリング制約を用いると性能低下の原因になりうることを発見した.
今後の課題は,新たな$distinct$制約の符号化の提案とチャネリング制約を用いた際の性能低下の原因の解消である.


%%% Local Variables:
%%% mode: latex
%%% TeX-master: "paper"
%%% End:
