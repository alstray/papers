%%%%%%%%%%%%%%%%%%%%%%%%%%%%%%%%%%%%%%%%%%%%%%%%%%%%%%%%%% 
\chapter*{概要}
\pagenumbering{roman}
%%%%%%%%%%%%%%%%%%%%%%%%%%%%%%%%%%%%%%%%%%%%%%%%%%%%%%%%%% 

% 表紙を情報工学コース用のスタイルにするために,作成した{\tt jbachelor.sty}が
% 必要である.
% TexStudioなどの便利なTex統合環境を利用するために,lualtexを使うとよい.
% platex と lualatex を切り替えるためには,このファイルの先頭を編集してlualatex用
% のltjbook.clsを使うようにする.
% \begin{verbatim}
% %%% for platex
% \documentclass[a4paper,12pt]{jbook}
% %%% for lualatex
% %\documentclass[a4paper,12pt]{ltjbook}
% \end{verbatim}

命題論理の充足可能性判定問題(Boolean SATisfiability; SAT)は,与えられ
た命題論理式の充足可能性を判定する問題である.
近年,SAT の解を求める SAT ソルバーの性能が大きく向上し,
SAT を拡張・発展させた問題を中心に,SAT 技術が大きな広がりを見せている.

なかでも,
背景理論付き SAT (Satisfiability Modulo Theories; SMT) は,
背景理論が扱えるよ
うに,SAT を拡張・発展させた技術である.
SMT は,背景理論をより表現能力の高い述語論理で記述できるため,
問題を簡潔に記述することができる点が特長である.
近年,SAT ソルバー
を拡張した高速 SMT ソルバーが開発され,制約充足問題,プログラム検証な
どへの応用が活発に研究されている.

制約プログラミングの言語である制約充足問題は,与えられた制約を満たす解
を探索する問題である.
制約充足問題では,グローバル制約を用いて,複数の変数に対する複雑な
制約を簡潔に表現できる点が特長の一つである.
代表的なグローバル制約$distinct(x_{1},x_{2},\ldots x_{n})$は,
$x_{i}$が互いに異なる値をとることを表す.
この$distinct$制約は,記述性の向上を目的として SMT ソルバー
にも取り入れられている.
しかしながら,制約充足問題に対する SMT ソルバーの求解性能は,
SAT 型制約ソルバーと比べて劣っているとの報告もあり,
$distinct$制約を含めその効率的な実装は重要な研究課題となっている.

本論文では,D.~E.~Knuth の教科書
The Art of Computer Programming
でも取り上げられているクイーングラフ彩
色問題を題材とし,その SMT 符号化と$distinct$制約の高速化について述べる.
クイーングラフ彩色問題の SMT 符号化として,
色変数モデル,位置変数モデル,0-1変数モデル,ハイブリッドモデルを実装した.
また,$distinct$制約の高速化として,
色変数モデルと位置変数モデルでは,
SAT型制約ソルバーで有効性が示されている鳩の巣原理等に基づくヒント制約を追加した.
0-1変数モデルとハイブリッドモデルでは,
$distinct$制約のPB符号化[大野,2019]を応用し,探索空間の枝刈りを実装した.

クイーングラフ彩色問題($5\leq N \leq 13$)を用いた実行実験の結果,
色変数モデル+ヒント制約と位置変数モデル+ヒント制約が,$N=11$まで解を求め,最も良い性能を示した.
これにより,$distinct$制約の高速化について,
SATでのヒント制約が SMT においても有効であることが確認できた.
その一方で,チャネリング制約を用いたハイブリッドモデルは,
SMTソルバーの場合,求解性能が低下することが確認された.

%%% Local Variables:
%%% mode: japanese-latex
%%% TeX-master: "paper"
%%% End:
