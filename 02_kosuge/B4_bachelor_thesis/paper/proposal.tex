%%%%%%%%%%%%%%%%%%%%%%%%%%%%%%%%%%%%%%%%%%%%%%%%%%%%%%%%%%
\chapter{SMTソルバーにおける$distinct$制約の高速化}
%%%%%%%%%%%%%%%%%%%%%%%%%%%%%%%%%%%%%%%%%%%%%%%%%%%%%%%%%%
本章ではSMTソルバーにおける$distinct$制約の高速化のために用いた手法についてそれぞれ述べる.
まず最初に,クイーングラフ彩色問題のSMT符号化について示し,次に高速化手法についてクイーングラフ彩色問題においてどのように表されるかを交えて説明を行う.

\section{クイーングラフ彩色問題のSMT符号化}
ここではN=5の場合のクイーングラフ彩色問題について使用したSMT符号化について説明する.
本研究ではチェス盤上の一番上の行を0行目,一番左の列を0列目とし,クイーンの色は整数として0から数えるものとする.

\subsection{色変数モデルを使ったSMT符号化}
まず,整数変数c\_0\_0は以下のように宣言される.
c\_i\_jはi行j列目に塗られる色を示している.
\begin{verbatim}
    (declare-const c_0_0 Int)
\end{verbatim}
また,c\_0\_0は0以上4以下であるので以下の制約を追加する.
\begin{verbatim}
    (assert (and (>= c_0_0 0) (<= c_0_0 4)))
\end{verbatim}
他の変数についても同様に宣言される.

次に,0行目に配置されるクイーンの色が互いに異なるという制約は以下のように宣言される.
\begin{verbatim}
    (assert (distinct c_0_0 c_0_1 c_0_2 c_0_3 c_0_4)))
\end{verbatim}

作成したN=5の場合の色変数モデルのコードをコード\ref{code:qgcp_5_col}に示す.
3から52行目が各整数変数の宣言とそのドメインを指定しており,
53から57行目が各行についての制約であり,
58から62行目が各列について,
63から69行目が各右上がり対角線について,
70から76行目が角右下がり対角線についてである.
また,77から81行目は対称性除去のために0行目の値を指定している.

\lstinputlisting[float=htbp,caption={%
N=5の色変数モデルのクイーングラフ彩色問題},%
captionpos=b,frame=single,label=code:qgcp_5_col,%
numbers=left,%
breaklines=true,%
columns=fullflexible,keepspaces=true,%
basicstyle=\ttfamily\tiny]{code/qgcp_5_m0__z3_489.smt2}


\subsection{位置変数モデルを使ったSMT符号化}
作成したN=5の場合の位置変数モデルのコードを\ref{code:qgcp_5_pos}に示す.
y\_i\_kはi行目のk色のクイーンが配置される場所を示している.
色変数モデルの時と同様にして,
3から52行目が各整数変数の宣言とそのドメインを指定しており,
53から72行目が各行,各列,各右上がり対角線,各右下がり対角線についてそれぞれ互いに異なるという制約を追加している.
また,73から77行目は対称性除去のために0行目に配置されるクイーンの列を指定して宣言している.

\lstinputlisting[float=htbp,caption={%
N=5の位置変数モデルのクイーングラフ彩色問題},%
captionpos=b,frame=single,label=code:qgcp_5_pos,%
numbers=left,%
breaklines=true,%
columns=fullflexible,keepspaces=true,%
basicstyle=\ttfamily\tiny]{code/qgcp_5_m7__z3_489.smt2}


%%%%%%%%%%%%%%%%%%%%%%%%%%%%%%%%%%%%%%%%%%%%%%%%%%%%
%
% ヒント制約の追加
%
%%%%%%%%%%%%%%%%%%%%%%%%%%%%%%%%%%%%%%%%%%%%%%%%%%%%
\section{$distinct$制約に対するヒント制約の追加}\label{sec:hint}
高速化のための手法の1つ目は$distinct$制約にヒント制約を追加することである.
追加したヒント制約は2つあり,1つは鳩の巣原理を用いたヒント制約で,もう一つはドメインサイズと要素数が等しい場合に追加するヒント制約である.

\subsection{鳩の巣原理のヒント}
1つ目の鳩の巣原理を用いたヒント制約は,参考文献\cite{}から引用したものであり,追加した制約は以下の通りである,

$distinct(x_1 ... x_n)$について,$x_i \in \{l, l+1, ..., u\}$であるとき
\begin{eqnarray}
&& \bigvee_{i=1}^n   x_i \geq l+n-1\\
\label{eq:h1_1}
&& \bigvee_{i=1}^n \lnot(x_i \geq u-n+1)
\label{eq:h1_2}
\end{eqnarray}
である.

(\ref{eq:h1_1})は,全ての$x_i$が$l+n-2$以下になることを禁止しており,
(\ref{eq:h1_2})は,全ての$x_i$が$u-n+1$以上になることを禁止している.

この制約についてSMT符号化を行うと.
例として,N=5での時に
\verb|(distinct c_0_3 c_1_2 c_2_1 c_3_0)|は以下のように宣言される.

\begin{verbatim}
(assert (distinct c_0_3 c_1_2 c_2_1 c_3_0))
(assert (or (>= c_0_3 3) (>= c_1_2 3) (>= c_2_1 3) (>= c_3_0 3)))
(assert (or (<= c_0_3 1) (<= c_1_2 1) (<= c_2_1 1) (<= c_3_0 1)))
\end{verbatim}

この制約を追加する利点としては,
上記の例においては,c\_0\_3,c\_1\_2,c\_2\_1がそれぞれ0,1,2の値を取るとしたときに,c\_3\_0の取り得る範囲が3以上4以下に制限される.
このように,制限を加わえることができるので,探索範囲が狭まり求解速度を上げることができる.


\subsection{ドメインサイズと要素数が等しい場合のヒント}
2つ目のヒント制約は,$distinct$制約の要素のドメインのサイズと要素数が等しい場合,つまり
$distinct(x_1 ... x_n)$について,$x_i \in \{l, l+1, ..., u\}$かつ$u-l=n-1$であるときに以下の制約を追加するものである.\\
\begin{eqnarray}
\bigvee_{i=1}^n x_i=a \; (a \in \{l, l+1, ..., u\})
\label{eq:h2}
\end{eqnarray}
この制約はドメインサイズと要素数が等しい場合には,値$a$に対してその値をとる変数$x_i$が存在するというヒントを追加している.

この制約についてSMT符号化を行うと.
例として,N=5の時に
\verb|(distinct c_0_0 c_0_1 c_0_2 c_0_3 c_0_4)|は以下のように宣言される.

\begin{verbatim}
(assert (distinct c_0_0 c_0_1 c_0_2 c_0_3 c_0_4))
(assert (or (= c_0_0 0) (= c_0_1 0) (= c_0_2 0) (= c_0_3 0) (= c_0_4 0)))
(assert (or (= c_0_0 1) (= c_0_1 1) (= c_0_2 1) (= c_0_3 1) (= c_0_4 1)))
(assert (or (= c_0_0 2) (= c_0_1 2) (= c_0_2 2) (= c_0_3 2) (= c_0_4 2)))
(assert (or (= c_0_0 3) (= c_0_1 3) (= c_0_2 3) (= c_0_3 3) (= c_0_4 3)))
(assert (or (= c_0_0 4) (= c_0_1 4) (= c_0_2 4) (= c_0_3 4) (= c_0_4 4)))
\end{verbatim}


この制約を追加する利点としては,
上記の例においては,探索を進めていくにつれてc\_0\_0,c\_0\_1,c\_0\_2,c\_0\_3,c\_0\_4が0にならないと分かった時点でその解が間違っているとわかるので枝刈りを行うことができ,
求解速度を上げることができる.



%%%%%%%%%%%%%%%%%%%%%%%%%%%%%%%%%%%%%%%%%%%%%%%%%%%%
%
% PB符号化
%
%%%%%%%%%%%%%%%%%%%%%%%%%%%%%%%%%%%%%%%%%%%%%%%%%%%%
\section{$distinct$制約に対するPB符号化} \label{sec:pb}
高速化のための2つ目の手法は$distinct$制約をPB符号化して解くというものである.
この手法では,求める解を整数変数としてではなく0-1変数として求める.

例としては,整数変数$x \in \{1,2,3\}$は0-1変数$x_1,x_2,x_3 \in {0,1}$として,$x_i=1$なら$x=i$というようにして解を求める.

\subsection{0-1変数のみを用いたクイーングラフ彩色問題のSMT符号化} \label{sec:pb4smt}
N=5の場合について本研究で使用したSMT符号化を説明する
まず,0-1変数c\_0\_0\_0は以下のように宣言される.
c\_i\_j\_kはi行j列目がk色に塗られるかどうかを示している.
\begin{verbatim}
    (declare-const c_0_0_0 Int)
\end{verbatim}
また,c\_0\_0\_0は0-1変数であるので以下の制約を追加する.
\begin{verbatim}
    (assert (and (>= c_0_0_0 0) (<= c_0_0_0 1)))
\end{verbatim}
他の変数についても同様に宣言される.
盤面上の一つのマスに塗られる色は一色だけなので,0行0列目が一つの色で塗られる制約は以下のようになる.
\begin{verbatim}
(assert (= (+ c_0_0_0 c_0_0_1 c_0_0_2 c_0_0_3 c_0_0_4) 1))
\end{verbatim}
他のマスについても同様に宣言される.

次に,0行目に配置されるクイーンの色が互いに異なるという制約は以下のように宣言される.
\begin{verbatim}
(assert (= (+ c_0_0_0 c_0_1_0 c_0_2_0 c_0_3_0 c_0_4_0) 1))
(assert (= (+ c_0_0_1 c_0_1_1 c_0_2_1 c_0_3_1 c_0_4_1) 1))
(assert (= (+ c_0_0_2 c_0_1_2 c_0_2_2 c_0_3_2 c_0_4_2) 1))
(assert (= (+ c_0_0_3 c_0_1_3 c_0_2_3 c_0_3_3 c_0_4_3) 1))
(assert (= (+ c_0_0_4 c_0_1_4 c_0_2_4 c_0_3_4 c_0_4_4) 1))
\end{verbatim}

作成したN=5の場合の色変数モデルのコードをコード\ref{code:qgcp_5_pb}に示す.
3から277行目が各整数変数の宣言とそのドメインと色の制約を追加しており,
278から425行目が各行,各列,各右上がり対角線,各右下がり対角線についてそれぞれ互いに異なるという制約を追加している.
また,426から430行目は対称性除去のために0行目の値を指定している.


\twocolumn
\lstinputlisting[float=htbp,caption={%
N=5の0-1変数のみを用いた色変数モデルのクイーングラフ彩色問題},%
captionpos=b,frame=single,label=code:qgcp_5_pb,%
numbers=left,%
classoffset=1,%
breaklines=true,%
columns=fullflexible,keepspaces=true,%
basicstyle=\ttfamily\tiny]{code/qgcp_5_m18__z3_489.smt2}
\onecolumn

本研究では,$distinct$制約を3つ方法で符号化した.
これらは,参考文献\cite{}から引用したものであり,
% 1つ目はブール基数制約符号化のみを使用したもので\ref{sec:pb4smt}に示したものである,2つ目と3つ目はブール基数制約符号化にヒント制約を追加したものである.
1つ目はat-most-one制約を用いてdistinct制約を表したもので,2つ目はと3つ目は新たに変数を導入することでヒントを追加し,distinct制約を表すものである.


\subsection{符号化方法1}\label{sec:pb_1}
1つ目の符号化は$distinct$制約を以下のように表すものである.

$distinct(x_1 ... x_n)$について,$x_i$は1以上$d$以下の値で$n \leq d$とし,
$x_{ij}$は$x_{ij} = 1 \leftrightarrow x_i = j$であるとすると,
\begin{eqnarray}
    x_{i1} + ... + x_{id}=1 \; (i=1 ... n) \label{eq:pb_1_1}\\
    x_{1j} + ... + x_{nj}\leq1 \; (j=1 ... d) \label{eq:pb_1_2}
\end{eqnarray}
(\ref{eq:pb_1_1})は$x_i$がちょうど1つの値しか取らないことを表し,
(\ref{eq:pb_1_2})は各値$j$が割り当て割れるのが高々1つの変数であることを表している.

また,$n=d$の時,(\ref{eq:pb_1_2})は以下の制約に置き換えられる.
\begin{eqnarray}
    x_{1j} + ... + x_{nj}=1 \; (j=1 ... d) \label{eq:pb_1_3}
\end{eqnarray}
これは各値jがちょうど1つの変数に割り当てられることを表している.

これらの制約についてSMT符号化を行うと,
例として,N=5の時に\verb|(distinct c_0_0 c_0_1 c_0_2 c_0_3 c_0_4)|は以下のように宣言される.

\begin{verbatim}
(assert (= (+ c_0_0_0 c_0_0_1 c_0_0_2 c_0_0_3 c_0_0_4) 1))
(assert (= (+ c_0_1_0 c_0_1_1 c_0_1_2 c_0_1_3 c_0_1_4) 1))
(assert (= (+ c_0_2_0 c_0_2_1 c_0_2_2 c_0_2_3 c_0_2_4) 1))
(assert (= (+ c_0_3_0 c_0_3_1 c_0_3_2 c_0_3_3 c_0_3_4) 1))
(assert (= (+ c_0_4_0 c_0_4_1 c_0_4_2 c_0_4_3 c_0_4_4) 1))
(assert (= (+ c_1_0_0 c_1_0_1 c_1_0_2 c_1_0_3 c_1_0_4) 1))
(assert (= (+ c_0_0_0 c_0_1_0 c_0_2_0 c_0_3_0 c_0_4_0) 1))
(assert (= (+ c_0_0_1 c_0_1_1 c_0_2_1 c_0_3_1 c_0_4_1) 1))
(assert (= (+ c_0_0_2 c_0_1_2 c_0_2_2 c_0_3_2 c_0_4_2) 1))
(assert (= (+ c_0_0_3 c_0_1_3 c_0_2_3 c_0_3_3 c_0_4_3) 1))
(assert (= (+ c_0_0_4 c_0_1_4 c_0_2_4 c_0_3_4 c_0_4_4) 1))
\end{verbatim}


\subsection{符号化方法2}
2つ目の符号化は$distinct$制約を以下のように表すものである.

$distinct(x_1 ... x_n)$について,$n < d$の時,
各値$j$ごとの$x_{ij}$の和を表す変数$y_j$を導入し,以下のように表す.
\begin{eqnarray}
    x_{i1} + ... + x_{id}=1 \; (i=1 ... n) \label{eq:pb_2_1}\\
    x_{1j} + ... + x_{nj}\leq1 \; (j=1 ... d) \label{eq:pb_2_2}\\
    y_j = x_{1j} + ... + x_{nj} \; (j=1 ... d) \label{eq:pb_2_3}\\
    y_1 + ... + y_d =n \label{eq:pb_2_4}
\end{eqnarray}

(\ref{eq:pb_2_3})(\ref{eq:pb_2_4})についてSMT符号化を行うと,
例として,N=5での時に\verb|(distinct c_0_1 c_1_0)|は以下のように宣言される.

\begin{verbatim}
(declare-const _T1_0 Int)
(assert (and (>= _T1_0 0) (<= _T1_0 1)))
(declare-const _T1_1 Int)
(assert (and (>= _T1_1 0) (<= _T1_1 1)))
(declare-const _T1_2 Int)
(assert (and (>= _T1_2 0) (<= _T1_2 1)))
(declare-const _T1_3 Int)
(assert (and (>= _T1_3 0) (<= _T1_3 1)))
(declare-const _T1_4 Int)
(assert (and (>= _T1_4 0) (<= _T1_4 1)))
(assert (= _T1_0 (+ c_0_1_0 c_1_0_0)))
(assert (= _T1_1 (+ c_0_1_1 c_1_0_1)))
(assert (= _T1_2 (+ c_0_1_2 c_1_0_2)))
(assert (= _T1_3 (+ c_0_1_3 c_1_0_3)))
(assert (= _T1_4 (+ c_0_1_4 c_1_0_4)))
(assert (= (+ _T1_0 _T1_1 _T1_2 _T1_3 _T1_4) 2))
\end{verbatim}

この制約を追加する利点としては,
上記の例においては,探索を進めていくにつれて
c\_0\_1,c\_1\_0が1,2,3,4にならないと分かった時点では,\ref{sec:pb_1}の制約には違反しないが,
\verb|_T1_0 + _T1_1 + _T1_2 + _T1_3 + _T1_4 = 2|が成り立たないことがわかるため
枝刈りを行うことができ,求解速度を上げることができる.

\subsection{符号化方法3}
3つ目の符号化は$distinct$制約を以下のように表すものである.

$distinct(x_1 ... x_n)$について,$n < d$の時,
各値$j$ごとに新たな変数$x_{(n+1)j}$を導入し,以下のように表す.
\begin{eqnarray}
    x_{i1} + ... + x_{id}=1 \; (i=1 ... n) \label{eq:pb_3_1}\\
    x_{(n+1)1} + ... + x_{(n+1)d}=d-n \label{eq:pb_3_2}\\
    x_{1j} + ... + x_{(n+1)j}=1 \; (j=1 ... d) \label{eq:pb_3_3}\\
\end{eqnarray}

(\ref{eq:pb_3_2})(\ref{eq:pb_3_3})についてSMT符号化を行うと,
例として,N=5での時に\verb|(distinct c_0_1 c_1_0)|は以下のように宣言される.

\begin{verbatim}
(declare-const _T1_0 Int)
(assert (and (>= _T1_0 0) (<= _T1_0 1)))
(declare-const _T1_1 Int)
(assert (and (>= _T1_1 0) (<= _T1_1 1)))
(declare-const _T1_2 Int)
(assert (and (>= _T1_2 0) (<= _T1_2 1)))
(declare-const _T1_3 Int)
(assert (and (>= _T1_3 0) (<= _T1_3 1)))
(declare-const _T1_4 Int)
(assert (and (>= _T1_4 0) (<= _T1_4 1)))
(assert (= (+ _T1_0 _T1_1 _T1_2 _T1_3 _T1_4) 3))
(assert (= (+ c_0_1_0 c_1_0_0 _T1_0) 1))
(assert (= (+ c_0_1_1 c_1_0_1 _T1_1) 1))
(assert (= (+ c_0_1_2 c_1_0_2 _T1_2) 1))
(assert (= (+ c_0_1_3 c_1_0_3 _T1_3) 1))
(assert (= (+ c_0_1_4 c_1_0_4 _T1_4) 1))
\end{verbatim}

この制約を追加する利点としては,
上記の例においては,探索を進めていくにつれて
c\_0\_1,c\_1\_0が1,2,3,4にならないと分かった時点では,\ref{sec:pb_1}の制約には違反しないが,
\verb|_T1_0 + _T1_1 + _T1_2 + _T1_3 + _T1_4 = 3|が成り立たないことがわかるため
枝刈りを行うことができ,求解速度を上げることができる.



%%%%%%%%%%%%%%%%%%%%%%%%%%%%%%%%%%%%%%%%%%%%%%%%%%%%
%
% チャネリング
%
%%%%%%%%%%%%%%%%%%%%%%%%%%%%%%%%%%%%%%%%%%%%%%%%%%%%
\section{チャネリング制約を用いた$distinct$制約のPB符号化}
高速化のための3つ目の手法は,チャネリング制約を用いて$distinct$制約をPB符号化して解くというものである.
この手法では求める解は整数変数として求めるが,$distinct$制約を解く場合にチャネリング制約を追加し求める整数変数を0-1変数に変換してからPB符号化で解くというものである.
N=5の時,c\_0\_0について追加したチャネリング制約は以下のものである.
\begin{verbatim}
(assert (=> (= c_0_0_0 1) (= c_0_0 0)))
(assert (=> (= c_0_0 0) (= c_0_0_0 1)))
(assert (=> (= c_0_0_1 1) (= c_0_0 1)))
(assert (=> (= c_0_0 1) (= c_0_0_1 1)))
(assert (=> (= c_0_0_2 1) (= c_0_0 2)))
(assert (=> (= c_0_0 2) (= c_0_0_2 1)))
(assert (=> (= c_0_0_3 1) (= c_0_0 3)))
(assert (=> (= c_0_0 3) (= c_0_0_3 1)))
(assert (=> (= c_0_0_4 1) (= c_0_0 4)))
(assert (=> (= c_0_0 4) (= c_0_0_4 1)))
\end{verbatim}



% この手法の利点としては上記\ref{sec:pb}の手法では$distinct$制約の要素に式が存在する場合,新たに変数を導入して

%%% Local Variables:
%%% mode: latex
%%% TeX-master: "paper"
%%% End:
