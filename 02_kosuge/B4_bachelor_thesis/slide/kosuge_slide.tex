\documentclass [dvipdfmx,12pt]{beamer}
\usepackage{bxdpx-beamer}
\usepackage{pxjahyper}
\usepackage{amsmath}
\usepackage{minijs}
\usepackage{tikz}
\usepackage{otf}
\renewcommand{\kanjifamilydefault}{\gtdefault}
\usetheme{Copenhagen}
\usetikzlibrary{intersections, calc, arrows}
\setbeamertemplate{navigation symbols}{}
\setbeamertemplate{itemize item}[circle]
\setbeamersize{text margin left=1.5em,text margin right=1.5em}
\setlength{\abovedisplayskip}{0pt} % 上部のマージン
\setlength{\belowdisplayskip}{0pt} % 下部のマージン
%
%
%
% footer setting %
\makeatother
\setbeamertemplate{footline}
{
    \leavevmode%
    \hbox{%
        \begin{beamercolorbox}[wd=.4\paperwidth,ht=2.25ex,dp=1ex,center]{author in head/foot}%
            \usebeamerfont{author in head/foot}\insertshortauthor
        \end{beamercolorbox}%
        \begin{beamercolorbox}[wd=.6\paperwidth,ht=2.25ex,dp=1ex,center]{title in head/foot}%
            \usebeamerfont{title in head/foot}\hspace*{1ex} \insertshorttitle\hspace*{3em}
            \textbf{ \insertframenumber{} / \inserttotalframenumber } \hspace*{1ex}
    \end{beamercolorbox}}%
    \vskip0pt%
}
\makeatletter
% exclude apprendix slides from framenumber %
\newcommand{\backupbegin}{
    \newcounter{framenumberappendix}
    \setcounter{framenumberappendix}{\value{framenumber}}
}
\newcommand{\backupend}{
    \addtocounter{framenumberappendix}{-\value{framenumber}}
    \addtocounter{framenumber}{\value{framenumberappendix}}
}


%%%%%%%%%%%% my macro %%%%%%%%%%%%%%%%%
\newcommand{\distinct}{$distinct$}
%%%%%%%%%%%%%%%%%%%%%%%%%%%%%%%%%%%%%%%

%%%%%%%%%%%%%%%%%%%%%%%%%%%%%%%%%%%%%%
% タイトル
%%%%%%%%%%%%%%%%%%%%%%%%%%%%%%%%%%%%%%
\title[]{SMTソルバーにおける \distinct 制約の高速化とクイーングラフ彩色問題への応用}
\author{小菅脩司}
\institute{番原研究室}
\date{卒業研究発表会\\2021年2月19日}
\begin{document}
\begin{frame} {}
    \titlepage
\end{frame}
%%%%%%%%%%%%%%%%%%%%%%%%%%%%%%%%%%%%%%



%%%%%%%%%%%%%%%%%%%%%%%%%%%%%%%%%%%%%%
% 背景理論付き(SATSatisfiability Modulo Theories; SMT)
%%%%%%%%%%%%%%%%%%%%%%%%%%%%%%%%%%%%%%
\begin{frame}
    \frametitle{背景理論付き(SATSatisfiability Modulo Theories; SMT)}
    \begin{itemize}
        \item SMTは等号や算術,配列やリスト,ビットベクターなど様々な背景 理論が扱えるように,SAT を拡張・発展させた技術である.
        \item 近年,SMTソルバーは制約充足問題,プログラム検証などへの応用が活発に研究されている.
    \end{itemize}
    \begin{alertblock}{SMTソルバーの特長}
        背景理論をより表現能力 の高い述語論理で記述できるため,問題を簡潔に記述することができる
    \end{alertblock}
\end{frame}



%%%%%%%%%%%%%%%%%%%%%%%%%%%%%%%%%%%%%%
% \distinct 制約
%%%%%%%%%%%%%%%%%%%%%%%%%%%%%%%%%%%%%%
\begin{frame}
    \frametitle{\distinct 制約}
    \begin{itemize}
        \item $distinct(x_1,x_2,...x_n)$は,$x_i$ が互いに異なる値をとることを表す.
        \item 制約充足問題に対する SMT ソルバーの 求解性能は,SAT 型制約ソルバーと比べて劣っているとの報告もあり, 
            distinct 制約を含めその効率的な実装は重要な研究課題となっている
    \end{itemize}
    \begin{block}{\distinct 制約の定式化}
        $distinct(x_1,x_2,...x_n)$は以下のように表される.\\
        $$\bigwedge_{1 \leq i < j \leq n} x_i \neq x_j$$
    \end{block}
\end{frame}

\begin{frame}
    \frametitle{\distinct 制約のPB符号化}
    \distinct 制約はnot-equalで表す他に以下のようにPB符号化して表すことができる.
    \begin{block}{ \distinct 制約のPB符号化}
        $distinct(x_1 ... x_n) \; (x_i \in \{l, l+1, ..., u\}, n-1 \leq u-l)$は
        $x_{ij} \in \{0,1\} \; (x_{ij}=1 \Leftrightarrow x_i=j)$を導入して以下のように表される.
        \begin{eqnarray}
            \sum_{j=l}^{u} x_{ij}=1 \; (i \in \{1,2,\ldots,n\}) \label{eq:pb_1}\\
    \vspace{-3mm}
            \begin{cases}
                \sum_{i=1}^{n} x_{ij} = 1 \; (j \in \{l,l+1,\ldots,u\}) & (n-1=u-l) \\
                \sum_{i=1}^{n} x_{ij} \leq 1 \; (j \in \{l,l+1,\ldots,u\}) & (otherwise)
            \end{cases}\label{eq:pb_2}
        \end{eqnarray}
    \end{block}
\end{frame}


%%%%%%%%%%%%%%%%%%%%%%%%%%%%%%%%%%%%%%
% クイーングラフ彩色問題
%%%%%%%%%%%%%%%%%%%%%%%%%%%%%%%%%%%%%%
\begin{frame}
    \frametitle{クイーングラフ彩色問題}
    \begin{center}
\scalebox{0.8}{
    \begin{tikzpicture}
        \fill[red]    (0.5,0.5) circle (0.25);
        \fill[red]    (1.5,2.5) circle (0.25);
        \fill[red]    (2.5,4.5) circle (0.25);
        \fill[red]    (3.5,1.5) circle (0.25);
        \fill[red]    (4.5,3.5) circle (0.25);
        \fill[blue]   (0.5,3.5) circle (0.25);
        \fill[blue]   (1.5,0.5) circle (0.25);
        \fill[blue]   (2.5,2.5) circle (0.25);
        \fill[blue]   (3.5,4.5) circle (0.25);
        \fill[blue]   (4.5,1.5) circle (0.25);
        \fill[green]  (0.5,1.5) circle (0.25);
        \fill[green]  (1.5,3.5) circle (0.25);
        \fill[green]  (2.5,0.5) circle (0.25);
        \fill[green]  (3.5,2.5) circle (0.25);
        \fill[green]  (4.5,4.5) circle (0.25);
        \fill[black] (0.5,4.5) circle (0.25);
        \fill[black] (1.5,1.5) circle (0.25);
        \fill[black] (2.5,3.5) circle (0.25);
        \fill[black] (3.5,0.5) circle (0.25);
        \fill[black] (4.5,2.5) circle (0.25);
        \fill[pink] (0.5,2.5) circle (0.25);
        \fill[pink] (1.5,4.5) circle (0.25);
        \fill[pink] (2.5,1.5) circle (0.25);
        \fill[pink] (3.5,3.5) circle (0.25);
        \fill[pink] (4.5,0.5) circle (0.25);
        \draw [step=10mm] (0,0) grid (5,5);
        \node at (2.5, 0) [below] {N=5のとき};
    \end{tikzpicture}
}
\end{center}

    \begin{itemize}
        \item N × N の大きさのチェス盤上に N 色のク イーンの駒を各色 N 個ずつ互いに取り合わないように配置する問題である.
        \item 各行,各列,各右上がり対角線,各右下 がり対角線に配置されるクイーンの色が互いに異なるということである
    \end{itemize}
\end{frame}



%%%%%%%%%%%%%%%%%%%%%%%%%%%%%%%%%%%%%%
% 研究概要
%%%%%%%%%%%%%%%%%%%%%%%%%%%%%%%%%%%%%%
\begin{frame}
    \frametitle{研究概要}
    \begin{alertblock}{研究目的}
        distinct制約に対して高速化手法を適用し, クイーングラフ彩色問題を用いて比較評価を行う.
    \end{alertblock}
    \begin{block}{研究内容}
        \begin{enumerate}
            \item \structure{クイーングラフ彩色問題の4種類のSMT符号化を考案}
                \begin{itemize}
                    \item 色変数モデル,位置変数モデル,0-1変数モデル,ハイブリッドモデルをSMT符号化した.
                \end{itemize}
            \item \structure{\distinct 制約の4種類の高速化手法を考案}
                \begin{itemize}
                    \item 考案した手法は,探索空間の枝刈りをするため求解速度の向上が期待できる.
                \end{itemize}
            \item \structure{クイーングラフ彩色問題を用いた実験}
                \begin{itemize}
                    \item N=11までの解を得ることが出来た.
                    \item 適用した高速化手法が求解速度を向上させることを確認した.
                    % \item ハイブリッドモデルは求解性能を低下させることを確認した.
                \end{itemize}
        \end{enumerate}
    \end{block}
\end{frame}



%%%%%%%%%%%%%%%%%%%%%%%%%%%%%%%%%%%%%%
% 考案したSMT符号化
%%%%%%%%%%%%%%%%%%%%%%%%%%%%%%%%%%%%%%
\begin{frame}
    \frametitle{考案したSMT符号化}
    \begin{block}{}
        \begin{itemize}
            \item \structure{ 色変数モデル}\\
                盤面に配置されるクイーンの駒の色を整数変数として解を求める制約モデル
            \item \structure{ 位置変数モデル}\\
                盤面上のある行に対して何列目にクイーンが配置されるのかを整数変数として解を求める制約モデル
            \item \structure{ 0-1変数モデル}\\
                盤面に配置されるクイーンの駒の色を0-1変数として解を求める制約モデル\\
                つまり,色変数モデルの整数変数を 0-1 変数で表したモデル
            \item \structure{ ハイブリッドモデル}\\
                色・位置変数モデルと0-1変数モデルを組み合わせたモデル\\
                チャネリング制約を用いて解を整数変数で求め,\distinct 制約を0-1変数で解く
        \end{itemize}
    \end{block}
\end{frame}



%%%%%%%%%%%%%%%%%%%%%%%%%%%%%%%%%%%%%%
% 高速化手法1
%%%%%%%%%%%%%%%%%%%%%%%%%%%%%%%%%%%%%%
\begin{frame}
    \frametitle{高速化手法}
    \vspace{-3mm}
    \begin{exampleblock}{鳩の巣原理を用いたヒント制約}
        $distinct(x_1 ... x_n)$について, $x_i \in \{l, l+1, ..., u\}$であるときに以下の制約を追加する.
        \vspace{-3mm}
        \begin{eqnarray}
            && \bigvee_{i=1}^n   x_i \geq l+n-1\\
            && \bigvee_{i=1}^n \lnot(x_i \geq u-n+1)
        \end{eqnarray}
    \end{exampleblock}
    \begin{exampleblock}{at-most-one制約を用いたヒント制約}
        $distinct(x_1 ... x_n)$について, $x_i \in \{l, l+1, ..., u\}$かつ$u-l=n-1$であるときに以下の制約を追加する.\\
        \vspace{-3mm}
        $$\bigvee_{i=1}^n x_i=a \; (a \in \{l, l+1, ..., u\})$$
    \end{exampleblock}
\end{frame}



%%%%%%%%%%%%%%%%%%%%%%%%%%%%%%%%%%%%%%
% 高速化手法2
%%%%%%%%%%%%%%%%%%%%%%%%%%%%%%%%%%%%%%

\begin{frame}\footnotesize
    \frametitle{高速化手法}
    PB符号化した\distinct 制約に対して2つの改良手法[大野,2019]を適用した.
    \begin{exampleblock}{改良手法1}
        $n-1 < u-l$の時,各値$j$ごとの$x_{ij}$の和を表す変数$y_{j}$を導入し,以下の制約を追加する.
    \vspace{-3mm}
        \begin{eqnarray}
            \sum_{i=1}^{n} x_{ij}=y_j \; (j \in \{l,l+1, \ldots,u\})\\
    \vspace{-3mm}
            \sum_{j=l}^{u} y_j = n
        \end{eqnarray}
    \end{exampleblock}
    \vspace{-3mm}
    \begin{exampleblock}{改良手法2}
        $n-1 < u-l$の時,各値$j$ごとに新たな変数$x_{(n+1)j}$を導入し,以下の制約を追加する.
    \vspace{-3mm}
        \begin{eqnarray}
            \sum_{i=1}^{n+1} x_{ij}=1 \; (j \in \{l,l+1, \ldots,u\})\\
    \vspace{-3mm}
            \sum_{j=l}^{u} x_{(n+1)j} = u-l-n
        \end{eqnarray}
    \end{exampleblock}
\end{frame}


%%%%%%%%%%%%%%%%%%%%%%%%%%%%%%%%%%%%%%
% 実験概要2
%%%%%%%%%%%%%%%%%%%%%%%%%%%%%%%%%%%%%%
\begin{frame}
    \frametitle{実験概要}
    ベンチマーク問題
    \begin{itemize}
        \item クイーングラフ彩色問題 N=5〜13
    \end{itemize}
    使用ソルバー
    \begin{itemize}
        \item SMTソルバー: \textit{z3}(ver.4.8.9)
    \end{itemize}
    制限時間
    \begin{itemize}
        \item 1問あたり2時間
    \end{itemize}
    実験環境:
    \begin{itemize}
        \item Mac mini,  3.2GHz,  64GB メモリ
    \end{itemize}
\end{frame}



%%%%%%%%%%%%%%%%%%%%%%%%%%%%%%%%%%%%%%
% 実験結果
%%%%%%%%%%%%%%%%%%%%%%%%%%%%%%%%%%%%%%
\begin{frame}
    \frametitle{実験結果}
    計測したCPU時間は以下の通りである.
    \begin{block}{}
        {\tiny \begin{tabular}{l|rr|rr} 
  & \multicolumn{2}{c|}{基本ソルバー} & \multicolumn{2}{c}{改良ソルバー} \\
  & \code{changed} & \code{unchanged} & \code{changed} & \code{unchanged} \\ \hline
  解けた問題数(到達可能) & 11 & 11 & 11 & 11 \\
  解けた問題数(到達不能) & 10 & 10 & 56 & \alert{60} \\\hline
  平均 CPU 時間(秒) & 223.796 & 151.341 & 101.758 & \alert{59.095} \\
\end{tabular} }
    \end{block}
    \begin{itemize}
        \item 高速化手法無しのdistinct制約では, 位置変数モデルの方が性能が良い.
        \item 高速化手法有りがN=11まで解けていることから, 高速化手法が有効であることがわかる.
    \end{itemize}
    \begin{tikzpicture}
 \draw (0,0)--(10,0);
\end{tikzpicture}
\\
    \vspace{-3mm}
    {\tiny COL: 色変数モデル, POS: 位置変数モデル}\\
    \vspace{-3mm}
    {\tiny H1: 鳩の巣原理を用いたヒント制約, H2: at-most-one制約を用いたヒント制約, H3: H1+H2}\\
    \vspace{-3mm}
    {\tiny PB1: 0-1変数モデル, PB2:0-1変数モデル+改良手法1, PB3:0-1変数モデル+改良手法2}\\
\end{frame}



%%%%%%%%%%%%%%%%%%%%%%%%%%%%%%%%%%%%%%
% まとめ
%%%%%%%%%%%%%%%%%%%%%%%%%%%%%%%%%%%%%%
\begin{frame}
    \frametitle{まとめ}
    \begin{enumerate}
        \item \structure{クイーングラフ彩色問題の4種類のSMT符号化を考案}
            \begin{itemize}
                \item 色変数モデル,位置変数モデル,0-1変数モデル,ハイブリッドモデルをSMT符号化した.
            \end{itemize}
        \item \structure{\distinct 制約の4種類の高速化手法を考案}
            \begin{itemize}
                \item 考案した手法は,探索空間の枝刈りをするため求解速度の向上が期待できる.
            \end{itemize}
        \item \structure{クイーングラフ彩色問題を用いた実験}
            \begin{itemize}
                \item N=11までの解を得ることが出来た.
                \item 適用した高速化手法が求解速度を向上させることを確認した.
                    % \item ハイブリッドモデルは求解性能を低下させることを確認した.
            \end{itemize}
    \end{enumerate}
    \begin{alertblock}{今後の課題}
        \begin{itemize}
            \item チャネリング制約を用いた際の性能低下の原因の解消
            \item 新たなdistinct制約の符号化の提案\\
        \end{itemize}
    \end{alertblock}
\end{frame}









%%%%%%%%%%%%%%%%%%%%%%%%%%%%%%%%%%%%%%%%%%%%%%%%%%%%%%%%%%%%%
% %%%% 補助スライド
%%%%%%%%%%%%%%%%%%%%%%%%%%%%%%%%%%%%%%%%%%%%%%%%%%%%%%%%%%%%%

\appendix

\backupbegin



%%%%%%%%%%%%%%%%%%%%%%%%%%%%%%%%%%%%%%
% ~
%%%%%%%%%%%%%%%%%%%%%%%%%%%%%%%%%%%%%%
\begin{frame}
    \frametitle{~}
    \centering
    - 補足用 -
\end{frame}


%%%%%%%%%%%%%%%%%%%%%%%%%%%%%%%%%%%%%%
% 色変数モデル
%%%%%%%%%%%%%%%%%%%%%%%%%%%%%%%%%%%%%%
\begin{frame}
    \frametitle{色変数モデル}
    {\small
        \alert{クイーンの色}を整数変数とした制約モデル\\
        \setlength{\abovedisplayskip}{1pt} % 上部のマージン
        \setlength{\belowdisplayskip}{1pt} % 下部のマージン
        \begin{block}{}
            \begin{itemize}
                \item 位置$(i, j)$に配置されたクイーンの色を整数変数$c_{ij} \in \bf N (i, j \in \bf N)$で表す
                \item \alert{各行}, \alert{各列}, \alert{各右上がり対角線}, \alert{各右下がり対角線}に配置されるクイーンの色がそれぞれ互いに異なることから以下の制約が得られる
            \end{itemize}
            \vspace{-0.5\baselineskip}           %余白
            \begin{eqnarray*}
& distinct\{c_{ij} | j \in \bf N\} \; & (i \in \bf N)\\
& distinct\{c_{ij} | i \in \bf N\} \; & (j \in \bf N)\\
& distinct\{c_{ij} | i, j \in \bf N,  i+j=u\} \; & (u \in \bf U)\\
& distinct\{c_{ij} | i, j \in \bf N,  i-j=d\} \; & (d \in \bf D)
            \end{eqnarray*}
        \end{block}
    }
    \begin{tikzpicture}
 \draw (0,0)--(10,0);
\end{tikzpicture}
\\
    {\footnotesize
        $\bf N$: 行$i$, 列$j$クイーンの色$k$の取り得る値の集合\\
        $\bf U$: $i+j$の取り得る値の集合, 右上がり対角線に対応する.\\
        $\bf D$: $i-j$の取り得る値の集合, 右下がり対角線に対応する.\\
    }
\end{frame}



%%%%%%%%%%%%%%%%%%%%%%%%%%%%%%%%%%%%%%
% 位置変数モデル
%%%%%%%%%%%%%%%%%%%%%%%%%%%%%%%%%%%%%%
\begin{frame}
    \frametitle{位置変数モデル}
    {\small
        各行に配置される\alert{クイーンの列番号}を整数変数とした制約モデル
    }
    {\footnotesize
        \setlength{\abovedisplayskip}{1pt} % 上部のマージン
        \setlength{\belowdisplayskip}{0pt} % 下部のマージン
        \begin{block}{}
            \setlength{\itemsep}{0pt}
            \setlength{\parskip}{0pt}
            \begin{itemize}
                \item 色$k$のクイーンが行$i$で配置されている列番号を整数変数$y_{ik} \in \bf N (i, k \in \bf N)$で表す
                \item 同一の\alert{行番号}に同色のクイーンが配置されないことより
                    $$distinct\{y_{ik} | i \in \bf N\} \; (k \in \bf N)$$
                \item 同一の\alert{列番号}に同色のクイーンが配置されないことより
                    $$distinct\{y_{ik} | k \in \bf N\} \; (i \in \bf N)$$
                \item \alert{右上がりの対角線上}に2つ以上の同色のクイーンが配置されないことより
                    $$distinct\{y_{ik}+i | i \in \bf N\} \; (k \in \bf N)$$
                \item \alert{右下がりの対角線上}に2つ以上の同色のクイーンが配置されないことより
                    $$distinct\{y_{ik}-i | i \in \bf N\} \; (k \in \bf N)$$
            \end{itemize}
        \end{block}
    }
\end{frame}



%%%%%%%%%%%%%%%%%%%%%%%%%%%%%%%%%%%%%%
% 0-1変数モデル
%%%%%%%%%%%%%%%%%%%%%%%%%%%%%%%%%%%%%%
\begin{frame}\small
    \frametitle{0-1変数モデル}
    \alert{k色のクイーンが配置されるどうか}を0-1変数とした制約モデル\\
    \setlength{\abovedisplayskip}{1pt} % 上部のマージン
    \setlength{\belowdisplayskip}{1pt} % 下部のマージン
    \begin{block}{}
        \begin{itemize}
            \item 位置$(i, j)$に$k$色のクイーンが配置されるかを0-1変数$c_{ijk} \in \bf N (i, j, k \in \bf N)$で表す
            \item $c_{ijk}$は色変数モデルの$c_{ij}$を用いて$c_{ijk}=1 \Leftrightarrow c_{ij}=k$と表すことができる.
            \item このモデルは色変数モデルの整数変数を0-1変数で表したモデルである.
            \item 色変数モデルと同様に\alert{各行}, \alert{各列}, \alert{各右上がり対角線}, \alert{各右下がり対角線}に配置されるクイーンの色がそれぞれ互いに異なるという制約を得られる.
        \end{itemize}
    \end{block}
\end{frame}


%%%%%%%%%%%%%%%%%%%%%%%%%%%%%%%%%%%%%%
% ハイブリッドモデル
%%%%%%%%%%%%%%%%%%%%%%%%%%%%%%%%%%%%%%
\begin{frame}\small
    \frametitle{ハイブリッドモデル}
    色・位置変数モデルと0-1変数モデルを組み合わせたモデル\\
    % 整数変数を用いる色変数モデルと位置変数モデルに\distinct 制約のPB符号化とその高速化手法を
    % 実装する
    \setlength{\abovedisplayskip}{1pt} % 上部のマージン
    \setlength{\belowdisplayskip}{1pt} % 下部のマージン
    \begin{block}{}
        \begin{itemize}
            \item \distinct 制約を解く際にチャネリング制約を追加し求める整数変数を0-1変数に変換してから解く
            \item 追加した制約は以下のものである.
                $$ c_{ijk}=1 \Rightarrow c_{ij}=k \; (i,j,k \in \bf N)$$
                $$ c_{ij}=k \Rightarrow c_{ijk}=1 \; (i,j,k \in \bf N)$$
        \end{itemize}
    \end{block}
\end{frame}










\backupend
\end{document}
