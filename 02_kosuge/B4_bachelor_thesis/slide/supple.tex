
%%%%%%%%%%%%%%%%%%%%%%%%%%%%%%%%%%%%%%%%%%%%%%%%%%%%%%%%%%%%%
% %%%% 補助スライド
%%%%%%%%%%%%%%%%%%%%%%%%%%%%%%%%%%%%%%%%%%%%%%%%%%%%%%%%%%%%%

\appendix

\backupbegin



%%%%%%%%%%%%%%%%%%%%%%%%%%%%%%%%%%%%%%
% ~
%%%%%%%%%%%%%%%%%%%%%%%%%%%%%%%%%%%%%%
\begin{frame}
    \frametitle{~}
    \centering
    - 補足用 -
\end{frame}


%%%%%%%%%%%%%%%%%%%%%%%%%%%%%%%%%%%%%%
% distinct制約のPB符号化
%%%%%%%%%%%%%%%%%%%%%%%%%%%%%%%%%%%%%%
\begin{frame}
    \frametitle{\distinct 制約のPB符号化}
    \distinct 制約はnot-equalで表す他に以下のようにPB符号化して表すことができる.
    \begin{block}{ \distinct 制約のPB符号化}
        $distinct(x_1 ... x_n) \; (x_i \in \{l, l+1, ..., u\}, n-1 \leq u-l)$は
        $x_{ij} \in \{0,1\} \; (x_{ij}=1 \Leftrightarrow x_i=j)$を導入して以下のように表される.
        \begin{eqnarray}
            \sum_{j=l}^{u} x_{ij}=1 \; (i \in \{1,2,\ldots,n\}) \label{eq:pb_1}\\
            \vspace{-3mm}
            \begin{cases}
                \sum_{i=1}^{n} x_{ij} = 1 \; (j \in \{l,l+1,\ldots,u\}) & (n-1=u-l) \\
                \sum_{i=1}^{n} x_{ij} \leq 1 \; (j \in \{l,l+1,\ldots,u\}) & (otherwise)
            \end{cases}\label{eq:pb_2}
        \end{eqnarray}
    \end{block}
    (\ref{eq:pb_1})は$x_i$が取り得る値が1つのみであることを表す.\\
    (\ref{eq:pb_2})は2つ以上同じ値にならないことを表す.
\end{frame}

%%%%%%%%%%%%%%%%%%%%%%%%%%%%%%%%%%%%%%
% 色変数モデル
%%%%%%%%%%%%%%%%%%%%%%%%%%%%%%%%%%%%%%
\begin{frame}
    \frametitle{色変数モデル}
    \begin{center}
\scalebox{0.4}{
    \begin{tikzpicture}
        \fill[red]    (0.5,0.5) circle (0.25);
        \fill[red]    (1.5,2.5) circle (0.25);
        \fill[red]    (2.5,4.5) circle (0.25);
        \fill[red]    (3.5,1.5) circle (0.25);
        \fill[red]    (4.5,3.5) circle (0.25);
        \fill[blue]   (0.5,3.5) circle (0.25);
        \fill[blue]   (1.5,0.5) circle (0.25);
        \fill[blue]   (2.5,2.5) circle (0.25);
        \fill[blue]   (3.5,4.5) circle (0.25);
        \fill[blue]   (4.5,1.5) circle (0.25);
        \fill[green]  (0.5,1.5) circle (0.25);
        \fill[green]  (1.5,3.5) circle (0.25);
        \fill[green]  (2.5,0.5) circle (0.25);
        \fill[green]  (3.5,2.5) circle (0.25);
        \fill[green]  (4.5,4.5) circle (0.25);
        \fill[black] (0.5,4.5) circle (0.25);
        \fill[black] (1.5,1.5) circle (0.25);
        \fill[black] (2.5,3.5) circle (0.25);
        \fill[black] (3.5,0.5) circle (0.25);
        \fill[black] (4.5,2.5) circle (0.25);
        \fill[pink] (0.5,2.5) circle (0.25);
        \fill[pink] (1.5,4.5) circle (0.25);
        \fill[pink] (2.5,1.5) circle (0.25);
        \fill[pink] (3.5,3.5) circle (0.25);
        \fill[pink] (4.5,0.5) circle (0.25);
        \draw [step=10mm] (0,0) grid (5,5);
        \node at (2.5, 0) [below] {N=5のとき(COL)};
        \node at ( 0,4.5) [left] {0};
        \node at ( 0,3.5) [left] {1};
        \node at ( 0,2.5) [left] {2};
        \node at ( 0,1.5) [left] {3};
        \node at ( 0,0.5) [left] {4};
        \node at (4.5, 5) [above] {4};
        \node at (3.5, 5) [above] {3};
        \node at (2.5, 5) [above] {2};
        \node at (1.5, 5) [above] {1};
        \node at (0.5, 5) [above] {0};
    \end{tikzpicture}
}
\end{center}

    {\footnotesize
        \alert{クイーンの色}を整数変数とした制約モデル\\
        \setlength{\abovedisplayskip}{1pt} % 上部のマージン
        \setlength{\belowdisplayskip}{1pt} % 下部のマージン
        \begin{block}{}
            \begin{itemize}
                \item 位置$(i, j)$に配置されたクイーンの色を整数変数$c_{ij} \in \bf N (i, j \in \bf N)$で表す
                \item \alert{各行}, \alert{各列}, \alert{各右上がり対角線}, \alert{各右下がり対角線}に配置されるクイーンの色がそれぞれ互いに異なることから以下の制約が得られる
            \end{itemize}
            \vspace{-0.5\baselineskip}           %余白
            \begin{eqnarray*}
& distinct\{c_{ij} | j \in \bf N\} \; & (i \in \bf N)\\
& distinct\{c_{ij} | i \in \bf N\} \; & (j \in \bf N)\\
& distinct\{c_{ij} | i, j \in \bf N,  i+j=u\} \; & (u \in \bf U)\\
& distinct\{c_{ij} | i, j \in \bf N,  i-j=d\} \; & (d \in \bf D)
            \end{eqnarray*}
        \end{block}
    }
    \begin{tikzpicture}
 \draw (0,0)--(10,0);
\end{tikzpicture}
\\
    {\tiny
        $\bf N$: 行$i$, 列$j$クイーンの色$k$の取り得る値の集合\\
        $\bf U$: $i+j$の取り得る値の集合, 右上がり対角線に対応する.\\
        $\bf D$: $i-j$の取り得る値の集合, 右下がり対角線に対応する.\\
    }
\end{frame}






%%%%%%%%%%%%%%%%%%%%%%%%%%%%%%%%%%%%%%
% 0-1変数モデル
%%%%%%%%%%%%%%%%%%%%%%%%%%%%%%%%%%%%%%
\begin{frame}\small
    \frametitle{0-1変数モデル}
    \alert{k色のクイーンが配置されるどうか}を0-1変数とした制約モデル\\
    \setlength{\abovedisplayskip}{1pt} % 上部のマージン
    \setlength{\belowdisplayskip}{1pt} % 下部のマージン
    \begin{block}{}
        \begin{itemize}
            \item 位置$(i, j)$に$k$色のクイーンが配置されるかを0-1変数$c_{ijk} \in \bf N (i, j, k \in \bf N)$で表す
            \item $c_{ijk}$は色変数モデルの$c_{ij}$を用いて$c_{ijk}=1 \Leftrightarrow c_{ij}=k$と表すことができる.
            \item このモデルは色変数モデルの整数変数を0-1変数で表したモデルである.
            \item 色変数モデルと同様に\alert{各行}, \alert{各列}, \alert{各右上がり対角線}, \alert{各右下がり対角線}に配置されるクイーンの色がそれぞれ互いに異なるという制約を得られる.
        \end{itemize}
    \end{block}
\end{frame}


%%%%%%%%%%%%%%%%%%%%%%%%%%%%%%%%%%%%%%
% ハイブリッドモデル
%%%%%%%%%%%%%%%%%%%%%%%%%%%%%%%%%%%%%%
\begin{frame}[fragile]\small
    \frametitle{ハイブリッドモデル}
    色・位置変数モデルと0-1変数モデルを組み合わせたモデル\\
    % 整数変数を用いる色変数モデルと位置変数モデルに\distinct 制約のPB符号化とその高速化手法を
    % 実装する
    \setlength{\abovedisplayskip}{1pt} % 上部のマージン
    \setlength{\belowdisplayskip}{1pt} % 下部のマージン
    \begin{block}{}
        \begin{itemize}
            \item \distinct 制約を解く際にチャネリング制約を追加し求める整数変数を0-1変数に変換してから解く
            \item 追加した制約は以下のものである.
                $$ c_{ijk}=1 \Rightarrow c_{ij}=k \; (i,j,k \in \bf N)$$
                $$ c_{ij}=k \Rightarrow c_{ijk}=1 \; (i,j,k \in \bf N)$$
        \end{itemize}
    \end{block}
    例えば,$i,j,k=0$の時に
    上記の制約はsmt-lib2形式では以下のように宣言できる.
    \begin{exampleblock}{}
\begin{verbatim}
(assert (=> (= c_0_0_0 1) (= c_0_0 0)))
(assert (=> (= c_0_0 0) (= c_0_0_0 1)))
\end{verbatim}
    \end{exampleblock}
\end{frame}

\begin{frame}
    \frametitle{既存のSMTソルバー}
    \begin{itemize}
        \item 既存のSMTソルバーとしては,OpenSMT,YICES,UCLID,CVC3,Z3などが挙げられる.\\
        \item 今回使用したZ3はMicrosoft社によって開発されたもので,プログラム解析,テスト,検証などに使用されている.
    \end{itemize}
\end{frame}


%%%%%%%%%%%%%%%%%%%%%%%%%%%%%%%%%%%%%%
% SMT-LIB
%%%%%%%%%%%%%%%%%%%%%%%%%%%%%%%%%%%%%%
\begin{frame}[fragile]\small
    \frametitle{SMT-LIB}
    \begin{itemize}
        \item SMT-LIBは,SMTの研究開発の促進を目的とした国際的な取り組みである.
        \item SMTソルバーで使用される背景理論の標準的な記述方法を提供している.
        \item SMTソルバーのための共通の入出力言語を開発している.
    \end{itemize}
\end{frame}

%%%%%%%%%%%%%%%%%%%%%%%%%%%%%%%%%%%%%%
% SMT-LIB形式での記述方法
%%%%%%%%%%%%%%%%%%%%%%%%%%%%%%%%%%%%%%
\begin{frame}[fragile]\small
    \frametitle{SMT-LIBでの記述方法}
    SMTソルバーを扱う際に使用した記述方法はsmt-lib2形式である.\\
    変数や制約を宣言する際には以下のように宣言される.
    \begin{exampleblock}{変数の宣言}
        整数変数\verb|x|は以下のように宣言される.
\begin{verbatim}
(declare-const x Int)
\end{verbatim}
    \end{exampleblock}
    \begin{exampleblock}{制約の宣言}
        変数\verb|x|が0以上4以下であることは以下のように宣言される.
\begin{verbatim}
(assert (and (>= x 0) (<= x 4)))
\end{verbatim}
    \end{exampleblock}
    \begin{exampleblock}{\distinct 制約の宣言}
        $distinct(x,y,z)$は以下のように宣言される.
\begin{verbatim}
(assert (distinct x y z))
\end{verbatim}
    \end{exampleblock}
\end{frame}


%%%%%%%%%%%%%%%%%%%%%%%%%%%%%%%%%%%%%%
% 鳩の巣原理
% at-least-one制約
%%%%%%%%%%%%%%%%%%%%%%%%%%%%%%%%%%%%%%
\begin{frame}
    \frametitle{高速化手法}
    \vspace{-3mm}
    \begin{block}{at-least-one制約を用いたヒント制約(H2)}
        $distinct(x_1 ... x_n)$について, $x_i \in \{l, l+1, ..., u\}$かつ$u-l=n-1$であるときに以下の制約を追加する.\\
        \vspace{-3mm}
        $$\bigvee_{i=1}^n x_i=a \; (a \in \{l, l+1, ..., u\})$$
    \end{block}
    \begin{exampleblock}{at-least-one制約を用いたヒント制約の例}
        $distinct(x_1, x_2, x_3, x_4)$について, $x_i \in \{1, 2, 3, 4\}$であるときには以下の制約が追加される.
        \vspace{-3mm}
        \begin{eqnarray*}
            (x_1=1) \lor (x_2=1) \lor (x_3=1) \lor (x_4=1)\\
            (x_1=2) \lor (x_2=2) \lor (x_3=2) \lor (x_4=2)\\
            (x_1=3) \lor (x_2=3) \lor (x_3=3) \lor (x_4=3)\\
            (x_1=4) \lor (x_2=4) \lor (x_3=4) \lor (x_4=4)
        \end{eqnarray*}

    \end{exampleblock}
\end{frame}

%%%%%%%%%%%%%%%%%%%%%%%%%%%%%%%%%%%%%%
% 鳩の巣原理のSMT符号化
%%%%%%%%%%%%%%%%%%%%%%%%%%%%%%%%%%%%%%
\begin{frame}[fragile]\small
    \frametitle{鳩の巣原理のSMT符号化}
    鳩の巣原理を用いた高速化手法についてのSMT符号化を示す.\\
    例として\verb|(distinct x_0 x_1 x_2 x_3)|は以下のように宣言される.\\
    (\verb|x_i|$\in \{0,1,2,3,4\}$とする)
    \begin{exampleblock}{}\scriptsize
\begin{verbatim}
(assert (distinct x_0 x_1 x_2 x_3))
(assert (or (>= x_0 3) (>= x_1 3) (>= x_2 3) (>= x_3 3)))
(assert (or (<= x_0 1) (<= x_1 1) (<= x_2 1) (<= x_3 1)))
\end{verbatim}
    \end{exampleblock}
    2,3行目が高速化のために追加した制約である.\\
    これにより,例えば\verb|x_0,x_1,x_2|にそれぞれ0,1,2が割り当てられた際に\verb|x_3|が3以上に決まるため,探索空間が削減される.
\end{frame}


%%%%%%%%%%%%%%%%%%%%%%%%%%%%%%%%%%%%%%
% at-least-one制約のSMT符号化
%%%%%%%%%%%%%%%%%%%%%%%%%%%%%%%%%%%%%%
\begin{frame}[fragile]\small
    \frametitle{at-least-one制約のSMT符号化}
    at-least-one制約を用いた高速化手法についてのSMT符号化を示す.\\
    例として,\verb|(distinct x_0 x_1 x_2 x_3 x_4)|は以下のように宣言される.\\
    (\verb|x_i|$\in \{0,1,2,3,4\}$とする)
    \begin{exampleblock}{}\scriptsize
\begin{verbatim}
(assert (distinct x_0 x_1 x_2 x_3 x_4))
(assert (or (= x_0 0) (= x_1 0) (= x_2 0) (= x_3 0) (= x_4 0)))
(assert (or (= x_0 1) (= x_1 1) (= x_2 1) (= x_3 1) (= x_4 1)))
(assert (or (= x_0 2) (= x_1 2) (= x_2 2) (= x_3 2) (= x_4 2)))
(assert (or (= x_0 3) (= x_1 3) (= x_2 3) (= x_3 3) (= x_4 3)))
(assert (or (= x_0 4) (= x_1 4) (= x_2 4) (= x_3 4) (= x_4 4)))
\end{verbatim}
    \end{exampleblock}
    2行目以降が高速化のために追加した制約である.\\
    これにより,例えば\verb|x_0,x_1,x_2,x_3|に0以外が割り当てられた際に\verb|x_4|が0になることが確定するので,探索空間が削減される.
\end{frame}

%%%%%%%%%%%%%%%%%%%%%%%%%%%%%%%%%%%%%%
% 高速化手法2
%%%%%%%%%%%%%%%%%%%%%%%%%%%%%%%%%%%%%%

\begin{frame}\footnotesize
    \frametitle{高速化手法}
    PB符号化した\distinct 制約に対して2つの改良手法[大野,2019]を適用した.
    \begin{block}{改良手法1}
        $n-1 < u-l$の時,各値$j$ごとの$x_{ij}$の和を表す変数$y_{j}$を導入し,以下の制約を追加する.
        \vspace{-3mm}
        \begin{eqnarray*}
            \sum_{i=1}^{n} x_{ij}=y_j \; (j \in \{l,l+1, \ldots,u\})\\
            \vspace{-3mm}
            \sum_{j=l}^{u} y_j = n
        \end{eqnarray*}
    \end{block}
    \vspace{-3mm}
    \begin{block}{改良手法2}
        $n-1 < u-l$の時,各値$j$ごとに新たな変数$x_{(n+1)j}$を導入し,以下の制約を追加する.
        \vspace{-3mm}
        \begin{eqnarray*}
            \sum_{i=1}^{n+1} x_{ij}=1 \; (j \in \{l,l+1, \ldots,u\})\\
            \vspace{-3mm}
            \sum_{j=l}^{u} x_{(n+1)j} = u-l-n
        \end{eqnarray*}
    \end{block}
\end{frame}

\begin{frame}
    \frametitle{高速化手法}
    \vspace{-3mm}
    \begin{exampleblock}{改良手法1の例}
        $distinct(x_1, x_2, x_3)$について, $x_i \in \{1, 2, 3, 4\}$であるときには以下の制約が追加される.
        \vspace{-3mm}
        \begin{eqnarray*}
            x_{11}+x_{21}+x_{31}=y_1\\
            x_{12}+x_{22}+x_{32}=y_2\\
            x_{13}+x_{23}+x_{33}=y_3\\
            x_{14}+x_{24}+x_{34}=y_4\\
            y_1+y_2+y_3+y_4=3
        \end{eqnarray*}
    \end{exampleblock}
\end{frame}
\begin{frame}
    \frametitle{高速化手法}
    \vspace{-3mm}
    \begin{exampleblock}{改良手法2の例}
        $distinct(x_1, x_2, x_3)$について, $x_i \in \{1, 2, 3, 4\}$であるときには以下の制約が追加される.
        \vspace{-3mm}
        \begin{eqnarray*}
            x_{11}+x_{21}+x_{31}+x_{41}=1\\
            x_{12}+x_{22}+x_{32}+x_{42}=1\\
            x_{13}+x_{23}+x_{33}+x_{43}=1\\
            x_{14}+x_{24}+x_{34}+x_{44}=1\\
            x_{41}+x_{42}+x_{43}+x_{44}=1
        \end{eqnarray*}

    \end{exampleblock}
\end{frame}
%%%%%%%%%%%%%%%%%%%%%%%%%%%%%%%%%%%%%%
% PB符号化のSMT符号化
%%%%%%%%%%%%%%%%%%%%%%%%%%%%%%%%%%%%%%
\begin{frame}[fragile]
    \frametitle{\distinct 制約のPB符号化手法のSMT符号化}
    \distinct 制約のPB符号化手法についてのSMT符号化を示す.\\
    例として,\verb|(distinct x_0 x_1 x_2)|は以下のように宣言される.\\
    (\verb|x_i|$\in \{0,1,2,3,4\}$とする)
    \begin{exampleblock}{}\scriptsize
\begin{verbatim}
(assert (= (+ x_0_0 x_0_1 x_0_2 x_0_3 x_0_4) 1))
(assert (= (+ x_1_0 x_1_1 x_1_2 x_1_3 x_1_4) 1))
(assert (= (+ x_2_0 x_2_1 x_2_2 x_2_3 x_2_4) 1))
(assert (<= (+ x_0_0 x_1_0 x_2_0) 1))
(assert (<= (+ x_0_1 x_1_1 x_2_1) 1))
(assert (<= (+ x_0_2 x_1_2 x_2_2) 1))
(assert (<= (+ x_0_3 x_1_3 x_2_3) 1))
(assert (<= (+ x_0_4 x_1_4 x_2_4) 1))
\end{verbatim}
    \end{exampleblock}
    \distinct 制約内の要素数が\verb|x_i|のドメインサイズと同じとき,上記の制約の"\verb|<=|"は"\verb|=|"となる.
\end{frame}



%%%%%%%%%%%%%%%%%%%%%%%%%%%%%%%%%%%%%%
% 改良手法1のSMT符号化
%%%%%%%%%%%%%%%%%%%%%%%%%%%%%%%%%%%%%%
\begin{frame}[fragile]\footnotesize
    \frametitle{改良手法1のSMT符号化}
    \distinct 制約のPB符号化手法の改良手法1についてのSMT符号化を示す.\\
    例として,\verb|(distinct x_0 x_1 x_2)|は以下のような制約が追加される.\\
    (\verb|x_i|$\in \{0,1,2,3,4\}$とする)

    \begin{columns}
        \begin{column}{0.5\textwidth}
            \begin{exampleblock}{}\scriptsize
\begin{verbatim}
(declare-const y_0 Int)
(assert (and (>= y_0 0) (<= y_0 1)))
(declare-const y_1 Int)
(assert (and (>= y_1 0) (<= y_1 1)))
(declare-const y_2 Int)
(assert (and (>= y_2 0) (<= y_2 1)))
(declare-const y_3 Int)
(assert (and (>= y_3 0) (<= y_3 1)))
(declare-const y_4 Int)
(assert (and (>= y_4 0) (<= y_4 1)))
(assert (= y_0 (+ x_0_0 x_1_0 x_2_0)))
(assert (= y_1 (+ x_0_1 x_1_1 x_2_1)))
(assert (= y_2 (+ x_0_2 x_1_2 x_2_2)))
(assert (= y_3 (+ x_0_3 x_1_3 x_2_3)))
(assert (= y_4 (+ x_0_4 x_1_4 x_2_4)))
(assert (= (+ y_0 y_1 y_2 y_3 y_4) 3))
\end{verbatim}
            \end{exampleblock}
        \end{column}
        \begin{column}{0.5\textwidth}\scriptsize
            これは,例えば\verb|x_0,x_1,x_2|が0,1,2にならないと分かった際に,\verb|y_0 + y_1 + y_2 + y_3 + y_4 = 3|が成り立たないとわかるため,探索を切り上げることができる.
        \end{column}
    \end{columns}
\end{frame}


%%%%%%%%%%%%%%%%%%%%%%%%%%%%%%%%%%%%%%
% 改良手法2のSMT符号化
%%%%%%%%%%%%%%%%%%%%%%%%%%%%%%%%%%%%%%
\begin{frame}[fragile]\footnotesize
    \frametitle{改良手法2のSMT符号化}
    \distinct 制約のPB符号化手法の改良手法1についてのSMT符号化を示す.\\
    例として,\verb|(distinct x_0 x_1 x_2)|は以下のような制約が追加される.\\
    (\verb|x_i|$\in \{0,1,2,3,4\}$とする)

    \begin{columns}
        \begin{column}{0.5\textwidth}
            \begin{exampleblock}{}\scriptsize
\begin{verbatim}
(declare-const y_0 Int)
(assert (and (>= y_0 0) (<= y_0 1)))
(declare-const y_1 Int)
(assert (and (>= y_1 0) (<= y_1 1)))
(declare-const y_2 Int)
(assert (and (>= y_2 0) (<= y_2 1)))
(declare-const y_3 Int)
(assert (and (>= y_3 0) (<= y_3 1)))
(declare-const y_4 Int)
(assert (and (>= y_4 0) (<= y_4 1)))
(assert (= (+ x_0_0 x_1_0 x_2_0 y_0) 1))
(assert (= (+ x_0_1 x_1_1 x_2_1 y_1) 1))
(assert (= (+ x_0_2 x_1_2 x_2_2 y_2) 1))
(assert (= (+ x_0_3 x_1_3 x_2_3 y_3) 1))
(assert (= (+ x_0_4 x_1_4 x_2_4 y_4) 1))
(assert (= (+ y_0 y_1 y_2 y_3 y_4) 2))
\end{verbatim}
            \end{exampleblock}
        \end{column}
        \begin{column}{0.5\textwidth}\scriptsize
            これは,例えば\verb|x_0,x_1,x_2|が0,1,2にならないと分かった際に,\verb|y_0 + y_1 + y_2 + y_3 + y_4 = 2|が成り立たないとわかるため,探索を切り上げることができる.
        \end{column}
    \end{columns}

\end{frame}

\begin{frame}
    \frametitle{クイーングラフ彩色問題:N=11}
    \begin{center}
\scalebox{0.5}{
    \begin{tikzpicture}
        \fill[red]          ( 0.5,10.5) circle (0.25);
        \fill[blue]         ( 1.5,10.5) circle (0.25);
        \fill[teal]         ( 2.5,10.5) circle (0.25);
        \fill[pink]         ( 3.5,10.5) circle (0.25);
        \fill[green]        ( 4.5,10.5) circle (0.25);
        \fill[olive]        ( 5.5,10.5) circle (0.25);
        \fill[brown]        ( 6.5,10.5) circle (0.25);
        \fill[black]        ( 7.5,10.5) circle (0.25);
        \fill[yellow]       ( 8.5,10.5) circle (0.25);
        \fill[violet]       ( 9.5,10.5) circle (0.25);
        \fill[lightgray]    (10.5,10.5) circle (0.25);
        \fill[red]          ( 6.5, 9.5) circle (0.25);
        \fill[blue]         ( 7.5, 9.5) circle (0.25);
        \fill[teal]         ( 8.5, 9.5) circle (0.25);
        \fill[pink]         ( 9.5, 9.5) circle (0.25);
        \fill[green]        (10.5, 9.5) circle (0.25);
        \fill[olive]        ( 0.5, 9.5) circle (0.25);
        \fill[brown]        ( 1.5, 9.5) circle (0.25);
        \fill[black]        ( 2.5, 9.5) circle (0.25);
        \fill[yellow]       ( 3.5, 9.5) circle (0.25);
        \fill[violet]       ( 4.5, 9.5) circle (0.25);
        \fill[lightgray]    ( 5.5, 9.5) circle (0.25);
        \fill[red]          ( 1.5, 8.5) circle (0.25);
        \fill[blue]         ( 2.5, 8.5) circle (0.25);
        \fill[teal]         ( 3.5, 8.5) circle (0.25);
        \fill[pink]         ( 4.5, 8.5) circle (0.25);
        \fill[green]        ( 5.5, 8.5) circle (0.25);
        \fill[olive]        ( 6.5, 8.5) circle (0.25);
        \fill[brown]        ( 7.5, 8.5) circle (0.25);
        \fill[black]        ( 8.5, 8.5) circle (0.25);
        \fill[yellow]       ( 9.5, 8.5) circle (0.25);
        \fill[violet]       (10.5, 8.5) circle (0.25);
        \fill[lightgray]    ( 0.5, 8.5) circle (0.25);
        \fill[red]          ( 7.5, 7.5) circle (0.25);
        \fill[blue]         ( 8.5, 7.5) circle (0.25);
        \fill[teal]         ( 9.5, 7.5) circle (0.25);
        \fill[pink]         (10.5, 7.5) circle (0.25);
        \fill[green]        ( 0.5, 7.5) circle (0.25);
        \fill[olive]        ( 1.5, 7.5) circle (0.25);
        \fill[brown]        ( 2.5, 7.5) circle (0.25);
        \fill[black]        ( 3.5, 7.5) circle (0.25);
        \fill[yellow]       ( 4.5, 7.5) circle (0.25);
        \fill[violet]       ( 5.5, 7.5) circle (0.25);
        \fill[lightgray]    ( 6.5, 7.5) circle (0.25);
        \fill[red]          ( 2.5, 6.5) circle (0.25);
        \fill[blue]         ( 3.5, 6.5) circle (0.25);
        \fill[teal]         ( 4.5, 6.5) circle (0.25);
        \fill[pink]         ( 5.5, 6.5) circle (0.25);
        \fill[green]        ( 6.5, 6.5) circle (0.25);
        \fill[olive]        ( 7.5, 6.5) circle (0.25);
        \fill[brown]        ( 8.5, 6.5) circle (0.25);
        \fill[black]        ( 9.5, 6.5) circle (0.25);
        \fill[yellow]       (10.5, 6.5) circle (0.25);
        \fill[violet]       ( 0.5, 6.5) circle (0.25);
        \fill[lightgray]    ( 1.5, 6.5) circle (0.25);
        \fill[red]          ( 8.5, 5.5) circle (0.25);
        \fill[blue]         ( 9.5, 5.5) circle (0.25);
        \fill[teal]         (10.5, 5.5) circle (0.25);
        \fill[pink]         ( 0.5, 5.5) circle (0.25);
        \fill[green]        ( 1.5, 5.5) circle (0.25);
        \fill[olive]        ( 2.5, 5.5) circle (0.25);
        \fill[brown]        ( 3.5, 5.5) circle (0.25);
        \fill[black]        ( 4.5, 5.5) circle (0.25);
        \fill[yellow]       ( 5.5, 5.5) circle (0.25);
        \fill[violet]       ( 6.5, 5.5) circle (0.25);
        \fill[lightgray]    ( 7.5, 5.5) circle (0.25);
        \fill[red]          ( 3.5, 4.5) circle (0.25);
        \fill[blue]         ( 4.5, 4.5) circle (0.25);
        \fill[teal]         ( 5.5, 4.5) circle (0.25);
        \fill[pink]         ( 6.5, 4.5) circle (0.25);
        \fill[green]        ( 7.5, 4.5) circle (0.25);
        \fill[olive]        ( 8.5, 4.5) circle (0.25);
        \fill[brown]        ( 9.5, 4.5) circle (0.25);
        \fill[black]        (10.5, 4.5) circle (0.25);
        \fill[yellow]       ( 0.5, 4.5) circle (0.25);
        \fill[violet]       ( 1.5, 4.5) circle (0.25);
        \fill[lightgray]    ( 2.5, 4.5) circle (0.25);
        \fill[red]          ( 9.5, 3.5) circle (0.25);
        \fill[blue]         (10.5, 3.5) circle (0.25);
        \fill[teal]         ( 0.5, 3.5) circle (0.25);
        \fill[pink]         ( 1.5, 3.5) circle (0.25);
        \fill[green]        ( 2.5, 3.5) circle (0.25);
        \fill[olive]        ( 3.5, 3.5) circle (0.25);
        \fill[brown]        ( 4.5, 3.5) circle (0.25);
        \fill[black]        ( 5.5, 3.5) circle (0.25);
        \fill[yellow]       ( 6.5, 3.5) circle (0.25);
        \fill[violet]       ( 7.5, 3.5) circle (0.25);
        \fill[lightgray]    ( 8.5, 3.5) circle (0.25);
        \fill[red]          ( 4.5, 2.5) circle (0.25);
        \fill[blue]         ( 5.5, 2.5) circle (0.25);
        \fill[teal]         ( 6.5, 2.5) circle (0.25);
        \fill[pink]         ( 7.5, 2.5) circle (0.25);
        \fill[green]        ( 8.5, 2.5) circle (0.25);
        \fill[olive]        ( 9.5, 2.5) circle (0.25);
        \fill[brown]        (10.5, 2.5) circle (0.25);
        \fill[black]        ( 0.5, 2.5) circle (0.25);
        \fill[yellow]       ( 1.5, 2.5) circle (0.25);
        \fill[violet]       ( 2.5, 2.5) circle (0.25);
        \fill[lightgray]    ( 3.5, 2.5) circle (0.25);
        \fill[red]          (10.5, 1.5) circle (0.25);
        \fill[blue]         ( 0.5, 1.5) circle (0.25);
        \fill[teal]         ( 1.5, 1.5) circle (0.25);
        \fill[pink]         ( 2.5, 1.5) circle (0.25);
        \fill[green]        ( 3.5, 1.5) circle (0.25);
        \fill[olive]        ( 4.5, 1.5) circle (0.25);
        \fill[brown]        ( 5.5, 1.5) circle (0.25);
        \fill[black]        ( 6.5, 1.5) circle (0.25);
        \fill[yellow]       ( 7.5, 1.5) circle (0.25);
        \fill[violet]       ( 8.5, 1.5) circle (0.25);
        \fill[lightgray]    ( 9.5, 1.5) circle (0.25);
        \fill[red]          ( 5.5, 0.5) circle (0.25);
        \fill[blue]         ( 6.5, 0.5) circle (0.25);
        \fill[teal]         ( 7.5, 0.5) circle (0.25);
        \fill[pink]         ( 8.5, 0.5) circle (0.25);
        \fill[green]        ( 9.5, 0.5) circle (0.25);
        \fill[olive]        (10.5, 0.5) circle (0.25);
        \fill[brown]        ( 0.5, 0.5) circle (0.25);
        \fill[black]        ( 1.5, 0.5) circle (0.25);
        \fill[yellow]       ( 2.5, 0.5) circle (0.25);
        \fill[violet]       ( 3.5, 0.5) circle (0.25);
        \fill[lightgray]    ( 4.5, 0.5) circle (0.25);
        \draw [step=10mm] (0,0) grid (11,11);
        \node at (5.5, 0) [below] {N=11のとき};
    \end{tikzpicture}
}
\end{center}

\end{frame}


\begin{frame}
    \frametitle{実験結果(ハイブリッドモデル):計測したCPU時間(秒)}
    計測したCPU時間は以下の通りである.
    \begin{block}{}
        {\tiny 
\begin{tabular}[c]{|c|r|r|r|r|r|r|r|r|r|}\hline
               & N=5  & N=6     & N=7          & N=8          &    N=9                &   N=10                   &    N=11& N=12& N=13 \\
               & SAT  & UNSAT   & SAT          & UNSAT        &  UNSAT        &  UNSAT         &    SAT & SAT & SAT \\\hline
    % COL        & 0.01 & 0.09    & 1.09         & 91.88        & TO            &      TO        &      TO&   TO& TO \\
    % COL+H1     & 0.01 & 0.02    & 0.11         & 7.41         & 4044.89       &      TO        &      TO&   TO& TO \\
    % COL+H2     & 0.01 & 0.03    & 0.08         & 0.35         & \textcolor{red}{30.96} &  \textcolor{red}{123.69}&      TO&   TO& TO \\
    % COL+H3     & 0.02 & 0.03    & 0.06         & \textcolor{red}{0.23} & 32.54         &  138.47        & 2338.89&   TO& TO \\
    % COL+PB1    & 0.17 & 0.51    & 2.41         & 6.88         & 401.54        & 3413.20        &      TO&   TO& TO \\
    % COL+PB2    & 0.21 & 0.72    & 3.49         & 9.26         & 530.82        & 6736.73        &      TO&   TO& TO \\
    % COL+PB3    & 0.47 & 1.39    & 193.88       & TO           & TO            & TO             &      TO&   TO& TO \\
    COL+PB1+C  & 0.03 & 0.04    & 0.10         & 2.68         & 141.00        & 2796.16        & \structure{3967.33} & TO & TO \\
    COL+PB2+C  & 0.14 & 905.22  & TO           & TO           & TO            & TO             & TO  & TO & TO \\
    COL+PB3+C  & 1.03 & 1481.66 & TO           & TO           & TO            & TO             & TO  & TO & TO \\
    % POS        & 0.01 & 0.02    & \textcolor{red}{0.05} & 1.00         & 58.37 & 1484.24&              TO&   TO& TO \\
    % POS+H1     & 0.01 & 0.02    & 0.06         & 0.96         & 62.46 & 2093.45&              TO&   TO& TO \\
    % POS+H2     & 0.01 & 0.02    & 0.06         & 0.28         & 46.77 &  167.53&         4677.32&   TO& TO \\
    % POS+H3     & 0.01 & 0.02    & \textcolor{red}{0.05} & 0.43         & 47.38 &  231.74& \textcolor{red}{1257.02}&   TO& TO \\ 
    POS+PB1+C  & 0.05 & 0.08    & 0.16         & 0.87         & 45.18 & 3770.98&             TO &  TO & TO \\
    POS+PB2+C  & 0.06 & 0.10    & 0.18         & 1.15         & 58.12 & 3155.96&             TO &  TO & TO \\
    POS+PB3+C  & 0.06 & 0.10    & 0.16                   & 1.49         & 56.63   & 3254.16                  &  TO & TO & TO\\\hline
\end{tabular}
%
% \begin{tabular}[c]{|c|r|r|r|}\\\hline
%                & N=5  & N=6     & N=7                    & N=8          &    N=9                  \\
%                & SAT  & UNSAT   & SAT                    & UNSAT        &  UNSAT                  \\\hline
%     COL        & 0.01 & 0.09    & 1.09                   & 91.88        & TO                      \\
%     COL+H1     & 0.01 & 0.02    & 0.11                   & 7.41         & 4044.89                 \\
%     COL+H2     & 0.01 & 0.03    & 0.08                   & 0.35         & \textcolor{red}{30.96}  \\
%     COL+H3     & 0.02 & 0.03    & 0.06                   & \textcolor{red}{0.23} & 32.54          \\
%     COL+PB1    & 0.17 & 0.51    & 2.41                   & 6.88         & 401.54                  \\
%     COL+PB2    & 0.21 & 0.72    & 3.49                   & 9.26         & 530.82                  \\
%     COL+PB3    & 0.47 & 1.39    & 193.88                 & TO           & TO                      \\
%     COL+PB1+C  & 0.03 & 0.04    & 0.10                   & 2.68         & 141.00                  \\
%     COL+PB2+C  & 0.14 & 905.22  & TO                     & TO           & TO                      \\
%     COL+PB3+C  & 1.03 & 1481.66 & TO                     & TO           & TO                      \\
%     POS        & 0.01 & 0.02    & \textcolor{red}{0.05}  & 1.00         & 58.37                   \\
%     POS+H1     & 0.01 & 0.02    & 0.06                   & 0.96         & 62.46                   \\
%     POS+H2     & 0.01 & 0.02    & 0.06                   & 0.28         & 46.77                   \\
%     POS+H3     & 0.01 & 0.02    & \textcolor{red}{0.05}  & 0.43         & 47.38                   \\ 
%     POS+PB1+C  & 0.05 & 0.08    & 0.16                   & 0.87         & 45.18                   \\ 
%     POS+PB2+C  & 0.06 & 0.10    & 0.18                   & 1.15         & 58.12                   \\ 
%     POS+PB3+C  & 0.06 & 0.10    & 0.16                   & 1.49         & 56.63                   \\\hline
% \end{tabular}
% \begin{tabular}[c]{|c|r|r|r|}\\\hline
%                 &   N=10                   &    N=11& N=12& N=13 \\
%                 &  UNSAT                   &    SAT & SAT & SAT \\\hline
%      COL        &      TO                  &      TO&   TO& TO \\
%      COL+H1     &      TO                  &      TO&   TO& TO \\
%      COL+H2     &  \textcolor{red}{123.69} &      TO&   TO& TO \\
%      COL+H3     &  138.47                  & 2338.89&   TO& TO \\
%      COL+PB1    & 3413.20                  &      TO&   TO& TO \\
%      COL+PB2    & 6736.73                  &      TO&   TO& TO \\
%      COL+PB3    & TO                       &      TO&   TO& TO \\
%      COL+PB1+C  & 2796.16                  & \structure{3967.33} & TO & TO \\
%      COL+PB2+C  & TO                       & TO  & TO & TO \\
%      COL+PB3+C  & TO                       & TO  & TO & TO \\
%      POS        & 1484.24                  &   TO& TO & TO\\
%      POS+H1     & 2093.45                  &   TO& TO & TO\\
%      POS+H2     &  167.53                  &   TO& TO & TO\\
%      POS+H3     &  231.74                  &   TO& TO & TO\\ 
%      POS+PB1+C  & 3770.98                  &  TO & TO & TO\\
%      POS+PB2+C  & 3155.96                  &  TO & TO & TO\\
%      POS+PB3+C  & 3254.16                  &  TO & TO & TO\\\hline
% \end{tabular}
 }
    \end{block}
    \begin{itemize}
        \item N=11でUNSATとなった.
        \item 色変数モデルでPB符号化の改良手法を用いると性能が低下した.
    \end{itemize}
    \begin{tikzpicture}
 \draw (0,0)--(10,0);
\end{tikzpicture}
\\
    \vspace{-3mm}
    {\footnotesize C:チャネリング制約を使用}\\
\end{frame}





\backupend

%%% Local Variables:
%%% mode: japanese-latex
%%% TeX-master: "kosuge_slide"
%%% End:
