\documentclass[uplatex,dvipdfmx,a4paper,twocolumn,base=11pt,jbase=11pt,ja=standard]{bxjsarticle}

%%%%%%%%%%%%%%%%%%%%%%%%%%%%%%%%%%%%%%%%%%%%%%%%%%%%%%%%%%%%%%%%
% User-defined Macro
%%%%%%%%%%%%%%%%%%%%%%%%%%%%%%%%%%%%%%%%%%%%%%%%%%%%%%%%%%%%%%%%
\newcommand{\compress}{\itemsep0pt\parsep0pt\parskip0pt\partopsep0pt}
% \newcommand{\compress}{\itemsep1pt plus1pt\parsep0pt\parskip0pt}
% \newcommand{\code}[1]{\lstinline[basicstyle=\ttfamily]{#1}}
\newcommand{\gringo}{\textit{gringo}}
\newcommand{\clasp}{\textit{clasp}}
\newcommand{\clingo}{\textit{clingo}}
\newcommand{\teaspoon}{\textit{teaspoon}}
\newcommand{\sat}{\textsf{SAT}}
\newcommand{\unsat}{\textsf{UNSAT}}
% \newcommand{\web}[2]{\href{#1}{#2\ \raisebox{-0.15ex}{\beamergotobutton{Web}}}}
% \newcommand{\doi}[2]{\href{#1}{#2\ \raisebox{-0.15ex}{\beamergotobutton{DOI}}}}
% \newcommand{\weblink}[1]{\web{#1}{#1}}
% \newcommand{\imp}{\mathrel{\Rightarrow}}
% \newcommand{\Iff}{\mathrel{\Leftrightarrow}}
% \newcommand{\mybox}[1]{\fbox{\rule[.2cm]{0cm}{0cm}\mbox{${#1}$}}}
% \newcommand{\mycbox}[2]{\tikz[baseline]\node[fill=#1!10,anchor=base,rounded corners=2pt] () {#2};}
% \newcommand{\naf}[1]{\ensuremath{{\sim\!\!{#1}}}}
% \newcommand{\head}[1]{\ensuremath{\mathit{head}(#1)}}
% \newcommand{\body}[1]{\ensuremath{\mathit{body}(#1)}}
% \newcommand{\atom}[1]{\ensuremath{\mathit{atom}(#1)}}
% \newcommand{\poslits}[1]{\ensuremath{{#1}^+}}
% \newcommand{\neglits}[1]{\ensuremath{{#1}^-}}
% \newcommand{\pbody}[1]{\poslits{\body{#1}}}
% \newcommand{\nbody}[1]{\neglits{\body{#1}}}
% \newcommand{\Cn}[1]{\ensuremath{\mathit{Cn}(#1)}}
% \newcommand{\reduct}[2]{\ensuremath{#1^{#2}}}
% \newcommand{\OK}{\mbox{\textcolor{green}{\Pisymbol{pzd}{52}}}}
% \newcommand{\KO}{\mbox{\textcolor{red}{\Pisymbol{pzd}{56}}}}
% \newcommand{\code}[1]{\lstinline[basicstyle=\ttfamily]{#1}}
% \newcommand{\lw}[1]{\smash{\lower2.ex\hbox{#1}}}
\newcommand{\llw}[1]{\smash{\lower3.ex\hbox{#1}}}

\newenvironment{tableC}{%
  \scriptsize
  \renewcommand{\arraystretch}{0.9}
  \tabcolsep = 0.6mm
  % \begin{tabular}[t]{p{6mm}|rlr|rlr|rlr|rlr|rlr}\hline
  %   \multicolumn{1}{l|}{\llw{問題   }} &
  \begin{tabular}[t]{l|rlr|rlr|rlr|rlr|rlr}\hline
    \multicolumn{1}{l|}{\llw{問題}} &
    \multicolumn{3}{c|}{UD1} &
    \multicolumn{3}{c|}{UD2} &
    \multicolumn{3}{c|}{UD3} &
    \multicolumn{3}{c|}{UD4} &
    \multicolumn{3}{c}{UD5} \\
    & 
    \multicolumn{1}{c}{既知の} & & \multicolumn{1}{c|}{ASP} & 
    \multicolumn{1}{c}{既知の} & & \multicolumn{1}{c|}{ASP} & 
    \multicolumn{1}{c}{既知の} & & \multicolumn{1}{c|}{ASP} & 
    \multicolumn{1}{c}{既知の} & & \multicolumn{1}{c|}{ASP} & 
    \multicolumn{1}{c}{既知の} & & \multicolumn{1}{c}{ASP} \\
    & 
    ベスト & &  & 
    ベスト & &  & 
    ベスト & &  & 
    ベスト & &  & 
    ベスト & &  \\
    \hline
  }{%
    \hline
  \end{tabular}
}


\title{チャネリング制約を用いた alldifferent 制約の SAT 符号化}%
      {SAT Encoding of Alldifferent Constraints with Channeling Constraints}
\author{名古屋大学}{小菅 脩司}{Shuji Kosuge, Nagoya University}
\author{神戸大学}{宋 剛秀}{Takehide Soh, Kobe University}
\author{神戸大学}{田村 直之}{Naoyuki Tamura, Kobe University}
\author{名古屋大学}{番原 睦則}{Mutsunori Banbara, Nagoya University}

\begin{document}
\maketitle

%%%%%%%%%%%%%%%%%%%%%%%%%%%%%%%%
%%% alldiff 制約とその重要性

\textbf{alldifferent制約}は,制約プログラミングにおける代表的なグロー
バル制約の一つである.
$\Alldiff$は,整数変数 $x_1$, $x_2$, \ldots, $x_n$ の値が互いに異なる
ことを表す制約である.すなわち
\begin{align*}\small
  \Alldiff & \Llra \bigwedge_{1 \le i < j \le n} x_i \ne x_j
\end{align*}
である.
各 $x_i$ が 1 以上 $d$ 以下の値を取る場合,
alldifferent 制約の解は $d$ 個から $n$ 個を取り出す順列に対応する.
alldifferent 制約は,人工知能分野の諸問題に頻繁に現れる.
そのため,alldifferent 制約の効率的な実装は重要な研究課題であり,
古くから研究がなされている.%~\cite{DBLP:reference/fai/HoeveK06}.

%%%%%%%%%%%%%%%%%%%%%%%%%%%%%%%%
%%% alldiff 制約の SAT 符号化

一方,2000年以降,
命題論理の充足可能性判定問題(Boolean SATisfiability; SAT)を解く
SAT ソルバーの性能が飛躍的に向上し,
alldifferent 制約を含む制約充足問題(制約プログラミングの言語)を
SAT に符号化して解く手法の研究が進められた.
\textbf{順序符号化法} (Order Encoding; OE) は,
各整数変数$x$と各整数定数$a \in \Dom(x)$に対して,
$x\le a$を意味する命題変数$\oE{x}{a}$を用いる
\cite{DBLP:journals/constraints/TamuraTKB09}.
%
この順序符号化法に基づいたSAT型制約ソルバー{\sf Sugar}
\footnote{\texttt{https://cspsat.gitlab.io/sugar/}}
は,
2008年国際制約ソルバー競技会
のグローバル制約部門で第1位になるなど,
alldifferent 制約に対して優れた性能を示している.

%%%%%%%%%%%%%%%%%%%%%%%%%%%%%%%%
%%% 提案

本発表では,alldifferent 制約の SAT 符号化として,
順序符号化法と直接符号化法をチャネリング制約を用いて
融合させた手法を提案する.
また,クイーングラフ彩色問題を用いた評価結果について述べる.
この問題は alldifferent 制約だけで記述でき,
D.~E.~Knuth の著書~\cite{Knuth:TAOCP:SAT}でも
%D.~E.~Knuth の著書でも
種々な SAT 符号化を比較するためのベンチマークとして用いられている.

提案手法の基本的アイデアは,
各整数変数$x$と各整数定数$a \in \Dom(x)$に対して,
順序符号化法と直接符号化法の両方の命題変数を導入し,
以下のようなチャネリング制約を追加する点である\footnote{%
$a-1\notin\Dom(x)$の場合は,$\lnot\oE{x}{a-1}$は省略}.
\begin{align*}
  \dE{x}{a} & \Llra \lnot\oE{x}{a-1} \land \oE{x}{a}
\end{align*}
$\dE{x}{a}$は
直接符号化法 (Direct Encoding; DE) の命題変数であり$x=a$を意味する.
これにより,提案手法は,
$\Alldiff$を $\bigwedge x_i \ne x_j$ の形に分解し,
各々の $x_i \ne x_j$ を順序符号化法(OE)と直接符号化法(DE)のいずれかを
用いて SAT に符号化できる.
また,
順序符号化法で有効性が確認されている鳩の巣原理
(Pigeon Hole Principle; PHP)に基づくヒント制約,
$n=d$の場合に直接符号化法で有効性が確認されている at-least-one 制約 (ALT1)
を組み合わせて利用できる.
さらに,
alldifferent 制約を
$\bigwedge x_i \ne x_j$ の形に分解する代わりに,
擬似ブール (Pseudo-Boolean; PB) 制約に符号化~\cite{Ono19:ai}し,その後,
SAT に符号化することもできる.

%%%%%%%%%%%%%%%%%%%%%%%%%%%%%%%%%%%%%%%%%%%%%%%
\begin{table*}[tb]
  %\tabcolsep = 2mm
  \small
  \renewcommand{\arraystretch}{0.8}
  \centering
  \caption{比較に用いた alldifferent 制約の符号化一覧}
  \label{table:model}
   \begin{tabular}[c] {c|c|c|c|c|c|c|c}
  符号化 & 整数変数の            & \multicolumn{4}{|c|}{alldifferent制
                                   約の表現} & PHP & ALT1 \\\cline{3-6}
        & 符号化法              & $\neq$分解 & 基本PB & PB3~\cite{Ono19:ai} & PB4~\cite{Ono19:ai} & & \\\hline
  0     & OE                    & OE      &    &       &             &     &      \\
  1     & OE                    & OE      &    &       &             & OE  &      \\
  2     & OE$\Leftrightarrow$DE & DE      &    &       &             &     &      \\
  3     & OE$\Leftrightarrow$DE & DE      &    &       &             &     & DE   \\
  4     & OE$\Leftrightarrow$DE & DE      &    &       &             & OE  &      \\
  5     & OE$\Leftrightarrow$DE & DE      &    &       &             & OE  & DE   \\
  6     & OE$\Leftrightarrow$DE &         & OE &       &             &     &      \\
  7     & OE$\Leftrightarrow$DE &         & OE &       &             & OE  &      \\
  8     & OE$\Leftrightarrow$DE &         &    & OE    &             &     &      \\
  9     & OE$\Leftrightarrow$DE &         &    & OE    &             & OE  &      \\
  10    & OE$\Leftrightarrow$DE &         &    &       & OE          &     &      \\
  11    & OE$\Leftrightarrow$DE &         &    &       & OE          & OE  &      \\\hline
 \end{tabular}

\end{table*}
%%%%%%%%%%%%%%%%%%%%%%%%%%%%%%%%%%%%%%%%%%%%%%%
\begin{table*}[t]
  %\tabcolsep = 2mm
  \small
  \renewcommand{\arraystretch}{0.8}
  \centering
  \caption{クイーングラフ彩色問題の実験結果:
    一問あたりの制限時間は2時間.
    ただし,$N=12$については,$N=11$で性能の良かった上位3モデルについ
    て制限時間を72時間に延ばして得られた結果である.}
    \label{table:result}
    \begin{tabular}{l|rr|rr} 
  & \multicolumn{2}{c|}{基本ソルバー} & \multicolumn{2}{c}{改良ソルバー} \\
  & \code{changed} & \code{unchanged} & \code{changed} & \code{unchanged} \\ \hline
  解けた問題数(到達可能) & 11 & 11 & 11 & 11 \\
  解けた問題数(到達不能) & 10 & 10 & 56 & \alert{60} \\\hline
  平均 CPU 時間(秒) & 223.796 & 151.341 & 101.758 & \alert{59.095} \\
\end{tabular}
\end{table*}
%%%%%%%%%%%%%%%%%%%%%%%%%%%%%%%%%%%%%%%%%%%%%%%

提案手法の有効性を評価するために,
クイーングラフ彩色問題($8\leq N\leq 12$)を用いた実行実験を行なった.
比較に用いた符号化を表~\ref{table:model}に,
結果を表~\ref{table:result}に示す.
表~\ref{table:model}の符号化1は,高速制約ソルバー
{\sf Sugar}のデフォルト設定と同じである.
符号化2--11はチャネリング制約を用いた提案手法である.
例えば,符号化9は,alldifferent 制約を
PB 符号化~\cite{Ono19:ai}したのち,
鳩の巣原理に基づくヒント制約と併せて,
順序符号化法を用いて SAT に符号化する.

表~\ref{table:result}より,提案手法は既存の順序符号化法を単体で用いる
よりも優れた結果を示している.
特に,符号化9は,
クイーングラフ彩色問題の SAT 解法について様々な手法を比較した論
文~\cite{Tamura:queen}でも成功していない$N=12$の発見に成功しており,
提案手法の有効性を示している.

%%%%%%%%%%%%%%%%%%%%%%%%%%%%%%%%%%%%%%%%%%%%%%%%%%%%%%%%%%%%%%%%%%%%%%%%%%%%%%%%%%%%%%%% 
\small
\bibliographystyle{jplain} % 参考文献スタイル
\bibliography{bachelor,aisat,local}    % 参考文献リスト
\end{document}

%%% Local Variables:
%%% mode: latex
%%% TeX-master: t
%%% End:
