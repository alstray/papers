% \chapter{alldifferent 制約の SAT 符号化}
\section{alldifferent 制約の SAT 符号化}

alldifferent 制約は alldifferent($x_1,\dots,x_n$) で表され,その要素$x_i$が互いに異なることを表す.
% この制約は
% $$\bigwedge_{1 \leq i < j \leq n} x_i \neq x_j$$
% を意味する.

直接符号化は各整数変数$x$とそのドメイン$a$について$x=a$であることを命題変数$p(x=a)$で表す.
$x$がそのドメイン$\{\ell,\ell+1,\dots,u\}$の中から少なくとも1つの値を取る制約を
$$p(x=\ell) \lor p(x=\ell+1) \lor \dots \lor p(x=u)$$
と表す.
$x$がそのドメイン$\{\ell,\ell+1,\dots,u\}$の中から多くとも1つの値を取る制約を
$$\lnot p(x=i) \land \lnot p(x=j) \;\; (\ell \leq i < j \leq u)$$
と表す.

順序符号化は各整数変数$x$とそのドメイン$a$について$x \leq a$であることを命題変数$p(x\leq a)$で表す.
$x$とそのドメイン$\{\ell,\ell+1,\dots,u\}$に対し,$p(x \leq \ell),p(x \leq \ell+1),\dots,p(x \leq u-1)$を導入し,その関係を
$$\lnot p(x \leq i) \lor p(x \leq i+1) \; (\ell \leq i \leq u-2)$$
と表す.


alldifferent$(x_1,\dots,x_n)$について$x_i\in \{ \ell,\ell+1,\dots,u \}$である時,
この制約は
$$\bigwedge_{1 \leq i < j \leq n} x_i \neq x_j$$
を意味する.
また,$x \neq x'$は
$$
\bigwedge_{\ell \leq a \leq u} (x \neq a \lor x' \neq a)
$$
で表すことができる.
直接符号化において,$x \neq a$は$\lnot p(x=a)$と符号化できる.
順序符号化において,$x \neq a$は$p(x \leq a-1) \lor \lnot p(x \leq a)$と符号化できる.


% alldifferent$(x_1,\dots,x_n)$について$x_i\in \{ \ell,\ell+1,\dots,u \}$である時,
% 直接符号化では
% $$\bigwedge_{1 \leq i < j \leq n} (p(x_i=a) \lor p(x_j=a)) \;\; (\ell \leq a \leq u)$$
% と符号化できる.
% 順序符号化では
% $$\bigwedge_{1 \leq i < j \leq n} (p(x_i \leq a-1) \lor \lnot p(x_i \leq a) \lor p(x_j \leq a-1) \lor \lnot p(x_j \leq a)) \;\; (\ell \leq a \leq u-2)$$
% と符号化できる.


%%% Local Variables:
%%% mode: latex
%%% TeX-master: "paper"
%%% End:
