% \chapter{チャネリング制約を用いた alldifferent 制約の SAT 符号化}
\section{チャネリング制約を用いた alldifferent 制約の SAT 符号化}

順序符号化と直接符号化を組み合わせる際に用いた制約は以下の通りである.
$$p(x=a) \Leftrightarrow p(x \leq a) \land \lnot p(x \leq a)$$

チャネリングさせることで
順序符号化で有効な鳩の巣原理を用いたヒント制約や,
直接符号化で有効なat-least-one制約を組み合わせることができる.
alldiferent制約はPB・大野\cite{Ono19:ai}の手法で表現することができる.


\begin{table}[]
    \caption{提案した符号化一覧}
    \label{table:model}
    % {\scriptsize  \begin{tabular}[c] {|c|c|c|c|c|}\hline
  model & 符号    & alldiff & PHP & ALT1 \\\hline
  0     & OE      & neq     &    &      \\
  1     & OE      & neq     & \checkmark   &      \\
  2     & OE      & neq     &    & \checkmark    \\
  3     & OE      & neq     & \checkmark   & \checkmark    \\
  4     & OE{\textless=\textgreater}DE & neq     &    &  \\
  5     & OE{\textless=\textgreater}DE & neq     & \checkmark   &  \\
  6     & OE{\textless=\textgreater}DE & neq     &    & \checkmark \\
  7     & OE{\textless=\textgreater}DE & neq     & \checkmark   & \checkmark \\
  8     & OE{\textless=\textgreater}DE & PB      &    &   \\
  9     & OE{\textless=\textgreater}DE & PB      & \checkmark   &   \\
  10    & OE{\textless=\textgreater}DE & 大野3   &    &   \\
  11    & OE{\textless=\textgreater}DE & 大野3   & \checkmark   &   \\
  12    & OE{\textless=\textgreater}DE & 大野4   &    &   \\
  13    & OE{\textless=\textgreater}DE & 大野4   & \checkmark   &   \\\hline
 \end{tabular}
}
    {\tiny  \begin{tabular}[c] {c|c|c|c|c|c|c|c}
  符号化 & 整数変数の            & \multicolumn{4}{|c|}{alldifferent制
                                   約の表現} & PHP & ALT1 \\\cline{3-6}
        & 符号化法              & $\neq$分解 & 基本PB & PB3~\cite{Ono19:ai} & PB4~\cite{Ono19:ai} & & \\\hline
  0     & OE                    & OE      &    &       &             &     &      \\
  1     & OE                    & OE      &    &       &             & OE  &      \\
  2     & OE$\Leftrightarrow$DE & DE      &    &       &             &     &      \\
  3     & OE$\Leftrightarrow$DE & DE      &    &       &             &     & DE   \\
  4     & OE$\Leftrightarrow$DE & DE      &    &       &             & OE  &      \\
  5     & OE$\Leftrightarrow$DE & DE      &    &       &             & OE  & DE   \\
  6     & OE$\Leftrightarrow$DE &         & OE &       &             &     &      \\
  7     & OE$\Leftrightarrow$DE &         & OE &       &             & OE  &      \\
  8     & OE$\Leftrightarrow$DE &         &    & OE    &             &     &      \\
  9     & OE$\Leftrightarrow$DE &         &    & OE    &             & OE  &      \\
  10    & OE$\Leftrightarrow$DE &         &    &       & OE          &     &      \\
  11    & OE$\Leftrightarrow$DE &         &    &       & OE          & OE  &      \\\hline
 \end{tabular}
}
\end{table}
% % model.tex の説明
% 1列目は提案した符号化の番号を表している.
% 2列目は alldifferent 制約の符号化手法を表している.
% OE は順序符号化法で符号化することを表している.
% OE\textless=\textgreater DE は順序符号化法と直接符号化法をチャネリングさせて符号化する符号化することを表している.
% 3列目は alldifferent 制約の分解方法を表している.
% neq はnot-equal 制約に分解することを表している.
% PB はブール基数制約を用いて分解することを表している.
% 大野3・大野4は[大野,'19]の手法3・4を用いて分解することを表している.
% 4列目は 鳩の巣原理を用いたヒント制約の有無を表している.
% チェックマークを付けたモデルにヒント制約として追加している.
% 5列目は At-least-one 制約の有無を表している
% チェックマークを付けたモデルにヒント制約として追加している.

% model2.tex の説明
表\ref{table:model}は提案した alldifferent 制約のSAT符号化手法の一覧である.
1列目は提案した符号化の番号を,
2列目は 整数変数の符号化手法を表している.
OE は順序符号化法で符号化することを,
OE$\Leftrightarrow$ DE は順序符号化法と直接符号化法をチャネリングさせて符号化する符号化することを表している.
3列目は alldifferent 制約の分解方法を表している.
その中で$\neq$ はnot-equal 制約,
PB はブール基数制約,
大野3・大野4は[大野,'19]の手法3・4を用いて分解することを表している.
そして,どの分解方法を用いるのかと,それぞれの分解方法が直接符号化と順序符号化のどちらで符号化されるかをDEとOEで表している.
4・5列目は 鳩の巣原理を用いたヒント制約とat-least-one制約を直接符号化と順序符号化のどちらで符号化するのかとその有無を表している.




%%% Local Variables:
%%% mode: latex
%%% TeX-master: "paper"
%%% End:
