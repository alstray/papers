
%%%%%%%%%%%%%%%%%%%%%%%%%%%%%%%%%%%%%%%%%%%%%%%%%%%%%%%%%%%%%
% %%%% 補助スライド
%%%%%%%%%%%%%%%%%%%%%%%%%%%%%%%%%%%%%%%%%%%%%%%%%%%%%%%%%%%%%

\appendix

\backupbegin

%%%%%%%%%%%%%%%%%%%%%%%%%%%%%%%%%%%%%%
% ~
%%%%%%%%%%%%%%%%%%%%%%%%%%%%%%%%%%%%%%
\begin{frame}
    \frametitle{~}
    \centering
    - 補足用 -
\end{frame}



%%%%%%%%%%%%%%%%%%%%%%%%%%%%%%%%%%%%%%
% 鳩の巣原理を用いたヒント制約(PHP)~[田島・田村,2008]
%%%%%%%%%%%%%%%%%%%%%%%%%%%%%%%%%%%%%%
\begin{frame}
    \frametitle{鳩の巣原理を用いたヒント制約(PHP)~[田島・田村,2008]}
    SAT符号化された{\alldifferent}制約に,鳩の巣原理を用いたヒントを加える
    と求解速度が向上することが知られている.
    \begin{block}{}
        $alldifferent(x_{1},\ldots,x_{n})$について,$x_i \in
        \{\ell,\ell+1,\ldots,u\}$であるとき,以下の2つの制約を追加する.
        \[
            \bigvee_{i=1}^{n}x_{i}\geq \ell+n-1 \qquad
            \bigvee_{i=1}^{n}x_{i}\leq u-n+1
        \]
    \end{block}
    \begin{exampleblock}{例}
        $alldifferent(x_1, x_2, x_3)$について, $x_i \in \{1,2,3\}$であるとき,以下の制約が追加される.
        \vspace{-3mm}
        \begin{eqnarray*}
            (x_1\geq 3) \lor (x_2 \geq 3) \lor (x_3 \geq 3)\\
            (x_1\leq 1) \lor (x_2 \leq 1) \lor (x_3 \leq 1)
        \end{eqnarray*}
    \end{exampleblock}
\end{frame}


%%%%%%%%%%%%%%%%%%%%%%%%%%%%%%%%%%%%%%
% at-least-one制約
%%%%%%%%%%%%%%%%%%%%%%%%%%%%%%%%%%%%%%
\begin{frame}
    \frametitle{at-least-one制約を用いたヒント制約(ALT1)}
    \vspace{-3mm}
    \begin{block}{}
        $alldfifferent(x_1,x_2,\ldots,x_n)$について, $x_i \in \{\ell, \ell+1,\ldots, u\}$かつ$u-\ell=n-1$であるときに以下の制約を追加する.\\
        \vspace{-3mm}
        $$\bigvee_{i=1}^n x_i=a \qquad (a \in \{\ell,\ldots, u\})$$
    \end{block}
    \begin{exampleblock}{at-least-one制約を用いたヒント制約の例}
        $alldifferent(x_1, x_2, x_3, x_4)$について, $x_i \in \{1, 2, 3, 4\}$であるときには以下の制約が追加される.
        \vspace{-3mm}
        \begin{eqnarray*}
            (x_1=1) \lor (x_2=1) \lor (x_3=1) \lor (x_4=1)\\
            (x_1=2) \lor (x_2=2) \lor (x_3=2) \lor (x_4=2)\\
            (x_1=3) \lor (x_2=3) \lor (x_3=3) \lor (x_4=3)\\
            (x_1=4) \lor (x_2=4) \lor (x_3=4) \lor (x_4=4)
        \end{eqnarray*}
    \end{exampleblock}
\end{frame}


%%%%%%%%%%%%%%%%%%%%%%%%%%%%%%%%%%%%%%
% alldifferent制約の擬似ブール符号化
%%%%%%%%%%%%%%%%%%%%%%%%%%%%%%%%%%%%%%
\begin{frame}
    \frametitle{{\alldiff}制約の擬似ブール符号化(PB) (\only<1>{1}\only<2>{2}/2)}
    % {\alldiff}制約をブール基数制約で表現することができる.
    \begin{exampleblock}{}
        $x_i \in \{ 1 \dots d \}, n \geq d$ である $\Alldiff$に対して,$p_{ij}=1 \Llra x_i=j$である$n$行$d$列の0-1行列($p_{ij}$)を導入する.
        % \vspace{-3mm}
        \begin{displaymath}
            \begin{array}{ccccc}
             & & & \begin{array}{cccc}  1\;\;&2&\;\dots&d \end{array}&\\
                (p_{ij})&=&
                \begin{array}{c}x_1\\ x_2\\ \vdots\\ x_n \end{array}&
                \left(
                    \begin{array}{cccc}
                        p_{11}&p_{12}&\dots&p_{1d}\\
                        p_{21}&p_{22}&\dots&p_{2d}\\
                        \vdots&\vdots&\ddots&\vdots\\
                        p_{n1}&p_{n2}&\dots&p_{nd}
                \end{array}\right)&
                \begin{array}{c} \alert<1>{= 1}\\ \alert<1>{= 1}\\ \alert<1>{\vdots}\\ \alert<1>{= 1} \end{array}\\
                & & & \begin{array}{cccc}  
                \alert<2>{\rotatebox[origin=c]{-90}{$\leq$}}\;\;&
                \alert<2>{\rotatebox[origin=c]{-90}{$\leq$}}&
                \alert<2>{\dots}\;&
                \alert<2>{\rotatebox[origin=c]{-90}{$\leq$}} 
                \end{array}&\\
                & & & \begin{array}{cccc}  \alert<2>{1}\;\;&\alert<2>{1}&\;\alert<2>{\dots}\;&\alert<2>{1} \end{array}&
            \end{array}
        \end{displaymath}
        \begin{itemize}
            \only<1>{
            \item 各$x_i$はちょうど一つの値をとる.
            \vspace{-3mm}
                % $$ \sum_{j=1}^{d} p_{ij}=1 \; (i \in \{1,2,\ldots,n\}) $$
            $$ p_{i1} + \ldots + p_{id} = 1 \; (i \in \{1,2,\ldots,n\})$$
            }
            \only<2>{
            \item 各列について1となるのは高々1つである. \footnote{n=dのときはちょうど1つである.}
            \vspace{-3mm}
            $$ p_{1j} + \ldots + p_{nj} \leq 1 \; (j \in \{1,2,\ldots,d\})$$
            }
                % $$ \sum_{i=1}^{n} p_{ij} \leq 1 \; (j \in \{l,l+1,\ldots,u\})$$
                % これは$n=d$の時には等号にできる
                % $$\sum_{i=1}^{n} p_{ij} = 1 \; (j \in \{l,l+1,\ldots,u\})$$
        \end{itemize}
    \end{exampleblock}
\end{frame}

%%%%%%%%%%%%%%%%%%%%%%%%%%%%%%%%%%%%%%
% PB3
%%%%%%%%%%%%%%%%%%%%%%%%%%%%%%%%%%%%%%
\begin{frame}
    \frametitle{{\alldiff}制約の擬似ブール符号化(PB3)[大野 '19]}
    \begin{exampleblock}{}
        PBの制約に加えて,各列$j$の和を表す変数$y_j$を追加を導入し以下の制約を追加する
        \begin{itemize}
            \item $p_{1j} + \ldots + p_{nj} = y_j$
            \item $y_1 + \ldots + y_d = n$
        \end{itemize}
        % \vspace{-3mm}
        \begin{displaymath}
            \begin{array}{ccccc}
             & & & \begin{array}{cccc}  1\;\;&2&\;\dots&d \end{array}&\\
                (p_{ij})&=&
                \begin{array}{c}x_1\\ x_2\\ \vdots\\ x_n \end{array}&
                \left(
                    \begin{array}{cccc}
                        p_{11}&p_{12}&\dots&p_{1d}\\
                        p_{21}&p_{22}&\dots&p_{2d}\\
                        \vdots&\vdots&\ddots&\vdots\\
                        p_{n1}&p_{n2}&\dots&p_{nd}
                \end{array}\right)&
                \begin{array}{c} = 1\\ = 1\\ \vdots\\ = 1 \end{array}\\
                & & & \begin{array}{cccc}
                    \alert{\rotatebox[origin=c]{-90}{$=$}}\;\;\;\;&
                    \alert{\rotatebox[origin=c]{-90}{$=$}}\;\;\;&
                    \alert{\dots}&
                    \;\alert{\rotatebox[origin=c]{-90}{$=$}}\;\;
                \end{array}&\\
                & & & \begin{array}{cccc}  \alert{y_1}\;\;&\alert{y_2}&\;\alert{\dots}\;&\alert{y_d} \end{array}&\alert{=n} 
            \end{array}
        \end{displaymath}
    \end{exampleblock}
\end{frame}


%%%%%%%%%%%%%%%%%%%%%%%%%%%%%%%%%%%%%%
% PB4
%%%%%%%%%%%%%%%%%%%%%%%%%%%%%%%%%%%%%%
\begin{frame}
    \frametitle{{\alldiff}制約の擬似ブール符号化(PB4)[大野 '19]}
    \begin{exampleblock}{}
        $n<d$の時,0-1行列($p_{ij}$)に$n+1$行目を追加し,{\alldifferent}制約を以下のように表す.
        \begin{itemize}
            \item $p_{i1} + \dots + p_{id} = 1$
            \item $p_{(n+1)1} + \dots + p_{(n+1)d} = d-n$
            \item $p_{1j} + \dots + p_{(n+1)d} = 1$
        \end{itemize}
        % \vspace{-3mm}
        \begin{displaymath}
            \begin{array}{ccccc}
             & & & \begin{array}{cccc}  \;\;\;\;1\;\;\;\;&\;\;\;\;2\;\;\;\;&\dots&\;\;\;\;d\;\;\;\; \end{array}&\\
                (p_{ij})&=&
                \begin{array}{c}x_1\\ x_2\\ \vdots\\ x_n \end{array}&
                \left(
                    \begin{array}{cccc}
                        \;\;\;\;p_{11\;\;\;\;}&\;\;\;\;p_{12\;\;\;\;}&\dots&\;\;\;\;p_{1d\;\;\;\;}\\
                        \;\;\;\;p_{21\;\;\;\;}&\;\;\;\;p_{22\;\;\;\;}&\dots&\;\;\;\;p_{2d\;\;\;\;}\\
                        \;\;\;\;\vdots\;\;\;\;&\;\;\;\;\vdots\;\;\;\;&\ddots&\;\;\;\;\vdots\;\;\;\;\\
                        \;\;\;\;p_{n1\;\;\;\;}&\;\;\;\;p_{n2\;\;\;\;}&\dots&\;\;\;\;p_{nd\;\;\;\;}
                \end{array}\right)&
                \begin{array}{l} = 1\\ = 1\\ \vdots\\ = 1  \end{array}\\
                                                                    & & \begin{array}{c}\alert{x_{n+1}}\\ \\ \\ \end{array}& \begin{array}{cccc}  
                \alert{p_{(n+1)1}}&\alert{p_{(n+1)2}}&\alert{\dots}&\alert{p_{(n+1)d}} \\
                \rotatebox[origin=c]{-90}{$=$}&\rotatebox[origin=c]{-90}{$=$}&\dots & \rotatebox[origin=c]{-90}{$=$}\\
                1&1&\dots&1
                \end{array}& \begin{array}{c}\alert{=d-n}\\ \\ \\ \end{array}
            \end{array}
        \end{displaymath}
    \end{exampleblock}
\end{frame}



\begin{frame}
    \frametitle{クイーングラフ彩色問題}
    \begin{itemize}
        \item クイーングラフ彩色問題の大きさ$N>9$で2か3の倍数である時,解を求めることが難しい.
        \item $N$が2と3の素数である時,チェスのナイトの動きで彩色すればN色で塗ることができる.
        \item 現在では上記のN以外に$N=\{12, 14, 15, 16, 18, 20, 21, 22, 24, 26, 28, 30, 32\}$に対して解が発見されている.
    \end{itemize}

\end{frame}
\backupend

%%% Local Variables:
%%% mode: japanese-latex
%%% TeX-master: "slide"
%%% End:
