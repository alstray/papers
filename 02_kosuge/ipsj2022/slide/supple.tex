
%%%%%%%%%%%%%%%%%%%%%%%%%%%%%%%%%%%%%%%%%%%%%%%%%%%%%%%%%%%%%
% %%%% 補助スライド
%%%%%%%%%%%%%%%%%%%%%%%%%%%%%%%%%%%%%%%%%%%%%%%%%%%%%%%%%%%%%

\appendix

\backupbegin

%%%%%%%%%%%%%%%%%%%%%%%%%%%%%%%%%%%%%%
% ~
%%%%%%%%%%%%%%%%%%%%%%%%%%%%%%%%%%%%%%
\begin{frame}
    \frametitle{~}
    \centering
    - 補足用 -
\end{frame}



%%%%%%%%%%%%%%%%%%%%%%%%%%%%%%%%%%%%%%
% 鳩の巣原理を用いたヒント制約(PHP)~[田島・田村,2008]
%%%%%%%%%%%%%%%%%%%%%%%%%%%%%%%%%%%%%%
\begin{frame}
    \frametitle{鳩の巣原理を用いたヒント制約(PHP)~[田島・田村,2008]}
    SAT符号化された{\alldifferent}制約に,鳩の巣原理を用いたヒントを加える
    と求解速度が向上することが知られている.
    \begin{block}{}
        $alldifferent(x_{1},\ldots,x_{n})$について,$x_i \in
        \{\ell,\ell+1,\ldots,u\}$であるとき,以下の2つの制約を追加する.
        \[
            \bigvee_{i=1}^{n}x_{i}\geq \ell+n-1 \qquad
            \bigvee_{i=1}^{n}x_{i}\leq u-n+1
        \]
    \end{block}
    \begin{exampleblock}{例}
        $alldifferent(x_1, x_2, x_3)$について, $x_i \in \{1,2,3\}$であるとき,以下の制約が追加される.
        \vspace{-3mm}
        \begin{eqnarray*}
            (x_1\geq 3) \lor (x_2 \geq 3) \lor (x_3 \geq 3)\\
            (x_1\leq 1) \lor (x_2 \leq 1) \lor (x_3 \leq 1)
        \end{eqnarray*}
    \end{exampleblock}
\end{frame}


%%%%%%%%%%%%%%%%%%%%%%%%%%%%%%%%%%%%%%
% at-least-one制約
%%%%%%%%%%%%%%%%%%%%%%%%%%%%%%%%%%%%%%
\begin{frame}
    \frametitle{at-least-one制約を用いたヒント制約(ALT1)}
    \vspace{-3mm}
    \begin{block}{}
        $alldfifferent(x_1,x_2,\ldots,x_n)$について, $x_i \in \{\ell, \ell+1,\ldots, u\}$かつ$u-\ell=n-1$であるときに以下の制約を追加する.\\
        \vspace{-3mm}
        $$\bigvee_{i=1}^n x_i=a \qquad (a \in \{\ell,\ldots, u\})$$
    \end{block}
    \begin{exampleblock}{at-least-one制約を用いたヒント制約の例}
        $alldifferent(x_1, x_2, x_3, x_4)$について, $x_i \in \{1, 2, 3, 4\}$であるときには以下の制約が追加される.
        \vspace{-3mm}
        \begin{eqnarray*}
            (x_1=1) \lor (x_2=1) \lor (x_3=1) \lor (x_4=1)\\
            (x_1=2) \lor (x_2=2) \lor (x_3=2) \lor (x_4=2)\\
            (x_1=3) \lor (x_2=3) \lor (x_3=3) \lor (x_4=3)\\
            (x_1=4) \lor (x_2=4) \lor (x_3=4) \lor (x_4=4)
        \end{eqnarray*}
    \end{exampleblock}
\end{frame}


%%%%%%%%%%%%%%%%%%%%%%%%%%%%%%%%%%%%%%
% alldifferent制約の擬似ブール符号化
%%%%%%%%%%%%%%%%%%%%%%%%%%%%%%%%%%%%%%
\begin{frame}
    \frametitle{{\alldiff}制約の擬似ブール符号化}
    {\alldiff}制約を$\neq$で表現する他にブール基数制約で表現することができる.
    \begin{exampleblock}{}
        $x_i \in \{ 1 \dots d \}, n \geq d$ である $\Alldiff$に対して,$p_{ij}=1 \Llra x_i=j$である$n$行$d$列の0-1行列($p_{ij}$)を導入する.
        \begin{displaymath}
            \begin{array}{cccc}
             & & &
             \begin{array}{cccc}
                 1&2&\dots&d
             \end{array}\\
                (p_{ij})&=&
                \begin{array}{c}x_1\\ x_2\\ \vdots\\ x_n \end{array}&
                \left(
                    \begin{array}{cccc}
                        p_{11}&p_{12}&\dots&p_{1d}\\
                        p_{21}&p_{22}&\dots&p_{2d}\\
                        \vdots&\vdots&\ddots&\vdots\\
                        p_{n1}&p_{n2}&\dots&p_{nd}
                \end{array}\right)
            \end{array}
        \end{displaymath}
        \begin{itemize}
            \item 各$x_i$はちょうど一つの値をとる.
                % $$ \sum_{j=l}^{u} p_{ij}=1 \; (i \in \{1,2,\ldots,n\}) $$
            \item 各列について1となるのは高々1つである.
                % $$ \sum_{i=1}^{n} p_{ij} \leq 1 \; (j \in \{l,l+1,\ldots,u\})$$
                % これは$n=d$の時には等号にできる
                % $$\sum_{i=1}^{n} p_{ij} = 1 \; (j \in \{l,l+1,\ldots,u\})$$
        \end{itemize}
    \end{exampleblock}
\end{frame}

\backupend

%%% Local Variables:
%%% mode: japanese-latex
%%% TeX-master: "slide"
%%% End:
