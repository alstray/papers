\documentclass [dvipdfmx,11pt]{beamer}
\usepackage{bxdpx-beamer}
\usepackage{pxjahyper}
\usepackage{amsmath}
\usepackage{bm}
\usepackage{minijs}
\usepackage{tikz}
\usepackage{multicol}
\usepackage{amssymb}
%\usepackage{otf}
%\renewcommand{\kanjifamilydefault}{\gtdefault}
%%% Beamer
%\AtBeginDvi{\special{pdf:tounicode EUC-UCS2}}
%\usetheme{Madrid}
%\usetheme{Copenhagen}
\usetheme{Warsaw}
% \renewcommand{\kanjifamilydefault}{\gtdefault}
\usefonttheme{structurebold}
%\usefonttheme{professionalfonts}
\setbeamertemplate{blocks}[shadow=true,rounded]
% \setbeamercolor{structure}{fg=blue!60!black}
\setbeamercolor{structure}{fg=blue!50!black}
\setbeamercolor{item projected}{fg=black,bg=blue!20!white}
%\setbeamercolor{alerted text}{fg=red!80!black}
\setbeamercolor{alerted text}{fg=red!70!black}
\setbeamertemplate{navigation symbols}{}
\useoutertheme[subsection=false]{miniframes}
\setbeamertemplate{footline}[frame number]
%%% Tikz
\usetikzlibrary{intersections, calc, arrows}
\setbeamertemplate{navigation symbols}{}
\setbeamertemplate{itemize item}[circle]
\setbeamersize{text margin left=1.5em,text margin right=1.5em}
\setlength{\abovedisplayskip}{0pt} % 上部のマージン
\setlength{\belowdisplayskip}{0pt} % 下部のマージン
\setlength{\columnsep}{0pt}
%
%
%
% footer setting %
\makeatother
\setbeamertemplate{footline}
{
    \leavevmode%
    \hbox{%
        \begin{beamercolorbox}[wd=.4\paperwidth,ht=2.25ex,dp=1ex,center]{author in head/foot}%
            \usebeamerfont{author in head/foot}\insertshortauthor
        \end{beamercolorbox}%
        \begin{beamercolorbox}[wd=.6\paperwidth,ht=2.25ex,dp=1ex,center]{title in head/foot}%
            \usebeamerfont{title in head/foot}\hspace*{1ex} \insertshorttitle\hspace*{2em}
            \textbf{ \insertframenumber{} / \inserttotalframenumber } \hspace*{1ex}
    \end{beamercolorbox}}%
    \vskip0pt%
}
\makeatletter
% exclude apprendix slides from framenumber %
\newcommand{\backupbegin}{
    \newcounter{framenumberappendix}
    \setcounter{framenumberappendix}{\value{framenumber}}
}
\newcommand{\backupend}{
    \addtocounter{framenumberappendix}{-\value{framenumber}}
    \addtocounter{framenumber}{\value{framenumberappendix}}
}


\usepackage{ipsj}
\usepackage{color}
\usepackage{amssymb}
\usepackage{amsmath}
\usepackage{amsthm}
\usepackage{multirow,bigdelim}
\newcommand{\la}{\leftarrow}
\newcommand{\Lra}{\Longrightarrow}
\newcommand{\Lla}{\Longleftarrow}
\newcommand{\Llra}{\Longleftrightarrow}
\newcommand{\lra}{\longrightarrow}
\newcommand{\dd}{\mathop{..}}
\newcommand{\range}[2]{\{#1\dd#2\}}
\newcommand{\imp}{\Rightarrow}
\newcommand{\equ}{\Leftrightarrow}
\renewcommand{\labelenumi}{(\arabic{enumi})}
\newcommand{\alldiff}{\textrm{alldifferent}}
\newcommand{\Alldiff}{\alldiff(x_1,x_2,\ldots,x_n)}
\newcommand{\SAT}{{\tt SAT}}
\newcommand{\UNSAT}{{\tt UNSAT}}
\newcommand{\Dom}{{\it Dom}}
% \newcommand{\p}[2]{p(#1,#2)}
\newcommand{\dE}[2]{p(#1=#2)}
\newcommand{\lE}[2]{p(#1^{(#2)})}
\newcommand{\oE}[2]{p(#1\le#2)}

%%%%%%%%%%%% my macro %%%%%%%%%%%%%%%%%
\newcommand{\alldifferent}{$alldifferent$}
%%%%%%%%%%%%%%%%%%%%%%%%%%%%%%%%%%%%%%%


\title{チャネリング制約を用いた\\ alldifferent 制約の SAT 符号化}
% \author{小菅脩司}
% \institute{名古屋大学}
\author{小菅 脩司\inst{1} \and 宋 剛秀\inst{2} \and 田村 直之\inst{2} \and 番原 睦則\inst{1}}
\institute{ \inst{1}名古屋大学 \ \  \inst{2}神戸大学 }
\date{情報処理学会第84回全国大会}
\begin{document}
\begin{frame} {}
    \titlepage
\end{frame}
%%%%%%%%%%%%%%%%%%%%%%%%%%%%%%%%%%%%%%




%%%%%%%%%%%%%%%%%%%%%%%%%%%%%%%%%%%%%%
% alldifferent制約と制約充足問題
%%%%%%%%%%%%%%%%%%%%%%%%%%%%%%%%%%%%%%
\begin{frame}
    \frametitle{{\alldifferent}制約と制約充足問題}
    \begin{alertblock}{}
        \bm{$$alldifferent(x_{1},x_{2},\ldots, x_{n})$$}
        {\alldifferent}制約は,整数上の変数$x_{i}$が互いに異なることを表す制約
        である.
    \end{alertblock}
    \begin{itemize}
        \item この制約は,
            $$\bigwedge_{1 \leq i < j \leq n} x_i \neq x_j$$
            を意味する.
        \item {\alldifferent}制約は,時間割問題,グラフ彩色問題,組合せデザイン
            など様々な制約充足問題に現れる.
        \item そのような問題に対し,{\alldifferent}制約を効率良く解くことは重要
            な研究課題である.
    \end{itemize}
\end{frame}




%%%%%%%%%%%%%%%%%%%%%%%%%%%%%%%%%%%%%%
% 順序符号化
%%%%%%%%%%%%%%%%%%%%%%%%%%%%%%%%%%%%%%
\begin{frame}
    \frametitle{{\alldiff}制約のSAT符号化法と順序符号化法(Order Encoding:OE)}
    \begin{itemize}
        \item 2000年以降,命題論理の充足可能性判定問題を解くSAT ソルバーの性能が飛躍的に向上し,{\alldiff}制約を含む制約充足問題をSAT に符号化して解く手法の研究が進められた.
    \end{itemize}
    \begin{alertblock}{順序符号化法}
        各整数変数$x$と各整数定数$a \in \Dom(x)$に対して,$x \le a$を意味する命題変数\alert{\bf $\oE{x}{a}$}を用いる.
    \end{alertblock}
    \begin{itemize}
        \item 順序符号化法に基づいたSAT型制約ソルバーSugarが,2008年国際制約ソルバー競技会のグローバル制約部門で第1位になるなど,{\alldiff}制約に対して優れた性能を示している.
    \end{itemize}
\end{frame}



%%%%%%%%%%%%%%%%%%%%%%%%%%%%%%%%%%%%%%
% 研究概要
%%%%%%%%%%%%%%%%%%%%%%%%%%%%%%%%%%%%%%
\begin{frame}
    \frametitle{研究概要}
    \begin{alertblock}{研究目的}
        チャネリング制約を用いて{\alldiff}制約をSAT符号化し,
        クイーングラフ彩色問題を題材として比較評価する.
    \end{alertblock}
    \begin{block}{研究内容}
        \begin{enumerate}
            % \item {\alldiff}制約を直接符号化法(DE)と順序符号化法(OE)のいずれかでSAT符号化
            \item {\alldiff}制約の整数変数を直接符号化法(DE)と順序符号化法(OE)のいずれかでSAT符号化
            \item {\alldiff}制約の表現方法を4種類実装
                \begin{itemize}
                    \item $\neq$の形に分解
                    \item {\alldiff}制約の擬似ブール符号化(基本PB,PB3,PB4)
                \end{itemize}
            \item {\alldiff}制約の高速化手法を2種類実装
                \begin{itemize}
                    \item 鳩の巣原理を用いたヒント制約(PHP)
                    \item at-least-one制約を用いたヒント制約(ALT1)
                \end{itemize}
            \item \structure{クイーングラフ彩色問題($5\leq N \leq 12$)を用いた評価実験}
        \end{enumerate}
    \end{block}
\end{frame}



%%%%%%%%%%%%%%%%%%%%%%%%%%%%%%%%%%%%%%
% alldifferent制約が現れる制約充足問題の例
%%%%%%%%%%%%%%%%%%%%%%%%%%%%%%%%%%%%%%
\begin{frame}
    \frametitle{{\alldifferent}制約が現れる制約充足問題の例}
    \begin{block}{クイーングラフ彩色問題}
        $N$個ずつの$N$個のグループからなるクイーン (計$N^2$個) を,
        $N\times N$のチェス盤に,同じグループのクイーン同士が互いに取られ
        ないように配置する問題
    \end{block}
    \begin{exampleblock}{}\centering
        \begin{columns}
            \begin{column}{0.48\textwidth}\centering
                \includegraphics[width=3cm]{images/qgcp_5.jpg}
            \end{column}
            \begin{column}{0.48\textwidth}\centering
                \includegraphics[width=3cm]{images/qgcp_5_c.jpg}
            \end{column}
        \end{columns}
    \end{exampleblock}
    \begin{itemize}
        \item この問題は{\alldifferent}制約のみを用いて記述できる.
        \item Knuthの教科書 The Art of Computer Programming でも
            取り上げられている.
    \end{itemize}
\end{frame}


%%%%%%%%%%%%%%%%%%%%%%%%%%%%%%%%%%%%%%
% 追加したチャネリング制約
%%%%%%%%%%%%%%%%%%%%%%%%%%%%%%%%%%%%%%
\begin{frame}
    \frametitle{追加したチャネリング制約}
    \begin{block}{基本的アイデア}
        各整数変数$x$と各整数定数$a \in \Dom(x)$に対して,順序符号化法(OE)と直接符号化法(DE)の両方の命題変数を導入し,以下のような制約を追加する.
        \footnote{$a-1 \in \Dom(x)$の場合は,$\lnot\oE{x}{a-1}$は省略}
        \[
            \dE{x}{a} \Llra \lnot\oE{x}{a-1} \land \oE{x}{a}
            \footnote{$\dE{x}{a}$はDEの命題変数であり$x=a$を意味する.}
        \]
    \end{block}
    \begin{alertblock}{チャネリング制約を用いる利点}
        \begin{itemize}
            % \item $\Alldiff$を$\land x_i \neq x_j$の形に分解し,各々の$x_i \neq x_j$をOEとDEのいずれかを用いてSAT符号化することができる.
            \item OEで有効性が確認されている鳩の巣原理(PHP)に基づくヒント制約と$n=d$の場合にDEで有効性が確認されているat-least-one制約(ALT1)を組み合わせて利用することができる.
            % \item {\alldiff}制約を$\land x_i \neq x_j$の形に分解する代わりに,擬似ブール(PB)制約に符号化し,その後,SAT符号化することができる.
        \end{itemize}
    \end{alertblock}
\end{frame}



%%%%%%%%%%%%%%%%%%%%%%%%%%%%%%%%%%%%%%
% 鳩の巣原理を用いたヒント制約(PHP)~[田島・田村,2008]
%%%%%%%%%%%%%%%%%%%%%%%%%%%%%%%%%%%%%%
\begin{frame}
    \frametitle{鳩の巣原理を用いたヒント制約(PHP)~[田島・田村,2008]}
    SAT符号化された{\alldifferent}制約に,鳩の巣原理を用いたヒントを加える
    と求解速度が向上することが知られている.
    \begin{block}{}
        $alldifferent(x_{1},\ldots,x_{n})$について,$x_i \in
        \{\ell,\ell+1,\ldots,u\}$であるとき,以下の2つの制約を追加する.
        \[
            \bigvee_{i=1}^{n}x_{i}\geq \ell+n-1 \qquad
            \bigvee_{i=1}^{n}x_{i}\leq u-n+1
        \]
    \end{block}
    \begin{exampleblock}{例}
        $alldifferent(x_1, x_2, x_3)$について, $x_i \in \{1,2,3\}$であるとき,以下の制約が追加される.
        \vspace{-3mm}
        \begin{eqnarray*}
            (x_1\geq 3) \lor (x_2 \geq 3) \lor (x_3 \geq 3)\\
            (x_1\leq 1) \lor (x_2 \leq 1) \lor (x_3 \leq 1)
        \end{eqnarray*}
    \end{exampleblock}
\end{frame}



%%%%%%%%%%%%%%%%%%%%%%%%%%%%%%%%%%%%%%
% at-least-one制約
%%%%%%%%%%%%%%%%%%%%%%%%%%%%%%%%%%%%%%
\begin{frame}
    \frametitle{at-least-one制約を用いたヒント制約(H2)}
    \vspace{-3mm}
    \begin{block}{}
        $distinct(x_1,x_2,\ldots,x_n)$について, $x_i \in \{\ell, \ell+1,\ldots, u\}$かつ$u-\ell=n-1$であるときに以下の制約を追加する.\\
        \vspace{-3mm}
        $$\bigvee_{i=1}^n x_i=a \qquad (a \in \{\ell,\ldots, u\})$$
    \end{block}
    \begin{exampleblock}{at-least-one制約を用いたヒント制約の例}
        $distinct(x_1, x_2, x_3, x_4)$について, $x_i \in \{1, 2, 3, 4\}$であるときには以下の制約が追加される.
        \vspace{-3mm}
        \begin{eqnarray*}
            (x_1=1) \lor (x_2=1) \lor (x_3=1) \lor (x_4=1)\\
            (x_1=2) \lor (x_2=2) \lor (x_3=2) \lor (x_4=2)\\
            (x_1=3) \lor (x_2=3) \lor (x_3=3) \lor (x_4=3)\\
            (x_1=4) \lor (x_2=4) \lor (x_3=4) \lor (x_4=4)
        \end{eqnarray*}

    \end{exampleblock}
\end{frame}



%%%%%%%%%%%%%%%%%%%%%%%%%%%%%%%%%%%%%%
% 実験概要
%%%%%%%%%%%%%%%%%%%%%%%%%%%%%%%%%%%%%%
\begin{frame}
    \frametitle{実験概要}
    実装したSAT符号化及び{\alldifferent}制約の高速化手法を評価するために,以下の実験を行なった.
    \begin{itemize}
        \item \structure{比較に用いた実装(12個)}
            % \vspace{-3mm}
            % \begin{block}{}\centering
            %     {\fontsize{4pt}{5pt}\selectfont  \begin{tabular}[c] {c|c|c|c|c|c|c|c|c}
   & 整数変数の & \multicolumn{4}{|c|}{alldifferent制約の分解} & PHP & ALT1 \\\cline{3-6}
   & 符号化法   & $\neq$分解 & PB & PB3 & PB4 & & &\\\hline\hline
  0     & OE                    & OE      &    &       &             &     &     & \\
  1     & OE                    & OE      &    &       &             & OE  &     & (Sugarと同じ)\\\hline
  2     & OE$\Leftrightarrow$DE & DE      &    &       &             &     &     & \\
  3     & OE$\Leftrightarrow$DE & DE      &    &       &             &     & DE  & [Gent+ '04]\\
  4     & OE$\Leftrightarrow$DE & DE      &    &       &             & OE  &     & \\
  5     & OE$\Leftrightarrow$DE & DE      &    &       &             & OE  & DE  & \\
  6     & OE$\Leftrightarrow$DE &         & OE &       &             &     &     & \\
  7     & OE$\Leftrightarrow$DE &         & OE &       &             & OE  &     & \\
  8     & OE$\Leftrightarrow$DE &         &    & OE    &             &     &     & \\
  9     & OE$\Leftrightarrow$DE &         &    & OE    &             & OE  &     & \\
  10    & OE$\Leftrightarrow$DE &         &    &       & OE          &     &     & \\
  11    & OE$\Leftrightarrow$DE &         &    &       & OE          & OE  &     &
 \end{tabular}
}
            % \end{block}
        \item \structure{ベンチマーク問題}
            \begin{itemize}
                \item $N$次クイーングラフ彩色問題 ($8\leq N\leq 12$)
            \end{itemize}
        \item \structure{SATソルバー}: Sugar ver.2.3.3 ,GlueMiniSat 2.2.10-193
        \item \structure{制限時間}: 1問あたり2時間
        \item \structure{実験環境}: Mac mini, 3.2GHz, 64GB メモリ
    \end{itemize}
\end{frame}

%%%%%%%%%%%%%%%%%%%%%%%%%%%%%%%%%%%%%%
% 比較に用いたalldifferent制約の符号化一覧
%%%%%%%%%%%%%%%%%%%%%%%%%%%%%%%%%%%%%%
\begin{frame}
    \frametitle{比較に用いた{\alldiff}制約の符号化一覧}
    \begin{block}{}\centering
        {\tiny  \begin{tabular}[c] {c|c|c|c|c|c|c|c|c}
   & 整数変数の & \multicolumn{4}{|c|}{alldifferent制約の分解} & PHP & ALT1 \\\cline{3-6}
   & 符号化法   & $\neq$分解 & PB & PB3 & PB4 & & &\\\hline\hline
  0     & OE                    & OE      &    &       &             &     &     & \\
  1     & OE                    & OE      &    &       &             & OE  &     & (Sugarと同じ)\\\hline
  2     & OE$\Leftrightarrow$DE & DE      &    &       &             &     &     & \\
  3     & OE$\Leftrightarrow$DE & DE      &    &       &             &     & DE  & [Gent+ '04]\\
  4     & OE$\Leftrightarrow$DE & DE      &    &       &             & OE  &     & \\
  5     & OE$\Leftrightarrow$DE & DE      &    &       &             & OE  & DE  & \\
  6     & OE$\Leftrightarrow$DE &         & OE &       &             &     &     & \\
  7     & OE$\Leftrightarrow$DE &         & OE &       &             & OE  &     & \\
  8     & OE$\Leftrightarrow$DE &         &    & OE    &             &     &     & \\
  9     & OE$\Leftrightarrow$DE &         &    & OE    &             & OE  &     & \\
  10    & OE$\Leftrightarrow$DE &         &    &       & OE          &     &     & \\
  11    & OE$\Leftrightarrow$DE &         &    &       & OE          & OE  &     &
 \end{tabular}
}
    \end{block}
    \begin{itemize}
        \item 符号化1は高速制約ソルバーSugarのデフォルト設定と同じである.
        \item 符号化2-11はチャネリング制約を用いた提案手法である.
        \item 3列目以降のDE・ODは直接符号化法と順序符号化法のどちらを用いてSAT符号化するのかを表している.
    \end{itemize}
\end{frame}



%%%%%%%%%%%%%%%%%%%%%%%%%%%%%%%%%%%%%%
% 実験結果
%%%%%%%%%%%%%%%%%%%%%%%%%%%%%%%%%%%%%%
\begin{frame}
    \frametitle{実験結果: 求解に要したCPU時間(秒)}
    \begin{block}{}\centering
        {\tiny \begin{tabular}{l|rrr} 
  & 到達可能 & 到達不能 & 合計 \\ \hline
  origin & 11 & 0 & 11 \\
  changed & 11 & 10 & 21 \\
  unchanged & 11 & 10 & 21 \\
\end{tabular}}
    \end{block}
    \begin{itemize}
        \item 既存の順序符号化法のみを用いた手法(符号化 1)よりもチャネリング制約を用いた手法(符号化 2-11)の方が優れている
        \item 解を求めることが困難な$N=12$について符号化 9で求めることができた.
    \end{itemize}
\end{frame}





%%%%%%%%%%%%%%%%%%%%%%%%%%%%%%%%%%%%%%
% まとめ
%%%%%%%%%%%%%%%%%%%%%%%%%%%%%%%%%%%%%%
\begin{frame}
    \frametitle{まとめ}
    \begin{enumerate}
        \item \structure{{\alldifferent}制約に対して,12個の実装方法を提案}
            \begin{itemize}
                % \item 比較のため高速制約ソルバー Sugarのデフォルト設定の手法とチャネリング制約を用いた提案手法に{\alldiff}制約の高速化手法を合わせて実装した.
                \item 整数変数の符号化方法(OE,OE $\equ$ DE),\\
                    {\alldiff}制約の表現手法($\neq$分解,基本PB,PB3,PB4)と\\
                    ヒント制約(PHP,ALT1)を組み合わせて実装した.
            \end{itemize}
        \item \structure{クイーングラフ彩色問題($8 \leq N \leq 12$)を用いた評価実験}
            \begin{itemize}
                \item 既存の手法に比べてチャネリング制約の優位性が確認できた.
                \item N=12のクイーングラフ彩色問題において解を求めることができた.
            \end{itemize}
    \end{enumerate}
\end{frame}



%%%%%%%%%%%%%%%%%%%%%%%%%%%%%%%%%%%%%%
% 12次クイーングラフ彩色問題の解
%%%%%%%%%%%%%%%%%%%%%%%%%%%%%%%%%%%%%%
\begin{frame}
    \frametitle{12次クイーングラフ彩色問題の解}
        \begin{exampleblock}{}\centering
            \includegraphics[width=5cm]{images/qgcp_12.jpg}
        \end{exampleblock}
        実験において$N=11$で性能の良かった上位3モデルを用いて,$N=12$を制限時間1週間で
        実験した結果を示す.
\end{frame}
% %%%% 補助スライド

% 
%%%%%%%%%%%%%%%%%%%%%%%%%%%%%%%%%%%%%%%%%%%%%%%%%%%%%%%%%%%%%
% %%%% 補助スライド
%%%%%%%%%%%%%%%%%%%%%%%%%%%%%%%%%%%%%%%%%%%%%%%%%%%%%%%%%%%%%

\appendix

\backupbegin

%%%%%%%%%%%%%%%%%%%%%%%%%%%%%%%%%%%%%%
% ~
%%%%%%%%%%%%%%%%%%%%%%%%%%%%%%%%%%%%%%
\begin{frame}
    \frametitle{~}
    \centering
    - 補足用 -
\end{frame}



%%%%%%%%%%%%%%%%%%%%%%%%%%%%%%%%%%%%%%
% 鳩の巣原理を用いたヒント制約(PHP)~[田島・田村,2008]
%%%%%%%%%%%%%%%%%%%%%%%%%%%%%%%%%%%%%%
\begin{frame}
    \frametitle{鳩の巣原理を用いたヒント制約(PHP)~[田島・田村,2008]}
    SAT符号化された{\alldifferent}制約に,鳩の巣原理を用いたヒントを加える
    と求解速度が向上することが知られている.
    \begin{block}{}
        $alldifferent(x_{1},\ldots,x_{n})$について,$x_i \in
        \{\ell,\ell+1,\ldots,u\}$であるとき,以下の2つの制約を追加する.
        \[
            \bigvee_{i=1}^{n}x_{i}\geq \ell+n-1 \qquad
            \bigvee_{i=1}^{n}x_{i}\leq u-n+1
        \]
    \end{block}
    \begin{exampleblock}{例}
        $alldifferent(x_1, x_2, x_3)$について, $x_i \in \{1,2,3\}$であるとき,以下の制約が追加される.
        \vspace{-3mm}
        \begin{eqnarray*}
            (x_1\geq 3) \lor (x_2 \geq 3) \lor (x_3 \geq 3)\\
            (x_1\leq 1) \lor (x_2 \leq 1) \lor (x_3 \leq 1)
        \end{eqnarray*}
    \end{exampleblock}
\end{frame}


%%%%%%%%%%%%%%%%%%%%%%%%%%%%%%%%%%%%%%
% at-least-one制約
%%%%%%%%%%%%%%%%%%%%%%%%%%%%%%%%%%%%%%
\begin{frame}
    \frametitle{at-least-one制約を用いたヒント制約(ALT1)}
    \vspace{-3mm}
    \begin{block}{}
        $alldfifferent(x_1,x_2,\ldots,x_n)$について, $x_i \in \{\ell, \ell+1,\ldots, u\}$かつ$u-\ell=n-1$であるときに以下の制約を追加する.\\
        \vspace{-3mm}
        $$\bigvee_{i=1}^n x_i=a \qquad (a \in \{\ell,\ldots, u\})$$
    \end{block}
    \begin{exampleblock}{at-least-one制約を用いたヒント制約の例}
        $alldifferent(x_1, x_2, x_3, x_4)$について, $x_i \in \{1, 2, 3, 4\}$であるときには以下の制約が追加される.
        \vspace{-3mm}
        \begin{eqnarray*}
            (x_1=1) \lor (x_2=1) \lor (x_3=1) \lor (x_4=1)\\
            (x_1=2) \lor (x_2=2) \lor (x_3=2) \lor (x_4=2)\\
            (x_1=3) \lor (x_2=3) \lor (x_3=3) \lor (x_4=3)\\
            (x_1=4) \lor (x_2=4) \lor (x_3=4) \lor (x_4=4)
        \end{eqnarray*}
    \end{exampleblock}
\end{frame}


%%%%%%%%%%%%%%%%%%%%%%%%%%%%%%%%%%%%%%
% alldifferent制約の擬似ブール符号化
%%%%%%%%%%%%%%%%%%%%%%%%%%%%%%%%%%%%%%
\begin{frame}
    \frametitle{{\alldiff}制約の擬似ブール符号化(PB)}
    {\alldiff}制約をブール基数制約で表現することができる.
    \begin{exampleblock}{}
        $x_i \in \{ 1 \dots d \}, n \geq d$ である $\Alldiff$に対して,$p_{ij}=1 \Llra x_i=j$である$n$行$d$列の0-1行列($p_{ij}$)を導入する.
        \vspace{-3mm}
        \begin{displaymath}
            \begin{array}{cccc}
             & & &
             \begin{array}{cccc}
                 1&2&\dots&d
             \end{array}\\
                (p_{ij})&=&
                \begin{array}{c}x_1\\ x_2\\ \vdots\\ x_n \end{array}&
                \left(
                    \begin{array}{cccc}
                        p_{11}&p_{12}&\dots&p_{1d}\\
                        p_{21}&p_{22}&\dots&p_{2d}\\
                        \vdots&\vdots&\ddots&\vdots\\
                        p_{n1}&p_{n2}&\dots&p_{nd}
                \end{array}\right)
            \end{array}
        \end{displaymath}
        \begin{itemize}
            \item 各$x_i$はちょうど一つの値をとる.
            \vspace{-3mm}
                % $$ \sum_{j=1}^{d} p_{ij}=1 \; (i \in \{1,2,\ldots,n\}) $$
            $$ p_{i1} + \ldots + p_{id} = 1 \; (i \in \{1,2,\ldots,n\})$$
            \item 各列について1となるのは高々1つである.
            \vspace{-3mm}
            $$ p_{1j} + \ldots + p_{nj} \leq 1 \; (j \in \{1,2,\ldots,d\})$$
                % $$ \sum_{i=1}^{n} p_{ij} \leq 1 \; (j \in \{l,l+1,\ldots,u\})$$
                % これは$n=d$の時には等号にできる
                % $$\sum_{i=1}^{n} p_{ij} = 1 \; (j \in \{l,l+1,\ldots,u\})$$
        \end{itemize}
    \end{exampleblock}
\end{frame}

% %%%%%%%%%%%%%%%%%%%%%%%%%%%%%%%%%%%%%%
% % PB3
% %%%%%%%%%%%%%%%%%%%%%%%%%%%%%%%%%%%%%%
% \begin{frame}
%     \frametitle{PB3}
%     {\alldiff}制約をブール基数制約
%     \begin{exampleblock}{}
%         $x_i \in \{ 1 \dots d \}, n \geq d$ である $\Alldiff$に対して,$p_{ij}=1 \Llra x_i=j$である$n$行$d$列の0-1行列($p_{ij}$)を導入する.
%         \vspace{-3mm}
%         \begin{displaymath}
%             \begin{array}{cccc}
%              & & &
%              \begin{array}{cccc}
%                  1&2&\dots&d
%              \end{array}\\
%                 (p_{ij})&=&
%                 \begin{array}{c}x_1\\ x_2\\ \vdots\\ x_n \end{array}&
%                 \left(
%                     \begin{array}{cccc}
%                         p_{11}&p_{12}&\dots&p_{1d}\\
%                         p_{21}&p_{22}&\dots&p_{2d}\\
%                         \vdots&\vdots&\ddots&\vdots\\
%                         p_{n1}&p_{n2}&\dots&p_{nd}
%                 \end{array}\right)
%             \end{array}
%         \end{displaymath}
%         \begin{itemize}
%             \item 各$x_i$はちょうど一つの値をとる.
%             \vspace{-3mm}
%                 % $$ \sum_{j=1}^{d} p_{ij}=1 \; (i \in \{1,2,\ldots,n\}) $$
%             $$ p_{i1} + \ldots + p_{id} = 1 \; (i \in \{1,2,\ldots,n\})$$
%             \item 各列について1となるのは高々1つである.
%             \vspace{-3mm}
%             $$ p_{1j} + \ldots + p_{nj} \leq 1 \; (j \in \{1,2,\ldots,d\})$$
%                 % $$ \sum_{i=1}^{n} p_{ij} \leq 1 \; (j \in \{l,l+1,\ldots,u\})$$
%                 % これは$n=d$の時には等号にできる
%                 % $$\sum_{i=1}^{n} p_{ij} = 1 \; (j \in \{l,l+1,\ldots,u\})$$
%         \end{itemize}
%     \end{exampleblock}
% \end{frame}
%
\backupend

%%% Local Variables:
%%% mode: japanese-latex
%%% TeX-master: "slide"
%%% End:

\end{document}
