\documentclass [dvipdfmx,11pt]{beamer}
\usepackage{bxdpx-beamer}
\usepackage{pxjahyper}
\usepackage{amsmath}
\usepackage{bm}
\usepackage{minijs}
\usepackage{tikz}
\usepackage{multicol}
\usepackage{amssymb}
%\usepackage{otf}
%\renewcommand{\kanjifamilydefault}{\gtdefault}
%%% Beamer
%\AtBeginDvi{\special{pdf:tounicode EUC-UCS2}}
%\usetheme{Madrid}
%\usetheme{Copenhagen}
\usetheme{Warsaw}
% \renewcommand{\kanjifamilydefault}{\gtdefault}
\usefonttheme{structurebold}
%\usefonttheme{professionalfonts}
\setbeamertemplate{blocks}[shadow=true,rounded]
% \setbeamercolor{structure}{fg=blue!60!black}
\setbeamercolor{structure}{fg=blue!50!black}
\setbeamercolor{item projected}{fg=black,bg=blue!20!white}
%\setbeamercolor{alerted text}{fg=red!80!black}
\setbeamercolor{alerted text}{fg=red!70!black}
\setbeamertemplate{navigation symbols}{}
\useoutertheme[subsection=false]{miniframes}
\setbeamertemplate{footline}[frame number]
%%% Tikz
\usetikzlibrary{intersections, calc, arrows}
\setbeamertemplate{navigation symbols}{}
\setbeamertemplate{itemize item}[circle]
\setbeamersize{text margin left=1.5em,text margin right=1.5em}
\setlength{\abovedisplayskip}{0pt} % 上部のマージン
\setlength{\belowdisplayskip}{0pt} % 下部のマージン
\setlength{\columnsep}{0pt}
%
%
%
% footer setting %
\makeatother
\setbeamertemplate{footline}
{
    \leavevmode%
    \hbox{%
        \begin{beamercolorbox}[wd=.4\paperwidth,ht=2.25ex,dp=1ex,center]{author in head/foot}%
            \usebeamerfont{author in head/foot}\insertshortauthor
        \end{beamercolorbox}%
        \begin{beamercolorbox}[wd=.6\paperwidth,ht=2.25ex,dp=1ex,center]{title in head/foot}%
            \usebeamerfont{title in head/foot}\hspace*{1ex} \insertshorttitle\hspace*{2em}
            \textbf{ \insertframenumber{} / \inserttotalframenumber } \hspace*{1ex}
    \end{beamercolorbox}}%
    \vskip0pt%
}
\makeatletter
% exclude apprendix slides from framenumber %
\newcommand{\backupbegin}{
    \newcounter{framenumberappendix}
    \setcounter{framenumberappendix}{\value{framenumber}}
}
\newcommand{\backupend}{
    \addtocounter{framenumberappendix}{-\value{framenumber}}
    \addtocounter{framenumber}{\value{framenumberappendix}}
}


\usepackage{ipsj}
\usepackage{color}
\usepackage{amssymb}
\usepackage{amsmath}
\usepackage{amsthm}
\usepackage{multirow,bigdelim}
\newcommand{\la}{\leftarrow}
\newcommand{\Lra}{\Longrightarrow}
\newcommand{\Lla}{\Longleftarrow}
\newcommand{\Llra}{\Longleftrightarrow}
\newcommand{\lra}{\longrightarrow}
\newcommand{\dd}{\mathop{..}}
\newcommand{\range}[2]{\{#1\dd#2\}}
\newcommand{\imp}{\Rightarrow}
\newcommand{\equ}{\Leftrightarrow}
\renewcommand{\labelenumi}{(\arabic{enumi})}
\newcommand{\alldiff}{\textrm{alldifferent}}
\newcommand{\Alldiff}{\alldiff(x_1,x_2,\ldots,x_n)}
\newcommand{\SAT}{{\tt SAT}}
\newcommand{\UNSAT}{{\tt UNSAT}}
\newcommand{\Dom}{{\it Dom}}
% \newcommand{\p}[2]{p(#1,#2)}
\newcommand{\dE}[2]{p(#1=#2)}
\newcommand{\lE}[2]{p(#1^{(#2)})}
\newcommand{\oE}[2]{p(#1\le#2)}

%%%%%%%%%%%% my macro %%%%%%%%%%%%%%%%%
\newcommand{\alldifferent}{$alldifferent$}
%%%%%%%%%%%%%%%%%%%%%%%%%%%%%%%%%%%%%%%


\title{チャネリング制約を用いた\\ alldifferent 制約の SAT 符号化}
% \author{小菅脩司}
% \institute{名古屋大学}
\author{小菅 脩司\inst{1} \and 宋 剛秀\inst{2} \and 田村 直之\inst{2} \and 番原 睦則\inst{1}}
\institute{ \inst{1}名古屋大学 \ \  \inst{2}神戸大学 }
\date{情報処理学会第84回全国大会}
\begin{document}
\begin{frame} {}
    \titlepage
\end{frame}
%%%%%%%%%%%%%%%%%%%%%%%%%%%%%%%%%%%%%%


%%%%%%%%%%%%%%%%%%%%%%%%%%%%%%%%%%%%%%
% alldifferent制約
%%%%%%%%%%%%%%%%%%%%%%%%%%%%%%%%%%%%%%
\begin{frame}
  \frametitle{{\alldifferent}制約}
  \begin{alertblock}{}\centering
    \alert{\bm{$alldifferent(x_{1},x_{2},\ldots, x_{n})$}}\\[1em]
    \begin{itemize}
    \item 整数上の変数$x_{i}$の値が互いに異なることを表す制約である.
    \item 制約プログラミングにおける代表的なグローバル制約の一つである.
    \end{itemize}
  \end{alertblock}
  \begin{itemize}
  \item この制約は,
    $$\bigwedge_{1 \leq i < j \leq n} x_i \neq x_j$$
    を意味する.
  \item {\alldifferent}制約は,時間割問題,グラフ彩色問題,組合せデザイン
    など様々な制約充足問題に現れる.
  \item {\alldifferent}制約を効率良く解くことは重要
    な研究課題である.
  \end{itemize}
\end{frame}
%%%%%%%%%%%%%%%%%%%%%%%%%%%%%%%%%%%%%%
% alldifferent制約の適用例
%%%%%%%%%%%%%%%%%%%%%%%%%%%%%%%%%%%%%%
\begin{frame}
  \frametitle{{\alldifferent}制約の適用例}
  \begin{block}{クイーングラフ彩色問題}
    $N$個ずつの$N$個のグループからなるクイーン (計$N^2$個) を,
    $N\times N$のチェス盤に,同じグループのクイーン同士が互いに取られ
    ないように配置する問題
  \end{block}
  \begin{exampleblock}{}\centering
    \begin{columns}
      \begin{column}{0.45\textwidth}\centering
        \includegraphics[width=3cm]{images/qgcp_5.jpg}
      \end{column}
      \begin{column}{0.45\textwidth}\centering
        \includegraphics[width=3cm]{images/qgcp_5_c.jpg}
      \end{column}
    \end{columns}
  \end{exampleblock}
  \begin{itemize}
  \item この問題は{\alldifferent}制約のみを用いて記述できる.
  \item Knuthの教科書 TAOCP でも取り上げられている~[Knuth '15].
  % \item 2005 年までに,11 以上 26 以下のすべての N に対し解が発見され
  %   たが,現時点でも $N = 27$の解が存在するかどうかは未解決である.
  \end{itemize}
\end{frame}
%%%%%%%%%%%%%%%%%%%%%%%%%%%%%%%%%%%%%%
% alldifferent制約の実装技術
%%%%%%%%%%%%%%%%%%%%%%%%%%%%%%%%%%%%%%
\begin{frame}
\frametitle{{\alldifferent}制約の既存研究}

\begin{itemize}
\item グラフ理論に基づくアーク整合性アルゴリズム[R\'{e}gin '94]
  \begin{itemize}
  \item R\'{e}gin のアルゴリズムに対して様々な改良が提案.
  \item サーベイ論文~[Gent '08]
  \end{itemize}
\item \alert{\bf SAT 符号化を用いた手法}
  \begin{itemize}
  \item 直接符号化法~[de Kleer '89, Walsh '00]
  \item 順序符号化法~[Crawford+ '94, 田村+ '06, 田村+ '09]
  \item ラダー符号化法~[Gent+ '04]
  \item 区間符号化法~[Bessiere+ '09]
  \end{itemize}
\item PB 符号化を用いた手法~[大野+ '19]
  \begin{itemize}
  \item alldifferent 制約を擬似ブール(Pseudo-Boolean; PB)制約に符号化.
  \item PB 制約をSATに符号化して解くことも可能.
  \end{itemize}
\end{itemize}
\end{frame}

%\begin{frame}
%   \frametitle{{\alldifferent}制約の実装に関する既存研究}
%   % alldifferent制約はおそらく制約プログラミングにおいて最もよく知られ,最も影響力があり,最も研究されているグローバル制約である.\\
%   % これはおそらくマッチング理論との関係によるもので,
%   % この理論計算機科学の重要な分野はalldifferent制約に対する効率的なフィルタリングアルゴリズムの基礎を提供した.\\
%   マッチング理論はalldifferent制約に対する効率的なフィルタリングアルゴリズムの基礎を提供した.\\

%   \begin{block}{マッチング理論に基づく{\alldiff}制約に対するアーク整合性アルゴリズム[Régin]}
%     変数の集合$X$,それらのドメインの和を$D(X)$とし,\\
%     二部グラフ$G = (X, D(X), E), E = \{\{x, d\} | x \in X, d \in D(x)\}$を表す.
%     \[
%       x_1 \in \{b, c, d, e\}, x_2 \in \{b, c\}, x_3 \in \{a, b, c, d\}, x_4 \in \{b, c\},
%     \]
%     \[
%       alldifferent(x_1, x_2, x_3, x_4)
%     \]
%     を考える.
%   \end{block}
% \end{frame}


%%%%%%%%%%%%%%%%%%%%%%%%%%%%%%%%%%%%%%
% 研究目的
%%%%%%%%%%%%%%%%%%%%%%%%%%%%%%%%%%%%%%
\begin{frame}
  \frametitle{研究目的}
  \begin{alertblock}{研究目的}\centering
    {\alldifferent}制約を効率よく解くために,新しい SAT 符号化法を考案・実装・評価する.
  \end{alertblock}
  \vfill
  \begin{itemize}
  \item \structure{方針}: 順序符号化法をベースに研究開発を進める.
  \item \structure{理由}: 順序符号化法に基づく SAT型 制約ソルバー 
    Sugar および Fun-sCOP が,国際制約ソルバー競技会で優勝するなど優れ
    た性能を示している.
  \end{itemize}
  \vfill  
  \begin{block}{研究内容}
    \begin{enumerate}
    \item \alert{\bf チャネリング制約を導入し},順序符号化法と直接符号
      化法を結合した SAT 符号化法を考案した.
    \item クイーングラフ彩色問題を用いて,提案手法を評価した.
    \end{enumerate}
  \end{block}
\end{frame}
%%%%%%%%%%%%%%%%%%%%%%%%%%%%%%%%%%%%%%
% alldifferent制約の直接符号化
%%%%%%%%%%%%%%%%%%%%%%%%%%%%%%%%%%%%%%
\begin{frame}
  \frametitle{直接符号化法を用いた手法~{\footnotesize [de Kleer '89, Walsh '00]}}
  \begin{block}{直接符号化法(Direct Encoding; DE)}
    各整数変数$x$と各整数定数$a \in \Dom(x)$に対して,
    $x = a$を意味する命題変数\alert{\bf $\dE{x}{a}$}を用いる.
  \end{block}
  \begin{exampleblock}{$alldifferent(x_{1},x_{2},\ldots, x_{n})$, $x_i\in \{\ell,\dots,u\}$の符号化}\small
    \[
      \begin{array}[t]{ccc}
        \bm{\displaystyle\bigwedge_{1 \leq i < j \leq n} \lnot\dE{x_i}{a} \lor \lnot \dE{x_j}{a}}
        & (a \in \{\ell,\ldots, u\}) & (\neq\textrm{分解})\\
        \bm{\displaystyle\bigvee_{i=1}^n \dE{x_i}{a}}
        & (a \in \{\ell,\ldots, u\}) & \textsf{(ALT1)}\\
        & (if\ n=u-\ell+1)& 
      \end{array}
    \]
  \end{exampleblock}

  \begin{alertblock}{利点}
    \begin{itemize}
    \item ($\neq$分解)する代わりに,{\alldifferent}制約のPB符号化を利用できる.
    \end{itemize}
  \end{alertblock}
\end{frame}
%%%%%%%%%%%%%%%%%%%%%%%%%%%%%%%%%%%%%%
% 順序符号化
%%%%%%%%%%%%%%%%%%%%%%%%%%%%%%%%%%%%%%
\begin{frame}
  \frametitle{順序符号化法を用いた手法~{\footnotesize [田村+ '06, 田村+ '09]}}
  \begin{block}{順序符号化法(Order Encoding; OE)}
    各整数変数$x$と各整数定数$a \in \Dom(x)$に対して,
    $x \le a$を意味する命題変数\alert{\bf $\oE{x}{a}$}を用いる.
  \end{block}
  \begin{exampleblock}{$alldifferent(x_{1},x_{2},\ldots, x_{n})$, $x_i\in \{\ell,\dots,u\}$の符号化}\small
    \[
      \begin{array}[t]{ccc}
        \bm{\displaystyle\bigwedge_{1 \leq i < j \leq n} x_i > x_j \lor x_i < x_j}
        & \textrm{と変換したのち符号化} & (\neq\textrm{分解})\\
        \bm{\displaystyle\bigvee_{i=1}^{n} \lnot\oE{x_i}{\ell+n-2}},
        &  \bm{\displaystyle\bigvee\limits_{i=1}^{n} \oE{x_i}{u-n+1}}& \textsf{(PHP)}
      \end{array}
    \]
  \end{exampleblock}

  \begin{alertblock}{利点}
    \begin{itemize}
    \item 鳩の巣原理を応用したヒント制約(PHP)が利用できる.
    \item 互いに異なる$n$変数が$n-1$のサイズのドメインに入らないことを表す~[田島+ '07]
    \end{itemize}
  \end{alertblock}  
\end{frame}

%%%%%%%%%%%%%%%%%%%%%%%%%%%%%%%%%%%%%%
% チャネリング制約を用いたalldifferent制約のSAT符号化
%%%%%%%%%%%%%%%%%%%%%%%%%%%%%%%%%%%%%%
\begin{frame}
  \frametitle{(提案) チャネリング制約を用いた手法}
  \begin{alertblock}{基本的アイデア}
    各整数変数$x$と各整数定数$a \in \Dom(x)$に対して,OE と DE の両方
    の命題変数を導入し,それらの変数をつなぐ\alert{\bf チャネリング制約}
    を追加する\footnotemark[1].
    \[
      \bm{\dE{x}{a} \Llra \lnot\oE{x}{a-1} \land \oE{x}{a}}
    \]
  \end{alertblock}
  \vfill
  \begin{block}{長所と短所}
    \begin{itemize}
    \item \structure{長所}: OE と DE の双方の長所をもつ
      \begin{itemize}
      \item {\alldifferent}制約を($\neq$分解)する代わりに,PB 符号化を利用できる.
      \item ヒント制約(ALT1)と(PHP)を同時に利用できる.
      \end{itemize}
    \item \structure{短所}: SAT符号化後の節数が増える.
    \end{itemize}
  \end{block}
  \footnotetext[1]{$a-1 \notin \Dom(x)$の場合は,$\lnot\oE{x}{a-1}$は省略}
\end{frame}

%%%%%%%%%%%%%%%%%%%%%%%%%%%%%%%%%%%%%%
% 実験概要
%%%%%%%%%%%%%%%%%%%%%%%%%%%%%%%%%%%%%%
\begin{frame}
  \frametitle{実験概要}

  \begin{block}\centering
    提案手法の有効性を評価するために,以下の実験を行なった.
  \end{block}

  \vfill

  \begin{itemize}
  \item \structure{比較に用いた{\alldifferent}制約の符号化法(計12個)}
  \item \structure{ベンチマーク問題}
    \begin{itemize}
    \item $N$次クイーングラフ彩色問題
    \item $8\leq N\leq 12$
    \end{itemize}
  \item \structure{PB符号化}: 論文~[大野+ '19]のPB, PB3, PB4 の3つを使用
  \item \structure{SATソルバー}: GlueMiniSat 2.2.10-193
  \item \structure{制限時間}: 1問あたり2時間
    \begin{itemize}
    \item $N=12$については,$N=11$で性能のよかった上位3つの符号化法に
      対して,制限時間を72時間に延ばして実験を行った.
    \end{itemize}
  \item \structure{実験環境}: Mac mini, 3.2GHz, 64GB メモリ
  \end{itemize}
\end{frame}

%%%%%%%%%%%%%%%%%%%%%%%%%%%%%%%%%%%%%% 
% 比較に用いたalldifferent制約の符号化一覧
%%%%%%%%%%%%%%%%%%%%%%%%%%%%%%%%%%%%%%
\begin{frame}
    \frametitle{比較に用いた{\alldifferent}制約の符号化法}
    \begin{block}{}\centering\footnotesize\scriptsize
       \begin{tabular}[c] {c|c|c|c|c|c|c|c|c}
   & 整数変数の & \multicolumn{4}{|c|}{alldifferent制約の分解} & PHP & ALT1 \\\cline{3-6}
   & 符号化法   & $\neq$分解 & PB & PB3 & PB4 & & &\\\hline\hline
  0     & OE                    & OE      &    &       &             &     &     & \\
  1     & OE                    & OE      &    &       &             & OE  &     & (Sugarと同じ)\\\hline
  2     & OE$\Leftrightarrow$DE & DE      &    &       &             &     &     & \\
  3     & OE$\Leftrightarrow$DE & DE      &    &       &             &     & DE  & [Gent+ '04]\\
  4     & OE$\Leftrightarrow$DE & DE      &    &       &             & OE  &     & \\
  5     & OE$\Leftrightarrow$DE & DE      &    &       &             & OE  & DE  & \\
  6     & OE$\Leftrightarrow$DE &         & OE &       &             &     &     & \\
  7     & OE$\Leftrightarrow$DE &         & OE &       &             & OE  &     & \\
  8     & OE$\Leftrightarrow$DE &         &    & OE    &             &     &     & \\
  9     & OE$\Leftrightarrow$DE &         &    & OE    &             & OE  &     & \\
  10    & OE$\Leftrightarrow$DE &         &    &       & OE          &     &     & \\
  11    & OE$\Leftrightarrow$DE &         &    &       & OE          & OE  &     &
 \end{tabular}

    \end{block}
    \begin{itemize}
        \item 0と1は,既存の順序符号化法のみを使用.
        \item 1は,高速SAT型制約ソルバー Sugar と同じ.
        \item 2--11は,チャネリング制約を用いた提案手法.
%        \item 符号化3は先行研究であるI.Gentのものと同じ設定である.
    \end{itemize}
\end{frame}

%%%%%%%%%%%%%%%%%%%%%%%%%%%%%%%%%%%%%%
% 実験結果
%%%%%%%%%%%%%%%%%%%%%%%%%%%%%%%%%%%%%%
\begin{frame}
  \frametitle{実験結果: 求解に要したCPU時間(秒)}
  \begin{block}{}\centering
    \tiny
    \begin{tabular}{l|rrr} 
  & 到達可能 & 到達不能 & 合計 \\ \hline
  origin & 11 & 0 & 11 \\
  changed & 11 & 10 & 21 \\
  unchanged & 11 & 10 & 21 \\
\end{tabular}
  \end{block}
  \begin{itemize}
  \item 既存の順序符号化法(0--1)と比較して,チャネリング制約を用
    いた提案手法は優れた性能を示している.
  \item クイーングラフ彩色問題の SAT 解法について様々な方法を調
    べた論文~[田村+'17]でも成功していない$N=12$を解くことができた.
  \end{itemize}
\end{frame}


%%%%%%%%%%%%%%%%%%%%%%%%%%%%%%%%%%%%%%
% まとめ
%%%%%%%%%%%%%%%%%%%%%%%%%%%%%%%%%%%%%%
\begin{frame}
  \frametitle{まとめ}

  \begin{itemize}
  \item {\alldifferent}制約の SAT 符号化について,チャネリング制約を導
    入し,既存の順序符号化法と直接符号化法を結合した SAT 符号化法を考案
    した.
  \item 提案手法を比較・評価するために,$N=8$ から $N=12$ までの
    クイーングラフ彩色問題を用いた実験を行った.
  \item 既存の順序符号化法と比較して,チャネリング制約を用いた提案手法
    の優位性が確認できた.
  \end{itemize}

\begin{exampleblock}{今後の課題}
  \begin{itemize}
  \item 提案手法を高速SAT型制約ソルバーSugarに組み込む.
  \item {\alldifferent}制約で記述できる他の問題での実験.
  \end{itemize}
\end{exampleblock}
\end{frame}

%%%%%%%%%%%%%%%%%%%%%%%%%%%%%%%%%%%%%% 
% 12次クイーングラフ彩色問題の解
%%%%%%%%%%%%%%%%%%%%%%%%%%%%%%%%%%%%%%
\begin{frame}
    \frametitle{12次クイーングラフ彩色問題の解}
    \begin{exampleblock}{}\centering
        \includegraphics[width=5cm]{images/qgcp_12.jpg}
    \end{exampleblock}
\end{frame}
% %%%% 補助スライド


%%%%%%%%%%%%%%%%%%%%%%%%%%%%%%%%%%%%%%%%%%%%%%%%%%%%%%%%%%%%%
% %%%% 補助スライド
%%%%%%%%%%%%%%%%%%%%%%%%%%%%%%%%%%%%%%%%%%%%%%%%%%%%%%%%%%%%%

\appendix

\backupbegin

%%%%%%%%%%%%%%%%%%%%%%%%%%%%%%%%%%%%%%
% ~
%%%%%%%%%%%%%%%%%%%%%%%%%%%%%%%%%%%%%%
\begin{frame}
    \frametitle{~}
    \centering
    - 補足用 -
\end{frame}



%%%%%%%%%%%%%%%%%%%%%%%%%%%%%%%%%%%%%%
% 鳩の巣原理を用いたヒント制約(PHP)~[田島・田村,2008]
%%%%%%%%%%%%%%%%%%%%%%%%%%%%%%%%%%%%%%
\begin{frame}
    \frametitle{鳩の巣原理を用いたヒント制約(PHP)~[田島・田村,2008]}
    SAT符号化された{\alldifferent}制約に,鳩の巣原理を用いたヒントを加える
    と求解速度が向上することが知られている.
    \begin{block}{}
        $alldifferent(x_{1},\ldots,x_{n})$について,$x_i \in
        \{\ell,\ell+1,\ldots,u\}$であるとき,以下の2つの制約を追加する.
        \[
            \bigvee_{i=1}^{n}x_{i}\geq \ell+n-1 \qquad
            \bigvee_{i=1}^{n}x_{i}\leq u-n+1
        \]
    \end{block}
    \begin{exampleblock}{例}
        $alldifferent(x_1, x_2, x_3)$について, $x_i \in \{1,2,3\}$であるとき,以下の制約が追加される.
        \vspace{-3mm}
        \begin{eqnarray*}
            (x_1\geq 3) \lor (x_2 \geq 3) \lor (x_3 \geq 3)\\
            (x_1\leq 1) \lor (x_2 \leq 1) \lor (x_3 \leq 1)
        \end{eqnarray*}
    \end{exampleblock}
\end{frame}


%%%%%%%%%%%%%%%%%%%%%%%%%%%%%%%%%%%%%%
% at-least-one制約
%%%%%%%%%%%%%%%%%%%%%%%%%%%%%%%%%%%%%%
\begin{frame}
    \frametitle{at-least-one制約を用いたヒント制約(ALT1)}
    \vspace{-3mm}
    \begin{block}{}
        $alldfifferent(x_1,x_2,\ldots,x_n)$について, $x_i \in \{\ell, \ell+1,\ldots, u\}$かつ$u-\ell=n-1$であるときに以下の制約を追加する.\\
        \vspace{-3mm}
        $$\bigvee_{i=1}^n x_i=a \qquad (a \in \{\ell,\ldots, u\})$$
    \end{block}
    \begin{exampleblock}{at-least-one制約を用いたヒント制約の例}
        $alldifferent(x_1, x_2, x_3, x_4)$について, $x_i \in \{1, 2, 3, 4\}$であるときには以下の制約が追加される.
        \vspace{-3mm}
        \begin{eqnarray*}
            (x_1=1) \lor (x_2=1) \lor (x_3=1) \lor (x_4=1)\\
            (x_1=2) \lor (x_2=2) \lor (x_3=2) \lor (x_4=2)\\
            (x_1=3) \lor (x_2=3) \lor (x_3=3) \lor (x_4=3)\\
            (x_1=4) \lor (x_2=4) \lor (x_3=4) \lor (x_4=4)
        \end{eqnarray*}
    \end{exampleblock}
\end{frame}


%%%%%%%%%%%%%%%%%%%%%%%%%%%%%%%%%%%%%%
% alldifferent制約の擬似ブール符号化
%%%%%%%%%%%%%%%%%%%%%%%%%%%%%%%%%%%%%%
\begin{frame}
    \frametitle{{\alldiff}制約の擬似ブール符号化(PB)}
    {\alldiff}制約をブール基数制約で表現することができる.
    \begin{exampleblock}{}
        $x_i \in \{ 1 \dots d \}, n \geq d$ である $\Alldiff$に対して,$p_{ij}=1 \Llra x_i=j$である$n$行$d$列の0-1行列($p_{ij}$)を導入する.
        \vspace{-3mm}
        \begin{displaymath}
            \begin{array}{cccc}
             & & &
             \begin{array}{cccc}
                 1&2&\dots&d
             \end{array}\\
                (p_{ij})&=&
                \begin{array}{c}x_1\\ x_2\\ \vdots\\ x_n \end{array}&
                \left(
                    \begin{array}{cccc}
                        p_{11}&p_{12}&\dots&p_{1d}\\
                        p_{21}&p_{22}&\dots&p_{2d}\\
                        \vdots&\vdots&\ddots&\vdots\\
                        p_{n1}&p_{n2}&\dots&p_{nd}
                \end{array}\right)
            \end{array}
        \end{displaymath}
        \begin{itemize}
            \item 各$x_i$はちょうど一つの値をとる.
            \vspace{-3mm}
                % $$ \sum_{j=1}^{d} p_{ij}=1 \; (i \in \{1,2,\ldots,n\}) $$
            $$ p_{i1} + \ldots + p_{id} = 1 \; (i \in \{1,2,\ldots,n\})$$
            \item 各列について1となるのは高々1つである.
            \vspace{-3mm}
            $$ p_{1j} + \ldots + p_{nj} \leq 1 \; (j \in \{1,2,\ldots,d\})$$
                % $$ \sum_{i=1}^{n} p_{ij} \leq 1 \; (j \in \{l,l+1,\ldots,u\})$$
                % これは$n=d$の時には等号にできる
                % $$\sum_{i=1}^{n} p_{ij} = 1 \; (j \in \{l,l+1,\ldots,u\})$$
        \end{itemize}
    \end{exampleblock}
\end{frame}

% %%%%%%%%%%%%%%%%%%%%%%%%%%%%%%%%%%%%%%
% % PB3
% %%%%%%%%%%%%%%%%%%%%%%%%%%%%%%%%%%%%%%
% \begin{frame}
%     \frametitle{PB3}
%     {\alldiff}制約をブール基数制約
%     \begin{exampleblock}{}
%         $x_i \in \{ 1 \dots d \}, n \geq d$ である $\Alldiff$に対して,$p_{ij}=1 \Llra x_i=j$である$n$行$d$列の0-1行列($p_{ij}$)を導入する.
%         \vspace{-3mm}
%         \begin{displaymath}
%             \begin{array}{cccc}
%              & & &
%              \begin{array}{cccc}
%                  1&2&\dots&d
%              \end{array}\\
%                 (p_{ij})&=&
%                 \begin{array}{c}x_1\\ x_2\\ \vdots\\ x_n \end{array}&
%                 \left(
%                     \begin{array}{cccc}
%                         p_{11}&p_{12}&\dots&p_{1d}\\
%                         p_{21}&p_{22}&\dots&p_{2d}\\
%                         \vdots&\vdots&\ddots&\vdots\\
%                         p_{n1}&p_{n2}&\dots&p_{nd}
%                 \end{array}\right)
%             \end{array}
%         \end{displaymath}
%         \begin{itemize}
%             \item 各$x_i$はちょうど一つの値をとる.
%             \vspace{-3mm}
%                 % $$ \sum_{j=1}^{d} p_{ij}=1 \; (i \in \{1,2,\ldots,n\}) $$
%             $$ p_{i1} + \ldots + p_{id} = 1 \; (i \in \{1,2,\ldots,n\})$$
%             \item 各列について1となるのは高々1つである.
%             \vspace{-3mm}
%             $$ p_{1j} + \ldots + p_{nj} \leq 1 \; (j \in \{1,2,\ldots,d\})$$
%                 % $$ \sum_{i=1}^{n} p_{ij} \leq 1 \; (j \in \{l,l+1,\ldots,u\})$$
%                 % これは$n=d$の時には等号にできる
%                 % $$\sum_{i=1}^{n} p_{ij} = 1 \; (j \in \{l,l+1,\ldots,u\})$$
%         \end{itemize}
%     \end{exampleblock}
% \end{frame}
%
\backupend

%%% Local Variables:
%%% mode: japanese-latex
%%% TeX-master: "slide"
%%% End:

\end{document}

%%% Local Variables:
%%% mode: japanese-latex
%%% TeX-master: t
%%% End:
