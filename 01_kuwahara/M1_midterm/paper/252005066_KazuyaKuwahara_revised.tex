\documentclass[a4j,10pt]{jarticle}
\usepackage{multicol}
\usepackage{caption}
%%%%% 余白の設定 %%%%%%
\setlength{\textheight}{\paperheight}
\setlength{\topmargin}{-15.4truemm}
\addtolength{\topmargin}{-\headheight}
\addtolength{\topmargin}{-\headsep}
\addtolength{\textheight}{-20truemm}
\setlength{\textwidth}{\paperwidth}
\setlength{\oddsidemargin}{-10.4truemm}
\setlength{\evensidemargin}{\oddsidemargin}
\addtolength{\textwidth}{-30truemm}
%%%%% 行間の設定 %%%%%
\renewcommand{\baselinestretch}{.95}
\newcommand{\lw}[1]{\smash{\lower2.ex\hbox{#1}}} % added by banbara
\pagestyle{empty}

%%%%% 題目 %%%%%
\title{解集合プログラミングに基づく\\系統的探索と確率的局所探索の統合的手法}
%%%%% 氏名 %%%%%
\author{桑原 和也(番原研究室)}

\date{\today}
\begin{document}
\maketitle
\thispagestyle{empty}
\begin{multicols}{1}
%%%%% 本文ここから %%%%%

%%%%%%%%%%%%%%%%%%%%%%%%%%%%%%%%%%%%%%%%%%%%%%%%%%%%%%%%%%%%%%%%
\section{研究概要}
%%%%%%%%%%%%%%%%%%%%%%%%%%%%%%%%%%%%%%%%%%%%%%%%%%%%%%%%%%%%%%%%
\textbf{解集合プログラミング}(Answer Set Programming; ASP)は,
論理プログラミングから派生した宣言的プログラミングパラダイムである.
ASP 言語は,命題表現に限られる SAT を拡張し,述語表現を許した宣言的知
識表現言語である.
ASP ソルバーは安定モデル意味論に基づく解集合を計算するプログラムである.
近年,SAT ソルバーの実装技術を応用した高速 ASP ソルバーが実現され,AI
分野の諸問題への実用的応用が急速に拡大している.

最近では,ASP の適用範囲を,プランニング,モデル検査,多目的最適化問題
などへ拡張する研究も進められている.
応用についても,時間割問題などの求解困難な組合せ最適化問題に対し,
系統的探索であることを活かして,未解決問題の最適値決定に成功している.
しかし,これまでの研究のほとんどは,ASP の系統的探索をベースにしたもの
であり,問題規模に対するスケーラビリティに限界があるのも事実である.

本研究では,ASP 技術を用いて,系統的探索と確率的局所探索
を統合的かつ効率的に扱う探索手法およびソルバーの実現を目指す.
組合せ最適化問題に対する近似解法の一種である
\textbf{巨大近傍探索} (Large Neighborhood Search; LNS~\cite{Psinger_10})
のアイディアをベースに,ASP に適した探索手法に関する研究開発を進める.
時間割問題等の重要な応用研究を通じて,提案手法およびソルバーを評価する.

%%%%%%%%%%%%%%%%%%%%%%%%%%%%%%%%%%%%%%%%%%%%%%%%%%%%%%%%%%%%%%%%
\section{研究成果}
%%%%%%%%%%%%%%%%%%%%%%%%%%%%%%%%%%%%%%%%%%%%%%%%%%%%%%%%%%%%%%%%
Pisinger の LNS は,解に含まれる変数の値割当ての一部をランダムに選んで
取り消し,その変数のみに対して再割当てを行うことで解を再構築する反復解
法である.
先行研究の
\textbf{優先度付き巨大近傍探索}(Large Neighborhood Prioritized Search; LNPS)
は,解の再構築の操作を,値割当てをなるべく維持したままでの再探索に置き
換えることで,取り消されなかった変数への再割当てを許す.これによって,
どの値割当てを取り消すかに依存しすぎない探索を行うことができる.
また,反復の終了条件を適切に設定することで解の最適性も保証できる.

高速 ASP ソルバー \textsf{clingo}上に,LNPS アルゴリズムを実装した.
\textsf{clingo} の探索ヒューリスティックの機能を利用することにより,
値割当てをなるべく維持したままでの再探索 (解の再構築) が,系統的探索で
自然に実現できる.

国際時間割競技会で使用されたカリキュラムベース・コース時間割の問題集を
用いて実験を行なった(下表参照).
その結果,提案手法は,21問中20問について,既存 ASP 解法と同じかより良
い解を生成した.さらに,
comp09 について,既知の最良値を更新することに成功した.

%----------------------------------------------
\begin{center}\footnotesize
\renewcommand{\arraystretch}{1.2}
\tabcolsep = 1.0mm
\begin{tabular}{c|r|rrrrrr|r}
\lw{問題名} & 既存 & \multicolumn{6}{c|}{提案手法} & 既知の\\
       & ASP  & R-0 & R-3 & DP & DR & SR-5 & SR-10 & 最良値\\\hline
comp01 & 129 & 13 & 13 & 13 & 11 & 11 & 11 & 11\\
comp02 & 331 & 334 & 239 &172 & 242 & 201 & 245 & 130\\
comp03 & 302 & 173 & 177 & 149 & 173 & 146 & 157 & 142\\
comp04 & 49 & 49 & 49 & 49 & 49 & 49 & 49 & 49\\
comp05 & 1,940 & 1,124 & 922 & 926 & 1116 & 891 & 861 & 570\\
comp06 & 822 & 220 & 166 & 140 & 204 & 118 & 106 & 85\\
comp07 & 924 & 225 & 131 & 118 & 129 & 72 & 77 & 42\\
comp08 & 55 & 55 & 55 & 55 & 55 & 55 & 55 & 55\\
comp09 & 254 & 155 & 154 & \textbf{149} & \textbf{146} & \textbf{145} & 151 & 150\\
comp10 & 822 & 231 & 220 & 167 & 133 & 97 & 109 & 72\\
comp11 & 0 & 0 & 0 & 0 & 0 & 0 & 0 & 0\\
comp12 & 1,246 & 834 & 740 & 705 & 694 & 756 & 809 & 483\\
comp13 & 301 & 183 & 180 & 168 & 165 & 151 & 155 & 147\\
comp14 & 67 & 189 & 186 & 179 & 165 & 104 & 83 & 67\\
comp15 & 607 & 244 & 238 & 234 & 238 & 215 & 224 & 176\\
comp16 & 944 & 197 & 156 & 180 & 162 & 223 & 158 & 96\\
comp17 & 412 & 254 & 226 & 230 & 234 & 199 & 208 & 155\\
comp18 & 471 & 191 & 170 & 152 & 158 & 144 & 149 & 137\\
comp19 & 890 & 231 & 192 & 187 & 219 & 176 & 163 & 125\\
comp20 & 1,386 & 373 & 274 & 280 & 305 & 293 & 265 & 124\\
comp21 & 310 & 285 & 219 & 222 & 235 & 192 & 178 & 151\\\hline
\end{tabular}
\end{center}
%----------------------------------------------

%%%%%%%%%%%%%%%%%%%%%%%%%%%%%%%%%%%%%%%%%%%%%%%%%%%%%%%%%%%%%%%%
\section{まとめと今後の課題}
%%%%%%%%%%%%%%%%%%%%%%%%%%%%%%%%%%%%%%%%%%%%%%%%%%%%%%%%%%%%%%%%
ASP 基づく系統的探索と確率的局所探索の統合的手法について,
研究概要とこれまでの研究成果を述べた.
今後の課題としては,
\textsf{clingo}の最新 Python インターフェイスを用いた LNPS の実装,
他の時間割問題への適用などが挙げられる.

%%%%%参考文献 %%%%%
\begin{thebibliography}{10}
\bibitem{Psinger_10} 
David Pisinger and Stefan Ropke. 
Large Neighborhood Search. 
In Handbook of Metaheuristics,
pp.399--419, Springer, 2010.
\end{thebibliography}

%%%%% 本文ここまで %%%%%
\end{multicols}
\vfill
\noindent
{\gt コメント欄}
{\footnotesize
(本資料をそのまま発表者に返却します.コメント欄以外にもコメントを書いていただいてもかまいません.)}
\\
\fbox{\begin{minipage}{\textwidth}\noindent\\\\\end{minipage}}	
\end{document}

%%% Local Variables:
%%% mode: japanese-latex
%%% TeX-master: t
%%% End:
