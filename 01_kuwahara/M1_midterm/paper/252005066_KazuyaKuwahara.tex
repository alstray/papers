\documentclass[a4j,10pt]{jarticle}
\usepackage{multicol}
\usepackage{caption}
%%%%% 余白の設定 %%%%%%
\setlength{\textheight}{\paperheight}
\setlength{\topmargin}{-15.4truemm}
\addtolength{\topmargin}{-\headheight}
\addtolength{\topmargin}{-\headsep}
\addtolength{\textheight}{-20truemm}
\setlength{\textwidth}{\paperwidth}
\setlength{\oddsidemargin}{-10.4truemm}
\setlength{\evensidemargin}{\oddsidemargin}
\addtolength{\textwidth}{-30truemm}
%%%%% 行間の設定 %%%%%
\renewcommand{\baselinestretch}{.95}

\pagestyle{empty}

%%%%% 題目 %%%%%
\title{解集合プログラミングに基づく系統的探索と確率的局所探索の統合的手法}
%%%%% 氏名 %%%%%
\author{桑原 和也(番原研究室)}

\date{\today}
\begin{document}
\maketitle
\thispagestyle{empty}
\begin{multicols}{1}
%%%%% 本文ここから %%%%%

\section{研究の概要}
本研究の目的は,
ASP 技術を用いて,
任意の組み合わせ問題を効率良く
解くシステムを実現することである.

\textbf{解集合プログラミング (Answer Set Programming; ASP)}
は,論理プログラミングから
派生した宣言的プログラミングパラダイムである.
ASP 言語は一階論理に基づく知識表現言語の一種である.
ASP システムは論理プログラムから
安定モデル意味論に基づく
解集合を計算するシステムである.
SAT 技術を応用した高速 ASP システムが実現され,
様々な分野への実用的応用が急速に拡大している.

本研究では,
ASP 技術を用い,
組合せ最適化問題に対して系統的探索と確率的局所探索を
統合的に適用する手法である
\textbf{優先度付き巨大近傍探索 (Large Neighborhood Prioritized Search; LNPS)}
を提案する.

LNPS は,
近似解法の一種である
\textbf{巨大近傍探索 (Large Neighborhood Search; LNS)}
のアイデアをベースにしている.
LNS は解に含まれる変数の値割当ての一部をランダムに選んで取り消し,
その変数のみに対して再割当てを行うことで
解を再構築する反復解法である\cite{Psinger_10}.
LNPS では,解の再構築の操作を,
値割当てをなるべく維持したままでの再探索に置き換えることで,
取り消されなかった変数への再割当てを許す.
これによって,
どの値割当てを取り消すかに依存しすぎない探索を行うことができる.

\section{現在の成果}
卒業研究では,
提案する LNPS の実行をするプログラムを一つに
統合できていなかった為,
新たに実装し直した.
LNPS の実装にはプログラミング言語 Python を用い,
実行には 本研究で使用する
ASP システムである
\textit{clingo} の Python インターフェースを利用した.
最適化を行う任意の ASP プログラムに適用して,
LNPS による探索を可能にするシステムを実現した.

LNPS の性能に重要な destroy 演算について,
卒業研究から新たに実装した.

表1に実行実験を行った結果を示す.
ベンチマーク問題には,
組合せ最適化問題の中から時間割問題を取り上げ,
時間割競技会の問題集を使用した.
左の列から順に,
問題名,既存 ASP 解法によって得られた値,
各種実装した destroy 演算を
使用した LNPS によって得られた値となっている.
提案手法では1問を除いて,
既存 ASP 解法より良いか同じ値が得られている.
また,comp09 については新たに実装した destroy 演算
によって既知の最適値の更新に成功した.

%----------------------------------------------
\begin{minipage}{1\linewidth}
 \centering
\captionof{table}{実験結果}
 \renewcommand{\arraystretch}{0.7}
 \tabcolsep = 0.9mm
 \begin{tabular}{c||r|r|r|r|r|r|r|r}
問題 & 既存 & R-0 & R-3 & R-5 & DP & DR & SR-5 & SR-10\\\hline
comp01 & 129 & 13 & 13 & 11 & 13 & 11 & 11 & 11\\
comp02 & 331 & 334 & 239 & 273 &172 & 242 & 201 & 245\\
comp03 & 302 & 173 & 177 & 154 & 149 & 173 & 146 & 157\\
comp04 & 49 & 49 & 49 & 49 & 49 & 49 & 49 & 49\\
comp05 & 1940 & 1124 & 922 & 797 & 926 & 1116 & 891 & 861\\
comp06 & 822 & 220 & 166 & 135 & 140 & 204 & 118 & 106\\
comp07 & 924 & 225 & 131 & 137 & 118 & 129 & 72 & 77\\
comp08 & 55 & 55 & 55 & 55 & 55 & 55 & 55 & 55\\
comp09 & 254 & 155 & 154 & 158 & 149 & 146 & 145 & 151\\
comp10 & 822 & 231 & 220 & 177 & 167 & 133 & 97 & 109\\
comp11 & 0 & 0 & 0 & 0 & 0 & 0 & 0 & 0\\
comp12 & 1246 & 834 & 740 & 728 & 705 & 694 & 756 & 809\\
comp13 & 301 & 183 & 180 & 163 & 168 & 165 & 151 & 155\\
comp14 & 67 & 189 & 186 & 145 & 179 & 165 & 104 & 83\\
comp15 & 607 & 244 & 238 & 213 & 234 & 238 & 215 & 224\\
comp16 & 944 & 197 & 156 & 178 & 180 & 162 & 223 & 158\\
comp17 & 412 & 254 & 226 & 259 & 230 & 234 & 199 & 208\\
comp18 & 471 & 191 & 170 & 168 & 152 & 158 & 144 & 149\\
comp19 & 890 & 231 & 192 & 197 & 187 & 219 & 176 & 163\\
comp20 & 1386 & 373 & 274 & 356 & 280 & 305 & 293 & 265\\
comp21 & 310 & 285 & 219 & 202 & 222 & 235 & 192 & 178\\\hline
 \end{tabular}
 \label{tab:result}
\end{minipage}
%----------------------------------------------

\section{まとめと今後の課題}
ASP 技術を用いて任意の最適化問題を解くシステムとして,
系統的探索と確率的局所探索を統合した LNPS を提案し,
その成果を述べた.

今後の課題としては,
新たな destroy 演算の実装や,
他の組合せ最適化問題への適用などが
挙げられる.

%%%%% 参考文献 %%%%%
\begin{thebibliography}{10}

\bibitem{Psinger_10} 
David Pisinger and Stefan Ropke. 
Large neighborhood search. 
In Handbook of metaheuristics, pp. 399–419. Springer, 2010.

\end{thebibliography}

%%%%% 本文ここまで %%%%%
\end{multicols}
\vfill
\noindent
{\gt コメント欄}
{\footnotesize
(本資料をそのまま発表者に返却します.コメント欄以外にもコメントを書いていただいてもかまいません.)}
\\
\fbox{\begin{minipage}{\textwidth}\noindent\\\\\end{minipage}}	
\end{document}
