\section{おわりに}

本論文では,
系統的探索と確率的局所探索を統合的に適用する手法として,
優先度付き巨大近傍探索(LNPS)を提案した.
提案手法 LNPS を 高速 ASP ソルバー{\clingo}上に実装し,
国際時間割競技会の CB-CTT 問題集(全21問)を用いて性能評価を行った.
その結果,提案手法は,通常の ASP 解法と比較して,多くの問題に対して
より精度の高い解を得ることができた.
さらに,1問について,既知の最良値を更新することに成功した.
今後の課題としては,
アダプティブ LNPS への拡張や他の時間割問題への適用などが挙げられる.


%%% Local Variables:
%%% mode: japanese-latex
%%% TeX-master: "paper"
%%% End:
