\section{ASP ソルバー 上での LNPS の実装}
%%%%%%%%%%%%%%%%%%%%%%%%%%%%%%%%%
  \thicklines
  \setlength{\unitlength}{1.28pt}
  \small
  \begin{picture}(280,57)(4,-10)
    \put(  0, 20){\dashbox(50,24){\shortstack{根付き全域森\\問題}}}
    \put( 60, 20){\framebox(50,24){変換器}}
    \put(120, 20){\dashbox(50,24){\shortstack{ASPファクト}}}
    \put(120,-10){\alert{\bf\dashbox(50,24){\scriptsize{\shortstack{ASP符号化\\(論理プログラム)}}}}}
    \put(180, 20){\framebox(50,24){ASPシステム}}
    \put(240, 20){\dashbox(50,24){\shortstack{根付き全域森\\問題の解}}}
    \put( 50, 32){\vector(1,0){10}}
    \put(110, 32){\vector(1,0){10}}
    \put(170, 32){\vector(1,0){10}}
    \put(230, 32){\vector(1,0){10}}
    \put(170, +2){\line(1,0){4}}
    \put(174, +2){\line(0,1){30}}
  \end{picture}  

%%%%%%%%%%%%%%%%%%%%%%%%%%%%%%%%%
提案手法 LNPS を 高速 ASP ソルバー{\clingo}上に実装した.
提案ソルバーは,
ASP ファクト形式の問題インスタンスと
問題を解く ASP 符号化を入力とし,
LNPS アルゴリズムを用いて解を求める(図~\ref{fig:arch}参照).
%
提案ソルバーは,{\clingo}の Python インターフェースを利用して実装され
ている.以下に,LNPS を実装する上で重要な役割を果たすASP 技術について
簡単にまとめる.

\textbf{探索ヒューリスティックス.}
ASP ソルバー{\clingo}では,\code{#heuristic}文を用いて,
探索ヒューリスティックスをカスタマイズすることができる\footnote{%
{\clingo}のデフォルト探索ヒューリスティックスは,
SAT ソルバーと同じ VSIDS (Variable State Independent Decaying Sum)}.
以下に例を示す.

%%%%%%%%%%%%%%%%%%%%%%%%%%%%%%%%%
\lstinputlisting[float=h,caption={%
\code{\#heuristic}文の例 (\code{heu.lp})},%
captionpos=b,frame=single,label=code:heu.lp,%
numbers=none,%
breaklines=true,%
columns=fullflexible,keepspaces=true,%
basicstyle=\ttfamily\footnotesize]{code/heu.lp}
%%%%%%%%%%%%%%%%%%%%%%%%%%%%%%%%% 
\lstinputlisting[float=t,caption={%
\code{\#heuristic}文を無効にした実行例},%
captionpos=b,frame=single,label=code:heu.log,%
numbers=none,%
breaklines=true,%
columns=fullflexible,keepspaces=true,%
basicstyle=\ttfamily\footnotesize]{code/heu.log}
%%%%%%%%%%%%%%%%%%%%%%%%%%%%%%%%% 
\lstinputlisting[float=t,caption={%
\code{\#heuristic}文を有効にした実行例},%
captionpos=b,frame=single,label=code:heu2.log,%
numbers=none,%
breaklines=true,%
columns=fullflexible,keepspaces=true,%
basicstyle=\ttfamily\footnotesize]{code/heu2.log}
%%%%%%%%%%%%%%%%%%%%%%%%%%%%%%%%%

1行目のルールは,選択子を使ってアトム
\code{a}と\code{b}を導入している.
2行目のルールは,\code{a}と\code{b}が同時に成り立たないことを表している.
このプログラム例の解集合は
\code{\{\}},\code{\{a\}},\code{\{b\}}の3つである.
3行目の \code{#heuristic}文は,
優先度1で\code{a}に真を割当てることを表している.
同様に,4行目は
優先度2で\code{b}に真を割当てることを表している.
アトムに対するデフォルトの優先度は0である.
%
コード~\ref{code:heu.log}に \code{#heuristic}文を無効にした実行例,
コード~\ref{code:heu2.log}に有効にした実行例を示す.
これら実行例より,\code{#heuristic}文を使うことによって,
解の探索において優先度の高いアトムから順に値が割当てられている
ことがわかる.

LNPS の実装では,この \code{#heuristic}文を利用して,
$destroy$で値割当てを取り消されなかった変数に対して,
優先的に同じ値を割当てる.
これにより,値割当てをなるべく維持したままでの解の再探索
($re\mathchar`-search$)を{\clingo}の系統的探索を用いて
自然に実装できる.

\textbf{マルチショット ASP 解法.}
ASP ソルバー{\clingo}は,
同様の探索失敗を防ぐために獲得した学習節を保持することで,無駄な探索を
行うことなく,ルールを追加・削除したプログラムを連続的に解くことができ
る.このマルチショット ASP 解法は,別名インクリメンタル ASP 解法とも呼
ばれ,モデル検査やプランニングに利用されている.

LNPS の実装では,この解法を利用することで,
$destroy$と$re\mathchar`-search$の反復実行を簡潔に実装できる.
動的なルールの削除は,\code{#external}文を利用して実装されるが,
詳細は省略する.

%%% Local Variables:
%%% mode: latex
%%% TeX-master: "paper"
%%% End:
