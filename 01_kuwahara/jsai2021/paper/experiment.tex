\section{評価実験}
提案手法の有効性を評価するために,実行実験を行った.
ベンチマーク問題には,国際時間割競技会 ITC2007 の comp01〜21の
全21問を使用し,
ソフト制約の多い問題集(UD5)を使用した.
実験に使用した手法としては,
既存 ASP 解法のみでの実行に加え,
LNPS と 各種 destroy 演算を加えたもので行い,
destroy 演算は,
random $N$ ($N$ = 0, 3, 5),
day-period,
day-room,
swap-room $N$ ($N$ = 5, 10)を選択した.
表 2 は,
それぞれの手法で得られた最適値・最良値を示している.
左の列から順に,
問題名,既存 ASP 解法によって得られた値,各種提案手法によって得られた値
となっている.
各問題ごとに実験した手法の中で最も良い値が得られたものは太字で示してある.
それぞれの手法で良い値が得られた問題数を比較すると,
順に swap-room 5,swap-room 10 が良い性能を示している.
表 3 では別のアプローチとの比較を行っている.
既知の最良値は先行研究によって得られた解の上界を表しており,
提案ベストの値は,
各種提案手法の中で最も良かった値としている.
それぞれ最良値との比では,
既知の最良値を 100\% として,
得られた解での増減を表している.
既存 ASP 解法では +437\% と 5 倍以上となっているが,
提案手法では +26\%にまで抑えられている.
また comp09 については -3\% と既知の最良値を更新できた.

\begin{table*}[tbp]
  \label{table:bench:result1}
  \begin{center}
  \caption{得られた最適値・最良値}
\begin{tableA}
    {comp01} & 129 & 13 & 13 & \bf{11} & 13 & \bf{11} & \bf{11} & \bf{11}\\
{comp02} & 331 & 334 & 239 & 273 & \bf{172} & 242 & 201 & 245\\
{comp03} & 302 & 173 & 177 & 154 & 149 & 173 & \bf{146} & 157\\
{comp04} & ${}^\ast$\bf{49} & ${}^\ast$\bf{49} & ${}^\ast$\bf{49} & ${}^\ast$\bf{49} & ${}^\ast$\bf{49} & ${}^\ast$\bf{49} & ${}^\ast$\bf{49} & ${}^\ast$\bf{49}\\
{comp05} & 1940 & 1124 & 922 & \bf{797} & 926 & 1116 & 891 & 861\\
{comp06} & 822 & 220 & 166 & 135 & 140 & 204 & 118 & \bf{106}\\
{comp07} & 924 & 225 & 131 & 137 & 118 & 129 & \bf{72} & 77\\
{comp08} & ${}^\ast$\bf{55} & ${}^\ast$\bf{55} & ${}^\ast$\bf{55} & ${}^\ast$\bf{55} & ${}^\ast$\bf{55} & ${}^\ast$\bf{55} & ${}^\ast$\bf{55} & ${}^\ast$\bf{55}\\
{comp09} & 254 & 155 & 154 & 158 & 149 & 146 & \bf{145} & 151\\
{comp10} & 822 & 231 & 220 & 177 & 167 & 133 & \bf{97} & 109\\
{comp11} & ${}^\ast$\bf{0} & ${}^\ast$\bf{0} & ${}^\ast$\bf{0} & ${}^\ast$\bf{0} & ${}^\ast$\bf{0} & ${}^\ast$\bf{0} & ${}^\ast$\bf{0} & ${}^\ast$\bf{0}\\
{comp12} & 1246 & 834 & 740 & 728 & 705 & \bf{694} & 756 & 809\\
{comp13} & 301 & 183 & 180 & 163 & 168 & 165 & \bf{151} & 155\\
{comp14} & ${}^\ast$\bf{67} & 189 & 186 & 145 & 179 & 165 & 104 & 83\\
{comp15} & 607 & 244 & 238 & \bf{213} & 234 & 238 & 215 & 224\\
{comp16} & 944 & 197 & \bf{156} & 178 & 180 & 162 & 223 & 158\\
{comp17} & 412 & 254 & 226 & 259 & 230 & 234 & \bf{199} & 208\\
{comp18} & 471 & 191 & 170 & 168 & 152 & 158 & \bf{144} & 149\\
{comp19} & 890 & 231 & 192 & 197 & 187 & 219 & 176 & \bf{163}\\
{comp20} & 1386 & 373 & 274 & 356 & 280 & 305 & 293 & \bf{265}\\
{comp21} & 310 & 285 & 219 & 202 & 222 & 235 & 192 & \bf{178}\\\hline
{\#最適値・最良値} & 4 & 3 & 4 & 6 & 4 & 5 & \bf{11} & 8\\

%%% Local Variables:
%%% mode: japanese-latex
%%% TeX-master: "../paper"
%%% End:

  \end{tableA}
  \end{center}
\end{table*}

\begin{table*}[tbp]
  \label{table:bench:result2}
  \centering
   \caption{既知の最良値との比較}
  \begin{tableB}
    {comp01} & 11 & 129 & +1,072 & 11 & 0\\
{comp02} & 130 & 331 & +154 & 172 & +32\\
{comp03} & 142 & 302 & +112 & 146 & +2\\
{comp04} & 49 & 49 & 0 & 49 & 0\\
{comp05} & 570 & 1,940 & +240 & 797 & +39\\
{comp06} & 85 & 822 & +867 & 106 & +24\\
{comp07} & 42 & 924 & +2,100 & 72 & +71\\
{comp08} & 55 & 55 & 0 & 55 & 0\\
{comp09} & 150 & 254 & +69 & 145 & \bf{-3}\\
{comp10} & 72 & 822 & +1,041 & 97 & +34\\
{comp11} & 0 & 0 & 0 & 0 & 0\\
{comp12} & 483 & 1,246 & +157 & 694 & +43\\
{comp13} & 147 & 301 & +104 & 151 & +2\\
{comp14} & 67 & 67 & 0 & 83 & +23\\
{comp15} & 176 & 607 & +244 & 213 & +21\\
{comp16} & 96 & 944 & +883 & 156 & +62\\
{comp17} & 155 & 412 & +165 & 199 & +28\\
{comp18} & 137 & 471 & +243 & 144 & +5\\
{comp19} & 125 & 890 & +612 & 163 & +30\\
{comp20} & 124 & 1,386 & +1,017 & 265 & +113\\
{comp21} & 151 & 310 & +105 & 178 & +17\\\hline
{$\sharp$との比の平均} & & & +437 & & +26\\
  \end{tableB}
\end{table*}