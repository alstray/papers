\section{評価実験}

%%%%%%%%%%%%%%%%%%%%%%%%%%%%%%%%%%%%%%%%%%%%%%
\begin{table*}[t]\centering
  \caption{実験結果: 得られた最適値と最良値}
  \vskip 1em
  \label{table:bench:result1}
  \begin{tableA}
    {comp01} & 129 & 13 & 13 & \alert{\bf 11} & 13 & \alert{\bf 11} & \alert{\bf 11} & \alert{\bf 11}\\
{comp02} & 331 & 334 & 239 & 273 & \alert{\bf 172} & 242 & 201 & 245\\
{comp03} & 302 & 173 & 177 & 154 & 149 & 173 & \alert{\bf 146} & 157\\
{comp04} & *\alert{\bf 49} & *\alert{\bf 49} & *\alert{\bf 49} & *\alert{\bf 49} & *\alert{\bf 49} & *\alert{\bf 49} & *\alert{\bf 49} & *\alert{\bf 49}\\
{comp05} & 1940 & 1124 & 922 & \alert{\bf 797} & 926 & 1116 & 891 & 861\\
{comp06} & 822 & 220 & 166 & 135 & 140 & 204 & 118 & \alert{\bf 106}\\
{comp07} & 924 & 225 & 131 & 137 & 118 & 129 & \alert{\bf 72} & 77\\
{comp08} & *\alert{\bf 55} & *\alert{\bf 55} & *\alert{\bf 55} & *\alert{\bf 55} & *\alert{\bf 55} & *\alert{\bf 55} & *\alert{\bf 55} & *\alert{\bf 55}\\
{comp09} & 254 & 155 & 154 & 158 & 149 & 146 & \alert{\bf 145} & 151\\
{comp10} & 822 & 231 & 220 & 177 & 167 & 133 & \alert{\bf 97} & 109\\
{comp11} & *\alert{\bf 0} & *\alert{\bf 0} & *\alert{\bf 0} & *\alert{\bf 0} & *\alert{\bf 0} & *\alert{\bf 0} & *\alert{\bf 0} & *\alert{\bf 0}\\
{comp12} & 1246 & 834 & 740 & 728 & 705 & \alert{\bf 694} & 756 & 809\\
{comp13} & 301 & 183 & 180 & 163 & 168 & 165 & \alert{\bf 151} & 155\\
{comp14} & *\alert{\bf 67} & 189 & 186 & 145 & 179 & 165 & 104 & 83\\
{comp15} & 607 & 244 & 238 & \alert{\bf 213} & 234 & 238 & 215 & 224\\
{comp16} & 944 & 197 & \alert{\bf 156} & 178 & 180 & 162 & 223 & 158\\
{comp17} & 412 & 254 & 226 & 259 & 230 & 234 & \alert{\bf 199} & 208\\
{comp18} & 471 & 191 & 170 & 168 & 152 & 158 & \alert{\bf 144} & 149\\
{comp19} & 890 & 231 & 192 & 197 & 187 & 219 & 176 & \alert{\bf 163}\\
{comp20} & 1386 & 373 & 274 & 356 & 280 & 305 & 293 & \alert{\bf 265}\\
{comp21} & 310 & 285 & 219 & 202 & 222 & 235 & 192 & \alert{\bf 178}\\\hline
{最適(良)値の数} & 4 & 3 & 4 & 6 & 4 & 5 & \alert{\bf 11} & 8\\
  \end{tableA}
\end{table*}
%%%%%%%%%%%%%%%%%%%%%%%%%%%%%%%%%%%%%%%%%%%%%%
\begin{table*}[t]\centering
  \caption{他のアプローチとの比較}
  \vskip 1em  
  \label{table:bench:result2}
  \begin{tableB}
    {comp01} & 11 & 129 & +1072.73\% & 11 & +0.00\%\\
{comp02} & 130 & 331 & +154.62\% & 172 & +32.31\%\\
{comp03} & 142 & 302 & +112.68\% & 146 & +2.82\%\\
{comp04} & 49 & 49 & +0.00\% & 49 & +0.00\%\\
{comp05} & 570 & 1940 & +240.35\% & 797 & +39.82\%\\
{comp06} & 85 & 822 & +867.06\% & 106 & +24.71\%\\
{comp07} & 42 & 924 & +2100.00\% & 72 & +71.43\%\\
{comp08} & 55 & 55 & +0.00\% & 55 & +0.00\%\\
{comp09} & 150 & 254 & +69.33\% & 145 & \alert{-3.33\%}\\
{comp10} & 72 & 822 & +1041.67\% & 97 & +34.72\%\\
{comp11} & 0 & 0 & +0.00\% & 0 & +0.00\%\\
{comp12} & 483 & 1246 & +157.97\% & 694 & +43.69\%\\
{comp13} & 147 & 301 & +104.76\% & 151 & +2.72\%\\
{comp14} & 67 & 67 & +0.00\% & 83 & +23.88\%\\
{comp15} & 176 & 607 & +244.89\% & 213 & +21.02\%\\
{comp16} & 96 & 944 & +883.33\% & 156 & +62.50\%\\
{comp17} & 155 & 412 & +165.81\% & 199 & +28.39\%\\
{comp18} & 137 & 471 & +243.80\% & 144 & +5.11\%\\
{comp19} & 125 & 890 & +612.00\% & 163 & +30.40\%\\
{comp20} & 124 & 1386 & +1017.74\% & 265 & +113.71\%\\
{comp21} & 151 & 310 & +105.30\% & 178 & +17.88\%\\\hline
{比の平均} & & & +437.81\% & & +26.27\%\\
  \end{tableB}
\end{table*}
%%%%%%%%%%%%%%%%%%%%%%%%%%%%%%%%%%%%%%%%%%%%%%

提案手法の有効性を評価するために実行実験を行った.
ベンチマーク問題には,
国際時間割競技会ITC2007~\footnote{%
  \url{http://www.cs.qub.ac.uk/itc2007/}}
で公開されているカリキュラムベース・コース時間割(CB-CTT)の
問題集(全21問)を使用し,
ソフト制約が最も多いUD5~\cite{DBLP:journals/anor/BonuttiCGS12}
で評価を行なった.
%
比較した手法は以下の2つである.
\begin{itemize}\compress
\item 既存手法: ASPソルバー{\clingo}
\item 提案手法: LNPSを{\clingo}上に実装
\end{itemize}

CB-CTT 問題を解くための ASP 符号化には,
\textsf{teaspoon}符号化~\cite{anor/Banbara2019}
を使用した.
LNPS の$destroy$と$re\mathchar`-search$については,
CB-CTTの既存研究~\cite{anor/Kiefer2017}を参考に,
以下の4種類を実装した.
\begin{itemize}\compress
\item \textsf{Random} $N$ (\textsf{R-$N$})\\
  暫定解から変数の値割当ての$N$\%をランダムに選んで取り消す.
\item \textsf{Day-Period} (\textsf{DP})\\
  曜日\textsf{D}と時限\textsf{P}をランダムに1組選び,
  暫定解から\textsf{D}曜\textsf{P}限に関する
  変数の値割当てをすべて取り消す.
\item \textsf{Day-Room} (\textsf{DR})\\
  曜日\textsf{D}と教室\textsf{R}をランダムに1組選び,
  暫定解から\textsf{D}曜日の\textsf{R}教室に関する
  変数の値割当てをすべて取り消す.
\item \textsf{Swap-Room} $N$ (\textsf{SR-$N$})\\
  \textsf{Random} $N$ と同様に,
  暫定解から変数の値割当ての$N$\%をランダムに選んで取り消す.
  ただし,科目に対する曜日と時限の割当ては,できるだけ維持する.
\end{itemize}

ASPソルバーには{\clingo}-5.4.0を利用し,
1問あたりの制限時間は1時間とした.
実験環境は,Mac OS, 3.2GHz Intel Core i7, 64GB メモリである.

表~\ref{table:bench:result1}に,各手法で得られた最適値および最良値を示す.
各問題ごとに最も良い値を太字で表している.
`$\ast$'付きの値は最適値を意味する.
提案手法は,1問を除くすべての問題に対して,既存 ASP と同じかより良い解
を得ている.
各問題に対する最適値および最良値を求めた数を比較すると,
\textsf{SR-5}が11問と最も多く,次いで
\textsf{SR-10}が8問と,
\textsf{Swap-Room}が優れた性能を示した.


表~\ref{table:bench:result2}に,他のアプローチとの比較結果を示す.
左から順に,
問題名,
既知の最良値($\sharp$),
既存 ASP の最良値と$\sharp$との比,
提案ベストの最良値と$\sharp$との比が示されている.
既知の最良値は,これまで,メタ戦略に基づく各種アルゴリズム,整数計画法,
SAT, MaxSAT, SMT など様々な手法で求められた結果である.
提案ベストとは,各問題に対して,表~\ref{table:bench:result1}中の提案
LNPS で得られた最も良い値を示している.
また,既知の最良値との比は,以下の計算結果を百分率で表したものである.
\[
既知の最良値との比 = \frac{得られた最良値 - 既知の最良値}{既知の最良値}
\]

既存 ASP は,既知の最良値との比が$+437\%$と高く,解の精度が悪い.
提案手法は,既知の最良値との比が$+26\%$まで抑えられており,
既存 ASP と比較して,解の精度が大きく改善されている.
さらに,提案手法は,comp09 について,既知の最良値を更新することに成功
した.

%%% Local Variables:
%%% mode: japanese-latex
%%% TeX-master: "paper"
%%% End:
