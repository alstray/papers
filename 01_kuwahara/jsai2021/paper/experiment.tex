\section{評価実験}
提案手法の有効性を評価するために,実行実験を行った.
ベンチマーク問題には,国際時間割競技会 ITC2007 の comp01〜21の
全21問を使用し,
ソフト制約の多い問題集(UD5)を使用した.
実験に使用した手法としては,
既存 ASP 解法のみでの実行に加え,
LNPS と 各種 destroy 演算を加えたもので行い,
destroy 演算は,
random $N$ ($N$ = 0, 3, 5),
day-period,
day-room,
swap-room $N$ ($N$ = 5, 10)を選択した.


\begin{table*}[tbp]
  \label{table:bench:result1}
  \begin{center}
  \caption{得られた最適値・最良値}
\begin{tableA}
    {comp01} & 129 & 13 & 13 & \bf{11} & 13 & \bf{11} & \bf{11} & \bf{11}\\
{comp02} & 331 & 334 & 239 & 273 & \bf{172} & 242 & 201 & 245\\
{comp03} & 302 & 173 & 177 & 154 & 149 & 173 & \bf{146} & 157\\
{comp04} & ${}^\ast$\bf{49} & ${}^\ast$\bf{49} & ${}^\ast$\bf{49} & ${}^\ast$\bf{49} & ${}^\ast$\bf{49} & ${}^\ast$\bf{49} & ${}^\ast$\bf{49} & ${}^\ast$\bf{49}\\
{comp05} & 1940 & 1124 & 922 & \bf{797} & 926 & 1116 & 891 & 861\\
{comp06} & 822 & 220 & 166 & 135 & 140 & 204 & 118 & \bf{106}\\
{comp07} & 924 & 225 & 131 & 137 & 118 & 129 & \bf{72} & 77\\
{comp08} & ${}^\ast$\bf{55} & ${}^\ast$\bf{55} & ${}^\ast$\bf{55} & ${}^\ast$\bf{55} & ${}^\ast$\bf{55} & ${}^\ast$\bf{55} & ${}^\ast$\bf{55} & ${}^\ast$\bf{55}\\
{comp09} & 254 & 155 & 154 & 158 & 149 & 146 & \bf{145} & 151\\
{comp10} & 822 & 231 & 220 & 177 & 167 & 133 & \bf{97} & 109\\
{comp11} & ${}^\ast$\bf{0} & ${}^\ast$\bf{0} & ${}^\ast$\bf{0} & ${}^\ast$\bf{0} & ${}^\ast$\bf{0} & ${}^\ast$\bf{0} & ${}^\ast$\bf{0} & ${}^\ast$\bf{0}\\
{comp12} & 1246 & 834 & 740 & 728 & 705 & \bf{694} & 756 & 809\\
{comp13} & 301 & 183 & 180 & 163 & 168 & 165 & \bf{151} & 155\\
{comp14} & ${}^\ast$\bf{67} & 189 & 186 & 145 & 179 & 165 & 104 & 83\\
{comp15} & 607 & 244 & 238 & \bf{213} & 234 & 238 & 215 & 224\\
{comp16} & 944 & 197 & \bf{156} & 178 & 180 & 162 & 223 & 158\\
{comp17} & 412 & 254 & 226 & 259 & 230 & 234 & \bf{199} & 208\\
{comp18} & 471 & 191 & 170 & 168 & 152 & 158 & \bf{144} & 149\\
{comp19} & 890 & 231 & 192 & 197 & 187 & 219 & 176 & \bf{163}\\
{comp20} & 1386 & 373 & 274 & 356 & 280 & 305 & 293 & \bf{265}\\
{comp21} & 310 & 285 & 219 & 202 & 222 & 235 & 192 & \bf{178}\\\hline
{\#最適値・最良値} & 4 & 3 & 4 & 6 & 4 & 5 & \bf{11} & 8\\

%%% Local Variables:
%%% mode: japanese-latex
%%% TeX-master: "../paper"
%%% End:

  \end{tableA}
  \end{center}
\end{table*}

\begin{table*}[tbp]
  \label{table:bench:result2}
  \centering
   \caption{既知の最良値を1とした場合の比}
  \begin{tableA}
    {comp01} & 11.73 & 1.18 & 1.18 & \alert{\bf 1.00} & 1.18 & \alert{\bf 1.00} & \alert{\bf 1.00} & \alert{\bf 1.00}\\
{comp02} & 2.55 & 2.57 & 1.84 & 2.10 & \alert{\bf 1.32} & 1.86 & 1.55 & 1.88\\
{comp03} & 32.13 & 1.22 & 1.25 & 1.08 & 1.05 & 1.22 & \alert{\bf 1.03} & 1.11\\
{comp04} & \alert{\bf 1.00} & \alert{\bf 1.00} & \alert{\bf 1.00} & \alert{\bf 1.00} & \alert{\bf 1.00} & \alert{\bf 1.00} & \alert{\bf 1.00} & \alert{\bf 1.00}\\
{comp05} & 3.40 & 1.97 & 1.62 & \alert{\bf 1.40} & 1.62 & 1.96 & 1.56 & 1.51\\
{comp06} & 9.67 & 2.59 & 1.95 & 1.59 & 1.65 & 2.40 & 1.39 & \alert{\bf 1.25}\\
{comp07} & 22.00 & 5.36 & 3.12 & 3.26 & 2.81 & 3.07 & \alert{\bf 1.71} & 1.83\\
{comp08} & \alert{\bf 1.00} & \alert{\bf 1.00} & \alert{\bf 1.00} & \alert{\bf 1.00} & \alert{\bf 1.00} & \alert{\bf 1.00} & \alert{\bf 1.00} & \alert{\bf 1.00}\\
{comp09} & 1.69 & 1.03 & 1.03 & 1.05 & 0.99 & \alert{\bf 0.97} & \alert{\bf 0.97} & 1.01\\
{comp10} & 11.42 & 3.21 & 3.06 & 2.46 & 2.32 & 1.85 & \alert{\bf 1.35} & 1.51\\
{comp11} & \alert{\bf 1.00} & \alert{\bf 1.00} & \alert{\bf 1.00} & \alert{\bf 1.00} & \alert{\bf 1.00} & \alert{\bf 1.00} & \alert{\bf 1.00} & \alert{\bf 1.00}\\
{comp12} & 2.58 & 1.73 & 1.53 & 1.51 & 1.46 & \alert{\bf 1.44} & 1.57 & 1.67\\
{comp13} & 2.05 & 1.24 & 1.22 & 1.11 & 1.14 & 1.12 & \alert{\bf 1.03} & 1.05\\
{comp14} & \alert{\bf 1.00} & 2.82 & 2.78 & 2.16 & 2.67 & 2.46 & 1.55 & 1.24\\
{comp15} & 3.45 & 1.39 & 1.35 & \alert{\bf 1.21} & 1.33 & 1.35 & 1.22 & 1.27\\
{comp16} & 9.83 & 2.05 & \alert{\bf 1.63} & 1.85 & 1.88 & 1.69 & 2.32 & 1.65\\
{comp17} & 2.66 & 1.64 & 1.46 & 1.67 & 1.48 & 1.51 & \alert{\bf 1.28} & 1.34\\
{comp18} & 3.44 & 1.39 & 1.24 & 1.23 & 1.11 & 1.15 & \alert{\bf 1.05} & 1.09\\
{comp19} & 7.12 & 1.85 & 1.54 & 1.58 & 1.50 & 1.75 & 1.41 & \alert{\bf 1.30}\\
{comp20} & 11.18 & 3.01 & 2.21 & 2.87 & 2.26 & 2.46 & 2.36 & \alert{\bf 2.14}\\
{comp21} & 2.05 & 1.89 & 1.45 & 1.34 & 1.47 & 1.56 & 1.27 & \alert{\bf 1.18}\\\hline
{平均} & 5.38& 1.96 & 1.64 & 1.59 & 1.54 & 1.61 & 1.36 & \alert{\bf 1.34}\\

  \end{tableA}
\end{table*}