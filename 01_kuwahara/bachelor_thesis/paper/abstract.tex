%%%%%%%%%%%%%%%%%%%%%%%%%%%%%%%%%%%%%%%%%%%%%%%%%%%%%%%%%% 
\chapter*{概要}
\pagenumbering{roman}
%%%%%%%%%%%%%%%%%%%%%%%%%%%%%%%%%%%%%%%%%%%%%%%%%%%%%%%%%% 


%現状では,質の高い時間割を編成するために多くの人間の労力が費やされている.

%解集合プログラミング (ASP) は,
%論理プログラミングから派生した宣言的プログラミングパラダイムの一つである.
%ASP 言語は一階論理に基づく知識表現言語の一種であり,論理プログラムは
%ASP のルールの有限集合である.ASP システムは論理プログラムから安定モデ
%ル意味論に基づく解集合を計算するシステムである.

時間割問題は, 
求解困難な組合せ最適化問題の一種である. 
カリキュラムベース・コース時間割 
(Curriculum-based Course Timetabling; CB-CTT) 
問題は, 大学等の1週間の講義スケジュールを求める問題であり, 
最も研究が盛んな教育時間割問題の一つである. 
近年, 解集合プログラミング(Answer Set Programming; ASP)
を用いたCB-CTT問題の解法が提案され, 成功を収めている. 
ASP は系統的探索であることを活かして,
未解決問題の最適値を決定するなど
優れた性能を示している.  
しかし, その一方で, 制約が多く含まれるような問題集において,
局所的探索を用いた解法より性能が劣っている場合が見られる.

この問題を解決するために,系統的探索と局所的探索を組み合わせた
Large Neighborhood Prioritized Search (LNPS)
が提案されている.
LNPS は,暫定解に含まれる変数の値割り当ての一部をランダムに選んで取り
消し,他の値割り当てをなるべく維持したままで解を再構築する反復法の一種
である.
LNPS の性能は,
暫定解の一部をランダムに選んで取り消す destroy 演算子に依存するが,
十分な研究がなされてない.

本論文では,LNPS を用いた CB-CTT 問題の解法について述べる.
CB-CTT に対する既存研究を応用して,
3種類の destroy 演算子 (random, day-period, day-room) をASP上に実装した.
random が問題の性質をまったく利用しないのに対し,
day-period と day-room は CB-CTT のソフト制約を考慮して
暫定解の一部をランダムに選んで取り消す点が特長である.
%
提案手法の有効性を評価するために,国際時間割競技会 ITC-2007 のベンチマー
ク問題(21問)を用いて実行実験を行った.その結果,
多くの問題に対して,day-period と day-room が既存 ASP 解法より良い解を
生成し,提案手法の有効性が確認できた.

%%% Local Variables:
%%% mode: japanese-latex
%%% TeX-master: "paper"
%%% End:
