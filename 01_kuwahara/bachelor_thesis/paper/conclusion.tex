%%%%%%%%%%%%%%%%%%%%%%%%%%%%%%%%%%%%%%%%%%%%%%%%%%%%%%%%%% 
\chapter{結論}
%%%%%%%%%%%%%%%%%%%%%%%%%%%%%%%%%%%%%%%%%%%%%%%%%%%%%%%%%%

本論文では, 
LNPS の性能に重要な役割を果たす destory 演算子について, 
性能評価を行うことを目的として, 
3種類の destory 演算子 (random, day-period, day-room) を実装した. 

提案手法の性能を評価するために, 国際時間割協議会 ITC-2007 の
ベンチマーク問題 (21 問) を用いて実行実験を行った. 
その結果, UD5 シナリオの多くの問題に対して, day-period と day-room が既存 ASP 解法より良い解を生成し, 提案手法の有効性が確認できた. 

今後の課題として, さらなる destroy 演算の考案や評価, 
狙いを持って実装した destroy 演算子が意図通りに働いているかの検証, 
re-search の手法等の destroy 演算以外での問題へのアプローチ, 
また, 他の組み合わせ最適化問題への適用などが挙げられる.


%%% Local Variables:
%%% mode: latex
%%% TeX-master: "paper"
%%% End:
