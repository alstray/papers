\documentclass{abstB}

\begin{document}

%%%%%%%%%%%%%%%%%%%%%%%%%%%%%%%%%%%%%%%%%%%%%%%%%%%%%%%%%%%%%%%%%%%
\研究室名{番原}
\氏名{桑~~原~~~和~~也}
\卒研題目{%
  解集合プログラミングを用いた時間割問題の解法に関する考察}

%%%%%%%%%%%%%%%%%%%%%%%%%%%%%%%%%%%%%%%%%%%%%%%%%%%%%%%%%%%%%%%%%%%
\卒研要旨{%
 \textbf{時間割問題} (Timetabling Problem) は,
求解困難な組合せ最適化問題の一種である.
この問題には,ハード制約とソフト制約が存在し,ソフト制約に違反するとペ
ナルティが与えられる.必ず満たすべきハード制約を満たしながら,ペナルティ
の総和を最小にするような解を求めることが目的である.
現状では,質の高い時間割を編成するために多くの人間の労力が費やされている.
このような背景から,時間割に関する国際会議
(Practice and Theory of Automated Timetabling; PATAT)
や国際時間割競技会 (International Timetabling Competition; ITC)
が開催され,時間割ソルバーの性能向上に貢献している.

\textbf{解集合プログラミング} (Answer Set Programming; ASP) は,
論理プログラミングから派生した宣言的プログラミングパラダイムの一つである.
ASP 言語は一階論理に基づく知識表現言語の一種であり,論理プログラムは
ASP のルールの有限集合である.ASP システムは論理プログラムから安定モデ
ル意味論に基づく解集合を計算するシステムである.
近年,SAT 技術を応用した高速 ASP システムが実現され,ロボット工学,シ
ステム生物学,システム検証,プランニングなど様々な分野への実用的応用が
急速に拡大している.

近年,\textbf{カリキュラムベース・コース時間割}
(Curriculum-based Course Timetabling; CB-CTT)
問題に対する ASP を用いた解法が提案され,成功を収めている.
CB-CTT 問題は,大学等の1週間の講義スケジュールを求める問題であり,
最も研究が盛んな教育時間割問題の一つである.
ASP は系統的探索であることを活かして,未解決問題の最適値を決定するなど
優れた性能を示している.
しかし,その一方で,ソフト制約が多く含まれるような問題集において,
局所的探索を用いた解法より性能が劣っている場合が見られる.

この問題を解決するために,系統的探索と局所的探索を組み合わせた
\textbf{Large Neighborhood Prioritized Search} (LNPS)
が提案されている.
LNPS は,暫定解に含まれる変数の値割り当ての一部をランダムに選んで取り
消し,他の値割り当てをなるべく維持したままで解を再構築する反復法の一種
である.
ASP を用いた LNPS の利点は,解の再構築を系統的探索で行え,値割り当てを
なるべく維持したままでの再構築が自然に実現できることである.
LNPS の性能は,
暫定解の一部をランダムに選んで取り消す destroy 演算子に依存するが,
十分な研究がなされてない.

本論文では,LNPS を用いたカリキュラムベース・コース時間割 (CB-CTT)
問題の解法について述べる.
CB-CTT に対する既存研究を応用して,
3種類の destroy 演算子 (random, day-period, day-room) を実装した.
random が問題の性質をまったく利用しないのに対し,
day-period と day-room は CB-CTT のソフト制約を考慮して
暫定解の一部をランダムに選んで取り消す点が特長である.
%
提案手法の有効性を評価するために,国際時間割競技会 ITC-2007 のベンチマー
ク問題(21問)を用いて実行実験を行った.その結果,
多くの問題に対して,day-period と day-room が既存 ASP 解法より良い解を
生成し,提案手法の有効性が確認できた.}

\end{document}
