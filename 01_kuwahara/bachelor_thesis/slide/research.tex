%%%%%%%%%%%%%%%%%%%%%%%%%%%%%%%%%%%%%%%%%%%%%%%%%%%%%%%%%%%%%%%%%%%%%%%
\begin{frame}{研究目的}
  \begin{exampleblock}{先行研究}
    \begin{itemize}
    \item 系統的探索と局所的探索を組み合わせた
      \alert{優先度付き大域近傍探索 (LNPS)} [坡山ほか '18]
      が提案されている.
    \item LNPSは,暫定解に含まれる変数の値割り当ての一部をランダムに選んで取り
      消し(\alert{destroy}),他の値割り当てをなるべく維持したままで解を
      再探索する方法である.
    \end{itemize}
  \end{exampleblock}

  \begin{alertblock}{研究目的}
    LNPSをASP上に実装し,CB-CTT に適用・評価する.
  \end{alertblock}

  \begin{block}{研究内容}
    \begin{enumerate}
    \item LNPS の性能に重要な役割を果たす destroy 演算について,
      CB-CTT に適した3種類の手法の実装. 
    \item CB-CTT に対する実験・評価
    \end{enumerate}
  \end{block}
\end{frame}
%%%%%%%%%%%%%%%%%%%%%%%%%%%%%%%%%%%%%%%%%%%%%%%%%%%%%%%%%%%%%%%%%%%%%%%

%%% Local Variables:
%%% mode: latex
%%% TeX-master: slide
%%% End:
