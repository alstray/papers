%%%%%%%%%%%%%%%%%%%%%%%%%%%%%%%%%%%%%%%%%%%%%%%%%%%%%%%%%%%%%%%%%%%%%%%
\begin{frame}{研究目的}

  \begin{alertblock}{研究目的}
    系統的探索と局所的探索を組み合わせた
    \alert{Large Neighborhood Prioritized Search (LNPS)} [坡山ほか '18]
    をASP上に実装し,CB-CTT に適用・評価する.
    LNPSは,暫定解に含まれる変数の値割り当ての一部をランダム
    に選んで取り消し,他の値割り当てをなるべく維持したままで
    解を再構築する反復法の一種である.
  \end{alertblock}

  \begin{block}{研究内容}
    \begin{enumerate}
    \item LNPS の性能に重要な役割を果たす destroy 演算について,
      CB-CTT に適した3種類の手法
       (random, day-period, day-room)
      を実装した. 
      \begin{itemize}
      \item randomは問題の性質を利用しない手法.
      \item day-period, day-roomはCB-CTTのソフト制約を考慮した手法.
      \end{itemize}
    \item CB-CTT に対する実験・評価
    \end{enumerate}
  \end{block}
\end{frame}
%%%%%%%%%%%%%%%%%%%%%%%%%%%%%%%%%%%%%%%%%%%%%%%%%%%%%%%%%%%%%%%%%%%%%%%
