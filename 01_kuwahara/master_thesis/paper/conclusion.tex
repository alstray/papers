\chapter{おわりに}

本論文では,
系統的探索と確率的局所探索を統合的に適用する手法である
優先度付き巨大近傍探索(LNPS)の ASP 上での実装について述べた.
開発ソルバー \textit{asprior} は提案手法 LNPS を
高速 ASP ソルバー{\clingo}上に実装し,
国際時間割競技会 ITC2007 で使用された問題を含む
 CB-CTT 問題集(全61問)を用いて性能評価を行った.
その結果,
既知の最良値との比について,
通常の ASP 解法が $+437\%$ であったのに対し,
提案手法 LNPS は,その比を $+16\%$ まで大幅に改善できた.
さらに,11 問について,既知の最良値を更新することに成功した.

今後の課題としては,
アダプティブ LNPS への拡張や他の組合せ最適化問題への適用などが挙げられる.
本論文では,destroy 演算子を固定した LNPS を扱ったが,
より効率良く局所最適解から脱出するために,
実行中に最適な destroy 演算子を選択して動的に切り替える機能を加えた
アダプティブな LNPS を実現することで,
より良い結果が期待できると考えている.
また,開発ソルバー \textit{asprior} は
任意の組合せ最適化問題を解く汎用ソルバーであり,
今回対象とした CB-CTT 以外にも
他の時間割問題や巡回セールスマン問題などの組合せ最適化問題に対して
性能評価を行うことで,ソルバーの汎用性を確かめることができる.
LNPS は,メタ戦略として LNS を取り上げて拡張したものであるが,
他の問題へ適用する際に,
問題によっては遺伝的アルゴリズムなどの異なるメタ戦略を
取り入れることで,
様々な問題に対応可能な枠組みになり得ると筆者は期待している.

%%% Local Variables:
%%% mode: japanese-latex
%%% TeX-master: "paper"
%%% End:
