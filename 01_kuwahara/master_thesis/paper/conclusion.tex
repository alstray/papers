\chapter{おわりに}

本論文では,
系統的探索と確率的局所探索を統合的に適用する手法である
優先度付き巨大近傍探索(LNPS)を ASP 上に実装について述べた.
開発ソルバー \textit{asprior} は提案手法 LNPS を
高速 ASP ソルバー{\clingo}上に実装し,
国際時間割競技会で使用された問題を含む
 CB-CTT 問題集(全 61 問)を用いて性能評価を行った.
その結果,
既知の最良値との比について,
通常の ASP 解法が $+437\%$ であったのに対し,
提案手法 LNPS は,その比を $+16\%$ まで大幅に改善できた.
さらに,6 問について,既知の最良値を更新することに成功した.
今後の課題としては,
アダプティブ LNPS への拡張や他の時間割問題や組合せ最適化問題への適用などが挙げられる.


%%% Local Variables:
%%% mode: japanese-latex
%%% TeX-master: "paper"
%%% End:
