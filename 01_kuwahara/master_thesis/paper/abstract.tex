%%%%%%%%%%%%%%%%%%%%%%%%%%%%%%%%%%%%%%%%%%%%%%%%%%%%%%%%%% 
\chapter*{概要}
\pagenumbering{roman}
%%%%%%%%%%%%%%%%%%%%%%%%%%%%%%%%%%%%%%%%%%%%%%%%%%%%%%%%%% 

% しかし,その一方で,ソフト制約の種類が多い問題集においては,
% 確率的局所探索に基づくメタ戦略が,
% 多くの問題に対してより高精度な解を求めている.
% 以上から,
% 系統的探索の長所である``最適性の保証''と確率的局所探索の長所である
% ``計算時間相応の解精度''の両方を備えた統合的探索手法を実現することは
% 重要な研究課題といえる.

本論文では,解集合プログラミング(Answer Set Programming; ASP)に基づく
優先度付き巨大近傍探索 (Large Neighborhood Prioritized Search; LNPS)
の実装,および,開発したソルバー \textit{asprior}の性能評価について述べる.
%
LNPS は,ASP の系統的探索と
メタ戦略の一種である巨大近傍探索
(Large Neighborhood Search; LNS)
を統合した探索手法である.
%
Pisinger らの LNS は解に含まれる変数の値割当ての一部をランダムに選んで取り消し,
その変数のみに対して再割当てを行うことで解を再構築する反復解法である.
これに対して,
LNPS は,解の再構築の操作を,値割当てをなるべく維持したまま
での再探索に置き換えることで,取り消されなかった変数への再割当てを許す.
これによって,どの値割当てを取り消すかに依存しすぎない探索を行うことが
できる点が特長である.
LNPS は,メタ戦略と同様に,近接最適性が成り立つ問題に対する有効性が期
待できる.また,各反復の終了条件を適切に設定することで解の最適性も保証で
きる.
開発した \textit{asprior} は,
ASP ファクト形式の問題インスタンスと
問題を解く ASP 符号化を入力とし,
LNPS アルゴリズムを用いて解を求める汎用的なソルバーである.
\textit{asprior} は,
高速 ASP ソルバー {\clingo}
の Python インターフェースを利用して実装されている.
%
提案手法を評価するために,国際時間割競技会 ITC2007
で使用された問題を含む CB-CTT 問題集(全61問)を用いて性能評価を行った.
その結果,
既知の最良値との比について,
通常の ASP 解法が$+472\%$であったのに対し,
提案手法は,その比を$+53\%$まで大幅に改善できた.
さらに,11問について,既知の最良値を更新することに成功した.

%%% Local Variables:
%%% mode: japanese-latex
%%% TeX-master: "paper"
%%% End:
