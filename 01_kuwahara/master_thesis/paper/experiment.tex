\chapter{評価実験}

提案手法の有効性を評価するために
カリキュラムベース・コース時間割問題
に対する実行実験を行なった.

\section{destroy 演算}
LNPS の$destroy$については,
CB-CTTの既存研究~\cite{anor/Kiefer2017}を参考に,
以下の3種類を実装した.
\begin{itemize}\compress
\item \textsf{Random} $N$ (\textsf{R-$N$})\\
  暫定解から変数の値割当ての$N$\%をランダムに選んで取り消す.
\item \textsf{Day-Period} (\textsf{DP})\\
  曜日\textsf{D}と時限\textsf{P}をランダムに1組選び,
  暫定解から\textsf{D}曜\textsf{P}限に関する
  変数の値割当てをすべて取り消す.
\item \textsf{Day-Room} (\textsf{DR})\\
  曜日\textsf{D}と教室\textsf{R}をランダムに1組選び,
  暫定解から\textsf{D}曜日の\textsf{R}教室に関する
  変数の値割当てをすべて取り消す.
\end{itemize}

また,上記の destroy 演算を参考に以下の
2種類を新たに考案し実装した.いずれも,
科目に対する曜日と時限の割当てはできるだけ維持する
ことで,教室の変更を促し,
教室に関するソフト制約への違反を減らすことを
狙いとしている.
\begin{itemize}
\item \textsf{Swap-Room} $N$ (\textsf{SR-$N$})\\
  \textsf{Random} $N$ と同様に,
  暫定解から変数の値割当ての$N$\%をランダムに選んで取り消す.
  ただし,科目に対する曜日と時限の割当てはできるだけ維持する.
\item \textsf{Day-Period-Swap-Room} $N$ (\textsf{DPSR-$N$})\\
  \textsf{Day-Period} と同様に,
  曜日\textsf{D}と時限\textsf{P}をランダムに $N$ 組選び,
  暫定解から\textsf{D}曜\textsf{P}限に関する変数の値割当てをすべて取り消す.
  ただし,科目に対する曜日と時限の割当てはできるだけ維持する.
\end{itemize}

\section{実験概要}
比較した手法は以下の2つである.
\begin{itemize}\compress
\item 既存手法: ASPソルバー{\clingo}
 \begin{itemize}
  \item 分枝限定法による探索(以下 bb)と充足不能コアを用いた探索(以下 usc)
  の 2 通りで実行を行い,オプションは先行研究~\cite{anor/Banbara2019}
  によって結果の良かったものを使用した.
 \end{itemize}
\item 提案手法: LNPSを{\clingo}上に実装
 \begin{itemize}
  \item destroy 演算子 : R-0/3/5, DP, DR, SR-5/10, DPSR-1/2/3 の 10 種類
  \item 初期解探索手法 : usc
  \item 初期回探索の打ち切り : conflict 数が 250 万以上,または restart 数が 5,000 以上
  \item $re\mathchar`-search$ 手法 : bb
  \item $re\mathchar`-search$ の打ち切り : conflict 数が 3 万以上
  
 \end{itemize}
\end{itemize}

ベンチマーク問題には,
国際時間割競技会ITC2007~\footnote{%
  \url{http://www.cs.qub.ac.uk/itc2007/}}
で公開されているカリキュラムベース・コース時間割(CB-CTT)の
問題集(全21問)と,
他の問題集(全40問)に対して,
ソフト制約が最も多い UD5 を使って
評価を行なった~\cite{GasperoMS/ITC2007,DBLP:journals/anor/BonuttiCGS12}.
CB-CTT 問題を解くための ASP 符号化には,
\textsf{teaspoon}符号化~\cite{anor/Banbara2019}
を使用した.
%

既存手法と提案手法ともに,
ASPソルバーには{\clingo}-5.4.0を利用し,
1問あたりの制限時間は1時間とした.
実験環境は,Mac OS, 3.2GHz Intel Core i7, 64GB メモリである.

初期解探索と$re\mathchar`-search$
の手法に関して,それぞれ usc と bb の両方
での予備実験を行い,平均的に良い性能を示した
探索手法を採用している.

\section{実験結果}

%%%%%%%%%%%%%%%%%%%%%%%%%%%%%%%%%%%%%%%%%%%%%%
\begin{table*}[pt]\centering
  \caption{ITC2007問題集の実験結果: 得られた最適値と最良値}
  \vskip 1em
  \label{table:bench:result1}
  \begin{tableA}
    {comp01} & 129 & 283 & 13 & \bf{11} & \bf{11} & \bf{11} & \bf{11} & \bf{11} & \bf{11} & \bf{11} & \bf{11} & \bf{11}\\
{comp02} & 1049 & 331 & 259 & 239 & 199 & \bf{172} & 235 & 201 & 213 & 260 & 227 & 236\\
{comp03} & 791 & 302 & 173 & 173 & 154 & 149 & 149 & 144 & \bf{143}& 145 & 154 & 144\\
{comp04} & 231 & ${}^\ast$\bf{49} & ${}^\ast$\bf{49} & ${}^\ast$\bf{49} & ${}^\ast$\bf{49} & ${}^\ast$\bf{49} & ${}^\ast$\bf{49} & ${}^\ast$\bf{49} & ${}^\ast$\bf{49} & ${}^\ast$\bf{49} & ${}^\ast$\bf{49} & ${}^\ast$\bf{49}\\
{comp05} & 2662 & 1940 & 1102 & 922 & 797 & 864 & 841 & 891 & 861 & 994 & \bf{776} & 907\\
{comp06} & 822 & 1025 & 216 & 162 & 135 & 135 & 166 & 112 & 106 & 119 & \bf{102} & 123\\
{comp07} & 924 & 1149 & 153 & 131 & 114 & 99 & 105 & 60 & 65 & 56 & 74 & \bf{40}\\
{comp08} & 348 & ${}^\ast$\bf{55} & ${}^\ast$\bf{55} & ${}^\ast$\bf{55} & ${}^\ast$\bf{55} & ${}^\ast$\bf{55} & ${}^\ast$\bf{55} & ${}^\ast$\bf{55} & ${}^\ast$\bf{55} & ${}^\ast$\bf{55} & ${}^\ast$\bf{55} & ${}^\ast$\bf{55}\\
{comp09} & 617 & 254 & 154 & 154 & 151 & 143 & 146 & 145 & 146& 141 & \bf{138} & 141\\
{comp10} & 822 & 1229 & 209 & 166 & 141 & 124 & 133 & 97 & \bf{80} & 97 & 98 & 101\\
{comp11} & 287 & ${}^\ast$\bf{0} & ${}^\ast$\bf{0} & ${}^\ast$\bf{0} & ${}^\ast$\bf{0} & ${}^\ast$\bf{0} & ${}^\ast$\bf{0} & ${}^\ast$\bf{0} & ${}^\ast$\bf{0} & ${}^\ast$\bf{0} & ${}^\ast$\bf{0} & ${}^\ast$\bf{0}\\
{comp12} & 2626 & 1246 & 787 & 740 & 728 & 705 & 694 & 729 & \bf{664} & 687 & 718 & 702\\
{comp13} & 661 & 301 & 171 & 158 & 163 & 168 & 165 & 147 & 152 & 147 & 149 & \bf{146}\\
{comp14} & 748 & ${}^\ast$\bf{67} & 189 & 156 & 145 & 143 & 158 & 104 & 83 & 123 & 130 & 128\\
{comp15} & 852 & 607 & 232 & 214 & 213 & 227 & 206 & 210 & 205 & \bf{198} & 212 & 213\\
{comp16} & 944 & 1090 & 197 & 156 & 168 & 140 & 162 & 167 & 151 & \bf{134} & 149 & 172\\
{comp17} & 979 & 412 & 244 & 226 & 211 & 209 & 214 & 196 & 204 & 200 & \bf{184} & 203\\
{comp18} & 673 & 471 & 180 & 156 & 148 & 151 & 155 & 140 & 149 & 146 & \bf{136} & 149\\
{comp19} & 890 & 919 & 231 & 192 & 190 & 165 & 199 & 176 & 163 & \bf{144} & 174 & 171\\
{comp20} & 3304 & 1386 & 373 & 274 & 356 & 268 & 273 & 272 & 246 & \bf{237} & 283 &291\\
{comp21} & 893 & 310 & 234 & 210 & 202 & 222 & 214 & 166 & 167 & \bf{161} & 192 &170\\\hline
{\#最適値} & \lw{0} & \lw{4} & \lw{3} & \lw{4} & \lw{4} & \lw{5} & \lw{4} & \lw{4} & \lw{7} & \lw{9} & \lw{9} & \lw{6}\\
{・最良値} & & & & & & & & & & & &\\

%%% Local Variables:
%%% mode: japanese-latex
%%% TeX-master: "../paper"
%%% End:

  \end{tableA}
\end{table*}
%%%%%%%%%%%%%%%%%%%%%%%%%%%%%%%%%%%%%%%%%%%%%%
\begin{table*}[pt]\centering
  \caption{ITC2007問題集: 他のアプローチとの比較}
  \vskip 1em  
  \label{table:bench:result2}
  \begin{tableB}
    {comp01} & 11 & 129 & +1,072 & 11 & 0\\
{comp02} & 130 & 331 & +154 & 172 & +32\\
{comp03} & 142 & 302 & +112 & 143 & +1\\
{comp04} & 49 & 49 & 0 & 49 & 0\\
{comp05} & 570 & 1,940 & +240 & 776 & +36\\
{comp06} & 85 & 822 & +867 & 102 & +20\\
{comp07} & 42 & 924 & +2,100 & \alert{\bf 40} & \alert{\bf -5}\\
{comp08} & 55 & 55 & 0 & 55 & 0\\
{comp09} & 150 & 254 & +69 & \alert{\bf 138} & \alert{\bf -8}\\
{comp10} & 72 & 822 & +1,041 & 80 & +11\\
{comp11} & 0 & 0 & 0 & 0 & 0\\
{comp12} & 483 & 1,246 & +157 & 664 & +37\\
{comp13} & 147 & 301 & +104 & \alert{\bf 146} & \alert{\bf -1}\\
{comp14} & 67 & 67 & 0 & 83 & +23\\
{comp15} & 176 & 607 & +244 & 198 & +13\\
{comp16} & 96 & 944 & +883 & 134 & +40\\
{comp17} & 155 & 412 & +165 & 184 & +19\\
{comp18} & 137 & 471 & +243 & \alert{\bf 136} & \alert{\bf -1}\\
{comp19} & 125 & 890 & +612 & 144 & +15\\
{comp20} & 124 & 1,386 & +1,017 & 237 & +91\\
{comp21} & 151 & 310 & +105 & 161 & +7\\\hline
{$\sharp$との比の平均} & & & +437 & & +16\\\hline
  \end{tableB}
\end{table*}
%%%%%%%%%%%%%%%%%%%%%%%%%%%%%%%%%%%%%%%%%%%%%%
\begin{table*}[pt]\centering
  \caption{他の問題集の実験結果: 得られた最適値と最良値}
  \vskip 1em
  \label{table:bench:result3}
  \begin{tableC}
    {DDS1} & 6536 & 7544 & 4864 & 2993 & 3317 & 2780 & 3047 & 2834 & 2879 & 3076 & 2739 & \bf{2700}\\
{DDS2} & 407 & 433 & 100 & 81 & 90 & \bf{68} & 76 & 79 & 70 & 78 & 74 & \bf{68}\\
{DDS3} & \bf{22} & 391 & 36 & 28 & \bf{22} & \bf{22} & \bf{22} & \bf{22} & \bf{22} & \bf{22} & \bf{22} & \bf{22}\\
{DDS4} & 10486 & 3123 & 1443 & 3056 & 2990 & 2690 & 1595 & 1305 & 1539 & 1183 & \bf{1162} & 1245\\
{DDS5} & 539 & ${}^\ast$\bf{76} & ${}^\ast$\bf{76} & ${}^\ast$\bf{76} & ${}^\ast$\bf{76} & ${}^\ast$\bf{76} & ${}^\ast$\bf{76} & ${}^\ast$\bf{76} & ${}^\ast$\bf{76} & ${}^\ast$\bf{76} & ${}^\ast$\bf{76} & ${}^\ast$\bf{76}\\
{DDS6} & 849 & 964 & 208 & 174 & 187 & 180 & 160 & \bf{158} & 203 & 162 & 187 & 202\\
{DDS7} & 645 & 1485 & 120 & 75 & 363 & 87 & 67 & 65 & 68 & 74 & \bf{58} & 66\\
{EA01} & 4708 & 807 & 269 & 226 & 226 & 225 & 249 & 222 & 216 & \bf{201} & 204 & 217\\
{EA02} & 1094 & 1215 & 140 & 146 & 132 & 135 & \bf{130} & 165 & 132 & 169 & 137 & 136\\
{EA03} & 8192 & 2555 & 586 & 1411 & 1486 & 829 & \bf{508} & 775 & 789 & 901 & 882 & 834\\
{EA04} & \bf{63} & 1873 & 388 & 1309 & 1611 & 681 & 359 & 130 & 842 & 169 & 168 & 120\\
{EA05} & 26 & 513 & 87 & 16 & 21 & 29 & 18 & \bf{14} & \bf{14} & \bf{14} & \bf{14} & \bf{14}\\
{EA06} & 543 & 1027 & 235 & 208 & 179 & 166 & 172 & 140 & \bf{125} & 169 & 161 & 164\\
{EA07} & 10122 & 2831 & 995 & 1653 & 2624 & 1199 & 851 & \bf{671} & 722 & 750 & 701 & 705\\
{EA08} & 480 & 991 & 166 & 60 & 685 & 60 & 48 & \bf{40} & 42 & 48 & 42 & 44\\
{EA09} & \bf{48} & 865 & \bf{48} & 53 & 53 & 50 & 52 & \bf{48} & \bf{48} & \bf{48} & \bf{48} & \bf{48}\\
{EA10} & 1711 & 444 & 355 & 456 & 1235 & \bf{273} & 306 & 420 & 427 & 410 & 396 & 382\\
{EA11} & 232 & 771 & 123 & 54 & 59 & 82 & 57 & 40 & 40 & \bf{36} & 41 & 40\\
{EA12} & 234 & 518 & 79 & 43 & 45 & 51 & 42 & \bf{27} & \bf{27} & \bf{27} & \bf{27} & \bf{27}\\
{erlangen2011\_2} & 17035 & 17263 & \bf{16772} & \bf{16772} & \bf{16772} & \bf{16772} & \bf{16772} & \bf{16772} & \bf{16772} & \bf{16772} & \bf{16772} & \bf{16772}\\
{erlangen2012\_1} & 32541 & \bf{25061} & 25538 & 25538 & 25538 & 25538 & 25538 & 25538 & 25538 & 25538 & 25538 & 25538\\
{erlangen2012\_2} & 38180 & 34360 & \bf{34218} & \bf{34218} & \bf{34218} & \bf{34218} & \bf{34218} & \bf{34218} & \bf{34218} & \bf{34218} & \bf{34218} & \bf{34218}\\
{erlangen2013\_1} & 29459 & 28302 & \bf{27923} & \bf{27923} & \bf{27923} & \bf{27923} & \bf{27923} & \bf{27923} & \bf{27923} & \bf{27923} & \bf{27923} & \bf{27923}\\
{erlangen2013\_2} & 34568 & \bf{29140} & 30117 & 30117 & 30117 & 30117 & 30117 & 30117 & 30117 & 30117 & 30117 & 30117\\
{erlangen2014\_1} & 30249 & 24510 & \bf{23461} & \bf{23461} & \bf{23461} & \bf{23461} & \bf{23461} & \bf{23461} & \bf{23461} & \bf{23461} & \bf{23461} & \bf{23461}\\
    {test1} & 619 & 751 & 334 & \bf{232} & \bf{232} & \bf{232} & \bf{232} & \bf{232} & \bf{232} & \bf{232} & \bf{232} & \bf{232}\\
{test2} & 24 & ${}^\ast$\bf{20} & \bf{20} & 21 & \bf{20} & \bf{20} & \bf{20} & \bf{20} & \bf{20} & \bf{20} & \bf{20} & \bf{20}\\
{test3} & 196 & 117 & 79 & 79 & 78 & 79 & 79 & \bf{71} & 76 & 73 & 73 & 76\\
{test4} & 705 & 260 & 181 & 172 & 144 & 138 & 150 & 159 & 165 & 140 & \bf{117} & 148\\
{toy} & ${}^\ast$\bf{0} & ${}^\ast$\bf{0} & ${}^\ast$\bf{0} & ${}^\ast$\bf{0} & ${}^\ast$\bf{0} & ${}^\ast$\bf{0} & ${}^\ast$\bf{0} & ${}^\ast$\bf{0} & ${}^\ast$\bf{0} & ${}^\ast$\bf{0} & ${}^\ast$\bf{0} & ${}^\ast$\bf{0}\\
{Udine1} & 1079 & 953 & 256 & 225 & 250 & 212 & 223 & 213 & 228 & 218 & 199 & \bf{185}\\
{Udine2} & 505 & 402 & 133 & 106 & 121 & 108 & 94 & 118 & 117 & \bf{89} & 93 & 122\\
{Udine3} & 343 & 385 & 119 & 100 & 73 & \bf{62} & 93 & 88 & 96 & 87 & 71 & 69\\
{Udine4} & 400 & ${}^\ast$\bf{106} & ${}^\ast$\bf{106} & ${}^\ast$\bf{106} & ${}^\ast$\bf{106} & ${}^\ast$\bf{106} & ${}^\ast$\bf{106} & ${}^\ast$\bf{106} & ${}^\ast$\bf{106} & ${}^\ast$\bf{106} & ${}^\ast$\bf{106} & ${}^\ast$\bf{106}\\
{Udine5} & 518 & 695 & 103 & 77 & 72 & 67 & 75 & 80 & 66 & 80 & \bf{61} & 76\\
{Udine6} & 366 & 628 & 156 & 54 & 45 & 43 & 47 & 38 & 40 & 38 & \bf{36} & 41\\
{Udine7} & 233 & 733 & 80 & 77 & 81 & 70 & 77 & 73 & \bf{67} & 76 & 72 & 76\\
{Udine8} & 606 & 941 & 110 & 105 & 112 & 108 & 110 & 111 & 109 & 105 & 100 & \bf{96}\\
{Udine9} & 610 & 163 & 87 & 74 & 72 & 70 & 74 & 69 & 63 & 65 & \bf{62} & 72\\
{UUMCAS\_A131} & 38929 & 28688 & 27397 & 27911 & 28081 & 27099 & 26843 & 27364 & 28037 & \bf{24579} & 25342 & 26832\\\hline
{\#最適値} & \lw{4} & \lw{6} & \lw{9} & \lw{8} & \lw{10} & \lw{13} & \lw{12} & \lw{17} & \lw{15} & \lw{17} & \lw{19} & \lw{17}\\
{・最良値} & & & & & & & & & & & &\\

%%% Local Variables:
%%% mode: japanese-latex
%%% TeX-master: "../paper"
%%% End:

  \end{tableC}
\end{table*}
%%%%%%%%%%%%%%%%%%%%%%%%%%%%%%%%%%%%%%%%%%%%%%
\begin{table*}[pt]\centering
  \caption{他の問題集: 他のアプローチとの比較}
  \vskip 1em  
  \label{table:bench:result4}
  \begin{tableB}
    {DDS1} & 1831 & 6536 & +257 & 2700 & +47\\
{DDS2} & 64 & 407 & +536 & 68 & +6\\
{DDS3} & 22 & 22 & 0 & 22 & 0\\
{DDS4} & 96 & 3123 & +3153 & 1162 & +1110\\
{DDS5} & 76 & 76 & 0 & 76 & 0\\
{DDS6} & 96 & 849 & +784 & 158 & +65\\
{DDS7} & 52 & 645 & +1140 & 58 & +12\\
{EA01} & 196 & 807 & +312 & 201 & +3\\
{EA02} & 128 & 1094 & +755 & 130 & +2\\
{EA03} & 90 & 2555 & +2739 & 508 & +464\\
{EA04} & 18 & 63 & +250 & 120 & +567\\
{EA05} & 14 & 26 & +86 & 14 & 0\\
{EA06} & 99 & 543 & +448 & 125 & +26\\
{EA07} & 205 & 2831 & +1281 & 671 & +227\\
{EA08} & 40 & 480 & +1100 & 40 & 0\\
{EA09} & 48 & 48 & 0 & 48 & 0\\
{EA10} & 93 & 444 & +377 & 273 & +194\\
{EA11} & 40 & 232 & +480 & 36 & \bf{-10}\\
{EA12} & 27 & 234 & +767 & 27 &0\\
{erlangen2011\_2} & 12353 & 17035 & +38 & 16772 & +36\\
{erlangen2012\_1} & 28236 & 25061 & \bf{-11} & 25538 & \bf{-10}\\
{erlangen2012\_2} & 37103 & 34360 & \bf{-7} & 34218 & \bf{-8}\\
{erlangen2013\_1} & 28997 & 28302 & \bf{-2} & 27923 & \bf{-4}\\
{erlangen2013\_2} & 30533 & 29140 & \bf{-5} & 30117 & \bf{-1}\\
{erlangen2014\_1} & 28655 & 24510 & \bf{-14} & 23461 & \bf{-18}\\
{test1} & 232 & 619 & +167 & 232 &0\\
{test2} & 20 & 20 & 0 & 20 & 0\\
{test3} & 68 & 117 & +72 & 71 & 4\\
{test4} & 166 & 260 & +57 & 117 & \bf{-30}\\
{toy} & 0 & 0 & 0 & 0 & 0\\
{Udine1} & 138 & 953 & +591 & 185 & +34\\
{Udine2} & 81 & 402 & +396 & 89 & +10\\
{Udine3} & 37 & 343 & +827 & 62 & +68\\
{Udine4} & 106 & 106 & 0 & 106 & 0\\
{Udine5} & 47 & 518 & +1002 & 61 & +30\\
{Udine6} & 36 & 366 & +917 & 36 & 0\\
{Udine7} & 64 & 233 & +264 & 67 & 5\\
{Udine8} & 88 & 606 & +589 & 96 & +9\\
{Udine9} & 56 & 163 & +191 & 62 & +11\\
{UUMCAS\_A131} & 19699 & 28688 & +46 & 24579 & +25\\\hline
{$\sharp$との比の平均} & & & +490 & & +72\\\hline

%%% Local Variables:
%%% mode: japanese-latex
%%% TeX-master: "../paper"
%%% End:

  \end{tableB}
\end{table*}
%%%%%%%%%%%%%%%%%%%%%%%%%%%%%%%%%%%%%%%%%%%%%%

表~\ref{table:bench:result1}に,ITC2007問題集に対して
各手法で得られた最適値および最良値を示す.
提案手法 LNPS については,3回の実験で得られた値のうち
最も良かった値が記されている.
各問題ごとに全手法の中で最も良かった値を太字で表している.
`$\ast$'付きの値は最適値を意味する.
提案手法で最適値決定ができた3問については,
いずれも初期解探索の段階で求められている.
提案手法は,1問を除くすべての問題に対して,既存手法と同じかより良い解
を得ていることが確認できる.
各手法について,最適値および最良値を求めた問題数を比較すると,
提案手法 \textsf{DPSR-1} と \textsf{DPSR-2} が9問と最も多く,次いで
\textsf{SR-10}が7問であった.

表~\ref{table:bench:result2}に,ITC2007問題集における
他のアプローチとの比較結果を示す.
左から順に,
問題名,
既知の最良値($\sharp$),
既存手法 ASP の最良値と$\sharp$との比,
提案手法 LNPS の最良値と$\sharp$との比が示されている.
既知の最良値は,これまでに,メタ戦略に基づく各種アルゴリズム,整数計画法,
SAT, MaxSAT, SMT など様々な手法で求められた結果である~\cite{anor/Banbara2019}.
既存手法の最良値は,各問題に対して,表~\ref{table:bench:result1}中の既存
手法(2種類)で得られた値の中で良かった値を示している.
同様に提案手法の最良値も,各問題に対して,表~\ref{table:bench:result1}中の提
案手法(10種類)で得られた値の中で最も良い値を示している.
また,既知の最良値との比は,以下の計算結果を百分率で表したものである.
\[
既知の最良値との比 = \frac{得られた最良値 - 既知の最良値}{既知の最良値}
\]
既存手法 ASP は,既知の最良値との比が$+437\%$と高く,解の精度が悪いこ
とが確認できる.
一方,提案手法 LNPS は,既知の最良値との比が$+16\%$まで抑えられており,
既存手法と比較して約 27 倍改善されている.
提案手法の比は,3回の実験で最も良かった値
をもとに計算されているが,
3回の実験で得られた値の平均値をもとに
同様の比を求めたところ$+29\%$となり,
値は劣るがこちらも既存手法と比較して大きく改善されている.
さらに,提案手法は,comp07, comp09, comp13, comp18 
の4問について,
既知の最良値を更新することに成功した.

表~\ref{table:bench:result3}に,他の問題集に対して
各手法で得られた最適値および最良値を示す.
各問題ごとに全手法の中で最も良かった値を太字で表している.
こちらは表~\ref{table:bench:result1}と異なり1回の実験で
得られた値が記されている.
`$\ast$'付きの値は最適値を意味する.
提案手法で最適値決定ができた3問については,
いずれも初期解探索の段階で求められている.
提案手法は,4問を除くすべての問題に対して,既存手法と同じかより良い解
を得ている.うち1問は,同じ値が得られたが,
既存手法のみ最適値決定ができている.
各手法について,最適値および最良値を求めた問題数を比較すると,
提案手法 \textsf{DPSR-2} が19問と最も多く,
次いで \textsf{SR-5, DPSR-1, DPSR-3} が17問であった.

表~\ref{table:bench:result4}に,他の問題集における
他のアプローチとの比較結果を示す.
表の見方は表~\ref{table:bench:result2}と同様である.
こちらも既存手法 ASP は,既知の最良値との比が$+490\%$と解の精度が悪く,
提案手法 LNPS では,既知の最良値との比が$+72\%$に抑えられており,
既存手法と比較して約 7 倍改善されていることが確認できる.
さらに,提案手法は,
EA11, 
test4, 
erlangen2012\_1, 
erlangen2012\_2, 
erlangen2013\_1, 
erlangen2013\_2, 
erlangen2014\_1 の7問について,
既知の最良値を更新することに成功した.
このうち,erlangenシリーズの5問については
実行終了まで初期解探索の打ち切り条件を満たさなかった.
今回,初期解探索および$re\mathchar`-search$の打ち切り条件について
ITC2007問題集のみを用いたチューニングを行なったため,
適切なタイミングでの$re\mathchar`-search$への移行や,
さらなる解精度の向上という点で改良の余地がある.

%%% Local Variables:
%%% mode: japanese-latex
%%% TeX-master: "paper"
%%% End:
