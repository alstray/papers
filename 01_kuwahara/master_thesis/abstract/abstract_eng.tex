\documentclass[dvipdfmx,a4paper]{jsarticle}

\title{ASP-based Integration of Systematic and Stochastic Local Search}
\author{Name:Kazuya Kuwahara}
\date{Student ID:252005066}

\usepackage{url}
\pagestyle{empty}

\begin{document}
\maketitle

%\textbf{解集合プログラミング}(Answer Set Programming; ASP)は,
%論理プログラミングから派生した宣言的プログラミングパラダイムである.
%ASP言語は一階論理に基づく知識表現言語の一種である.
%ASP ソルバーは安定モデル意味論に基づ
%く解集合を計算するプログラムである.
%近年,SAT ソルバーの実装技術を応用した高速 ASP ソルバーが実現され,
%人工知能分野の諸問題を中心に実用的応用が急速に拡大している.

\textbf{Answer Set Programming (ASP)} is 
a declarative programming paradigm derived from logic programming. 
The ASP language is one of knowledge representation languages 
based on first-order predicate logic. 
ASP solvers are computer programs that calculate answer sets of logic
programs based on stable model semantics.
In recent years, 
efficient ASP solvers utilizing 
SAT techniques enable us to solve a wide range of
practical problems in AI field.

%解集合プログラミングが成功した応用事例の一つに,
%\textbf{カリキュラムベース・コース時間割}
%(Curriculum-Based Course TimeTabling; CB-CTT)がある.
%CB-CTT は大学等での一週間の講義スケジュールを編成する
%求解困難な組合せ最適化問題である.
%CB-CTT は必ず満たすべきハード制約と,
%できるだけ満たしたい重み付きソフト制約から構成される.
%違反するソフト制約の重み(ペナルティ)の総和の最小化が目的となる.
%解集合プログラミングは,この問題に対し,
%ASP ソルバーの系統的探索を活かして,未解決問題の最適値決定を含む優れた
%性能を示している.

\textbf{Curriculum-Based Course TimeTabling (CB-CTT)} 
is one of the most successful ASP applications.
The CB-CTT problem is a difficult combinatorial optimization problem
defined as the task of constructing a weekly schedule of lectures in
universities.
The problem has a set of hard and soft constraints.
Hard constraints must be strictly satisfied.
Soft constraints are not necessarily satisfied,
but the sum of their violations should be minimal.
For this problem, 
ASP has succeeded in finding new optimal solutions of previously
unsolved problems by taking advantage of its powerful systematic
search.

%しかし,その一方で,ソフト制約の種類が多い問題集においては,
%確率的局所探索に基づくメタ戦略が,
%多くの問題に対してより高精度な解を求めている.
%以上から,
%系統的探索の長所である\textbf{``最適性の保証''}と確率的局所探索の長所である
%\textbf{``計算時間相応の解精度''}の両方を備えた統合的探索手法を実現することは
%重要な研究課題といえる.

On the other hand, 
meta-heuristics based on stochastic local search 
can often find better solutions for 
large CB-CTT instances having more soft constraints. 
It is therefore important to investigate 
an integrated method that combines
strength of systematic and stochastic local search, such as
``guarantee of optimality'' and 
``solution quality meeting the cost of computation time''.

%本論文では,解集合プログラミングに基づく
%\textbf{優先度付き巨大近傍探索}
%(Large Neighborhood Prioritized Search; LNPS)
%の実装,および,
%開発したソルバー \textit{asprior}の性能評価について述べる.
%LNPSは,
%ASP の系統的探索と
%メタ戦略の一種である巨大近傍探索
%(Large Neighborhood Search; LNS)
%を統合した探索手法である.
%LNS は解に含まれる変数の値割当ての一部をランダムに選んで取り消し,
%その変数のみに対して再割当てを行うことで解を再構築する反復解法である.
%これに対して,
%LNPS は,解の再構築の操作を,値割当てをなるべく維持したまま
%での再探索に置き換えることで,取り消されなかった変数への再割当てを許す.
%これによって,どの値割当てを取り消すかに依存しすぎない探索を行うことが
%できる点が特長である.
%LNPS は,メタ戦略と同様に,近接最適性が成り立つ問題に対する有効性が期
%待できる.また,各反復の終了条件を適切に設定することで解の最適性も保証で
%きる.
%開発した \textit{asprior} は,
%ASP ファクト形式の問題インスタンスと
%問題を解く ASP 符号化を入力とし,
%LNPS アルゴリズムを用いて解を求める汎用的なソルバーである.
%\textit{asprior} は,
%高速 ASP ソルバー {\textit clingo}
%の Python インターフェースを利用して実装されている.

In this paper, 
we describe an ASP-based implementation of 
   \textbf{Large Neighborhood Prioritized Search (LNPS)} 
and present the resulting solver \textit{asprior}. 
%
LNPS is an integrated search method of
systematic search and Large Neighborhood Search (LNS). 
During optimal search,
LNS destroys the value assignments of some variables in a
temporary solution, and then repairs them by re-assigning values to
the destroyed variables only.
In contrast, LNPS allows for the re-assignment of undestroyed variables
as well as the destroyed ones by 
replacing re-assigning with re-search,
while keeping the assignments of undestroyed variables as much as possible. 
Thus, LNPS is able to search solutions without strongly 
depending on the destroyed variables. 
Furthermore, LNPS can guarantee the optimality of obtained solutions
and can be effective for a wide range of problems having the proximate
optimality property.
%
For a given a problem instance and ASP encoding for problem solving,
the resulting solver \textit{asprior} finds solutions based on LNPS
algorithm.
\textit{asprior} is implemented by using Python interface of efficient
ASP solver \textit{clingo}.

%提案手法を評価するために,国際時間割競技会 ITC2007
%で使用された問題を含む CB-CTT 問題集(全61問)を用いて性能評価を行った.
%その結果,
%既知の最良値との比について,
%通常の ASP 解法が$+472\%$であったのに対し,
%提案手法は,その比を$+53\%$まで大幅に改善できた.
%さらに,11問について,既知の最良値を更新することに成功した.

To evaluate our approach,
we carried out experiments on CB-CTT problems (61 instances in total)
including ITC2007 competition instances.
As a result, 
the proposed approach is able to greatly improve
the ratio of previously best known bounds
from $+472\%$ of standard ASP solving to $+53\%$.
Moreover, we succeed in finding new upper bounds of 11 instances.

\end{document}

% LocalWords:  PDNP PDNRP reachability PDNRPs PDNPs clingo encodings
% LocalWords:  acyclicity DNET fukui tepco TimeTabling CTT optimality

%%% Local Variables:
%%% mode: japanese-latex
%%% TeX-master: t
%%% End:
% LocalWords:  combinatorial LNPS asprior LNS undestroyed ITC
