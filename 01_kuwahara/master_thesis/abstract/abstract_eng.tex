\documentclass[dvipdfmx,a4paper]{jsarticle}

\title{A Research on ASP-based Integration of\\
Systematic and Stochastic Local Search}
\author{Name:Kazuya Kuwahara}
\date{Student ID:252005066}

\usepackage{url}
\pagestyle{empty}

\begin{document}
\maketitle

%\textbf{解集合プログラミング}(Answer Set Programming; ASP)は,
%論理プログラミングから派生した宣言的プログラミングパラダイムである.
%ASP言語は一階論理に基づく知識表現言語の一種である.
%ASP ソルバーは安定モデル意味論に基づ
%く解集合を計算するプログラムである.
%近年,SAT ソルバーの実装技術を応用した高速 ASP ソルバーが実現され,
%人工知能分野の諸問題を中心に実用的応用が急速に拡大している.

\textbf{Answer Set Programming} (ASP) is 
a declarative programming paradigm derived from 
logic programming. 
The ASP language is one of 
knowledge representation languages 
based on first-order predicate logic. 
ASP solvers are programs 
calculating an answer set based on stable model semantics. 
In recent years, 
efficient ASP solvers utilizing 
SAT solving implementation techniques 
enable us to solve practical problems, 
especially in the field of AI.

%解集合プログラミングが成功した応用事例の一つに,
%\textbf{カリキュラムベース・コース時間割}
%(Curriculum-Based Course TimeTabling; CB-CTT)がある.
%CB-CTT は大学等での一週間の講義スケジュールを編成する
%求解困難な組合せ最適化問題である.
%CB-CTT は必ず満たすべきハード制約と,
%できるだけ満たしたい重み付きソフト制約から構成される.
%違反するソフト制約の重み(ペナルティ)の総和の最小化が目的となる.
%解集合プログラミングは,この問題に対し,
%ASP ソルバーの系統的探索を活かして,未解決問題の最適値決定を含む優れた
%性能を示している.

\textbf{Curriculum-Based Course TimeTabling} (CB-CTT) 
is one of the most successful 
ASP applications. 
CB-CTT problem is 
a difficult combinatorial problem 
defined as the task of 
making a weekly schedule of 
lectures in universities. 
The problem is composed of 
hard constraints which must be strictly satisfied 
and soft ones which are not necessarily satisfied. 
The objective of the problem is 
to minimize the sum of soft constraints' 
violations (penalty). 
For this problem, 
ASP shows an excellent performance 
including optimal value determination 
of open problems, 
taking advantage of its systematic search.

%しかし,その一方で,ソフト制約の種類が多い問題集においては,
%確率的局所探索に基づくメタ戦略が,
%多くの問題に対してより高精度な解を求めている.
%以上から,
%系統的探索の長所である\textbf{``最適性の保証''}と確率的局所探索の長所である
%\textbf{``計算時間相応の解精度''}の両方を備えた統合的探索手法を実現することは
%重要な研究課題といえる.

On the other hand, 
meta-heuristics based on 
stochastic local search 
find better solutions 
at instances with many kinds of 
soft constraints. 
Therefore, 
To actualize integrated search method 
with both strength of systematic search 
\textbf{``certificate optimality''} 
and strength of stochastic local search 
\textbf{``appropriate solution for computation time''} 
is important research task.

%本論文では,解集合プログラミングに基づく
%\textbf{優先度付き巨大近傍探索}
%(Large Neighborhood Prioritized Search; LNPS)
%の実装,および,
%開発したソルバー \textit{asprior}の性能評価について述べる.
%LNPSは,
%ASP の系統的探索と
%メタ戦略の一種である巨大近傍探索
%(Large Neighborhood Search; LNS)
%を統合した探索手法である.
%LNS は解に含まれる変数の値割当ての一部をランダムに選んで取り消し,
%その変数のみに対して再割当てを行うことで解を再構築する反復解法である.
%これに対して,
%LNPS は,解の再構築の操作を,値割当てをなるべく維持したまま
%での再探索に置き換えることで,取り消されなかった変数への再割当てを許す.
%これによって,どの値割当てを取り消すかに依存しすぎない探索を行うことが
%できる点が特長である.
%LNPS は,メタ戦略と同様に,近接最適性が成り立つ問題に対する有効性が期
%待できる.また,各反復の終了条件を適切に設定することで解の最適性も保証で
%きる.
%開発した \textit{asprior} は,
%ASP ファクト形式の問題インスタンスと
%問題を解く ASP 符号化を入力とし,
%LNPS アルゴリズムを用いて解を求める汎用的なソルバーである.
%\textit{asprior} は,
%高速 ASP ソルバー {\textit clingo}
%の Python インターフェースを利用して実装されている.

In this paper, 
we discuss implementation of 
ASP-based \textbf{Large Neighborhood Prioritized Search} (LNPS) 
and evaluation of developed solver \textit{asprior}. 
LNPS is a search method 
integrating systematic search of ASP with 
one of meta heuristics, Large Neighborhood Search (LNS). 
LNS cancels assignments of 
some variables in a solution, 
and rebuild it by reassigning only canceled variables. 
In contrast, LNPS 
allows variables which is not canceled 
to be reassigned 
by replacing rebuilding with 
re-search which just keeps assignments 
as much as possible. 
By this, we can search 
which is not too dependent on 
canceled variables. 
LNPS as well as meta heuristics 
can be expected effectiveness to 
problems which have 
proximal optimality. 
Moreover, LNPS can certify 
optimality of solutions by 
setting appropriate end conditions of each iteration. 
A general-purpose solver \textit{asprior} finds solutions 
with LNPS algorithm, 
taking a instance in the form of ASP facts and 
ASP encoding for problems as input. 
We implement \textit{asprior} using 
Python interface of efficient ASP solver \textit{clingo}.

%提案手法を評価するために,国際時間割競技会 ITC2007
%で使用された問題を含む CB-CTT 問題集(全61問)を用いて性能評価を行った.
%その結果,
%既知の最良値との比について,
%通常の ASP 解法が$+472\%$であったのに対し,
%提案手法は,その比を$+53\%$まで大幅に改善できた.
%さらに,11問について,既知の最良値を更新することに成功した.

We carried out experiments on CB-CTT problems (61 instances)
including ITC2007 problems 
in order to evaluate our approach. 
As a result, 
the proposed approach greatly improves 
the ratio to known upper bounds to $+53\%$ 
from existing ASP solving's $+472\%$. 
Moreover, it successfully 
found new upper bounds of 11 instances.

\end{document}

% LocalWords:  PDNP PDNRP reachability PDNRPs PDNPs clingo encodings
% LocalWords:  acyclicity DNET fukui tepco
