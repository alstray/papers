\documentclass[dvipdfmx,a4paper]{jsarticle}

\title{\vspace{-3cm}解集合プログラミングに基づく系統的探索と確率的局所探索の統合的手法に関する研究}
\author{氏名:桑原 和也}
\date{学生番号:252005066}

\usepackage{url}
\usepackage{graphicx}
\usepackage{multirow}
\pagestyle{empty}

\setlength{\textheight}{247truemm}

\begin{document}
\maketitle

本論文では,解集合プログラミング(Answer Set Programming; ASP)に基づく
優先度付き巨大近傍探索 (Large Neighborhood Prioritized Search; LNPS)
の実装,および,開発したソルバー\textit{asprior}の性能評価について述べる.
%
LNPSは,ASPの系統的探索と
メタ戦略の一種である巨大近傍探索
(Large Neighborhood Search; LNS)
を統合した探索手法である.
%
PisingerらのLNSは解に含まれる変数の値割当ての一部をランダムに選んで取り消し,
その変数のみに対して再割当てを行うことで解を再構築する反復解法である.
これに対して,
LNPSは,解の再構築の操作を,値割当てをなるべく維持したまま
での再探索に置き換えることで,取り消されなかった変数への再割当てを許す.
これによって,どの値割当てを取り消すかに依存しすぎない探索を行うことが
できる点が特長である.
LNPSは,メタ戦略と同様に,近接最適性が成り立つ問題に対する有効性が期
待できる.また,各反復の終了条件を適切に設定することで解の最適性も保証で
きる.
開発した \textit{asprior}は,
ASPファクト形式の問題インスタンスと
問題を解くASP符号化を入力とし,
LNPSアルゴリズムを用いて解を求める汎用的なソルバーである.
\textit{asprior}は,
高速ASPソルバー\textit{clingo}
のPythonインターフェースを利用して実装されている.
%
提案手法を評価するために,国際時間割競技会ITC2007
で使用された問題を含むCB-CTT問題集(全61問)を用いて性能評価を行った.
その結果,
既知の最良値との比について,
通常のASP解法が$+472\%$であったのに対し,
提案手法は,その比を$+53\%$まで大幅に改善できた.
さらに,11問について,既知の最良値を更新することに成功した.

\begin{table}[h]
 \centering
 \footnotesize
 \caption{ITC2007問題集: 他のアプローチとの比較}
{
\begin{tabular}{c|r|rr|rr}
%\noalign{\hrule height 1pt}
\hline
    \multirow{2}{*}{問題名} & \multirow{2}{*}{既知の最良値($\sharp$)} & 
    \multicolumn{2}{c|}{既存手法ASP} & \multicolumn{2}{c}{提案手法LNPS}\\\cline{3-6}
    &  & 最良値 & $\sharp$との比(\%) & 最良値 & $\sharp$との比(\%)\\\hline
%\noalign{\hrule height 1pt}
{comp01} & 11 & 129 & +1,072 & 11 & 0\\
{comp02} & 130 & 331 & +154 & 172 & +32\\
{comp03} & 142 & 302 & +112 & 143 & +1\\
{comp04} & 49 & 49 & 0 & 49 & 0\\
{comp05} & 570 & 1,940 & +240 & 776 & +36\\
{comp06} & 85 & 822 & +867 & 102 & +20\\
{comp07} & 42 & 924 & +2,100 & \bf{40} & \bf{-5}\\
{comp08} & 55 & 55 & 0 & 55 & 0\\
{comp09} & 150 & 254 & +69 & \bf{138} & \bf{-8}\\
{comp10} & 72 & 822 & +1,041 & 80 & +11\\
{comp11} & 0 & 0 & 0 & 0 & 0\\
{comp12} & 483 & 1,246 & +157 & 664 & +37\\
{comp13} & 147 & 301 & +104 & \bf{146} & \bf{-1}\\
{comp14} & 67 & 67 & 0 & 83 & +23\\
{comp15} & 176 & 607 & +244 & 198 & +13\\
{comp16} & 96 & 944 & +883 & 134 & +40\\
{comp17} & 155 & 412 & +165 & 184 & +19\\
{comp18} & 137 & 471 & +243 & \bf{136} & \bf{-1}\\
{comp19} & 125 & 890 & +612 & 144 & +15\\
{comp20} & 124 & 1,386 & +1,017 & 237 & +91\\
{comp21} & 151 & 310 & +105 & 161 & +7\\\hline
{$\sharp$との比の平均} & & & +437 & & +16\\\hline
%\noalign{\hrule height 1pt}
 \end{tabular}
}
\end{table}

\textbf{研究業績}
\begin{itemize}
 \small
 \item 発表.「解集合プログラミングに基づく系統的探索と確率的局所探索の統合的手法に関する一考察」.
	   2021年度人工知能学会全国大会(第35回).
 \item 発表予定.「解集合プログラミングを用いた優先度付き巨大近傍探索の実装と評価 」.
      2022年情報処理学会第84回全国大会.
\end{itemize}
\end{document}