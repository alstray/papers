\section{評価実験}

%%%%%%%%%%%%%%%%%%%%%%%%%%%%%%%%%%%%%%%%%%%%%%
\begin{table*}[t]\centering
  \caption{実験結果: 得られた最適値と最良値}
  \vskip 1em
  \label{table:bench:result1}
  \begin{tableA}
    {comp01} & 129 & 13 & 13 & \bf{11} & 13 & \bf{11} & \bf{11} & \bf{11}\\
{comp02} & 331 & 334 & 239 & 273 & \bf{172} & 242 & 201 & 245\\
{comp03} & 302 & 173 & 177 & 154 & 149 & 173 & \bf{146} & 157\\
{comp04} & ${}^\ast$\bf{49} & ${}^\ast$\bf{49} & ${}^\ast$\bf{49} & ${}^\ast$\bf{49} & ${}^\ast$\bf{49} & ${}^\ast$\bf{49} & ${}^\ast$\bf{49} & ${}^\ast$\bf{49}\\
{comp05} & 1940 & 1124 & 922 & \bf{797} & 926 & 1116 & 891 & 861\\
{comp06} & 822 & 220 & 166 & 135 & 140 & 204 & 118 & \bf{106}\\
{comp07} & 924 & 225 & 131 & 137 & 118 & 129 & \bf{72} & 77\\
{comp08} & ${}^\ast$\bf{55} & ${}^\ast$\bf{55} & ${}^\ast$\bf{55} & ${}^\ast$\bf{55} & ${}^\ast$\bf{55} & ${}^\ast$\bf{55} & ${}^\ast$\bf{55} & ${}^\ast$\bf{55}\\
{comp09} & 254 & 155 & 154 & 158 & 149 & 146 & \bf{145} & 151\\
{comp10} & 822 & 231 & 220 & 177 & 167 & 133 & \bf{97} & 109\\
{comp11} & ${}^\ast$\bf{0} & ${}^\ast$\bf{0} & ${}^\ast$\bf{0} & ${}^\ast$\bf{0} & ${}^\ast$\bf{0} & ${}^\ast$\bf{0} & ${}^\ast$\bf{0} & ${}^\ast$\bf{0}\\
{comp12} & 1246 & 834 & 740 & 728 & 705 & \bf{694} & 756 & 809\\
{comp13} & 301 & 183 & 180 & 163 & 168 & 165 & \bf{151} & 155\\
{comp14} & ${}^\ast$\bf{67} & 189 & 186 & 145 & 179 & 165 & 104 & 83\\
{comp15} & 607 & 244 & 238 & \bf{213} & 234 & 238 & 215 & 224\\
{comp16} & 944 & 197 & \bf{156} & 178 & 180 & 162 & 223 & 158\\
{comp17} & 412 & 254 & 226 & 259 & 230 & 234 & \bf{199} & 208\\
{comp18} & 471 & 191 & 170 & 168 & 152 & 158 & \bf{144} & 149\\
{comp19} & 890 & 231 & 192 & 197 & 187 & 219 & 176 & \bf{163}\\
{comp20} & 1386 & 373 & 274 & 356 & 280 & 305 & 293 & \bf{265}\\
{comp21} & 310 & 285 & 219 & 202 & 222 & 235 & 192 & \bf{178}\\\hline
{\#最適値・最良値} & 4 & 3 & 4 & 6 & 4 & 5 & \bf{11} & 8\\

%%% Local Variables:
%%% mode: japanese-latex
%%% TeX-master: "../paper"
%%% End:

  \end{tableA}
\end{table*}
%%%%%%%%%%%%%%%%%%%%%%%%%%%%%%%%%%%%%%%%%%%%%%
\begin{table*}[t]\centering
  \caption{他のアプローチとの比較}
  \vskip 1em  
  \label{table:bench:result2}
  \begin{tableB}
    {comp01} & 11 & 129 & +1,072 & 11 & 0\\
{comp02} & 130 & 331 & +154 & 172 & +32\\
{comp03} & 142 & 302 & +112 & 146 & +2\\
{comp04} & 49 & 49 & 0 & 49 & 0\\
{comp05} & 570 & 1,940 & +240 & 797 & +39\\
{comp06} & 85 & 822 & +867 & 106 & +24\\
{comp07} & 42 & 924 & +2,100 & 72 & +71\\
{comp08} & 55 & 55 & 0 & 55 & 0\\
{comp09} & 150 & 254 & +69 & 145 & \bf{-3}\\
{comp10} & 72 & 822 & +1,041 & 97 & +34\\
{comp11} & 0 & 0 & 0 & 0 & 0\\
{comp12} & 483 & 1,246 & +157 & 694 & +43\\
{comp13} & 147 & 301 & +104 & 151 & +2\\
{comp14} & 67 & 67 & 0 & 83 & +23\\
{comp15} & 176 & 607 & +244 & 213 & +21\\
{comp16} & 96 & 944 & +883 & 156 & +62\\
{comp17} & 155 & 412 & +165 & 199 & +28\\
{comp18} & 137 & 471 & +243 & 144 & +5\\
{comp19} & 125 & 890 & +612 & 163 & +30\\
{comp20} & 124 & 1,386 & +1,017 & 265 & +113\\
{comp21} & 151 & 310 & +105 & 178 & +17\\\hline
{$\sharp$との比の平均} & & & +437 & & +26\\
  \end{tableB}
\end{table*}
%%%%%%%%%%%%%%%%%%%%%%%%%%%%%%%%%%%%%%%%%%%%%%

提案手法の有効性を評価するために実行実験を行った.
ベンチマーク問題には,
国際時間割競技会ITC2007~\footnote{%
  \url{http://www.cs.qub.ac.uk/itc2007/}}
で公開されているカリキュラムベース・コース時間割(CB-CTT)の
問題集(全21問)に対し,ソフト制約が最も多い UD5 を使って
評価を行なった~\cite{GasperoMS/ITC2007,DBLP:journals/anor/BonuttiCGS12}.
%
比較した手法は以下の2つである.
\begin{itemize}\compress
\item 既存手法: ASPソルバー{\clingo}
\item 提案手法: LNPSを{\clingo}上に実装
\end{itemize}

CB-CTT 問題を解くための ASP 符号化には,
\textsf{teaspoon}符号化~\cite{anor/Banbara2019}
を使用した.
LNPS の$destroy$と$re\mathchar`-search$については,
CB-CTTの既存研究~\cite{anor/Kiefer2017}を参考に,
以下の4種類を実装した.
\begin{itemize}\compress
\item \textsf{Random} $N$ (\textsf{R-$N$})\\
  暫定解から変数の値割当ての$N$\%をランダムに選んで取り消す.
\item \textsf{Day-Period} (\textsf{DP})\\
  曜日\textsf{D}と時限\textsf{P}をランダムに1組選び,
  暫定解から\textsf{D}曜\textsf{P}限に関する
  変数の値割当てをすべて取り消す.
\item \textsf{Day-Room} (\textsf{DR})\\
  曜日\textsf{D}と教室\textsf{R}をランダムに1組選び,
  暫定解から\textsf{D}曜日の\textsf{R}教室に関する
  変数の値割当てをすべて取り消す.
\item \textsf{Swap-Room} $N$ (\textsf{SR-$N$})\\
  \textsf{Random} $N$ と同様に,
  暫定解から変数の値割当ての$N$\%をランダムに選んで取り消す.
  ただし,科目に対する曜日と時限の割当てはできるだけ維持する.
\end{itemize}

既存手法と提案手法ともに,
ASPソルバーには{\clingo}-5.4.0を利用し,
1問あたりの制限時間は1時間とした.
実験環境は,Mac OS, 3.2GHz Intel Core i7, 64GB メモリである.

表~\ref{table:bench:result1}に,各手法で得られた最適値および最良値を示す.
各問題ごとに最も良い値を太字で表している.
`$\ast$'付きの値は最適値を意味する.
提案手法は,1問を除くすべての問題に対して,既存手法と同じかより良い解
を得ている.
各手法について,最適値および最良値を求めた問題数を比較すると,
提案手法\textsf{SR-5}が11問と最も多く,次いで
\textsf{SR-10}が8問と,
\textsf{Swap-Room}を用いた手法が優れた性能を示した.


表~\ref{table:bench:result2}に,他のアプローチとの比較結果を示す.
左から順に,
問題名,
既知の最良値($\sharp$),
既存手法 ASP の最良値と$\sharp$との比,
提案手法 LNPS の最良値と$\sharp$との比が示されている.
既知の最良値は,これまで,メタ戦略に基づく各種アルゴリズム,整数計画法,
SAT, MaxSAT, SMT など様々な手法で求められた結果である~\cite{anor/Banbara2019}.
提案手法の最良値は,各問題に対して,表~\ref{table:bench:result1}中の提
案手法(7種類)で得られた値の中で最も良い値を示している.
また,既知の最良値との比は,以下の計算結果を百分率で表したものである.
\begin{align*}
既知の&最良値との比\\
&= \frac{得られた最良値 - 既知の最良値}{既知の最良値}
\end{align*}

既存手法 ASP は,既知の最良値との比が$+437\%$と高く,解の精度が悪いこ
とが確認できる.
一方,提案手法 LNPS は,既知の最良値との比が$+26\%$まで抑えられており,
既存手法と比較して約16倍改善されている.
さらに,提案手法は,comp09 について,既知の最良値を更新することに成功
した.

%%% Local Variables:
%%% mode: japanese-latex
%%% TeX-master: "paper"
%%% End:
