\section{評価実験}

%%%%%%%%%%%%%%%%%%%%%%%%%%%%%%%%%%%%%%%%%%%%%%
\begin{table*}[t]\centering
  \caption{実験結果: 得られた最適値と最良値}
  \vskip 1em
  \label{table:bench:result1}
  \begin{tableA}
    {comp01} & 129 & 283 & 13 & \bf{11} & \bf{11} & \bf{11} & \bf{11} & \bf{11} & \bf{11} & \bf{11} & \bf{11} & \bf{11}\\
{comp02} & 1049 & 331 & 259 & 239 & 199 & \bf{172} & 235 & 201 & 213 & 260 & 227 & 236\\
{comp03} & 791 & 302 & 173 & 173 & 154 & 149 & 149 & 144 & \bf{143}& 145 & 154 & 144\\
{comp04} & 231 & ${}^\ast$\bf{49} & ${}^\ast$\bf{49} & ${}^\ast$\bf{49} & ${}^\ast$\bf{49} & ${}^\ast$\bf{49} & ${}^\ast$\bf{49} & ${}^\ast$\bf{49} & ${}^\ast$\bf{49} & ${}^\ast$\bf{49} & ${}^\ast$\bf{49} & ${}^\ast$\bf{49}\\
{comp05} & 2662 & 1940 & 1102 & 922 & 797 & 864 & 841 & 891 & 861 & 994 & \bf{776} & 907\\
{comp06} & 822 & 1025 & 216 & 162 & 135 & 135 & 166 & 112 & 106 & 119 & \bf{102} & 123\\
{comp07} & 924 & 1149 & 153 & 131 & 114 & 99 & 105 & 60 & 65 & 56 & 74 & \bf{40}\\
{comp08} & 348 & ${}^\ast$\bf{55} & ${}^\ast$\bf{55} & ${}^\ast$\bf{55} & ${}^\ast$\bf{55} & ${}^\ast$\bf{55} & ${}^\ast$\bf{55} & ${}^\ast$\bf{55} & ${}^\ast$\bf{55} & ${}^\ast$\bf{55} & ${}^\ast$\bf{55} & ${}^\ast$\bf{55}\\
{comp09} & 617 & 254 & 154 & 154 & 151 & 143 & 146 & 145 & 146& 141 & \bf{138} & 141\\
{comp10} & 822 & 1229 & 209 & 166 & 141 & 124 & 133 & 97 & \bf{80} & 97 & 98 & 101\\
{comp11} & 287 & ${}^\ast$\bf{0} & ${}^\ast$\bf{0} & ${}^\ast$\bf{0} & ${}^\ast$\bf{0} & ${}^\ast$\bf{0} & ${}^\ast$\bf{0} & ${}^\ast$\bf{0} & ${}^\ast$\bf{0} & ${}^\ast$\bf{0} & ${}^\ast$\bf{0} & ${}^\ast$\bf{0}\\
{comp12} & 2626 & 1246 & 787 & 740 & 728 & 705 & 694 & 729 & \bf{664} & 687 & 718 & 702\\
{comp13} & 661 & 301 & 171 & 158 & 163 & 168 & 165 & 147 & 152 & 147 & 149 & \bf{146}\\
{comp14} & 748 & ${}^\ast$\bf{67} & 189 & 156 & 145 & 143 & 158 & 104 & 83 & 123 & 130 & 128\\
{comp15} & 852 & 607 & 232 & 214 & 213 & 227 & 206 & 210 & 205 & \bf{198} & 212 & 213\\
{comp16} & 944 & 1090 & 197 & 156 & 168 & 140 & 162 & 167 & 151 & \bf{134} & 149 & 172\\
{comp17} & 979 & 412 & 244 & 226 & 211 & 209 & 214 & 196 & 204 & 200 & \bf{184} & 203\\
{comp18} & 673 & 471 & 180 & 156 & 148 & 151 & 155 & 140 & 149 & 146 & \bf{136} & 149\\
{comp19} & 890 & 919 & 231 & 192 & 190 & 165 & 199 & 176 & 163 & \bf{144} & 174 & 171\\
{comp20} & 3304 & 1386 & 373 & 274 & 356 & 268 & 273 & 272 & 246 & \bf{237} & 283 &291\\
{comp21} & 893 & 310 & 234 & 210 & 202 & 222 & 214 & 166 & 167 & \bf{161} & 192 &170\\\hline
{\#最適値} & \lw{0} & \lw{4} & \lw{3} & \lw{4} & \lw{4} & \lw{5} & \lw{4} & \lw{4} & \lw{7} & \lw{9} & \lw{9} & \lw{6}\\
{・最良値} & & & & & & & & & & & &\\

%%% Local Variables:
%%% mode: japanese-latex
%%% TeX-master: "../paper"
%%% End:

  \end{tableA}
\end{table*}
%%%%%%%%%%%%%%%%%%%%%%%%%%%%%%%%%%%%%%%%%%%%%%
\begin{table*}[t]\centering
  \caption{他のアプローチとの比較}
  \vskip 1em  
  \label{table:bench:result2}
  \begin{tableB}
    {comp01} & 11 & 129 & +1,072 & 11 & 0\\
{comp02} & 130 & 331 & +154 & 172 & +32\\
{comp03} & 142 & 302 & +112 & 143 & +1\\
{comp04} & 49 & 49 & 0 & 49 & 0\\
{comp05} & 570 & 1,940 & +240 & 776 & +36\\
{comp06} & 85 & 822 & +867 & 102 & +20\\
{comp07} & 42 & 924 & +2,100 & \alert{\bf 40} & \alert{\bf -5}\\
{comp08} & 55 & 55 & 0 & 55 & 0\\
{comp09} & 150 & 254 & +69 & \alert{\bf 138} & \alert{\bf -8}\\
{comp10} & 72 & 822 & +1,041 & 80 & +11\\
{comp11} & 0 & 0 & 0 & 0 & 0\\
{comp12} & 483 & 1,246 & +157 & 664 & +37\\
{comp13} & 147 & 301 & +104 & \alert{\bf 146} & \alert{\bf -1}\\
{comp14} & 67 & 67 & 0 & 83 & +23\\
{comp15} & 176 & 607 & +244 & 198 & +13\\
{comp16} & 96 & 944 & +883 & 134 & +40\\
{comp17} & 155 & 412 & +165 & 184 & +19\\
{comp18} & 137 & 471 & +243 & \alert{\bf 136} & \alert{\bf -1}\\
{comp19} & 125 & 890 & +612 & 144 & +15\\
{comp20} & 124 & 1,386 & +1,017 & 237 & +91\\
{comp21} & 151 & 310 & +105 & 161 & +7\\\hline
{$\sharp$との比の平均} & & & +437 & & +16\\\hline
  \end{tableB}
\end{table*}
%%%%%%%%%%%%%%%%%%%%%%%%%%%%%%%%%%%%%%%%%%%%%%

提案手法の有効性を評価するために実行実験を行った.
ベンチマーク問題には,
国際時間割競技会ITC2007~\footnote{%
  \url{http://www.cs.qub.ac.uk/itc2007/}}
で公開されているカリキュラムベース・コース時間割(CB-CTT)の
問題集(全21問)に対し,ソフト制約が最も多い UD5 を使って
評価を行なった~\cite{GasperoMS/ITC2007,DBLP:journals/anor/BonuttiCGS12}.
%
比較した手法は以下の2つである.
\begin{itemize}\compress
\item 既存手法: ASPソルバー{\clingo}
\item 提案手法: LNPSを{\clingo}上に実装
\end{itemize}

CB-CTT 問題を解くための ASP 符号化には,
\textsf{teaspoon}符号化~\cite{anor/Banbara2019}
を使用した.
LNPS の$destroy$と$re\mathchar`-search$については,
CB-CTTの既存研究~\cite{anor/Kiefer2017}を参考に,
以下の4種類を実装した.
\begin{itemize}\compress
\item \textsf{Random} $N$ (\textsf{R-$N$})\\
  暫定解から変数の値割当ての$N$\%をランダムに選んで取り消す.
\item \textsf{Day-Period} (\textsf{DP})\\
  曜日\textsf{D}と時限\textsf{P}をランダムに1組選び,
  暫定解から\textsf{D}曜\textsf{P}限に関する
  変数の値割当てをすべて取り消す.
\item \textsf{Day-Room} (\textsf{DR})\\
  曜日\textsf{D}と教室\textsf{R}をランダムに1組選び,
  暫定解から\textsf{D}曜日の\textsf{R}教室に関する
  変数の値割当てをすべて取り消す.
\item \textsf{Swap-Room} $N$ (\textsf{SR-$N$})\\
  \textsf{Random} $N$ と同様に,
  暫定解から変数の値割当ての$N$\%をランダムに選んで取り消す.
  ただし,科目に対する曜日と時限の割当てはできるだけ維持する.
\end{itemize}

既存手法と提案手法ともに,
ASPソルバーには{\clingo}-5.4.0を利用し,
1問あたりの制限時間は1時間とした.
実験環境は,Mac OS, 3.2GHz Intel Core i7, 64GB メモリである.

表~\ref{table:bench:result1}に,各手法で得られた最適値および最良値を示す.
各問題ごとに最も良い値を太字で表している.
`$\ast$'付きの値は最適値を意味する.
提案手法は,1問を除くすべての問題に対して,既存手法と同じかより良い解
を得ている.
各手法について,最適値および最良値を求めた問題数を比較すると,
提案手法\textsf{SR-5}が11問と最も多く,次いで
\textsf{SR-10}が8問と,
\textsf{Swap-Room}を用いた手法が優れた性能を示した.


表~\ref{table:bench:result2}に,他のアプローチとの比較結果を示す.
左から順に,
問題名,
既知の最良値($\sharp$),
既存手法 ASP の最良値と$\sharp$との比,
提案手法 LNPS の最良値と$\sharp$との比が示されている.
既知の最良値は,これまで,メタ戦略に基づく各種アルゴリズム,整数計画法,
SAT, MaxSAT, SMT など様々な手法で求められた結果である~\cite{anor/Banbara2019}.
提案手法の最良値は,各問題に対して,表~\ref{table:bench:result1}中の提
案手法(7種類)で得られた値の中で最も良い値を示している.
また,既知の最良値との比は,以下の計算結果を百分率で表したものである.
\begin{align*}
既知の&最良値との比\\
&= \frac{得られた最良値 - 既知の最良値}{既知の最良値}
\end{align*}

既存手法 ASP は,既知の最良値との比が$+437\%$と高く,解の精度が悪いこ
とが確認できる.
一方,提案手法 LNPS は,既知の最良値との比が$+26\%$まで抑えられており,
既存手法と比較して約16倍改善されている.
さらに,提案手法は,comp09 について,既知の最良値を更新することに成功
した.

%%% Local Variables:
%%% mode: japanese-latex
%%% TeX-master: "paper"
%%% End:
