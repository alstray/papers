%\section{はじめに}
%%%%%%%%%%%%%%%%%%%%%%%%%%%%%%%%%%%%%%%%%%%%%%
\begin{figure*}[t]
  \centering
  \thicklines
  \setlength{\unitlength}{1.28pt}
  \small
  \begin{picture}(280,57)(4,-10)
    \put( -35, 20){\dashbox(70,24){\shortstack{組合せ最適化問題\\のインスタンス}}}
    \put( 45, 20){\framebox(50,24){変換器}}
    \put(105, 20){\dashbox(70,24){\shortstack{ASPファクト}}}
    \put(105,-10){\dashbox(70,24){\shortstack{ASP符号化\\(論理プログラム)}}}
    \put(185,-10){\framebox(60,54){}}
    \put(189, 25){\framebox(52,12){ASPソルバー}}
    \put(190, -5){\framebox(50,12){LNPS}}
    % \put(180, 20){\framebox(50,24){ASPシステム}}
    \put(255, 20){\dashbox(70,24){\shortstack{組合せ最適化問題\\の最適解}}}
    \put(  35, 32){\vector(1,0){10}}
    \put(  95, 32){\vector(1,0){10}}
    \put(175, 32){\vector(1,0){10}}
    \put(245, 32){\vector(1,0){10}}
    \put(175, +2){\line(1,0){4}}
    \put(179, +2){\line(0,1){30}}
    \put(205,  7){\vector(0,1){17}}
    \put(225, 24){\vector(0,-1){17}}
    \put(190, 48){提案ソルバー}
  \end{picture}  
\caption{提案ソルバー\textit{asprior}の構成}
\label{fig:arch}
\end{figure*}

%%% Local Variables: 
%%% mode: latex
%%% TeX-master: "paper"
%%% End: 

%%%%%%%%%%%%%%%%%%%%%%%%%%%%%%%%%%%%%%%%%%%%%%
\begin{table*}[t]\centering
  \caption{他のアプローチとの比較結果}
%  \vskip 1em  
  \label{table:bench:result2}
  \begin{tableB}
    {comp01} & 11 & 129 & +1,072 & 11 & 0\\
{comp02} & 130 & 331 & +154 & 172 & +32\\
{comp03} & 142 & 302 & +112 & 143 & +1\\
{comp04} & 49 & 49 & 0 & 49 & 0\\
{comp05} & 570 & 1,940 & +240 & 776 & +36\\
{comp06} & 85 & 822 & +867 & 102 & +20\\
{comp07} & 42 & 924 & +2,100 & \alert{\bf 40} & \alert{\bf -5}\\
{comp08} & 55 & 55 & 0 & 55 & 0\\
{comp09} & 150 & 254 & +69 & \alert{\bf 138} & \alert{\bf -8}\\
{comp10} & 72 & 822 & +1,041 & 80 & +11\\
{comp11} & 0 & 0 & 0 & 0 & 0\\
{comp12} & 483 & 1,246 & +157 & 664 & +37\\
{comp13} & 147 & 301 & +104 & \alert{\bf 146} & \alert{\bf -1}\\
{comp14} & 67 & 67 & 0 & 83 & +23\\
{comp15} & 176 & 607 & +244 & 198 & +13\\
{comp16} & 96 & 944 & +883 & 134 & +40\\
{comp17} & 155 & 412 & +165 & 184 & +19\\
{comp18} & 137 & 471 & +243 & \alert{\bf 136} & \alert{\bf -1}\\
{comp19} & 125 & 890 & +612 & 144 & +15\\
{comp20} & 124 & 1,386 & +1,017 & 237 & +91\\
{comp21} & 151 & 310 & +105 & 161 & +7\\\hline
{$\sharp$との比の平均} & & & +437 & & +16\\\hline
  \end{tableB}
\end{table*}
%%%%%%%%%%%%%%%%%%%%%%%%%%%%%%%%%%%%%%%%%%%%%%


\textbf{解集合プログラミング}
(Answer Set Programming; ASP
\cite{DBLP:conf/iclp/GelfondL88})は,
論理プログラミングから派生した宣言的プログラミングパラダイムである.
ASP言語は一階論理に基づく知識表現言語の一種である.
ASP ソルバーは安定モデル意味論~\cite{DBLP:conf/iclp/GelfondL88}に基づ
く解集合を計算するプログラムである.
近年,SAT 技術を応用した高速 ASP ソルバーが実現され,
人工知能分野の諸問題を中心に実用的応用が急速に拡大している.%~\cite{ergele16a}.

解集合プログラミングが成功した応用事例の一つに,
\textbf{カリキュラムベース・コース時間割}
(Curriculum-Based Course TimeTabling; CB-CTT)がある.
CB-CTT は大学等での一週間の講義スケジュールを作成する
求解困難な組合せ最適化問題である.
CB-CTT は必ず満たすべきハード制約と,できるだけ満たしたい
重み付きソフト制約から構成される.
違反するソフト制約の重みの総和の最小化が目的となる.
解集合プログラミングは,この問題に対し,
ASP ソルバーの系統的探索を活かして,未解決問題の最適値決定を含む優れた
性能を示している~\cite{anor/Banbara2019}.

しかし,その一方で,ソフト制約の種類が多い問題集においては,
確率的局所探索に基づくメタ戦略が,多くの問題に対してより高精度な解を求めている.
以上から,
系統的探索の長所である``最適性の保証''と確率的局所探索の長所である
``計算時間相応の解精度''の両方を備えた統合的探索手法を実現することは
重要な研究課題といえる.

本発表では,ASP 技術を用いた
\textbf{優先度付き巨大近傍探索}
(Large Neighborhood Prioritized Search; LNPS)
の実装,および,
開発したソルバー \textit{asprior}の性能評価について述べる.

先行研究で提案した LNPS~\cite{jsai2021:kutaba}は,
ASP の系統的探索と
メタ戦略の一種である巨大近傍探索
(Large Neighborhood Search; LNS~\cite{Pisinger10})
を統合した探索手法である.
%
LNS は解に含まれる変数の値割当ての一部をランダムに選んで取り消し,
その変数のみに対して再割当てを行うことで解を再構築する反復解法である.
これに対して,
LNPS は,解の再構築の操作を,値割当てをなるべく維持したまま
での再探索に置き換えることで,取り消されなかった変数への再割当てを許す.
これによって,どの値割当てを取り消すかに依存しすぎない探索を行うことが
できる点が特長である.
% LNPS は,メタ戦略と同様に,近接最適性が成り立つ問題に対する有効性が期
% 待できる.また,各反復の終了条件を適切に設定することで解の最適性も保証で
% きる.

開発した \textit{asprior} は,
ASP ファクト形式の問題インスタンスと
問題を解く ASP 符号化を入力とし,
LNPS アルゴリズムを用いて解を求める汎用的なソルバーである
(図~\ref{fig:arch}参照).
\textit{asprior} は,
高速 ASP ソルバー {\clingo}%~\footnote{\url{https://potassco.org/clingo/}}
の Python インターフェースを利用して実装されている.

提案手法を評価するために,
国際時間割競技会の CB-CTT 問題集(全21問)を用いて性能評価を行った.
その結果,
表~\ref{table:bench:result2}に示すように,
既知の最良値との比について,
通常の ASP 解法が$+437\%$であったのに対し,
提案手法は,その比を$+16\%$まで大幅に改善できた.
%
さらに,
comp07,
comp09,
comp13,
comp18
について,既知の最良値を更新することに成功した.
今後の課題としては,
アダプティブ LNPS への拡張や他の時間割問題への適用などが挙げられる.

%%%%%%%%%%%%%%%%%%%%%%%%%%%%%%%%%
% \begin{figure*}[t]
  \centering
  \thicklines
  \setlength{\unitlength}{1.28pt}
  \small
  \begin{picture}(280,57)(4,-10)
    \put( -35, 20){\dashbox(70,24){\shortstack{組合せ最適化問題\\のインスタンス}}}
    \put( 45, 20){\framebox(50,24){変換器}}
    \put(105, 20){\dashbox(70,24){\shortstack{ASPファクト}}}
    \put(105,-10){\dashbox(70,24){\shortstack{ASP符号化\\(論理プログラム)}}}
    \put(185,-10){\framebox(60,54){}}
    \put(189, 25){\framebox(52,12){ASPソルバー}}
    \put(190, -5){\framebox(50,12){LNPS}}
    % \put(180, 20){\framebox(50,24){ASPシステム}}
    \put(255, 20){\dashbox(70,24){\shortstack{組合せ最適化問題\\の最適解}}}
    \put(  35, 32){\vector(1,0){10}}
    \put(  95, 32){\vector(1,0){10}}
    \put(175, 32){\vector(1,0){10}}
    \put(245, 32){\vector(1,0){10}}
    \put(175, +2){\line(1,0){4}}
    \put(179, +2){\line(0,1){30}}
    \put(205,  7){\vector(0,1){17}}
    \put(225, 24){\vector(0,-1){17}}
    \put(190, 48){提案ソルバー}
  \end{picture}  
\caption{提案ソルバー\textit{asprior}の構成}
\label{fig:arch}
\end{figure*}

%%% Local Variables: 
%%% mode: latex
%%% TeX-master: "paper"
%%% End: 

%%%%%%%%%%%%%%%%%%%%%%%%%%%%%%%%%

% %%%%%%%%%%%%%%%%%%%%%%%%%%%%%%%%%%%%%%%%%%%%%%
% \begin{table*}[t]\centering
%   \caption{実験結果: 得られた最適値と最良値}
%   \vskip 1em
%   \label{table:bench:result1}
%   \begin{tableA}
%     {comp01} & 129 & 283 & 13 & \alert{\bf 11} & \alert{\bf 11} & \alert{\bf 11} & \alert{\bf 11} & \alert{\bf 11} & \alert{\bf 11} & \alert{\bf 11} & \alert{\bf 11} & \alert{\bf 11}\\
{comp02} & 1049 & 331 & 259 & 239 & 199 & \alert{\bf 172} & 235 & 201 & 213 & 260 & 227 & 236\\
{comp03} & 791 & 302 & 173 & 173 & 154 & 149 & 149 & 144 & \alert{\bf 143}& 145 & 154 & 144\\
{comp04} & 231 & ${}^\ast$\alert{\bf 49} & ${}^\ast$\alert{\bf 49} & ${}^\ast$\alert{\bf 49} & ${}^\ast$\alert{\bf 49} & ${}^\ast$\alert{\bf 49} & ${}^\ast$\alert{\bf 49} & ${}^\ast$\alert{\bf 49} & ${}^\ast$\alert{\bf 49} & ${}^\ast$\alert{\bf 49} & ${}^\ast$\alert{\bf 49} & ${}^\ast$\alert{\bf 49}\\
{comp05} & 2662 & 1940 & 1102 & 922 & 797 & 864 & 841 & 891 & 861 & 994 & \alert{\bf 776} & 907\\
{comp06} & 822 & 1025 & 216 & 162 & 135 & 135 & 166 & 112 & 106 & 119 & \alert{\bf 102} & 123\\
{comp07} & 924 & 1149 & 153 & 131 & 114 & 99 & 105 & 60 & 65 & 56 & 74 & \alert{\bf 40}\\
{comp08} & 348 & ${}^\ast$\alert{\bf 55} & ${}^\ast$\alert{\bf 55} & ${}^\ast$\alert{\bf 55} & ${}^\ast$\alert{\bf 55} & ${}^\ast$\alert{\bf 55} & ${}^\ast$\alert{\bf 55} & ${}^\ast$\alert{\bf 55} & ${}^\ast$\alert{\bf 55} & ${}^\ast$\alert{\bf 55} & ${}^\ast$\alert{\bf 55} & ${}^\ast$\alert{\bf 55}\\
{comp09} & 617 & 254 & 154 & 154 & 151 & 143 & 146 & 145 & 146& 141 & \alert{\bf 138} & 141\\
{comp10} & 822 & 1229 & 209 & 166 & 141 & 124 & 133 & 97 & \alert{\bf 80} & 97 & 98 & 101\\
{comp11} & 287 & ${}^\ast$\alert{\bf 0} & ${}^\ast$\alert{\bf 0} & ${}^\ast$\alert{\bf 0} & ${}^\ast$\alert{\bf 0} & ${}^\ast$\alert{\bf 0} & ${}^\ast$\alert{\bf 0} & ${}^\ast$\alert{\bf 0} & ${}^\ast$\alert{\bf 0} & ${}^\ast$\alert{\bf 0} & ${}^\ast$\alert{\bf 0} & ${}^\ast$\alert{\bf 0}\\
{comp12} & 2626 & 1246 & 787 & 740 & 728 & 705 & 694 & 729 & \alert{\bf 664} & 687 & 718 & 702\\
{comp13} & 661 & 301 & 171 & 158 & 163 & 168 & 165 & 147 & 152 & 147 & 149 & \alert{\bf 146}\\
{comp14} & 748 & ${}^\ast$\alert{\bf 67} & 189 & 156 & 145 & 143 & 158 & 104 & 83 & 123 & 130 & 128\\
{comp15} & 852 & 607 & 232 & 214 & 213 & 227 & 206 & 210 & 205 & \alert{\bf 198} & 212 & 213\\
{comp16} & 944 & 1090 & 197 & 156 & 168 & 140 & 162 & 167 & 151 & \alert{\bf 134} & 149 & 172\\
{comp17} & 979 & 412 & 244 & 226 & 211 & 209 & 214 & 196 & 204 & 200 & \alert{\bf 184} & 203\\
{comp18} & 673 & 471 & 180 & 156 & 148 & 151 & 155 & 140 & 149 & 146 & \alert{\bf 136} & 149\\
{comp19} & 890 & 919 & 231 & 192 & 190 & 165 & 199 & 176 & 163 & \alert{\bf 144} & 174 & 171\\
{comp20} & 3304 & 1386 & 373 & 274 & 356 & 268 & 273 & 272 & 246 & \alert{\bf 237} & 283 &291\\
{comp21} & 893 & 310 & 234 & 210 & 202 & 222 & 214 & 166 & 167 & \alert{\bf 161} & 192 &170\\\hline
{\#最適値} & \multirow{2}{*}{0} & \multirow{2}{*}{4} & \multirow{2}{*}{3} & \multirow{2}{*}{4} & \multirow{2}{*}{4} & \multirow{2}{*}{5} & \multirow{2}{*}{4} & \multirow{2}{*}{4} & \multirow{2}{*}{7} & \multirow{2}{*}{\alert{\bf 9}} & \multirow{2}{*}{\alert{\bf 9}} & \multirow{2}{*}{6}\\
{・最良値} & & & & & & & & & & & &\\\hline

%%% Local Variables:
%%% mode: japanese-latex
%%% TeX-master: "../paper"
%%% End:

%   \end{tableA}
% \end{table*}


%%% Local Variables:
%%% mode: japanese-latex
%%% TeX-master: "paper"
%%% End:

