%%% Packages
\usepackage[dvipdfmx]{graphicx}
\usepackage{lineno}
%\usepackage{bm}
%\usepackage{array}
\usepackage{url}
\usepackage{alltt}
\usepackage{amsmath}
\usepackage{ascmac}
%\usepackage{tikz}
%\usetikzlibrary{arrows,shapes}
%\usetikzlibrary{positioning}
\usepackage{listings}
\usepackage{plistings}
\def\lstlistingname{コード}
\usepackage{color}

%%% For ASP
\newcommand{\asap}{\textit{teaspoon}}
\newcommand{\gringo}{\textit{gringo}}
\newcommand{\clingo}{\textit{clingo}}
\newcommand{\dlv}{\textit{DLV}}
\newcommand{\wasp}{\textit{WASP}}
\newcommand{\code}[1]{\lstinline[basicstyle=\ttfamily]{#1}}
\newcommand{\naf}[1]{\ensuremath{{\sim\!\!{#1}}}}
\newcommand{\head}[1]{\ensuremath{\mathit{head}(#1)}}
\newcommand{\body}[1]{\ensuremath{\mathit{body}(#1)}}
%\newcommand{\atom}[1]{\ensuremath{\mathit{atom}(#1)}}
\newcommand{\poslits}[1]{\ensuremath{{#1}^+}}
\newcommand{\neglits}[1]{\ensuremath{{#1}^-}}
\newcommand{\pbody}[1]{\poslits{\body{#1}}}
\newcommand{\nbody}[1]{\neglits{\body{#1}}}
%\newcommand{\Cn}[1]{\ensuremath{\mathit{Cn}(#1)}}
\newcommand{\reduct}[2]{\ensuremath{#1^{#2}}}%
% \newcommand{\web}[2]{\href{#1}{#2\ \raisebox{-0.15ex}{\beamergotobutton{Web}}}}
% \newcommand{\doi}[2]{\href{#1}{#2\ \raisebox{-0.15ex}{\beamergotobutton{DOI}}}}
% \newcommand{\weblink}[1]{\web{#1}{#1}}
% \newcommand{\imp}{\mathrel{\Rightarrow}}
% \newcommand{\Iff}{\mathrel{\Leftrightarrow}}
% \newcommand{\mybox}[1]{\fbox{\rule[.2cm]{0cm}{0cm}\mbox{${#1}$}}}
% \newcommand{\mycbox}[2]{\tikz[baseline]\node[fill=#1!10,anchor=base,rounded corners=2pt] () {#2};}
% \newcommand{\naf}[1]{\ensuremath{{\sim\!\!{#1}}}}
% \newcommand{\head}[1]{\ensuremath{\mathit{head}(#1)}}
% \newcommand{\body}[1]{\ensuremath{\mathit{body}(#1)}}
% \newcommand{\atom}[1]{\ensuremath{\mathit{atom}(#1)}}
% \newcommand{\poslits}[1]{\ensuremath{{#1}^+}}
% \newcommand{\neglits}[1]{\ensuremath{{#1}^-}}
% \newcommand{\pbody}[1]{\poslits{\body{#1}}}
% \newcommand{\nbody}[1]{\neglits{\body{#1}}}
% \newcommand{\Cn}[1]{\ensuremath{\mathit{Cn}(#1)}}
% \newcommand{\reduct}[2]{\ensuremath{#1^{#2}}}
% \newcommand{\OK}{\mbox{\textcolor{green}{\Pisymbol{pzd}{52}}}}
% \newcommand{\KO}{\mbox{\textcolor{red}{\Pisymbol{pzd}{56}}}}
% \newcommand{\code}[1]{\lstinline[basicstyle=\ttfamily]{#1}}

\newcommand{\lw}[1]{\smash{\lower1.ex\hbox{#1}}}
\newcommand{\llw}[1]{\smash{\lower3.ex\hbox{#1}}}
\newcommand{\compress}{\itemsep0pt\parsep0pt\parskip0pt\partopsep0pt}

%%% Table
\newenvironment{tableA}{%
  \small
%  \tabcolsep = 2mm
  %\renewcommand{\arraystretch}{1.1}
  \begin{tabular}{c|rr|rrrrrrrrrr}
    \lw{問題名} & \multicolumn{2}{c|}{既存手法ASP} & \multicolumn{7}{c}{提案手法LNPS}\\\cline{2-13}
     & bb & usc & \textsf{R-0} & \textsf{R-3} & \textsf{R-5} & \textsf{DP} & \textsf{DR} & \textsf{SR-5} & \textsf{SR-10} & \textsf{DPSR-1} & \textsf{DPSR-2} & \textsf{DPSR-3}\\\hline
    }{%
    \hline
  \end{tabular}
}

\newenvironment{tableB}{%
 \small
 % \tabcolsep = 2mm
 % \renewcommand{\arraystretch}{1.1}						
  \begin{tabular}[t]{c|r|rr|rr}
    \lw{問題名} & \lw{既知の最良値($\sharp$)} & 
    \multicolumn{2}{c|}{既存手法ASP} & \multicolumn{2}{c}{提案手法LNPS}\\\cline{3-6}
    &  & 最良値 & $\sharp$との比(\%) & 最良値 & $\sharp$との比(\%)\\\hline
    }{%
    \hline
  \end{tabular}
}

%%% Local Variables:
%%% mode: japanese-latex
%%% TeX-master: "paper"
%%% End:
