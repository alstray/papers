% 以下の3行は変更しないこと.
\documentclass[T]{compsoft}
\taikai{2020}
\pagestyle {empty}

\usepackage [dvipdfmx] {graphicx}

% ユーザが定義したマクロなどはここに置く.ただし学会誌のスタイルの
% 再定義は原則として避けること.

\begin{document}

\title{受賞者による受賞研究紹介}

\author{竹内 頼人}
\maketitle 
\section{車両装備仕様問題に対する解集合プログラミングの適用 (竹内 頼人)}
車両装備仕様とは,簡単に言うと自動車のカタログに記載されている
モデル/グレードと装備の組合せのことです.
車両装備仕様問題とは,与えられた装備仕様の個数,装備タイプの集合,
装備オプションの集合などから,装備及び燃費に関する制約を満たしつつ,
予想販売台数が最大となる装備仕様を求める問題です.
現状では,車両の装備仕様決定には専門知識をもつ技術者の多大な労力が
費やされており,その効率化・自動化は自動車メーカーにとって
重要な課題のひとつとなっています.
そこで,解集合プログラミング(ASP)を用いてこの問題を解決できないか,
というところから本研究は始まりました.

ASPの言語は標準論理プログラムをベースとしており,各制約をルールとして
宣言的に記述することが可能で,複数の制約から成り立つ車両装備仕様問題を
簡潔に記述することができます.
さらに,制約や目的関数の追加なども柔軟に行うことができ,拡張性に優れています.
車両の装備に関する制約は,まず企画部門で設定され,開発部門,生産部門,
販売部門に受け渡され,各部門で制約が追加されながら徐々に成熟していきます.
また,環境対策などの観点から燃費に関する制約は国によって細かく定められており,
年々厳しくなっています.
このような,制約が複雑に絡み合い,頻繁に追加や更新が行われる問題に対して,
記述性や拡張性に優れるASPはぴったりだと思うのです.

本研究では,燃費に関する制約として,日本では2020年度から導入されている
企業平均燃費(CAFE)方式に基づく車両装備仕様問題を扱っています.
そして,この問題を解くためのソルバー(CAFE問題ソルバー)の設計・実装を行いました.
その結果,制約の符号化を工夫することで大規模な問題にも適用可能であることや,
装備オプション数の最小化などの拡張も可能であることを確認しました.
今後は,排出ガスを一切出さない電気自動車や燃料電池車を一定割合以上
売らなければならないZEV規制や,より利幅の大きい車種を多く売りたい
という自動車メーカーの要望など,新たな制約や目的関数を実装することにより,
本ソルバーをさらに拡張していくことを課題として,この研究が装備仕様決定の
効率化・自動化につながるよう,邁進して参ります.


最後に,指導教員と共著者の皆様,研究に関して様々な意見をくださった共同研究者の皆様,
そして,コロナ禍で大変な中このような機会を設けてくださった大会関係者の皆様に
感謝申し上げます.


\end{document}
