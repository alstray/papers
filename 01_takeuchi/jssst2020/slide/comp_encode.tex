{\renewcommand{\arraystretch}{0.6}
\begin{tikzpicture}
 \coordinate (zero) at (0,0);
 \draw (zero) node [below,color=blue]{0};
 \uncover<1>{
 \draw (zero) + (1,-0.9) node [color=blue]{\scriptsize{(オプションを一つも装備しない)}};
 }
 \coordinate (all) at (10,0);
 \draw (all) node [below,color=blue] {$\sum_{j\in V}w_{j}$};
 \uncover<1>{
 \draw (all) + (-1,-0.9) node [color=blue]{\scriptsize{(すべてのオプションを装備)}};
 }
 \draw [|-|,thick] (zero)  -- (all);
 \draw [color=blue] (zero) to [out=30, in=150] (all);
 \draw (5,1.5) node [fill=white,blue]{基本符号化};
 
 \uncover<2>{
 \coordinate (min) at (3,0);
 \draw (min) node [below,color=red]{
   \begin{tabular}{c}
    $\sum_{i\in VP^{*}}\min_{j\in V_{i}}w_{j}$ \\
    \\
    \scriptsize{(必須タイプのみでIWR値が} \\
    \scriptsize{最小のオプションを装備)}
   \end{tabular}
 };
 \coordinate (max) at (7,0);
 \draw [|-|,thick] (min) -- (max);
 \draw (max) node [below,color=red]{
   \begin{tabular}{c}
    $\sum_{i\in VP}\max_{j\in V_{i}}w_{j}$ \\
    \\
    \scriptsize{(すべてのタイプでIWR値が} \\
    \scriptsize{最大のオプションを装備)}
   \end{tabular}
 }; 
 \draw [color=red] (min) to [out=45, in=135] (max);
 \draw (5,0.7) node [fill=white,red]{改良符号化};
 }
\end{tikzpicture}
}
