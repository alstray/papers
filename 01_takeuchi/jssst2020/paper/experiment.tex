\section{実験}
\begin{table}[tb]
 \caption{CAFE問題のインスタンス}
 \centering
 \begin{tabular}{crrrr} \bhline
  インスタンス & VP	& V	& require & exclude	\\\hline
  small	    & 8		& 21	& 4	  & 0	        \\
  medium    & 86	& 229	& 147	  & 0	        \\
  big	    & 315	& 1,337	& 0	  & 0        	\\\hline
 \end{tabular}
 \label{tab:instance}
\end{table}

\begin{table}[tb]
 \caption{実験結果}
 \centering
 \begin{tabular}{c|r|rr} \bhline
  インス & CAFE & \multicolumn{2}{c}{販売台数} \\ \cline{3-4}
  タンス & 基準値                & 基本符号化 & 改良符号化 \\\hline    
  small & 8.5   & \textbf{6,021*}       & \textbf{6,021*}       \\
   & 9.0   & \textbf{5,007*}       & \textbf{5,007*}       \\
   & 9.5   & \textbf{2,688*}       & \textbf{2,688*}       \\
   & 10.0  & \textbf{1,318*}       & \textbf{1,318*}       \\
   & 10.5  & \textbf{UNSAT*}       & \textbf{UNSAT*}       \\ \hline
  medium & 8.5  & 6,010        & \textbf{6,021}        \\
   & 9.0  & \textbf{5,595}        & \textbf{5,595}        \\
   & 9.5  & \textbf{3,447}        & 3,430        \\
   & 10.0 & 2,245        & \textbf{2,250}        \\
   & 10.5 & 1,690        & \textbf{1,845}        \\ \hline
  big & 8.5     & UNKNOWN     & \textbf{3,877}        \\
   & 9.0     & 1,038        & \textbf{4,623}        \\
   & 9.5     & 688         & \textbf{3,121}        \\
   & 10.0    & 1,634        & \textbf{2,064}        \\
   & 10.5    & 538         & \textbf{904}         \\ \hline \hline
  \multicolumn{2}{c}{最良値の数} & 6 & 13         \\ \hline
 \end{tabular}
 \label{tab:result}
\end{table}
本ソルバーの性能評価のために,前節で述べた基本符号化と改良符号化を用いて実験を行った.
インスタンスには,表\ref{tab:instance}に示す企業から提供されたCAFE問題のインスタンス
3問を用いた.特に,インスタンスmediumは現実的な問題のサイズとなっている.また,この実験で
用いたインスタンスでは排除制約は考慮しない.各インスタンスでCAFE基準値8.5,9.0,9.5,10.0,10.5km/L
の5通りに対して,3種類の装備仕様による解を求めた.

ASPシステムには,広く普及している高速ASPソルバーclingo-5.4.0を使用し,タイムアウトは
7200秒(2時間)とした.実験環境は,Mac mini(3.2GHz, 64GBメモリ)である.

表\ref{tab:result}に,各インスタンスとCAFE基準値に対して,基本符号化と改良符号化で
得られた装備仕様の予想販売台数を示す.* 付きの値は最適解であることを示し,太字の値は2つの
符号化の間で販売台数が最良値であることを意味する.また,UNSATは,制約を充足する解が一つも
存在しないことを意味し,UNKNOWNは制限時間内に解を見つけることができなかったことを意味する.


小規模なインスタンスsmallについては,基本符号化,改良符号化ともに5問すべてに対して最適解,
あるいはUNSATを得ることに成功した.
最良値を得た問題数は,基本符号化が6問,改良符号化が13問で,改良符号化の方が多くの問題に
対して優れた解を得たことがわかる.特に,大規模なインスタンスbigについては,改良符号化が
5問すべてに対して基本符号化よりも優れた解を得ており,ASPのルール数を削減することが規模の大きい問題
に対して有効であることがわかった.