\section{実行実験}

% 提案手法の有効性を評価するために,企業から提供 を受けた小規模・実用規
% 模・大規模の CAFE 問題に 対して実行実験を行った結果,基本符号化と改良
% 符号 化はどちらも小規模な問題の最適解を求めることが できた.また,実用
% 規模およびより大規模な問題に対 しては,改良符号化が基本符号化より優れ
% た結果を示




提案符号化の有効性を評価するために,開発したCAFE問題ソルバーの実行実験
を行った.
ベンチマーク問題(表~\ref{tab:bench}参照)には,
企業から提供を受けた小規模(small)・実用規模(medium)・大規模(big)の
CAFE問題(3問)に対して,
5種類の CAFE 基準値 (8.5, 9.0, 9.5, 10.0, 10.5km/L)
を適用した問題インスタンス(合計15問)を使用した.
求めたい装備仕様の個数は3とした.

ASPシステムには,広く普及している高速 ASP システム{\clingo}-5.4.0を使
用し,一問当たりの時間制限は 7,200 秒とした.
実験環境は,
Mac mini (3.2GHz Intel Core i7, 64GBメモリ)
である.

%----------------------------------------------
\begin{table}[tb]
  \caption{ベンチマーク問題}
  \centering
  %\renewcommand{\arraystretch}{0.9}
  \tabcolsep = 0.9mm
  \begin{tabular}{crrrr}
    & & & \\\bhline
    問題名 & タイプ数	& オプション数	& 要求制約数 	\\\hline
    small	    & 8		& 21	& 4	  	        \\
    medium    & 86	& 229	& 147	  	        \\
    big	    & 315	& 1,337	& 0	          	\\\hline
    & & & \\
  \end{tabular}
 \label{tab:bench}
\end{table}
%----------------------------------------------

表\ref{tab:result}に実験結果を示す.
左から順に,問題名,CAFE 基準値,基本符号化と改良符号化
の予想販売台数(目的関数の値)となっている.
記号`$\ast$'付きの値は最適値であることを表す.
各問題インスタンスごとに,
より良い予想販売台数をボールド体で示している.
また,\textsf{UNSAT}は制約を満たす実行可能解が存在しないことを意味し,
\textsf{UNKNOWN}は制限時間内に実行可能解が一つも求められなかったことを
意味する.

最適値・最良値を求めた問題数は,基本符号化が6問,改良符号化が13問であり,
改良符号化がより多くの問題に対して優れた結果を示している.
特に,大規模インスタンス big については,5問すべてに対して,改良符号化
が基本符号化よりも優れた解を得ることに成功している.
これにより,改良符号化の大規模な問題への有効性が確認できた.
%
小規模なインスタンス small については,基本符号化と改良符号化ともに5問
すべてに対して最適解(あるいは\textsf{UNSAT})を求めることができた.

表\ref{tab:cpu_time}に,
最適解,あるいは \textsf{UNSAT} が得られた問題インスタンスについて,
それぞれの符号化で解を得るまでに要した CPU 時間を示す.
各問題インスタンスごとに,より短い CPU 時間をボールド体で示している.
5問中4問で改良符号化の方が CPU 時間が短く,
全体の平均も改良符号化の方が短い.
これらの結果から,
平均的に改良符号化の方が高速に解を求めることができることがわかる.




%----------------------------------------------
\begin{table}[tb]
  \caption{実験結果}
  \centering
  %\renewcommand{\arraystretch}{0.9}
  \tabcolsep = 0.9mm
  \begin{tabular}{c|r|rr} \bhline
    問題名 & CAFE & \multicolumn{2}{c}{予想販売台数} \\ \cline{3-4}
           & 基準値 & 基本符号化             & 改良符号化 \\\hline    
    small & 8.5   & \textbf{6,021*}       & \textbf{6,021*}       \\
           & 9.0   & \textbf{5,007*}       & \textbf{5,007*}       \\
           & 9.5   & \textbf{2,688*}       & \textbf{2,688*}       \\
           & 10.0  & \textbf{1,318*}       & \textbf{1,318*}       \\
           & 10.5  & \textsf{UNSAT}        & \textsf{UNSAT}       \\ \hline
    medium & 8.5  & 6,010                 & \textbf{6,021}        \\
           & 9.0  & \textbf{5,595}        & \textbf{5,595}        \\
           & 9.5  & \textbf{3,447}        & 3,430                 \\
           & 10.0 & 2,245                 & \textbf{2,250}        \\
           & 10.5 & 1,690                 & \textbf{1,845}        \\ \hline
    big   & 8.5   & \textsf{UNKNOWN}       & \textbf{3,877}        \\
           & 9.0   & 1,038                 & \textbf{4,623}        \\
           & 9.5   & 688                   & \textbf{3,121}        \\
           & 10.0  & 1,634                 & \textbf{2,064}        \\
           & 10.5  & 538                   & \textbf{904}          \\ \hline \hline
    \multicolumn{2}{c}{最適値・最良値の数} & 6 & 13                       \\ \hline
 \end{tabular}
 \label{tab:result}
\end{table}
%----------------------------------------------
\begin{table}[tbp]
 \caption{解を得るまでに要したCPU時間(秒)}
 \centering
 \begin{tabular}{c|r|rr}\bhline
  問題名 & CAFE  & \multicolumn{2}{c}{CPU時間} \\ \cline{3-4}   
        & 基準値 & 基本符号化  & 改良符号化 \\ \hline
  small & 8.5  & 37.868          & \textbf{23.318}   \\
  	& 9.0  & 48.965          & \textbf{43.362}   \\
  	& 9.5  & \textbf{95.110} & 173.172           \\
  	& 10.0 & 99.954          & \textbf{0.343}    \\
  	& 10.5 & 439.613         & \textbf{0.080}    \\ \hline \hline
  \multicolumn{2}{c}{平均}  & 144.302         & \textbf{48.055}   \\ \hline
  % 幾何平均 & 	 & 95.028s & 5.449s \\
 \end{tabular}
 \label{tab:cpu_time}
\end{table}
%-------------------------------------------------------
\begin{table}[tb]
 \caption{比較結果: 基礎化後のルール数.CAFE基準値: 9.0km/L}
 \centering
 \begin{tabular}{crr} \bhline
  問題名    & 基本符号化    & 改良符号化    \\ \hline
  small	    &  83,520 (1.00)  & 32,855 (0.39) \\ 
  medium    &  93,017 (1.00)  & 56,940 (0.61) \\
  big	    & 155,654 (1.00)  & 42,190 (0.27) \\ \hline
 \end{tabular}
 \label{tab:rule}
\end{table}
%-------------------------------------------------------

次に,基本符号化と改良符号化とで,基礎化後のルール数を比較した結果を
表\ref{tab:rule}に示す.
括弧内の数字は,基本符号化のルール数を1としたときの比率である.
表\ref{tab:rule}より,
改良符号化は,基本符号化と比較して,基礎化後のルール数を少なく抑えられ
ていることがわかる.
特に,big では,70\%以上のルール数の削減に成功している.
このルール数の削減が,が改良符号化の大規模問題に対する優位性につながっ
ていると考えられる.

さらに,インスタンス small を用いて,求める装備仕様の個数を増やして実験を行った.
装備仕様の個数は3, 6, 9とし,CAFE基準値は前述の実験と同じ5種類を用いた.
初期制約を用いて,タイプ\textsf{Grade}のオプション\textsf{STD, DX, LX}
それぞれを選択する装備仕様の数が等しくなるようにした.
制限時間は1問あたり1時間とした.
装備仕様の個数3では,すべての問題インスタンスに対して,
基本符号化・改良符号化ともに最適値あるいはUNSATが得られたが,
装備仕様の個数6では最適値の求まらなかった問題や
実行可能解が一つも見つけられなかった問題が存在した.
さらに,装備仕様の個数9では,
最適値を一つも得ることができなかった.
これらの結果から,装備仕様の個数は求解の難しさに大きく関係していることがわかる.




% \begin{table}[tb]
%  \caption{実験結果2}
%  \centering
%  \small
%  \begin{tabular}{l|c|c|c|c}\bhline
%   問題名 & CAFE  & 装備仕様 & \multicolumn{2}{c}{CPU時間(秒)}\\ \cline{4-5}
%         & 基準値 & の個数   & 基本符号化 & 改良符号化 \\ \hline
%   small & 9.0   & 3       & 25.648    & 12.837   \\   
%         &       & 6       & 1386.491  & 1509.845 \\
%         &       & 12      & \textsf{Time Over} & \textsf{Time Over} \\ \hline
%  \end{tabular}
% \end{table}

%%% Local Variables:
%%% mode: japanese-latex
%%% TeX-master: "paper"
%%% End:
