\section{解集合プログラミング}\label{sec:asp}

解集合プログラミング(Answer Set Programming;ASP \cite{%
  Baral03:cambridge,%
  Gelfond88:iclp,%
  Inoue08:jssst})
は論理プログラミングから派生した宣言的論理パラダイムである.
\textbf{ASPシステム}は,与えられた論理プログラムから,
安定モデル意味論~\cite{Gelfond88:iclp}
に基づく解集合を計算するシステムである.
近年,SATソルバー技術を応用した高速ASPシステムが実現され,ロボット工学,システム生物学,
システム検証,制約充足問題,プランニングなど様々な分野への
実用的応用が急速に拡大している\cite{Gelfond16:aim}.

ASP言語は一階論理に基づく知識表現言語の一種であり,
一般拡張選言プログラムをベースとしている. 
本発表では説明の簡略化のため,そのサブクラスである
標準論理プログラムについて説明する.
以降,標準論理プログラムを単に論理プログラムと呼ぶ.

\textbf{論理プログラム}は,以下の形式の\textbf{ルール}の有限集合である.
\begin{equation}
  \label{eq:rule}
  a_0\leftarrow a_1,\dots,a_m,\naf{a_{m+1}},\dots,\naf{a_n}
\end{equation}
このルールの直観的な意味は,
「$a_1,\ldots,a_m$がすべて成り立ち,$a_{m+1},\ldots,a_n$のそれぞれが成
り立たないならば,$a_0$が成り立つ」である.
ここで,
$0\leq m\leq n$ であり,
各$a_i$はアトム,
$\naf{}$は\textbf{デフォルトの否定}
\footnote{\textbf{失敗による否定}とも呼ばれる.述語論理で定義される否定($\neg$)とは意味が異なる.},
``$,$''は連言を表す.
$\leftarrow$の左側を\textbf{ヘッド},右側を\textbf{ボディ}と呼ぶ.
ボディが空のルール(すなわち\(a_0\leftarrow\))を\textbf{ファクト}と呼び,
$\leftarrow$を省略してよい.

ヘッドが空のルールを\textbf{一貫性制約}と呼び,以下のように表す.
\begin{equation}
  \label{eq:constr}
  \leftarrow a_1,\dots,a_m,\naf{a_{m+1}},\dots,\naf{a_n}
\end{equation}
例えば,一貫性制約
``\(\leftarrow a_1,a_2\)''は,「$a_1$と$a_2$が両方同時に成り立つことはない」を意味し,
``\(\leftarrow a_1, \naf{a_{2}}\)''は,「$a_1$が成り立つならば,$a_2$が成り立つ」を意味する.



ASP言語には,組合せ問題を簡潔に記述するために,
\textbf{アグリゲート}(aggregate)と呼ばれる拡張構文がいくつか用意されている.
例えば,\textbf{選択子}``$\{a_1;...;a_n\}$.''は,
集合\(\{a_1;...;a_n\}\)の任意の部分集合を解集合に含めることを意味する.
\textbf{個数制約}は選択子の両端に選択可能な個数の上下限を付けたものである.
例えば,``\(lb\ \{a_1;\dots;a_n\}\ ub \leftarrow Body\)''と書くと,
「$Body$が成り立つならば,$a_1,\dots,a_n$のうち,$lb$個以上$ub$個以下
が成り立つ」を意味する.
\textbf{重み付き個数制約}``\#sum\(\{w_1,a_1;...;w_n,a_n\}\) = c.''は,
$a_1,\dots,a_n$のうち真となるアトムの重み和が整数定数cに等しくなることを意味する.
重み$w_i$は整数定数であり,演算子としては``="以外にも``$<$",``$>$"などを使用できる.
さらに,重み付き個数制約の``\#sum''を,``\#max''や``\#min''に書き換えると,重み和ではなく,
真となるアトムの重みの最大値や最小値を求めることができる.
また,組合せ最適化問題を解くために,最小化関数
(\code{#minimize})・最大化関数(\code{#maximize})等も用意されている.



近年,
{\clingo}~\footnote{\url{https://potassco.org/}},
{\dlv}~\footnote{\url{http://www.dlvsystem.com/dlv/}},
{\wasp}~\footnote{\url{https://www.mat.unical.it/ricca/wasp/}}
など,SATソルバー技術を応用した高速なASPシステムが開発されている.
なかでも{\clingo}は,高性能かつ高機能なASPシステムとして世界中で広く使われている.
これらの高速ASPシステムは,一階ASPプログラムを命題ASPプログラムに変換(\textbf{基礎化})
したのち,ASPソルバーを用いて解集合を計算する.
本論文で使用するASPシステム{\clingo}は,基礎化のためのグラウンダー{\gringo}とASPソルバー
{\clasp}をシームレスに結合したものである.

\begin{table}[tb]
  \centering
  \caption{論理プログラムとソースコードの対応}
  \begin{tabular}{l|*{4}{p{5mm}}}
    論理プログラム &   $\leftarrow$ & $,$        & $;$        & $\sim$       \\\hline
    ソースコード   &   \texttt{:-}  & \texttt{,} & \texttt{;} & \texttt{not}
  \end{tabular}
  \label{tbl:map}
\end{table}

解集合プログラミングを用いた問題解法プロセスは,3つのステップからなる.
まず最初に,解きたい問題を論理プログラムとして表現する.
つぎに,ASP システムを用いて,論理プログラムの解集合を計算する.
最後に,解集合を解釈して元の問題の解を得る.
以降の節で示す論理プログラムのソースコードはすべて{\gringo}言語で書かれて
おり,表記上の対応については表~\ref{tbl:map}の通りである.


%%% Local Variables:
%%% mode: japanese-latex
%%% TeX-master: "paper"
%%% End:
