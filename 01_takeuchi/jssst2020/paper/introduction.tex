\section{はじめに}
自動車の車両を設計する\textbf{車両装備仕様問題}は,求解困難な組合せ最
適化問題の一種である.
%
車両の装備仕様を決めるには,販売される国や地域の法規や規制,地域や市場
の特性,市場の嗜好や競合など十分に考慮する必要があり,
現状では専門知識をもつ技術者の多大な労力が費やされている.
そのため,車両装備仕様問題の効率のよい解法は重要な研究課題である.

車両装備仕様問題は,\textbf{装備タイプ}とそれに付随する\textbf{装備オ
プション}から構成される.
装備タイプはエンジンやトランスミッションなどの装備の種類を表し,
装備オプションは,4気筒エンジン,CVTなどの具体的な装備を表す.
%
\textbf{装備仕様}とは,タイプとオプションの組合せであり,
(エンジン, 4気筒), (トランスミッション,CVT)などの対の集合で表される.
車両装備仕様問題の目的は,与えられたタイプとオプションの集合から,
装備および燃費に関する制約を満たしつつ,販売台数を最大化する装備仕様を
求めることである.

本発表では,燃費に関する制約として,
企業平均燃費 (Corporate Average Fuel Efficiency; CAFE)方式を採用
する.
このCAFE方式は,自動車の燃費規制で,車種別ではなくメーカー全体で出荷台数
を加味した平均燃費を算出する方式である.
日本では2020年度基準から採用されている.
以降では,燃費規制にこのCAFE方式を用いる車両装備仕様問題のことを
\textbf{CAFE問題}と呼ぶ.



\textbf{解集合プログラミング}(Answer Set Programing; ASP \cite{%
  Baral03:cambridge,%
  Gelfond88:iclp,%
  Inoue08:jssst})
は,論理プログラミングから派生した宣言的プログラミングパラダイムである.
ASP言語は一階論理に基づく知識表現言語の一種であり,
論理プログラムはASPのルールの有限集合である.ASPシステムは論理プログラム
から安定モデル意味論 \cite{Gelfond88:iclp}
に基づく解集合を計算するシステムである.近年,SATソルバー技術を応用した
高速ASPシステムが
実現され,ロボット工学,システム生物学,システム検証,制約充足問題,
プランニングなど様々な分野への実用的応用が急速に拡大している\cite{%
Gelfond16:aim}.

本発表では,解集合プログラミング(ASP)を用いたCAFE問題ソルバーを提案する.
CAFE問題ソルバーでは,まず与えられたCAFE問題のインスタンスを
ASPのファクト形式に変換し,CAFE問題の制約のASP符号化と結合した後に,
ASPシステムを用いて解を求める.
ASP符号化として,基本符号化と改良符号化の2種類を考案した.
特に,改良符号化は燃費や販売台数を算出するために必要な値
IWR(Inertial Working Rating)の和の上下限を厳密に見積もることにより,
基礎化後のルール数を少なく抑えるよう工夫されている.
% 無駄な探索空間を削減するように工夫されている.
これにより,より大規模な問題への有効性が期待できる.
また,CAFE問題ソルバーでは解の列挙が可能であり,
販売台数の等しい最適解が複数存在する場合に,
より優れた解を求めるためにオプション数最小化という目的関数の追加をした
拡張符号化も考案した.この目的関数には,オプションの種類を減らして製造
ラインの削減や大量生産を促進する狙いがある.


CAFE問題ソルバーの有効性を評価するために,企業から提供された現実の
問題(3問)に対して,
5種類のCAFE基準値(8.5, 9.0, 9.5, 10.0,10.5km/L)に対する
問題インスタンス(計15問)を生成し,実行実験を行った.
%
小規模(タイプ数8,オプション数21)な問題インスタンス(5問)については,
両符号化とも最適解を求めることができた.
大規模(タイプ数315,オプション数1337)な問題インスタンス(5問)につい
ては,改良符号化が,基本符号化より優れた解を得ることに成功した.
これにより,大規模な問題に対する改良符号化の有効性が確認できた.
また,最適解を複数もつような問題インスタンスに対して
拡張符号化を用いることで,最もオプションの種類が少ない
解を求めることができることを確認した.

