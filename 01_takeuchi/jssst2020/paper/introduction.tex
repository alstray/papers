\section{はじめに}
% ----------------------------------------------------------------------
  \thicklines
  \setlength{\unitlength}{1.28pt}
  \small
  \begin{picture}(280,57)(4,-10)
    \put(  0, 20){\dashbox(50,24){\shortstack{根付き全域森\\問題}}}
    \put( 60, 20){\framebox(50,24){変換器}}
    \put(120, 20){\dashbox(50,24){\shortstack{ASPファクト}}}
    \put(120,-10){\alert{\bf\dashbox(50,24){\scriptsize{\shortstack{ASP符号化\\(論理プログラム)}}}}}
    \put(180, 20){\framebox(50,24){ASPシステム}}
    \put(240, 20){\dashbox(50,24){\shortstack{根付き全域森\\問題の解}}}
    \put( 50, 32){\vector(1,0){10}}
    \put(110, 32){\vector(1,0){10}}
    \put(170, 32){\vector(1,0){10}}
    \put(230, 32){\vector(1,0){10}}
    \put(170, +2){\line(1,0){4}}
    \put(174, +2){\line(0,1){30}}
  \end{picture}  

% ----------------------------------------------------------------------

自動車の車両の装備仕様を決めるには,販売される国や地域の法規や規制,
地域や市場の特性,市場の嗜好や競合など十分に考慮する必要があり,
現状では専門知識をもつ技術者の多大な労力が費やされている.
そのため,装備仕様決定の自動化・効率化は自動車メーカーにとって
重要な課題の一つである.

\textbf{車両装備仕様問題}は組合せ最適化問題の一種であり,
装備タイプと装備オプションから構成される.
\textbf{装備タイプ}はエンジンやトランスミッションなどの装備の種類を表す.
\textbf{装備オプション}は4気筒エンジン,CVTなどの具体的な装備を表す.
\textbf{装備仕様}とは,装備タイプと装備オプションの組合せであり,
(エンジン, 4気筒), (トランスミッション,CVT)などの対の集合で表される.
車両装備仕様問題の目的は,与えられた装備タイプの集合,各装備タイプに付
随する装備オプションの集合,および,
求めたい装備仕様の数から\footnote{派生車両の数と考えてよい},
装備および燃費に関する制約を満たしつつ,
販売台数を最大化する装備仕様を求めることである.

本研究では,燃費に関する制約として,日本でも2020年度から導入されている
\textbf{企業平均燃費}
(Corporate Average Fuel Efficiency; CAFE~\cite{metimlit18:cafe})
方式を用いる.
このCAFE方式は自動車の燃費規制で,車種別ではなくメーカー全体での出荷台数
を加味した平均燃費を算出し,規制をかける方式である.
CAFE方式の特長は,
ある特定の車種では燃費基準を達成できなくても,他の車種の燃費を向上させ
ることで基準を達成できることが可能な点である.
本論文では,CAFE方式に基づく車両装備仕様問題(以降,\textbf{CAFE問題}と
呼ぶ)を対象とする.

\textbf{解集合プログラミング}
(Answer Set Programing; ASP~\cite{%
  Baral03:cambridge,%
  Gelfond88:iclp,%
  Inoue08:jssst})
は,論理プログラミングから派生した宣言的プログラミングパラダイムである.
ASP言語は一階論理に基づく知識表現言語の一種である.
論理プログラムはASPのルールの有限集合である.
ASPシステムは論理プログラムから安定モデル意味論~\cite{Gelfond88:iclp}
に基づく解集合を計算するシステムである.
近年,SATソルバー技術を応用した高速ASPシステムが実現され,
ロボット工学,システム生物学,システム検証,制約充足問題,プランニング
など様々な分野への実用的応用が急速に拡大している~\cite{Gelfond16:aim}.


本論文では,解集合プログラミングを用いたCAFE問題ソルバーの実装と評価に
ついて述べる.
提案ソルバーは,まず与えられた問題インスタンスをASPのファクト形式に変
換し,CAFE問題を解くためのASP符号化と結合した上で,
高速ASPシステムを用いて解を求める(図~\ref{fig:arch}参照).

ASP符号化として,基本符号化と改良符号化の2種類を考案した.
基本符号化は,CAFE問題の制約をASPのルール18個で簡潔に表現できる.
改良符号化は,装備仕様の燃費や販売台数を算出するために必要な
IWR (Inertial Working Rating) 値の総和の上下限を厳密に計算
することにより,基礎化後のルール数を少なく抑えるよう工夫されている.
これにより,改良符号化は大規模な問題への有効性が期待できる.
また,販売台数の最大化に加えて,製造ラインの削減や大量生産を促進するこ
とを狙いとして,装備オプション数の最小化に関する拡張も行った.


提案手法の有効性を評価するために,企業から提供を受けた
実用規模を含むCAFE問題(3問)に対して,
5種類のCAFE基準値(8.5km/L〜10.5km/L)を適用した
問題インスタンスを生成し,実行実験を行った.
その結果,基本符号化と改良符号化のどちらも,
小規模インスタンスの最適解を求めることができた.
また,実用規模およびより大規模なインスタンスに対しては,
改良符号化が基本符号化より優れた結果を示し,
その有効性が確認できた.

%%% Local Variables:
%%% mode: latex
%%% TeX-master: "paper"
%%% End:
