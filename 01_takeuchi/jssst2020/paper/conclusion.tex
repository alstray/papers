\section{おわりに}

本発表では,CAFE方式と呼ばれる燃費基準に基づく車両装備仕様問題に対して,
解集合プログラミングを用いて求解を行うCAFE問題ソルバーを提案した.
CAFE問題の制約のASP符号化として,基本符号化と改良符号化の2種類を考案し,
改良符号化では基礎化後のルール数を少なく抑えるように工夫をした.
さらに,最適解が複数存在する場合に,それらの解に優劣をつけるために,
オプションの種類の数を最小にするような目的関数を追加した拡張符号化も
考案した.
本ソルバーの性能評価には実用規模のインスタンスを含む問題集を用いた
実行実験を行い,小規模な問題で最適解が得られることや,
大規模な問題に対して改良符号化が有効であることなどを確認した.
さらに,拡張符号化を用いることで,インスタンスによっては複数解に
優劣をつけ,解を一つに絞ることに成功した.

今後の課題としては,企業の技術者へのヒアリングに基づく問題拡張や
CAFE問題以外の,一般的なOVMの問題への対応を可能にすることなど
が挙げられる他,CAFE問題を定式化することで,
制約充足問題(Constraint Satisfiabilty Problem;CSP)ソルバー
などの,ASP以外の技術を用いた求解方法との性能比較などが考えられる.
