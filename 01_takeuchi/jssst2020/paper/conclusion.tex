\section{おわりに}

本論文では,解集合プログラミングを用いた CAFE 問題ソルバーの実装と評価
について述べた.
CAFE 問題に対する2種類の ASP 符号化を考案し,問題を簡潔に記述できるこ
とを確認した.
改良符号化は,装備仕様の燃費や予想販売台数を算出するために必要な IWR
値の総和の上下限を厳密に計算することにより,基礎化後のルール数を少な
く抑えるよう工夫されている点が特長である.
企業から提供を受けた実用規模を含むCAFE問題を用いた実験の結果,
基本符号化と改良符号化のどちらも,小規模インスタンスの最適解を求めるこ
とができた.また,実用規模およびより大規模なインスタンスに対しては,
改良符号化の優位性が確認できた.

今後の課題としては,企業の技術者へのヒアリングに基づくCAFE問題の拡張が
挙げられる.また,SATソルバー,制約プログラミング等の他のアプローチと
の比較も重要である.

%%% Local Variables:
%%% mode: japanese-latex
%%% TeX-master: "paper"
%%% End:
