\section{ASPに基づくCAFE問題ソルバー}

提案するCAFE問題ソルバーは,与えられた問題インスタンスをASPの
ファクト形式に変換した後,それらファクトとCAFE問題を解くためのASP符号
化(論理プログラム)を結合した上で,高速ASPシステム{\clingo}を用いて解
を求める(図~\ref{fig:arch}参照).
本節では,CAFE問題の入力と制約を論理プログラムとして表現する方法に
ついて述べる.

%%%%%%%%%%%%%%%%%%%%%%%%%%%%%%%%%%%%
\subsection{ASPファクト形式}

%----------------------------------------
\lstinputlisting[float=htbp,caption={%
  CAFE問題の例(図~\ref{fig:ovm_example})のファクト表現.
  装備仕様の個数:3,
  装備仕様とオプションの依存関係:
  (装備仕様1, \textsf{STD}),
  (装備仕様2, \textsf{DX}),
  (装備仕様3, \textsf{LX})},%
captionpos=b,frame=single,label=code:ovm.lp,%
numbers=none,%
breaklines=true,%
columns=fullflexible,keepspaces=true,%
xrightmargin=1zw,% 
xleftmargin=1zw,% 
basicstyle=\ttfamily\scriptsize]{codes/ovm.lp} 
%----------------------------------------

CAFE 問題の入力は,CAFE 基準値(入力~\ref{input:cafe})
を除いて,ASP のファクトで表される.
CAFE 基準値は ASP の定数\code{t}で表すものとし,
実行時に{\clingo}のオプションから値を指定する.

CAFE 問題の例(図~\ref{fig:ovm_example})
をファクトで表現したものをコード~\ref{code:ovm.lp}に示す.
この2つを見比べると,可変性モデルの
可変点,
バリアント,
制約依存性(要求),
制約依存性(排除)が,
それぞれファクト
\code{vp_def/1}, 
\code{v_def/3},
\code{require_v_v/2},
\code{exclude_v_v/2}
に対応していることがわかる.
例えば,バリアント\textsf{V4}は,ファクト
\code{v_def("V4", "Engine", 120).}に対応している.
%
可変性モデルは CAFE 問題の入力~\ref{input:vp}〜\ref{input:iwr}を含んでいる.
%
この他,
\code{require_vp/1}は必須タイプ,
\code{group/1}は装備仕様の識別子(入力~\ref{input:g}),
\code{require_g_v/2}は
各装備仕様とオプション間の依存関係を表している(入力~\ref{input:init}).
例えば,\code{require_g_v(1, "STD").}は,装備仕様~\code{1}が
オプション\code{STD}を要求することを表す.

IWR 値の総和と燃費の対応表(入力~\ref{input:fe})は,
ファクト\texttt{fe\_map($S$,$FE$)}の有限集合で表される.
ここで,$S$はIWR 値の総和,$FE$は燃費を表す.
同様にして,
IWR 値の総和と予想販売台数の対応表(入力~\ref{input:sv})は,
ファクト\texttt{sv\_map($S$,$SV$)}の有限集合で表される.

%%%%%%%%%%%%%%%%%%%%%%%%%%%%%%%%%%%%
\subsection{CAFE問題の ASP 符号化}

%----------------------------------------
\lstinputlisting[float=tb,caption={%
  基本符号化},%
captionpos=b,frame=single,label=code:basic1.lp,%
numbers=left,%
breaklines=true,%
columns=fullflexible,keepspaces=true,%
xrightmargin=1zw,% 
xleftmargin=1zw,% 
basicstyle=\ttfamily\scriptsize]{codes/basic1.lp} 
%----------------------------------------

CAFE問題の各制約は,ASP の個数制約および一貫性制約を使って簡潔に表現される.
CAFE問題の制約と目的関数を表す論理プログラムをコード\ref{code:basic1.lp}に示す.
装備仕様\code{G}がタイプ\code{VP}を装備することを意味するアトム\code{vp(VP,G)}を導入する.
また,
装備仕様\code{G}がオプション\code{V}を実装することを意味するアトム
\code{v(V,G)}も併せて導入する.

1行目のルールは,
各装備仕様\code{G},各タイプ\code{VP}に対して,
\code{vp(VP,G)}の候補を生成する.
2行目のルールは,各装備仕様\code{G}に対して,
\code{VP}が必須タイプならば,
\code{G}は\code{VP}を装備しなければならないことを表す.

範囲制約は5行目のルールで表される.
このルールは,タイプ\code{VP}を装備する装備仕様\code{G}に対して,
\code{VP}のオプションからちょうど1個を実装することを表す.

燃費制約は8〜13行目のルールで表される.
8行目の\code{iwr(S,G)}は,装備仕様\code{G}のIWR値の総和が\code{S}であ
ることを表す.11〜12行目の
\code{fe(FE,G)}と\code{sv(SV,G)}は,それぞれ,
装備仕様\code{G}の燃費\code{FE}と予想販売台数\code{SV}を表す.
13行目のルールは,第~\ref{sec:background}節で示した
CAFE 規制の式を以下のように変形し,ASPの重み付き個数制約で表している.
\begin{center}
\(\sum_{i=1}^{n} (FE_{i}-t)\cdot SV_{i} \geq 0\)  
\end{center}

要求制約は16〜19行目のルールで表される.
例えば,
16行目のルールは,装備仕様\code{G}がオプション\code{V1}
を実装し,かつ,\code{V1}が\code{V2}を要求するならば,
\code{G}は\code{V2}を実装しなければならないことを表す.
他の要求制約,排除制約,初期制約も同様に,一貫性制約を用いて
簡潔に表現できる.

CAFE 問題の目的は,予想販売台数を最大化することである.
この目的関数は,31行目の最大化関数\code{#maximize}によって表される.
%
本節で述べた論理プログラム(コード\ref{code:basic1.lp})を
基本符号化と呼ぶ.

%%%%%%%%%%%%%%%%%%%%%%%%%%%%%%%%%%%%
\subsection{符号化の改良}

%----------------------------------------
\lstinputlisting[float=tb,caption={%
  基本符号化(コード\ref{code:basic1.lp}の8〜10行目)の改良},%
captionpos=b,frame=single,label=code:basic2.lp,%
numbers=none,%
breaklines=true,%
columns=fullflexible,keepspaces=true,%
xrightmargin=1zw,% 
xleftmargin=1zw,% 
basicstyle=\ttfamily\scriptsize]{codes/basic2.lp} 
%----------------------------------------

前節で述べた基本符号化はCAFE問題の制約を簡潔に表現できるが,
改良できる点も残っている.
本節では,説明の簡単化のため,
各タイプが選択可能なオプション数の上下限値を1とする.

タイプの集合を$VP$,
必須タイプの集合を$VP^{*}$,
オプションの集合を$V$,
タイプ$i\in VP$が実装可能なオプションの集合を$V_{i}$,
オプション$j\in V$のIWR値を$w_{j}$とする.
このとき,
基本符号化(コード\ref{code:basic1.lp})の8〜10行目のルールで生成され
るアトム\code{iwr(S,G)}について,
IWR値の総和を表す\code{S}の上下限は,
\begin{center}
\(0 \leq \texttt{S} \leq \sum_{j\in V}w_{j}\)
\end{center}
である.
しかし,これは自明な上下限であり,
各タイプが選択可能なオプション数の上下限値,および,
必須かどうかの別を考慮することで,
\begin{center}
\(
\sum_{i\in VP^{*}}\min_{j\in V_{i}}w_{j}
\leq \texttt{S} \leq
\sum_{i\in VP}\max_{j\in V_{i}}w_{j}
\)
\end{center}
のように\code{S}の上下限をより厳密に計算することができる.
下限値は,各必須タイプが選択可能なオプションのIWR値の最小値の総和である.
上限値は,各タイプが選択可能なオプションのIWR値の最大値の総和である.
このアイデアを元にした論理プログラムをコード~\ref{code:basic2.lp}に示す.
基本符号化(コード\ref{code:basic1.lp})の8〜10行目を,
コード~\ref{code:basic2.lp}で置き換えた符号化を,
改良符号化と呼ぶ.
改良符号化は,基本符号化と比較して,
\code{iwr(S,G)}に関する基礎化後のルール数を少なく抑えることができる.
そのため,大規模な問題への有効性が期待できる.

%%%%%%%%%%%%%%%%%%%%%%%%%%%%%%%%%%%%
\subsection{符号化の拡張}

%----------------------------------------
\lstinputlisting[float=tb,caption={%
  オプション数の最小化},%
captionpos=b,frame=single,label=code:obj_op.lp,%
numbers=left,%
breaklines=true,%
columns=fullflexible,keepspaces=true,%
xrightmargin=1zw,% 
xleftmargin=1zw,% 
basicstyle=\ttfamily\scriptsize]{codes/obj_op.lp} 
%----------------------------------------

%-----------------------------------------------------------
\begin{table*}[htbp]
  \caption{CAFE問題(図~\ref{fig:ovm_example})の最適解全列挙.CAFE基準値: 8.5km/L}
  \centering
%  \small
  %\renewcommand{\arraystretch}{0.9}
  \tabcolsep = 0.9mm
  \begin{tabular}{l|l|c|c|c||c|c|c||c|c|c} \bhline
    \multicolumn{2}{l|}{} & \multicolumn{3}{c||}{解1} & \multicolumn{3}{c||}{解2} & \multicolumn{3}{c}{\bf{解3}}\\ \hline
    \multicolumn{2}{l|}{装備仕様} & 1 & 2 & 3 & 1 & 2 & 3 & 1 & 2 & 3 \\ \hline
    装備 & \textsf{Grade}        & \textsf{STD}& \textsf{DX} & \textsf{LX}& \textsf{STD} & \textsf{DX}  & \textsf{LX}    & \textsf{STD} & \textsf{DX}  & \textsf{LX}       \\
        & \textsf{Drive\_Type}  & \textsf{4WD}  & \textsf{2WD} & \textsf{4WD} & \textsf{2WD} & \textsf{2WD} & \textsf{4WD} & \textsf{2WD} & \textsf{2WD} & \textsf{4WD}  \\
        & \textsf{Engine} & \textsf{V4} & \textsf{V6} & \textsf{V6} & \textsf{V6} & \textsf{V6} & \textsf{V6} & \textsf{V6} & \textsf{V6} & \textsf{V6}      \\ 
        & \textsf{Tire} & \textsf{16}	& \textsf{17} & \textsf{18} & \textsf{16}  & \textsf{17}  & \textsf{18} & \textsf{16} & \textsf{17} & \textsf{18} \\
        & \textsf{Transmission} & \textsf{5MT} & \textsf{HEV} & \textsf{10AT} & \textsf{CVT} & \textsf{HEV} & \textsf{10AT}  & \textsf{6AT} & \textsf{HEV} & \textsf{10AT}     \\
        & \textsf{Sun\_Roof} & \textsf{Panorama} & -   & -       & \textsf{Nomal} & -  & -     & -   & -   & -       \\ \hline
    \multicolumn{2}{l|}{IWR値の総和}  & 1,128 & 1,130   & 1,255    & 1,130 & 1,130&1,255  & 1,130& 1,130& 1,255     \\ %\hline
    \multicolumn{2}{l|}{燃費(km/L)}    & 8.9 & 8.8    & 8.0     & 8.8 & 8.8  & 8.0 & 8.8  & 8.8  & 8.0         \\ %\hline
    \multicolumn{2}{l|}{予想販売台数}  & 2,007  & 2,007   & 1,511   & 2,007 & 2,007 & 1,511 & 2,007& 2,007& 1,511       \\ \hline
    \multicolumn{2}{l|}{平均燃費(km/L)} & \multicolumn{3}{c||}{8.6} & \multicolumn{3}{c||}{8.5} & \multicolumn{3}{c}{8.5}\\ 
    \multicolumn{2}{l|}{予想販売台数(合計)}  & \multicolumn{3}{c||}{5,525} & \multicolumn{3}{c||}{5,525}  &\multicolumn{3}{c}{5,525}\\
    \multicolumn{2}{l|}{オプション数}  & \multicolumn{3}{c||}{14} & \multicolumn{3}{c||}{13}  &\multicolumn{3}{c}{12}\\ \hline
  \end{tabular}
  \label{tab:optN}
\end{table*}
%-----------------------------------------------------------

製造ラインの削減や大量生産を促進することを狙いとして,
予想販売台数の最大化に加えて,
オプション数の最小化も可能なように拡張する.
コード~\ref{code:obj_op.lp}に拡張のための論理プログラムを示す.

予想販売台数の最大化(2行目)ついては,これまでと同じである.
5行目の\code{used_v(V)}は,
オプション\code{V}がいづれかの装備仕様において実装されたことを意味する.
オプション数の最小化は,\code{used_v(V)}が成り立つ\code{V}の数を最小化
することにより,容易に実現される(6行目).
目的関数中の\code{@}の右側はその優先度を表しており,今回の拡張例では,
予想販売台数の最大化(優先度2),オプション数の最小化(優先度1)の順に最適
化される.
改良符号化の目的関数を,コード~\ref{code:obj_op.lp}で置き換えた符号化を
拡張符号化と呼ぶ.

図~\ref{fig:ovm_example}のCAFE問題に対して,最適解の全列挙をした結果を
表~\ref{tab:optN}に示す.
CAFE基準値が8.5km/Lのとき,予想販売台数が5,525台である最適解が,
改良符号化では(解1,解2,解3)の3つ存在した.
一方で,オプション数の最小化を加えた拡張符号化では,
最適解は(解3)のただ1つだけとなる.
このように,ASPに基づくCAFE問題ソルバーでは,ユーザの選好にあわせて目
的関数を柔軟に追加することにより,複数ある最適解に対して優劣をつけるこ
とができる.

% \begin{table}[tb]
%  \caption{最適解の個数の比較}
%  \centering
%   \begin{tabular}{r|r|c|c} \bhline
%     \small{CAFE} & \small{販売台数} & \multicolumn{2}{c}{\small{最適解の個数}} \\ \cline{3-4}
%     \small{基準値} &     & \small{改良符号化} & \small{拡張符号化}\\ \hline        
%     8.5       & 5,525   & 3 & \textbf{1}\\ 
%     9.0       & 3,904   & 2 & 2 \\
%     9.5       & 1,699   & 1 & 1 \\ \hline
%   \end{tabular}
%  \label{tab:option}
% \end{table}

%%% Local Variables:
%%% mode: japanese-latex
%%% TeX-master: "paper"
%%% End:
