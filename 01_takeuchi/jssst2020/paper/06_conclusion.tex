\section{おわりに}

本論文では,CAFE 方式と呼ばれる新しい燃費規制に基づく車両装備仕様問題
に対して,解集合プログラミング技術をはじめて適用し,
その記述性の高さ,柔軟な最適解探索などの利点を確認した.
また,効率性についても,企業から提供された問題集を用いた実験結果から,
実用規模の問題に対して適用可能であることが確認できた.

装備の組合せに関する制約は,まず企画部門で設定され,そのあと開発部門,
生産部門,販売部門に受け渡され,各部門で制約が追加されながら徐々に成熟
していく.
これら様々な制約を調査・整理・実装することにより,CAFE問題ソルバーを拡
張することが今後の課題である.

%%% Local Variables:
%%% mode: japanese-latex
%%% TeX-master: "paper"
%%% End:
