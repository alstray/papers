\documentclass[a4j,10pt]{jarticle}
\usepackage{multicol}
\usepackage{caption}
%%%%% 余白の設定 %%%%%%
\setlength{\textheight}{\paperheight}
\setlength{\topmargin}{-15.4truemm}
\addtolength{\topmargin}{-\headheight}
\addtolength{\topmargin}{-\headsep}
\addtolength{\textheight}{-20truemm}
\setlength{\textwidth}{\paperwidth}
\setlength{\oddsidemargin}{-10.4truemm}
\setlength{\evensidemargin}{\oddsidemargin}
\addtolength{\textwidth}{-30truemm}
%%%%% 行間の設定 %%%%%
\renewcommand{\baselinestretch}{.95}

\pagestyle{empty}

%%%%% 題目 %%%%%
\title{解集合プログラミングを用いた車両装備仕様問題の解法}
%%%%% 氏名 %%%%%
\author{竹内 頼人(番原研究室)}

\date{\today}
\begin{document}
\maketitle
\thispagestyle{empty}
\begin{multicols}{1}
%%%%% 本文ここから %%%%%

\section{研究の概要}
車両装備仕様とは,簡単に言うと,
自動車のカタログに記載されているモデル/グレードと装備の組合せのことである.
車両装備仕様を決めるには,販売される国や地域の法規や規制,
地域や市場の特性,市場の嗜好や競合など十分に考慮する必要があり,
現状では専門知識をもつ技術者の多大な労力が費やされている.
そのため,車両装備仕様探索の自動化・効率化は自動車メーカーにとって
重要な課題の一つである.
{\bf 車両装備仕様問題}は,組合せ最適化問題の一種であり,
与えられたモデル/グレードの個数,装備タイプの集合,装備オプションの集合などから,
装備および燃費に関する制約を満たしつつ,予想販売台数を最大化する車両装備仕様を
求める問題である.
本研究の目的は,解集合プログラミング\cite{baral_03,gelfond_88}の技術を利用して大規模な車両装備仕様問題を効率よく
解くシステムを実現することである.
本研究では,企業別平均燃費(Corporate Average Fuel Economy; CAFE)
方式と呼ばれる燃費規制に基づく車両装備仕様問題({\bf CAFE問題})を扱う.
CAFE方式とは,メーカー全体での出荷台数を加味した平均燃費が定められたCAFE基準値を
越えなければならない,という制約である.
また,予想販売台数の最大化に加えて,装備オプション数の最小化など
トレードオフの関係にある複数の目的関数のもとで最適な車両装備仕様を求める問題である
{\bf 多目的車両装備仕様問題(多目的CAFE問題)}についても扱う.


% 主に装備タイプと装備オプションから構成される.
% \textbf{装備タイプ}はエンジンやトランスミッションなどの装備の種類を表す.
% \textbf{装備オプション}は4気筒エンジン,CVTなどの具体的な装備を表す.
% 車両装備仕様問題の目的は,与えられた装備仕様の個数,装備タイプの集合,
% 装備オプションの集合から,装備および燃費に関する制約を満たしつつ,
% 予想販売台数を最大化する車両装備仕様を求めることである.



\section{これまでの成果}
提案手法では,可変性モデルで表現された問題インスタンスを ASP のファクト形式に変換した後,
それらファクトとCAFE問題を解くためのASP符号化と結合し,
高速ASPシステムを用いて解を求める.

卒業研究では,単目的のCAFE問題を扱い,ASP符号化として基本符号化,改良符号化の
2種類を考案した.
基本符号化はCAFE問題の制約をASPのルール17個で簡潔に表現できる.
改良符号化は装備仕様の燃費や予想販売台数を算出するために必要なASPのルール数を少なく
抑えるよう工夫されており,大規模な問題への有効性が期待できる.
提案手法の評価として,企業から提供を受けたベンチマーク問題を用いた実行実験の結果,
小規模な問題では最適解を求めることができ,実用規模およびより大規模な問題に対しては,
改良符号化が基本符号化より優れた結果を示し,その優位性が確認できた.

その後,選好関係の宣言・評価を可能にするASPソルバー{\it asprin}\cite{brewka_15}を使用することで,
予想販売台数の最大化と装備オプション数の最小化の2つの目的関数をもつ
多目的CAFE問題に対してパレート最適解を列挙できるようにシステムの拡張を行った.
表\ref{tab:result}に小規模な問題での実験結果を示す.
左の列から順に,問題名,CAFE基準値,得られたパレート最適解の個数,CPU時間となっている.
記号'*'付きの最適解の個数は全列挙が完了しており,これ以上最適解が存在しないことを表す.
制限時間(10800秒)以内に最適解の全列挙最適解の全列挙が完了しなかった場合は,
CPU時間の欄にTime Outと示し,それまでに求めることができた最適解の個数を記載する.
また,{\bf UNSAT}は制約を満たす実行可能解(装備仕様)が1つも存在しないことを意味する.
実験の結果,小規模な問題では,5問中4問に対してパレート最適解の全列挙あるいはUNSATを
得ることに成功し,そのうちの3問では高速にすべての解を求めていることがわかる.
しかし,中・大規模な問題では,
すべての問題インスタンスに対して最適解を得ることができなかった.

%----------------------------------------------
\begin{minipage}{1\linewidth}
% \begin{table}[htb]
%  \caption{実験結果}
 \centering
\captionof{table}{実験結果}
 %\renewcommand{\arraystretch}{0.9}
 \tabcolsep = 0.9mm
 \begin{tabular}{c|r|rr} \hline
  問題名 & CAFE & 最適解の数 & CPU時間(秒) \\ \hline
  small  & 8.5   & {\bf 8*}             & 35.136            \\
  small  & 9.0   & {\bf 5*}             & 1085.354          \\
  small  & 9.5   & 0             & Time Out          \\
  small  & 10.0  & {\bf 1*}             & 1.863             \\
  small  & 10.5  & {\bf UNSAT}         & 0.221             \\ \hline
 \end{tabular}
 \label{tab:result}
\end{minipage}
% \end{table}
%----------------------------------------------

\section{まとめと今後の課題}
本研究は,解集合プログラミングの技術を利用して大規模な車両装備仕様問題を
効率よく解くシステムの実現を目的としている.
卒業研究では,改良符号化を考案することで実用規模,より大規模なCAFE問題に対する有効性を確認した.
その後の研究では,ASPソルバー{\it asprin}を使用することで,
多目的CAFE問題にも適用できるようにシステムを拡張し,
小規模な問題に対してパレート最適解を全列挙することができた.

装備の組合せに関する制約は,まず企画部門で設定され,
そのあと開発部門,生産部門,販売部門に受け渡され,
各部門で制約が追加されながら徐々に成熟していく.
これら様々な制約を調査・整理・実装することにより,
より実用的なシステムへと拡張することが今後の課題である.

\section{研究業績}
\begin{itemize}
 \item NII共同研究第1回研究会合で発表
 \item 日本ソフトウェア科学会第37回大会ソフトウェアセッションで発表,学生奨励賞を受賞
 \item 企業とのCAFE問題に関する意見交換会で発表
\end{itemize}

% \begin{itemize}
%  \item 発表. NII 共同研究 第1回研究会合
%  \item 発表. 日本ソフトウェア科学会 第37回大会 ソフトウェアセッション       
%  \item 発表. 企業とのCAFE問題に関する意見交換会
% \end{itemize}

%%%%% 参考文献 %%%%%
\begin{thebibliography}{10}

\bibitem{baral_03} Baral, C.: {\it Knowledge Representation, Reasoning and Declarative Problem Solving}, Cambridge University Press, 2003.

\bibitem{gelfond_88} Gelfond, M. and Lifschitz, V.: The Stable Model Semantics for Logic Programming, {\it Proceedings of the Fifth International Conference and Symposium on Logic Programming}, MIT Press, 1988, pp. 1070– 1080.

\bibitem{brewka_15}Brewka, G., Delgrande, J. P., Romero, J., and Schaub, T.: asprin: Customizing Answer Set Preferences without a Headache, in {\it AAAI}, pp. 1467–1474, AAAI Press (2015)
\end{thebibliography}

%%%%% 本文ここまで %%%%%
\end{multicols}
\vfill
\noindent
{\gt コメント欄}
{\footnotesize
(本資料をそのまま発表者に返却します.コメント欄以外にもコメントを書いていただいてもかまいません.)}
\\
\fbox{\begin{minipage}{\textwidth}\noindent\\\\\end{minipage}}	
\end{document}
