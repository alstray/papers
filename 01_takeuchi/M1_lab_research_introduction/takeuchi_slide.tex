% Created by Takeuchi on Feb. 2020
\documentclass[dvipdfmx, 11pt,]{beamer}

%%%% Packages %%%%%
%\usepackage{bxdpx-beamer}
%\usepackage{minijs}
%\usepackage{otf}
%\usepackage{tabularx}
%\usepackage{graphicx}
% \usepackage{graphicx}
% \usepackage{amsmath,amssymb,amsthm}
% \usepackage{multirow}
% \usepackage{url}
\usepackage{tikz}
\usetikzlibrary{arrows,shapes}
\usetikzlibrary{positioning}
% \usepackage{alltt}
% \usepackage{bm}
% \usepackage{listings,jlisting}
% \usepackage{listings}
% \lstset{
%  basicstyle=\ttfamily\scriptsize,
%  keepspaces=true,
%  escapechar=|,
%  columns=[l]{fullflexible}
% }

%%%% Fonts %%%%%
\renewcommand{\kanjifamilydefault}{\gtdefault}
% \usepackage{otf} % otfパッケージ
\usepackage[deluxe]{otf} 
\usepackage{txfonts} % 数式・英文ローマン体を Lxfont にする
% \usepackage[T1]{fontenc} % 8bit フォント
% \usepackage{minijs}
% \usepackage{textcomp} % 欧文フォントの追加
% \usepackage[utf8]{inputenc} % 文字コードをUTF-8

%%%%% Beamer %%%%%
\usetheme{Madrid}
\useinnertheme{rectangles}
%\useoutertheme{smoothbars}
\setbeamercolor{enumerate}{fg=white, bg=black}
\usefonttheme{professionalfonts}
\setbeamertemplate{frametitle}[default][center]
\setbeamertemplate{navigation symbols}{}
% \setbeamercovered{transparent} % 好みに応じてどうぞ
\setbeamertemplate{footline}[frame number]
\setbeamercolor{page number in head/foot}{fg=black} % ページ数を表示する
% \setbeamerfont{footline}{size=\normalsize,series=\bfseries}
\setbeamerfont{footline}{size=\scriptsize,series=\mdseries}
\setbeamercolor{footline}{fg=black,bg=black}
\setbeamertemplate{blocks}[rounded][shadow=true]
\setbeamertemplate{items}[ball]
% \setbeamertemplate{enumerate items}[default]
% \setbeamerfont{alerted text}{series=\bfseries}

%%%% My macro %%%%%
%%%%%%%%%%%%%%%%%%%%%%%%%%%%%%%%%%%%%%%%%%%%%%%%%%%%%%%%%%%%%%%%
% User-defined Macro
%%%%%%%%%%%%%%%%%%%%%%%%%%%%%%%%%%%%%%%%%%%%%%%%%%%%%%%%%%%%%%%%
\newcommand{\compress}{\itemsep0pt\parsep0pt\parskip0pt\partopsep0pt}
% \newcommand{\compress}{\itemsep1pt plus1pt\parsep0pt\parskip0pt}
% \newcommand{\code}[1]{\lstinline[basicstyle=\ttfamily]{#1}}
\newcommand{\gringo}{\textit{gringo}}
\newcommand{\clasp}{\textit{clasp}}
\newcommand{\clingo}{\textit{clingo}}
\newcommand{\teaspoon}{\textit{teaspoon}}
\newcommand{\sat}{\textsf{SAT}}
\newcommand{\unsat}{\textsf{UNSAT}}
% \newcommand{\web}[2]{\href{#1}{#2\ \raisebox{-0.15ex}{\beamergotobutton{Web}}}}
% \newcommand{\doi}[2]{\href{#1}{#2\ \raisebox{-0.15ex}{\beamergotobutton{DOI}}}}
% \newcommand{\weblink}[1]{\web{#1}{#1}}
% \newcommand{\imp}{\mathrel{\Rightarrow}}
% \newcommand{\Iff}{\mathrel{\Leftrightarrow}}
% \newcommand{\mybox}[1]{\fbox{\rule[.2cm]{0cm}{0cm}\mbox{${#1}$}}}
% \newcommand{\mycbox}[2]{\tikz[baseline]\node[fill=#1!10,anchor=base,rounded corners=2pt] () {#2};}
% \newcommand{\naf}[1]{\ensuremath{{\sim\!\!{#1}}}}
% \newcommand{\head}[1]{\ensuremath{\mathit{head}(#1)}}
% \newcommand{\body}[1]{\ensuremath{\mathit{body}(#1)}}
% \newcommand{\atom}[1]{\ensuremath{\mathit{atom}(#1)}}
% \newcommand{\poslits}[1]{\ensuremath{{#1}^+}}
% \newcommand{\neglits}[1]{\ensuremath{{#1}^-}}
% \newcommand{\pbody}[1]{\poslits{\body{#1}}}
% \newcommand{\nbody}[1]{\neglits{\body{#1}}}
% \newcommand{\Cn}[1]{\ensuremath{\mathit{Cn}(#1)}}
% \newcommand{\reduct}[2]{\ensuremath{#1^{#2}}}
% \newcommand{\OK}{\mbox{\textcolor{green}{\Pisymbol{pzd}{52}}}}
% \newcommand{\KO}{\mbox{\textcolor{red}{\Pisymbol{pzd}{56}}}}
% \newcommand{\code}[1]{\lstinline[basicstyle=\ttfamily]{#1}}
% \newcommand{\lw}[1]{\smash{\lower2.ex\hbox{#1}}}
\newcommand{\llw}[1]{\smash{\lower3.ex\hbox{#1}}}

\newenvironment{tableC}{%
  \scriptsize
  \renewcommand{\arraystretch}{0.9}
  \tabcolsep = 0.6mm
  % \begin{tabular}[t]{p{6mm}|rlr|rlr|rlr|rlr|rlr}\hline
  %   \multicolumn{1}{l|}{\llw{問題   }} &
  \begin{tabular}[t]{l|rlr|rlr|rlr|rlr|rlr}\hline
    \multicolumn{1}{l|}{\llw{問題}} &
    \multicolumn{3}{c|}{UD1} &
    \multicolumn{3}{c|}{UD2} &
    \multicolumn{3}{c|}{UD3} &
    \multicolumn{3}{c|}{UD4} &
    \multicolumn{3}{c}{UD5} \\
    & 
    \multicolumn{1}{c}{既知の} & & \multicolumn{1}{c|}{ASP} & 
    \multicolumn{1}{c}{既知の} & & \multicolumn{1}{c|}{ASP} & 
    \multicolumn{1}{c}{既知の} & & \multicolumn{1}{c|}{ASP} & 
    \multicolumn{1}{c}{既知の} & & \multicolumn{1}{c|}{ASP} & 
    \multicolumn{1}{c}{既知の} & & \multicolumn{1}{c}{ASP} \\
    & 
    ベスト & &  & 
    ベスト & &  & 
    ベスト & &  & 
    ベスト & &  & 
    ベスト & &  \\
    \hline
  }{%
    \hline
  \end{tabular}
}


%%%%%%%%%%%%%%%%%%%%%%%%%%%%%%%%%%%%%%%%%%%%%%%%%%%%
\title{解集合プログラミングを用いた\\車両装備仕様問題の解法}
\author{竹内 頼人}
\institute{番原研究室}
\date{M1研究紹介\\2020年5月22日}
\begin{document}
\begin{frame} {}
 \titlepage
\end{frame}
%%%%%%%%%%%%%%%%%%%%%%%%%%%%%%%%%%%%%%%%%%%%%%%%%%%%
\begin{frame}{車両装備仕様問題}
  \begin{itemize}
  \item 車両の装備を決定するには,販売される国や地域の法規や規制,
    地域や市場の特性,市場の嗜好や競合など十分に考慮する必要がある.
  \item 現状では専門知識をもつ技術者の多大な労力が費やされている.
  \item 日本では2020年度から,
    \structure{\bf 企業別平均燃費}
    (Corporate Average Fuel Efficiency; \structure{\bf CAFE})基準
    と呼ばれる燃費規制が採用される.
  \end{itemize}
  \vfill
  \begin{alertblock}{車両装備仕様問題}
    \begin{itemize}
    \item 組合せ最適化問題として定式化される.
    \item \alert{個数制約},\alert{燃費制約}(CAFE基準),\alert{要求制約}から構成される.
    \item \alert{販売台数を最大化}する\textbf{装備仕様}
      (装備タイプと装備オプションの組合せ)を求めることが目的である.
    \end{itemize}
  \end{alertblock}
\end{frame}
%%%%%%%%%%%%%%%%%%%%%%%%%%%%%%%%%%%%%%%%%%%%%%%%%%%%
\begin{frame}{車両装備仕様問題の例}
\begin{exampleblock}{}\centering
  \begin{tabular}{l|l|r|c|c|c}
    \lw{装備タイプ}& \lw{装備オプション}& \lw{IWR値}& \multicolumn{3}{c}{装備仕様} \\\cline{4-6}
                &		&	& 1	& 2	& 3	\\\cline{1-6}
    グレード 	& STD 		& 700	& \onslide<2->{\OK}	&	&	\\%\cline{2-6}
                & DX 		& 700	&	& \onslide<2->{\OK}	&	\\%\cline{2-6}	
		& LX 		& 700	& 	&	& \onslide<2->{\OK}	\\\cline{1-6}
    エンジン	& V4 		& 120	&	&	& \onslide<2->{\OK}	\\%\cline{2-6}
                & V6 		& 200	& \onslide<2->{\OK}	& \onslide<2->{\OK}	& \\\cline{1-6}
    タイヤ	& 16インチ	& 110	& \onslide<2->{\OK}	&	&	\\%\cline{2-6}
		& 17インチ 	& 130	&	& \onslide<2->{\OK}	&	\\%\cline{2-6}
		& 18インチ 	& 150	&       &	& \onslide<2->{\OK}	\\\cline{1-6}
    トランス	& MT		& 55	&	& &	\\%\cline{2-6}
    ミッション	& HEV 	        & 95	&	\onslide<2->{\OK}& \onslide<2->{\OK}	&\\%\cline{2-6}
                & AT 		& 115	&	& 	& \onslide<2->{\OK}
  \end{tabular}
\end{exampleblock}
%
\begin{itemize}
\item \onslide<1->{装備タイプはすべて\structure{\bf 必須}とする.}
\item \onslide<2>{実行可能解の例を{\OK}マークで示す.}
\end{itemize}
\end{frame}
%%%%%%%%%%%%%%%%%%%%%%%%%%%%%%%%%%%%%%%%%%%%%%%%%%%%
\begin{frame}[shrink]{個数制約}
\begin{alertblock}{}
  各装備仕様$g$, 各装備タイプ$i$に対して,$i$が$g$で選択されるならば,
  $i$の装備オプションのうち,ちょうど1つが$g$で選択される.
\end{alertblock}
\begin{exampleblock}{}\centering
  \begin{tabular}{l|l|r|c|c|c} %\cline{1-6}
    \lw{装備タイプ}& \lw{装備オプション}	& \lw{IWR値}& \multicolumn{3}{c}{装備仕様} \\\cline{4-6}
                &			&		& 1	& 2	& 3	\\\cline{1-6}
    グレード 	&STD 			& 700	& \OK	&	&	\\%\cline{2-6}
                & DX 			& 700	&	& \OK	&	\\%\cline{2-6}	
                & LX 			& 700	& 	&	& \OK	\\\cline{1-6}
    エンジン	& V4 			& 120	&	&	& \OK	\\%\cline{2-6}
                & V6 			& 200	& \OK	& \OK	& \\\cline{1-6}
    タイヤ	& 16インチ	& 110	&  \OK	&	&	\\%\cline{2-6}
                & 17インチ 	& 130	&	& \OK	&	\\%\cline{2-6}
                & 18インチ 	& 150	& &	& \OK	\\\cline{1-6}
    トランス	& MT		& 55	&	 &  &	\\%\cline{2-6}
    ミッション	& HEV 	& 95	& \OK	& \OK	&	\\%\cline{2-6}
                & AT 		& 115	&	& 	& \OK %\\\cline{1-6}
  \end{tabular}
\end{exampleblock}
\end{frame}
%%%%%%%%%%%%%%%%%%%%%%%%%%%%%%%%%%%%%%%%%%%%%%%%%%%%
\begin{frame}{燃費制約 (CAFE基準)と目的関数}
\footnotesize
\begin{alertblock}{}
  \begin{itemize}\compress
  \item 装備仕様$g$の燃費を$FE_{g}$, 予想販売台数を$SV_{g}$,CAFE基準
    値を$X$とする.
    \[ 
      \frac{FE_{1}\cdot SV_1 + FE_{2} \cdot SV_2 + FE_{3} \cdot SV_3 }{SV_1 + SV_2 + SV_3}
      \geq X
      \qquad\qquad
      SV_1 + SV_2 + SV_3\longrightarrow \textrm{最大}
    \]
  \item $FE_{g}$と$SV_{g}$は,装備仕様$g$の\structure{\bf IWR値の和}か
    ら算出される.
  \end{itemize}
\end{alertblock}
\begin{exampleblock}{}\centering
  \renewcommand{\arraystretch}{0.9}
  \begin{tabular}{l|l|r|c|c|c} %\cline{1-6}
    \lw{装備タイプ}& \lw{装備オプション}& \lw{IWR値}& \multicolumn{3}{c}{装備仕様} \\\cline{4-6}
                &		&	& 1	& 2	& 3	\\\cline{1-6}
    グレード 	&STD 		& 700	& \OK	&	&	\\%\cline{2-6}
                & DX 		& 700	&	& \OK	&	\\%\cline{2-6}	
                & LX 		& 700	& 	&	& \OK	\\\cline{1-6}
    エンジン	& V4 		& 120	&	&	& \OK	\\%\cline{2-6}
                & V6 		& 200	& \OK	& \OK	& \\\cline{1-6}
    タイヤ	& 16インチ	& 110	&  \OK	&	&	\\%\cline{2-6}
                & 17インチ 	& 130	&	& \OK	&	\\%\cline{2-6}
                & 18インチ 	& 150	& &	& \OK	\\\cline{1-6}
    トランス	& MT		& 55	&	 &  &	\\%\cline{2-6}
    ミッション	& HEV 	& 95	& \OK	& \OK	&	\\%\cline{2-6}
                & AT 		& 115	&	& 	& \OK \\\cline{1-6}
    \multicolumn{1}{l}{} & \multicolumn{2}{r}{\structure{\bf IWR値の和}} & \multicolumn{1}{c}{1,105} &%
		\multicolumn{1}{c}{1,125} & \multicolumn{1}{c}{1,085}\\ 
	\end{tabular}
\end{exampleblock}
\end{frame}
%%%%%%%%%%%%%%%%%%%%%%%%%%%%%%%%%%%%%%%%%%%%%%%%%%%%
\begin{frame}{解集合プログラミング(Answer Set Programing; ASP)}
 \begin{itemize}
 \item \structure{\bf ASP言語}は,一階論理に基づく知識表現言語の一種である.
 \item \structure{\bf ASPプログラム}は,ASPルールの有限集合である.
 \item \structure{\bf ASPシステム}は,安定モデル意味論~[Gelfond and Lifschitz '88]
   に基づく解集合を計算するシステムである.
 \item 近年,SAT技術を利用した高速なASPシステムが開発され,
%   ロボット工学,システム検証,システム生物学,
   スケジューリング,システム生物学
   など様々な分野への実用的応用が急速に拡大している.
 \end{itemize}
\vfill
 \begin{alertblock}{車両装備仕様問題に対してASPを用いる利点}
   \begin{itemize} 
   \item ASP言語の高い表現力により,各種制約を簡潔に記述できる.
   \item 高速なASPシステムを利用できる.
     \begin{itemize}
     \item 一階ASPプログラムを命題ASPプログラムに\alert{\bf 基礎化}した後,
       解集合を求めるシステムが主流
     \end{itemize}
   \item 解の最適性を保証できる.最適解の列挙も可能である.
   \end{itemize}
 \end{alertblock}
\end{frame}
%%%%%%%%%%%%%%%%%%%%%%%%%%%%%%%%%%%%%%%%%%%%%%%%%%%%
\begin{frame}{研究目的}
  \begin{alertblock}{目的}
    ASP技術を活用して,大規模な車両装備仕様問題を効率よく解くシステム
    を実現すること.
  \end{alertblock}

  \begin{block}{研究内容}
    \begin{enumerate}
    \item \structure{\bf 車両装備仕様問題に対する2種類のASP符号化(基本・改良)を考案}
      \begin{itemize}
      \item 問題を簡潔に記述できることを確認した
        (ASPのルール\alert{\bf 11}個).
      \item 改良符号化は,基本符号化と比較して,
        \alert{\bf IWR値の和の上下限を厳密に計算する}ように改良されている.
	\item この改良により,基礎化後のルール数を少なく抑えることができ,
		大規模な問題に対する有効性が期待できる.
      \end{itemize}
    \item \structure{\bf 企業から提供された実データを用いた実験・評価}
      \begin{itemize}
      \item 小規模な問題に対して,その最適解を得ることができた.
      \item 大規模な問題に対して,改良符号化の優位性が確認できた.
      \end{itemize}
    \end{enumerate}
  \end{block}
  \pause
  \begin{itemize}
  	\item IWR値の和を表す制約モデルを用いて,2種類の提案符号化の違いを説明する.
  \end{itemize}
\end{frame}
% %%%%%%%%%%%%%%%%%%%%%%%%%%%%%%%%%%%%%%%%%%%%%%%%%%%% 
% \begin{frame}
% \frametitle{ASPを用いた問題解法}

% \begin{center}%\small
% \begin{tikzpicture}[
%   ->,>=stealth',shorten >=1pt,auto,node distance=1cm,
%   rectangle/.style = {
%     draw,thick,fill=red!10,rounded corners=2pt,
%     minimum width=3cm,minimum height=1.0cm}
%   ]
%   \node[rectangle] (n1) {車両装備仕様問題};
%   \node[rectangle] (n2) [right=3cm of n1] {ASPプログラム};
%   \path[->,draw,line width=1pt] (n1) edge node {モデリング} (n2);
%   \node[rectangle] (n3) [below=1cm of n2] {解集合};
%   \path[->,draw,line width=1pt] (n2) edge node {ASPシステム} (n3);
%   \node[rectangle] (n4) [below=1cm of n1] {問題の解};
%   \path[->,draw,line width=1pt] (n3) edge node {解釈} (n4);
%   \path[->,draw,dashed,line width=1pt] (n1) edge node {} (n4);
% %  \only<2>{\draw[line width=1.2pt,color=red,dashed] (4,1) -- (8,1) -- (8,-1) -- (4,-1) -- cycle;}
% %  \only<3>{\draw[line width=1.2pt,color=red,dashed] (4,1) -- (8,1) -- (8,-3) -- (4,-3) -- cycle;}
% %  \only<4>{\draw[line width=1.2pt,color=red,dashed] (-2,-1) -- (8,-1) -- (8,-3) -- (-2,-3) -- cycle;}
% \end{tikzpicture}
% \end{center}
% \begin{enumerate}
% \item 問題を\structure{ASPプログラム}としてモデリングする.
% \item ASPシステムは,一階ASPプログラムを命題ASPプログラムに基礎化した後,
%   \structure{解集合}(一種の最小モデル)を計算する.
% \item 解集合を解釈して元の問題の解を得る.
% \end{enumerate}

% \end{frame}
%%%%%%%%%%%%%%%%%%%%%%%%%%%%%%%%%%%%%%%%%%%%%%%%%%%%
\begin{frame}{基本符号化のベースとなる制約モデル}
  \begin{block}{問題の入力 (一部)}
    \begin{itemize}\compress
     \item $V$: 装備オプションの集合
     \item $w_{j}$: 装備オプション$j\in V$のIWR値
     \item $G$: 求めたい装備仕様の数
    \end{itemize}
  \end{block}

  \begin{block}{IWR値の和に関する制約}
  \[
    \begin{array}{lr}
      x_{jg}\in\{0,1\} & j\in V, g\in\{1,\ldots,G\} \\
      y_{g}\in\{0,\ldots, \displaystyle\sum_{j\in V}w_{j}\} & g\in\{1,\ldots,G\} \\
      y_{g} = \displaystyle\sum_{j\in V}w_{j}x_{jg} & g\in\{1,\ldots,G\}
    \end{array}
  \]
  \end{block}

  \begin{itemize}\small
  \item 0-1変数$x_{jg}$は,装備オプション$j$が装備仕様$g$で選択されることを意味
  \item 整数変数$y_{g}$は,装備仕様$g$のIWR値の和を表す.
  \end{itemize}
\end{frame}
%%%%%%%%%%%%%%%%%%%%%%%%%%%%%%%%%%%%%%%%%%%%%%%%%%%%
\begin{frame}{改良符号化での制約モデル}
  \begin{alertblock}{}
      必須装備タイプと個数制約の性質を利用して,
      IWR 値の和($y_g$)の上下限を厳密に計算することができる.
  \end{alertblock}

  \begin{block}{問題の入力 (一部)}
    \begin{itemize}\compress
    % \item $V$: 装備オプションの集合
    % \item $w_{j}$: 装備オプション$j\in V$のIWR値
    % \item $G$: 求めたい装備仕様の数
    \item $VP$: 装備タイプの集合
    \item $VP^{*}\subseteq VP$: 必須装備タイプの集合
    \item $V_{i}\subseteq V$: 装備タイプ$i\in VP$が選択できる装備オプションの集合
    \end{itemize}
  \end{block}

  \begin{block}{IWR値の和に関する制約}
  \[
    \begin{array}{lr}
      x_{jg}\in\{0,1\} & j\in V, g\in\{1,\ldots,G\} \\
      \alert{y_{g}\in\{\displaystyle\sum_{i\in VP^{*}}\min_{j\in
      V_{i}}w_{j},\ldots, \sum_{i\in VP}\max_{j\in V_{i}}w_{j}\}} 
      & g\in\{1,\ldots,G\} \\
      y_{g} = \displaystyle\sum_{j\in V}w_{j}x_{jg} & g\in\{1,\ldots,G\}
    \end{array}
  \]
  \end{block}
\end{frame}
% %%%%%%%%%%%%%%%%%%%%%%%%%%%%%%%%%%%%%%%%%%%%%%%%%%%% 
% \begin{frame}{IWR値の和}

%   \begin{itemize}
%   \item $G$: 求めたい装備仕様の数
%   \item $V$: 装備オプションの集合
%   \item $w_{j}$: 装備オプション$j\in V$のIWR値
%   \end{itemize}
%   \begin{itemize}
%   \item $x_{jg}$: 装備オプション$j\in V$が装備仕様$g\in\{1,\ldots,G\}$で選択されることを意味
%     する0-1変数
%   \item $y_{g}$: 装備仕様$g\in\{1,\ldots,G\}$のIWR値の和を表す整数変数
%   \end{itemize}
%   \[
%     x_{jg}\in\{0,1\} \qquad
%     y_{g}\in\{0,\ldots, \sum_{j\in V}w_{j}\} \qquad
%     y_{g} = \sum_{j\in V}w_{j}x_{jg}
%   \]

%   \[
%     x_{jg}\in\{0,1\} \qquad
%     y_{g}\in\{\sum_{i\in VP}\min_{j\in V_{i}}w_{j},\ldots, \sum_{i\in VP}\max_{j\in V_{i}}w_{j}\} \qquad
%     y_{g} = \sum_{j\in V}w_{j}x_{jg}
%   \]
% \end{frame}
%%%%%%%%%%%%%%%%%%%%%%%%%%%%%%%%%%%%%%%%%%%%%%%%%%%%
\begin{frame}{実験概要}
 考案したASP符号化の有効性を評価するために実験を行なった.
 \begin{itemize}
  \item ベンチマーク問題(計15問)
	\begin{itemize}
	 \item 企業から提供された問題(3問)に対して
	 \item 5通りのCAFE基準値$X \in \{8.5, 9.0, 9.5, 10.0, 10.5km/L\}$で生成
	 \item 求める装備仕様の数$G = 3$
	\end{itemize}
	\begin{exampleblock}\small
	 \centering
	 \begin{tabular}{ l|r r r }
	  問題		& 装備タイプ数	& 装備オプション数& 要求制約数 	\\ \hline
	  small	& 8			& 21		& 4		\\
	  medium	& 86		& 226		& 147	\\
	  big		& 315		& 1,337		& 0		\\
	 \end{tabular}
	\end{exampleblock}
  \item ASPシステム: \textit{clingo-5.4.0}
  \item 制限時間: 1問あたり2時間
  \item 実験環境: Mac mini, 3.2GHz, 64GB メモリ
 \end{itemize}
\end{frame}
%%%%%%%%%%%%%%%%%%%%%%%%%%%%%%%%%%%%%%%%%%%%%%%%%%%%
\begin{frame}{実験結果: 得られた最適値と最良値}
 \begin{exampleblock}{}
  \centering
  \scriptsize
  \begin{tabular}{l|r|r|r}
   \lw{問題} & \lw{CAFE基準値} & \multicolumn{2}{c}{販売台数} \\ \cline{3-4}
            &                 & 基本符号化 & 改良符号化 \\\hline    
   small & 8.5   & \alert{6,021*} & \alert{6,021*}       \\
   small & 9.0   & \alert{5,007*} & \alert{5,007*}       \\
   small & 9.5   & \alert{2,688*} & \alert{2,688*}       \\
   small & 10.0  & \alert{1,318*} & \alert{1,318*}       \\
   small & 10.5  & UNSAT          & UNSAT    \\\hline
   medium & 8.5  & 6,010          & \alert{6,021}        \\
   medium & 9.0  & \alert{5,595}  & \alert{5,595}        \\
   medium & 9.5  & \alert{3,447}  & 3,430        \\
   medium & 10.0 & 2,245          & \alert{2,250}        \\
   medium & 10.5 & 1,690          & \alert{1,845}        \\\hline
   big & 8.5     & TO             & \alert{3,877}        \\
   big & 9.0     & 1,038          & \alert{4,623}        \\
   big & 9.5     & 688            & \alert{3,121}        \\
   big & 10.0    & 1,634          & \alert{2,064}        \\
   big & 10.5    & 538            & \alert{904}         \\\hline
   \multicolumn{2}{l}{最適値・最良値の数} & \multicolumn{1}{r}{6} & \alert{13} \\
  \end{tabular}
 \end{exampleblock}
 \begin{itemize}
  \item 改良符号化は,基本符号化と比較して,多くの問題に対してより良い解を得た.
  \item 大規模な問題に対する改良符号化の優位性が確認できた.
 \end{itemize}	
\end{frame}
%%%%%%%%%%%%%%%%%%%%%%%%%%%%%%%%%%%%%%%%%%%%%%%%%%%%
\begin{frame}{まとめ}
 \begin{itemize}
  \item \structure{\bf 車両装備仕様問題に対する2種類のASP符号化を考案}
	\begin{itemize}
	 \item 問題を簡潔に記述できることを確認した
	       (ASPのルール\alert{\bf 11}個).
	 \item 改良符号化は,基本符号化と比較して,
	       \alert{\bf IWR値の和の上下限を厳密に計算する}ように改良されている.
	\end{itemize}
  \item \structure{\bf 企業から提供された実データを用いた実験・評価}
	\begin{itemize}
	 \item 小規模な問題に対して,その最適解を得ることができた.
	 \item 大規模な問題に対して,改良符号化の優位性が確認できた.
	\end{itemize}
 \end{itemize}
 \vfill
 \begin{alertblock}{今後の課題}
  \begin{itemize}
   \item 企業の技術者へのヒアリングに基づく問題拡張
	 \begin{itemize}
	  \item 装備オプション間の排他制約などの新たな制約の追加
	  \item 共通部品の最大化などの新たな目的関数の追加
	 \end{itemize}
  \end{itemize}
 \end{alertblock}
\end{frame}
%%%%%%%%%%%%%%%%%%%%%%%%%%%%%%%%%%%%%%%%%%%%%%%%%%%%% 
\begin{frame}{最適解の個数}
 \begin{itemize}
  \item clingoはオプションを加えることで最適解の列挙が可能
  \item 改良符号化を用いて,問題smallに対して最適解の個数を調査
 \end{itemize}

 \begin{exampleblock}{}\centering
  \begin{tabular}{l|r|r|c}
   問題 & CAFE基準値 & 販売台数 & 最適解の個数 \\ \hline
   small & 8.5   & 6,021 & \alert{2} \\ 
   small & 9.0   & 5,007 & \alert{2} \\
   small & 9.5   & 2,688 & 1 \\
   small & 10.0  & 1,318 & 1 \\
   small & 10.5  & UNSAT & - \\ 
  \end{tabular}
 \end{exampleblock}
 \begin{itemize}
  \item CAFE基準値が8.5,9.0km/Lのとき,最適解が複数存在 
  \item 最適解の中でもより優れた解を選択したい
 \end{itemize}

 \begin{alertblock}{提案}
  目的関数として,共通部品数の最大化を追加
 \end{alertblock}
\end{frame}
%%%%%%%%%%%%%%%%%%%%%%%%%%%%%%%%%%%%%%%%%%%%%%%%%%%%% 
\begin{frame}{共通部品数の最大化}
 \begin{exampleblock}{}\centering
  \small
  \begin{tabular}{l|l|r|c|c|c} %\cline{1-6}
    \lw{装備タイプ}& \lw{装備オプション}	& \lw{IWR値}& \multicolumn{3}{c}{装備仕様} \\\cline{4-6}
                &			&		& 1	& 2	& 3	\\\cline{1-6}
    グレード 	& \alert{STD}		& 700	& \OK	&	&	\\%\cline{2-6}
                & \alert{DX}		& 700	&	& \OK	&	\\%\cline{2-6}	
                & \alert{LX}		& 700	& 	&	& \OK	\\\cline{1-6}
    エンジン	& \alert{V4} 			& 120	&	&	& \OK	\\%\cline{2-6}
                & \alert{V6} 			& 200	& \OK	& \OK	& \\\cline{1-6}
    タイヤ	& \alert{16インチ}	& 110	&  \OK	&	&	\\%\cline{2-6}
                & \alert{17インチ} 	& 130	&	& \OK	&	\\%\cline{2-6}
                & \alert{18インチ} 	& 150	& &	& \OK	\\\cline{1-6}
    トランス	& MT		& 55	&	 &  &	\\%\cline{2-6}
    ミッション	& \alert{HEV} 	& 95	& \OK	& \OK	&	\\%\cline{2-6}
                & \alert{AT} 		& 115	&	& 	& \OK \\\cline{1-6}
    \multicolumn{1}{c}{選択オプション数} & \multicolumn{1}{l}{\alert{10}} & \multicolumn{4}{r}

  \end{tabular}
 \end{exampleblock}

 \begin{alertblock}{提案手法}
  選択される装備オプションの数を最小化する.
 \end{alertblock}
\end{frame}
%%%%%%%%%%%%%%%%%%%%%%%%%%%%%%%%%%%%%%%%%%%%%%%%%%%%% 
\begin{frame}{共通部品数の最大化}
 \begin{block}{実験概要}
  \begin{itemize}
   \item 改良符号化に共通部品数最大化の目的関数を追加
	 \begin{itemize}
	  \item 優先度: 販売台数最大化 $>$ 共通部品数最大化
	 \end{itemize}
   \item 共通部品数最大化の有無で最適解の個数を比較

  \end{itemize}
 \end{block}
 
 \begin{exampleblock}{実験結果}\centering
  \begin{tabular}{l|r|r|c|c}
   \lw{問題} & \lw{CAFE基準値} & \lw{販売台数} & \multicolumn{2}{c}{最適解の個数} \\ \cline{4-5}
            &                &              & 改良符号化 & +共通部品数最大化 \\ \hline        
   small & 8.5   & 6,021 & 2 & 2\\ 
   small & 9.0   & 5,007 & \alert{2} & \alert{1}\\
   small & 9.5   & 2,688 & 1 & 1 \\
   small & 10.0  & 1,318 & 1 & 1 \\
   small & 10.5  & UNSAT & - & - \\ 
  \end{tabular}
 \end{exampleblock}
 \begin{itemize}
  \item CAFE基準値9.0km/Lでは,最適解の個数が減少した
  \item CAFE基準値8.5km/Lでは,最適解が複数のままだった
 \end{itemize}
\end{frame}
%%%%%%%%%%%%%%%%%%%%%%%%%%%%%%%%%%%%%%%%%%%%%%%%%%%%% 
\appendix
%%%%%%%%%%%%%%%%%%%%%%%%%%%%%%%%%%%%%%%%%%%%%%%%%%%%
\begin{frame}{要求制約}
 \begin{alertblock}{}
  要求制約$X\longrightarrow Y$は,装備オプション$X$が選択されるならば,
  装備オプション$Y$も選択されなければならないことを意味する.
 \end{alertblock}
 \begin{center}
  STD$\longrightarrow$16インチ,\quad
  DX$\longrightarrow$17インチ,\quad
  LX$\longrightarrow$18インチ,\quad
  LX$\longrightarrow$AT
 \end{center}
 \begin{exampleblock}{}\small
  \centering
  \begin{tabular}{l|l|r|c|c|c} %\cline{1-6}
   装備タイプ	& 装備オプション 	& IWR値	& \multicolumn{3}{c}{装備仕様} \\\cline{4-6}
                &		&	& 1	& 2	& 3	\\\cline{1-6}
   グレード 	& STD 		& 700	& \OK	&	&	\\\cline{2-6}
                & DX 		& 700	&	& \OK	&	\\\cline{2-6}	
                & LX 		& 700	& 	&	& \OK	\\\cline{1-6}
   エンジン	& V4 		& 120	&	&	& \OK	\\\cline{2-6}
                & V6 		& 200	& \OK	& \OK	& \\\cline{1-6}
   タイヤ	& 16インチ	& 110	&  \OK	&	&	\\\cline{2-6}
                & 17インチ 	& 130	&	& \OK	&	\\\cline{2-6}
                & 18インチ 	& 150	& &	& \OK	\\\cline{1-6}
   トランス	& MT		& 55	&	 &  &	\\\cline{2-6}
   ミッション	& HEV 	        & 95	& \OK	& \OK	&	\\\cline{2-6}
                & AT 		& 115	&	& 	& \OK %\\\cline{1-6}
  \end{tabular}
 \end{exampleblock}
\end{frame}
%%%%%%%%%%%%%%%%%%%%%%%%%%%%%%%%%%%%%%%%%%%%%%%%%%%%
\begin{tabular}{l|p{8cm}}
  符号化名 & 遷移制約 \\\hline
  origin & 任意の二つの頂点に対し違反する組合せを列挙することで表現  \\ \hline
  changed & 遷移制約を ASP の個数制約を用いて表現\\\hline
  unchanged & 「各遷移で色が変化しない頂点は$|V|-1$個」という制約を,
         ASPの個数制約を用いて表現
\end{tabular}
%%%%%%%%%%%%%%%%%%%%%%%%%%%%%%%%%%%%%%%%%%%%%%%%%%%%
\begin{frame}{比較結果: 基礎化後のルール数} 
 \begin{exampleblock}{IWR値の和に関する制約の基礎化後のルール数} \centering 
  \begin{tabular}{c|r|r|r}
   問題 	& 基本符号化 	& 改良符号化 & 削減率(\%)	\\\hline
   small 	& 2,820		& \alert{576} & 80\% $\downarrow$ \\
   medium 	& 7,545		& \alert{1,758} & 80\% $\downarrow$ \\
   big 		& 62,745	& \alert{1,488} & 98\% $\downarrow$ \\
  \end{tabular}
 \end{exampleblock}
 \begin{itemize}
  \item 改良符号化では,IWR値の和の上下限を厳密に計算することで,
	基礎化後のルール数が大幅に削減されることが確認できた.
 \end{itemize}
\end{frame}
%%%%%%%%%%%%%%%%%%%%%%%%%%%%%%%%%%%%%%%%%%%%%%%%%%%%
\begin{frame}{比較結果: CPU時間}
 最適解が求まった問題インスタンスでのCPU時間を比較する.
 \begin{exampleblock}{} \centering 
  \begin{tabular}{crrr}
   問題		& CAFE基準値(km/L)	 & 基本符号化(s)  & 改良符号化(s)   \\\hline
   small 	& 8.5  & 37.868          	& \alert{23.318}          \\
   small	& 9.0  & 48.965          & \alert{43.362}          \\
   small	& 9.5  & \alert{95.110}          & 173.172         \\
   small	& 10.0 & 99.954          & \alert{0.343}           \\
   small	& 10.5 & 439.613         & \alert{0.080}           \\\hline
   \multicolumn{2}{r}{平均}		 & 144.302			& \alert{48.055}
  \end{tabular}
 \end{exampleblock}
 \begin{itemize}
  \item 5問中4問に対して,改良符号化の方が高速に最適解を求めた.
  \item 平均では,改良符号化のCPU時間は基本符号化の約1/3であった.
 \end{itemize}
\end{frame}
%%%%%%%%%%%%%%%%%%%%%%%%%%%%%%%%%%%%%%%%%%%%%%%%%%%% 
\begin{frame}
 \frametitle{ASPを用いた問題解法}
 \begin{center}%\small
  \begin{tikzpicture}[
   ->,>=stealth',shorten >=1pt,auto,node distance=1cm,
   rectangle/.style = {
   draw,thick,fill=red!10,rounded corners=2pt,
   minimum width=3cm,minimum height=1.0cm}
   ]
   \node[rectangle] (n1) {車両装備仕様問題};
   \node[rectangle] (n2) [right=3cm of n1] {ASPプログラム};
   \path[->,draw,line width=1pt] (n1) edge node {モデリング} (n2);
   \node[rectangle] (n3) [below=1cm of n2] {解集合};
   \path[->,draw,line width=1pt] (n2) edge node {ASPシステム} (n3);
   \node[rectangle] (n4) [below=1cm of n1] {問題の解};
   \path[->,draw,line width=1pt] (n3) edge node {解釈} (n4);
   \path[->,draw,dashed,line width=1pt] (n1) edge node {} (n4);
   %  \only<2>{\draw[line width=1.2pt,color=red,dashed] (4,1) -- (8,1) -- (8,-1) -- (4,-1) -- cycle;}
   %  \only<3>{\draw[line width=1.2pt,color=red,dashed] (4,1) -- (8,1) -- (8,-3) -- (4,-3) -- cycle;}
   %  \only<4>{\draw[line width=1.2pt,color=red,dashed] (-2,-1) -- (8,-1) -- (8,-3) -- (-2,-3) -- cycle;}
  \end{tikzpicture}
 \end{center}
 \begin{enumerate}
  \item 問題を\structure{ASPプログラム}としてモデリングする.
  \item ASPシステムは,一階ASPプログラムを命題ASPプログラムに基礎化した後,
  \item 解集合を解釈して元の問題の解を得る.
	\structure{解集合}(一種の最小モデル)を計算する.
 \end{enumerate}

\end{frame}
%%%%%%%%%%%%%%%%%%%%%%%%%%%%%%%%%%%%%%%%%%%%%%%%%%%%

\end{document}
%%% Local Variables:
%%% mode: japanese-latex
%%% TeX-master: t
%%% End:
