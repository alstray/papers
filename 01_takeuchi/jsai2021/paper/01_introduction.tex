\section{はじめに}\label{sec:introduction}
% ------------------------------------------
  \thicklines
  \setlength{\unitlength}{1.28pt}
  \small
  \begin{picture}(280,57)(4,-10)
    \put(  0, 20){\dashbox(50,24){\shortstack{根付き全域森\\問題}}}
    \put( 60, 20){\framebox(50,24){変換器}}
    \put(120, 20){\dashbox(50,24){\shortstack{ASPファクト}}}
    \put(120,-10){\alert{\bf\dashbox(50,24){\scriptsize{\shortstack{ASP符号化\\(論理プログラム)}}}}}
    \put(180, 20){\framebox(50,24){ASPシステム}}
    \put(240, 20){\dashbox(50,24){\shortstack{根付き全域森\\問題の解}}}
    \put( 50, 32){\vector(1,0){10}}
    \put(110, 32){\vector(1,0){10}}
    \put(170, 32){\vector(1,0){10}}
    \put(230, 32){\vector(1,0){10}}
    \put(170, +2){\line(1,0){4}}
    \put(174, +2){\line(0,1){30}}
  \end{picture}  

% ------------------------------------------

車両装備仕様とは,簡単に言うと,自動車のカタログに記載されているモデ
ル/グレードと装備の組合せのことである.
車両装備仕様を決めるには,販売される国や地域の法規や規制,
地域や市場の特性,市場の嗜好や競合など十分に考慮する必要があり,
現状では専門知識をもつ技術者の多大な労力が費やされている.
そのため,車両装備仕様探索の自動化・効率化は自動車メーカーにとって
重要な課題の一つである.

\textbf{多目的車両装備仕様問題}は組合せ最適化問題の一種であり,主に装備タイ
プと装備オプションから構成される.
\textbf{装備タイプ}はエンジンやトランスミッションなどの装備の種類を表す.
\textbf{装備オプション}は4気筒エンジン,CVTなどの具体的な装備を表す.
多目的車両装備仕様問題の目的は,
与えられた装備仕様の個数,
装備タイプの集合,
装備オプションの集合から,
装備および燃費に関する制約を満たしつつ,
予想販売台数の最大化や装備オプション数の最小化など,トレードオフの関係にある
複数の目的関数のもとで最適な車両装備仕様を求めることである.

本研究では,燃費に関する制約として,欧米で採用され日本でも2020年度から
導入されている\textbf{企業平均燃費}
(Corporate Average Fuel Efficiency; CAFE~\cite{metimlit18:cafe})
方式を用いる.
このCAFE方式は自動車の燃費規制で,車種別ではなくメーカー全体での出荷台
数を加味した平均燃費を算出し,規制をかける方式である.
CAFE方式の特長は,ある特定の車種では燃費基準を達成できなくても,他の車
種の燃費を向上させることで基準を達成できることが可能な点である.
本論文では,CAFE方式に基づく多目的車両装備仕様問題(以降,\textbf{多目的CAFE問題}と
呼ぶ)を対象とする.

\textbf{解集合プログラミング}
(Answer Set Programming; ASP~\cite{%
  Baral03:cambridge,%
  Gelfond88:iclp,%
  Inoue08:jssst})
は,論理プログラミングから派生した宣言的プログラミングパラダイムである.
ASP言語は一階論理に基づく知識表現言語の一種である.
論理プログラムはASPのルールの有限集合である.
ASPシステムは論理プログラムから安定モデル意味論~\cite{Gelfond88:iclp}
に基づく解集合を計算するシステムである.
近年,SATソルバー技術を応用した高速ASPシステムが実現され,
ロボット工学,システム生物学,システム検証,制約充足問題,プランニング
など様々な分野への実用的応用が急速に拡大している~\cite{Gelfond16:aim}.


提案手法の有効性を評価するために,企業から提供を受けた小規模・中規模・大規模の
多目的CAFE問題に対して実行実験を行った結果,
小規模な問題では5問中4問で最適解の全列挙をすることができた.