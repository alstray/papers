\section{はじめに}\label{sec:introduction}
% ------------------------------------------
\begin{figure*}[t]
  \centering
  \thicklines
  \setlength{\unitlength}{1.28pt}
  \small
  \begin{picture}(280,57)(4,-10)
    \put( -35, 20){\dashbox(70,24){\shortstack{組合せ最適化問題\\のインスタンス}}}
    \put( 45, 20){\framebox(50,24){変換器}}
    \put(105, 20){\dashbox(70,24){\shortstack{ASPファクト}}}
    \put(105,-10){\dashbox(70,24){\shortstack{ASP符号化\\(論理プログラム)}}}
    \put(185,-10){\framebox(60,54){}}
    \put(189, 25){\framebox(52,12){ASPソルバー}}
    \put(190, -5){\framebox(50,12){LNPS}}
    % \put(180, 20){\framebox(50,24){ASPシステム}}
    \put(255, 20){\dashbox(70,24){\shortstack{組合せ最適化問題\\の最適解}}}
    \put(  35, 32){\vector(1,0){10}}
    \put(  95, 32){\vector(1,0){10}}
    \put(175, 32){\vector(1,0){10}}
    \put(245, 32){\vector(1,0){10}}
    \put(175, +2){\line(1,0){4}}
    \put(179, +2){\line(0,1){30}}
    \put(205,  7){\vector(0,1){17}}
    \put(225, 24){\vector(0,-1){17}}
    \put(190, 48){提案ソルバー}
  \end{picture}  
\caption{提案ソルバー\textit{asprior}の構成}
\label{fig:arch}
\end{figure*}

%%% Local Variables: 
%%% mode: latex
%%% TeX-master: "paper"
%%% End: 

% ------------------------------------------

\textbf{車両装備仕様}とは,簡単に言うと,自動車のカタログに記載されて
いる車種(モデルやグレード)と装備の組合せのことである.
車両装備仕様を決めるには,販売される国や地域の法規や規制,
地域や市場の特性,市場の嗜好や競合など十分に考慮する必要がある.
しかし,現状では専門知識をもつ技術者の多大な労力が費やされている.
そのため,車両装備仕様の自動生成は自動車メーカーにとって重要な課題の一
つである.

\textbf{企業平均燃費}
(Corporate Average Fuel Efficiency; CAFE
\cite{mlit18:cafe})
は自動車の燃費規制で,車種別ではなくメーカー全体での出荷
台数を加味した平均燃費を算出し,規制をかける方式である.
この方式の特長は,ある車種では燃費基準を達成できなくても,
他の車種の燃費を向上させることで基準を達成できることが可能な点である.
このCAFE方式は,欧米で採用され,日本でも2020年度から導入されている.

\textbf{多目的車両装備仕様問題}
(Multi-objective Vehicle Equipment Specification Problem)
は,与えられた車種の数,
装備タイプの集合,
装備オプションの集合から,
装備および燃費に関する制約を満たしつつ,
予想販売台数の最大化や装備オプション数の最小化など,トレードオフの関係にある
複数の目的関数のもとで最適な車両装備仕様を求める組合せ最適化問題の一種である.
本論文では,CAFE 方式に基づく多目的車両装備仕様問題
(以降,\textbf{多目的CAFE問題}と呼ぶ)を対象とする.

\textbf{解集合プログラミング}
(Answer Set Programming; ASP
\cite{Gelfond88:iclp,Inoue08:jssst})
は,論理プログラミングから派生した宣言的プログラミングパラダイムである.
ASP言語は一階論理に基づく知識表現言語の一種である.
論理プログラムは ASP のルールの有限集合である.
ASPシステムは論理プログラムから安定モデル意味論~\cite{Gelfond88:iclp}
に基づく解集合を計算するシステムである.
近年,SAT 技術を応用した高速 ASP システムが実現され,
システム検証,システム生物学,プランニングなど様々な分野への実用的応用
が急速に拡大している~\cite{Erdem16:ai-magazine}.

本論文では,ASP に基づく多目的 CAFE 問題ソルバーの実装と評価ついて述べる.
提案ソルバーは,可変性モデルで表現された問題インスタンスを ASP のファ
クト形式に変換した後,それらファクトと多目的 CAFE 問題を解くための ASP
符号化と結合し,高速 ASP システムを用いて
パレート最適解を列挙する(図~\ref{fig:arch}参照).
%
考案した ASP 符号化は,多目的 CAFE 問題の制約と2つの目的関数を簡潔に表
現できる.この符号化は,単目的 CAFE 問題の ASP 符号化~\cite{Takeuchi20:jssst}
の自然な拡張になっている.
%
ASP システムとしては,{\asprin}システム~\cite{Brewka15:casp}を使用する.
{\asprin}は,広く普及している{\clingo}をベースに,
多目的最適化や解集合の間の選好順序を記述できるように拡張されたシステム
である.
%
企業から提供を受けた小規模・中規模の多目的 CAFE 問題に対して実行実験を
行った結果,小規模な問題のパレート最適解を全列挙をすることができた.

%%% Local Variables:
%%% mode: latex
%%% TeX-master: "paper"
%%% End:
