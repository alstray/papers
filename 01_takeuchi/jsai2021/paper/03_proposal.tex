\section{多目的CAFE問題のASP符号化}\label{sec:proposal}

ASPの言語は,
一般拡張選言プログラム
(General Extended Disjunctive Program)
をベースとしている\cite{Inoue08:jssst}.
本節では,説明の簡略化のため,そのサブクラスである
標準論理プログラムについて説明する.
以降,標準論理プログラムを単に{\bf 論理プログラム}と呼ぶ.

論理プログラムは,以下の形式の\textbf{ルール}の有限集合である.
\begin{equation}
  \label{eq:rule}
  a_0\leftarrow a_1,\dots,a_m,\naf{a_{m+1}},\dots,\naf{a_n}
\end{equation}
このルールの直観的な意味は,
「$a_1,\ldots,a_m$がすべて成り立ち,$a_{m+1},\ldots,a_n$のそれぞれが成
り立たないならば,$a_0$が成り立つ」である.
ここで,
$0\leq m\leq n$ であり,
各$a_i$はアトム,
$\naf{}$は\textbf{デフォルトの否定}
\footnote{\textbf{失敗による否定}とも呼ばれる.述語論理で定義される否定($\neg$)とは意味が異なる.},
``$,$''は連言を表す.
$\leftarrow$の左側を\textbf{ヘッド},右側を\textbf{ボディ}と呼ぶ.
ボディが空のルール(すなわち\(a_0\leftarrow\))を\textbf{ファクト}と呼び,
$\leftarrow$を省略してよい.

ヘッドが空のルールを\textbf{一貫性制約}と呼び,以下のように表す.
\begin{equation}
  \label{eq:constr}
  \leftarrow a_1,\dots,a_m,\naf{a_{m+1}},\dots,\naf{a_n}
\end{equation}
例えば,一貫性制約
``\(\leftarrow a_1,a_2\)''は,「$a_1$と$a_2$が両方同時に成り立つことはない」を意味し,
``\(\leftarrow a_1, \naf{a_{2}}\)''は,「$a_1$が成り立つならば,$a_2$が成り立つ」を意味する.

ASP言語には,組合せ問題を簡潔に記述するために,
\textbf{アグリゲート}(aggregate)と呼ばれる拡張構文がいくつか用意されている.
例えば,\textbf{選択子}``$\{a_1;\ldots;a_n\}$.''は,
アトム集合\(\{a_1,\ldots,a_n\}\)の任意の部分集合を解集合に含めることを意味する.
\textbf{個数制約}は選択子の両端に選択可能な個数の上下限を付けたものである.
例えば,``\(lb\ \{a_1;\dots;a_n\}\ ub \leftarrow Body\)''と書くと,
「$Body$が成り立つならば,$a_1,\dots,a_n$のうち,$lb$個以上$ub$個以下
が成り立つ」を意味する.
\textbf{重み付き個数制約}``\(t = \#sum\ \{w_1:a_1;\ldots;w_n:a_n\}\).''は,
$a_1,\dots,a_n$のうち真となるアトムの重み和が項$t$に等しくなることを意味する.
項$w_i$は重みを表し,演算子としては``=''以外にも``$\leq$'',``$\geq$''などを使用できる.
さらに,重み付き個数制約の``$\#sum$''を,``$\#max$''や``$\#min$''に書
き換えると,重み和ではなく,真となるアトムの重みの最大値や最小値を求め
ることができる.

近年,
{\clingo}~\footnote{\url{https://potassco.org/}},
{\dlv}~\footnote{\url{http://www.dlvsystem.com/dlv/}},
{\wasp}~\footnote{\url{https://www.mat.unical.it/ricca/wasp/}}
など,SATソルバー技術を応用した高速なASPシステムが開発されている.
なかでも{\clingo}は,高性能かつ高機能なASPシステムとして世界中で広く使われている.
これらの高速ASPシステムは,変数を含む論理プログラムを変数を含まない論
理プログラムに変換(\textbf{基礎化})したのち,ASPソルバーを用いて解集合を計算する.
本論文で使用するASPシステム{\clingo}は,基礎化のためのグラウンダー
{\gringo}とASPソルバー{\clasp}をシームレスに結合したものである.

本論文では,論理プログラムの解集合の中で,選好関係の宣言・評価を可能にするシステム
{\asprin}~\cite{Brewka15:casp}も使用する.
選好関係は,{\bf 選好文}と呼ばれるプログラムの中で,次のように宣言される.
\begin{equation}
  \label{eq:preference}
  \#preference(s,t)\{e_1,\dots,e_n\}
\end{equation}
ここで,$s$ と $t$ はそれぞれ選好の名称とタイプであり,
引数の各 $e_j$ は選好の要素である. 
{\asprin} では,新たに選好タイプを定義することも可能だが,
いくつかの選好タイプはあらかじめライブラリに定義されており,
それを利用することで簡潔に選好関係を記述することができる.
% 本発表では選好のタイプとして$less(weight), more(weight), pareto$ を使用する.
そして,次のように最適化命令を記述することで,選好関係 $s$ について
最適な解集合を得ることができる.
\begin{equation}
 \label{eq:optimize}
 \#optimize(s)
\end{equation} 
このとき,{\asprin}では繰り返しASPソルバーを呼び出すことで,
最適な解集合を段階的に計算する.

本論文で提案する多目的CAFE問題の解法では,
与えられた問題インスタンスをASPのファクト形式に変換した後,
それらファクトと多目的CAFE問題を解くためのASP符号化(論理プログラム)を結合した上で,
高速ASPシステム{\clingo}と{\asprin}を用いて解を求める(図\ref{fig:arch}参照).
本論文では,説明の簡略化のため,各タイプが選択可能なオプション数の上下限値を1とする.

多目的CAFE問題の問題インスタンスは,CAFE基準値を除いてアトムに変換され,
ファクトとして表現される.
CAFE基準値はASPの定数$t$で表すものとし,実行時に\clingo のオプションから値を指定する.
各アトムは,表\ref{tab:fact}のように第\ref{sec:background}節の各入力に対応する.
\begin{table}[t]
 \caption{問題インスタンスのアトム}
 \centering
 \tabcolsep = 1mm
 \begin{tabular}{ll} \bhline
  アトム & 対応する入力番号 \\ \hline
  $vp\_def(vp)$ & 入力\ref{input:vp} \\
  $v\_def(v,vp,x)$ & 入力\ref{input:v}, \ref{input:vp-v}, \ref{input:iwr} \\
  $require\_vp(vp)$ & 入力\ref{input:req_vp} \\
  $require\_v\_v(v_1,v_2)$ & 入力\ref{input:dependency} \\
  $exclude\_v\_v(v_1,v_2)$ & 入力\ref{input:dependency} \\
  $group(1..n)$ & 入力\ref{input:g} \\ 
  $require\_g\_v(i,v)$ & 入力\ref{input:init} \\
  $fe\_map(s,fe)$ & 入力\ref{input:fe} \\
  $sv\_map(s,fe)$ & 入力\ref{input:sv} \\ \hline
 \end{tabular}
 \label{tab:fact}
\end{table}

多目的CAFE問題の各制約は,ASPの個数制約および一貫性制約を使用して
簡潔に表現される(コード\ref{code:basic.lp}).
また,コード\ref{code:basic.lp}の8〜12行目では,
各タイプが選択可能なオプション数の上下限値,および,必須かどうかの別を考慮することで,
IWR値の和である\code{S}の上下限をより厳密に計算するように工夫がされている.
これにより,\code{iwr(S,G)}に関する基礎化後のルール数を少なく抑えることができる.

%----------------------------------------
\lstinputlisting[float=tb,caption={%
  制約の符号化},%
captionpos=b,frame=single,label=code:basic.lp,%
numbers=left,%
breaklines=true,%
columns=fullflexible,keepspaces=true,%
xrightmargin=1zw,% 
xleftmargin=1zw,% 
basicstyle=\ttfamily\footnotesize]{codes/basic.lp} 
%----------------------------------------


多目的CAFE問題の選好文は,コード\ref{code:preference.lp}のように定義される.
目的関数である予想販売台数の最大化とオプション数の最小化は,
それぞれ選好名\code{max\_sv}, \code{min\_op} として定義している.
9行目で使用している選好タイプ\code{pareto} は,
選好文中で定義された選好名を引数に取り,
それらの選好関係についてパレートな解を得ることができる.
すなわち,解集合$S_1$が$S_2$よりも厳密に好まれるならば,かつそのときに限り,
引数のすべての選好関係に関して,少なくとも$S_1$は$S_2$と同等で,かつ,
少なくとも1つの選好関係に関して$S_1$は$S_2$より厳密に優れている.

%----------------------------------------
\lstinputlisting[float=tb,caption={%
  選好文},%
captionpos=b,frame=single,label=code:preference.lp,%
numbers=left,%
breaklines=true,%
columns=fullflexible,keepspaces=true,%
xrightmargin=1zw,% 
xleftmargin=1zw,% 
basicstyle=\ttfamily]{codes/preference.lp} 
%----------------------------------------

第\ref{sec:background}節と同様に多目的CAFE問題の例
(図\ref{fig:ovm_example})を用いて,
最適解をすべて列挙した結果を表~\ref{tab:prt_ans}に示す.
予想販売台数とオプション数の値から,
得られた各装備仕様が互いにトレードオフの関係にあることが分かる.
本解法の利点は,このような定性的に得られた複数の解を人が見比べることで,
定量的な判断により最終的な決定をすることができる点である.

%----------------------------------------
\begin{table*}[t]
   \centering
 % \small
 \tabcolsep=1mm
  \caption{CAFE問題(図~\ref{fig:ovm_example})のパレート解}
  \begin{tabular}{l|l|c|c|c||c|c|c||c|c|c||c|c|c} \bhline
   \multicolumn{2}{l|}{} & \multicolumn{3}{c||}{解1} & \multicolumn{3}{c||}{解2} & \multicolumn{3}{c||}{解3} & \multicolumn{3}{c}{解4}\\ \hline
   \multicolumn{2}{l|}{装備仕様} & 1 & 2 & 3 & 1 & 2 & 3 & 1 & 2 & 3 & 1 & 2 & 3 \\ \hline
   装備 & \textsf{Grade} & \textsf{STD} & \textsf{DX} & \textsf{LX} & \textsf{STD} & \textsf{DX} & \textsf{LX} & \textsf{STD} & \textsf{DX} & \textsf{LX} & \textsf{STD} & \textsf{DX} & \textsf{LX} \\
       & \textsf{Drive\_Type} & \textsf{2WD} & \textsf{2WD} & \textsf{4WD} & \textsf{2WD} & \textsf{2WD} & \textsf{4WD} & \textsf{2WD} & \textsf{2WD} & \textsf{2WD} & \textsf{2WD} & \textsf{2WD} & \textsf{2WD} \\
       & \textsf{Engine} & \textsf{V6} & \textsf{V6} & \textsf{V6} & \textsf{V6} & \textsf{V6} & \textsf{V6} & \textsf{V6} & \textsf{V6} & \textsf{V6} & \textsf{V6} & \textsf{V6} & \textsf{V6} \\
       & \textsf{Tire} & \textsf{16inch} & \textsf{17inch} & \textsf{18inch} & \textsf{16inch} & \textsf{17inch} & \textsf{18inch} & \textsf{16inch} & \textsf{17inch} & \textsf{18inch} & \textsf{16inch} & \textsf{17inch} & \textsf{18inch} \\
       & \textsf{Transmission} & \textsf{6AT} & \textsf{HEV} & \textsf{10AT} & \textsf{10AT} & \textsf{HEV} & \textsf{10AT} & \textsf{10AT} & \textsf{HEV} & \textsf{10AT} & \textsf{10AT} & \textsf{10AT} & \textsf{10AT} \\
       & \textsf{Sun\_Roof} & \textsf{-} & \textsf{-} & \textsf{-} & \textsf{-} & \textsf{-} & \textsf{-} & \textsf{-} & \textsf{-} & \textsf{-} & \textsf{-} & \textsf{-} & \textsf{-} \\ \hline
   \multicolumn{2}{l|}{予想販売台数(合計)}  & \multicolumn{3}{c||}{\bf 5,525} & \multicolumn{3}{c||}{\bf 5,475} & \multicolumn{3}{c||}{\bf 5,135} & \multicolumn{3}{c}{\bf 4,723} \\ 
   \multicolumn{2}{l|}{オプション数} & \multicolumn{3}{c||}{\bf 12} & \multicolumn{3}{c||}{\bf 11} & \multicolumn{3}{c||}{\bf 10} & \multicolumn{3}{c}{\bf 9} \\ \hline
   %\multicolumn{14}{c}{}
  \end{tabular}
 \label{tab:prt_ans}
\end{table*}
%----------------------------------------