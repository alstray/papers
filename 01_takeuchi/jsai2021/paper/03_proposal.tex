\section{ASPに基づく多目的CAFE問題ソルバー}\label{sec:proposal}

%----------------------------------------
\lstinputlisting[float=t,caption={%
多目的 CAFE 問題(図~\ref{fig:ovm})のASPファクト表現 (車種数$n=3$)},%
captionpos=b,frame=single,label=code:ovm.lp,%
numbers=none,%
breaklines=true,%
columns=fullflexible,keepspaces=true,%
xrightmargin=1zw,% 
xleftmargin=1zw,% 
basicstyle=\ttfamily\scriptsize]{codes/ovm.lp} 
%----------------------------------------

%----------------------------------------
\lstinputlisting[float=tb,caption={%
  多目的 CAFE 問題の ASP 符号化},%
captionpos=b,frame=single,label=code:basic.lp,%
numbers=left,%
breaklines=true,%
columns=fullflexible,keepspaces=true,%
xrightmargin=1zw,% 
xleftmargin=1zw,% 
basicstyle=\ttfamily\scriptsize]{codes/basic.lp} 
%----------------------------------------

提案ソルバーは,可変性モデルで表現された問題インスタンスを ASP のファ
クト形式に変換した後,それらファクトと多目的 CAFE 問題を解くための ASP
符号化と結合し,ASP システム
{\asprin}\footnote{\url{https://potassco.org/asprin/}}
を用いてパレート最適解を求める(図~\ref{fig:arch}参照).
本節では,ASP ファクト形式と ASP 符号化について述べる.
なお,説明の簡単化のため,各装備タイプが選択可能な
装備オプション数の上下限値を1とする.

%%%%%%%%%%%%%%%%%%%%%%%%%%%%%%%%%%%%
\textbf{ASP ファクト形式.}
コード~\ref{code:ovm.lp}は,
CAFE 問題(図~\ref{fig:ovm})を ASP のファクトで表したものである.
%
アトム\code{vp_def/1}は装備タイプ,
\code{v_def/3}は装備オプション,
\code{require_v_v/2}は依存制約(要求),
\code{exclude_v_v/2}は依存制約(排他)
を表している.
例えば,装備タイプ\code{Engine}は,
アトム\code{vp_def("Engine")}で,
その装備オプション\code{V4}は,
アトム\code{v_def("V4", "Engine", 120)}
で表されている.
この他,
\code{require_vp/1}は必須タイプ,
\code{group/1}は車種の識別子を表している.

%%%%%%%%%%%%%%%%%%%%%%%%%%%%%%%%%%%%
\textbf{制約の ASP 符号化.}
多目的 CAFE 問題の ASP 符号化をコード\ref{code:basic.lp}に示す.
%
1行目のルールは,
各車種\code{G},各装備タイプ\code{VP}に対して,
\code{G}が\code{VP}を装備することを意味する
アトム\code{vp(VP,G)}を,ASP の選択子を用いて導入している.
%
2行目のルールは,各車種\code{G}に対して,
\code{VP}が必須タイプならば,
\code{vp(VP,G)}が成り立たなければならないという制約を表す.
%
5行目のルールは範囲制約を表す.
アトム\code{v(V,G)}は,車種\code{G}が装備オプション\code{V}
を実装することを意味する.
このルールは,車種\code{G}が装備タイプ\code{VP}を装備するならば,
\code{G}は\code{VP}の装備オプションからちょうど1個を選択する制約を表す.
%
16行目のルールは燃費制約を表す.
このルールは,
CAFE 規制の式を以下のように変形し,ASPの重み付き個数制約で表している.
\begin{center}
\(\sum_{g=1}^{n} (FE_{g}-t)\cdot SV_{g} \geq 0\)  
\end{center}
ルール中の定数\code{t}は CAFE 基準値を表し,
{\asprin}の実行時のオプションによって与えられる.
\code{fe(FE,G)}と\code{sv(SV,G)}は,それぞれ,
$FE_{g}$と$SV_{g}$に対応している.
%
19〜22行目のルールは,依存制約(要求)を表す.
例えば,19行目のルールは,
装備オプション\code{V1}と\code{V2}の間に要求的な依存関係があり,
車種\code{G}が\code{V1}を実装するならば,
\code{G}は\code{V2}を実装しなければならない制約を表す.
同様にして,依存制約(排他)も一貫性制約を用いて簡潔に表される.

%%%%%%%%%%%%%%%%%%%%%%%%%%%%%%%%%%%%
\textbf{目的関数の ASP 符号化.}
{\asprin}言語は,解集合の間の選好順序や多目的最適化を記述できるように
拡張されている.
解集合の間の選好順序は,以下の\code{#preference}文によって定義される.
\begin{center}
  \code{#preference(}$s,t$\code{)\{}$e_1,\dots,e_n$\code{\}.}
\end{center}
ここで,$s$は選好名称,$t$は選好タイプ,$e_j$は要素を表す.
{\asprin} では,新たに選好タイプを定義することも可能だが,
\code{subset}, 
\code{less(weight)},
\code{more(weight)},
\code{pareto}
などいくつかの選好タイプはあらかじめ用意されている.
あとは,\code{#optimize(}$s$\code{)}と記述することによって,
選好名称\code{s}について最適な解集合を得ることができる.
% \begin{center}
%   \code{#optimize(}$s$\code{).}
% \end{center}

コード\ref{code:basic.lp}の31行目の\code{#preference}文は,
予想販売台数の最大化を\code{max_sv}という名称で定義している.
同様にして,35行目では,装備オプション数の最小化を
\code{min_op}という名称で定義している.
要素の\code{used_v(V)}は,
装備オプション\code{V}がいずれかの車種において実装されたことを意味する
補助アトムである.
続いて,38行目の\code{#preference}文は,
\code{max_sv}と\code{min_op}の2目的のパレート最適化を
\code{all}という名称で定義している.
最後に,41行目の\code{#optimize}文の要素に\code{all}を与えることによっ
て,多目的 CAFE 問題のパレート最適解を得る.

図~\ref{aaaa}に,
問題インスタンス (図~\ref{fig:ovm}),
車種数$n=3$,
CAFE 基準値$t=8.5$に対して,
パレート最適解を全列挙をした結果を示す.
解は全部で4つあり,
予想販売台数は右から左へ大きくなり,
装備オプション数は左から右へ小さくなっていることがわかる.

以上のように,ASP に基づく CAFE 問題ソルバーでは,
ASP 言語の表現力の高さを活かし,
多目的 CAFE 問題の制約と目的関数を簡潔に表現できる.
特に,\code{#preference}文を利用することにより,
複数の目的関数を定義し組み合わせることで,柔軟な最適化が可能な点が大きな特長である.

%----------------------------------------

  

\begin{figure*}[t]\centering
  \tabcolsep=1mm
  \begin{tabular}{l|c|c|c||c|c|c||c|c|c||c|c|c}\bhline
    & \multicolumn{3}{c||}{解1} & \multicolumn{3}{c||}{解2} & \multicolumn{3}{c||}{解3} & \multicolumn{3}{c}{解4}\\ \hline
    車種 & 1 & 2 & 3 & 1 & 2 & 3 & 1 & 2 & 3 & 1 & 2 & 3 \\ \hline
    \textsf{Grade} & \textsf{STD} & \textsf{DX} & \textsf{LX} & \textsf{STD} & \textsf{DX} & \textsf{LX} & \textsf{STD} & \textsf{DX} & \textsf{LX} & \textsf{STD} & \textsf{DX} & \textsf{LX} \\
    \textsf{Drive\_Type} & \textsf{2WD} & \textsf{2WD} & \textsf{4WD} & \textsf{2WD} & \textsf{2WD} & \textsf{4WD} & \textsf{2WD} & \textsf{2WD} & \textsf{2WD} & \textsf{2WD} & \textsf{2WD} & \textsf{2WD} \\
    \textsf{Engine} & \textsf{V6} & \textsf{V6} & \textsf{V6} & \textsf{V6} & \textsf{V6} & \textsf{V6} & \textsf{V6} & \textsf{V6} & \textsf{V6} & \textsf{V6} & \textsf{V6} & \textsf{V6} \\
    \textsf{Tire} & \textsf{16inch} & \textsf{17inch} & \textsf{18inch} & \textsf{16inch} & \textsf{17inch} & \textsf{18inch} & \textsf{16inch} & \textsf{17inch} & \textsf{18inch} & \textsf{16inch} & \textsf{17inch} & \textsf{18inch} \\
    \textsf{Transmission} & \textsf{6AT} & \textsf{HEV} & \textsf{10AT} & \textsf{10AT} & \textsf{HEV} & \textsf{10AT} & \textsf{10AT} & \textsf{HEV} & \textsf{10AT} & \textsf{10AT} & \textsf{10AT} & \textsf{10AT} \\
    \textsf{Sun\_Roof} & \textsf{-} & \textsf{-} & \textsf{-} & \textsf{-} & \textsf{-} & \textsf{-} & \textsf{-} & \textsf{-} & \textsf{-} & \textsf{-} & \textsf{-} & \textsf{-} \\\bhline
    予想販売台数(合計)  & \multicolumn{3}{c||}{5,525} & \multicolumn{3}{c||}{5,475} & \multicolumn{3}{c||}{5,135} & \multicolumn{3}{c}{4,723} \\ 
   装備オプション数 & \multicolumn{3}{c||}{12} & \multicolumn{3}{c||}{11} & \multicolumn{3}{c||}{10} & \multicolumn{3}{c}{9} \\ \hline
  \end{tabular}
  \caption{多目的 CAFE 問題(図~\ref{fig:ovm})のパレート最適解全列挙}
  \label{aaaa}
\end{figure*}


%----------------------------------------
%%% Local Variables:
%%% mode: japanese-latex
%%% TeX-master: "paper"
%%% End:
