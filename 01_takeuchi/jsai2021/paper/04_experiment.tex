\section{実行実験}
%----------------------------------------------
\begin{table}[t]\centering
  \caption{ベンチマーク問題}
  \vskip 1em
  % \renewcommand{\arraystretch}{0.9}
  % \tabcolsep = 0.9mm
  \begin{tabular}{lrrrr}\hline
    問題名 & \#装備タイプ	& \#装備オプション	& \#依存制約 	\\\hline
    small   & 8		& 21	& 4	  	        \\
    medium  & 86	& 229	& 147	  	        \\
    big	    & 315	& 1,337	& 0	          	\\\hline
    & & & \\
  \end{tabular}
 \label{tab:bench}
\end{table}
%----------------------------------------------

%----------------------------------------------
\begin{table}[t]\centering
  \caption{実験結果: CPU時間}
  \vskip 1em  
  % \renewcommand{\arraystretch}{0.9}
  % \tabcolsep = 0.9mm
  \begin{tabular}{cr|rr}\hline
    \lw{問題名} & CAFE & パレート   & \lw{CPU時間(秒)} \\
           & 基準値$t$ & 最適解の数 &  \\ \hline
    small  & 8.5   & 8*      &   35.136     \\
    small  & 9.0   & 5*      & 1085.354     \\
    small  & 9.5   & --       & $timeout$    \\
    small  & 10.0  & 1*      &    1.863     \\
    small  & 10.5  & 0*      &    0.221     \\\hline
  \end{tabular}
  \label{tab:result}
\end{table}
%----------------------------------------------

提案ソルバーの有効性を評価するために,多目的 CAFE 問題のパレート
最適解を全列挙する実行実験を行った.
ベンチマーク問題には,企業から提供を受けた3問を使用した(表~\ref{tab:bench}参照).
small は小規模な問題,
medium は現実規模の問題,
big は大規模な問題である.
各問題に対して,
5種類の CAFE 基準値$t = 8.5, 9.0, 9.5, 10.0, 10.5$km/Lを適用し,
計15問に対して実験を行なった.車種の数は$n=3$とした.
ASPシステムには{\asprin}-3.1.1を利用し,
1問あたりの制限時間は3時間とした.
実験環境は,Mac OS, 3.2GHz Intel Core i7, 64GB メモリである.

表~\ref{tab:result}に小規模な問題smallの実験結果を示す.
左の列から順に,問題名,CAFE基準値$t$,得られたパレート最適解の数,
解の全列挙に要した CPU 時間となっている.
記号`$\ast$'は,パレート最適解の全列挙に成功したことを意味する.
逆に,記号`--'は,パレート最適解が一つも得られなかったことを意味する.
%
実験の結果,5問中3問に対して,パレート最適解を全列挙することができた.
また,$t=10.5$の場合のパレート最適解の数は0であり,実行可能解が存在し
ないことが確認できた.
medium と big については,実行可能解は得られたものの,パレート最適解を
得ることはできなかった.

%%% Local Variables:
%%% mode: japanese-latex
%%% TeX-master: "paper"
%%% End:
