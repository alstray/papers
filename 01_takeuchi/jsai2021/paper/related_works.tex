\section{関連研究}
\cite{walter13:ceur}では,MaxSATソルバーを用いることで,顧客の制約を満たしていない
車両装備仕様の注文に対して,最小の変更による修正を可能にする.
また,顧客がすでに所有する車両に対して,オプションの追加,変更,除去を行うときに,
顧客の要望を最大限満たすような,車両の再構成に対する応用手法も提案している.
ドイツの自動車メーカーBMWから提供を受けた問題インスタンスから生成される
ベンチマークを用いて,これらの手法を評価しており,ほどよい計算時間で解を求めている.
\cite{genay19:jim}では,ASPを工業規模の自動車の製品構成問題に適用している.
顧客からの注文として(部分的な)装備仕様が与えられたときに,
満たされない制約の検証や,すべての装備仕様で選択される,
あるいはどの装備仕様にも選択されないオプションの発見など,
4種類のASPの応用を提案している.
工業用のデータセットでの実験の結果,標準的な計算機で最先端のASPソルバーを使用し,
数秒単位での計算が可能であることを確認している.

これらの関連研究では,装備仕様をあらかじめ設定し,それに対して矛盾の検証や
特徴的なオプションの発見などを行っている.
それに対して我々の研究では,問題インスタンスと制約のみを与え,
予想販売台数の最大化などの目的関数を最適化するような装備仕様の探索を目的としている.
また,燃費制約として,CAFE方式という最新の燃費規制を導入している点も,
本研究の特長である.