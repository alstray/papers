\section{おわりに}

本論文では,ASP に基づく多目的 CAFE 問題ソルバーの実装と評価について述べた.
ASP 言語の表現力の高さを活かし,
多目的 CAFE 問題の制約と目的関数を簡潔に表現できることを確認した.
企業から提供を受けたベンチマーク問題を用いて評価実験を行った結果,
小規模な問題に対して,パレート最適解を全列挙をすることができた.

我々の知る限り,多目的CAFE問題を対象とする関連研究は存在しない.
広い意味での関連研究としては,車両装備仕様問題のすべての実行可能解に共通する装備タイプと
装備オプションの組合せを求める問題\cite{genay19:jim}や,
車両装備仕様問題と顧客の希望する装備タイプと装備オプションの組合せが与えられたとき,
顧客の希望との差異が最小となる解を求める問題\cite{walter13:ceur}などがある.

今後の課題としては,実用規模の問題を解くためのASP符号化の改良が挙げられる.
さらに,ソルバーの実用性を高めるために,
協力企業から提案を受けている認証制約,適用タイミング制約,
IWRテーブル制約など,CAFE問題に対する様々な追加制約の実装も重要な研究課題である.
%%%%%%%%%%%%%%%%%%%%%%%%%%%%%%%%%%%%%%%%%%%%%%%%%%%
% \vskip 1em
% \begin{itemize}\Large
% \item 関連研究(DONE)
% \item 参考文献追加(DONE)
% \item 今後の課題(DONE)
% \end{itemize}
%%%%%%%%%%%%%%%%%%%%%%%%%%%%%%%%%%%%%%%%%%%%%%%%%%%


%%% Local Variables:
%%% mode: latex
%%% TeX-master: "paper"
%%% End:
