%%%%%%%%%%%%%%%%%%%%%%%%%%%%%%%%%%%%%%%%%%%%%%%%%%%%%%%%%% 
\chapter{車両装備仕様問題のASP符号化}
%%%%%%%%%%%%%%%%%%%%%%%%%%%%%%%%%%%%%%%%%%%%%%%%%%%%%%%%%% 

先にも述べたように,本研究では,問題インスタンスをASPのファクト形式に変換し,
車両装備仕様問題を解くASP符号化と結合した後に,ASPシステムを用いて解を求める.
本章では,ASPファクト形式,基本のASP符号化,改良したASP符号化について述べる.

\section{ASPファクト形式}
\lstinputlisting[float=htbp,caption={%
入力例のASPファクト表現},%
captionpos=b,frame=single,label=code:example.lp,%
numbers=left,%
breaklines=true,%
columns=fullflexible,keepspaces=true,%
basicstyle=\ttfamily\scriptsize]{code/example.lp}

車両装備仕様問題のインスタンスをASPのファクトとして表現する方法について述べる.第2章の問題の例(表\ref{tab:input})を,
ファクトで表現したものをコード\ref{code:example.lp}に表す.

2〜31行目は,タイプとオプションの集合を表している.
%タイプを\code{vp\_def}で,オプションを\code{v\_def}で表す.
例えば,2行目の\code{vp\_def("Drive\_Type")}はタイプのDrive\_Typeを表すファクトであり,
3行目の\code{v\_def("2WD","Drive\_Type",125)}はオプションの2WDを表すファクトで,
属するタイプはDrive\_Type,IWR値は125であることを示す.
%
34〜38行目は,必須であるタイプを表している.
例えば,34行目の\code{require\_vp("Drive\_Type")}は,
タイプDrive\_Typeが必須であることを示す.
%
40〜43行目は,オプションによるオプションの要求を表している.
例えば,40行目の\code{require\_v\_v("STD","16\_inch\_Tire")}は,オプションSTDを装備する場合,
オプション16\_inch\_Tireも装備しなければならないことを示す.
%
46〜48行目は,複数の装備仕様を区別するためのグループ(\code{group})を定義している.
この例では3種類の装備仕様を想定しているため,グループは3つである.
また,49〜51行目の一貫性制約で,それぞれのグループが
GradeのSTD, DX, LXを装備するようになっている.

装備仕様の燃費と販売台数を示すテーブルも,ASPのファクトとして表現できる.
コード\ref{code:tableFE.lp}, \ref{code:tableSV.lp}にて,各テーブルの一部を示す.
ファクト\code{fe\_map(S,FE)}, \code{sv\_map(S,SV)}は,
装備仕様のIWR値の和がSであるとき,燃費と販売台数がFEとSVであることを示している.
例えば,コード\ref{code:tableFE.lp}の1行目は,装備仕様のIWR値の和が900であるとき,
燃費は11.1km/Lであることを表し,コード\ref{code:tableSV.lp}の1行目は,予想される販売台数が
97台であることを表している.ASPでは整数しか扱えないため,燃費の値は10倍にして記述している.
また,販売台数のテーブルはIWR値の和が5の倍数であるもののみで構成されているため,
販売台数を求めるときにIWR値の和の端数を丸める必要がある.





\lstinputlisting[float=t,caption={%
燃費を示すテーブルのASPファクト表現},%
captionpos=b,frame=single,label=code:tableFE.lp,%
numbers=left,%
breaklines=true,%
columns=fullflexible,keepspaces=true,%
basicstyle=\ttfamily\scriptsize]{code/tableFE.lp}

\lstinputlisting[float=t,caption={%
販売台数示すテーブルのASPファクト表現},%
captionpos=b,frame=single,label=code:tableSV.lp,%
numbers=left,%
breaklines=true,%
columns=fullflexible,keepspaces=true,%
basicstyle=\ttfamily\scriptsize]{code/tableSV.lp}



\lstinputlisting[float=t,caption={%
基本符号化},%
captionpos=b,frame=single,label=code:encode01,%
numbers=left,%
breaklines=true,%
columns=fullflexible,keepspaces=true,%
basicstyle=\ttfamily\scriptsize]{code/encode01.lp}

\section{基本符号化}
車両装備仕様問題の各種制約はASPの一貫性制約およびアグリゲートを用いて,簡潔に符号化される.
制約を表す論理プログラムをコード\ref{code:encode01}に示す.

コード\ref{code:encode01}は,まず1行目で,CAFE基準値として変数\code{t}に90を代入している.
変数\code{t}の値は,プログラム実行時にオプションを加えることで,任意の値に変更することができる.
アトム\code{vp(VP,G)}, \code{v(V,G)}は,真であるとき,グループ\code{G}の装備仕様が,それぞれタイプ\code{VP},
オプション\code{V}を装備することを意味する.
3行目は,任意のタイプ\code{VP},任意のグループ\code{G}に対して,タイプの解候補となる
\code{vp(VP,G)}を導入し,個数制約を用いて,グループ\code{G}がタイプ\code{VP}を装備するか否かの
選択をするということを表している.
6行目はオプション個数制約を表しており,3行目のルールで真となった任意のアトム
\code{vp(VP,G)}に対して,オプションの解候補となる\code{v(V,G)}を導入し,個数制約を用いて,
タイプ\code{VP}に付随するオプションの中で,装備するオプションの数をちょうど1つに制限している.


9〜12行目は,一貫性制約を用いて要求制約を表している.
9行目は,任意の必須であるタイプ\code{VP},任意のグループ\code{G}に対して,
\code{vp(VP,G)}は真であることを意味する.
10行目は,任意のオプション\code{V},任意のグループ\code{G}に対して,
\code{require\_v\_v(V,W)}かつ\code{v(V,G)}ならば,アトム\code{v(W,G)}は真であることを意味する.
%10行目は,グループ\code{G}がオプション\code{V}を装備するとき,
%オプション\code{V}がオプション\code{W}を要求するならば,
%オプション\code{W}も装備しなければならないということを意味する.
11, 12行目は,同様にして,タイプによるオプションの要求,オプションによるタイプの要求を
記述している.


15〜18行目は,燃費制約を表している.
15行目は,重み付き個数制約を用いて,任意のグループ\code{G}に対して,
真となるアトム\code{v(V,G)}の\code{IWR}の和を変数\code{S}に代入し,
アトム\code{iwr(S,G)}を生成する.このアトム\code{iwr(S,G)}は,
グループ\code{G}の装備仕様のIWR値の和が\code{S}であることを表している.
%
16行目は,任意のグループ\code{G}と,任意のIWR値の和\code{S},任意の燃費\code{FE}に対して,
\code{iwr(S,G)}かつ\code{fe\_map(S,FE)}ならば,アトム\code{fe(FE,G)}を生成する.
このアトム\code{fe(FE,G)}は,グループ\code{G}の装備仕様の燃費が値\code{FE}であることを表している.
%
17行目は,任意のグループ\code{G}と,任意のIWR値の和\code{S},任意の販売台数\code{SV}に対して,
\code{S}に最も近い5の倍数\code{S1}を求める計算をし,\code{iwr(S,G)}かつ
\code{sv\_map(S1,SV)}ならば,アトム\code{sv(SV,G)}を生成する.
このアトム\code{sv(SV,G)}は,グループ\code{G}の装備仕様の販売台数が値\code{SV}であることを
表している.
%グループGのIWR値の和と燃費,販売台数のテーブルから,
%各装備仕様の燃費と販売台数を求めるためのルールで,
%アトムfe(FE,G), sv(SV,G)はそれぞれ,グループGの燃費がFE,販売台数がSVであることを表している.
%ただし,販売台数のテーブルではIWR値が5刻みになっているため,近似の値を取る.
18行目は,「販売台数を加味した全体の平均燃費が基準値t以上でなければならない」
という燃費制約を表している.第2章の燃費制約の式\ref{eq:cafe1}を変形すると,以下のようになる.
\begin{equation}
\label{eq:cafe2}
(FE_{1} - t) \times SV_{1} + (FE_{2} - t) \times SV_{2} + (FE_{3} - t) \times SV_{3} \geq 0
\end{equation}
つまり,すべてのグループの\code{(FE - t)*SV}の和が0以上でなければならないという制約を,
一貫性制約と重み付き個数制約を用いて表すことで燃費制約を実現している.

最後に目的関数について述べる.車両装備仕様問題の目的は制約をすべて満たし,
各グループの販売台数の和が最大の解を求めることである.これは,21行目の
最大化関数(\code{\#maximize})を用いて,各グループの販売台数\code{SV}の和を
最大化することによって実現される.

\section{改良符号化}


コード\ref{code:encode01}の15行目の重み付き個数制約は,基礎化の際に,
任意のグループ\code{G}に対して,
アトム\code{v(V,G)}のすべての部分集合による\code{IWR}の和を求めるルールを生成している.
しかし,実際はオプション個数制約や要求制約があり,IWR値の和が極端に大きい,
あるいは,極端に小さい範囲には解が存在しないため,無駄な探索空間を広げてしまっている.
そこで,IWR値の和の上下限を厳密に見積もるように改良したASP符号化を
コード\ref{code:encode02}に表す.
変更点としては15〜18行目を追加し,コード\ref{code:encode01}の15行目を,
21, 22行目のように変更した.
%
15行目は,任意のタイプ\code{V}に対して,それに付随するオプション\code{V}の中で最大の
\code{IWR}の値を変数\code{UB}に代入し,アトム\code{ub\_vp(UB,VP)}を生成する.
このアトム\code{ub\_vp(UB,VP)}は,タイプ\code{VP}に付随するオプションで最大のIWR値は
\code{UB}であることを表している.
%
16行目は,重み付き個数制約を用いて,すべてのアトム\code{ub\_vp(UB,VP)}の\code{UB}の和を
変数\code{S}に代入し,アトム\code{ub\_iwr(S)}を生成する.
このアトム\code{ub\_iwr(S)}はIWR値の和の上限が\code{S}であることを表している.
%
17行目は,15行目と同様にして,タイプ\code{VP}に付随するオプションで最小のIWR値は
\code{LB}であることを表すアトム\code{lb\_vp(LB,VP)}を生成する.
%
18行目は,アトム\code{lb\_vp(LB,VP)}の\code{VP}で,\code{require\_vp(VP)}が真であるものだけの
\code{LB}の和を変数\code{S}に代入し,アトム\code{lb\_iwr(S)}を生成する.
このアトムはIWR値の和の下限が\code{S}であることを表している.
%
21, 22行目は,IWR値の和\code{S}が\code{UB}以下かつ\code{LB}以上になるように,ボディにアトムを
加えている.
%
%15, 17行目のルールは,各タイプに属するオプションの中で,IWR値が最大のものと最小のものを求めている.
%16行目のルールは,すべてのタイプの最大のIWR値の和を求め,ファクトub\_iwr(S)の変数Sに代入している.
%18行目のルールは,必須であるタイプの最小のIWR値の和を求め,ファクトlb\_iwr(S)の変数Sに代入している.
%21行目のルールは,部分集合によるIWR値の和がLB以上,UB以下になるようにボディにアトムを加えている.
%このようにすることで,実現不可能な組合せが生まれるIWR値の和の範囲を大きく削減することが期待できる.

以上のようにして,上限をすべてのタイプでIWR値が最大のオプションを装備した場合の
和,下限を必須であるタイプのみでIWR値が最小のオプションを装備した場合の和として,
求めるIWR値の和の範囲を限定している.
\lstinputlisting[float=htbp,caption={%
改良符号化},%
captionpos=b,frame=single,label=code:encode02,%
numbers=left,%
breaklines=true,%
columns=fullflexible,keepspaces=true,%
basicstyle=\ttfamily\scriptsize]{code/encode02.lp}



%%% Local Variables:
%%% mode: latex
%%% TeX-master: "paper"
%%% End:
