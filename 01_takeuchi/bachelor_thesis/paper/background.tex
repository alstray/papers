%%%%%%%%%%%%%%%%%%%%%%%%%%%%%%%%%%%%%%%%%%%%%%%%%%%%%%%%%% 
\chapter{車両装備仕様問題}\label{chap:backgraound}
%%%%%%%%%%%%%%%%%%%%%%%%%%%%%%%%%%%%%%%%%%%%%%%%%%%%%%%%%% 
\begin{table}[tbp]
	\centering
	\caption{車両装備仕様問題の例}
	\label{tab:input}
	\begin{tabular}{|l|c|c|l|r|l|} \hline
	タイプ		& 必須	&上下限	& オプション 		& IWR	& 要求	\\\hline
	Grade 		& ○		& 1..1 	&STD 			& 350	& 16\_inch\_Tire	\\\cline{4-6}
				&		&		& DX 			& 350	& 17\_inch\_Tire	\\\cline{4-6}	
		 		& 		&		& LX 			& 350  	& 18\_inch\_Tire, AT\_typeA		\\\hline
	Drive\_Type 	& ○		& 1..1	& 2WD 			& 125	& 		\\\cline{4-6}
				& 		&		& 4WD 			& 200	&		\\\hline
	Engine		& ○		& 1..1 	& V4 			& 120	&		\\\cline{4-6}
				& 		&		& V6 			& 200	&		\\\hline
	Sun\_Roof 	& 		& 1..1 	& Nomal 		& 35 	& 		\\\cline{4-6}
				& 		&		& Panorama 		& 70	&		\\\hline
	Tire			& ○		& 1..1	& 15\_inch\_Tire 	& 90	&		\\\cline{4-6}
				&		&		& 16\_inch\_Tire 	& 110	&		\\\cline{4-6}
	 			& 		&		& 17\_inch\_Tire 	& 130	&		\\\cline{4-6}
	 			& 		&		& 18\_inch\_Tire 	& 150	&		\\\hline
	Transmission 	& ○ 	& 1..1	& AT\_typeA 		& 115	&		\\\cline{4-6}
			 	& 		&		& AT\_typeB 		& 125	&		\\\cline{4-6}
			 	& 		&		& HEV 			& 95	&		\\\cline{4-6}
			 	& 		&		& CVT 			& 80	&		\\\cline{4-6}
			 	& 		&		& MT\_typeA		& 48	&		\\\cline{4-6}
				& 		&		& MT\_typeB		& 55	&		\\\hline
	\end{tabular}
\end{table}

車両装備仕様問題は,\textbf{装備タイプ}とそれに付随する\textbf{装備オプション}から構成される.
装備タイプはエンジンやタイヤなどの装備の種類を表し,
装備オプションは,4気筒エンジン,15インチタイヤなどの具体的な装備を表す.
以降,装備タイプをタイプ,装備オプションをオプションと呼ぶことにする.
表\ref{tab:input}に6個のタイプと19個のオプションからなる問題の例を示す.
表の左から順に,タイプ名,タイプが必須か否か,オプション数の上下限,オプション名,
IWR(Inertial Working Rating)値,要求オプションを表す.
例えば,3行目のEngineは必須タイプで,2つのオプションV4, V6をもつ.


\textbf{装備仕様}とは,タイプとオプションの組合せであり,
(エンジン, 4気筒), (タイヤ, 15インチ)などの対の集合で表される.
車両装備仕様問題の目的は,与えられたタイプとオプションの集合から,
装備および燃費に関する制約を満たしつつ,販売台数を最大化する装備仕様を求める問題である.
本研究では,燃費に関する制約として,
企業別平均燃費 (Corporate Average Fuel Efficiency; CAFE)基準を採用する.
このCAFE基準は,自動車の燃費規制で,車種別ではなくメーカー全体で出荷台数を加味した
平均燃費を算出する方式である.
アメリカや EU で採用されており,日本でも2020年度燃費基準に採用されることが決定している.

車両装備仕様問題の制約は,以下の通りである.
\begin{itemize}
	\item \textbf{オプション個数制約}:各タイプごとに装備するオプション数は上下限に従わなければならない.
	表\ref{tab:input}の例では,Engineのオプションを決める際,上下限が1..1であるため,
	V4, V6からちょうど1つを装備しなければならない.
	\item \textbf{要求制約}:必須であるタイプは必ず装備しなければならなない.
	表\ref{tab:input}の例では,Drive\_TypeやEngineなど,必須の欄に丸がついているものは必ず装備しなければ
	ならないが,Sun\_Roofは装備しなくても構わない.
	また,要求をもつオプションを装備するとき,要求されたオプションも装備しなければならない.
	表\ref{tab:input}の例では,GradeタイプのSTDを装備する場合,16\_inch\_Tireを装備しなければならない.
	この例ではオプションがオプションを要求している場合のみだが,実際にはタイプがオプションを,
	オプションがタイプを要求する場合もある.
	\item \textbf{燃費制約}:装備仕様ごとの出荷台数も加味した全体の平均燃費が与えられた
	\textbf{CAFE基準値}以上にならなければならない.式で表現すると以下のようになる.
	 \begin{equation}
	 \label{eq:cafe1} 
 	\underbrace{
		\frac{Fe_{1} \times V_1 + Fe_{2} \times V_2 + Fe_{3} \times V_3 }{V_1 + V_2 + V_3}
	}_{\mbox{平均燃費}}
	\geq 
	\underbrace{t}_{\mbox{CAFE基準値}}
	\end{equation}

	各装備仕様で装備するオプションのIWR値の総和を求め,その総和を引数として,
	あらかじめ与えられているテーブルより,装備仕様ごとの燃費と予想販売台数を得る.
	この例では装備仕様iの燃費を$Fe_i$,販売台数を$V_i$とする.また,3種類の装備仕様を想定しているため,
	$1 \leq i \leq 3$である.
\end{itemize}

%本節では車両装備仕様問題について説明する.
%車両装備仕様問題の詳細に入る前に,いくつかの用語を説明する.
%車両を構成する個々の装備を\textbf{装備オプション}あるいは単に\textbf{オプション}と呼ぶ.
%装備オプションはエンジンやタイヤなど,いくつかの種類に分類され,その種類を\textbf{装備タイプ}あるいは単に\textbf{タイプ}と呼ぶ.
%各装備オプションに対し,燃費や予想販売台数の計算に利用される\textbf{IWR}(Inertial Working Rating)と呼ばれる値が定まっている.

%車両装備仕様問題の入力は,装備オプションと装備タイプの集合,グレードの数,CAFE基準値,依存関係のある装備タイプや装備オプションの組み合わせからなる制約の集合である.
%出力は,各グレードの仕様となる装備オプションの組合せである.

%表\ref{tab:input}に入力例を示す.
%各タイプは1つ以上のオプションを持ち,オプションの上下限によって,いくつのオプションを選択するか決定する.
%表\ref{tab:input}の例では,タイプEngineではオプションV4,V6からちょうど1つが選択されるが,タイプSun\_Roofではオプションが選択されない場合もある.
%各オプションにはIWRが定められている.また,特定のオプションは要求を持っており,そのオプションが選択された場合は,要求にあるオプションやタイプも選択されなければならない.表\ref{tab:input}の例では,オプションSTDが選択された場合,オプション16\_inch\_Tireも選択される.

表\ref{tab:output}にて,表\ref{tab:input}に対する解の例を示す.このとき,求める装備仕様を3種類,
CAFE基準値を9.5km/Lとする.
各装備仕様の列に装備するオプションが示されており,IWR値の和からそれぞれの燃費と販売台数が得られている.
例えば,装備仕様1を表すタイプとオプションの組合せは(Grade, STD),(Drive\_Type, 4WD),
(Engine, V6),(Sun\_Roof, Panorama),(Tire, 16\_inch\_Tire),(Transmission, MT\_typeA)で,
IWR値の和が978となり,燃費が10.2km/L,予想される販売台数が750台となっている.
すべての装備仕様で,各タイプごとにちょうど1つのオプションを選択しており,
オプション個数制約が満たされていることがわかる.
また,必須であるタイプは全て装備されていて,オプションの要求も
すべて満たされていることから,要求制約が満たされていることがわかる.
燃費制約の式は,
\begin{equation}
 	\frac{10.2\times750 + 9.3\times820 + 9.2\times1083}{750 + 820 +1083} = 9.52
	\geq 
	9.5
\end{equation}
となり,全体の平均燃費は9.52km/Lで,これはCAFE基準値の9.5km/L以上であるため,
燃費制約を満たしていることがわかる.
また,この解は最適解であるため,制約を満たす3種類の装備仕様の組の中で,
合計の販売台数2653(=750+820+1083)台が最大の販売台数である.
\begin{table}
	\centering
	\caption{車両装備仕様問題の解の例(t=9.5)}
	\label{tab:output}
	\begin{tabular}{|l|l|r|c|c|c|} \cline{1-6}
	タイプ		& オプション 		& IWR	& \multicolumn{3}{c|}{装備仕様} \\\cline{4-6}
				&				&		& 1	& 2	& 3	\\\cline{1-6}
	Grade 		&STD 			& 350	& ○	&	&	\\\cline{2-6}
				& DX 			& 350	&	& ○	&	\\\cline{2-6}	
		 		& LX 			& 350  	& 	&	& ○	\\\cline{1-6}
	Drive\_Type 	& 2WD 			& 125	& 	&	&	\\\cline{2-6}
				& 4WD 			& 200	& ○	& ○	& ○	\\\cline{1-6}
	Engine		& V4 			& 120	&	&	&	\\\cline{2-6}
				& V6 			& 200	& ○	& ○	& ○	\\\cline{1-6}
	Sun\_Roof 	& Nomal 		& 35 	&	& 	&	\\\cline{2-6}
				& Panorama 		& 70	& ○	& ○	& ○	\\\cline{1-6}
	Tire			& 15\_inch\_Tire 	& 90	&	&	&	\\\cline{2-6}
				& 16\_inch\_Tire 	& 110	& ○	&	&	\\\cline{2-6}
	 			& 17\_inch\_Tire 	& 130	&	& ○	&	\\\cline{2-6}
	 			& 18\_inch\_Tire 	& 150	&	&	& ○	\\\cline{1-6}
	Transmission 	& AT\_typeA 		& 115	&	&	& ○	\\\cline{2-6}
			 	& AT\_typeB 		& 125	&	& ○	&	\\\cline{2-6}
			 	& HEV 			& 95	&	&	&	\\\cline{2-6}
			 	& CVT 			& 80	&	&	&	\\\cline{2-6}
			 	& MT\_typeA		& 48	& ○	&	&	\\\cline{2-6}
				& MT\_typeB		& 55	&	&	&	\\\cline{1-6}
	\multicolumn{3}{r}{IWRの和} & \multicolumn{1}{c}{978} & \multicolumn{1}{c}{1075} & \multicolumn{1}{c}{1085}\\ 
	\multicolumn{3}{r}{燃費} & \multicolumn{1}{c}{10.2} & \multicolumn{1}{c}{9.3} & \multicolumn{1}{c}{9.2}\\ 
	\multicolumn{3}{r}{販売台数} & \multicolumn{1}{c}{750} & \multicolumn{1}{c}{820} & \multicolumn{1}{c}{1083}\\ 

	\end{tabular}
\end{table}


%%% Local Variables:
%%% mode: latex
%%% TeX-master: "paper"
%%% End:
