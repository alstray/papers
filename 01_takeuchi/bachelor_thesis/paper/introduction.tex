%%%%%%%%%%%%%%%%%%%%%%%%%%%%%%%%%%%%%%%%%%%%%%%%%%%%%%%%%% 
\chapter{序論}
\pagenumbering{arabic}
%%%%%%%%%%%%%%%%%%%%%%%%%%%%%%%%%%%%%%%%%%%%%%%%%%%%%%%%%% 
自動車の車両を設計する\textbf{車両装備仕様問題}は,求解困難な組合せ最
適化問題の一種である.
%
車両の装備仕様を決めるには,販売される国や地域の法規や規制,地域や市場
の特性,市場の嗜好や競合など十分に考慮する必要があり,
現状では専門知識をもつ技術者の多大な労力が費やされている.
そのため,車両装備仕様問題の効率のよい解法は重要な研究課題である.

車両装備仕様問題は,\textbf{装備タイプ}とそれに付随する\textbf{装備オプション}から構成される.
装備タイプはエンジンやタイヤなどの装備の種類を表し,
装備オプションは,4気筒エンジン,15インチタイヤなどの具体的な装備を表す.
%
\textbf{装備仕様}とは,タイプとオプションの組合せであり,
(エンジン, 4気筒), (タイヤ, 15インチ)などの対の集合で表される.
車両装備仕様問題の目的は,与えられたタイプとオプションの集合から,
装備および燃費に関する制約を満たしつつ,販売台数を最大化する装備仕様を
求めることである.

本研究では,燃費に関する制約として,
企業別平均燃費 (Corporate Average Fuel Efficiency; CAFE)基準を採用する.
このCAFE基準は,自動車の燃費規制で,車種別ではなくメーカー全体で出荷台数を加味した
平均燃費を算出する方式である.
アメリカや EU で採用されており,日本でも2020年度燃費基準 に採用されることが決定している.

\textbf{解集合プログラミング}(Answer Set Programing; ASP \cite{%
  Baral03:cambridge,%
  Gelfond88:iclp, %
  Inoue08:jssst})
は,論理プログラミングから派生した宣言的プログラミングパラダイムである.
ASP言語は一階論理に基づく知識表現言語の一種であり,
論理プログラムはASPのルールの有限集合である.ASPシステムは論理プログラムから
安定モデル意味論 \cite{Gelfond88:iclp}
に基づく解集合を計算するシステムである.近年,SATソルバー技術を応用した高速ASPシステムが
実現され,ロボット工学,システム生物学,システム検証,制約充足問題,プランニングなど
様々な分野への実用的応用が急速に拡大している\cite{Gelfond16:aim}.

本論文では,解集合プログラミング(ASP)を用いた車両装備仕様問題の解法に
ついて述べる.本解法では,まず与えらた問題インスタンスを ASP のファクト
形式に変換し,車両装備仕様問題を解くASP符号化と結合した後に,ASPシステ
ムを用いて解を求める.
ASP符号化として,基本符号化と改良符号化の2種類を考案した.
特に,改良符号化は,燃費を算出するために必要な
IWR (Inertial Working Rating) 値の上下限を厳密に見積もることにより,
基礎化後のルール数を少なく抑えるよう工夫されている.
これにより,より大規模な問題への有効性が期待できる.

提案手法の有効性を評価するために,企業から提供された現実の問題(3問)に対して,
5種類のCAFE基準値(8.5, 9.0, 9.5, 10.0,10.5km/L)に対する
問題インスタンス(計15問)を生成し,実行実験を行った.
%
小規模(タイプ数8,オプション数21)な問題インスタンス(5問)については,
両符号化とも最適解を求めることができた.
%中規模(タイプ数86,オプション数229)と
大規模(タイプ数315,オプション数1337)な問題インスタンス(5問)につい
ては,改良符号化が,基本符号化より優れた解を得ることに成功した.
これにより,大規模な問題に対する改良符号化の有効性が確認できた.

本論文の構成は,以下の通りである.
第2章では車両装備仕様問題について,第3章では解集合プログラミングについて説明を行う.
第4章では車両装備仕様問題のASP符号化について開発したプログラムを示しながら説明し,
第5章ではプログラムの比較実験とその考察について述べる.
%第6章では,本論文のまとめと,今後の課題について述べる.

% \vskip 3em

% まず最初に,問題インスタンスをASPファクトとして表現する方法を述べる.
% ASPは記号を(整数値などで置換することなく)そのまま扱えるため,問題を簡潔に記述できる.
% 次に,車両装備仕様問題の制約のASP符号化について述べる.各制約は,ASPの一貫性制約や
% 重み付き個数制約を用いて簡潔に表現される.
% 加えて,符号化の際に無駄な探索空間を削減するための処理を行う改善手法について説明する.
% 最後に,元の符号化と改善した符号化を比較するために,
% 規模の異なる3種類のベンチマークを用いた計15問の比較実験を行った.
% その結果,目的関数の値は,15問中8問で改善した符号化の方が高い値を示し,
% 6問は同じ値(内5問は最適値)を示した.
% このように,符号化の改善手法の有効性を示すことができた.


% \begin{itemize}
% 	\item \textbf{オプション個数制約}:各タイプごとに装備するオプション数は上下限に従わなければならない.
% 	表\ref{tab:input}の例では,Engineのオプションを決める際,上下限が1..1であるため,
% 	V4, V6からちょうど1つを装備しなければならない.
% 	\item \textbf{要求制約}:必須であるタイプは必ず装備しなければならなない.
% 	表\ref{tab:input}の例では,Drive\_TypeやEngineなど,必須の欄に丸がついているものは必ず装備しなければ
% 	ならないが,Sun\_Roofは装備しなくても構わない.
% 	また,要求をもつオプションを装備するとき,要求されたオプションも装備しなければならない.
% 	表\ref{tab:input}の例では,GradeタイプのSTDを選択した場合,16\_inch\_Tireを装備しなければならない.
% 	今回の例ではオプションがオプションを要求している場合のみだが,実際にはタイプがオプションを,
% 	オプションがタイプを要求する場合もある.
% 	\item \textbf{燃費制約}:装備仕様ごとの出荷台数も加味した全体の平均燃費が与えられた
% 	\textbf{CAFE基準値}以上にならなければならない.式で表現すると以下のようになる.
% 	 \[ 
%  	\underbrace{
% 		\frac{Fe_{1} \times V_1 + Fe_{2} \times V_2 + Fe_{3} \times V_3 }{V_1 + V_2 + V_3}
% 	}_{\mbox{平均燃費}}
% 	\geq 
% 	\underbrace{X}_{\mbox{CAFE基準値}}
%  	\]

% 	各装備仕様で選択されたオプションのIWRの総和を求め,その総和を引数として,
% 	あらかじめ与えられているテーブルより,装備仕様ごとの燃費と予想販売台数を得る.
% 	この例では装備仕様iの燃費を$Fe_i$,販売台数を$V_i$とする.また,3種類の装備仕様を想定しているため,
% 	$1 \leq i \leq 3$である.
% \end{itemize}


%%% Local Variables:
%%% mode: latex
%%% TeX-master: "paper"
%%% End:
