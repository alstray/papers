%%%%%%%%%%%%%%%%%%%%%%%%%%%%%%%%%%%%%%%%%%%%%%%%%%%%%%%%%% 
\chapter{結論}
%%%%%%%%%%%%%%%%%%%%%%%%%%%%%%%%%%%%%%%%%%%%%%%%%%%%%%%%%%
本論文では,解集合プログラミング(ASP)を用いた車両装備仕様問題の解法について述べた.
本解法では,まず与えられた問題インスタンスを ASP のファクト形式に変換し,
車両装備仕様問題を解くASP符号化と結合した後に,ASPシステムを用いて解を求めた.
ASP符号化として,基本符号化と改良符号化の2種類を考案した.
特に,改良符号化は,燃費を算出するために必要な
IWR (Inertial Working Rating) 値の上加減を厳密に見積もることにより,
基礎化後のルール数を少なく抑えるよう工夫されている.
 %
実行実験では,小規模な問題については,両符号化とも最適解を求めることができ,
大規模な問題については,改良符号化が,基本符号化より優れた解を得ることに成功した.
これにより,大規模な問題に対する改良符号化の有効性が確認できた.

%今後の課題として,以下が挙げられる.
%\begin{itemize}
%	\item 中規模,大規模な問題インスタンスに対する現実的な時間内での最適解の求解
%	\item 最適解が複数ある場合の列挙
%	\item さらなる制約や目的関数の追加
%\end{itemize}
%現状では,中規模,大規模な問題インスタンスに対しては最適解を求めることができていないため,
%これの実現
%さらなる制約としては,同時に装備できないオプション間の排他性による制約や,
今後の課題は,中規模,大規模な問題インスタンスに対しても,最適解を現実的な時間で求めることである.
その他,車両の生産コストを抑えるために重要な,共通部品の最大化を目的関数として加えることや,
オプション間の排他性など新たな制約の追加,
最適解が複数ある場合の列挙を可能にすることなどが今後の課題として挙げられる.
%%% Local Variables:
%%% mode: latex
%%% TeX-master: "paper"
%%% End:
