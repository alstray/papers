%%%%%%%%%%%%%%%%%%%%%%%%%%%%%%%%%%%%%%%%%%%%%%%%%%%%%%%%%% 
\chapter*{概要}
\pagenumbering{roman}
%%%%%%%%%%%%%%%%%%%%%%%%%%%%%%%%%%%%%%%%%%%%%%%%%%%%%%%%%% 

自動車の車両を設計する車両装備仕様問題は,求解困難な組合せ最適化問題の一種である.
車両装備仕様問題の目的は,与えられた装備タイプとそれに付随する装備オプションの集合から,
装備および燃費に関する制約を満たしつつ,販売台数を最大化する装備仕様を
求めることである.
%車両の装備仕様を決めるには,販売される国や地域の法規や規制,地域や市場
%の特性,市場の嗜好や競合など十分に考慮する必要があり,
%現状では専門知識をもつ技術者の多大な労力が費やされている.

解集合プログラミング(Answer Set Programing; ASP)
は,論理プログラミングから派生した宣言的プログラミングパラダイムである.
ASP言語は一階論理に基づく知識表現言語の一種であり,
論理プログラムはASPのルールの有限集合である.ASPシステムは論理プログラムから
安定モデル意味論に基づく解集合を計算するシステムである.
近年,SATソルバー技術を応用した高速ASPシステムが
実現され,様々な分野への実用的応用が急速に拡大している.

本論文では,解集合プログラミング(ASP)を用いた車両装備仕様問題の解法に
ついて述べる.本解法では,まず与えらた問題インスタンスを ASP のファクト
形式に変換し,車両装備仕様問題を解くASP符号化と結合した後に,ASPシステ
ムを用いて解を求める.
ASP符号化として,基本符号化と改良符号化の2種類を考案した.
特に,改良符号化は,燃費を算出するために必要な
IWR (Inertial Working Rating) 値の上下限を厳密に見積もることにより,
基礎化後のルール数を少なく抑えるよう工夫されている.
これにより,より大規模な問題への有効性が期待できる.

問題インスタンス(計15問)を用いた実行実験の結果,
小規模な問題インスタンス(5問)については,両符号化とも最適解を求めることができた.
大規模な問題インスタンス(5問)については,改良符号化が,基本符号化より優れた解を得ることに成功した.
これにより,大規模な問題に対する改良符号化の有効性が確認できた.

%表紙を情報工学コース用のスタイルにするために,作成した{\tt jbachelor.sty}が
%必要である.
%TexStudioなどの便利なTex統合環境を利用するために,lualtexを使うとよい.
%platex と lualatex を切り替えるためには,このファイルの先頭を編集してlualatex用
%のltjbook.clsを使うようにする.
%\begin{verbatim}
%%% for platex
%\documentclass[a4paper,12pt]{jbook}
%%% for lualatex
%\documentclass[a4paper,12pt]{ltjbook}
%\end{verbatim}

%%% Local Variables:
%%% mode: japanese-latex
%%% TeX-master: "paper"
%%% End: